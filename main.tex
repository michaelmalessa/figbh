% !TEX TS-program = lualatex
% !TEX encoding = UTF-8 Unicode

% The code needs to be compiled with the LuaLaTeX compiler.

\documentclass[11pt,letterpaper]{scrbook}

\usepackage[bidi=basic,hebrew,greek,english,provide*=*]{babel}
\usepackage{csquotes}
\usepackage[margin=3cm]{geometry}
\usepackage{fontspec}
\usepackage{microtype}
\usepackage[raggedrightboxes]{ragged2e}
\usepackage{hanging}
\usepackage{enumitem}
\usepackage{multirow}
\usepackage{multicol}
\usepackage{longtable}
\usepackage[bottom]{footmisc}
\usepackage{colortbl}
\usepackage[svgnames]{xcolor}
\usepackage{float}
\usepackage{graphicx}
\usepackage{lastpage}

% Fonts
\babelfont{rm}{Charis SIL}
\babelfont[hebrew]{rm}[Renderer=HarfBuzz]{Ezra SIL}
\babelfont[greek]{rm}[Renderer=HarfBuzz]{GFS Didot}

% KOMA-Script formatting
\renewcommand{\chapterformat}{}
\renewcommand{\chaptermark}[1]{\markboth{#1}{}}
\renewcommand\addchaptertocentry[2]{\addtocentrydefault{chapter}{}{#2}}

\setkomafont{title}{\normalfont\huge\bfseries}
\setkomafont{chapter}{\normalfont\huge\bfseries}
\setkomafont{section}{\normalfont\Large\bfseries}
\setkomafont{subsection}{\normalfont\large\bfseries}
\setkomafont{subsubsection}{\normalfont\large\bfseries}
\setkomafont{chapterentry}{\normalfont\bfseries}

\setcounter{tocdepth}{2}
\setcounter{secnumdepth}{2}

% LTRfootnotes in RTL Text after \selectlanguage{hebrew} (footnote is left-aligned, footnote separator left)
\newcommand{\LTRfootnote}[1]{%
	\unskip% This removes any accidental space before the command
	{\textdir TLT\selectlanguage{english}\footnote{#1}}%
}

% Command for making the use of the longtable environment possible in a two column setup
\newsavebox\ltmcbox % needed for the glossary

% Load hyperref late
\usepackage[pdfusetitle,hidelinks]{hyperref}


% Metadata
\title{A Free Introductory Grammar \\ of Biblical Hebrew}
\author{Michael Malessa}
\date{}
\newcommand{\theversion}{2025.12}


% Code for tables
% Table for examples:
%\begin{tabular}{>{\raggedleft}p{0.35\linewidth} p{0.55\linewidth}}
%	\foreignlanguage{hebrew}{} & \textit{} () \\
%\end{tabular}

% Text for Exercises:
%{}~~\foreignlanguage{hebrew}{}  \hspace{0.3cm}
%{}~~\foreignlanguage{hebrew}{}\LTRfootnote{\space \foreignlanguage{hebrew}{text} \textit{}} \foreignlanguage{hebrew}{} \hspace{0.3cm}

% Text for Hebrew readings:
%\selectlanguage{hebrew}
%\noindent
%\textsuperscript{}~\foreignlanguage{hebrew}{} \hspace{0.3cm}
%\textsuperscript{}~\foreignlanguage{hebrew}{} \hspace{0.3cm}
%\textsuperscript{}~\foreignlanguage{hebrew}{} \hspace{0.3cm}
%\selectlanguage{english}

% Table for notes of Hebrew readings:
%\hspace*{-0.5cm}\begin{longtable}{p{0.075\linewidth} p{0.1\linewidth}p{0.725\linewidth}}
	%	 & \foreignlanguage{hebrew}{} & \\
	%	 & \foreignlanguage{hebrew}{} & \textit{} \\
	%\end{longtable}

% Key to Exercises, Translations of verbal forms
%\textbf{} \textit{} \hspace{0.15cm} \foreignlanguage{hebrew}{} \hspace{0.3cm}

\begin{document}

\frontmatter
\maketitle

\newpage

\vspace*{\fill}

\noindent Typeset in \LaTeX.

\bigskip

\noindent Version \theversion

\bigskip

% The license information is based on https://github.com/tbazett/diffyqs/blob/master/diffyqs.tex or
% https://web.uvic.ca/~tbazett/diffyqs/colophon-1.html

\noindent A Free Introductory Grammar of Biblical Hebrew \copyright \space 2025 by Michael Malessa is licensed under the Creative Commons Attribution-Share Alike 4.0 International License (CC BY-SA 4.0).

\bigskip

\includegraphics[scale=0.8]{CC-BY-SA}

\bigskip

\noindent To view a copy of this license, visit \url{https://creativecommons.org/licenses/by-sa/4.0/}.

\bigskip

\noindent You may use, print, duplicate, share this work as much as you want. You may also create a derivative work (including a translation), if it is under the same Creative Commons Attribution-Share Alike 4.0 International License (CC BY-SA 4.0). A derivative work must be prominently marked as such.

\bigskip

\noindent The \LaTeX \space source for the book and a PDF file of the latest version are available at GitHub: \url{https://github.com/michaelmalessa/figbh}.

\chapter{Preface}

The Free Introductory Grammar of Biblical Hebrew is intended to enable students to read the Hebrew Bible with the help of a lexicon. Verbal morphology is introduced at a very early stage because of its importance in the grammar of Biblical Hebrew and the biblical texts themselves. The vocabulary includes almost all words that occur fifty times or more in the Hebrew Bible and a number of less frequent but important words. All exercises and examples are taken from the Hebrew Bible.

Special thanks go to my students at the Biblical Seminary of the Philippines in Valenzuela City and the Asian Theological Seminary in Quezon City for letting me try this introductory grammar on them in the academic year 2024--25.

With the plethora of Hebrew introductory grammars on the market the natural question is why another one is presented to the public. There are two reasons for this.

First, this grammar was written with the intention of making a Hebrew grammar for beginners available at no costs and without a restrictive license. Therefore, it is licensed under the Creative Commons Attribution-Share Alike 4.0 International License. For more information see the copyright page.

Second, this grammar introduces the most frequent verbal forms including weak forms at a very early stage and uses only authentic Biblical Hebrew for the examples and the exercises. In this respect it follows the model of Ernst Jenni's \textit{Lehrbuch der hebräischen Sprache des Alten Testaments} (4th ed.; Basel: Schwabe, 2009). This approach makes it possible to present authentic Hebrew sentences already in Chapter 3 and to avoid the pitfalls of self-composed Hebrew.

This book was typeset in \LaTeX \space because of the capabilities of \LaTeX \space and the fact that it is open source software and platform independent. The use of open source software is a prerequisite for a truly free work. This applies to fonts, as well. As a consequence, only fonts under a free license are used.

More information about \LaTeX \space is available on Wikipedia \space \url{https://en.wikipedia.org/wiki/LaTeX} and the website of the \TeX \space Users Group \url{https://www.tug.org/}. Information about the installation of \LaTeX \space can be found here: \url{https://www.tug.org/texlive/}.

\medskip
\noindent
December 2025

\medskip
\noindent
Michael Malessa

\tableofcontents
\addcontentsline{toc}{chapter}{Contents}
% The command in the line adds an entry in the document outline of the PDF file and an entry in the TOC


\mainmatter




\chapter{Chapter 1}

%\section{Test}
%
%Text \hebrewfontonetest{וכל יקרא בשם יהוה ימלט}
%Text \hebrewfonttwotest{וכל יקרא בשם יהוה ימלט}

\section{The Hebrew Alphabet}
	\noindent The Hebrew alphabet consists of 22 letters, originally representing only consonants. Hebrew is written from right to left. One letter, the letter \foreignlanguage{hebrew}{ש}, represents two distinct consonants.

	\renewcommand\arraystretch{1.3}

	\begin{center}
		\begin{tabular}{|rp{0.1mm}p{3mm}|rl|l|l|cr|}

			\hline
			\multicolumn{3}{|c|}{I} & \multicolumn{2}{c|}{II} & \multicolumn{1}{c|}{III} & \multicolumn{1}{c|}{IV} & \multicolumn{2}{c|}{V}\\
			\hline
			\foreignlanguage{hebrew}{א} & & & \foreignlanguage{hebrew}{אָ֫לֶף} & \textit{ʾālæp̄} & \textit{ʾ} & ˀ & & 1 \\
			\foreignlanguage{hebrew}{ב} & & & \foreignlanguage{hebrew}{בֵּית} & \textit{bēt} & \textit{b}, \textit{ḇ/v} & bet, vet & & 2 \\
			\foreignlanguage{hebrew}{ג} & & & \foreignlanguage{hebrew}{גִּ֫ימֶל} & \textit{gīmæl} & \textit{g} & get & & 3 \\
			\foreignlanguage{hebrew}{ד} & & & \foreignlanguage{hebrew}{דָּ֫לֶת} & \textit{dālæt} & \textit{d} & dot & & 4 \\
			\foreignlanguage{hebrew}{ה} & & & \foreignlanguage{hebrew}{הֵא} & \textit{hē} & \textit{h} & hat & & 5 \\
			\foreignlanguage{hebrew}{ו} & & & \foreignlanguage{hebrew}{וָו} & \textit{wāw} & \textit{w} & vase & & 6 \\
			\foreignlanguage{hebrew}{ז} & & & \foreignlanguage{hebrew}{זַ֫יִן} & \textit{zayin} & \textit{z} & zoo & & 7 \\
			\foreignlanguage{hebrew}{ח} & & & \foreignlanguage{hebrew}{חֵית} & \textit{ḥēt} & \textit{ḥ} & Ba\textit{ch} & & 8 \\
			\foreignlanguage{hebrew}{ט} & & & \foreignlanguage{hebrew}{טֵית} & \textit{ṭēt} & \textit{ṭ} & ton & & 9 \\
			\foreignlanguage{hebrew}{י} & & & \foreignlanguage{hebrew}{יוֺד} & \textit{yōd} & \textit{y} & yard & & 10 \\
			\foreignlanguage{hebrew}{ך} ,\foreignlanguage{hebrew}{כ} & & & \foreignlanguage{hebrew}{כַּף} & \textit{kap̄} & \textit{k}, \textit{ḵ} & kite, Ba\textit{ch} & & 20 \\
			\foreignlanguage{hebrew}{ל} & & & \foreignlanguage{hebrew}{לָ֫מֶד} & \textit{lāmæd} & \textit{l} & long &  & 30 \\
			\foreignlanguage{hebrew}{ם} ,\foreignlanguage{hebrew}{מ} & & & \foreignlanguage{hebrew}{מֵם} & \textit{mēm} & \textit{m} & map & & 40 \\
			\foreignlanguage{hebrew}{ן} ,\foreignlanguage{hebrew}{נ} & & & \foreignlanguage{hebrew}{נוּן} & \textit{nūn} & \textit{n} & nun & & 50 \\
			\foreignlanguage{hebrew}{ס} & & & \foreignlanguage{hebrew}{סָ֫מֶךְ} & \textit{sāmæḵ} & \textit{s} & sun & & 60 \\
			\foreignlanguage{hebrew}{ע} & & & \foreignlanguage{hebrew}{עַ֫יִן} & \textit{ʿayin} & \textit{ʿ} & ˁ & & 70 \\
			\foreignlanguage{hebrew}{ף} ,\foreignlanguage{hebrew}{פ} & & & \foreignlanguage{hebrew}{פֵּא} & \textit{pē} & \textit{p}, \textit{p̄/f} & pan, fun & & 80 \\
			\foreignlanguage{hebrew}{ץ} ,\foreignlanguage{hebrew}{צ} & & & \foreignlanguage{hebrew}{צָדֵי} & \textit{ṣādē} & \textit{ṣ} & tsar & & 90 \\
			\foreignlanguage{hebrew}{ק} & & & \foreignlanguage{hebrew}{קוֺף} & \textit{qōp̄} & \textit{q} & kaddish & & 100 \\
			\foreignlanguage{hebrew}{ר} & & & \foreignlanguage{hebrew}{רֵישׁ} & \textit{rēš} & \textit{r} & run & & 200 \\
			\multirow{2}{*}{\foreignlanguage{hebrew}{ש}} & \multirow{2}{*}{\Large\{} & \foreignlanguage{hebrew}{שׂ} & \foreignlanguage{hebrew}{שִׂין} & \textit{śīn} & \textit{ś} & sin & \multirow{2}{*}{\Large\}} & \multirow{2}{*}{300} \\
			& & \foreignlanguage{hebrew}{שׁ} & \foreignlanguage{hebrew}{שִׁין} & \textit{šīn} & \textit{š} & shin & & \\
			\foreignlanguage{hebrew}{ת} & & & \foreignlanguage{hebrew}{תָּו} & \textit{tāw} & \textit{t} & tag & & 400 \\
			\hline

		\end{tabular}
	\end{center}

\newpage

\noindent \textbf{Key to the Table}

\vspace{0.5cm}

\renewcommand\arraystretch{1}

\begin{tabular}{p{0.025\linewidth}p{0.85\linewidth}}
		I & Hebrew letters in square script (with final letters after the comma) \\
		II & Name of the letters in Hebrew and transliteration \\
		III & Transliteration \\
		IV & Pronunciation \\
		V & Numerical value \\
\end{tabular}

\renewcommand\arraystretch{1.4}

\vspace{0.75cm}

\noindent \textbf{Additional Information}

\noindent The Hebrew term for alphabet is \foreignlanguage{hebrew}{א״ב} which is pronounced \textit{ʾālæp̄-bēt}.

The pronunciation of the consonants is indicated by English words beginning with a corresponding consonant except for the letter \foreignlanguage{hebrew}{ח}. At times, the English word gives the pronunciation of the Hebrew sound only by approximation.

The second letter after the comma (when read from right to left) in the first column is the shape of the letter when it occurs at the end of a word. These letters with a distinct shape at the end of a word are called final letters or \foreignlanguage{hebrew}{כַּמְנַפֵּץ} \textit{kamnappēṣ} letters.

The \foreignlanguage{hebrew}{ר} is pronounced with the tip of the tongue (as \textit{r} in Tagalog).

The letter \foreignlanguage{hebrew}{ש} originally represented two distinct consonants, i.e., /ś/ and /š/. Traditionally, \foreignlanguage{hebrew}{ש} representing /ś/ is pronounced as a voiceless sibilant [s]. Its original pronunciation was probably that of a lateral-fricative sibilant [ɬ], i.e., a sibilant [s] combined with an [l]-element. The graphical distinction between /ś/ and /š/ was only introduced by the Masoretes by placing a diacritic point either on the right or the left upper corner of the letter. The result were two distinct letters (or graphemes) with /ś/ being represented by \foreignlanguage{hebrew}{שׂ} and /š/ by \foreignlanguage{hebrew}{שׁ}. Before the introduction of the diacritic points only the correct understanding of a text made it possible for the reader to distinguish between the two consonants. The letter \foreignlanguage{hebrew}{שׁ} is much more frequent in Hebrew than the letter \foreignlanguage{hebrew}{שׂ}.

It is important to distinguish letters that look alike and consonants that sound alike or identical in the applied reading tradition.

\section{\textit{Bəgadkəfat} Letters}

The letters \foreignlanguage{hebrew}{ב}, \foreignlanguage{hebrew}{כ} and \foreignlanguage{hebrew}{פ} have a double pronunciation in the table above. When they are preceded by a consonant or when they are doubled (geminated), they are pronounced as plosives whereas they are pronounced as fricatives when they are preceded by a vocalic element, either vowel or a reduced vowel.

\medskip

\begin{center}
	\begin{tabular}{|r|c|c|}
		\hline
		& plosive & fricative \\
		\hline
		\foreignlanguage{hebrew}{ב} & \textit{b} & \textit{ḇ/v} \\
		\foreignlanguage{hebrew}{ך} ,\foreignlanguage{hebrew}{כ} & \textit{k} & \textit{ḵ} \\
		\foreignlanguage{hebrew}{ף} ,\foreignlanguage{hebrew}{פ} & \textit{p} & \textit{p̄/f} \\
		\hline
	\end{tabular}
\end{center}

\bigskip

\noindent Final \foreignlanguage{hebrew}{ף} is always pronounced as a fricative. Final \foreignlanguage{hebrew}{ך} is mostly pronounced as a fricative but it can represent a plosive sound, too.

In Masoretic Hebrew three more consonants had a double pronunciation, i.e., as plosives after a consonant or when doubled or as fricatives after a vowel.

\medskip

\begin{center}
	\begin{tabular}{|r|c|cl|}
		\hline
		& plosive & fricative & \\
		\hline
		\foreignlanguage{hebrew}{ג} & \textit{g} & \textit{ḡ} & like \foreignlanguage{greek}{γ} in Mod. Gr. \foreignlanguage{greek}{γάλα}\\
		\foreignlanguage{hebrew}{ד} & \textit{d} & \textit{ḏ} & like \textit{th} in Engl. \textit{this} \\
		\foreignlanguage{hebrew}{ת} & \textit{t} & \textit{ṯ} & like \textit{th} in Engl. \textit{thing} \\
		\hline
	\end{tabular}
\end{center}

\bigskip

As a group, these six letters are called \foreignlanguage{hebrew}{בְּגַדְכְּפַת} \textit{bəgadkəfat} letters (\foreignlanguage{hebrew}{אותיות בג״ד כפ״ת}).

In Modern Hebrew, only the letters \foreignlanguage{hebrew}{ב}, \foreignlanguage{hebrew}{כ} and \foreignlanguage{hebrew}{פ} have a double pronunciation. This distinction is followed for the pronunciation of these letters in Biblical Hebrew. The double pronunciation of \foreignlanguage{hebrew}{ג}, \foreignlanguage{hebrew}{ד} and \foreignlanguage{hebrew}{ת} in Masoretic Hebrew is usually ignored.

In pointed (or vocalized) Hebrew (see the next chapter), a dot called \textit{dageš} is added to \textit{bəgadkəfat} letters to distinguish the plosive pronunciation (with \textit{dageš}) from the fricative pronunciation (without \textit{dageš}).

\section{Gutturals}
The four consonants \foreignlanguage{hebrew}{א},  \foreignlanguage{hebrew}{ה},  \foreignlanguage{hebrew}{ח} and \foreignlanguage{hebrew}{ע} represent guttural sounds which means that they are articulated in the throat. (The term \textit{guttural} is derived from Latin \textit{guttur throat}.)

The letter \foreignlanguage{hebrew}{א} stands for the glottal stop. In English the glottal stop is found in the negative \textit{uh-uh} and in Tagalog between the two vowels of the word \textit{daan}. At the end of syllables it became silent (quiescent) in most cases, e.g., \foreignlanguage{hebrew}{צבא} \textit{ṣāḇā(ʾ)} \textit{army}. In this grammar, the quiescient \foreignlanguage{hebrew}{א} is transcribed \textit{(ʾ)} to indicate the presence of the letter \foreignlanguage{hebrew}{א} and the fact that is no longer a consonant.

The  \foreignlanguage{hebrew}{ח} /ḥ/ is a fricative sound that corresponds to /ḥ/ in Arabic, e.g., \textit{Muhammad}. As in Modern Hebrew, the /ḥ/ is usually pronounced like the fricative \foreignlanguage{hebrew}{כ} \textit{ḵ} as the final \textit{-ch} in the name of the German composer Bach or the Scottish-Gaelic word \textit{loch} (as in Loch Ness, the name of a lake on Scotland).

The \foreignlanguage{hebrew}{ע} is the voiced counterpart to the \foreignlanguage{hebrew}{ח} /ḥ/. It is the same sound as the initial sound of the Arabic word \textit{ʿayn} \enquote{eye}. As in Modern Hebrew, the \foreignlanguage{hebrew}{ע} usually pronounced as the \foreignlanguage{hebrew}{א}.  For a clearer distinction of the two consonants it is recommended to pronounce the \foreignlanguage{hebrew}{ע} with more force than the \foreignlanguage{hebrew}{א}.


\section{Emphatic Consonants}
The three consonants \foreignlanguage{hebrew}{ט}, \foreignlanguage{hebrew}{צ} and \foreignlanguage{hebrew}{ק} are called emphatic because they are \enquote{articulated farther back in the mouth in the region called the soft palate, and with greater tension of the articulary organs than is the case for the non-emphatics} (JM §\,5\textit{i}).

\medskip

	\begin{center}
		\begin{tabular}{|l|c|c|}
			\hline
			& emphatic & non-emphatic \\
			\hline
			Dental  & \foreignlanguage{hebrew}{ט} \textit{ṭ} & \foreignlanguage{hebrew}{ת} \textit{t} \\
			Palatal-velar & \foreignlanguage{hebrew}{ק} \textit{q} & \foreignlanguage{hebrew}{כ} \textit{k} \\
			Sibilant  & \foreignlanguage{hebrew}{צ} \textit{ṣ} & \foreignlanguage{hebrew}{ס} \textit{s} \\
			\hline
		\end{tabular}
	\end{center}

	\bigskip

The emphatics \foreignlanguage{hebrew}{ט} and \foreignlanguage{hebrew}{ק} should be pronounced with more force or emphasis than their non-emphatic counterparts \foreignlanguage{hebrew}{ת} and \foreignlanguage{hebrew}{כ} to avoid confusion. The traditional pronunciation of \foreignlanguage{hebrew}{צ} as \textit{ts} makes confusion with \foreignlanguage{hebrew}{ס} \textit{s} impossible.


\section{Semivowels}
The letters \foreignlanguage{hebrew}{ו} and \foreignlanguage{hebrew}{י} represent so-called semivowels (sometimes also called semi-consonants). Technically, the semivowels are considered consonants. Originally, \foreignlanguage{hebrew}{ו} was pronounced as a bilabial sound \textit{w} as in Engl.\ \textit{well}. In Modern Hebrew it is pronounced as /v/ as in Engl. \textit{vase}. Its transliteration, however, is \textit{w}.

% The term semi-consonants is used in Blau, Phonology and Morphology

Semivowels often interact with vowels by merging with them to form long vowels. This process is called contraction.

\section{Vowel Letters}
Originally, the letters of the Hebrew alphabet only indicated consonants. As a consequence, readers of Hebrew texts had to supply all vowels by using their knowledge of the language.  Before 700 BCE, however, scribes introduced the use of three consonants for indicating vowels.

The letters \foreignlanguage{hebrew}{ו} and \foreignlanguage{hebrew}{י} are used to indicate long vowels in the middle of words and at the end of words. The letter \foreignlanguage{hebrew}{ה} only indicates vowels at the end of words. Originally, vowel letters were only used to indicate long vowels. The following vowels are indicated by these three vowel letters:

\bigskip

\begin{center}

\begin{tabular}{>{\raggedleft}p{0.1\linewidth} p{0.70\linewidth}}
	\foreignlanguage{hebrew}{ו} & \textit{ō}, \textit{ū}, e.g., \foreignlanguage{hebrew}{יום} \textit{yōm} \textit{day}, \foreignlanguage{hebrew}{סוס} \textit{sūs} \textit{horse}   \\
	\foreignlanguage{hebrew}{י} & \textit{ī}, \textit{ē}, \textit{ǣ}, e.g., \foreignlanguage{hebrew}{איש} \textit{ʾīš} \textit{man}, \foreignlanguage{hebrew}{בין} \textit{bēn} \textit{between}, \foreignlanguage{hebrew}{בניך} \textit{bānǣḵā} \textit{your sons} \\
	\foreignlanguage{hebrew}{ה} & \textit{ā}, \textit{ǣ}, \textit{ē}, \textit{ō} (only at the end of words), e.g., \foreignlanguage{hebrew}{אשה} \textit{ʾiššā} \textit{woman}, \foreignlanguage{hebrew}{זה} \textit{zæ̂} \textit{this}, \foreignlanguage{hebrew}{הנה} \textit{hinnê} \textit{look!}, \foreignlanguage{hebrew}{כה} \textit{kô} \textit{thus}, \\
\end{tabular}

\end{center}

\bigskip

When vowels are written with a vowel letter, it is called \textit{plene} spelling (Latin \textit{scriptio plena} from Hebrew \foreignlanguage{hebrew}{כתיב מלא} \textit{kətīḇ mālē(ʾ)} \textit{full spelling}). When vowels that could be spelled \textit{plene}, are spelled without vowel letter it is called defective spelling (Latin \textit{scriptio defectiva} from Hebrew \foreignlanguage{hebrew}{כתיב מלא} \textit{kətīḇ ḥāsēr} \textit{defective spelling}).

Sometimes, vowel letters are used for short vowels, e.g., \foreignlanguage{hebrew}{כולם} \textit{kullām} \textit{all of them} (Jer 31:34; the form is otherwise spelled \foreignlanguage{hebrew}{כלם}͏͏͏).

Traditional terms for vowel letters are the Latin \textit{mater lectionis} (singular) and \textit{matres lectionis} (plural) which are calques of the Hebrew terms \foreignlanguage{hebrew}{אם קריאה} \textit{ʾēm qərīʾā} \textit{mother of reading} (singular) and \foreignlanguage{hebrew}{אמות קריאה} \textit{ʾimmōt qərīʾā} \textit{mothers of reading} (plural), respectively. In English, these terms are often shortened to \textit{mater} or \textit{matres}.

The letter \foreignlanguage{hebrew}{א} is not a vowel letter in the proper sense. At the end of syllables /ʾ/ often became silent -- this happened always at the end of words -- but was retained in writing so that vowels appear to be written with \foreignlanguage{hebrew}{א}. But unlike \foreignlanguage{hebrew}{ו} and \foreignlanguage{hebrew}{י} which can be used as vowel letters even in positions where there was no semivowel /w/ or /y/ before, silent \foreignlanguage{hebrew}{א} only occurs where there was once a consonantal /ʾ/ (exceptions are possible).

\section{Position of the Stress}
In Masoretic Hebrew, most words are stressed on the last (or ultimate) syllable. Exceptions to this are nouns with originally two final consonants (see Chapter 2) and a number of verbal forms.

In this grammar, the position of stress of a Hebrew word is indicated by the sign \foreignlanguage{hebrew}{ ֫} on the stressed syllable when it is useful, e.g., \foreignlanguage{hebrew}{בָּאָ֫רֶץ} \textit{bāʾā́ræṣ} \textit{in the land}.


\section{Numerical Values of Letters}
The numerical value for Hebrew letters is a feature of postbiblical times. In Hebrew inscriptions from the First Temple period, hieratic number signs were used. In extant biblical manuscripts, number words are used instead of number signs. In English, this corresponds to using the word \textit{seventy} for the number 70.


\section{Exercises}

\subsection{Proper Nouns without Vowel Letters}
Identify the following proper nouns from the Hebrew Bible. The names do not contain vowel letters; every Hebrew character indicates a consonant. Use the vowels a, e, i, o, u (as they sound in Tagalog). Reminder: The letter \foreignlanguage{hebrew}{ש} may represent either the sound /š/ or the sound /ś/.

\subsubsection{Names of Persons}

\selectlanguage{hebrew}
\noindent
1~~\foreignlanguage{hebrew}{אברהם} \hspace{0.20cm}
2~~\foreignlanguage{hebrew}{אבשלם} \hspace{0.20cm}
3~~\foreignlanguage{hebrew}{אדם} \hspace{0.20cm}
4~~\foreignlanguage{hebrew}{אהרן} \hspace{0.20cm}
5~~\foreignlanguage{hebrew}{אחאב} \hspace{0.20cm}
6~~\foreignlanguage{hebrew}{אסף} \hspace{0.20cm}
7~~\foreignlanguage{hebrew}{אסתר} \hspace{0.20cm}
8~~\foreignlanguage{hebrew}{אפרים} \hspace{0.20cm}
9~~\foreignlanguage{hebrew}{אשר} \hspace{0.20cm}
10~~\foreignlanguage{hebrew}{בלעם} \hspace{0.20cm}
11~~\foreignlanguage{hebrew}{בנימן} \hspace{0.20cm}
12~~\foreignlanguage{hebrew}{ברק} \hspace{0.20cm}
13~~\foreignlanguage{hebrew}{גד} \hspace{0.20cm}
14~~\foreignlanguage{hebrew}{גלית} \hspace{0.20cm}
15~~\foreignlanguage{hebrew}{דוד} \hspace{0.20cm}
16~~\foreignlanguage{hebrew}{דן} \hspace{0.20cm}
17~~\foreignlanguage{hebrew}{הבל} \hspace{0.20cm}
18~~\foreignlanguage{hebrew}{הגר} \hspace{0.20cm}
19~~\foreignlanguage{hebrew}{חגי} \hspace{0.20cm}
20~~\foreignlanguage{hebrew}{יעקב} \hspace{0.20cm}
21~~\foreignlanguage{hebrew}{יפת} \hspace{0.20cm}
22~~\foreignlanguage{hebrew}{יפתח} \hspace{0.20cm}
23~~\foreignlanguage{hebrew}{יצחק} \hspace{0.20cm}
24~~\foreignlanguage{hebrew}{ירבעם} \hspace{0.20cm}
25~~\foreignlanguage{hebrew}{ישראל} \hspace{0.20cm}
26~~\foreignlanguage{hebrew}{כלב} \hspace{0.20cm}
27~~\foreignlanguage{hebrew}{לבן} \hspace{0.20cm}
28~~\foreignlanguage{hebrew}{למך} \hspace{0.20cm}
29~~\foreignlanguage{hebrew}{מנחם} \hspace{0.20cm}
30~~\foreignlanguage{hebrew}{מרים} \hspace{0.20cm}
31~~\foreignlanguage{hebrew}{נבכדנצר} \hspace{0.20cm}
32~~\foreignlanguage{hebrew}{נעמן} \hspace{0.20cm}
33~~\foreignlanguage{hebrew}{נתן} \hspace{0.20cm}
34~~\foreignlanguage{hebrew}{נתנאל} \hspace{0.20cm}
35~~\foreignlanguage{hebrew}{עשו} \hspace{0.20cm}
36~~\foreignlanguage{hebrew}{קין} \hspace{0.20cm}
37~~\foreignlanguage{hebrew}{רחבעם} \hspace{0.20cm}
38~~\foreignlanguage{hebrew}{רחל} \hspace{0.20cm}
39~~\foreignlanguage{hebrew}{שם} \hspace{0.20cm}
40~~\foreignlanguage{hebrew}{תמר} \hspace{0.20cm}
\selectlanguage{english}

\subsubsection{Geographical Names}

\selectlanguage{hebrew}
1~~\foreignlanguage{hebrew}{אררט}  \hspace{0.20cm}
2~~\foreignlanguage{hebrew}{אר~שבע}  \hspace{0.20cm}
3~~\foreignlanguage{hebrew}{גלגל}  \hspace{0.20cm}
4~~\foreignlanguage{hebrew}{גלעד}  \hspace{0.20cm}
5~~\foreignlanguage{hebrew}{גת}  \hspace{0.20cm}
6~~\foreignlanguage{hebrew}{דמשק}  \hspace{0.20cm}
7~~\foreignlanguage{hebrew}{דן}  \hspace{0.20cm}
8~~\foreignlanguage{hebrew}{חרן}  \hspace{0.20cm}
9~~\foreignlanguage{hebrew}{ירדן}  \hspace{0.20cm}
10~~\foreignlanguage{hebrew}{כנען}  \hspace{0.20cm}
11~~\foreignlanguage{hebrew}{כרמל}  \hspace{0.20cm}
12~~\foreignlanguage{hebrew}{מדין}  \hspace{0.20cm}
13~~\foreignlanguage{hebrew}{סדם}  \hspace{0.20cm}
14~~\foreignlanguage{hebrew}{עדן}  \hspace{0.20cm}
15~~\foreignlanguage{hebrew}{שכם}  \hspace{0.20cm}
\selectlanguage{english}


\subsection{Proper Nouns with Vowel Letters}

Identify the following proper nouns with vowel letters from the Hebrew Bible. Use the vowels a, e, i, o, u (as they sound in Tagalog). Reminder: The letter \foreignlanguage{hebrew}{ש} may represent either the sound /š/ or the sound /ś/.

\subsubsection{Names of Persons}

\selectlanguage{hebrew}
1~~\foreignlanguage{hebrew}{אביגיל}  \hspace{0.20cm}
2~~\foreignlanguage{hebrew}{אבימלך}  \hspace{0.20cm}
3~~\foreignlanguage{hebrew}{אבשלום}  \hspace{0.20cm}
4~~\foreignlanguage{hebrew}{אליעזר}  \hspace{0.20cm}
5~~\foreignlanguage{hebrew}{אלישע}  \hspace{0.20cm}
6~~\foreignlanguage{hebrew}{ברוך}  \hspace{0.20cm}
7~~\foreignlanguage{hebrew}{גדעון}  \hspace{0.20cm}
8~~\foreignlanguage{hebrew}{דויד}  \hspace{0.20cm}
9~~\foreignlanguage{hebrew}{זבולן}  \hspace{0.20cm}
10~~\foreignlanguage{hebrew}{זכריה}  \hspace{0.20cm}
11~~\foreignlanguage{hebrew}{חנה}  \hspace{0.20cm}
12~~\foreignlanguage{hebrew}{חפני}  \hspace{0.20cm}
13~~\foreignlanguage{hebrew}{יהוא}  \hspace{0.20cm}
14~~\foreignlanguage{hebrew}{יהודה}  \hspace{0.20cm}
15~~\foreignlanguage{hebrew}{יהונתן}  \hspace{0.20cm}
16~~\foreignlanguage{hebrew}{יהורם}  \hspace{0.20cm}
17~~\foreignlanguage{hebrew}{יהושע}  \hspace{0.20cm}
18~~\foreignlanguage{hebrew}{יהושפט}  \hspace{0.20cm}
19~~\foreignlanguage{hebrew}{יואב}  \hspace{0.20cm}
20~~\foreignlanguage{hebrew}{יואל}  \hspace{0.20cm}
21~~\foreignlanguage{hebrew}{יונה}  \hspace{0.20cm}
22~~\foreignlanguage{hebrew}{יוסף}  \hspace{0.20cm}
23~~\foreignlanguage{hebrew}{ירמיה}  \hspace{0.20cm}
24~~\foreignlanguage{hebrew}{ישעיה}  \hspace{0.20cm}
25~~\foreignlanguage{hebrew}{לאה}  \hspace{0.20cm}
26~~\foreignlanguage{hebrew}{לוט}  \hspace{0.20cm}
27~~\foreignlanguage{hebrew}{לוי}  \hspace{0.20cm}
28~~\foreignlanguage{hebrew}{מיכל}  \hspace{0.20cm}
29~~\foreignlanguage{hebrew}{מנשה}  \hspace{0.20cm}
30~~\foreignlanguage{hebrew}{משה}  \hspace{0.20cm}
31~~\foreignlanguage{hebrew}{נחמיה}  \hspace{0.20cm}
32~~\foreignlanguage{hebrew}{נעמי}  \hspace{0.20cm}
33~~\foreignlanguage{hebrew}{נפתלי}  \hspace{0.20cm}
34~~\foreignlanguage{hebrew}{עזרא}  \hspace{0.20cm}
35~~\foreignlanguage{hebrew}{עלי}  \hspace{0.20cm}
36~~\foreignlanguage{hebrew}{עמוס}  \hspace{0.20cm}
37~~\foreignlanguage{hebrew}{פינחס}  \hspace{0.20cm}
38~~\foreignlanguage{hebrew}{צפניה}  \hspace{0.20cm}
39~~\foreignlanguage{hebrew}{ראובן}  \hspace{0.20cm}
40~~\foreignlanguage{hebrew}{רבקה}  \hspace{0.20cm}
41~~\foreignlanguage{hebrew}{רות}  \hspace{0.20cm}
42~~\foreignlanguage{hebrew}{שאול}  \hspace{0.20cm}
43~~\foreignlanguage{hebrew}{שלמה}  \hspace{0.20cm}
44~~\foreignlanguage{hebrew}{שמואל}  \hspace{0.20cm}
45~~\foreignlanguage{hebrew}{שמעון}  \hspace{0.20cm}
46~~\foreignlanguage{hebrew}{שמעי}  \hspace{0.20cm}
47~~\foreignlanguage{hebrew}{שמשון}  \hspace{0.20cm}
48~~\foreignlanguage{hebrew}{שרה}  \hspace{0.20cm}

\selectlanguage{english}


\subsubsection{Geographical Names}

\selectlanguage{hebrew}
1~~\foreignlanguage{hebrew}{אדום}  \hspace{0.20cm}
2~~\foreignlanguage{hebrew}{אור}  \hspace{0.20cm}
3~~\foreignlanguage{hebrew}{אשור}  \hspace{0.20cm}
4~~\foreignlanguage{hebrew}{בית אל}  \hspace{0.20cm}
5~~\foreignlanguage{hebrew}{חברון}  \hspace{0.20cm}
6~~\foreignlanguage{hebrew}{יריחו}  \hspace{0.20cm}
7~~\foreignlanguage{hebrew}{לבנון}  \hspace{0.20cm}
8~~\foreignlanguage{hebrew}{לכיש}  \hspace{0.20cm}
9~~\foreignlanguage{hebrew}{מגדו}  \hspace{0.20cm}
10~~\foreignlanguage{hebrew}{מואב}  \hspace{0.20cm}
11~~\foreignlanguage{hebrew}{סיני}  \hspace{0.20cm}
12~~\foreignlanguage{hebrew}{עמון}  \hspace{0.20cm}
13~~\foreignlanguage{hebrew}{עמרה}  \hspace{0.20cm}
14~~\foreignlanguage{hebrew}{ציון}  \hspace{0.20cm}
15~~\foreignlanguage{hebrew}{שמרון}  \hspace{0.20cm}
\selectlanguage{english}

\subsection{Modern Hebrew}

Identify the following foreign words in Modern Hebrew (with and without vowel letters). You know them from English or Tagalog.

\subsubsection{Foreign Words in Modern Hebrew without Vowel Letters}

\selectlanguage{hebrew}
\noindent
\textbf{1}~~\foreignlanguage{hebrew}{בנק} \hspace{0.15cm}
\textbf{2}~~\foreignlanguage{hebrew}{בר} \hspace{0.15cm}
\textbf{3}~~\foreignlanguage{hebrew}{גז} \hspace{0.15cm}
\textbf{4}~~\foreignlanguage{hebrew}{גנרל} \hspace{0.15cm}
\textbf{5}~~\foreignlanguage{hebrew}{טנק} \hspace{0.15cm}
\textbf{6}~~\foreignlanguage{hebrew}{מגנט} \hspace{0.15cm}
\textbf{7}~~\foreignlanguage{hebrew}{מטר} \hspace{0.15cm}
\textbf{8}~~\foreignlanguage{hebrew}{סלט} \hspace{0.15cm}
\textbf{9}~~\foreignlanguage{hebrew}{פנתר} \hspace{0.15cm}
\textbf{10}~~\foreignlanguage{hebrew}{פרק} \hspace{0.15cm}
\textbf{11}~~\foreignlanguage{hebrew}{צמנט} \hspace{0.15cm}
\selectlanguage{english}


\subsubsection{Foreign Words in Modern Hebrew with Vowel Letters}

\selectlanguage{hebrew}
\noindent
\textbf{1}~~\foreignlanguage{hebrew}{שימפנזה} \hspace{0.15cm}
\textbf{2}~~\foreignlanguage{hebrew}{גורילה} \hspace{0.15cm}
\textbf{3}~~\foreignlanguage{hebrew}{פינגוין} \hspace{0.15cm}
\textbf{4}~~\foreignlanguage{hebrew}{קנגורו} \hspace{0.15cm}
\textbf{5}~~\foreignlanguage{hebrew}{יגואר} \hspace{0.15cm}
\textbf{6}~~\foreignlanguage{hebrew}{סטודנט} \hspace{0.15cm}
\textbf{7}~~\foreignlanguage{hebrew}{פרופסור} \hspace{0.15cm}
\textbf{8}~~\foreignlanguage{hebrew}{דוקטור} \hspace{0.15cm}
\textbf{9}~~\foreignlanguage{hebrew}{מיניסטר} \hspace{0.15cm}
\textbf{10}~~\foreignlanguage{hebrew}{דיפלומט} \hspace{0.15cm}
\textbf{11}~~\foreignlanguage{hebrew}{קונסול} \hspace{0.15cm}
\textbf{12}~~\foreignlanguage{hebrew}{בישוף} \hspace{0.15cm}
\textbf{13}~~\foreignlanguage{hebrew}{מונרך} \hspace{0.15cm}
\textbf{14}~~\foreignlanguage{hebrew}{מיליונר} \hspace{0.15cm}
\textbf{15}~~\foreignlanguage{hebrew}{אידיוט} \hspace{0.15cm}
\textbf{16}~~\foreignlanguage{hebrew}{דמוקרט} \hspace{0.15cm}
\textbf{17}~~\foreignlanguage{hebrew}{טרוריסט} \hspace{0.15cm}
\textbf{18}~~\foreignlanguage{hebrew}{סוציאליסט} \hspace{0.15cm}
\textbf{19}~~\foreignlanguage{hebrew}{קומוניסט} \hspace{0.15cm}
\textbf{20}~~\foreignlanguage{hebrew}{אופוזיציה} \hspace{0.15cm}
\textbf{21}~~\foreignlanguage{hebrew}{אופטימיסט} \hspace{0.15cm}
\textbf{22}~~\foreignlanguage{hebrew}{פסימיסט} \hspace{0.15cm}
\textbf{23}~~\foreignlanguage{hebrew}{ריאליסט} \hspace{0.15cm}
\textbf{24}~~\foreignlanguage{hebrew}{פרוטסטנט} \hspace{0.15cm}
\textbf{25}~~\foreignlanguage{hebrew}{לוקומוטיב} \hspace{0.15cm}
\textbf{26}~~\foreignlanguage{hebrew}{רדיו} \hspace{0.15cm}
\textbf{27}~~\foreignlanguage{hebrew}{טלפון} \hspace{0.15cm}
\textbf{28}~~\foreignlanguage{hebrew}{פילם} \hspace{0.15cm}
\textbf{29}~~\foreignlanguage{hebrew}{הומור} \hspace{0.15cm}
\textbf{30}~~\foreignlanguage{hebrew}{דרמה} \hspace{0.15cm}
\textbf{31}~~\foreignlanguage{hebrew}{קונצרט} \hspace{0.15cm}
\textbf{32}~~\foreignlanguage{hebrew}{קפה} \hspace{0.15cm}
\textbf{33}~~\foreignlanguage{hebrew}{קופיאין} \hspace{0.15cm}
\textbf{34}~~\foreignlanguage{hebrew}{תה} \hspace{0.15cm}
\textbf{35}~~\foreignlanguage{hebrew}{אלכוהול} \hspace{0.15cm}
\textbf{36}~~\foreignlanguage{hebrew}{קוניק} \hspace{0.15cm}
\textbf{37}~~\foreignlanguage{hebrew}{רום} \hspace{0.15cm}
\textbf{38}~~\foreignlanguage{hebrew}{פיקניק} \hspace{0.15cm}
\textbf{39}~~\foreignlanguage{hebrew}{שוקולד} \hspace{0.15cm}
\textbf{40}~~\foreignlanguage{hebrew}{מקרוני} \hspace{0.15cm}
\textbf{41}~~\foreignlanguage{hebrew}{ספורט} \hspace{0.15cm}
\textbf{42}~~\foreignlanguage{hebrew}{פינג־פונג} \hspace{0.15cm}
\textbf{43}~~\foreignlanguage{hebrew}{גולף} \hspace{0.15cm}
\textbf{44}~~\foreignlanguage{hebrew}{טניס} \hspace{0.15cm}
\selectlanguage{english}


\chapter{Chapter 2}

\section{The Vocalization of the Hebrew Bible}

The Masoretes, the Jewish scholars who worked on the transmission of the Hebrew Bible between 700 and 1000 CE, added an elaborate system of vowel signs and other signs to the biblical text to ensure its correct pronunciation. Each vowel was now indicated by a sign. As a result, some vowels are indicated twice -- once with a vowel letter and once with the vowel sign that was added by the Masoretes.

During the period from 700--1000~CE three competing vocalization systems were developed, the Tiberian, the Babylonian and the Palestinian system. In the history of the transmission of the Hebrew Bible, the Tiberian system became the standard vocalization system. It consists of sublinaer and supralinear vowel signs and additional signs to indicate the absence of a vowel, the plosive pronunciation of \textit{bəgadkəfat} letters and the doubling of consonants and other phonetic features.  The Hebrew term for the vocalization system is \foreignlanguage{hebrew}{ניקוד} (vocalized \foreignlanguage{hebrew}{נִקּוּד}) \textit{niqqūd}.

The existing biblical text to which the Masoretes added the pointing is often called the consonantal text. This term is not accurate because the biblical text contained vowel letters, i.e., letters that indicated vowels and not consonants. In this work, the term \enquote{unpointed text} is used instead of \enquote{consonantal text.}

\section{Full Vowels}

% The transliteration of vowels is taken from Blau, Phonology and Morphology, p. 66-67

Biblical Hebrew according to the Tiberian vocalization system has seven distinct vowels. Originally, the vowel signs only indicated vowel quality but not vowel quantity (length). The Tiberian system was later reinterpreted according to another reading tradition. This reinterpretation is presented here.

\renewcommand\arraystretch{1.4}

\begin{Center}
	\begin{tabular}{|c|c|c|c|l|r|l|}
		\hline
		I & II & III & IV & \multicolumn{1}{c|}{V}  & \multicolumn{2}{c|}{VI} \\
		\hline
		\foreignlanguage{hebrew}{מִ} & i & i & \textit{i, ī} & kit, breeze & \foreignlanguage{hebrew}{חִ֫ירֶק} & \textit{ḥī́ræq}  \\
		\foreignlanguage{hebrew}{מֵ} & e & ɛ & \textit{ē, e} & bet & \foreignlanguage{hebrew}{צֵרֵי} & \textit{ṣērē} \\
		\foreignlanguage{hebrew}{מֶ} & æ & ɛ & \textit{æ, ǣ} & hat & \foreignlanguage{hebrew}{סְגוֺל} & \textit{səgōl} \\
		\foreignlanguage{hebrew}{מַ} & a & a & \textit{a} & cut & \foreignlanguage{hebrew}{פַּ֫תַח} & \textit{pátaḥ} \\
		\foreignlanguage{hebrew}{מָ} & ɔ & a, ɔ & \textit{ā, ɔ} & balm [bɑːm], got & \foreignlanguage{hebrew}{קָ֫מֶץ} & \textit{qā́mæṣ} \\
		\foreignlanguage{hebrew}{מֹ} & o & o & \textit{ō, o} & as British law, got & \foreignlanguage{hebrew}{ח֫וֺלֶם} & \textit{ḥṓlæm} \\
		\foreignlanguage{hebrew}{מֻ} & u & u & \textit{u, ū} & rule & \foreignlanguage{hebrew}{קִבּוּץ} & \textit{qibbūṣ} \\
		\foreignlanguage{hebrew}{וּ} & u & u & \textit{ū} & rule & \foreignlanguage{hebrew}{שׁ֫וּרֶק} & \textit{šū́ræq} \\
		\hline
	\end{tabular}
\end{Center}

\newpage

\noindent \textbf{Key to the Table}

\vspace{0.25cm}

\renewcommand\arraystretch{1}

\begin{tabular}{p{0.025\linewidth}p{0.85\linewidth}}
	I & Vowel Sign \\
	II & Masoretic Vowel Quality \\
	III & Modern Hebrew \\
	IV & Approximate Pronunciation \\
	V & Hebrew Name \\
\end{tabular}

\renewcommand\arraystretch{1.4}

\vspace{0.5cm}

\noindent \textbf{Notes}
\nopagebreak

\noindent For grammatical explanations a simplified transliteration of the names of vowels and other signs will be used (\textit{ḥireq}, \textit{ṣere}, \textit{segol}, \textit{pataḥ}, \textit{qameṣ}, \textit{ḥolem}, \textit{qibbuṣ}, \textit{šureq}).

Vowels are pronounced after the consonant they are placed with. The combination \foreignlanguage{hebrew}{מַ}, for example, stands thus for /ma/.

The approximate pronunciation does not distinguish vowel length. Vowel quality is indicated only by approximation.

The \textit{qameṣ} indicating \textit{ā} is called \textit{qāmæṣ gādōl} (simplified \textit{qameṣ gadol}). The \textit{qameṣ} indicating \textit{ɔ} is called \textit{qāmæṣ qāṭān} or \textit{qāmæṣ ḥāṭūf} (simplified \textit{qameṣ qaṭan} and \textit{qameṣ ḥaṭuf}, respectively).

For the vowel /ū/ the sign \textit{qibbuṣ} is used when there is no vowel letter whereas \textit{šureq} is used when there is a vowel \foreignlanguage{hebrew}{ו} in the unpointed text, e.g., \foreignlanguage{hebrew}{פִּקּוּדֶ֫יךָ} and \foreignlanguage{hebrew}{פִּקֻּדֶ֫יךָ} \textit{your precepts} (Ps 119:15, 27).

The vowel \textit{ḥolem} is the only vowel sign that is placed on the top left corner of the letter (except with \foreignlanguage{hebrew}{ל}, e.g., \foreignlanguage{hebrew}{חֲלֹם} \textit{dream}). If combined with the vowel letter \foreignlanguage{hebrew}{ו} the \textit{ḥolem} is placed on the  \foreignlanguage{hebrew}{ו}, e.g., \foreignlanguage{hebrew}{לוֺט} \textit{Lot}. If combined with a silent \foreignlanguage{hebrew}{א} the \textit{ḥolem} is placed on the top right corner of the \foreignlanguage{hebrew}{א}, e.g., \foreignlanguage{hebrew}{רֹאשׁ} \textit{head}. In some printed works the \textit{ḥolem} merges with the diacritic point of a following \foreignlanguage{hebrew}{שׁ} or a preceding \foreignlanguage{hebrew}{שׂ}. Examples: \foreignlanguage{hebrew}{משֶׁה} instead of \foreignlanguage{hebrew}{מֹשֶׁה}; \foreignlanguage{hebrew}{שׂנַאֲךָ} instead of \foreignlanguage{hebrew}{שֹׂנַאֲךָ} \textit{one who hates you}.

\section{The Distinction between \textit{qameṣ gadol} and \textit{qameṣ qaṭan}}
In the Tiberian vocalization system as it was intended, the \textit{qameṣ} only represented the vowel \textit{ɔ}. The distinction between \textit{qameṣ gadol} and \textit{qameṣ qaṭan} is the result of a later reinterpretation of the Tiberian vocalization system and is linguistically justified. A \textit{qameṣ} is a \textit{qameṣ qaṭan} (or \textit{qameṣ ḥaṭuf}) in the following cases:

\begin{enumerate}[noitemsep]
	\item In closed, unstressed syllables, i.e., before silent \textit{šwa}, before \textit{maqqef} (see below) and before \textit{dageš forte} (infrequent), \foreignlanguage{hebrew}{אָזְנִי} \textit{ʾɔznī} \textit{my ear} (from sg. \foreignlanguage{hebrew}{אֹ֫זֶן}), \foreignlanguage{hebrew}{שְׁמָר־לְךָ} \textit{šəmɔr ləḵā} \textit{keep} (cf. \foreignlanguage{hebrew}{שְׁמֹר} without \textit{maqqef}), \foreignlanguage{hebrew}{כָּסּוּ} \textit{kɔssū} \textit{they were covered}.
	\item In the first syllable of the plural noun \foreignlanguage{hebrew}{קָֽדָשִׁים} \textit{qɔdāšīm} (usually spelled with \textit{meteg}) because of the vowel \textit{ō} (< \textit{*u}) in the first syllable of the singular form \foreignlanguage{hebrew}{קֹ֫דֶשׁ}.
\end{enumerate}

The vowel \textit{qameṣ qaṭan} \textit{ɔ} always originates from a short /u/, whereas \textit{qameṣ gadol} goes back to an original /a/ or -- less frequently -- /ā/. If a word has the vowel \textit{qameṣ qaṭan} in a certain form, the vowel may appear either as \textit{ḥolem} or \textit{qibbuṣ} in other forms of the word, e.g., \foreignlanguage{hebrew}{חֹק} \textit{ḥōq} \textit{statue}, \foreignlanguage{hebrew}{חֻקִּים} \textit{ḥuqqīm} \textit{statues} (pl.), \foreignlanguage{hebrew}{חָק־} \textit{ḥɔq} \textit{statue of} (construct state).

\vspace{0.5cm}

\noindent \textbf{Note}
\nopagebreak

\noindent The plural of the noun \foreignlanguage{hebrew}{בַּ֫יִת} \textit{báyit} \textit{house} is spelled \foreignlanguage{hebrew}{בָּתִּים} but pronounced \textit{bātīm} with \textit{qameṣ gadol} and a single plosive /t/.


\section{The \textit{šwa}}

The sign \textit{šəwā} (simplified \textit{šwa}) consists of two dots under the letter that are arranged vertically: \foreignlanguage{hebrew}{͏◌ְ} \space. Phonetically, the \textit{šwa} indicates the lack of a vowel. For the morphological analysis of words three different types of \textit{šwa} can be distinguished, the silent \textit{šwa}, the vocal \textit{šwa} and the medial \textit{šwa}.

\subsection{Silent \textit{šwa}}
The silent \textit{šwa} is used to mark vowelless consonants at the end of closed syllables except at the end of words, e.g., \foreignlanguage{hebrew}{אַבְרָהָם} \textit{ʾaḇrāhām}, \foreignlanguage{hebrew}{מִדְבָּר} \textit{midbār} \textit{desert}.

In three cases silent \textit{šwa} is found at the end of words:

\begin{enumerate}[noitemsep]
	\item Words ending with the consonant /k/, e.g., \foreignlanguage{hebrew}{מַלְאָךְ} \textit{messenger, angel}, \foreignlanguage{hebrew}{אָבִיךְ} \textit{your} (fem.) \textit{father} (but \foreignlanguage{hebrew}{אָבִ֫יךָ} \textit{your} (masc.) \textit{father} with \foreignlanguage{hebrew}{ך} followed by a vowel)
	\item Words with two consonants at the end of the word, e.g., \foreignlanguage{hebrew}{שָׁאַלְתְּ} \textit{you have asked}, \foreignlanguage{hebrew}{וַיֵּשְׁתְּ} \textit{and he drank}
	\item The independent personal pronoun \foreignlanguage{hebrew}{אַתְּ} \textit{ʾat} \textit{you} (fem.\ sg.)
\end{enumerate}

The only words with silent \textit{šwa} at the beginning of the word are \foreignlanguage{hebrew}{שְׁתַּ֫יִם} \textit{štayim} and \foreignlanguage{hebrew}{שְׁנַ֫יִם} \textit{šnayim} \textit{two}.

\subsection{Vocal \textit{šwa}}
The vocal \textit{šwa} is used with the initial consonant of an originally open syllable in which the original short vowel was reduced, e.g., \foreignlanguage{hebrew}{דְּבָרִים} \textit{words} (\textit{< *dabarīm} with a reduced vowel in the first syllable; pl. of \foreignlanguage{hebrew}{דָּבָר} \textit{word} with two full vowels).

If possible, a consonant with vocal \textit{šwa} should be pronounced vowelless. For the sake of convenience, the vocal \textit{šwa} may be pronounced as a hurried \textit{e} as in the first syllable of Engl. \textit{atone} [əˈtoʊn], e.g., \foreignlanguage{hebrew}{בְּגָדִים} \textit{bəgādīm} \textit{clothes} (with two plosives) as opposed to \foreignlanguage{hebrew}{סְפָרִים}, transliterated \textit{səfārīm}, pronounced [sfaːriːm] \textit{books}. Vocal \textit{šwa} is always transliterated \textit{ə} despite its vowelless pronunciation.

The following three criteria help to distinguish a vocal \textit{šwa} from a silent \textit{šwa}. A \textit{šwa} is a vocal \textit{šwa} in the following cases:

\begin{enumerate}[noitemsep]
	\item \textit{šwa} at the beginning of the word, e.g., \foreignlanguage{hebrew}{פְּרִי} \textit{pərī} [priː] \textit{fruit}
	\item \textit{šwa} following \textit{šwa}, e.g., \foreignlanguage{hebrew}{מִשְׁפְּטֵי} \textit{mišpəṭē} \textit{precepts of}
	\item \textit{šwa} following a long vowel, e.g., \foreignlanguage{hebrew}{שֹׁפְטִים} \textit{šōp̄əṭīm} \textit{judges}, \foreignlanguage{hebrew}{אָמְרָה} \textit{ʾāmərā} \textit{she said}
	\item \textit{šwa} with a consonant with \textit{dageš forte} (see below), e.g., \foreignlanguage{hebrew}{דַּבְּרוּ} \textit{dabbərū} \textit{speak!} (impv.\ masc.\ pl.)
\end{enumerate}

\subsection{Medial \textit{šwa}}

When two open syllables with reduced vowels would have to follow each other immediately, the first consonant gets a full short vowel (often /i/) and the second consonant gets a silent \textit{šwa}. This particular \textit{šwa}, however, does not cause the following \textit{bəgadkəfat} letter to become plosive, e.g., \foreignlanguage{hebrew}{כִּזְבֵי} \textit{kizḇē- < *kazaḇē} \textit{lies of somebody} (from \foreignlanguage{hebrew}{כָּזָב} \textit{lie}). In this respect this \textit{šwa} is different from the regular silent \textit{šwa} (cf. \foreignlanguage{hebrew}{מִדְבָּר} \textit{midbār} above) and therefore it is called medial \textit{šwa} or \textit{šwa medium}.

\subsection{Composite \textit{šwa} with Gutturals}

Gutturals do not take simple vocal \textit{šwa}. Instead they take \textit{ḥatef šwa} (also called composite \textit{šwa}). When a guttural would have to take silent \textit{šwa}, it is frequently replaced by \textit{ḥatef šwa} of the same vowel quality as the preceding full vowel.


\begin{Center}
	\begin{tabular}{|c|c|l|r|}
		\hline
		Vowel Sign & Transl. & \multicolumn{2}{c|}{Hebrew Name} \\
		\hline
		\foreignlanguage{hebrew}{חֲ} & \textit{ă} & \textit{ḥaṭef pataḥ} & \foreignlanguage{hebrew}{חָטֵף פַּ֫תַח} \\
		\foreignlanguage{hebrew}{חֱ} & \textit{æ̆} & \textit{ḥaṭef səgol} & \foreignlanguage{hebrew}{חָטֵף סְגוֺל} \\
		\foreignlanguage{hebrew}{חֳ} & \textit{ɔ̆} & \textit{ḥaṭef qamæṣ} & \foreignlanguage{hebrew}{חָטֵף קָ֫מֶץ} \\
		\hline
	\end{tabular}
\end{Center}

\textit{Ḥatef šwa} for for vocal \textit{šwa} is found in the following words: \foreignlanguage{hebrew}{חֲצִי} \textit{ḥăṣī} \textit{half} (cf. \foreignlanguage{hebrew}{פְּרִי} \textit{pərī} \textit{fruit}), \foreignlanguage{hebrew}{אֱמֶת} \textit{ʾæ̆mæt} \textit{truth}, \foreignlanguage{hebrew}{עֳנִי} \textit{ʿɔ̆nī} \textit{misery}.

\textit{Ḥatef šwa} for silent \textit{šwa} is found in the following words: \foreignlanguage{hebrew}{יַעֲמֹד} \textit{yaʿămōd < *yaʿmōd} \textit{he will stand}, \foreignlanguage{hebrew}{יֶאֱהַב} \textit{yæʾæ̆haḇ < *yæʾhaḇ} \textit{he will love}, \foreignlanguage{hebrew}{יָעֳמַד} \textit{yɔʿɔ̆mad < *yɔʿmad} \textit{he shall be placed}.

At times, even non-gutturals may have \textit{ḥatef šwa} instead vocal \textit{šwa}, e.g., \foreignlanguage{hebrew}{סוֺרֲרִים} \textit{sōrărĭm} \textit{stubborn ones} (Ps 68:7).


\section{The \textit{dageš}}

The \textit{dāgēš} (simplified \textit{dageš}) is a point in the middle of the letter: \foreignlanguage{hebrew}{◌ּ} \space. It has two functions. The \textit{dageš} either indicates the doubling of a consonant or the plosive pronunciation of \textit{bəgadkəfat} letters.

\subsection{\textit{dageš forte}}

The \textit{dageš} in a consonant with a preceding full short vowel indicates \emph{doubling} (or \emph{gemination}) of the consonant, i.e., its long pronunciation, e.g., \foreignlanguage{hebrew}{רַכָּב} \textit{rakkāḇ} \textit{charioteer}, \foreignlanguage{hebrew}{עַמִּים} \textit{ʿammīm} \textit{nations}. When used in this way, the \textit{dageš} is called \textit{dageš forte} or \foreignlanguage{hebrew}{דָּגֵשׁ חָזָק}. A \textit{dageš forte} in \textit{bəgadkəfat} letters causes them to be pronounced as plosives, e.g., \foreignlanguage{hebrew}{לִבּוֺ} \textit{libbō} \textit{his heart}.

In the course of the history of Hebrew, gutturals and \foreignlanguage{hebrew}{ר} lost the ability to be doubled (with few exceptional cases of doubled \foreignlanguage{hebrew}{ר}). As a result, the letters \foreignlanguage{hebrew}{א}, \foreignlanguage{hebrew}{ה}, \foreignlanguage{hebrew}{ח}, \foreignlanguage{hebrew}{ע} and \foreignlanguage{hebrew}{ר} do not take \textit{dageš forte}. In Biblical Hebrew, the vowels preceding \foreignlanguage{hebrew}{ה} and \foreignlanguage{hebrew}{ח} that stand in a position where other consonants would be doubled, are normally preserved unchanged despite the fact that they now are in an open syllable (see Syllable Structure below). Vowels preceding \foreignlanguage{hebrew}{א} and \foreignlanguage{hebrew}{ע} are usually lengthened (/a/ > /ā/, /i/ > /ē/, /u/ > /ō/) although preservation of the short vowel may also occur. Before \foreignlanguage{hebrew}{ר} the vowels are always lengthened (/a/ > /ā/, /i/ > /ē/, /u/ > /ō/). Examples:

\begin{center}
	\begin{tabular}{rll}
		\foreignlanguage{hebrew}{רַחוּם} & \textit{raḥūm < *raḥḥūm} & \textit{compassionate} \\
		\foreignlanguage{hebrew}{רָעִים} & \textit{rāʿīm < *raʿʿīm} & \textit{bad, evil} (pl.) \\
		\foreignlanguage{hebrew}{חָרָשׁ} & \textit{ḥārāš < *ḥarrāš} & \textit{craftsman} \\
	\end{tabular}
\end{center}

\subsection{\textit{dageš lene}}

In \textit{bəgadkəfat} letters, the \textit{dageš lene} indicates the plosive pronunciation as /b/, /g/, /d/, /k/, /p/ and /t/, respectively. These letters are pronounced as plosives and are therefore marked with a \textit{dageš lene} when 

\begin{enumerate}[noitemsep]
	\item At the beginning of a sentence, e.g., \foreignlanguage{hebrew}{} \textit{My friend had a vineyard} (Isa 5:1b)
	\item At the beginning of a word following a word ending with a consonant, e.g., \foreignlanguage{hebrew}{} \textit{I raised sons} (Isa 1:2a)
	\item In the middle of a word immediately after a consonant, e.g., \foreignlanguage{hebrew}{מִדְבָּר} \textit{midbār} \textit{desert}, \foreignlanguage{hebrew}{מִשְׁכָּב} \textit{miškāḇ} \textit{place of lying}, \foreignlanguage{hebrew}{מִשְׁפָּט} \textit{mišpāṭ} \textit{judgment}
\end{enumerate}

A \textit{dageš} in a \textit{bəgadkəfat} letter can be either be a \textit{dageš forte} or a \textit{dageš lene}. If a full short vowel precedes the \textit{bəgadkəfat} letter, it is a \textit{dageš forte}. In all other cases it is a \textit{dageš lene}. The Hebrew term for \textit{dageš lene} is \foreignlanguage{hebrew}{דָּגֵשׁ קַל}.

\section{\textit{Mappiq}}
If the \foreignlanguage{hebrew}{ה} at the end of a word is not a vowel letter but indicates a consonant, it is marked by a point in the middle of the \foreignlanguage{hebrew}{ה} called \textit{mappīq} (simplified \textit{mappiq}). Examples: \foreignlanguage{hebrew}{שְׁמָהּ} \textit{šəmāh} \textit{her name}, \foreignlanguage{hebrew}{מַלְכָּהּ} \textit{malkāh} \textit{her king} (as opposed to \foreignlanguage{hebrew}{מַלְכָּה} \textit{malkā} \textit{queen} with the femimine ending \textit{-ā}), \foreignlanguage{hebrew}{אֱלֹהַּ} \textit{ʾæ̆lṓah} \textit{God} (with \textit{furtive pataḥ}).

\section{\textit{Furtive pataḥ}}
Words ending with the gutturals \foreignlanguage{hebrew}{ה}, \foreignlanguage{hebrew}{ח} and \foreignlanguage{hebrew}{ע} preceded by a long vowel other than /ā/ require the insertion of a short vowel /a/ before the guttural. This /a/ is called \textit{furtive pataḥ}. It is placed under the guttural but pronounced before it. The \textit{furtive pataḥ} is \emph{not} stressed. Examples: \foreignlanguage{hebrew}{מָשִׁיחַ} \textit{māšī́aḥ} \textit{anointed one}, \foreignlanguage{hebrew}{רֵעַ} \textit{rḗaʿ} \textit{friend}, \foreignlanguage{hebrew}{גָּבֹהַּ} \textit{gāḇṓah} \textit{high, tall}, \foreignlanguage{hebrew}{רוּחַ} \textit{rū́aḥ} \textit{wind, spirit}.

\section{\textit{Meteg}}
The \textit{méteg} is a short vertical stroke to the left of the vowel sign. It indicates that the pronunciation of the syllable should not be hasty (\textit{meteg} means \textit{bridle}). It often marks the secondary stress of a word or expression. In some cases, it helps distinguish forms that look identical but are to be pronounced differently and have a different meaning. In biblical manuscripts the \textit{meteg} is not used consistently. Examples: \foreignlanguage{hebrew}{וַיִּֽרְאוּ} \textit{wayyīrəʾū} \textit{and they were afraid} (1 Sam 7:7; often the form is spelled plene \foreignlanguage{hebrew}{וַיִּירְאוּ}) as opposed to \foreignlanguage{hebrew}{וַיִּרְאוּ} \textit{wayyirʾū} \textit{and they saw} (never with \textit{meteg}).


\section{\textit{Maqqef}}
The \textit{maqqēf} line (simplified \textit{maqqef}) or \textit{linea maqqef} connects two or more words so that they form a stress unit with only one main stressed syllable in  the last word of the unit, e.g., \foreignlanguage{hebrew}{בֵּית־אָבִ֫יךָ} (Gen 24:23). In many cases the last vowel of the word before the \textit{maqqēf} line is changed, e.g., \foreignlanguage{hebrew}{כְּתֹב} \textit{kətoḇ} \textit{write} (impv.) and \foreignlanguage{hebrew}{כְּתָב־לְךָ} \textit{kətɔḇ ləkā} \textit{write} (impv.) (\foreignlanguage{hebrew}{לְךָ} is extremely difficult to translate into English; Exod 34:27).

\section{Syllable Structure}
Biblical Hebrew has a relatively simple syllable structure. Most syllables are either open syllables that end in a vowel (or vocal \textit{šwa} or \textit{ḥatef šwa}) (CV) or closed syllables that end in a consonant (CVC). Doubly closed syllables ending in two consonants (a so-called consonant cluster) (CVCC) are infrequent. The following table gives an overview of possible syllables:

\bigskip

\begin{center}
	\begin{tabular}{|l|l|}
		\hline
		Open syllable & CV \\
		Closed syllable & CVC \\
		Doubly closed syllable & CVCC \\
		\hline
	\end{tabular}
\end{center}



\bigskip

\noindent For a proper understanding of syllable structure, the following notes are important:

\begin{enumerate}[noitemsep]
	\item Every syllable must begin with a consonant.
	\item Open syllables (CV) include syllables with vocal \textit{šwa} or \textit{ḥatef šwa} for historical reasons. The vocal \textit{šwa} or \textit{ḥatef šwa} are the result of the reduction of short vowels.
	\item In unstressed closed syllables only short vowel are possible.
	\item In unstressed open syllables short vowels are not possible, only long vowels or reduced vowels are possible. An exception to this rule are cases of the loss of gemination in gutturals with preservation of the short vowel, e.g., \foreignlanguage{hebrew}{רַחוּם} \textit{raḥūm < *raḥḥūm} \textit{compassionate}.
	\item Syllables with the structure CVCC only occur at the end of words and almost exclusively as the final syllable of verbal forms.
	\item Exceptions to these syllabification rules are the words for \textit{two} \foreignlanguage{hebrew}{שְׁנַ֫יִם} \textit{šnayim} (masc.) and \foreignlanguage{hebrew}{שְׁתַּ֫יִם} \textit{štayim} (fem.) with two initial consonants.
\end{enumerate}


\section{Segolate Nouns}

% More information is contained in the file Hebrew_Segolate_nouns.odt. The contents have been simplified here.

Biblical Hebrew words are usually stressed on the final syllable. A relatively high number of words, however, are stressed on the penultimate syllable. To this group belong a number of verbal forms and the so-called segolate nouns. As they are relatively frequent it is important to know the patterns of segolate nouns so that their correct pronunciation is ensured.

Originally Hebrew had only three short vowels (/a/, /i/, /u/) and three long vowels. These original vowels were subject to change over time. Segolate nouns give evidence to this.

Segolate nouns are originally monosyllabic nouns with a CVCC syllable structure. After the loss of the final short vowel indicating case (e.g., \textit{*malku > *malk}) in an early phase of Hebrew, the resulting final consonant cluster -CC was broken up by insertion of a helping vowel (also called epenthetic or anaptyptic vowel). In most cases the helping vowel is \textit{segol} (therefore the term \textit{segolate nouns}). Three basic patterns (\textit{*qaṭl},\textit{*qiṭl}, \textit{*quṭl}) can be distinguished.

\begin{center}
	\begin{tabular}{|l|rl|}
		\hline
		\textit{*qaṭl} & \foreignlanguage{hebrew}{מֶ֫לֶךְ} & \textit{king} \\
		\hline
		\textit{*qiṭl} & \foreignlanguage{hebrew}{סֵ֫פֶר} & \textit{book} \\
		\hline
		\textit{*quṭl} & \foreignlanguage{hebrew}{קֹ֫דֶשׁ} & \textit{holiness} \\
		\hline
	\end{tabular}
\end{center}

\noindent If the second or/and the third root consonants are gutturals or semi-vowels, segolate nouns have different vowel patterns than the basic patterns. Except for roots with the semivowel \textit{y} in third position the stress is on the original vowel and thus on the penultimate syllable.

\begin{center}
	\begin{longtable}{|ll|rl|}
		\hline
		\textit{*qaṭl} & & \foreignlanguage{hebrew}{מֶ֫לֶךְ} & \textit{king} \\
		& II\,gutt. & \foreignlanguage{hebrew}{נַ֫עַר} & \textit{boy} \\
		& III\,gutt. & \foreignlanguage{hebrew}{זֶ֫רַע} & \textit{seed} \\
		& II\,\textit{w} & \foreignlanguage{hebrew}{מָ֫וֶת} & \textit{death} \\
		& III\,\textit{w} & \foreignlanguage{hebrew}{שָׂ֫חוּ} & \textit{swimming} \\
		& II\,\textit{y} & \foreignlanguage{hebrew}{זַ֫יִת} & \textit{olive} \\
		& III\,\textit{y} & \foreignlanguage{hebrew}{גְּדִי} & \textit{kid} \\
		\hline
		\textit{*qiṭl} & & \foreignlanguage{hebrew}{סֵ֫פֶר} & \textit{book} \\
		& II\,gutt. & \foreignlanguage{hebrew}{נֵ֫צַח} & \textit{everlastingness} \\
		& III\,\textit{y} & \foreignlanguage{hebrew}{שְׁבִי} & \textit{captivity} \\
		\hline
		\textit{*quṭl} & & \foreignlanguage{hebrew}{קֹ֫דֶשׁ} & \textit{holiness} \\
		& II\,gutt. & \foreignlanguage{hebrew}{פֹּ֫עַל} & \textit{deed, work} \\
		& III\,gutt. & \foreignlanguage{hebrew}{רֹ֫מַח} & \textit{spear} \\
		& III\,\textit{w} & \foreignlanguage{hebrew}{תֹּ֫הוּ} & \textit{wilderness} \\
		& III\,\textit{y} & \foreignlanguage{hebrew}{חֳלִי} & \textit{sickness} \\
		\hline
	\end{longtable}
\end{center}

It is important to be familiar with these patterns. When one encounters them in the biblical text with different consonants, one should be able to recognize the pattern as a segolate formation.

\bigskip

\noindent \textbf{Notes}
\nopagebreak

\noindent In segolate nouns of the \textit{qaṭl} type the original vowel /a/ has been assimilated to the anaptyptic vowel /æ/, e.g., \foreignlanguage{hebrew}{דֶּ֫לֶת} \textit{door}. In roots with a guttural the helping vowel is usually /a/ because of the preference of gutturals for /a/.

Some nouns in the II gutt.\ category have the helping vowel /æ/ instead of /a/, e.g., \foreignlanguage{hebrew}{לֶ֫חֶם} \textit{bread}, \foreignlanguage{hebrew}{רֶ֫חֶם} \textit{womb}, \foreignlanguage{hebrew}{אׂ֫הֶל} \textit{tent}, \foreignlanguage{hebrew}{בֹּ֫הֶן} \textit{thumb}.

The first vowel of pausal forms of \textit{qaṭl} segolate nouns is usually changed to /ā/, e.g., \foreignlanguage{hebrew}{דָּ֑רֶךְ} \textit{way} (context form \foreignlanguage{hebrew}{דֶּ֫רֶךְ}; but \foreignlanguage{hebrew}{מֶ֑לֶךְ} \textit{king})

In some nouns the vowel may shift from the position following the first root consonant to the position between the second and third radical, e.g., \foreignlanguage{hebrew}{דְּבַשׁ} \textit{honey}, \foreignlanguage{hebrew}{שְׁכֶם} \textit{shoulder}, \foreignlanguage{hebrew}{בְּאֵר} \textit{well}.

Segolate endings also appear in polysyllabic nouns, dual endings of nouns, and verbal forms with an original final syllable with the structure CVCC, e.g., \foreignlanguage{hebrew}{תְּכֵ֫לֶת} \textit{purple wool}, \foreignlanguage{hebrew}{נְחׂ֫שֶׁת} \textit{bronze}, \foreignlanguage{hebrew}{יָדַ֫יִם} \textit{hands} (dual), \foreignlanguage{hebrew}{וַיִּ֫בֶן} \textit{he built}, \foreignlanguage{hebrew}{וַיִּשְׁתַּ֫חוּ} \textit{he prostrated}, \foreignlanguage{hebrew}{יְהִי} \textit{may it be} (< \textit{*yihy}).

Proper nouns can be segolate nouns or contain a segolate ending, e.g., \foreignlanguage{hebrew}{יֶ֫רֶד} \textit{Jared}, \foreignlanguage{hebrew}{קַ֫יִן} \textit{Cain}, \foreignlanguage{hebrew}{אֵלִיָּ֫הוּ} \textit{Elijah}.


\section{Pausal Forms}

When the reader or speaker is supposed to slow down at the end of a sentence and pause briefly before continuing with the next sentence, the form of the word my slightly change at times. The meaning of the form is not affected by this change. This happens frequently at the end of a verse or the middle of verse, but it may happen also in other places in a verse. These forms are called \emph{pausal forms} as opposed to the more frequent \emph{context forms}. The following changes may can be observed:

% References for the examples: Ruth 4:2; Gen 14:19; Gen 18:2; Gen 24:52; Gen 4:9; Gen 7:4

\begin{itemize}[noitemsep]
	\item[--] Lengthening of the stressed vowel, e.g., pausal form \foreignlanguage{hebrew}{אָ֫רְצָה} vs. the context form \foreignlanguage{hebrew}{אַ֫רְצָה} \textit{to the ground}
	\item[--] Change of vowel quality, e.g., pausal form \foreignlanguage{hebrew}{אָ֫רֶץ} vs. the context form \foreignlanguage{hebrew}{אֶ֫רֶץ} \textit{land} 
	\item[--] Stress moves to a different syllable, e.g., pausal form \foreignlanguage{hebrew}{אָנֹ֫כִי} vs. the context form \foreignlanguage{hebrew}{אָנֹכִ֫י} \textit{I}
	\item[--] Restoration of a reduced vowel which takes the stress, e.g., pausal form \foreignlanguage{hebrew}{וַיֵּשֵׁ֫בוּ} vs. the context form \foreignlanguage{hebrew}{וַיֵּשְׁב֫וּ} \textit{and they sat down, settled}
\end{itemize}

In this grammar, both context forms and pausal forms are included in the explanation of forms. Pausal forms are indicated by the accent \textit{atnaḥ} \foreignlanguage{hebrew}{ ֑}, e.g., \foreignlanguage{hebrew}{אָנֹ֑כִי}.


\section{Exercises}

\subsection{Proper Nouns}
Read the following proper nouns. Identify the syllable structure of the names 1--20.

\vspace{0.5cm}

\selectlanguage{hebrew}
\noindent
1~~\foreignlanguage{hebrew}{אַבְרָהָם} \hspace{0.20cm}
2~~\foreignlanguage{hebrew}{אַבְשָׁלוֺם} \hspace{0.20cm}
3~~\foreignlanguage{hebrew}{אָדָם} \hspace{0.20cm}
4~~\foreignlanguage{hebrew}{אַהֲרֹן} \hspace{0.20cm}
5~~\foreignlanguage{hebrew}{אַחְאָב} \hspace{0.20cm}
6~~\foreignlanguage{hebrew}{אָחָז} \hspace{0.20cm}
7~~\foreignlanguage{hebrew}{אִיּוֺב} \hspace{0.20cm}
8~~\foreignlanguage{hebrew}{אָמוֹץ} \hspace{0.20cm}
9~~\foreignlanguage{hebrew}{אָסָף} \hspace{0.20cm}
10~~\foreignlanguage{hebrew}{אֶסְתֵּר} \hspace{0.20cm}
11~~\foreignlanguage{hebrew}{אֶפְרַ֫יִם} \hspace{0.20cm}
12~~\foreignlanguage{hebrew}{אָשֵׁר} \hspace{0.20cm}
13~~\foreignlanguage{hebrew}{בִּלְעָם} \hspace{0.20cm}
14~~\foreignlanguage{hebrew}{בִּנְיָמִין} \hspace{0.20cm}
15~~\foreignlanguage{hebrew}{בָּרָק} \hspace{0.20cm}
16~~\foreignlanguage{hebrew}{גָּד} \hspace{0.20cm}
17~~\foreignlanguage{hebrew}{גָּלְיָת} \hspace{0.20cm}
18~~\foreignlanguage{hebrew}{דָּוִד} \hspace{0.20cm}
19~~\foreignlanguage{hebrew}{דָּן} \hspace{0.20cm}
20~~\foreignlanguage{hebrew}{הָגָר} \hspace{0.20cm}
21~~\foreignlanguage{hebrew}{הוֹשֵׁעַ} \hspace{0.20cm}
22~~\foreignlanguage{hebrew}{זְבוּלֻן} \hspace{0.20cm}
23~~\foreignlanguage{hebrew}{חַגַּי} \hspace{0.20cm}
24~~\foreignlanguage{hebrew}{יְהוּדָה} \hspace{0.20cm}
25~~\foreignlanguage{hebrew}{יְהוֹשֻׁעַ} \hspace{0.20cm}
26~~\foreignlanguage{hebrew}{יְהוֺשָׁפָט} \hspace{0.20cm}
27~~\foreignlanguage{hebrew}{יַעֲקֹב} \hspace{0.20cm}
28~~\foreignlanguage{hebrew}{יִפְתָּח} \hspace{0.20cm}
29~~\foreignlanguage{hebrew}{יִצְחָק} \hspace{0.20cm}
30~~\foreignlanguage{hebrew}{יָרָבְעָם} \hspace{0.20cm}
31~~\foreignlanguage{hebrew}{יִשְׂרָאֵל} \hspace{0.20cm}
32~~\foreignlanguage{hebrew}{כָּלֵב} \hspace{0.20cm}
33~~\foreignlanguage{hebrew}{לָבָן} \hspace{0.20cm}
34~~\foreignlanguage{hebrew}{לֵוִי} \hspace{0.20cm}
35~~\foreignlanguage{hebrew}{מְנַחֵם} \hspace{0.20cm}
36~~\foreignlanguage{hebrew}{מִרְיָם} \hspace{0.20cm}
37~~\foreignlanguage{hebrew}{נְבוּכַדְנֶאצָּר} \hspace{0.20cm}
38~~\foreignlanguage{hebrew}{נְחֶמְיָה} \hspace{0.20cm}
39~~\foreignlanguage{hebrew}{נַעֲמָן} \hspace{0.20cm}
40~~\foreignlanguage{hebrew}{נַפְתָּלִי} \hspace{0.20cm}
41~~\foreignlanguage{hebrew}{נָתָן} \hspace{0.20cm}
42~~\foreignlanguage{hebrew}{נְתַנְאֵל} \hspace{0.20cm}
43~~\foreignlanguage{hebrew}{עֵלִי} \hspace{0.20cm}
44~~\foreignlanguage{hebrew}{עָמְרִי} \hspace{0.20cm}
45~~\foreignlanguage{hebrew}{עֵשָׂו} \hspace{0.20cm}
46~~\foreignlanguage{hebrew}{רְאוּבֵן} \hspace{0.20cm}
47~~\foreignlanguage{hebrew}{רוּת} \hspace{0.20cm}
48~~\foreignlanguage{hebrew}{רְחַבְעָם} \hspace{0.20cm}
49~~\foreignlanguage{hebrew}{רָחֵל} \hspace{0.20cm}
50~~\foreignlanguage{hebrew}{שֵׁם} \hspace{0.20cm}
51~~\foreignlanguage{hebrew}{שִׁמְעוֹן} \hspace{0.20cm}
52~~\foreignlanguage{hebrew}{תָּמָר} \hspace{0.20cm}
\selectlanguage{english}

%\noindent The second \foreignlanguage{hebrew}{ש} in the name \foreignlanguage{hebrew}{יִשָּׂשכָר} is not pronounced; it is a remnant of the earlier form of the name with noun \foreignlanguage{hebrew}{שָׂכָר} \textit{reward} as second element.

\subsection{Proper Nouns (Segolate Nouns and Endings)}

\selectlanguage{hebrew}
\noindent
1~~\foreignlanguage{hebrew}{אֲבִיגַיִל} \hspace{0.20cm}
2~~\foreignlanguage{hebrew}{אֲחִימֶלֶךְ} \hspace{0.20cm}
3~~\foreignlanguage{hebrew}{אֲחִינֹעַם} \hspace{0.20cm}
4~~\foreignlanguage{hebrew}{אִיזֶבֶל} \hspace{0.20cm}
5~~\foreignlanguage{hebrew}{אֱלִיעֶזֶר} \hspace{0.20cm}
6~~\foreignlanguage{hebrew}{אֶפְרַיִם} \hspace{0.20cm}
7~~\foreignlanguage{hebrew}{בֹּעַז} \hspace{0.20cm}
8~~\foreignlanguage{hebrew}{הֶבֶל} \hspace{0.20cm}
9~~\foreignlanguage{hebrew}{יָפֶת} \hspace{0.20cm}
10~~\foreignlanguage{hebrew}{יְרֻבַּעַל} \hspace{0.20cm}
11~~\foreignlanguage{hebrew}{יִרְמְיָהוּ} \hspace{0.20cm}
12~~\foreignlanguage{hebrew}{לָמֶךְ} \hspace{0.20cm}
13~~\foreignlanguage{hebrew}{פֶּקַח} \hspace{0.20cm}
14~~\foreignlanguage{hebrew}{צִדְקִיָּהוּ} \hspace{0.20cm}
15~~\foreignlanguage{hebrew}{קַיִן} \hspace{0.20cm}
\selectlanguage{english}


\subsection{Reading Exercise}
Read Gen 1:1--8. Stress of words that are stressed on the penultimate syllable is marked.

% Stress is marked because I don't want students to put the stress on the wrong syllable when reading a text from the Bible for the first time.
% One could add assignments to this, such as searching for vowel letters, saying the names of letters and vowel signs, identifying segolate formations.

\bigskip

\selectlanguage{hebrew}
\noindent
\textsuperscript{1}~\foreignlanguage{hebrew}{בְּרֵאשִׁית בָּרָא אֱלֹהִים אֵת הַשָּׁמַ֫יִם וְאֵת הָאָ֫רֶץ} \hspace{0.3cm}
\textsuperscript{2}~\foreignlanguage{hebrew}{וְהָאָ֫רֶץ הָיְתָה תֹ֫הוּ וָבֹ֫הוּ וְחֹ֫שֶׁךְ עַל־פְּנֵי תְהוֹם וְרוּחַ אֱלֹהִים מְרַחֶ֫פֶת עַל־פְּנֵי הַמָּ֫יִם} \hspace{0.3cm}
\textsuperscript{3}~\foreignlanguage{hebrew}{וַיֹּ֫אמֶר אֱלֹהִים יְהִי אוֹר וַיְהִי־אוֹר} \hspace{0.3cm}
\textsuperscript{4}~\foreignlanguage{hebrew}{וַיַּרְא אֱלֹהִים אֶת־הָאוֹר כִּי־טוֹב וַיַּבְדֵּל אֱלֹהִים בֵּין הָאוֹר וּבֵין הַחֹ֫שֶׁךְ} \hspace{0.3cm}
\textsuperscript{5}~\foreignlanguage{hebrew}{וַיִּקְרָא אֱלֹהִים לָאוֹר יוֹם וְלַחֹ֫שֶׁךְ קָ֫רָא לָ֫יְלָה וַיְהִי־עֶ֫רֶב וַיְהִי־בֹ֫קֶר יוֹם אֶחָד} \hspace{0.3cm}
\textsuperscript{6}~\foreignlanguage{hebrew}{וַיֹּאמֶר אֱלֹהִים יְהִי רָקִיעַ בְּתוֹךְ הַמָּ֫יִם וִיהִי מַבְדִּיל בֵּין מַ֫יִם לָמָּ֫יִם} \hspace{0.3cm}
\textsuperscript{7}~\foreignlanguage{hebrew}{וַיַּ֫עַשׂ אֱלֹהִים אֶת־הָרָקִיעַ וַיַּבְדֵּל בֵּין הַמַּ֫יִם אֲשֶׁר מִתַּ֫חַת לָרָקִיעַ וּבֵין הַמַּ֫יִם אֲשֶׁר מֵעַל לָרָקִיעַ וַֽיְהִי־כֵן} \hspace{0.3cm}
\textsuperscript{8}~\foreignlanguage{hebrew}{וַיִּקְרָא אֱלֹהִים לָרָקִיעַ שָׁמָ֫יִם וַיְהִי־עֶ֫רֶב וַיְהִי־בֹ֫קֶר יוֹם שֵׁנִי} \hspace{0.3cm}
\selectlanguage{english}


\chapter{Chapter 3}

\section{Vocabulary}

\subsection{Verbs}

\begin{center}
	
	% For the centering of the separation between the two columns see the documentation of the array package, page 2 
	
	\begin{tabular}{>{\raggedleft}p{0.175\linewidth} p{0.75\linewidth}}
		\foreignlanguage{hebrew}{עָשָׂה} & \textit{he did, he made} \\
		\foreignlanguage{hebrew}{וַיַּ֫עַשׂ} & \textit{and he did, and he made} \\
		\foreignlanguage{hebrew}{לָקַח} & \textit{he took} \\
		\foreignlanguage{hebrew}{וַיִּקַּח} & \textit{and he took} \\
		\foreignlanguage{hebrew}{נָתַן} & \textit{he gave} \\
		\foreignlanguage{hebrew}{וַיִּתֵּן} & \textit{and he gave} \\
		\foreignlanguage{hebrew}{רָאָה} & \textit{he saw}  \\
		\foreignlanguage{hebrew}{וַיַּרְא} & \textit{and he saw} (\textit{wayyar(ʾ)} with silent \foreignlanguage{hebrew}{א}) \\
		\foreignlanguage{hebrew}{שָׁמַע} & \textit{he heard, he listened} \\
		\foreignlanguage{hebrew}{וַיִּשְׁמַע} & \textit{and he heard, and he listened} \\
	\end{tabular}
\end{center}

\subsection{Nouns}

\begin{center}
	\begin{longtable}{>{\raggedleft}p{0.175\linewidth} p{0.75\linewidth}}
		\foreignlanguage{hebrew}{אָב} & \textit{father} \\
		\foreignlanguage{hebrew}{אִישׁ} & \textit{man} \\
		\foreignlanguage{hebrew}{אֱלֹהִים} & \textit{God, god, gods} \\
		\foreignlanguage{hebrew}{אֱלוֺהַּ} & \textit{god, God} (sg. of \foreignlanguage{hebrew}{אֱלֹהִים}; much less frequent) \\
		\foreignlanguage{hebrew}{אֶרֶץ} & \textit{earth, land} (fem.) (pl. \foreignlanguage{hebrew}{אֲרָצוֺת}; with the article \foreignlanguage{hebrew}{הָאָ֫רֶץ}, pausal form \foreignlanguage{hebrew}{אָ֑רֶץ})\\
		\foreignlanguage{hebrew}{בֵּן} & \textit{son} \\
		\foreignlanguage{hebrew}{בָּקָר} & \textit{cattle, herd} (coll.) \\
		\foreignlanguage{hebrew}{דָּבָר} & \textit{word, matter, thing} \\
		\foreignlanguage{hebrew}{דָּם} & \textit{blood} \\
		\foreignlanguage{hebrew}{זָהָב} & \textit{gold} \\
		\foreignlanguage{hebrew}{חֶרֶב} & \textit{sword} (fem.) \\
		\foreignlanguage{hebrew}{יוֺם} & \textit{day} (\foreignlanguage{hebrew}{הַיּוֺם} as an adverb \textit{today}) \\
		\foreignlanguage{hebrew}{יֶלֶד} & \textit{boy} \\
		\foreignlanguage{hebrew}{כֶּסֶף} & \textit{silver} \\
		\foreignlanguage{hebrew}{מִזְבֵּחַ} & \textit{altar} (pl. \foreignlanguage{hebrew}{מִזְבְּחוֺת}) \\
		\foreignlanguage{hebrew}{מֶלֶךְ} & \textit{king} \\
		\foreignlanguage{hebrew}{מָקוֺם} & \textit{place} (pl. \foreignlanguage{hebrew}{מְקֹמוֺת}) \\
		\foreignlanguage{hebrew}{מִשְׁתֶּה} & \textit{feast, banquet} \\ % only 46 occurrences
		\foreignlanguage{hebrew}{עֶבֶד} & \textit{slave, servant} \\
		\foreignlanguage{hebrew}{עַיִן} & \textit{eye, spring} (fem.) \\
		\foreignlanguage{hebrew}{עִיר} & \textit{city, town} (fem.) \\
		\foreignlanguage{hebrew}{עַם} & \textit{people, nation} (also \foreignlanguage{hebrew}{עָם}) \\
		\foreignlanguage{hebrew}{פַּרְעֹה} & \textit{Pharaoh} \\
		\foreignlanguage{hebrew}{צֹאן} & \textit{small cattle, sheep and goats, flock, flocks} (coll., fem.) \\
		\foreignlanguage{hebrew}{שָׂדֶה} & \textit{field} (pl. \foreignlanguage{hebrew}{שָׂדוֺת}) \\
	\end{longtable}
\end{center}

\subsection{Prepositions}

\begin{center}
	\begin{tabular}{>{\raggedleft}p{0.175\linewidth} p{0.75\linewidth}}
		\foreignlanguage{hebrew}{לְ} & \textit{to, for, regarding}, also indicating indirect objects \\
		\foreignlanguage{hebrew}{בְּ} & \textit{in, at, with} (local and temporal)\\
		\foreignlanguage{hebrew}{כְּ} & \textit{like, as, according to} \\
	\end{tabular}
\end{center}


\subsection{Other Parts of Speech}

\begin{center}
	\begin{tabular}{>{\raggedleft}p{0.175\linewidth} p{0.75\linewidth}}
		\foreignlanguage{hebrew}{אֵת} & object marker indicating the direct object \\
		\foreignlanguage{hebrew}{זֹאת} ,\foreignlanguage{hebrew}{זֶה} & \textit{this} (masc., fem.) \\
		\foreignlanguage{hebrew}{אֵ֫לֶּה} & \textit{these} (comm.) \\
		\foreignlanguage{hebrew}{הִנֵּה} & \textit{look, behold} \\
		\foreignlanguage{hebrew}{יוֹמָם} & \textit{by day} (adv.) \\
		\foreignlanguage{hebrew}{כֵּן} & \textit{so, thus} (Modern Hebrew: \textit{yes}) \\
		\foreignlanguage{hebrew}{שָׁם} & \textit{there} (adv.) \\
	\end{tabular}
\end{center}


\section{Frequent Verbal Forms}
Verbs are the most important part of speech in Biblical Hebrew. Among the verbal forms, weak forms are very frequent, i.e., verbal forms that do not have three root consonants but only two or sometimes only one instead of the three root consonants of the strong verb. In order to make it easier to become familiar with verbal forms, the most frequent verbal forms are introduced as vocabulary in chapters 3 and 4.

The first set of ten forms of five verbs is in the section Vocabulary above; the second set is in the vocabulary of chapter 4. At this state, it is sufficient to memorize the forms and their meanings without understanding how the forms are actually formed.

The twenty verbal forms in the vocabulary of chapters 3 and 4 are about 10\,\% of all verbal forms in the Hebrew Bible. All forms of these ten verbs make up about 30\,\% of the total number of verbal forms in the Hebrew Bible. These figures make it clear that it is essential to be familiar with these twenty verbal forms in particular and the ten verbs in general. Only two of the ten verbs are strong verbs, i.e., verbs that have three root consonants in \emph{all} their forms.

The most frequent forms that are introduced in chapters 3 and 4, are the 3 masc.\ sg.\ suffix conjugation (SC) form  and the 3 masc.\ sg.\ \textit{waw}-prefix conjugation (\textit{waw}-PC) or \textit{wayyiqṭol} form. The SC usually has past tense meaning (simple past, present perfect or past perfect depending on the context). The \textit{waw}-PC serves as simple past (preterite). Chains of \textit{waw}-PC forms are the backbone of narratives in the Hebrew Bible. \textit{waw}-PC forms are always placed in first position in the clause. The other forms of the SC and PC will be introduced in later chapters.

\section{The Verbal Clause}
A clause with a finite verbal form (i.e., a form that is marked for person, 1st \textit{I}, 2nd \textit{you}, or 3rd person \textit{he, she, it}) as predicate is called \textit{verbal clause}. Most clauses in biblical literature are verbal clauses. The other type of clause is the verbless or nominal clause.

With a finite verb as predicate there is no need for an explicit (overt) subject because the subject is contained in the verbal form, e.g., \foreignlanguage{hebrew}{וַיַּרְא} \textit{and he saw}. With an overt subject the verbal form is to be rendered only as a verbal form, e.g., \foreignlanguage{hebrew}{וַיַּרְא יַעֲקֹב} \textit{and Jacob saw}.

If a clause includes an overt subject, the subject and the verbal predicate agree with each other in number, gender and -- in the case of 1st or 2nd person verbal forms -- in person. This means, for example, that a common noun as subject that is masculine and singular, is combined with a masc.\ sg.\ verbal form (in the 3rd person). There are, however, exceptions to this rule.

The most common constituent order is verb-subject-object (VSO). As a consequence, the function of a constituent is often indicated by its position in the clause. Subjects normally precede direct objects, indirect objects or prepositional objects. Indirect objects usually follow direct objects. Adverbial phrases tend to be put at the end of the clause although temporal adverbial expressions are often placed up front. The length of constituents also plays a role. Short constituents, e.g., pronouns or short adverbs, have the tendency to be placed earlier in the clause. In most cases, the meaning of the verb and context help in identifying the function of clause constituents.

Any constituent maybe be fronted, that is, placed before the verb. The result is marked constituent order. There is usually a specific reason for employing marked constituent order. Usually, it is the activation or reactivation of a topic (also for contrast) or putting a constituent in focus, that is, highlighting it as the most salient constituent of the clause.


\section{The Definite Article}
Different from English but similar to Greek, Biblical Hebrew only has a definite article. Its use is similar to the use of the definite article \textit{the} in English but there are quite many differences in the use of the definite article between the two languages. Most frequently, the definite article is used when an entity can be identified by both the speaker and the addressee in the context of the communication. 


The  Hebrew definite article is directly prefixed to the noun. Its basic form is \foreignlanguage{hebrew}{◌ּ}\space\foreignlanguage{hebrew}{הַ}, i.e., \textit{ha-} followed by \textit{dageš forte} in the first consonant of the noun. This is illustrated by the following examples:

\begin{center}
	\begin{tabular}{rlrl}
		\foreignlanguage{hebrew}{מֶלֶךְ} & \textit{king} & \foreignlanguage{hebrew}{הַמֶּלֶךְ} & \textit{the king} \\
		\foreignlanguage{hebrew}{יָד} & \textit{hand} & \foreignlanguage{hebrew}{הַיָּד} & \textit{the hand} \\
		\foreignlanguage{hebrew}{מִזְבֵּחַ} & \textit{altar} & \foreignlanguage{hebrew}{הַמִּזְבֵּחַ} & \textit{the altar} \\
	\end{tabular}
\end{center}

If the noun begins with \foreignlanguage{hebrew}{יְ} \textit{yə-} or \foreignlanguage{hebrew}{מְ} \textit{mə-}, the \textit{dageš forte} of the article is usually lost, e.g., \foreignlanguage{hebrew}{הַיְהוּדִים} \textit{the Judeans}, \foreignlanguage{hebrew}{הַמְרַגְּלִים} \textit{the spies}.

As the consonants \foreignlanguage{hebrew}{א}, \foreignlanguage{hebrew}{ה}, \foreignlanguage{hebrew}{ח}, \foreignlanguage{hebrew}{ע} and \foreignlanguage{hebrew}{ר} are not geminated there is no \textit{dageš} in the first consonant of the noun if it begins with a guttural or /r/. The following rules apply with in regard to the vowel of the article:

\begin{enumerate}[noitemsep]
	\item If the noun begins with \foreignlanguage{hebrew}{ה} or \foreignlanguage{hebrew}{ח}, the \textit{pataḥ} of the article is preserved in the open syllable, e.g., \foreignlanguage{hebrew}{הַהֵיכָל} \textit{the temple}, \foreignlanguage{hebrew}{הַחֶרֶב} \textit{the sword}. Frequent exceptions to this rule are the noun \foreignlanguage{hebrew}{הַר} \textit{mountain}, which is  \foreignlanguage{hebrew}{הָהָר} with the article, and the independent personal pronoun 3 m./f.\ pl. with the article as demonstrative pronoun \foreignlanguage{hebrew}{הָהֵם}, \foreignlanguage{hebrew}{הָהֵ֫מָּה} (masc.), \foreignlanguage{hebrew}{הָהֵ֫נָּה} [fem.] (in the singular, however one finds \foreignlanguage{hebrew}{הַהוּא} [masc.] and \foreignlanguage{hebrew}{הַהִיא} [fem.], see below).
	\item If the noun begins with \foreignlanguage{hebrew}{א}, \foreignlanguage{hebrew}{ע} or \foreignlanguage{hebrew}{ר}, the \textit{pataḥ} of the article is changed to \textit{qameṣ}, e.g., \foreignlanguage{hebrew}{הָאֵשׁ} \textit{the fire}, \foreignlanguage{hebrew}{הָעִיר} \textit{the city}, \foreignlanguage{hebrew}{הָרֹאשׁ} \textit{the head}.
	\item If the noun begins with \foreignlanguage{hebrew}{ה} or \foreignlanguage{hebrew}{ע} followed by unstressed \textit{qameṣ}, the \textit{pataḥ} of the article is changed to \textit{segol}, e.g., \foreignlanguage{hebrew}{הֶהָרִים} \textit{the mountains}, \foreignlanguage{hebrew}{הֶעָרִים} \textit{the cities}.
	\item Likewise, if the noun begins with \foreignlanguage{hebrew}{ח} followed by stressed or unstressed \textit{qameṣ gadol} or \textit{ḥatef qameṣ}, the \textit{pataḥ} of the article is changed to \textit{segol}, e.g., \foreignlanguage{hebrew}{הֶחָכָם} \textit{the wise one}, \foreignlanguage{hebrew}{הֶחָג} \textit{the feast}, \foreignlanguage{hebrew}{הֶחֳדָשִׁים} \textit{the new moons} (i.e., the festivals held at new moon).
\end{enumerate}

Some nouns change their vowels to the vowel of the pausal form of the noun when they are used with the article, e.g., \foreignlanguage{hebrew}{הָאָ֫רֶץ} \textit{the land, the earth} (context form \foreignlanguage{hebrew}{אֶרֶץ} \textit{land, earth}), \foreignlanguage{hebrew}{הָהָר} \textit{the mountain} (context form \foreignlanguage{hebrew}{הַר} \textit{mountain}). In a similar way, the noun \foreignlanguage{hebrew}{אֲרוֺן} \textit{ark, chest} with the article is \foreignlanguage{hebrew}{הָאָרוֺן} \textit{the ark}.

Common nouns may be used with the article to indicate the vocative function, e.g., \foreignlanguage{hebrew}{וַתֹּ֫אמֶר הוֹשִׁ֫עָה הַמֶּ֫לֶךְ} \textit{And she said, \enquote{Help, O king!}} (2\,Sam 14:4); \foreignlanguage{hebrew}{בֶּן־מִי אַתָּה הַנָּעַר} \textit{Whose son are you, young man?} (1\,Sam 17:58).

As Hebrew does not have an indefinite article, an expression like \textit{a great nation} in English is simply \foreignlanguage{hebrew}{גּוֺי גָּדוֺל} in Hebrew, i.e., only the noun \foreignlanguage{hebrew}{גּוֺי} with the attributive adjective \foreignlanguage{hebrew}{גָּדוֺל}.


\section{Demonstrative Pronouns}
Unlike English with \textit{this}, \textit{these} and \textit{that}, \textit{those} or Greek with \foreignlanguage{greek}{οὗτος} etc.\ \textit{this} and \foreignlanguage{greek}{ἐκεῖνος} etc. \textit{that}, Biblical Hebrew has only one set of demonstrative pronouns. In the singular there are distinct forms for masc.\ and fem.\ while in the plural there is only one form with common gender. The function of the demonstrative pronouns as deictic elements is to point to specific entities in the communicative situation or the immediate context.

\medskip

\begin{center}
	\begin{tabular}{|ll|rl|}
		\hline
		\multirow{2}{*}{sg.} & masc. & \foreignlanguage{hebrew}{זֶה} & \multirow{2}{*}{\textit{this}} \\
		& fem. &  \foreignlanguage{hebrew}{זֹאת} & \\
		\hline
		pl. & comm. & \foreignlanguage{hebrew}{אֵ֫לֶּה} & \textit{these} \\
		\hline
	\end{tabular}
\end{center}

\vspace{0.5cm}

\noindent \textbf{Note}
\nopagebreak

\noindent The \foreignlanguage{hebrew}{א} in the fem.\ sg.\ form \foreignlanguage{hebrew}{זֹאת} \textit{zō(ʾ)t} is silent.

\vspace{0.5cm}

Demonstrative pronouns can either be used independently as subject or direct object replacing a noun or they can be used as attributes depending on nouns. When they are used as attributes, they follow the noun and both the noun and the demonstrative pronoun have the definite article. The demonstrative pronoun agrees with the noun it modifies in number and -- in the singular -- in gender. As the plural form is common gender it can be used with both masc. and fem. nouns.

\begin{center}
	\begin{tabular}{rl}
		\foreignlanguage{hebrew}{הָאִישׁ הַזֶּה} & \textit{this man} (masc.) \\
		\foreignlanguage{hebrew}{הָאִשָּׁה הַזֹּאת} & \textit{this woman} (fem.) \\
		\foreignlanguage{hebrew}{הַדְּבָרִים הָאֵ֫לֶּה} & \textit{these words} (masc.) \\
		\foreignlanguage{hebrew}{הַבְּרָכוֹת הָאֵ֫לֶּה} & \textit{these blessings} (fem.) \\
	\end{tabular}
\end{center}

For far deixis (pointing to something at a distance, e.g., Engl. \textit{that} or Greek \foreignlanguage{greek}{ἐκεῖνος}, Biblical Hebrew uses the 3rd person personal pronoun following the modified noun. Both the noun and the personal pronoun have the article. The personal pronoun agrees with the noun in number and gender.

\begin{center}
	\begin{tabular}{rl}
		\foreignlanguage{hebrew}{הָאִישׁ הַהוּא} & \textit{that man} (masc.) \\
		\foreignlanguage{hebrew}{הָאִשָּׁה הַהִיא} & \textit{that woman} (fem.) \\
		\foreignlanguage{hebrew}{הָאֲנָשִׁים הָהֵ֫מָּה} & \textit{those men} (masc.) \\
	\end{tabular}
\end{center}

The demonstrative pronoun \foreignlanguage{hebrew}{זֶה} can be used as adverb with local or temporal deictic function with the meanings \textit{here}, \textit{there} or \textit{now}.

Rare forms of the demonstrative pronoun are \foreignlanguage{hebrew}{זֹה} \textit{this} (fem.), \foreignlanguage{hebrew}{הַלָּזֶה} \textit{this} (masc.) and \foreignlanguage{hebrew}{הַלָּז} \textit{this} (comm.).

\section{The Prepositions \foreignlanguage{hebrew}{לְ}, \foreignlanguage{hebrew}{בְּ} and \foreignlanguage{hebrew}{כְּ}}
Prepositions are a small set of expressions that are frequently used and very important. Their function is to denote the relations between elements in the outside world that speech refers to, or between elements of a clause. In this section the three prepositions \foreignlanguage{hebrew}{לְ}, \foreignlanguage{hebrew}{בְּ} and \foreignlanguage{hebrew}{כְּ} are introduced. They are extremely frequent with about 20,275, 15,570 and 2,895 occurrences in the Hebrew Bible, respectively.

% Figures for ל and ב are taken from Jenni's monographs on these prepositions.

Formally, these three prepositions are characterized by the fact that they are not separate words but are attached directly to the following word. Therefore, they are called \emph{proclitic} prepositions (as opposed to the other prepositions which are separate words).

The prepositions \foreignlanguage{hebrew}{לְ}, \foreignlanguage{hebrew}{בְּ} and \foreignlanguage{hebrew}{כְּ} have a number of different meanings and functions that cannot be exhaustively explained here. The standard lexica contain more detailed information. In this context a short overview of meanings must be sufficient.

\begin{center}
	\begin{tabular}{rl}
		\foreignlanguage{hebrew}{לְ} & \textit{to, for, in regard to}, also indicating indirect objects \\
		\foreignlanguage{hebrew}{בְּ} & \textit{in, at, with} (local and temporal)\\
		\foreignlanguage{hebrew}{כְּ} & \textit{like, as, according to} \\
	\end{tabular}
\end{center}

When prefixed to a word beginning with a syllable with a full vowel (except the definite article) the prepositions \foreignlanguage{hebrew}{לְ}, \foreignlanguage{hebrew}{בְּ} and \foreignlanguage{hebrew}{כְּ} appear in the forms as in the overview above, e.g., \foreignlanguage{hebrew}{לְמֹשֶׁה}, \foreignlanguage{hebrew}{בְּמֹשֶׁה} and \foreignlanguage{hebrew}{כְּמֹשֶׁה}.

The prepositions \foreignlanguage{hebrew}{לְ}, \foreignlanguage{hebrew}{בְּ} and \foreignlanguage{hebrew}{כְּ} have deviating forms in the following circumstances:

\begin{enumerate}[noitemsep]
	\item If the prepositions \foreignlanguage{hebrew}{לְ}, \foreignlanguage{hebrew}{בְּ} and \foreignlanguage{hebrew}{כְּ} are prefixed to a noun with the article, the /h/ of the article is lost but the vowel of the article is preserved, e.g., \foreignlanguage{hebrew}{לַמֶּלֶךְ} \textit{to the king}, \foreignlanguage{hebrew}{לָעָם} \textit{for the people}, \foreignlanguage{hebrew}{בֶּהָרִים} \textit{in the mountains}.
	\item If prefixed to a word beginning with vocal \textit{šwa} (except words beginning with \foreignlanguage{hebrew}{יְ} \textit{yə-}), the three prepositions have the full vowel /i/ and the vocal \textit{šwa} becomes a medial \textit{šwa}, \foreignlanguage{hebrew}{לִבְרִית עוֹלָם} \textit{for an everlasting covenant}.
	\item If prefixed to a word beginning with \textit{ḥatef šwa}, the three prepositions take the full vowel that corresponds to the \textit{ḥatef šwa} and the \textit{ḥatef šwa} itself remains unchanged, e.g., \foreignlanguage{hebrew}{לַחֲמֹרוֺ} \textit{to his donkey}, \foreignlanguage{hebrew}{בֶּאֱמֶת} \textit{in faithfulness, faithfully}, \foreignlanguage{hebrew}{לָחֳלִי} \textit{lɔḥɔ̆lī} \textit{to sickness}.
	\item If prefixed to a word beginning with \foreignlanguage{hebrew}{יְ} \textit{yə-}, the preposition is followed by a long vowel /ī/ which is the result of the contraction of the short vowel /i/ and the following semivowel (\textit{*bəyə- > *biy- > bī-}), e.g., \foreignlanguage{hebrew}{בִּיהוּדָה} \textit{in Judah} (basic form \foreignlanguage{hebrew}{יְהוּדָה})
	\item Before words that are stressed on the first syllable the prepositions sometimes have the vowel \textit{qameṣ}, e.g., \foreignlanguage{hebrew}{לָרֹב} \textit{abundantly}, \foreignlanguage{hebrew}{בָּזֶה} \textit{here}, \foreignlanguage{hebrew}{כָּאֵלֶּה} \textit{like them}.
	\item If the three preposition are to be prefixed to the noun \foreignlanguage{hebrew}{אֱלֹהִים} \textit{God} or to a form of the noun \foreignlanguage{hebrew}{אָדוֺן} \textit{lord} other than the citation form, the \foreignlanguage{hebrew}{א} of these nouns becomes silent although it is retained in writing, and the following \textit{ḥatef šwa} is changed to a full vowel, e.g., \foreignlanguage{hebrew}{לֵאלֹהִים} \textit{lē(ʾ)lōhīm} \textit{to God} and \foreignlanguage{hebrew}{לַאדֹנִי} \textit{la(ʾ)dōnī} \textit{to my lord} (\foreignlanguage{hebrew}{לְ} + \foreignlanguage{hebrew}{אֲדֹנִי}).
\end{enumerate}



\section{The Conjunction \foreignlanguage{hebrew}{וְ}}
The function of the conjunction \foreignlanguage{hebrew}{וְ} is to connect words, phrases and clauses. Its meaning is \textit{and} although in many cases a different rendering like \textit{but} or \textit{or} is more appropriate in English. It is also called a coordinating conjunction.

The most common form of the conjunction is \foreignlanguage{hebrew}{וְ} \textit{wə-} which is directly prefixed to the following word, e.g., \foreignlanguage{hebrew}{וְאַבְרָהָם} \textit{and Abraham}.

The conjunction \foreignlanguage{hebrew}{וְ} has deviating forms in the following circumstances:

\begin{enumerate}[noitemsep]
	\item If the following word begins with a vocal \textit{šwa}, the conjunction is \foreignlanguage{hebrew}{וּ} \textit{ū}, e.g., \foreignlanguage{hebrew}{וּדְבַר־הַמֶּלֶךְ} \textit{and the word of the king} (This is an exception to the rule that every word must begin with a consonant.)
	\item If the following word begins with a labial consonant, i.e., \foreignlanguage{hebrew}{ב}, \foreignlanguage{hebrew}{מ}, and \foreignlanguage{hebrew}{פ}, the conjunction is \foreignlanguage{hebrew}{וּ} \textit{ū}, e.g., \foreignlanguage{hebrew}{וּמֹשֶׁה} \textit{and Moses}. (This is another exception to the rule that every word must begin with a consonant.)
	\item If the conjunction connects two words and the second one is stressed on the first syllable, especially in the case of common word pairs, the form of the conjunction is \foreignlanguage{hebrew}{וָ} \textit{wā-}, e.g., \foreignlanguage{hebrew}{קֹר וָחֹם וְקַ֫יִץ וָחֹ֫רֶף וְיוֹם וָלַ֫יְלָה} \textit{cold and heat, and summer and winter, and day and night} (Gen 8:22 with three word pairs).
	\item With words beginning with a vocal \textit{šwa} or \textit{ḥatef šwa}, the vowel of the conjunction \foreignlanguage{hebrew}{וְ} resembles the vowels of the proclitic prepositions \foreignlanguage{hebrew}{לְ}, \foreignlanguage{hebrew}{בְּ} and \foreignlanguage{hebrew}{כְּ}.
	\begin{enumerate}[noitemsep]
		\item Conjunction \foreignlanguage{hebrew}{וְ} + word beginning with \textit{ḥatef šwa}: \foreignlanguage{hebrew}{וַחֲמֹרִים} \textit{and donkeys}, \foreignlanguage{hebrew}{וֶאֱמֶת} \textit{and truth}, \foreignlanguage{hebrew}{וָחֳלִי} \textit{wɔḥɔ̆lī} \textit{and sickness}
		\item Conjunction \foreignlanguage{hebrew}{וְ} + word beginning with \foreignlanguage{hebrew}{יְ} \textit{yə-}: \foreignlanguage{hebrew}{וִיהוּדָה} \textit{and Judah}
		\item Conjunction \foreignlanguage{hebrew}{וְ} + the noun \foreignlanguage{hebrew}{אֱלֹהִים} or a form of the noun \foreignlanguage{hebrew}{אָדוֺן} \textit{lord} other than the citation form: \foreignlanguage{hebrew}{וֵאלֹהִים} \textit{wē(ʾ)lōhĭm} \textit{and God}, \foreignlanguage{hebrew}{וַאדֹנִי} \textit{wa(ʾ)dōnĭ} \textit{and my lord}, both with silent \foreignlanguage{hebrew}{א}.
	\end{enumerate}
	\item In the case of \textit{waw}-PC form (or \textit{wayyiqṭol} form) the form of the conjunction is \foreignlanguage{hebrew}{◌ּ}\space\foreignlanguage{hebrew}{וַ} (\textit{wa-} with following \textit{dageš forte}).
\end{enumerate}

The /h/ of the article is preserved after the conjunction \foreignlanguage{hebrew}{וְ}, e.g., \foreignlanguage{hebrew}{וְהָאָרֶץ} \textit{and the earth}.


\section{Differential Object Marking}
Differential object marking means that only a more or less well defined part of direct objects is explicitly marked as such while the rest of the direct objects is not.

\subsection{The Use of the Object Marker \foreignlanguage{hebrew}{את}}
In Biblical Hebrew only definite direct objects are marked with the object marker \foreignlanguage{hebrew}{אֵת} (with \textit{maqqef} \foreignlanguage{hebrew}{אֶת־}). This word does not have a meaning and, therefore, does not have a translation equivalent. It only has the function of marking the direct object.

Not all definite direct objects, however, are marked. Proper nouns as direct objects have the highest percentage of marked elements in prose texts (97\,\%) whereas only 73\,\% of definite common nouns as direct objects are marked. It should, therefore, not be surprising to find relatively many definite direct objects that are not marked with \foreignlanguage{hebrew}{אֵת}. The following nouns are definite:

\begin{enumerate}[noitemsep]
	\item Proper nouns, e.g., \foreignlanguage{hebrew}{יִצְחָק}, \foreignlanguage{hebrew}{חֶבְרוֺן}
	\item Nouns with the definite article, e.g., \foreignlanguage{hebrew}{הַמֶּלֶךְ} \textit{the king}
	\item Nouns with enclitic personal pronouns, e.g., \foreignlanguage{hebrew}{בֵּיתִי} \textit{my house}
	\item Nouns with a definite dependent noun in a construct chain, e.g., \foreignlanguage{hebrew}{בֵּית הַמֶּלֶךְ} \textit{the house of the king}
\end{enumerate}

The use of the object marker \foreignlanguage{hebrew}{אֵת} is illustrated in the exercises.

In the Hebrew Bible, there are only very few indefinite direct objects that are marked with \foreignlanguage{hebrew}{אֵת}.

Sometimes, the object marker \foreignlanguage{hebrew}{אֵת} is used with nouns as adverbial expressions of time and place.

\subsection{Ellipsis of Direct Objects}

Direct objects do not need to be included in the clause if they can be easily retrieved from the immediate context. As a consequence, Biblical Hebrew does not need to have a direct object where English has a personal pronoun as direct object.

\vspace{0.5cm}

\begin{tabular}{>{\raggedleft}p{0.35\linewidth} p{0.55\linewidth}}
	\foreignlanguage{hebrew}{וַיִּקַּח אַבְרָהָם צֹאן וּבָקָר וַיִּתֵּן לַאֲבִימֶלֶךְ } & \textit{And Abraham took sheep and oxen and gave [them]  to Abimelech.} (Gen 21:27a) \\
\end{tabular}

\vspace{0.5cm}

Gen 21:27a consists of two clauses. In the first one, the verbal form \foreignlanguage{hebrew}{וַיִּקַּח} has the direct object \foreignlanguage{hebrew}{צֹאן וּבָקָר} (two coordinated indefinite nouns). In the second clause with the verbal form \foreignlanguage{hebrew}{וַיִּתֵּן} there is only the indirect object \foreignlanguage{hebrew}{לַאֲבִימֶלֶךְ} and no direct object because it can be easily supplied from the immediately preceding clause.

Direct objects may also be omitted when they can be easily supplied from common knowledge.

\section{Exercises}

\subsection{Frequent Verbal Forms}
Translate the following clauses from the Hebrew Bible. In some clauses there is a fronted constituent. The fronting, however, may be ignored in these exercises. Names of persons and geographical names in these sentences: \foreignlanguage{hebrew}{אֲבִימֶלֶךְ}, \foreignlanguage{hebrew}{אַבְרָהָם}, \foreignlanguage{hebrew}{אֵהוּד}, \foreignlanguage{hebrew}{אֵלִיָּהוּ}, \foreignlanguage{hebrew}{בִּלְעָם}, \foreignlanguage{hebrew}{בָּלָק}, \foreignlanguage{hebrew}{בְּנָיָהוּ}, \foreignlanguage{hebrew}{גִּדְעוֹן}, \foreignlanguage{hebrew}{גַּעַל}, \foreignlanguage{hebrew}{יְהוֹאָחָז}, \foreignlanguage{hebrew}{יְהוֹיָדָע}, \foreignlanguage{hebrew}{יְהוֹיָקִים}, \foreignlanguage{hebrew}{יוֹאָשׁ}, \foreignlanguage{hebrew}{יַעֲקֹב}, \foreignlanguage{hebrew}{יִשְׂרָאֵל}, \foreignlanguage{hebrew}{מֹשֶׁה}, \foreignlanguage{hebrew}{רִבְקָה}, \foreignlanguage{hebrew}{שָׂרָה}, \foreignlanguage{hebrew}{שָׁאוּל},\foreignlanguage{hebrew}{שְׁמוּאֵל} \foreignlanguage{hebrew}{שִׁמְשׁוֹן}.

\bigskip

\selectlanguage{hebrew}

\noindent
1~~\foreignlanguage{hebrew}{גִּדְעוֹן בֶּן־יוֹאָשׁ}\LTRfootnote{\space \foreignlanguage{hebrew}{בֶּן־יוֹאָשׁ} \textit{son of Joash}; apposition to \foreignlanguage{hebrew}{גִּדְעוֹן}} \space \foreignlanguage{hebrew}{עָשָׂה הַדָּבָר הַזֶּה} \hspace{0.3cm}
2~~\foreignlanguage{hebrew}{אֵלֶּה עָשָׂה בְּנָיָהוּ בֶּן־יְהוֹיָדָע}\LTRfootnote{\space \foreignlanguage{hebrew}{בֶּן־יְהוֹיָדָע} \textit{son of Jehoiada}; apposition to \foreignlanguage{hebrew}{בְּנָיָהוּ}} \hspace{0.3cm}
3 ~~\foreignlanguage{hebrew}{וַיַּעַשׂ מִשְׁתֶּה}  \hspace{0.3cm}
4~~\foreignlanguage{hebrew}{וַיַּעַשׂ יַעֲקֹב כֵּן} \hspace{0.3cm}
5~~\foreignlanguage{hebrew}{וַיַּעַשׂ לוֹ}\LTRfootnote{\space \foreignlanguage{hebrew}{לוֹ} \textit{for himself}} \space \foreignlanguage{hebrew}{אֵהוּד חֶרֶב}  \hspace{0.3cm}
6~~\foreignlanguage{hebrew}{וַיַּעַשׂ שָׁם שִׁמְשׁוֹן מִשְׁתֶּה}  \hspace{0.3cm}
7~~\foreignlanguage{hebrew}{וְאֶת־יְהוֹאָחָז לָקָח}\LTRfootnote{\space \foreignlanguage{hebrew}{לָקָח} pausal form for \foreignlanguage{hebrew}{לָקַח}}  \hspace{0.3cm}
8~\hspace{0.15cm}~\foreignlanguage{hebrew}{וַיִּקַּח אֶת־שָׂרָה} \hspace{0.3cm}
9~~\foreignlanguage{hebrew}{וַיִּקַּח הָעֶבֶד אֶת־רִבְקָה}  \hspace{0.3cm}
10~~\foreignlanguage{hebrew}{וַיִּקַּח מֹשֶׁה אֶת־הַדָּם}  \hspace{0.3cm}
11~~\foreignlanguage{hebrew}{וַיִּקַּח בָּלָק אֶת־בִּלְעָם}  \hspace{0.3cm}
12~~\foreignlanguage{hebrew}{וַיִּקַּח אֶת־הָעָם}  \hspace{0.3cm}
13~~\foreignlanguage{hebrew}{וַיִּקַּח שָׁאוּל אֶת־הַחֶרֶב}  \hspace{0.3cm}
14~~\foreignlanguage{hebrew}{וַיִּקַּח אֵלִיָּהוּ אֶת־הַיֶּלֶד}  \hspace{0.3cm}
15~~\foreignlanguage{hebrew}{וְהַכֶּסֶף וְהַזָּהָב נָתַן יְהוֹיָקִים לְפַרְעֹה}  \hspace{0.3cm}
16~~\foreignlanguage{hebrew}{וַיִּקַּח אַבְרָהָם צֹאן וּבָקָר וַיִּתֵּן לַאֲבִימֶלֶךְ}  \hspace{0.3cm}
17~~\foreignlanguage{hebrew}{וַיִּתֵּן יְהוָה}\LTRfootnote{\space \foreignlanguage{hebrew}{יהוה} read as \foreignlanguage{hebrew}{אֲדֹנָי} (Engl. \textit{Lord})} \foreignlanguage{hebrew}{לְיִשְׂרָאֵל מוֹשִׁיעַ}\LTRfootnote{\space \foreignlanguage{hebrew}{מוֹשִׁיעַ} \textit{deliverer, savior}} \hspace{0.3cm}
18~~\foreignlanguage{hebrew}{וּשְׁמוּאֵל רָאָה אֶת־שָׁאוּל} \hspace{0.3cm}
19~~\foreignlanguage{hebrew}{וַיַּרְא אֱלֹהִים אֶת־הָאָרֶץ} \hspace{0.3cm}
20~~\foreignlanguage{hebrew}{וַיַּרְא וְהִנֵּה בְאֵר}\LTRfootnote{\space \foreignlanguage{hebrew}{בְּאֵר} \textit{well}} \space\foreignlanguage{hebrew}{בַּשָּׂדֶה} \hspace{0.3cm}
21~~\foreignlanguage{hebrew}{וַיַּרְא־גַּעַל אֶת־הָעָם}  \hspace{0.3cm}
22~~\foreignlanguage{hebrew}{וַיַּרְא אֶת־הַמִּזְבֵּחַ}  \hspace{0.3cm}
23~~\foreignlanguage{hebrew}{וַיַּרְא אֶת־הַמָּקוֹם מֵרָחֹק}\LTRfootnote{\space \foreignlanguage{hebrew}{מֵרָחֹק} \textit{from a distance}} \hspace{0.3cm}
24~~\foreignlanguage{hebrew}{וַיִּשְׁמַע פַּרְעֹה אֶת־הַדָּבָר הַזֶּה}

\selectlanguage{english}



\subsection{Exercises with the Article}
Write down every noun in the vocabulary section (except the word \foreignlanguage{hebrew}{פַּרְעֹה}) with the definite article attached to it and read it out loud. (The word \foreignlanguage{hebrew}{פַּרְעֹה} is never used with the article.)

\chapter{Chapter 4}

\renewcommand\arraystretch{1.4}

\section{Vocabulary}

\subsection{Verbs}

\begin{center}
	
	% For the centering of the separation between the two columns see the documentation of the array package, page 2 
	
	\begin{tabular}{>{\raggedleft}p{0.175\linewidth} p{0.75\linewidth}}
		\foreignlanguage{hebrew}{אָמַר} & \textit{he said} \\
		\foreignlanguage{hebrew}{וַיֹּ֫אמֶר} & \textit{and he said} \\
		\foreignlanguage{hebrew}{דִּבֶּר} & \textit{he spoke, he said} (pausal form \foreignlanguage{hebrew}{דִּבֵּ֑ר}) \\
		\foreignlanguage{hebrew}{וַיְדַבֵּר} & \textit{and he spoke, and he said} \\
		\foreignlanguage{hebrew}{הָלַךְ} & \textit{he went} \\
		\foreignlanguage{hebrew}{וַיֵּ֫לֶךְ} & \textit{and he went} \\
		\foreignlanguage{hebrew}{בָּא} & \textit{he came, he went inside} (\textit{bā(ʾ)} with silent \foreignlanguage{hebrew}{א}) \\
		\foreignlanguage{hebrew}{וַיָּבֹא} & \textit{and he came, and he went inside} (\textit{wayyāḇō(ʾ)} with silent \foreignlanguage{hebrew}{א}) \\
		\foreignlanguage{hebrew}{הָיָה} & \textit{he was, he became} \\
		\foreignlanguage{hebrew}{וַיְהִי} & \textit{and he was, and he became} \\
	\end{tabular}
\end{center}

\subsection{Nouns}

\begin{center}
	\begin{longtable}{>{\raggedleft}p{0.175\linewidth} p{0.75\linewidth}}
		\foreignlanguage{hebrew}{אִשָּׁה} & \textit{woman} \\
		\foreignlanguage{hebrew}{בֹּקֶר} & \textit{morning} \\
		\foreignlanguage{hebrew}{גִּבּוֺר} & \textit{hero} (as substantive), \textit{strong, mighty} (adj.) \\
		\foreignlanguage{hebrew}{דֶּרֶךְ} & \textit{way, road} \\
		\foreignlanguage{hebrew}{הָר} & \textit{mountain, hill, hill-country} \\
		\foreignlanguage{hebrew}{חַיִל} & \textit{strength, wealth, army} \\
		\foreignlanguage{hebrew}{חֲלוֹם} & \textit{dream} (masc.; pl. \foreignlanguage{hebrew}{חֲלֹמוֺת})  \\
		\foreignlanguage{hebrew}{לַ֫יְלָה} & \textit{night} \\
		\foreignlanguage{hebrew}{מִגְדָּל} & \textit{tower} \\
		\foreignlanguage{hebrew}{מִדְבָּר} & \textit{wilderness, desert} \\
		\foreignlanguage{hebrew}{מַחֲנֶה} & \textit{camp} \\
		\foreignlanguage{hebrew}{עֹז} & \textit{might, strength} \\
		\foreignlanguage{hebrew}{עָנָן} & \textit{cloud, cloud-mass} \\
		\foreignlanguage{hebrew}{עֶרֶב} & \textit{evening} \\
		\foreignlanguage{hebrew}{רָעָב} & \textit{famine, hunger} \\
		\foreignlanguage{hebrew}{שָׁלוֺם} & \textit{welfare, peace, completeness} \\
		\foreignlanguage{hebrew}{שֵׁם} & \textit{name} (pl. \foreignlanguage{hebrew}{שֵׁמוֺת}) \\
		\foreignlanguage{hebrew}{שֶׁמֶשׁ} & \textit{sun} (fem.) \\
		\foreignlanguage{hebrew}{שָׁנָה} & \textit{year} (pl. \foreignlanguage{hebrew}{שָׁנִים}) \\
		\foreignlanguage{hebrew}{תָּוֶךְ} & \textit{midst} (cs.\ st.\ \foreignlanguage{hebrew}{תּוֺךְ}) \\
	\end{longtable}
\end{center}

\subsection{Prepositions}

\begin{center}
	\begin{tabular}{>{\raggedleft}p{0.175\linewidth} p{0.75\linewidth}}
		\foreignlanguage{hebrew}{אֶל} & \textit{to, towards} (spatial), \textit{at, by} (local), \textit{against, according to} \\
		\foreignlanguage{hebrew}{עַל} & \textit{on, upon, above, at, by, against, on account of, because of, about,} \\
		& \textit{concerning} \\
		\foreignlanguage{hebrew}{מִן} & \textit{from, away from, out of}; in comparisons: \textit{(more) than} \\
	\end{tabular}
\end{center}


\subsection{Other Parts of Speech}

\begin{center}
	\begin{tabular}{>{\raggedleft}p{0.175\linewidth} p{0.75\linewidth}}
		\foreignlanguage{hebrew}{כַּאֲשֶׁר} & \textit{as, according to} (modal), \textit{when} (temporal), \textit{because} (causal) (conj.) \\
		\foreignlanguage{hebrew}{כֹּה} & \textit{thus, here} (adv.) \\
		\foreignlanguage{hebrew}{כִּי} & \textit{for, because} (causal), \textit{if} (conditional), \textit{when} (temporal) \textit{that}, etc. (conj.) \\
		\foreignlanguage{hebrew}{עוֺד} & \textit{still, yet, again, besides} (adv.) \\
	\end{tabular}
\end{center}


\section{The Prepositions \foreignlanguage{hebrew}{אֶל}, \foreignlanguage{hebrew}{עַל} and \foreignlanguage{hebrew}{מִן}}
The prepositions \foreignlanguage{hebrew}{אֶל} and \foreignlanguage{hebrew}{עַל} are always separate words. The Preposition \foreignlanguage{hebrew}{עַל} is very often followed by \textit{maqqef}, \foreignlanguage{hebrew}{אֶל} almost always.

The preposition \foreignlanguage{hebrew}{מִן} is a separate word before nouns with the definite article, e.g., \foreignlanguage{hebrew}{מִן־הַמִּזְבֵּחַ} \textit{from the alter}, \foreignlanguage{hebrew}{מִן־הַשָּׂדֶה} \textit{from the field}.

In other cases, \foreignlanguage{hebrew}{מִן} is put immediately in front of the following word. As a consequence, the /n/ of the preposition is immediately followed by another consonant and therefore it is assimilated to that consonant. The following sound changes take place:

\begin{enumerate}[noitemsep]
	\item Assimilation of the /n/ to the first consonant of the following noun which is doubled as a consequence, e.g., \textit{*min-miṣráyim > mimmiṣráyim} \foreignlanguage{hebrew}{מִמִּצְרַיִם}  \textit{from Egypt}, \textit{*min-gilʿād > miggilʿād} \foreignlanguage{hebrew}{מִגִּלְעָד} \textit{from Gilead}
	\item If the following word begins with a guttural or /r/, the /n/ is also assimilated to the  first consonant of the following word but the vowel of the preposition is changed to \textit{ṣere} and the first consonant of the following word is \emph{not} doubled, e.g., \foreignlanguage{hebrew}{מֵאֶפְרַיִם} \textit{mēʾæp̄ráyim < *miʾʾæp̄ráyim < *min-ʾæp̄ráyim} \textit{from Ephraim}, \foreignlanguage{hebrew}{מֵחֶבְרוֺן} \textit{mēḥæḇrōn < *miḥḥæḇrōn < min-ḥæḇrōn} \textit{from Hebron}, \foreignlanguage{hebrew}{מֵרָחוֺק} \textit{from a distance}, \foreignlanguage{hebrew}{מֵאֱלֹהִים} \textit{from God}. This may also happen with the direct article, e.g., \foreignlanguage{hebrew}{מֵהַמֶּלֶךְ}  \textit{from the king}.
	\item The doubling of the first consonant before vocal \textit{šwa} is often lost, e.g. \foreignlanguage{hebrew}{מִקְצֵה} \textit{miqṣē < miqqəṣē} \textit{at the end of} (noun \foreignlanguage{hebrew}{קָצֶה} \textit{end} with the prep. \foreignlanguage{hebrew}{מִן}).
	\item With a word beginning with \foreignlanguage{hebrew}{יְ} \textit{yə-} assimilation of the /n/, loss of the doubling of the /y/ and contraction of \textit{-iy- > -ī-} takes place, e.g., \foreignlanguage{hebrew}{מִיהוּדָה} \textit{mīhūdā < miyyəhūdā < min-yəhūdā}.
\end{enumerate}


\section{Other Local Adverbial Expressions}
In addition to prepositional phrases, Biblical Hebrew uses other means to express spatial and local relations (e.g., \textit{to, towards, in} etc.)

\subsection{The Locative He}
The locative He is the ending \textit{-ā} that is attached to a noun or adverb to indicate the goal of a motion. The locative He is \emph{not} stressed and can be used with common nouns, proper nouns or adverbs, e.g., \foreignlanguage{hebrew}{הַבַּ֫יְתָה} \textit{into the house}, \foreignlanguage{hebrew}{חֶבְר֫וֺנָה} \textit{to Hebron}, \foreignlanguage{hebrew}{שָׁ֫מָּה} \textit{there} (as in \textit{They went there}).

With feminine nouns with the ending \textit{-ā} the locative He is attached to the construct state form (see below), e.g., \foreignlanguage{hebrew}{עַזָּה} \textit{Gaza}, with locative He \foreignlanguage{hebrew}{עַזָּ֫תָה} \textit{to Gaza}.

% Other example \foreignlanguage{hebrew}{רַבָּה} \textit{Rabbah}, with locative He \foreignlanguage{hebrew}{רַבָּ֫תָה} \textit{to Rabbah} as in \foreignlanguage{hebrew}{וַיֵּלֶךְ רַבָּ֫תָה} \textit{and he went to Rabbah} (2\,Sam 12:29)

\subsection{Noun Phrases}
Bare noun phrases without preposition or the locative He can also be used as adverbial expressions indicating spatial relations (\textit{to, towards, in, at} etc.), e.g., the place name \foreignlanguage{hebrew}{בֵּית־אֵל} in the clause \foreignlanguage{hebrew}{וַיָּבֹא הָעָם בֵּית־אֵל} \textit{And the people came to Bethel}.

Bare noun phrases may also be used to indicate time, e.g., \foreignlanguage{hebrew}{הַלַּיְלָה} \textit{this night, tonight} (2\,Sam 19:8), \foreignlanguage{hebrew}{הַשָּׁנִים הָאֵלֶּה} \textit{these years, during these years} (1\,Kgs 17:1). Other adverbial functions are possible, too.


\section{Vowel Changes}
Originally, Biblical Hebrew had the following vowels:

\renewcommand\arraystretch{1.4}

\begin{Center}
	\begin{tabular}{|c|c|}
		\hline
		Short Vowels & Long Vowels \\
		\hline
		\textit{i} & \textit{ī} \\
		\textit{a} & \textit{ā} \\
		\textit{u} & \textit{ū} \\
		\hline
	\end{tabular}
\end{Center}

\vspace{0.2cm}
\noindent These vowels were subject to change in various ways over the course of time. The following outline summarizes the main changes that are relevant for understanding the morphology of Biblical Hebrew.

\begin{enumerate}[noitemsep]
	\item The long vowel /ā/ changed to /ō/ in stressed syllables at a very early date (cp. Hebrew \foreignlanguage{hebrew}{שָׁלוֺם} \textit{peace} with its cognates in Aramaic \foreignlanguage{hebrew}{שְׁלָם} and Arabic \textit{salām}).
	\item Originally long vowels are usually not subject to change. This applies also to long vowels originating from contraction of diphthongs (semivowel /y/ or /w/ + vowel), e.g., \textit{-ay} > \textit{-ē} or \textit{-aw} > \textit{-ō}.
	\item Originally short vowels in closed syllables are preserved.
	\item Originally short vowels in open syllables may be subject to change depending on vowel quality and position of the syllable in the word.
	\begin{enumerate}
		\item In propretonic syllables all short vowels in open syllables are reduced to vocal \textit{šwa} (or \textit{ḥatef šwa}), e.g., \foreignlanguage{hebrew}{צְדָקָה} \textit{< ṣadaqā} \textit{justice}, \foreignlanguage{hebrew}{עֲבֹדָה} \textit{ʿibādā} \textit{work.}
		\item In pretonic syllables original /a/ is preserved in disyllabic words before syllables with originally short vowel or syllables with an originally long vowel, e.g., \foreignlanguage{hebrew}{דָּבָר} \textit{< *dabar} and \foreignlanguage{hebrew}{שָׁלוֺם} \textit{< *šalām}.
		\item Originally short /i/ is preserved in disyllabic words if the next syllable has an originally short vowel, e.g., \foreignlanguage{hebrew}{לֵבָב} \textit{< *libab}. If it is a long vowel, originally short /i/ is reduced to \textit{šwa}, e.g. \foreignlanguage{hebrew}{חֲמוֺר} \textit{< *ḥimār}.
		\item Originally short /u/ in pretonic syllables is always reduced to \textit{šwa}, e.g., \foreignlanguage{hebrew}{רְחוֺב} \textit{< *ruḥāb} \textit{broad open place, plaza}.
		\item If an open syllable with an orginally short vowel is stressed as in some pausal forms of verbs, these vowels are preserved, e.g., \foreignlanguage{hebrew}{יִשְׁמֹ֔רוּ} as opposed to the context form \foreignlanguage{hebrew}{יִשְׁמְרוּ}.
	\end{enumerate}
	\item If preserved, originally short vowels have the following forms in Biblical Hebrew according to the Tiberian Masoretic tradition:
	\begin{enumerate}
		\item The originally short vowel /a/ appears usually either as \textit{qameṣ} or \textit{pataḥ} and at times also as \textit{segol}. If the syllable is stressed, \textit{pataḥ} only occurs in verbal forms; in other forms /a/ becomes \textit{qameṣ} /ā/.
		\item The vowel /i/ appears either as \textit{ḥireq}, \textit{ṣere} or \textit{segol}.
		\item The vowel /u/ appears either as \textit{ḥolem}, \textit{qameṣ ḥaṭuf} or \textit{qibbuṣ}. A \textit{ḥolem} originating from /u/ can only change into \textit{qameṣ ḥaṭuf} or \textit{qibbuṣ}, e.g., \foreignlanguage{hebrew}{כֹּל} \textit{all of}, with maqqef \foreignlanguage{hebrew}{כָּל־} \textit{kɔl} \textit{all of}; \foreignlanguage{hebrew}{דֹּב} \textit{dōḇ < dubb} \textit{bear}, plural \foreignlanguage{hebrew}{דֻּבִּים} \textit{dubbīm}.
	\end{enumerate}
	\item Original short /a/ in an unstressed closed syllable may change to /i/ (law of attenuation), e.g., the proper noun \foreignlanguage{hebrew}{מִרְיָם} which is \foreignlanguage{greek}{Μαριαμ} in the LXX.
	\item Original short /i/ in a closed stressed syllable may change to /a/ (Philippi's Law); e.g., \textit{*bint} > \textit{bat} \foreignlanguage{hebrew}{בַּת} \textit{daughter}.
\end{enumerate}

\noindent \textbf{Notes}
\nopagebreak

\noindent The law of attenuation and Philippi's law are not laws in the strict sense as there are exceptions to them (cf.\ JM §\,29\textit{aa}). They rather reflect a strong tendency to these vowel changes in defined circumstances.




\section{Inflection of Nouns}
Inflection of nouns happens in the following cases: 

\begin{enumerate}[noitemsep]
	\item change of number from singular to plural or dual
	\item in constructions called construct chains in which two (or more) nouns are joined to express the possessive relation and other relations as in \foreignlanguage{hebrew}{בֵּית הַמֶּלֶךְ} \textit{the house of the king}
	\item the attachment of enclitic personal pronouns (ePP) (also called pronominal suffixes in other grammars) as in \foreignlanguage{hebrew}{בֵּיתוֺ} \textit{his house}
\end{enumerate}

\subsection{Gender and Number}
Biblical Hebrew has two categories of gender, masculine (masc.\ or m.) and feminine (fem.\ or f.) and three categories of number, the singular (sg.), the plural (pl.) and for some nouns and uses, the dual (du.).

Masc.\ nouns in the singular are always without an ending whereas fem.\ nouns are often marked as fem.\ by the endings \textit{-ā} or \textit{-t}. A number of fem.\ nouns, though, are not marked as fem.\ by means of a fem.\ ending. An example of a marked fem.\ noun is \foreignlanguage{hebrew}{אִשָּׁה} \textit{woman} with the fem.\ ending; an example of unmarked fem.\ noun is \foreignlanguage{hebrew}{אֵם} \textit{mother} without the fem.\ ending.

In Biblical Hebrew the plural is used with most nouns if two or more entities are meant. The dual is restricted to measures of time and space and for body parts that come in pairs. When used for body parts, the dual replaces the plural; it is used even when more than two items are meant.

\subsection{Absolute State and Construct State}
For expressing the possessive relationship as in the noun phrase \textit{the house of the king} Biblical Hebrew uses  a construction called construct chain in which two or more nouns are joined to form a noun phrase. The following two examples illustrate this.

\begin{Center}
	\begin{tabular}{rl}
		\foreignlanguage{hebrew}{קוֺל יַעֲקֹב}	& \textit{Jacob's voice} (Gen 27:22)\\
		\foreignlanguage{hebrew}{בֵּית הַמֶּלֶךְ}& \textit{the house of the king} (2\,Sam 11:2) \\
	\end{tabular}
\end{Center}

\noindent In noun phrases like these, the first noun is in the construct state (cs.\ st.) and the second noun is in the absolute state (abs.\ st.). The first noun is the governing noun (in Latin \textit{nomen regens}) and the second noun is the dependent noun (in Latin \textit{nomen rectum}). When nouns are used independently or with adjectives as attributes or with another noun in apposition, they are in the absolute state. The form in the abs.\ st.\ is also the lexical form (or citation form).

Construct chains form a stress unit. As the first element loses its main stress there is only one syllable in a construct chain carrying the main stress. This results in vowel changes in the noun in the construct state (see below).

Details of the use of the construct state are presented after the explanation of the vowel changes that come with the inflection of nouns.

\subsection{Nominal Endings}
The following nominal endings are found in Biblical Hebrew. They are attached to the adjective \foreignlanguage{hebrew}{טוֺב} \textit{good} because the vowel of the adjective is unchangeable and it is attested in all forms mentioned. The endings of the dual forms of nouns with the fem.\ ending \textit{-ā} are illustrated with the noun \foreignlanguage{hebrew}{שָׂפָה} \textit{lip, edge} which has changeable vowels. The noun \foreignlanguage{hebrew}{מֹאזְנַ֫יִם} \textit{scale, balance} is always used in the dual (because of the two plates or bowls).


\begin{center}
	\begin{tabular}{|ll|lr|lr|}
		\hline
		& & \multicolumn{2}{c|}{masculine} & \multicolumn{2}{c|}{feminine}  \\
		\hline
		\multirow{2}{*}{sg.} & abs. st. & ∅ & \foreignlanguage{hebrew}{טוֺב} & \textit{-ā} & \foreignlanguage{hebrew}{טוֺבָה} \\
		& cs. st. & ∅ & \foreignlanguage{hebrew}{טוֺב} & \textit{-at} & \foreignlanguage{hebrew}{טוֺבַת} \\
		\hline
		\multirow{2}{*}{pl.} & abs. st. & \textit{-īm} & \foreignlanguage{hebrew}{טוֺבִים} & \textit{-ōt} & \foreignlanguage{hebrew}{טוֺבוֺת} \\
		& cs. st. &  \textit{-ē}  & \foreignlanguage{hebrew}{טוֺבֵי} & \textit{-ōt} & \foreignlanguage{hebrew}{טוֺבוֺת} \\
		\hline
		\multirow{2}{*}{dual} & abs. st. & \textit{-áyim} & \foreignlanguage{hebrew}{מֹאזְנַ֫יִם} & \textit{-ātáyim} & \foreignlanguage{hebrew}{שְׂפָתַ֫יִם} \\
		& cs. st. & \textit{-ē} & \foreignlanguage{hebrew}{מֹאזְנֵי} & \textit{-tē} & \foreignlanguage{hebrew}{שִׂפְתֵי} \\
		\hline
	\end{tabular}
\end{center}

Normally, masc.\ nouns have the plural ending \textit{-īm}, e.g., \foreignlanguage{hebrew}{דָּבָר} \textit{word}, pl. \foreignlanguage{hebrew}{דְּבָרִים}, and fem.\ nouns have the plural ending \textit{-ōt}, e.g., \foreignlanguage{hebrew}{בְּרָכָה} \textit{blessing}, pl. \foreignlanguage{hebrew}{בְּרָכוֺת}. There are, however, masc.\ nouns with the plural ending \textit{-ōt}, e.g., \foreignlanguage{hebrew}{מָקוֺם} \textit{place}, pl. \foreignlanguage{hebrew}{מְקֹמוֺת}, and some fem.\ nouns with the plural ending \textit{-īm}, e.g, \foreignlanguage{hebrew}{שָׁנָה} \textit{year}, pl. \foreignlanguage{hebrew}{שָׁנִים}.

Endings distinguish construct state forms from absolute state forms only in nouns with the feminine singular ending \textit{-ā}, plural nouns with the ending \textit{-īm} and masculine and feminine dual nouns. Masculine singular nouns, feminine nouns without ending like \foreignlanguage{hebrew}{אֶרֶץ}, and plural nouns with the ending \textit{-ōt} are not distinguished by their endings from their counterparts in the absolute state.

Beside the distinct endings in some nouns in construct state, construct chains are recognizable by the presence of a dependent noun and the vowel changes in many nouns in construct state as compared to their form in the absolute state.


\subsection{Vowel Changes}
When nouns are used in the plural or dual or in the construct state, vowel changes may happen. With the addition of endings the position of the stress moves. Nouns in a construct chain form a stress unit with the main stress on the stressed syllable of the last noun in the construct chain. As a result vowels in unstressed open syllables of the noun in construct state are subject to change unless they are originally long. The following explanations focus on nouns with changeable vowels.

\subsubsection{Monosyllabic nouns in the singular}

\vspace{0.25cm}

\begin{Center}
	\begin{tabular}{|l|r|r|l|}
		\hline
		& \multicolumn{1}{c|}{abs.\,st.} & \multicolumn{2}{c|}{cs.\,st.} \\
		\hline
		\multirow{2}{*}{sg.} & \foreignlanguage{hebrew}{דָּם} & \multicolumn{1}{r}{\foreignlanguage{hebrew}{דַּם}} & \textit{blood of} \\
		& \foreignlanguage{hebrew}{עֵץ} & \multicolumn{1}{r}{\foreignlanguage{hebrew}{עֵץ}} & \textit{tree of} \\
		\hline
		\multirow{2}{*}{pl.} & \foreignlanguage{hebrew}{דָּמִים} & \multicolumn{1}{r}{\foreignlanguage{hebrew}{דְּמֵי}} & \\
		& \foreignlanguage{hebrew}{עֵצִים} & \multicolumn{1}{r}{\foreignlanguage{hebrew}{עֲצֵי}} &  \\
		\hline
	\end{tabular}
\end{Center}

\subsubsection{Disyllabic nouns}
Masculine nouns have the following forms:

\begin{table}[H]
	\begin{Center}
		\begin{tabular}{|l|r|r|}
			\hline
			& abs. st. & cs. st. \\
			\hline
			\multirow{3}{*}{sg.} & \foreignlanguage{hebrew}{דָּבָר} & \foreignlanguage{hebrew}{דְּבַר} \\
			& \foreignlanguage{hebrew}{זָקֵן} & \foreignlanguage{hebrew}{זְקַן} \\
			& \foreignlanguage{hebrew}{לֵבָב} & \foreignlanguage{hebrew}{לְבַב} \\
			\hline
			\multirow{3}{*}{pl.} & \foreignlanguage{hebrew}{דְּבָרִים} & \foreignlanguage{hebrew}{דִּבְרֵי} \\
			& \foreignlanguage{hebrew}{זְקָנִים} & \foreignlanguage{hebrew}{זִקְנֵי} \\
			& \foreignlanguage{hebrew}{עֲנָבִים} & \foreignlanguage{hebrew}{עִנְּבֵי} \\
			\hline
		\end{tabular}
	\end{Center}
\end{table}

\noindent Note: The \textit{dageš} forte in \foreignlanguage{hebrew}{עִנְּבֵי} is secondary and has the purpose of making the \textit{šwa} more audible (cf.\ GKC §\,20h).

\medskip

\noindent As an example of a feminine disyllabic noun \foreignlanguage{hebrew}{צְדָקָה} \textit{righteousness} can be mentioned.

\begin{Center}
	\begin{tabular}{|l|r|r|}
		\hline
		& abs. st. & cs. st. \\
		\hline
		sg. & \foreignlanguage{hebrew}{צְדָקָה} & \foreignlanguage{hebrew}{צִדְקַת} \\
		\hline
		pl.	& \foreignlanguage{hebrew}{צְדָקוֺת} & \foreignlanguage{hebrew}{צִדְקוֺת} \\
		\hline
	\end{tabular}
\end{Center}

\subsubsection{Segolate nouns}

In the singular, the cs.\ st.\ of segolate nouns is identical to the abs.\ st.\ except in segolate nouns with a semivowel as second root consonant. In the cs.\ st., the original diphthong is contracted to a long vowel. 

\begin{Center}
	\begin{tabular}{|lr|rl|}
		\hline
		& abs. st. & \multicolumn{2}{c|}{cs. st.}  \\
		\hline
		\multirow{2}{*}{sg.} & \foreignlanguage{hebrew}{בַּיִת} & \foreignlanguage{hebrew}{בֵּית} & \textit{house of} \\
		& \foreignlanguage{hebrew}{מָוֶת} & \foreignlanguage{hebrew}{מוֺת} & \textit{death of} \\
		\hline
	\end{tabular}
\end{Center}

In the plural, segolate nouns use disyllabic bases with two originally short vowels. In the cs.\ st., the original vowel of the first root consonant of the singular form is found (with \textit{ɔ < *u}); the second root consonant has medial \textit{šwa}.

\begin{Center}
	\begin{tabular}{|lr|rl|rl|}
		\hline
		& abs. st. & \multicolumn{2}{c|}{cs. st.} & \multicolumn{2}{c|}{sg. form} \\
		\hline
		\multirow{3}{*}{pl.} & \foreignlanguage{hebrew}{מְלָכִים} & \foreignlanguage{hebrew}{מַלְכֵי} & \textit{kings of} & \foreignlanguage{hebrew}{מֶלֶךְ} & < \textit{*malk} \\
		& \foreignlanguage{hebrew}{שְׁבָטִים} & \foreignlanguage{hebrew}{שִׁבְטֵי} & \textit{tribes of}  & \foreignlanguage{hebrew}{שֵׁבֶט} & < \textit{*šibṭ} \\
		& \foreignlanguage{hebrew}{חֳדָשִׁים} & \foreignlanguage{hebrew}{חָדְשֵׁי} & \textit{months of}  & \foreignlanguage{hebrew}{חׂדֶשׁ} & < \textit{*ḥudš} \\
		\hline
	\end{tabular}
\end{Center}

In segolate nouns with a semivowel as second root consonant the diphthong (or triphthong) is always contracted in the cs.\ st.\ while in the abs.\ st.\ forms with contraction are attested as well as forms without.

\begin{Center}
	\begin{tabular}{|lr|rl|}
		\hline
		& abs. st. & \multicolumn{2}{c|}{cs. st.}  \\
		\hline
		\multirow{3}{*}{pl.} & \foreignlanguage{hebrew}{זֵיתִים} & \foreignlanguage{hebrew}{זֵיתֵי} & \textit{olive trees of} \\
		& \foreignlanguage{hebrew}{חֲיָלִים} & \foreignlanguage{hebrew}{חֵילֵי} & \textit{armies of} \\
		& & \foreignlanguage{hebrew}{מוֺתֵי} & \textit{deaths of} \\
		\hline
	\end{tabular}
\end{Center}

\subsubsection{Geminate nouns} 
Geminate nouns are nouns with an originally geminated final consonant that was simplified after the loss of the short final vowel: \foreignlanguage{hebrew}{עַם} \textit{ʿam < ʿamm < ʿammu}. In forms with an ending the gemination of the final consonant is restored.

\begin{table}[H]
	\begin{Center}
		\begin{tabular}{|l|r|r|}
			\hline
			& abs. st. & cs. st. \\
			\hline
			\multirow{3}{*}{sg.} & \foreignlanguage{hebrew}{עַם} & \foreignlanguage{hebrew}{עַם} \\
			& \foreignlanguage{hebrew}{צֵל} & \foreignlanguage{hebrew}{צֵל} \\
			& \foreignlanguage{hebrew}{דֹּב} & \foreignlanguage{hebrew}{דֹּב} \\
			\hline
			\multirow{3}{*}{pl.} & \foreignlanguage{hebrew}{עַמִּים} & \foreignlanguage{hebrew}{עַמֵּי} \\
			& \foreignlanguage{hebrew}{חִצִּים} & \foreignlanguage{hebrew}{חִצֵּי} \\
			& \foreignlanguage{hebrew}{דֻּבִּים} & \foreignlanguage{hebrew}{דֻּבֵּי}* \\
			\hline
		\end{tabular}
	\end{Center}
\end{table}

\subsubsection{Nouns ending in \textit{segol-He}}
Nouns with the ending \textit{-ǣ} have the cs.\ st.\ ending \textit{-ē}. The endings of the plural are attached to the last consonant of the noun.

\begin{Center}
	\begin{tabular}{|l|r|rl|}
		\hline
		& abs.\ st. & \multicolumn{2}{c|}{cs.\ st.} \\
		\hline
		\multirow{2}{*}{sg.} & \foreignlanguage{hebrew}{קָנֶה} & \foreignlanguage{hebrew}{קְנֵה} & \textit{reed of} \\
		& \foreignlanguage{hebrew}{שָׂדֶה} & \foreignlanguage{hebrew}{שְׂדֵה} & \textit{field of} \\
		\hline
		\multirow{2}{*}{pl.} & \foreignlanguage{hebrew}{קָנִים} & \foreignlanguage{hebrew}{קְנֵי} & \\
		& \foreignlanguage{hebrew}{שָׂדוֺת} & \foreignlanguage{hebrew}{שְׂדוֺת} & \\
		\hline
	\end{tabular}
\end{Center}




\subsection{Construct Chains}

\subsubsection{The Syntax of Construct Chains}
The definiteness of construct chains is determined by the noun in the absolute state, that is the last noun in the construct chain. If it is indefinite, the construct chain is indefinite (Judg 20:40; 1\,Sam 10:3).

\vspace{0.5cm}

\begin{tabular}{>{\raggedleft}p{0.35\linewidth} p{0.55\linewidth}}
	\foreignlanguage{hebrew}{רֹאשׁ־חֲמוֺר} & \textit{a head of a donkey} (2\,Kgs 6:25) \\
	\foreignlanguage{hebrew}{פַת־לֶחֶם} & \textit{a morsel of bread} (Gen 18:5) \\
\end{tabular}

\vspace{0.5cm}

If the dependent noun is definite, the entire construct chain becomes definite (Gen 21:17; Gen 32:31). (Nouns are made definite by the definite article, by a enclitic personal pronoun or a definite dependent noun; proper nouns are definite in themselves; cf.\ Chapter 3, Section 8.1). In a construct chain, only the noun in the absolute state may be definite.

\vspace{0.5cm}

\begin{tabular}{>{\raggedleft}p{0.35\linewidth} p{0.55\linewidth}}
	\foreignlanguage{hebrew}{קוֹל הַנַּעַר} & \textit{the voice of the boy} (Gen 21:17) \\
	\foreignlanguage{hebrew}{שֵׁם הַמָּקוֺם} & \textit{the name of the place} (Gen 32:31) \\
\end{tabular}

\vspace{0.5cm}

\noindent Note:

\noindent If it needs to be avoided that the entire phrase becomes definite because of a definite noun in it as in \textit{a son of Jesse}, Biblical Hebrew uses constructions with the preposition \foreignlanguage{hebrew}{לְ}, e.g., \foreignlanguage{hebrew}{בֵּן לְיִשַׁי בֵּית הַלַּחְמִי} \textit{a son of Jesse the Bethlehemite} (1\,Sam 16:18).

\bigskip

Construct chains may contain more than two nouns.  If there are more than two nouns in a construct chain, only the last one is in the absolute state. The other nouns in the construct chain are in the construct state.

% The following is probably too complicated and not really necessary: While the first noun in construct state is only a governing noun and the noun in the absolute state at the end is only a dependent noun, the nouns in between are the governing noun of the following dependent noun and, at the same time, the dependent noun of the preceding governing noun in the construct chain.

\vspace{0.5cm}

\begin{tabular}{>{\raggedleft}p{0.35\linewidth} p{0.55\linewidth}}
	\foreignlanguage{hebrew}{אַב הֲמוֹן גּוֺיִם} & \textit{the father of a multitude of nations} (Gen 17:4) \\
	\foreignlanguage{hebrew}{יְמֵי שְׁנֵי חַיֵּי אֲבֹתַי} & \multicolumn{1}{p{0.6\linewidth}}{\textit{the days of the years of the lives of my fathers} (Gen 47:9)} \\
	\foreignlanguage{hebrew}{קַרְנוֹת מִזְבַּח קְטֹרֶת הַסַּמִּים} & \textit{the horns of the altar of the incense of spices} (Lev 4:7) \\
\end{tabular}

\vspace{0.5cm}

In a construct chain, nouns in the construct state may have two or more dependent nouns (e.g, 1\,Sam 23:7; Num 20:5) although this is often avoided.\footnote{\space Cf.\ \foreignlanguage{hebrew}{אֶל־נְבִיאֵי אָבִיךָ וְאֶל־נְבִיאֵי אִמֶּךָ} \textit{to the prophets of your fathers and to the prophets of your mother} (2\,Kgs 3:13)} On the other hand, a dependent noun may only have one governing noun which means that an English phrase like \textit{the officials and elders of Succoth} needs to be broken up into two noun phrases in Biblical Hebrew (Judg 8:14).

\vspace{0.5cm}

\begin{tabular}{>{\raggedleft}p{0.35\linewidth} p{0.55\linewidth}}
	\foreignlanguage{hebrew}{עִיר דְּלָתַיִם וּבְרִיחַ} & \textit{a city with gates and bars} (1\,Sam 23:7) \\
	\foreignlanguage{hebrew}{מְקוֹם זֶרַע וּתְאֵנָה וְגֶפֶן וְרִמּוֹן} & \multicolumn{1}{p{0.6\linewidth}}{\textit{a place of seed or figs or vines or pomegranates} (Num 20:5)} \\
	\foreignlanguage{hebrew}{אֶת־שָׂרֵי סֻכּוֹת וְאֶת־זְקֵנֶיהָ} & \textit{the officials of Succoth and its elders} (Judg 8:14) \\
\end{tabular}

\vspace{0.5cm}

As a construct chain is a very close unit nothing may separate its nouns, the only exception being the locative He.

\vspace{0.5cm}

\begin{tabular}{>{\raggedleft}p{0.35\linewidth} p{0.55\linewidth}}
	\foreignlanguage{hebrew}{בֵּ֫יתָה יוֹסֵף} & \textit{to the house of Joseph} (Gen 43:17) \\
\end{tabular}

\vspace{0.5cm}

Attributive adjectives that modify the governing noun of a construct chain and demonstrative pronouns follow the dependent noun at the end of the construct chain.

% Other examples Jer 11:10; 2 Kgs 15:35

\vspace{0.5cm}

\begin{tabular}{>{\raggedleft}p{0.35\linewidth} p{0.55\linewidth}}
	\foreignlanguage{hebrew}{בַּעַל הַבַּיִת הַזָּקֵן} & \textit{the old owner of the house} (Judg 19:22) \\ %cf. 19:16
	\foreignlanguage{hebrew}{פַּךְ הַשֶּׁמֶן הַזֶּה} & \textit{this flask of oil} (2\,Kgs 9:1) \\
\end{tabular}

\vspace{0.5cm}

In the construct chain \foreignlanguage{hebrew}{בַּעַל הַבַּיִת הַזָּקֵן} in Judg 19:22 the adjective \foreignlanguage{hebrew}{זָקֵן} modifies the noun \foreignlanguage{hebrew}{בַּעַל} as shown by the noun phrases \foreignlanguage{hebrew}{אִישׁ זָקֵן} and \foreignlanguage{hebrew}{הָאִישׁ הַזָּקֵן} in Judg 19:16–17 although it would be grammatically possible to translate the construct chain \textit{the owner of the old house}. In 2\,Kgs 9:1, the antecedent of the demonstrative pronoun \foreignlanguage{hebrew}{זֶה} is the noun \foreignlanguage{hebrew}{פַךְ} and not the noun \foreignlanguage{hebrew}{שֶׁמֶן} (cf. the differing renderings of the phrase in Vg and LXX).

\subsubsection{The Functions of Construct Chains}
Besides the possessive relation construct chains express a number of different concepts. Important concepts are:

\vspace{0.5cm}

\noindent Quality or characteristic

\vspace{0.5cm}

\begin{tabular}{>{\raggedleft}p{0.35\linewidth} p{0.55\linewidth}}
	\foreignlanguage{hebrew}{הַר הַקֹֹּדֶשׁ} & \textit{the holy mountain} (lit. \textit{the mountain of holiness}) (Isa 27:13) \\
	\foreignlanguage{hebrew}{אֱלֹהֵי הַנֵּכָר} & \textit{the foreign gods} (lit. \textit{the gods of foreignness}) (1\,Sam 7:3) \\
	\foreignlanguage{hebrew}{מִגְדַּל עֹז} & \textit{a strong tower} (lit. \textit{a tower of strength}) (Judg 9:51) \\
\end{tabular}

\vspace{0.5cm}

\noindent The matter out of which something is made, or the content of something

\vspace{0.5cm}

\begin{tabular}{>{\raggedleft}p{0.35\linewidth} p{0.55\linewidth}}
	\foreignlanguage{hebrew}{רֶכֶב בַּרְזֶל} & \textit{iron chariots} (Judg 1:19) \\
	\foreignlanguage{hebrew}{נְצִיב מֶלַח} & \textit{a pillar of salt} (Gen 19:26) \\
	\foreignlanguage{hebrew}{בַּקְבֻּק דְּבַשׁ} & \textit{a bottle of honey} (1\,Kgs 14:3) \\
\end{tabular}

\vspace{0.5cm}

\noindent If the governing noun expresses a verbal idea, the dependent noun can express the object of the action.

\vspace{0.5cm}

\begin{tabular}{>{\raggedleft}p{0.35\linewidth} p{0.55\linewidth}}
	\foreignlanguage{hebrew}{חֲמַס אָחִיךָ} & \textit{violence against your brother} (Ob 10) \\
	\foreignlanguage{hebrew}{נִקְמַת הֵיכָלוֹ} & \textit{vengeance for his temple} (Jer 50:28) \\
\end{tabular}

\vspace{0.5cm}

\noindent Construct chains can express the partitive idea, i.e., someone or something belonging to a group.

\vspace{0.5cm}

\begin{tabular}{>{\raggedleft}p{0.35\linewidth} p{0.55\linewidth}}
	\foreignlanguage{hebrew}{אַחַד הָעָם} & \textit{one of the people} (1\,Sam 26:15) \\
\end{tabular}

\vspace{0.5cm}

\noindent Adjectives can be used in the cs.\ st.\ as normal nouns as in 1\,Sam 16:4 or for expressing various other ideas.

\vspace{0.5cm}

\begin{tabular}{>{\raggedleft}p{0.35\linewidth} p{0.55\linewidth}}
	\foreignlanguage{hebrew}{זִקְנֵי הָעִיר} & \textit{the elders of the city} (1\,Sam 16:4) \\
	\foreignlanguage{hebrew}{עַם־קְשֵׁה־עֹרֶף} & \textit{a stiff-necked people} (Exod 32:9) \\
	\foreignlanguage{hebrew}{שְׂבַע יָמִים} & \textit{full of days} (Gen 35:29) \\ %(lit. \textit{satiate with years})?
	\foreignlanguage{hebrew}{צִחֵה צָמָא} & \textit{parched with thirst} (Isa 5:13) \\
\end{tabular}

\vspace{0.5cm}


\noindent Although proper nouns are definite in themselves they may occur in special phrases as governing noun in a construct chain.

\vspace{0.5cm}

\begin{tabular}{>{\raggedleft}p{0.35\linewidth} p{0.55\linewidth}}
	\foreignlanguage{hebrew}{יהוה צְבָאוֺת} & \textit{the Lord of hosts} (1\,Sam 1:11) \\
	\foreignlanguage{hebrew}{יַרְדֵּן יְרִיחוֹ} & \textit{the Jordan at Jericho} (Num 33:48) \\
	\foreignlanguage{hebrew}{בֵּית לֶחֶם יְהוּדָה} & \textit{Bethlehem of Judah} (Judg 19:1)) \\
\end{tabular}


% More information about construct chains can be found in JM §§\,92, 129.

\section{Exercises}

\subsection{Sentences with Frequent Verbal Forms}
Translate the following sentences from the Hebrew Bible. Names of persons and geographical names in these sentences: \foreignlanguage{hebrew}{אֲבִימֶלֶךְ}, \foreignlanguage{hebrew}{אַבְנֵר}, \foreignlanguage{hebrew}{אַהֲרֹן}, \foreignlanguage{hebrew}{בִּלְעָם}, \foreignlanguage{hebrew}{בָּלָק}, \foreignlanguage{hebrew}{דָּוִד}, \foreignlanguage{hebrew}{חֹרֵב}, \foreignlanguage{hebrew}{חָרָן}, \foreignlanguage{hebrew}{יַעֲקֹב}, \foreignlanguage{hebrew}{יִפְתָּח}, \foreignlanguage{hebrew}{יִצְחָק}, \foreignlanguage{hebrew}{יִשְׁמָעֵאל}, \foreignlanguage{hebrew}{כַּרְמֶל}, \foreignlanguage{hebrew}{לוֹט}, \foreignlanguage{hebrew}{מֹשֶׁה}, \foreignlanguage{hebrew}{עַזָּה}, \foreignlanguage{hebrew}{עֵלִי}, \foreignlanguage{hebrew}{עֵשָׂו}, \foreignlanguage{hebrew}{צֹעַר}.

\vspace{0.5cm}

\selectlanguage{hebrew}
\noindent
1~~\foreignlanguage{hebrew}{כֹּה־אָמַר הַמֶּלֶךְ}  \hspace{0.3cm}
2~~ \foreignlanguage{hebrew}{וַיֹּאמֶר לוֹ}\LTRfootnote{ \foreignlanguage{hebrew}{לוֹ} \textit{to him}} \foreignlanguage{hebrew}{הָעָם כַּדָּבָר הַזֶּה}  \hspace{0.3cm}
3~~\foreignlanguage{hebrew}{וַיַּעַשׂ בָּלָק כַּאֲשֶׁר דִּבֶּר בִּלְעָם}  \hspace{0.3cm}
4~~\foreignlanguage{hebrew}{כַּדְּבָרִים הָאֵלֶּה דִּבֶּר דָּוִד}  \hspace{0.3cm}
5~~\foreignlanguage{hebrew}{וַיְדַבֵּר יהוה}\LTRfootnote{\space \foreignlanguage{hebrew}{יהוה} read as \foreignlanguage{hebrew}{אֲדֹנָי} (Engl. \textit{Lord})} \foreignlanguage{hebrew}{אֶל־מֹשֶׁה וְאֶל־אַהֲרֹן} \hspace{0.3cm}
6~~\foreignlanguage{hebrew}{וְהוּא־הָלַךְ}\LTRfootnote{ \foreignlanguage{hebrew}{הוּא} \textit{he} (independent personal pronoun)} \foreignlanguage{hebrew}{בַּמִּדְבָּר דֶּרֶךְ יוֹם}  \hspace{0.3cm} \hspace{0.3cm}
7~~\foreignlanguage{hebrew}{וַיֵּלֶךְ עֵשָׂו אֶל־יִשְׁמָעֵאל}  \hspace{0.3cm}
8~~\foreignlanguage{hebrew}{וַיֵּלֶךְ מִשָּׁם יִצְחָק}  \hspace{0.3cm}
9~~\foreignlanguage{hebrew}{וַיֵּלֶךְ חָרָ֫נָה}  \hspace{0.3cm}
10~~\foreignlanguage{hebrew}{וַיֵּלֶךְ שִׁמְשׁוֹן עַזָּ֫תָה}  \hspace{0.3cm}
11~~\foreignlanguage{hebrew}{וַיֵּלֶךְ בְּשָׁלוֹם}  \hspace{0.3cm}
12~~\foreignlanguage{hebrew}{וַיֵּלֶךְ מִשָּׁם אֶל־הַר הַכַּרְמֶל}  \hspace{0.3cm}
13~~\foreignlanguage{hebrew}{הַשֶּׁמֶשׁ יָצָא}\LTRfootnote{\space \foreignlanguage{hebrew}{יָצָא} \textit{yāṣā(ʾ)} \textit{he went out}} \foreignlanguage{hebrew}{עַל־הָאָרֶץ וְלוֹט בָּא צֹ֫עֲרָה}  \hspace{0.3cm}
14~~\foreignlanguage{hebrew}{וַיִּשְׁמַע בָּלָק כִּי בָא בִלְעָם}  \hspace{0.3cm}
15~~\foreignlanguage{hebrew}{בָּא־אַבְנֵר בֶּן־נֵר אֶל־הַמֶּלֶךְ}\LTRfootnote{\space \foreignlanguage{hebrew}{בֶּן־נֵר} \textit{the son of Ner}} \foreignlanguage{hebrew}{אֶל־הַמֶּלֶךְ}  \hspace{0.3cm}
16~~\foreignlanguage{hebrew}{בָּא אֱלֹהִים אֶל־הַמַּחֲנֶה}  \hspace{0.3cm}
17~~\foreignlanguage{hebrew}{וַיָּבֹא יַעֲקֹב מִן־הַשָּׂדֶה בָּעֶרֶב}  \hspace{0.3cm}
18~~\foreignlanguage{hebrew}{וַיָּבֹא אֱלֹהִים אֶל־אֲבִימֶלֶךְ בַּחֲלוֹם הַלָּ֫יְלָה}\LTRfootnote{\space \foreignlanguage{hebrew}{הַלָּ֫יְלָה} pausal form of \foreignlanguage{hebrew}{הַלַּ֫יְלָה}}  \hspace{0.3cm}
19~~\foreignlanguage{hebrew}{וַיָּבֹא אֶל־הַר הָאֱלֹהִים חֹרֵ֫בָה}
\hspace{0.3cm} 20~~\foreignlanguage{hebrew}{וַיָּבֹא מֹשֶׁה וְאַהֲרֹן}\LTRfootnote{\space With a compound subject like \foreignlanguage{hebrew}{מֹשֶׁה וְאַהֲרֹן} the preceding verbal predicate is often in the singular.} \foreignlanguage{hebrew}{אֶל־פַּרְעֹה}  \hspace{0.3cm}
21~~\foreignlanguage{hebrew}{וַיָּבֹא מֹשֶׁה בְּתוֹךְ הֶעָנָן}  \hspace{0.3cm}
22~~\foreignlanguage{hebrew}{וַיָּבֹא אֱלֹהִים אֶל־בִּלְעָם לַיְלָה}  \hspace{0.3cm}
23~~\foreignlanguage{hebrew}{יָּבֹא מַלְאַךְ הָאֱלֹהִים עוֹד אֶל־הָאִשָּׁה}  \hspace{0.3cm}
24~~\foreignlanguage{hebrew}{וַיָּבֹא אִישׁ־אֱלֹהִים אֶל־עֵלִי}  \hspace{0.3cm}
25~~\foreignlanguage{hebrew}{וּמִגְדַּל־עֹז הָיָה בְתוֹךְ־הָעִיר}  \hspace{0.3cm}
26~~\foreignlanguage{hebrew}{וְיִפְתָּח הַגִּלְעָדִי}\LTRfootnote{\space \foreignlanguage{hebrew}{גִּלְעָדִי} \textit{Gileadite}} \foreignlanguage{hebrew}{הָיָה גִּבּוֹר חַיִל}  \hspace{0.3cm}
27~~\foreignlanguage{hebrew}{וַיְהִי רָעָב בָּאָרֶץ}

\selectlanguage{english}

\subsection{Construct Chains}
Identify the construct chains in the clauses in Section 6.1 and write down the abs.\ st.\ forms of the nouns in the cs.\ st. and identify the vowel changes.

\hspace{0.5cm}

\subsection{Translation into English}
Translate the following construct chains into English:
\medskip

\noindent \foreignlanguage{hebrew}{גִּבּוֺר חַיִל}

\noindent \foreignlanguage{hebrew}{דֶּרֶךְ מִדְבַּר מוֺאָב}

\noindent \foreignlanguage{hebrew}{͏͏͏͏͏יְמֵי הַמִּשְׁתֶּה} (\foreignlanguage{hebrew}{יָמִים} plural of \foreignlanguage{hebrew}{יוֺם})

\hspace{0.5cm}

\subsection{Translation into Hebrew}
Translate into Hebrew using construct chains:

\medskip

\noindent man of words

\noindent a strong tower (lit. a tower of strength)

\noindent a man of blood (use the plural of the word for blood)

\noindent the servant of the king

\noindent the servants of the king of \foreignlanguage{hebrew}{יִשְׂרָאֵל}



\chapter{Chapter 5}

\section{Vocabulary}

\subsection{Verbs}

\begin{center}
	
	% For the centering of the separation between the two columns see the documentation of the array package, page 2 
	
	\begin{tabular}{>{\raggedleft}p{0.175\linewidth} p{0.75\linewidth}}
		\foreignlanguage{hebrew}{אכל} & Q.\ \textit{to eat} \\
		\foreignlanguage{hebrew}{אמר} & Q.\ \textit{to say} \\
		\foreignlanguage{hebrew}{ידע} & Q.\ \textit{to know} \\
		\foreignlanguage{hebrew}{ילד} & Q.\ \textit{to bear, beget} \\
		\foreignlanguage{hebrew}{ירד} & Q.\ \textit{to go down, come down, descend} \\
		\foreignlanguage{hebrew}{ישׁב} & Q.\ \textit{to sit, remain, dwell} \\
		\foreignlanguage{hebrew}{כבד} & Q.\ \textit{to be heavy; to be insensitive, dull; to be honored} (stative verb, SC \foreignlanguage{hebrew}{כָּבֵד}, PC \foreignlanguage{hebrew}{יִכְבַּד}) \\
		\foreignlanguage{hebrew}{לקח} & Q.\ \textit{to take} \\
		\foreignlanguage{hebrew}{מלך} & Q.\ \textit{to reign} (as king) \textit{to become king} \\
		\foreignlanguage{hebrew}{נפל} & Q.\ \textit{to fall} \\
		\foreignlanguage{hebrew}{נתן} & Q.\ \textit{to give, to put, to make into, to allow} \\
		\foreignlanguage{hebrew}{ראה} & Q.\ \textit{to see} \\
		\foreignlanguage{hebrew}{שׁאל} & Q.\ \textit{to ask, inquire} \\
		\foreignlanguage{hebrew}{שׁלח} & Q.\ \textit{to send} \\
		\foreignlanguage{hebrew}{שׁמע} & Q.\ \textit{to hear, to listen to} \\
		\foreignlanguage{hebrew}{שׁמר} & Q.\ \textit{to keep, watch, preserve} \\
	\end{tabular}
\end{center}

\subsection{Nouns}

\begin{center}
	\begin{longtable}{>{\raggedleft}p{0.175\linewidth} p{0.75\linewidth}}
		\foreignlanguage{hebrew}{אָב} & \textit{father} \\
		\foreignlanguage{hebrew}{אָדָם} & \textit{man, mankind} \\
		\foreignlanguage{hebrew}{אָח} & \textit{brother} \\
		\foreignlanguage{hebrew}{אָחוֺת} & \textit{sister} \\
		\foreignlanguage{hebrew}{אָמָה} & \textit{slave, handmaid} \\
		\foreignlanguage{hebrew}{אֲרוֺן} & \textit{chest, ark} (with the article \foreignlanguage{hebrew}{הָאָרוֺן}) \\
		\foreignlanguage{hebrew}{אֵשׁ} & \textit{fire} (fem.) \\
		\foreignlanguage{hebrew}{בַּיִת} & \textit{house} \\
		\foreignlanguage{hebrew}{בַּת} & \textit{daughter} \\
		\foreignlanguage{hebrew}{כָּבוֺד} & \textit{glory, honor} \\
		\foreignlanguage{hebrew}{כֹּל} & \textit{the whole, every, all} \\
		\foreignlanguage{hebrew}{כְּלִי} & \textit{article, utensil, vessel, weapon} \\
		\foreignlanguage{hebrew}{לֵבָב} & \textit{heart} (cs.\ st. \foreignlanguage{hebrew}{לְבַב}) \\
		\foreignlanguage{hebrew}{לֶחֶם} & \textit{bread, food} \\
		\foreignlanguage{hebrew}{מַיִם} & \textit{water} \\
		\foreignlanguage{hebrew}{נַעַר} & \textit{boy, lad, youth} \\
		\foreignlanguage{hebrew}{נַעֲרָה} & \textit{young woman, girl; maid, attendant, servant} \\ % DCH
		\foreignlanguage{hebrew}{פֶּה} & \textit{mouth} \\
		\foreignlanguage{hebrew}{קוֺל} & \textit{voice, sound} (masc., pl.\ \foreignlanguage{hebrew}{קֹלוֺת}) \\
		\foreignlanguage{hebrew}{רֹאשׁ} & \textit{head} \textit{rō(ʾ)š} \\
		\foreignlanguage{hebrew}{שָׁמַיִם} & \textit{heaven, sky} \\
	\end{longtable}
\end{center}

\subsection{Other Parts of Speech}

\begin{center}
	\begin{tabular}{>{\raggedleft}p{0.175\linewidth} p{0.75\linewidth}}
		\foreignlanguage{hebrew}{לֹא} & \textit{not, no} \\
	\end{tabular}
\end{center}



\section{Irregular Nouns in Biblical Hebrew}
A number of nouns have irregular forms in the absolute state in the plural and in the construct state in both the singular and the plural. Most of these nouns are very frequent in the Hebrew Bible. Therefore, familiarity with these forms is important.

\begin{center}
	\begin{longtable}{|r|r|r|r|r|r|l|}
		\hline
		\multicolumn{3}{|c|}{Singular} & \multicolumn{3}{c|}{Plural} & Meaning\\
		\cline{1-6}\foreignlanguage{hebrew}{}
		abs.\ st. & cs.\ st. & with ePP & abs.\ st. & cs.\ st. & with ePP & \\
		\hline
		\endhead
		\hline
		\endfoot
		\foreignlanguage{hebrew}{אָב} & \foreignlanguage{hebrew}{אֲבִי} & \foreignlanguage{hebrew}{אָבִ֫ינוּ} & \foreignlanguage{hebrew}{אָבוֹת} & \foreignlanguage{hebrew}{אֲבוֹת} & \foreignlanguage{hebrew}{אֲבוֹתַי} & \textit{father} \\
		& & \foreignlanguage{hebrew}{אֲבִיכֶם} & & & \foreignlanguage{hebrew}{אֲבוֹתֵיכֶם} & \\
		\foreignlanguage{hebrew}{אָח} & \foreignlanguage{hebrew}{אֲחִי} & \foreignlanguage{hebrew}{אָחִ֫ינוּ} & \foreignlanguage{hebrew}{אַחִים} & \foreignlanguage{hebrew}{אֲחֵי} &  \foreignlanguage{hebrew}{אַחַי} & \textit{brother}\\
		& & \foreignlanguage{hebrew}{אֲחִיכֶם} & & & \foreignlanguage{hebrew}{אֲחֵיהֶם} & \\
		\newpage
		\foreignlanguage{hebrew}{אָחוֹת} & \foreignlanguage{hebrew}{אֲחוֹת} & \foreignlanguage{hebrew}{אֲחוֹתִי} & \foreignlanguage{hebrew}{*אֲחָיוֹת} & \foreignlanguage{hebrew}{*אַחֲיוֹת} & \foreignlanguage{hebrew}{אַחְיוֹתָיו} & \textit{sister}\\
		& & & & & \foreignlanguage{hebrew}{אֲחוֹתֵיכֶם} & \\
		\foreignlanguage{hebrew}{בֵּן} & \foreignlanguage{hebrew}{בֶּן} & \foreignlanguage{hebrew}{בְּנִי} & \foreignlanguage{hebrew}{בָּנִים} & \foreignlanguage{hebrew}{בְּנֵי} & \foreignlanguage{hebrew}{בָּנַי} & \textit{son}\\
		& & \foreignlanguage{hebrew}{בִּנְךָ} & & & \foreignlanguage{hebrew}{בְּנֵיכֶם} & \\
		\foreignlanguage{hebrew}{בַּת} & \foreignlanguage{hebrew}{בַּת} & \foreignlanguage{hebrew}{בִּתִּי} & \foreignlanguage{hebrew}{בָּנוֹת} & \foreignlanguage{hebrew}{בְּנוֹת} & \foreignlanguage{hebrew}{בְּנוֹתַי} & \textit{daughter}\\
		& & & & & \foreignlanguage{hebrew}{בְּנוֹתֵיכֶם} & \\
		\foreignlanguage{hebrew}{אִישׁ} & \foreignlanguage{hebrew}{אִישׁ} & \foreignlanguage{hebrew}{אִישִׁי} & \foreignlanguage{hebrew}{אֲנָשִׁים} & \foreignlanguage{hebrew}{אַנְשֵׁי} & \foreignlanguage{hebrew}{אֲנָשַׁי} & \textit{man}\\
		& & & & & \foreignlanguage{hebrew}{אַנְשֵׁיהֶם} & \\
		\foreignlanguage{hebrew}{אִשָּׁה} & \foreignlanguage{hebrew}{אֵ֫שֶׁת} & \foreignlanguage{hebrew}{אִשְׁתִּי} & \foreignlanguage{hebrew}{נָשִׁים} &
		\foreignlanguage{hebrew}{נְשֵׁי} & \foreignlanguage{hebrew}{נָשַׁי} & \textit{woman}\\
		& & & & & \foreignlanguage{hebrew}{נְשֵׁיכֶם} & \\
		\foreignlanguage{hebrew}{אָמָה} & \foreignlanguage{hebrew}{*אֲמַת} & \foreignlanguage{hebrew}{אֲמָתִי} & \foreignlanguage{hebrew}{אֲמָהוֹת} & \foreignlanguage{hebrew}{אַמְהוֹת} & \foreignlanguage{hebrew}{אַמְהׂתַי} & \textit{female slave}\\
		& & & & & \foreignlanguage{hebrew}{אַמְהׂתֵיכֶם} & \\
		\foreignlanguage{hebrew}{בַּ֫יִת} & \foreignlanguage{hebrew}{בֵּית} & \foreignlanguage{hebrew}{בֵּיתִי} & \foreignlanguage{hebrew}{בָּתִּים} & \foreignlanguage{hebrew}{בָּתֵּי} & \foreignlanguage{hebrew}{בָּתֵּינוּ} & \textit{house}\\
		\foreignlanguage{hebrew}{יוֹם} & \foreignlanguage{hebrew}{יוֹם} & \foreignlanguage{hebrew}{יוֹמוֹ} & \foreignlanguage{hebrew}{יָמִים} & \foreignlanguage{hebrew}{יְמֵי} & \foreignlanguage{hebrew}{יָמַי} & \textit{day}\\
		& & & & & \foreignlanguage{hebrew}{יְמֵיכֶם} & \\
		\foreignlanguage{hebrew}{רֹאשׁ} & \foreignlanguage{hebrew}{רֹאשׁ} & \foreignlanguage{hebrew}{רֹאשִׁי} & \foreignlanguage{hebrew}{רָאשִׁים} & \foreignlanguage{hebrew}{רָאשֵׁי} & \foreignlanguage{hebrew}{רָאשֶׁ֫יךָ} & \textit{head}\\
		& & & & & \foreignlanguage{hebrew}{רָאשֵׁיכֶם} & \\
		\foreignlanguage{hebrew}{עִיר} & \foreignlanguage{hebrew}{עִיר} & \foreignlanguage{hebrew}{עִירִי} & \foreignlanguage{hebrew}{עָרִים} & \foreignlanguage{hebrew}{עָרֵי} & \foreignlanguage{hebrew}{עָרָיו} & \textit{city, town}\\
		& & & & & \foreignlanguage{hebrew}{עָרֵיכֶם} & \\
		\foreignlanguage{hebrew}{פֶּה} & \foreignlanguage{hebrew}{פִּי} & \foreignlanguage{hebrew}{פִּי} & \foreignlanguage{hebrew}{פֵּיוֹת} & \foreignlanguage{hebrew}{} & \foreignlanguage{hebrew}{} & \textit{mouth}\\
		& & \foreignlanguage{hebrew}{פִּיכֶם} & or \foreignlanguage{hebrew}{פִּיּוֹת} & & & \\
		\foreignlanguage{hebrew}{כְּלִי	} & \foreignlanguage{hebrew}{כְּלִי} & \foreignlanguage{hebrew}{כִּלְיְךָ} & \foreignlanguage{hebrew}{כֵּלִים} & \foreignlanguage{hebrew}{כְּלֵי} & \foreignlanguage{hebrew}{כֵּלַי} & \textit{vessel}\\
		& & & & & \foreignlanguage{hebrew}{כְּלֵיכֶם} & \\
		\foreignlanguage{hebrew}{} & \foreignlanguage{hebrew}{} & \foreignlanguage{hebrew}{} & \foreignlanguage{hebrew}{מַ֫יִם} & \foreignlanguage{hebrew}{מֵי} & \foreignlanguage{hebrew}{מֵימֶ֫יךָ} & \textit{water} \\
		& & & & or \foreignlanguage{hebrew}{מֵימֵי} & & \\
		& & & \foreignlanguage{hebrew}{שָׁמַ֫יִם} & \foreignlanguage{hebrew}{שְׁמֵי} & \foreignlanguage{hebrew}{שָׁמֶ֫יךָ} & \textit{sky, heaven}\\
		& & & & & \foreignlanguage{hebrew}{שְׁמֵיכֶם} & \\
		\hline
	\end{longtable}
\end{center}

\noindent \textbf{Notes}
\nopagebreak
\begin{enumerate}[noitemsep]
	\item The asterisk * indicates that the form is not attested in Biblical Hebrew.
	\item More information can be found in P. Joüon and T. Muraoka, \textit{A Grammar of Biblical Hebrew}, §\,98 (abbr.\ JM). Other irregular nouns that are mentioned there, are \foreignlanguage{hebrew}{חָם} \textit{husband’s father,} \foreignlanguage{hebrew}{חָמוֹת}* \textit{husband’s mother,} \foreignlanguage{hebrew}{שֶׂה} \textit{a head of small livestock}.
	\item The noun \foreignlanguage{hebrew}{אָח} \textit{brother} has the following forms with enclitic personal pronoun that need to be mentioned: \foreignlanguage{hebrew}{אֶחָ֑י} \textit{my brothers} (pausal form), \foreignlanguage{hebrew}{אֶחָיו} \textit{ʾæḥāw} \textit{his brothers.} (with silent /y/)
	\item The plural form \foreignlanguage{hebrew}{אֲחָיוֹת} \textit{sisters} is attested in Rabbinic Hebrew but not in Biblical Hebrew.
	\item The \textit{dageš} in the \foreignlanguage{hebrew}{ת} in the plural forms \foreignlanguage{hebrew}{בָּתִּים} \textit{bātīm}, \foreignlanguage{hebrew}{בָּתֵּי} \textit{bātē}, \foreignlanguage{hebrew}{בָּתֵּינוּ} \textit{bātēnū} \textit{our houses}, etc.\ is an exceptional \textit{dageš lene} because it is preceded by a full (long) vowel. According to a different explanation, the \textit{dageš} is a \textit{dageš forte}; the forms are then pronounced as \textit{bāttīm}, \textit{bāttē}, \textit{bāttēnū}, etc., with an exceptional \textit{qameṣ gadol} in a closed unstressed syllable.
	\item The nouns \foreignlanguage{hebrew}{מַ֫יִם} and \foreignlanguage{hebrew}{שָׁמַ֫יִם} are exceptional plural nouns (JM §\,90f, 91f).
\end{enumerate}

% For the plural forms of the noun \foreignlanguage{hebrew}{בית} cf. the literature quoted in \textit{HALOT}/Ges18, esp. Brockelmann, GVG I, 430 (§ 229); for the different analysis cf. JM § 98f, Lettinga/von Siegenthal §38, 333 (p. 108).

\section{The Noun \foreignlanguage{hebrew}{כֹּל}}
The noun \foreignlanguage{hebrew}{כֹּל} can be used as a substantive in the abs.\ st.\ meaning \textit{the whole, entirety, everything}, e.g., \foreignlanguage{hebrew}{וַיהוָה בֵּרַךְ אֶת־אַבְרָהָם בַּכֹּל} \textit{And the Lord had blessed Abraham in all things} (Gen 24:1).


It is much more frequently used as the governing noun in construct chains with the cs.\ st.\ forms \foreignlanguage{hebrew}{כֹּל}, \foreignlanguage{hebrew}{כָּל} and \foreignlanguage{hebrew}{כָּל־}. Used in this way, \foreignlanguage{hebrew}{כֹּל} means \textit{all (of)}, \textit{every} or \textit{any}. Illustrative examples are:

\bigskip

\begin{tabular}{>{\raggedleft}p{0.35\linewidth} p{0.55\linewidth}}
	\foreignlanguage{hebrew}{כָּל־מָקוֺם} & \textit{every place} (Deut 12:13) \\
	\foreignlanguage{hebrew}{כָל־אִישׁ} & \textit{everyone} (lit. \textit{every man} [person]) (Gen 45:1) \\
	\foreignlanguage{hebrew}{כָּל־תְּפִלָּה כָל־תְּחִנָּה} & \textit{any prayer, any supplication} (1\, Kgs 8:38) \\
	\foreignlanguage{hebrew}{כָּל־מַמְלָכוֹת} & \textit{all kingdoms} (1\,Kgs 10:20) \\
	\foreignlanguage{hebrew}{כָּל־הָאָרֶץ} & \textit{all the land, the entire land} (Gen 13:15) \\
	\foreignlanguage{hebrew}{כֹּל הָאֲרָצוֹת} & \textit{all the lands} (Gen 41:54) \\
\end{tabular}

\medskip

\noindent \textit{Note}

\noindent The particular combination of signs \foreignlanguage{hebrew}{כָּל} is to be pronounced \textit{kāl} only \emph{once} in the Hebrew Bible in Isa 40:12 (3 m.\ sg.\ suffix conjugation of the verb \foreignlanguage{hebrew}{כול}). In all other instances \foreignlanguage{hebrew}{כָּל} is the cs.\ st. of \foreignlanguage{hebrew}{כֹּל} and is to be pronounced \textit{kɔl}.



\section{Stem Formations and Conjugations in Biblical Hebrew}
The morphology of the verb in Biblical Hebrew is characterized by a set of \emph{stem formations} (Hebrew \foreignlanguage{hebrew}{בִּנְיָנִים}, singular \foreignlanguage{hebrew}{בִּנְיָן}) and a set of \emph{conjugations} consisting of finite forms expressing tense, aspect and mood and the participle and the two infinitives as non-finite nouns (verbal nouns).

The finite forms with the suffix conjugation, prefix conjugation, jussive, cohortative and imperative are introduced in Chapters 5--7, the non-finite forms with the active and the passive participle and the infinitive construct and infinitive absolute in Chapters 8--9.

The system of the \textit{binyanim} consists of the seven main \textit{binyanim}: Qal, Niphal, Hiphil (with its passive counterpart Hophal), Piel (with its passive and reflexive counterparts Pual and Hitpael, respectively). Compared to the Qal, the other \textit{binyanim} are characterized by prefixes or the gemination of the second root consonant. These additional elements modify the meaning of forms. The following examples illustrate this.

\begin{center}
	\begin{tabular}{ll}
		Qal \foreignlanguage{hebrew}{מָלַךְ} \textit{he reigned as king} & Hiphil \foreignlanguage{hebrew}{הִמְלִיךְ} \textit{he made (someone) king} \\
		Qal \foreignlanguage{hebrew}{אָבַד} \textit{he perished} & Piel \foreignlanguage{hebrew}{אִבַּד} \textit{he made (someone) perish} \\
		Qal \foreignlanguage{hebrew}{מָכַר} \textit{he bought} & Niphal \foreignlanguage{hebrew}{נִמְכַּר} \textit{he was sold} \\
	\end{tabular}
\end{center}

The Qal forms are introduced in Chapters 5--15; the other \textit{binyanim} are introduced beginning in Chapter 16. About 69\,\% of all verbal forms are Qal forms, making this \textit{binyan} most frequent one by far. Hiphil forms are about 13\,\%, Piel forms about 9\,\% and Niphal forms about 6\,\%.


\section{The Suffix Conjugation Qal}

\subsection{The Forms of the Suffix Conjugation Qal}

\noindent The suffix conjugation is characterized by suffixes that are attached to the verbal root at the end. Other terms for the suffix conjugation are \textit{perfect} or in Hebrew \foreignlanguage{hebrew}{עבר} (\foreignlanguage{hebrew}{עָבַר}) \textit{past tense}. As the terms \textit{perfect} and \foreignlanguage{hebrew}{עבר} make implicit statements about the meaning and use of these forms, the term \textit{suffix conjugation} is preferred here because it pertains exclusively to formal features.

In the suffix conjugation Qal, Hebrew makes a formal distinction between fientive or active verbs on the one side and stative verbs on the other side. Fientive verbs are much more frequent than stative verbs. Among the stative verbs, Hebrew distinguishes between verbs with the thematic vowel /e/ and verbs with the thematic vowel /o/. The last category is very infrequent except for the verb \foreignlanguage{hebrew}{יָכֹל} \textit{to be able to, prevail} with 182 occurrences.

\medskip

\renewcommand\arraystretch{1.4}

\begin{Center}
	\begin{tabular}{|ll|l|r|r|r|}
		\hline
		\multicolumn{2}{|c|}{} & suffix & fientive verb & \multicolumn{2}{c|}{stative verb} \\
		\hline 
		sg. & 3 m. & -∅ & \foreignlanguage{hebrew}{כָּתַב} &  \foreignlanguage{hebrew}{כָּבֵד} & \foreignlanguage{hebrew}{קָטֹן} \\
		& 3 f. & \textit{-ā} & \foreignlanguage{hebrew}{כָּתְבָָה} &  \foreignlanguage{hebrew}{כָּבְדָה} &  \foreignlanguage{hebrew}{קָטְנָה} \\
		& 2 m. & \textit{-tā} & \foreignlanguage{hebrew}{כָּתַ֫בְתָּ} & \foreignlanguage{hebrew}{כָּבַ֫דְתָּ} & \foreignlanguage{hebrew}{קָטֹ֫נְתָּ} \\
		& 2 f. & \textit{-t} & \foreignlanguage{hebrew}{כָּתַבְתְּ} & \foreignlanguage{hebrew}{כָּבַדְתְּ} & \foreignlanguage{hebrew}{קָטֹנְתְּ} \\
		& 1 c. & \textit{-tī} & \foreignlanguage{hebrew}{כָּתַ֫בְתִּי} & \foreignlanguage{hebrew}{כָּבַ֫דְתִּי} & \foreignlanguage{hebrew}{קָטֹ֫נְתִּי} \\
		pl. & 3 c. & \textit{-ū} & \foreignlanguage{hebrew}{כָּתְבוּ} &  \foreignlanguage{hebrew}{כָּבְדוּ} & \foreignlanguage{hebrew}{קָטְנוּ} \\
		& 2 m. & \textit{-tæm} & \foreignlanguage{hebrew}{כְּתַבְתֶּם} & \foreignlanguage{hebrew}{כְּבַדְתֶּם} & \foreignlanguage{hebrew}{קְטָנְתֶּם} \\
		& 2 f. & \textit{-tæn} & \foreignlanguage{hebrew}{כְּתַבְתֶּן} & \foreignlanguage{hebrew}{כְּבַדְתֶּן} & \foreignlanguage{hebrew}{קְטָנְתֶּן} \\
		& 1 c. & \textit{-nū} & \foreignlanguage{hebrew}{כָּתַ֫בְנוּ} & \foreignlanguage{hebrew}{בָּבַ֫דְנוּ} & \foreignlanguage{hebrew}{קָטֹ֫נּוּ} \\
		\hline
	\end{tabular}
\end{Center}

\bigskip

The suffix conjugation forms include an expression of the subject in the suffixes. An overt subject is grammatically not necessary but frequently present. 

The suffixes can be categorized into three groups: The vocalic suffixes in the 3 f.\  sg.\ and 3 c.\ pl.\ forms consist of only a vowel. These suffixes are stressed in context forms and unstressed in pausal forms. Among the consonantal suffixes -- those that begin with a consonant -- light and heavy consonantal suffixes can be distinguished. Whereas the light suffixes in the 2 m./f.\  sg., 1 c.\ sg.\ and 1 c.\ pl.\ forms are not stressed (except in a certain set of forms), the heavy suffixes in the 2 m./f.\  pl.\ forms are stressed.

Attaching vocalic suffixes or heavy suffixes leads to vowel reduction. In the case of heavy suffixes the first vowel is reduced. When vocalic suffixes are attached, the second vowel is reduced in context forms. In pausal forms the thematic vowel is preserved and stressed, e.g., \foreignlanguage{hebrew}{כָּתָ֑בוּ}.

As examples of distinct pausal forms can be mentioned: 3 m.\ sg.\ \foreignlanguage{hebrew}{כָּתָ֑ב}; forms with vocalic suffix: 3 f.\  sg.\  \foreignlanguage{hebrew}{כָּתָ֑בָה}, \foreignlanguage{hebrew}{כָּבֵ֑דָה} and 3 c.\ pl.\ \foreignlanguage{hebrew}{כָּתָ֑בוּ}, \foreignlanguage{hebrew}{יָכֹ֑לוּ}; forms with light suffix with lengthening of the thematic vowel: \foreignlanguage{hebrew}{כָּתָ֑בְתָּ} etc.

The thematic vowel of static verbs of the type \foreignlanguage{hebrew}{כָּבֵד} changes from /e/ to /a/ in all forms with consonantal suffix.

Sometimes the suffix of the 2 m.\  sg.\ is spelled with a vowel letter: \foreignlanguage{hebrew}{כָּתַ֫בְתָּה}.

% German Kontaktstellung; JM §42e "the coming-together of results in gemination" (or something like that)

In 2\textsuperscript{nd} person forms and 1 c.\ sg.\ forms of verbs with /t/ as third radical, the /t/ of the root and the initial /t/ of the suffix stand next to each other; the \foreignlanguage{hebrew}{ת} is spelled only once and it is doubled, e.g., \textit{kāráttī} \foreignlanguage{hebrew}{כָּרַ֫תִּי} \textit{I have cut}.

Although in forms of verbs with /n/ as third radical the /n/ is immediately followed by the consonant /t/ in all forms with consonantal suffix except the 1 c.\ pl.\ forms, the /n/ of the root is \textit{not} assimilated to the following /t/, e.g., \foreignlanguage{hebrew}{שָׁכַ֫נְתִּי} \textit{I dwelt}. An exception is the verb \foreignlanguage{hebrew}{נתן}, e.g., \foreignlanguage{hebrew}{נָתַ֫תִּי} < \textit{nātantī} \textit{I gave}, \foreignlanguage{hebrew}{נְתַתֶּם} < \textit{nətantæm} \textit{you gave}. %LvS 496

In 1 c.\ pl.\ forms of verbs with /n/ as third radical (e.g., \foreignlanguage{hebrew}{נתן}) the /n/ of the root and the initial /n/ of the suffix \textit{-nū} stand next to each other; the \foreignlanguage{hebrew}{נ} is spelled only once and it is doubled, e.g., \textit{nātánnū} \foreignlanguage{hebrew}{נָתַ֫נּוּ} \textit{we gave}.

The original forms of the suffixes may have bearing on verbal forms with enclitic personal pronouns. The zero morpheme of the 3 m.\  sg.\ form originates from the loss of a short final vowel *\textit{kataba > kātab} without an ending. The original forms of the suffixes of the 3 f.\  sg.\ and 2 m./f.\  sg.\ need to be mentioned here, too.

\begin{Center}
	\begin{tabular}{lllr}
		3 m. sg. & \textit{-a} > ∅ &  \textit{*kataba > kātaḇ} & \foreignlanguage{hebrew}{כָּתַב} \\
		3 f. sg. & \textit{-at} > \textit{-ā} &  \textit{*katabat > kātǝḇā} & \foreignlanguage{hebrew}{כָּתְבָה} \\
		2 m. sg. & \textit{-ta} > \textit{-tā} &  \textit{*katabta > kātáḇtā} & \foreignlanguage{hebrew}{כָּתַ֫בְתָּ} \\
		2 f. sg. & \textit{-ti} > \textit{-t} &  \textit{*katabti > kātaḇt} & \foreignlanguage{hebrew}{כָּתַבְתְּ} \\
	\end{tabular}
\end{Center} 

\bigskip

\subsection{The Suffix Conjugation Qal of Guttural Verbs}

\noindent Suffix conjugation forms of guttural verbs are characterized by only few differences in comparison with the verbs without gutturals. In verbs I gutt.\ and II gutt.\ the vocal \textit{šwa} is replaced by \textit{ḥatef pataḥ} in the 3 f.\  sg.\ and 3 c.\ pl.\ and the 2 m./f.\  pl., respectively. In verbs III gutt., the consonant cluster in the 2 f.\  sg.\ form is broken up by insertion of a vowel. The \textit{dageš lene} in the /t/ of the suffix that is found in the verbal forms without guttural is retained.

\begin{Center}
	\begin{tabular}{llrrr}
		& & I gutt. & II gutt. & III gutt. \\
		sg. & 3 f. & & \foreignlanguage{hebrew}{בָּחֲרָה} & \\
		& 2 f. & & & \foreignlanguage{hebrew}{שָׁלַ֫חַתְּ} \\
		pl. & 3 c. & & \foreignlanguage{hebrew}{בָּחֲרוּ} & \\
		& 2 m. & \foreignlanguage{hebrew}{עֲזַבְתֶּם} & & \\
		& 2 f. & \foreignlanguage{hebrew}{עֲזַבְתֶּן} & & \\
	\end{tabular}
\end{Center}


\subsection{The Use of the Suffix Conjugation}
The use of the suffix conjugation and how it is translated depends on the verb type. Suffix conjugations forms of fientive verbs have different translation options than the same forms of stative verbs.

\subsubsection{Fientive verbs}

The suffix conjugation is used for events in the past.
\medskip

\begin{tabular}{>{\raggedleft}p{0.35\linewidth} p{0.55\linewidth}}
	\foreignlanguage{hebrew}{מַיִם שָׁאַל חָלָב נָתָ֫נָה} & \textit{He asked for water, she gave} [him] \textit{milk.} (Judg 5:25) \\
\end{tabular}

\medskip
\noindent Sometimes the past event has bearing on the present. Then the present perfect should be used in the English translation.
\medskip

\begin{tabular}{>{\raggedleft}p{0.35\linewidth} p{0.55\linewidth}}
	\foreignlanguage{hebrew}{וְהִנֵּה נָתַן יְהוָה עֲלֵיכֶם מֶלֶךְ} & \textit{... and see, the Lord has set a king over you.} (1\,Sam 12:13) \\
	\foreignlanguage{hebrew}{קָרַע יְהוָה אֶת־מַמְלְכוּת יִשְׂרָאֵל מֵעָלֶיךָ הַיּוֹם} & \textit{The Lord has torn the kingdom of Israel from you today.} (1 Sam 15:28) \\
\end{tabular}

\medskip
\noindent In narratives the suffix conjugation is used instead of the \textit{wayyiqṭol} form when a constituent is fronted (often for emphasis or contrast).
\medskip

\begin{tabular}{>{\raggedleft}p{0.35\linewidth} p{0.55\linewidth}}
	\foreignlanguage{hebrew}{וְהָעִיר שָׂרְפוּ בָאֵשׁ} & \textit{And the city they burnt with fire.} (Jos 6:24) \\
	\foreignlanguage{hebrew}{וְאֶת־בְּנוֹתֵיהֶם נָתְנוּ לִבְנֵיהֶם} & \textit{... and their daughters they gave to their sons.} (Judg 3:6) \\
	\foreignlanguage{hebrew}{וְצֵידָה נָתַן לוֹ} & \textit{... and he gave him provisions.} (1 Sam 22:10) \\
\end{tabular}

\medskip
\noindent The suffix conjugation is used for anterior events (in a past tense context the pluperfect is used in English).
\medskip

\begin{tabular}{>{\raggedleft}p{0.35\linewidth} p{0.55\linewidth}}
	\foreignlanguage{hebrew}{וַיִּקְחוּ אֶת־הַפָּר אֲשֶׁר־נָתַן לָהֶם} & \textit{And they took the bull that he had given them.} (1 Kgs 18:26) \\
	\foreignlanguage{hebrew}{וַיִּשָּׁפֵךְ הַדֶּשֶׁן מִן־הַמִּזְבֵּחַ כַּמּוֹפֵת אֲשֶׁר נָתַן אִישׁ הָאֱלֹהִים בִּדְבַר יְהוָה} & \textit{And the fat ashes poured out from the altar according to the sign that the man of God had given.} (1\,Kgs 13:5) \\
\end{tabular}

\subsubsection{Stative verbs}

Suffix conjugation forms of stative verbs refer to states in the present. This applies also to verbs of mental perception (e.g., \foreignlanguage{hebrew}{ידע} Q. \textit{to know}) and verbs that express emotions (e.g., \foreignlanguage{hebrew}{אהב} Q. \textit{to love}).
\medskip

\begin{tabular}{>{\raggedleft}p{0.35\linewidth} p{0.55\linewidth}}
	\foreignlanguage{hebrew}{עַל־כֵּן גָּדַלְתָּ  אֲדֹנָי יְהוִה} & \textit{Therefore you are great, O Lord GOD} (2\,Sam 7:22) \\
	\foreignlanguage{hebrew}{קָטֹנְתִּי מִכֹּל הַחֲסָדִים} & \textit{I am too insignificant for all your kindness} (Gen 32:11) \\
	\foreignlanguage{hebrew}{לֹא יָדַעְתִּי  מִי עָשָׂה אֶת־הַדָּבָר הַזֶּה} & \textit{I do not know who did this thing} (Gen 21:26) \\
	\foreignlanguage{hebrew}{אָהַבְתִּי אֶת־אֲדֹנִי אֶת־אִשְׁתִּי וְאֶת־בָּנָי} & \textit{I love my lord and my wife and my children} (Exod 21:5) \\
\end{tabular}

\vspace{0.5cm}

\noindent Suffix conjugation forms of stative verbs can also express a state in the past or the entrance into a state in the past.

\vspace{0.5cm}

% \foreignlanguage{hebrew}{כִּי יָגֹרְתִּי \colorbox{Gainsboro}{יָגֹרְתִּי} מִפְּנֵי הָאַף וְהַחֵמָה אֲשֶׁר קָצַף יְהוָה עֲלֵיכֶם
	
	\begin{tabular}{>{\raggedleft}p{0.35\linewidth} p{0.55\linewidth}}
		\foreignlanguage{hebrew}{כִּי יָגֹרְתִּי מִפְּנֵי הָאַף וְהַחֵמָה אֲשֶׁר קָצַף יְהוָה עֲלֵיכֶם}
		& \textit{For I was afraid of the anger and wrath that the LORD had against you.} (Deut 9:19) \\
		\foreignlanguage{hebrew}{וַיִּוָּעַץ אֶת־הַיְלָדִים אֲשֶׁר גָּדְלוּ אִתּוֹ} & \textit{And he exchanged counsel with the young men that had grown up with him.} [= that had become grown-up with him] (1\,Kgs 12:8) \\
	\end{tabular}
	
	\vspace{0.5cm}
	
	\noindent The suffix conjugation has other functions that do not need to be introduced here.
	
	\subsection{Subject Expression and Agreement}
	As explained above, the suffix conjugation forms include an expression of the subject in the suffixes marking the forms for person, gender and number. An overt subject is grammatically not necessary but frequently present. If an overt subject is present, it agrees with the verbal form in gender and number. In Gen 13:12 the verb and the subject are in the singular; in Judg 3:5 they are both in the plural.
	
	\begin{longtable}{>{\raggedleft}p{0.35\linewidth} p{0.55\linewidth}}
		\foreignlanguage{hebrew}{אַבְרָם יָשַׁב בְּאֶֽרֶץ־כְּנָעַן וְלוֹט יָשַׁב בְּעָרֵי הַכִּכָּר} & \textit{Abram settled in the land of Canaan while Lot settled in the cities in the vicinity [of the Jordan]} (Gen 13:12) \\
		\foreignlanguage{hebrew}{וּבְנֵי יִשְׂרָאֵל יָשְׁבוּ בְּקֶרֶב הַכְּנַעֲנִי} & \textit{So the Isrealites settled among the Canaanites} (Judg 3:5) \\
	\end{longtable}
	
	While agreement is the norm, lack of agreement does occur, especially when the verb is at the beginning of the clause and followed by a compound subject consisting of more than one entity.
	
	\vspace{0.5cm}
	
	\begin{tabular}{>{\raggedleft}p{0.35\linewidth} p{0.55\linewidth}}
		\foreignlanguage{hebrew}{וַיָּבֹא מֹשֶׁה וְאַהֲרֹן אֶל־פַּרְעֹה} & \textit{Moses and Aaron came to Pharao} (Exod 7:10) \\
	\end{tabular}
	
	\vspace{0.5cm}
	
	With 1st and 2nd person verbal forms only independent personal pronouns may be used as overt subjects. They agree with the verbal form in person, number and gender.
	
	\vspace{0.5cm}
	
	\begin{tabular}{>{\raggedleft}p{0.35\linewidth} p{0.55\linewidth}}
		\foreignlanguage{hebrew}{אֲנִי שָׁמָ֑עְתִּי} & \textit{I have heard [it]} (Ezek 35:13) \\
	\end{tabular}
	
	\vspace{0.5cm}
	
	The rules for agreement and the expression for the subject apply to the other finite verbal forms, as well, i.e., the prefix conjugation, the volitive forms jussive, cohortative and imperative, and the \textit{wayyiqṭol} and \textit{wɔqaṭaltí} forms.
	
	Other terms for \textit{agreement} and \textit{lack of agreement} are \textit{congruency} and \textit{incongruency}.
	
	
	\section{The Negative \foreignlanguage{hebrew}{לֹא}}
	The most common negative in Biblical Hebrew is \foreignlanguage{hebrew}{לֹא} \textit{lō(ʾ)} \textit{not} (with silent \foreignlanguage{hebrew}{א}). It is used with verbal forms (except infinitives) as negation of the entire clause. The negative \foreignlanguage{hebrew}{לֹא} is usually placed immediately before the verbal form, e.g., \foreignlanguage{hebrew}{לֹא צָחַקְתִּי} \textit{I have not laughed} (Gen 18:15).
	
	The negative \foreignlanguage{hebrew}{לֹא} is also used as the negative answer \textit{no} to a yes-no question, request or statement (e.g., Gen 23:11; 1\,Sam 1:15).
	
	It can be used to negate a single word as in \foreignlanguage{hebrew}{דֶּרֶךְ לֹא־טוֹב} \textit{a way not good} (Ps 36:5), \foreignlanguage{hebrew}{בֵּן לֹא חָכָם} \textit{an unwise son} (Hos 13:13), \foreignlanguage{hebrew}{בְלֹא־אֵל} \textit{with a not-God} (Deut 32:21).
	
	
	\section{Exercises}
	
	\subsection{Translation of Verbal Forms}
	Translate the following verbal forms. Identify the gender (masc., fem., comm.) and number (sg., pl.) the forms of which the English translation is ambiguous (i.e., \textit{you}, \textit{they}). Mark the stressed syllable if stress is not on the last syllable.
	
	\hspace{0.5cm}
	
	\selectlanguage{hebrew}
	
	\noindent
	1~~\foreignlanguage{hebrew}{אָמְרוּ}  \hspace{0.3cm}
	2~~\foreignlanguage{hebrew}{הָלַכְתִּי}  \hspace{0.3cm}
	3~~\foreignlanguage{hebrew}{לָקַח}  \hspace{0.3cm}
	4~~\foreignlanguage{hebrew}{לָקְחָה}  \hspace{0.3cm}
	5~~\foreignlanguage{hebrew}{שָׁלַחְתִּי}  \hspace{0.3cm}
	6~~\foreignlanguage{hebrew}{יָדַעְתָּ}  \hspace{0.3cm}
	7~~\foreignlanguage{hebrew}{יָשְׁבוּ}  \hspace{0.3cm}
	8~~\foreignlanguage{hebrew}{הָלַכְנוּ}  \hspace{0.3cm}
	9~~\foreignlanguage{hebrew}{שָׁמַעַתְּ}  \hspace{0.3cm}
	10~~\foreignlanguage{hebrew}{נָפַל}  \hspace{0.3cm}
	11~~\foreignlanguage{hebrew}{שְׁמַרְתֶּם}  \hspace{0.3cm}
	12~~\foreignlanguage{hebrew}{שָׁאַלְתְּ}  \hspace{0.3cm}
	13~~\foreignlanguage{hebrew}{יָלַדְתְּ}  \hspace{0.3cm}
	14~~\foreignlanguage{hebrew}{אֲכַלְתֶּם}  \hspace{0.3cm}
	15~~\foreignlanguage{hebrew}{מָלַכְתָּ}  \hspace{0.3cm}
	16~~\foreignlanguage{hebrew}{יְדַעְתֶּן}  \hspace{0.3cm}
	17~~\foreignlanguage{hebrew}{אָמְרָה}  \hspace{0.3cm}
	18~~\foreignlanguage{hebrew}{נָתְנוּ}  \hspace{0.3cm}
	19~~\foreignlanguage{hebrew}{נָתַתָּה}  \hspace{0.3cm}
	20~~\foreignlanguage{hebrew}{נָתַתָּ}  \hspace{0.3cm}
	
	\selectlanguage{english}
	
	\subsection{Translation of Sentences}
	Translate the following sentences from the Hebrew Bible. Names of persons and geographical names in these sentences: \foreignlanguage{hebrew}{אָסָא}, \foreignlanguage{hebrew}{בֶּן־הֲדַד}, \foreignlanguage{hebrew}{מוֹאָב}, \foreignlanguage{hebrew}{נַעֲמָן}, \foreignlanguage{hebrew}{עַמּוֹן}, \foreignlanguage{hebrew}{שָׁאוּל}.
	
	\vspace{0.5cm}
	
	\selectlanguage{hebrew}
	
	\noindent
	1~~\foreignlanguage{hebrew}{לֶחֶם לֹא אֲכַלְתֶּם}  \hspace{0.3cm}
	2~~\foreignlanguage{hebrew}{הִנֵּה יָרְדָה אֵשׁ מִן־הַשָּׁמַיִם}  \hspace{0.3cm}
	3~~\foreignlanguage{hebrew}{וּפְלִשְׁתִּים}\LTRfootnote{\space \foreignlanguage{hebrew}{פְּלִשְׁתִּים} \textit{the Philistines} (usually without the article; sg.\ \foreignlanguage{hebrew}{פְּלִשְׁתִּי})} \foreignlanguage{hebrew}{לָקְחוּ אֵת אֲרוֹן הָאֱלֹהִים}  \hspace{0.3cm}
	4~~\foreignlanguage{hebrew}{כִּי־אַתָּה}\LTRfootnote{\space \foreignlanguage{hebrew}{אַתָּה} \textit{you} (independent personal pronoun, 2 m.\ sg.)} \foreignlanguage{hebrew}{יָדַעְתָּ לְבַדְּךָ}\LTRfootnote{\space \foreignlanguage{hebrew}{לְבַדְּךָ} \textit{you alone}} \foreignlanguage{hebrew}{אֶת־לְבַב כָּל־בְּנֵי הָאָדָם} \hspace{0.3cm}
	5~~\foreignlanguage{hebrew}{שָׁמַעְתִּי אֶת־קוֹל דִּבְרֵי הָעָם הַזֶּה אֲשֶׁר דִּבְּרוּ אֵלֶ֫יךָ}\LTRfootnote{\space \foreignlanguage{hebrew}{אֲשֶׁר דִּבְּרוּ אֵלֶיךָ} \textit{which they have spoken to you}}  \hspace{0.3cm}
	6~~\foreignlanguage{hebrew}{הִנֵּה שָׁלַחְתִּי אֵלֶ֫יךָ}\LTRfootnote{\space \foreignlanguage{hebrew}{אֵלֶיךָ} \textit{to you} (preposition \foreignlanguage{hebrew}{אֶל} with enclitic pronoun 2 m.\ sg.)} \foreignlanguage{hebrew}{אֶת־נַעֲמָן עַבְדִּי}\LTRfootnote{\space \foreignlanguage{hebrew}{עַבְדִּי} \textit{my servant} (\foreignlanguage{hebrew}{עֶבֶד} with enclitic personal pronoun 2 m.\ sg.)} \hspace{0.3cm}
	7~~\foreignlanguage{hebrew}{יָדַעְתִּי כִּי־נָתַן יְהוָה}\LTRfootnote{\space \foreignlanguage{hebrew}{יהוה} read as \foreignlanguage{hebrew}{אֲדֹנָי} (Engl. \textit{Lord})} \foreignlanguage{hebrew}{לָכֶם}\LTRfootnote{\space \foreignlanguage{hebrew}{לָכֶם} \textit{to you} (preposition \foreignlanguage{hebrew}{לְ} with enclitic personal pronoun 2 m.\ pl.)} \foreignlanguage{hebrew}{אֶת־הָאָרֶץ}  \hspace{0.3cm}
	8~~\foreignlanguage{hebrew}{מָה}\LTRfootnote{\space \foreignlanguage{hebrew}{מָה} \textit{what?} (interrogative pronoun)} \foreignlanguage{hebrew}{אָמְרוּ הָאֲנָשִׁים הָאֵלֶּה}  \hspace{0.3cm}
	9~~\foreignlanguage{hebrew}{וַיַּרְא אֱלֹהִים אֶת־בְּנֵי יִשְׂרָאֵל}  \hspace{0.3cm}
	10~~\foreignlanguage{hebrew}{וַיִּשְׁמַע אֱלֹהִים אֶת־קוֹל הַנַּעַר}  \hspace{0.3cm}
	11~~\foreignlanguage{hebrew}{וַיִּשְׁמַע שָׁאוּל וְכָל־יִשְׂרָאֵל אֶת־דִּבְרֵי הַפְּלִשְׁתִּי}\LTRfootnote{\space \foreignlanguage{hebrew}{פְּלִשְׁתִּי} \textit{Philistine} (sg.)} \foreignlanguage{hebrew}{הָאֵלֶּה} \hspace{0.3cm}
	12~~\foreignlanguage{hebrew}{וַיֹּאמֶר לוֹ}\LTRfootnote{\space \foreignlanguage{hebrew}{לוֹ} \textit{to him} (preposition \foreignlanguage{hebrew}{לְ} with enclitic pronoun 3 m.\ sg.)} \foreignlanguage{hebrew}{כֹּה אָמַר יִפְתָּח לֹא־לָקַח יִשְׂרָאֵל אֶת־אֶרֶץ מוֹאָב וְאֶת־אֶרֶץ בְּנֵי עַמּוֹן}  \hspace{0.3cm}
	13~~\foreignlanguage{hebrew}{וַיִּשְׁמַע בֶּן־הֲדַד אֶל־הַמֶּלֶךְ אָסָא}  \hspace{0.3cm}
	14~~\foreignlanguage{hebrew}{כֹּה־אָמַר יְהוָה}\LTRfootnote{\space \foreignlanguage{hebrew}{יהוה} read as \foreignlanguage{hebrew}{אֲדֹנָי} (Engl. \textit{Lord})} \foreignlanguage{hebrew}{אֱלֹהֵי יִשְׂרָאֵל} \hspace{0.3cm}
	
	\selectlanguage{english}
	
	\chapter{Chapter 6}
	
	\renewcommand\arraystretch{1.4}
	
	\section{Vocabulary}
	
	\subsection{Verbs}
	
	\begin{center}
		
		% For the centering of the separation between the two columns see the documentation of the array package, page 2 
		
		\begin{longtable}{>{\raggedleft}p{0.175\linewidth} p{0.75\linewidth}}
			\foreignlanguage{hebrew}{זכר} & Q.\ \textit{to remember} \\
			\foreignlanguage{hebrew}{חזק} & Q.\ \textit{to be firm, strong} (stative verb, PC \foreignlanguage{hebrew}{יֶחֱזַק} but SC \foreignlanguage{hebrew}{חָזַק}) \\
			\foreignlanguage{hebrew}{כרת} & Q.\ \textit{to cut} (frequently with the noun \foreignlanguage{hebrew}{בְּרִית} \textit{covenant} as direct object) \\
			\foreignlanguage{hebrew}{כתב} & Q.\ \textit{to write} \\
			\foreignlanguage{hebrew}{עבד} & Q.\ \textit{to serve, work} \\
			\foreignlanguage{hebrew}{פקד} & Q.\ \textit{to visit, attend to, muster, appoint} \\
			\foreignlanguage{hebrew}{קרב} & Q.\ \textit{to draw near, approach} (stative verb; SC \foreignlanguage{hebrew}{קָרַב} and \foreignlanguage{hebrew}{קָרֵב}; PC \foreignlanguage{hebrew}{יִקְרַב}) \\
		\end{longtable}
	\end{center}
	
	\subsection{Nouns}
	
	\begin{center}
		\begin{longtable}{>{\raggedleft}p{0.175\linewidth} p{0.75\linewidth}}
			\foreignlanguage{hebrew}{אֹהֶל} & \textit{tent} (pl.\ abs.\ st.\ \foreignlanguage{hebrew}{אֹהָלִים}, pl.\ cs.\ st.\ \foreignlanguage{hebrew}{אָהֳלֵי} \textit{ʾɔhɔ̆lē}) \\
			\foreignlanguage{hebrew}{אוֺת} & \textit{sign} (fem., pl.\ \foreignlanguage{hebrew}{אֹתוֺת}) \\
			\foreignlanguage{hebrew}{בְּרִית} & \textit{covenant, agreement, contract} (fem.) \\
			\foreignlanguage{hebrew}{גְּבוּל} & \textit{border, boundary, territory} \\
			\foreignlanguage{hebrew}{יָם} & \textit{sea}, west (pl.\ \foreignlanguage{hebrew}{יַמִּים}) \\
			\foreignlanguage{hebrew}{כָּבֵד} & \textit{heavy; oppressing; weighty} (adj.) \\ % HALOT
			\foreignlanguage{hebrew}{כֹּהֵן} & \textit{priest} \\
			\foreignlanguage{hebrew}{לֵב} & \textit{heart} (gem. noun; pl.\ \foreignlanguage{hebrew}{לִבּוֺת}) \\
			\foreignlanguage{hebrew}{מַלְאָךְ} & \textit{messenger, angel} \\
			\foreignlanguage{hebrew}{מִצְוָה} & \textit{commandment} \\
			\foreignlanguage{hebrew}{נָבִיא} & \textit{prophet} \\
			\foreignlanguage{hebrew}{פָּנִים} & \textit{face, surface} (only pl.) \\
			\foreignlanguage{hebrew}{קָהָל} & \textit{assembly} \\
			\foreignlanguage{hebrew}{רוּחַ} & \textit{spirit, wind} (fem.\ or masc.) \\
		\end{longtable}
	\end{center}
	
	\subsection{Other Parts of Speech}
	
	\begin{center}
		\begin{tabular}{>{\raggedleft}p{0.175\linewidth} p{0.75\linewidth}}
			\foreignlanguage{hebrew}{אַל} & \textit{not} (with the jussive or cohortative) \\
		\end{tabular}
	\end{center}
	
	
	\section{Prefix Conjugation Qal}
	
	\subsection{The forms of the Prefix Conjugation}
	
	The prefix conjugation is characterized by prefixes in all forms of the paradigm. In some forms suffixes are attached to the form as well.
	
	In the prefix conjugation Biblical Hebrew distinguishes two forms with different thematic vowels, i.e., the vowels between the second and third root consonants. The form \foreignlanguage{hebrew}{יִקְטֹל} \textit{yiqṭōl} with the thematic vowel /ō/ is used for fientive verbs while the form \foreignlanguage{hebrew}{יִכְבַּד} \textit{yikbad} with the thematic vowel /a/ is used for stative verbs.
	
	For a proper understanding of the forms of the prefix conjugation forms of verbs with gutturals and some weak verbs it is important to note that the form \foreignlanguage{hebrew}{יִקְטֹל} \textit{yiqṭol} is the result of the development \textit{*yaqṭulu > yiqṭol} (with loss of the final short vowel as usual in BH).\footnote{\space The original vowels are preserved in Classical Arabic, e.g., \textit{yafʿulu} \textit{he will do}. The forms of the prefix conjugation forms of verbs I\,ʾ, II\,\textit{w/y} and II\,gem.\ can only be explained with the original forms with the vowel /a/ in the prefix.}  The change of the prefix vowel /a/ > /i/ is due to the law of attenuation (/a/ > /i/ in unstressed closed syllables). The change of /u/ > /ō/ in stressed syllables is regular.
	
	The prefix vowel /i/ and thematic vowel /a/ of the forms \foreignlanguage{hebrew}{יִכְבַּד} \textit{yikbad}, etc., of the stative verb are the original vowels; they were not subject to change.
	
	\bigskip
	
	\renewcommand\arraystretch{1.4}
	
	\begin{center}
		\begin{tabular}{|ll|c|c|r|r|}
			\hline
			\multicolumn{2}{|c|}{} & pref. & suff. & fientive & stative \\
			\hline 
			sg. & 3 m. & \textit{y-} & & \foreignlanguage{hebrew}{יִכְתֹּב} &  \foreignlanguage{hebrew}{יִכְבַּד} \\
			& 3 f. & \textit{t-} & & \foreignlanguage{hebrew}{תִִּכְתֹּב} &  \foreignlanguage{hebrew}{תִּכְבַּד} \\
			& 2 m. & \textit{t-} & & \foreignlanguage{hebrew}{תִִּכְתֹּב} & \foreignlanguage{hebrew}{תִּכְבַּד} \\
			& 2 f. & \textit{t-} & \textit{-ī} & \foreignlanguage{hebrew}{תִִּכְתְּבִי} & \foreignlanguage{hebrew}{תִּכְבְּדִי} \\
			& 1 c. & \textit{ʾ-} & & \foreignlanguage{hebrew}{אֶכְתֹּב} & \foreignlanguage{hebrew}{אֶכְבַּד} \\
			pl & 3 m. & \textit{y-} & \textit{-ū} & \foreignlanguage{hebrew}{יִכְתְּבוּ} &  \foreignlanguage{hebrew}{יִכְבְּדוּ} \\
			& 3 f. & \textit{t-} & \textit{-nā} & \foreignlanguage{hebrew}{תִּכְתֹּ֫בְנָה} &  \foreignlanguage{hebrew}{תִּכְבַּ֫דְנָה} \\
			& 2 m. & \textit{t-} & \textit{-ū} & \foreignlanguage{hebrew}{תִּכְתְּבוּ} & \foreignlanguage{hebrew}{תִּכְבְּדוּ} \\
			& 2 f. & \textit{t-} & \textit{-nā} & \foreignlanguage{hebrew}{תִּכְתֹּ֫בְנָה} & \foreignlanguage{hebrew}{תִּכְבַּ֫דְנָה} \\
			& 1 c. & \textit{n-} & & \foreignlanguage{hebrew}{נִכְתֹּב} & \foreignlanguage{hebrew}{נִכְבַּד} \\
			\hline
		\end{tabular}
	\end{center}
	
	\vspace{0.5cm}
	
	\noindent \textbf{Notes}
	\nopagebreak
	
	\noindent The vowel /æ/ in the prefix of the 1 c.\ sg.\ form needs to be noted. The consonant /ʾ/ often triggers the change of a vowel to /æ/.
	
	The forms with stressed vocalic suffix (2\ f.\ sg.\ and 3/2\ m.\ pl.) have distinct pausal forms. The stress is on the penultimate syllable and the original thematic vowel is preserved instead of being reduced as in the context forms: e.g., \foreignlanguage{hebrew}{יִכְתֹּ֑בוּ}, \foreignlanguage{hebrew}{יִכְבָּ֑דוּ}.
	
	In forms with /a/ as thematic vowel the \textit{pataḥ} is lengthened to \textit{qameṣ} in pausal forms, e.g., \foreignlanguage{hebrew}{יִכְבָּ֑ד}.
	
	The forms 3/2m.\ pl.\ are quite often extended by an additional \foreignlanguage{hebrew}{ן} called paragogic Nun, e.g., \foreignlanguage{hebrew}{תִּשְׁמְרוּן} \textit{you shall keep} (Deut 6:17),  \foreignlanguage{hebrew}{יִשְׁמְעוּן} \textit{they will hear} (Deut 4:6). Most of the 305 examples in the Hebrew Bible are found in Deut, Isa, Job and Ps 104 (JM §\,44e). There is no difference in meaning between the normal forms and the forms with paragogic Nun. In pausal forms the thematic vowel is preserved despite the position of the stress, e.g., \foreignlanguage{hebrew}{תִּכְרֹת֑וּן} \textit{you shall cut down} (Exod 34:13), \foreignlanguage{hebrew}{תִּדְבָּק֑וּן} \textit{you shall cling} (Deut 13:5). Sometimes pausal forms with paragogic Nun are found in context. In five cases 2 f.\ sg.\ forms also have a paragogic Nun, e.g., \foreignlanguage{hebrew}{תִּדְבָּקִין} \textit{you shall cling} (Ruth 2:8).
	
	The possible confusion of identical forms (3 f.\ sg.\ and 2 m.\ sg.\ forms and 3/2 fem.\ pl.) is hardly ever an issue in Biblical Hebrew texts.
	
\subsection{Historical Background}
In the earliest phase of Northwest Semitic the prefix conjugation had four distinct forms. Two forms had a short final vowel: \textit{yaqṭulu} and \textit{yaqṭula}. The form \textit{yaqṭulu} functioned as an indicative form while the form \textit{yaqṭula} was used as a subjunctive. Vestiges of the form \textit{yaqṭula} can still be found in the cohortative and the long form of the imperative.
	
The third form was a short form without a final vowel \textit{yaqṭul} which functioned either as a perfective past tense (or preterite) or as a volitive form. This form is preserved in the \textit{wayyiqṭol} form and the jussive in Biblical Hebrew.
	
Because of the general loss of short final vowels the formal difference between the forms \textit{yaqṭulu} and \textit{yaqṭul} is lost in most forms. It is reflected only in some volitive forms and \textit{wayyiqṭol} forms of certain classes of weak verbs and in some Hiphil forms.
	
A fourth form was the so-called energic mode \textit{yaqṭulan(na)} which is preserved only in some forms of the prefix conjugation with enclitic personal pronouns.
	
\subsection{Verbs with Gutturals in the Prefix Conjugation}
	
	\begin{center}
		\begin{longtable}{|ll|r|r|r|r|}
		\hline
		\multicolumn{2}{|c|}{} & \multicolumn{2}{c|}{I gutt.} & \multicolumn{1}{c|}{II gutt.} & \multicolumn{1}{c|}{III gutt.} \\
		\cline{3-4}
		\multicolumn{2}{|c|}{} & fientive & stative & & \\
		\hline
		\endhead
		\hline
		\endfoot
%		\hline
			sg. & 3 m. & \foreignlanguage{hebrew}{יַעֲזֹב} &  \foreignlanguage{hebrew}{יֶחֱזַק} & \foreignlanguage{hebrew}{יִבְחַר} & \foreignlanguage{hebrew}{יִשְׁלַח} \\
			& 3 f. & \foreignlanguage{hebrew}{תַּעֲזֹב} &  \foreignlanguage{hebrew}{תֶּחֱזַק} &  \foreignlanguage{hebrew}{תִּבְחַר} & \foreignlanguage{hebrew}{תִּשְׁלַח} \\
			& 2 m. & \foreignlanguage{hebrew}{תַּעֲזֹב} & \foreignlanguage{hebrew}{תֶּחֱזַק} & \foreignlanguage{hebrew}{תִּבְחַר} & \foreignlanguage{hebrew}{תִּשְׁלַח} \\
			& 2 f. & \foreignlanguage{hebrew}{תַּעַזְבִי} & \foreignlanguage{hebrew}{תֶּחֶזְקִי} & \foreignlanguage{hebrew}{תִּבְחֲרִי} & \foreignlanguage{hebrew}{תִּשְׁלְחִי} \\
			& 1 c. & \foreignlanguage{hebrew}{אֶעֱזֹב} & \foreignlanguage{hebrew}{אֶחֱזַק} & \foreignlanguage{hebrew}{אֶבְחַר} & \foreignlanguage{hebrew}{אֶשְׁלַח} \\
			pl. & 3 m. & \foreignlanguage{hebrew}{יַעַזְבוּ} &  \foreignlanguage{hebrew}{יֶחֶזְקוּ} & \foreignlanguage{hebrew}{יִבְחֲרוּ} & \foreignlanguage{hebrew}{תִּשְׁלְחוּ} \\
			& 3 f. & \foreignlanguage{hebrew}{תַּעֲזֹ֫בְנָה} &  \foreignlanguage{hebrew}{תֶּחֱזַ֫קְנָה} & \foreignlanguage{hebrew}{תִּבְחַ֫רְנָה} & \foreignlanguage{hebrew}{תִּשְׁלַ֫חְנָה} \\
			& 2 m. & \foreignlanguage{hebrew}{תַּעַזְבוּ} & \foreignlanguage{hebrew}{תֶּחֶזְקוּ} & \foreignlanguage{hebrew}{תִּבְחֲרוּ} & \foreignlanguage{hebrew}{תִּשְׁלְחוּ} \\
			& 2 f. & \foreignlanguage{hebrew}{תַּעֲזֹ֫בְנָה} & \foreignlanguage{hebrew}{תֶּחֱזַ֫קְנָה} & \foreignlanguage{hebrew}{תִּבְחַ֫רְנָה} & \foreignlanguage{hebrew}{תִּשְׁלַ֫חְנָה} \\
			& 1 c. & \foreignlanguage{hebrew}{נַעֲזֹב} & \foreignlanguage{hebrew}{נֶחֱזַק} & \foreignlanguage{hebrew}{נִבְחַר} & \foreignlanguage{hebrew}{נִשְׁלַח} \\
			%		\hline
		\end{longtable}
	\end{center}
	
	\noindent \textbf{Notes}
	\nopagebreak
	
	\noindent The gutturals  \foreignlanguage{hebrew}{ה},  \foreignlanguage{hebrew}{ח} and  \foreignlanguage{hebrew}{ע} prefer the vowel /a/ in the positions before and after the guttural. There are, however, some I\,gutt.\ verbs with \textit{yaqṭul} prefix conjugation forms that have the prefix vowel /æ/ instead of /a/, e.g., \foreignlanguage{hebrew}{יֶהְדֹּף} \textit{he will push}. When the guttural \foreignlanguage{hebrew}{א} is preserved as a consonant, the vowel /æ/ is preferred, e.g., \foreignlanguage{hebrew}{יֶאְסֹר} \textit{he will bind}.
	
	Many I\,gutt.\ verbs have a \textit{ḥatef} vowel with the first root consonant instead of silent \textit{šwa} (cp.\ \foreignlanguage{hebrew}{יַעֲזֹב} with \foreignlanguage{hebrew}{יִכְתֹּב}) while other I\,gutt.\ verbs tolerate the silent \textit{šwa} (e.g., \foreignlanguage{hebrew}{יַחְמֹד} \textit{he shall desire}).
	
	In context forms with vocalic suffix (2 f.\ sg.\ and 2/3 m.\ pl.) of I\,gutt.\ verbs the two consecutive vocal \textit{šwa} are changed to a closed syllable with the vowel of the preceding vowel and medial \textit{šwa} which does not trigger a \textit{dageš lene} in the following consonant, e.g., \textit{*taʿăzǝḇū > taʿazḇū} \foreignlanguage{hebrew}{תַּעַזְבוּ} and \textit{*yæḥæ̆zǝqū > yæḥæzqū} \foreignlanguage{hebrew}{יֶחֶזְקוּ}. In forms like these Biblical Hebrew tolerates short vowels in open syllables.
	
	Fientive verbs I\,gutt.\ have the vowel pattern \textit{a--u} with preservation of the original vowel of the prefixes while stative verbs I\,gutt.\ have the vowel pattern \textit{i--a} (with change of the /i/ to /æ/ to bring the prefix vowel closer to the vowel /a/ that is preferred by gutturals).
	
	Verbs II\,gutt.\ usually have the vowel pattern \textit{i--a} irrespective of their semantic classification as fientive or stative. The stem vowel /ō/ can be found with a few verbs, e.g., \foreignlanguage{hebrew}{יֶאֱחֹז} \textit{he shall grasp}. Verbs III\,gutt.\ almost always show the vowel pattern \textit{i--a}.
	
	In pausal forms with vocalic suffix the original thematic vowel is preserved and the stress is on the penultimate syllable, e.g., \foreignlanguage{hebrew}{יַעֲזֹ֑בוּ}, \foreignlanguage{hebrew}{תִּשְׁלָ֑חוּ}.
	
	In forms with /a/ as thematic vowel the \textit{pataḥ} is lengthened to \textit{qameṣ} in pausal forms, e.g., \foreignlanguage{hebrew}{יִבְחָ֑ר}.
	
	
	\subsection{The Use of the Prefix Conjugation}
	
	\noindent The prefix conjugation has three main uses:
	
	\begin{enumerate}[noitemsep]
		\item as a future tense and -- almost exclusively occurring in questions and subordinate clauses -- present tense (1\,Sam 9:16; Gen 24:31)
		\item as a form expressing imperfective aspect in the past for durative or iterative (repeated) past-time events (1\,Kgs 17:6)
		\item as a modal form expressing obligation, capability etc.\ (expressed in English with auxiliary verbs like \textit{may, would, shall, can, must, want, could, should}, etc.), e.g., Exod 20:9; Deut 23:7. With the negative \foreignlanguage{hebrew}{לֹא} the prefix conjugation is used for general or strict prohibitions (e.g., Exod 20:15). This use is called vetitive and forms a subgroup of the modal use of the prefix conjugation.
	\end{enumerate}
	
	
	\begin{tabular}{>{\raggedleft}p{0.35\linewidth} p{0.55\linewidth}}
		\foreignlanguage{hebrew}{כָּעֵת מָחָר אֶשְׁלַח אֵלֶיךָ אִישׁ מֵאֶרֶץ בִּנְיָמִן} & \textit{Tomorrow about this time I will send to you a man from the land of Benjamin} (1 Sam 9:16) \\
		\foreignlanguage{hebrew}{לָמָּה תַעֲמֹד בַּחוּץ} & \textit{Why are you standing outside?} (Gen 24:31) \\
		\foreignlanguage{hebrew}{וּמִן־הַנַּחַל יִשְׁתֶּה} & \textit{... and from the brook he used to drink.} (1 Kgs 17:6b) \\
		\foreignlanguage{hebrew}{שֵׁשֶׁת יָמִים תַּעֲבֹד} & \textit{Six days you may work.} (Exod 20:9) \\
		\foreignlanguage{hebrew}{לֹא תִּגְנֹב} & \textit{You shall not steal.} (Exod 20:15) \\
	\end{tabular}
	
	\subsection{The \textit{wayyiqṭol} Form}
	The \textit{wayyiqṭol} form is a past tense form that is based on the short form of the prefix conjugation \textit{yaqṭul} in its function as perfective past tense. It is formed by a special form of the conjunction \textit{waw} with \textit{pataḥ} und \textit{dageš forte} in the consonant of the prefix syllable. As the \foreignlanguage{hebrew}{א} of the 1 c.\ sg.\ prefix does not take \textit{dageš forte} the vowel is changed to \textit{qameṣ}.
	
	This special form of the conjunction is called \textit{waw consecutive} (sometimes called \textit{waw inversive} [Hebrew \foreignlanguage{hebrew}{וָו הַהִפּוּךְ}] although this term is rather misleading and historically incorrect). In the strong verb, the actual form of the \textit{wayyiqṭol} form is identical with the normal prefix conjugation. This applies to pausal forms as well. Only with weak verbs and in derived conjugations deviating forms occur.
	
	
	\begin{center}
		\begin{tabular}{|ll|r|r|}
			\hline
			\multicolumn{2}{|c|}{} & \multicolumn{2}{c|}{\textit{wayyiqṭol}} \\
			\cline{3-4}
			\multicolumn{2}{|c|}{} & fientive & stative \\
			\hline 
			sg. & 3 m. & \foreignlanguage{hebrew}{וַיִּכְתֹּב} &  \foreignlanguage{hebrew}{וַיִּכְבַּד} \\
			& 3 f. & \foreignlanguage{hebrew}{וַתִּכְתֹּב} &  \foreignlanguage{hebrew}{וַתִּכְבַּד} \\
			& 2 m. & \foreignlanguage{hebrew}{וַתִּכְתֹּב} & \foreignlanguage{hebrew}{וַתִּכְבַּד} \\
			& 2 f. & \foreignlanguage{hebrew}{וַתִּכְתְּבִי} & \foreignlanguage{hebrew}{וַתִּכְבְדִי} \\
			& 1 c. & \foreignlanguage{hebrew}{וָאֶכְתֹּב} & \foreignlanguage{hebrew}{וָאֶכְבַּד} \\
			pl. & 3 m. & \foreignlanguage{hebrew}{וַיִּכְתְּבוּ} &  \foreignlanguage{hebrew}{וַיִּכְבְּדוּ} \\
			& 3 f. & \foreignlanguage{hebrew}{וַתִּכְתֹּ֫בְנָה} &  \foreignlanguage{hebrew}{וַתִּכְבַּ֫דְנָה} \\
			& 2 m. & \foreignlanguage{hebrew}{וַתִּכְתְּבוּ} & \foreignlanguage{hebrew}{וַתִּכְבְּדוּ} \\
			& 2 f. & \foreignlanguage{hebrew}{וַתִּכְתֹּ֫בְנָה} & \foreignlanguage{hebrew}{וַתִּכְבַּ֫דְנָה} \\
			& 1 c. & \foreignlanguage{hebrew}{וַנִּכְתֹּב} & \foreignlanguage{hebrew}{וַנִּכְבַּד} \\
			\hline
		\end{tabular}
	\end{center}
	
	
	The \textit{wayyiqṭol} form is used as a past tense form, e.g., Exod 24:4. Chains of \textit{wayyiqṭol} forms form the backbone of narratives. \textit{wayyiqṭol} forms need to be in clause initial position.\footnote{\space Whenever another element needs to be placed in clause initial position (e.g., the negative \foreignlanguage{hebrew}{לֹא}), different verbal forms need to be used.} With stative verbs the \textit{wayyiqṭol} form usually has ingressive meaning, i.e., \textit{began to $\ldots$}, e.g., Gen 41:56.
	
	\medskip
	\begin{tabular}{>{\raggedleft}p{0.35\linewidth} p{0.55\linewidth}}
		\foreignlanguage{hebrew}{וַיִּכְתֹּב מֹשֶׁה אֵת כָּל־דִּבְרֵי יְהוָה} & \textit{And Moses wrote down all the words of the Lord.} (Exod 24:4) \\
		\foreignlanguage{hebrew}{וַיֶּחֱזַק הָֽרָעָב בְּאֶרֶץ מִצְרָיִם} & \textit{And the famine became severe in the land of Egypt.} (Gen 41:56) \\
	\end{tabular}
	
	
	
	\section{Volitive Forms}
	
	\subsection{Introduction}
	Volitive forms are verbal forms that express an intention or desire. Biblical Hebrew has three volitive forms, the cohortative (1st person), the imperative (2nd person) and the jussive (2nd and 3rd person). The most frequent volitive form is the imperative which is introduced in Chapter 7.
	
	\subsection{The Jussive}
	The jussive is a volitive form that can be used in the 3rd or 2nd person. Historically, the jussive is based on the short form of the prefix conjugation \textit{yaqṭul}. The jussive forms of strong verbs are identical to the prefix conjugation forms discussed above. Only weak verbs of the types III\,\textit{y} and II\,\textit{w/y} have distinct jussive forms in the Qal.
	
	The jussive is used for commands, instructions, requests and petitions in the 3rd person and in the 2nd person. It is used both in positive and negative clauses. With negative commands, the negative \foreignlanguage{hebrew}{אַל} is used. Positive clauses with the 2nd person jussive forms as predicate, however, are very infrequent.
	
	\medskip
	\begin{tabular}{>{\raggedleft}p{0.35\linewidth} p{0.55\linewidth}}
		\foreignlanguage{hebrew}{יִשְׁפֹּט יְהוָה בֵּינִי וּבֵינֶ֫ךָ} & \textit{Let the Lord judge between me and you.} (Gen 16:5) \\
		\foreignlanguage{hebrew}{יָבֹא־נָא אֵלַי} & \textit{Let him come to me.} (2 Kgs 5:8) \\
		\foreignlanguage{hebrew}{אַל־תִּשְׁלַח יָדְךָ אֶל־הַנַּעַר} & \textit{Do not lay your hand on the boy.} (Gen 22:12) \\
	\end{tabular}
	
	\subsection{The Cohortative}
	
	\noindent The cohortative -- the first person volitive form -- is used for exhortations, requests, wishes, or declarations of intent. It is marked by the ending \textit{-ā} that is attached to the 1 c.\ sg./pl.\ prefix conjugation form. The thematic vowel is reduced in context forms as the ending is stressed.
	
	\begin{center}
		\begin{tabular}{|ll|r|}
			\hline
			\multicolumn{2}{|c|}{} & cohortative \\
			\hline
			sg. & 1 c. & \foreignlanguage{hebrew}{אֶכְתְּבָה} \\
			\hline
			pl. & 1 c. & \foreignlanguage{hebrew}{נִכְתְּבָה} \\
			\hline
		\end{tabular}
	\end{center}
	
	\noindent In pausal forms the thematic vowel is preserved and stressed: \foreignlanguage{hebrew}{אֶכְתֹּ֫בָה}, \foreignlanguage{hebrew}{נִכְתֹּ֫בָה}
	
	\medskip
	\begin{tabular}{>{\raggedleft}p{0.35\linewidth} p{0.55\linewidth}}
		\foreignlanguage{hebrew}{וְאֶקְבְּרָה אֶת־אָבִי} & \textit{and let me bury my father.} (Gen 50:5) \\
		\foreignlanguage{hebrew}{הָ֫בָה נִלְבְּנָה לְבֵנִים} & \textit{Come now, let us make bricks.} (Gen 11:3) \\
	\end{tabular}
	
	\section{The Marker of Subordination \foreignlanguage{hebrew}{אֲשֶׁר}}
	The word \foreignlanguage{hebrew}{אֲשֶׁר} serves as a marker of subordination introducing dependent clauses among which relative clauses are the most frequent. Because of this, \foreignlanguage{hebrew}{אֲשֶׁר} is sometimes called relative particle.
	
	\subsection{Relative Clauses}
	Relative clauses can either function as dependent attributive clauses modifying a noun or be substantivized replacing a noun. Unlike the relative pronouns \textit{which} and \textit{that} in English or \foreignlanguage{greek}{ ὅς, ἥ, ὅ } etc. in Greek, \foreignlanguage{hebrew}{אֲשֶׁר} is not a relative pronoun in the sense that it is a constituent of the relative clause. It rather functions as a marker of subordination. As a consequence, one often finds in the relative clause a resumptive pronoun or adverb which refers back to the antecedent. If the resumptive pronoun would be the direct object of the predicate of the relative clause, it is usually dropped because it can be easily retrieved from the immediate context.\footnote{\space An example of an resumptive pronominal element as direct object is found in 2\,Kgs 16:3 \foreignlanguage{hebrew}{כְּתֹעֲבוֹת הַגּוֹיִם אֲשֶׁר הוֹרִישׁ יְהוָה אֹתָם מִפְּנֵי בְּנֵי יִשְׂרָאֵל} \textit{according to the abominations of the peoples that the Lord had driven out before the Israelites}.} Other resumptive elements (e.g., prepositional phrases) may be dropped, too.
	
	If the relative clause serves as attribute to a noun, it is possible to first render \foreignlanguage{hebrew}{אֲשֶׁר} as \textit{about which can be said} and translate the relative clause as if it were an independent clause. In the next step, the relative clause can be integrated smoothly into the main clause.
	
	\vspace{0.5cm}
	
	\begin{tabular}{>{\raggedleft}p{0.35\linewidth} p{0.55\linewidth}}
		\foreignlanguage{hebrew}{מִן־הַמָּקוֹם אֲשֶׁר־אַתָּה שָׁם} & \textit{from the place about which can be said: you are there} > \textit{from the place where you are} (Gen 13:14) \\
	\end{tabular}
	
	\vspace{0.5cm}
	
	In the overview below it is indicated which element is dropped from the relative clause.
	
	\begin{longtable}{>{\raggedleft}p{0.35\linewidth} p{0.55\linewidth}}
		\foreignlanguage{hebrew}{אַיֵּה הָאֲנָשִׁים אֲשֶׁר־בָּ֫אוּ אֵלֶ֫יךָ הַלָּ֫יְלָה} & \textit{Where are the men who came to you tonight?} (Gen 19:5) \\
		\foreignlanguage{hebrew}{וַיִּקְרָא אַבְרָם שֶׁם־בְּנוֹ אֲשֶׁר־יָלְדָה הָגָר יִשְׁמָעֵאל} & \textit{And Abram called the name of his son whom Hagar bore, Ishmael} (Gen 16:15; direct object) \\
		\foreignlanguage{hebrew}{עַל־הַדָּבָר אֲשֶׁר־דִּבֶּר אֵלָיו נָבוֹת הַיִּזְרְעֵאלִי} & \textit{\dots \space because of the word that Naboth, the Jezreelite, had said to him} (1\,Kgs 21:4; direct object) \\
		\foreignlanguage{hebrew}{וַיַּעַל מֵעָלָיו אֱלֹהִים בַּמָּקוֹם אֲשֶׁר־דִּבֶּר אִתּוֺ} & \textit{And God went up from him in the place where he had spoken with him} (Gen 35:13; local adverbial phrase, cf.\ Gen 35:15) \\
	\end{longtable}
	
	\noindent The following two clauses contain substantivized clauses with \foreignlanguage{hebrew}{אֲשֶׁר} as subject (Gen 38:10) and direct object (Exod 18:1; Exod 11:7 where \foreignlanguage{hebrew}{אֲשֶׁר} is equivalent to \foreignlanguage{hebrew}{כִּי}), respectively.
	
	\vspace{0.5cm}
	
	\begin{tabular}{>{\raggedleft}p{0.35\linewidth} p{0.55\linewidth}}
		\foreignlanguage{hebrew}{וַיֵּ֫רַע בְּעֵינֵי יְהוָה אֲשֶׁר עָשָׂה} & \textit{And what he did was evil in the eyes of the Lord} (Gen 38:10) \\
		\foreignlanguage{hebrew}{וַיִּשְׁמַע יִתְרוֹ כֹהֵן מִדְיָן חֹתֵן מֹשֶׁה אֵת כָּל־אֲשֶׁר עָשָׂה אֱלֹהִים לְמֹשֶׁה וּלְיִשְׂרָאֵל עַמּוֹ} & \textit{And Jethro the priest of Midian, Moses' father-in-law, heard everything that God had done to Moses and his people Israel \dots} (Exod 18:1) \\
		\foreignlanguage{hebrew}{לְמַעַן תֵּדְעוּן אֲשֶׁר יַפְלֶה יְהוָה בֵּין מִצְרַיִם וּבֵין יִשְׂרָאֵל} & \textit{\dots \space so that you know that the Lord will make a distinction between Egypt and Israel} (Exod 11:7) \\
	\end{tabular}
	
	\vspace{0.5cm}
	
	\noindent Substantivized clauses with \foreignlanguage{hebrew}{אֲשֶׁר} may also be dependent on a preposition.
	
	\begin{longtable}{>{\raggedleft}p{0.35\linewidth} p{0.55\linewidth}}
		\foreignlanguage{hebrew}{כִּי־שָׁמַע אֱלֹהִים אֶל־קוֹל הַנַּעַר בַּאֲשֶׁר הוּא־שָׁם} & \textit{\dots \space for God has heard the voice of the boy where he is} (Gen 21:17) \\
		\foreignlanguage{hebrew}{כִּי אֶסְלַח לַאֲשֶׁר אַשְׁאִיר} & \textit{\dots \space for I will forgive those whom I will leave as a remnant} (Jer 50:20; with ellipsis of the direct object) \\
		\foreignlanguage{hebrew}{וַיֵּלֶךְ אַבְרָם כַּאֲשֶׁר דִּבֶּר אֵלָיו יְהוָה} & \textit{And Abram went according to what the Lord had said to him} (Gen 12:4) \\
	\end{longtable}
	
	\subsection{Other Subordinate Clauses Introduced with \foreignlanguage{hebrew}{אֲשֶׁר}}
	The subordination marker \foreignlanguage{hebrew}{אֲשֶׁר} maybe also be used as a conjunction with causal, concessive, final, consecutive or temporal subordinate clauses.
	
	\section{Exercises}
	
	\subsection{Translation of Verbal Forms}
	Translate the following verbal forms. Identify the gender (masc., fem., comm.) and number (sg., pl.) the forms of which the English translation is ambiguous (i.e., \textit{you}, \textit{they}). Mark the stressed syllable if stress is not on the last syllable.
	
	\hspace{0.5cm}
	
	\selectlanguage{hebrew}
	
	\noindent
	1~~\foreignlanguage{hebrew}{תִּזְכְּרוּ}  \hspace{0.3cm}
	2~~\foreignlanguage{hebrew}{אֶכְרֹת}  \hspace{0.3cm}
	3~~\foreignlanguage{hebrew}{תַּעֲבֹד}  \hspace{0.3cm}
	4~~\foreignlanguage{hebrew}{תִּשְׁמְרוּן}  \hspace{0.3cm}
	5~~\foreignlanguage{hebrew}{תִּקְרַב}  \hspace{0.3cm}
	6~~\foreignlanguage{hebrew}{תִּזְכְּרִי}  \hspace{0.3cm}
	7~~\foreignlanguage{hebrew}{נִשְׁמַע}  \hspace{0.3cm}
	8~~\foreignlanguage{hebrew}{אֶשְׁלַח}  \hspace{0.3cm}
	9~~\foreignlanguage{hebrew}{יִמְלֹ֫כוּ}  \hspace{0.3cm}
	10~~\foreignlanguage{hebrew}{תִּשְׁלַחְנָה}  \hspace{0.3cm}
	
	\selectlanguage{english}
	
	\subsection{Translation of Sentences}
	Translate the following sentences from the Hebrew Bible. Names of persons and geographical names in these sentences: \foreignlanguage{hebrew}{אַהֲרֹן}, \foreignlanguage{hebrew}{אַחְאָב}, \foreignlanguage{hebrew}{אֶפְרַיִם}, \foreignlanguage{hebrew}{בְנָיָהוּ}, \foreignlanguage{hebrew}{גִּדְעוֹן}, \foreignlanguage{hebrew}{חִזְקִיָּהוּ}, \foreignlanguage{hebrew}{יָבֵישׁ גִּלְעָד}, \foreignlanguage{hebrew}{יִפְתָּח}, \foreignlanguage{hebrew}{מִצְרַיִם}, \foreignlanguage{hebrew}{מֹשֶׁה}, \foreignlanguage{hebrew}{עַמּוֹן} \foreignlanguage{hebrew}{שָׂרָה}, \foreignlanguage{hebrew}{שָׂרַי}, \foreignlanguage{hebrew}{שְׁלֹמֹה}, \foreignlanguage{hebrew}{שְׁמוּאֵל}.
	
	\vspace{0.5cm}
	
	\selectlanguage{hebrew}
	
	% Comments: no. 3 is not ideal
	
	\noindent
	1~~\foreignlanguage{hebrew}{לֹא שָׁמַרְתָּ אֶת־מִצְוַת יְהוָה אֱלֹהֶ֫יךָ}\LTRfootnote{\space \foreignlanguage{hebrew}{יהוה} read as \foreignlanguage{hebrew}{אֲדֹנָי} (Engl. \textit{Lord}); \foreignlanguage{hebrew}{אֱלֹהֶ֫יךָ} \textit{your God}} \foreignlanguage{hebrew}{אֲשֶׁר צִוָּךְ}\LTRfootnote{\space \foreignlanguage{hebrew}{צִוָּךְ} \textit{he commanded you}.}  \hspace{0.3cm}
	2~~\foreignlanguage{hebrew}{וַיהוָה}\LTRfootnote{\space \foreignlanguage{hebrew}{יהוה} read as \foreignlanguage{hebrew}{אֲדֹנָי} (Engl. \textit{Lord}), with the conjunction \foreignlanguage{hebrew}{וְ} (see chapter 3)} \foreignlanguage{hebrew}{פָּקַד אֶת־שָׂרָה כַּאֲשֶׁר אָמָר}\LTRfootnote{\space \foreignlanguage{hebrew}{אָמָר} pausal form of \foreignlanguage{hebrew}{אָמַר}} \foreignlanguage{hebrew}{וַיַּעַשׂ יְהוָה לְשָׂרָה כַּאֲשֶׁר דִּבֵּר}\LTRfootnote{\space \foreignlanguage{hebrew}{דִּבֵּר} pausal form of \foreignlanguage{hebrew}{דִּבֶּר}} \hspace{0.3cm}
	3~~\foreignlanguage{hebrew}{וּמַלְאָכִים שָׁלַח גִּדְעוֹן בְּכָל־הַר אֶפְרַיִם}  \hspace{0.3cm}
	4~~\foreignlanguage{hebrew}{וְלֹא שָׁמַע מֶלֶךְ בְּנֵי עַמּוֹן אֶל־דִּבְרֵי יִפְתָּח אֲשֶׁר שָׁלַח אֵלָיו}\LTRfootnote{\space \foreignlanguage{hebrew}{אֵלָיו} \textit{to him} (\textit{ʾēlāw} with silent /y/; prep. \foreignlanguage{hebrew}{אֵל} with enclitic pronoun 3 m.\ sg.)} \hspace{0.3cm}
	5~~\foreignlanguage{hebrew}{וּבְנֵי יִשְׂרָאֵל הָלְכוּ בַיַּבָּשָׁה}\LTRfootnote{\space \foreignlanguage{hebrew}{יַבָּשָׁה} \textit{dry ground}} \foreignlanguage{hebrew}{בְּתוֹךְ הַיָּם}  \hspace{0.3cm}
	6~~\foreignlanguage{hebrew}{וַיְהִי הַמֶּלֶךְ שְׁלֹמֹה מֶלֶךְ עַל־כָּל־יִשְׂרָאֵל} \hspace{0.3cm}
	7~~\foreignlanguage{hebrew}{וַיְהִי דְבַר־שְׁמוּאֵל לְכָל־יִשְׂרָאֵל}  \hspace{0.3cm}
	8~~\foreignlanguage{hebrew}{וְהָרָעָב הָיָה עַל כָּל־פְּנֵי הָאָרֶץ}  \hspace{0.3cm}
	9~~\foreignlanguage{hebrew}{וַיָּבֹא בְנָיָהוּ אֶל־אֹהֶל יְהוָה}\LTRfootnote{\space \foreignlanguage{hebrew}{יהוה} read as \foreignlanguage{hebrew}{אֲדֹנָי} (Engl. \textit{Lord})}  \hspace{0.3cm}
	10~~\foreignlanguage{hebrew}{הִנֵּה לֹא בָא־אִישׁ אֶל־הַמַּחֲנֶה מִיָּבֵישׁ גִּלְעָד אֶל־הַקָּהָל}  \hspace{0.3cm}
	11~~\foreignlanguage{hebrew}{וַיֵּלֶךְ בְּכָל־הַדֶּרֶךְ אֲשֶׁר־הָלַךְ אָבִיו}\LTRfootnote{\space \foreignlanguage{hebrew}{אָבִיו} \textit{ʾāḇīw} \textit{his father}} \hspace{0.3cm}
	12~~\foreignlanguage{hebrew}{וַיְדַבֵּר אַהֲרֹן אֵת כָּל־הַדְּבָרִים אֲשֶׁר־דִּבֶּר יְהוָה}\LTRfootnote{\space \foreignlanguage{hebrew}{יהוה} read as \foreignlanguage{hebrew}{אֲדֹנָי} (Engl. \textit{Lord})} \foreignlanguage{hebrew}{אֶל־מֹשֶׁה וַיַּעַשׂ הָאֹתֹת לְעֵינֵי הָעָם} \hspace{0.3cm}
	13~~\foreignlanguage{hebrew}{וַיֹּאמֶר לָהֶם}\LTRfootnote{\space \foreignlanguage{hebrew}{לָהֶם} \textit{to them}} \foreignlanguage{hebrew}{הִנֵּה בֶן־הַמֶּלֶךְ יִמְלֹךְ כַּאֲשֶׁר דִּבֶּר יְהוָה עַל־בְּנֵי דָוִיד} \hspace{0.3cm}
	14~~\foreignlanguage{hebrew}{וַיִּשְׁלַח יְהוָה אִישׁ נָבִיא אֶל־בְּנֵי יִשְׂרָאֵל}  \hspace{0.3cm}
	15~~\foreignlanguage{hebrew}{וַיִּשְׁלַח מַלְאָכִים אֶל־אַחְאָב מֶלֶךְ־יִשְׂרָאֵל הָעִ֫ירָה}  \hspace{0.3cm}
	16~~\foreignlanguage{hebrew}{וְנִשְׁלְחָה מַלְאָכִים בְּכֹל גְּבוּל יִשְׂרָאֵל}  \hspace{0.3cm}
	17~~\foreignlanguage{hebrew}{וַיִּשְׁמְעוּ}\LTRfootnote{\space \foreignlanguage{hebrew}{וַיִּשְׁמְעוּ} \textit{and they heard} (plural of \foreignlanguage{hebrew}{וַיִשְׁמַע})} \foreignlanguage{hebrew}{כִּי־פָקַד יְהוָה אֶת־בְּנֵי יִשְׂרָאֵל} \hspace{0.3cm}
	18~~\foreignlanguage{hebrew}{וַיִּשְׁמַע אַבְרָם לְקוֹל שָׂרָי}\LTRfootnote{\space \foreignlanguage{hebrew}{שָׂרָי} pausal form of the name \foreignlanguage{hebrew}{שָׂרַי}}  \hspace{0.3cm}
	19~~\foreignlanguage{hebrew}{אַל־תִּשְׁמְעוּ אֶל־חִזְקִיָּהוּ}  \hspace{0.3cm}
	20~~\foreignlanguage{hebrew}{וַיִּשְׁמַע שְׁמוּאֵל אֵת כָּל־דִּבְרֵי הָעָם}  \hspace{0.3cm}
	21~~\foreignlanguage{hebrew}{וַיֶּחֱזַק לֵב פַּרְעֹה וְלֹא שָׁמַע אֲלֵהֶם}\LTRfootnote{\space \foreignlanguage{hebrew}{אֲלֵהֶם} \textit{to them}} \foreignlanguage{hebrew}{כַּאֲשֶׁר דִּבֶּר יְהוָה} \hspace{0.3cm}
	22~~\foreignlanguage{hebrew}{וַיַּעַבְדוּ אֶת־יְהוָה}  \hspace{0.3cm}
	23~~\foreignlanguage{hebrew}{וַיֹּאמֶר הַכֹּהֵן נִקְרְבָה הֲלֹם}\LTRfootnote{\space \foreignlanguage{hebrew}{הֲלֹם} \textit{to this place, here} (adv.)} \foreignlanguage{hebrew}{אֶל־הָאֱלֹהִים} \hspace{0.3cm}
	24~~\foreignlanguage{hebrew}{בְּהוֹצִיאֲךָ}\LTRfootnote{\space \foreignlanguage{hebrew}{בְּהוֹצִיאֲךָ} \textit{when you have brought out}}  \foreignlanguage{hebrew}{אֶת־הָעָם מִמִּצְרַיִם תַּעַבְדוּן אֶת־הָאֱלֹהִים עַל הָהָר הַזֶּה} \hspace{0.3cm}
	
	\selectlanguage{english}
	
	
	\chapter{Chapter 7}
	
	\renewcommand\arraystretch{1.4}
	
	\section{Vocabulary}
	
	\subsection{Verbs}
	
	\begin{center}
		
		% For the centering of the separation between the two columns see the documentation of the array package, page 2 
		
		\begin{tabular}{>{\raggedleft}p{0.175\linewidth} p{0.75\linewidth}}
			\foreignlanguage{hebrew}{בחר} & Q.\ \textit{to choose} (with prepositional object with \foreignlanguage{hebrew}{בְּ} or direct object) \\
			\foreignlanguage{hebrew}{דרשׁ} & Q.\ \textit{to seek, resort to} \\
			\foreignlanguage{hebrew}{הרג} & Q.\ \textit{to kill, slay} \\
			\foreignlanguage{hebrew}{זבח} & Q.\ \textit{to slaughter for sacrifice, slaughter} \\
			\foreignlanguage{hebrew}{עזב} & Q.\ \textit{to leave, forsake} \\
			\foreignlanguage{hebrew}{פתח} & Q.\ \textit{to open} \\
			\foreignlanguage{hebrew}{קבר} & Q.\ \textit{to bury} \\
			\foreignlanguage{hebrew}{רדף} & Q.\ \textit{to pursue, chase, persecute} \\
			\foreignlanguage{hebrew}{שׂמח} & Q.\ \textit{to rejoice, be glad} (stative verb, 3 m.\ sg.\ SC pausal form \foreignlanguage{hebrew}{שָׂמֵ֑חַ}) \\
			\foreignlanguage{hebrew}{שׁבר} & Q.\ \textit{to break, break in pieces} \\
			\foreignlanguage{hebrew}{שׁחט} & Q.\ \textit{to slaughter} \\
			\foreignlanguage{hebrew}{שׁכב} & Q.\ \textit{to lie down} (3 m.\ sg.\ PC \foreignlanguage{hebrew}{יִשְׁכַּב}) \\
			\foreignlanguage{hebrew}{שׁפט} & Q.\ \textit{to judge, govern} \\
		\end{tabular}
	\end{center}
	
	\subsection{Adjectives}
	
	\begin{center}
		\begin{longtable}{>{\raggedleft}p{0.175\linewidth} p{0.75\linewidth}}
			\foreignlanguage{hebrew}{גָדוֺל} & \textit{great} \\
			\foreignlanguage{hebrew}{חַי} & \textit{alive, living} (pl. \foreignlanguage{hebrew}{חַיִּים}) \\
			\foreignlanguage{hebrew}{חָכָם} & \textit{wise} \\
			\foreignlanguage{hebrew}{טוֺב} & \textit{good} \\
			\foreignlanguage{hebrew}{יָשָׁר} & \textit{straight, right} \\
			\foreignlanguage{hebrew}{קָדוֺשׁ} & \textit{holy} \\
			\foreignlanguage{hebrew}{רַב} & \textit{much, many, great} (pl. \foreignlanguage{hebrew}{רַבִּים}) \\
			\foreignlanguage{hebrew}{רַע} & \textit{bad, evil} (pl. \foreignlanguage{hebrew}{רָעִים}) \\
		\end{longtable}
	\end{center}
	
	\newpage
	
	\subsection{Nouns}
	
	\begin{center}
		\begin{longtable}{>{\raggedleft}p{0.175\linewidth} p{0.75\linewidth}}
			\foreignlanguage{hebrew}{בָּמָה} & \textit{hill, high place (often as place of worship)} \\
			\foreignlanguage{hebrew}{חַטָּאת} & \textit{sin, sin offering} (cs.\ \foreignlanguage{hebrew}{חַטַּאת}; pl.\ abs.\ \foreignlanguage{hebrew}{חַטָּאוֺת}; pl.\ cs.\ \foreignlanguage{hebrew}{חַטֹּאת}) \\
			\foreignlanguage{hebrew}{עוֹלָם} & \textit{long time, duration, future time, long time back} \\
			\foreignlanguage{hebrew}{צָבָא} & \textit{army, host, war, warfare} (pl.\ \foreignlanguage{hebrew}{צְבָאוֺת}) \\
			\foreignlanguage{hebrew}{קֹדֶשׁ} & \textit{holiness} \\
			\foreignlanguage{hebrew}{שִׁפְחָה} & \textit{maid, maid-servant} (pl.\ \foreignlanguage{hebrew}{שְׁפָחוֺת}) \\
		\end{longtable}
	\end{center}
	
	\subsection{Other Parts of Speech}
	
	\begin{center}
		\begin{tabular}{>{\raggedleft}p{0.175\linewidth} p{0.75\linewidth}}
			\foreignlanguage{hebrew}{אַל} & \textit{not} (with the jussive) \\ % Action point; already in Chapter 6; why here again?
			\foreignlanguage{hebrew}{גַּם} & \textit{also, even} \\
			\foreignlanguage{hebrew}{מְאֹד} & \textit{very} (degree particle); as a noun \textit{strength, power} \\
			\foreignlanguage{hebrew}{עַתָּה} & \textit{now} (adv. of time) \\
		\end{tabular}
	\end{center}
	
	
	\section{The Independent Personal Pronoun}
	
	\subsection{The Forms of the Independent Personal Pronoun}
	The independent personal pronoun (iPP) has the following forms:
	
	\renewcommand\arraystretch{1.4}
	
	\begin{center}
		\begin{tabular}{|ll|r|l|ll|r|l|}
			\hline
			\multicolumn{4}{|c|}{Singular} & \multicolumn{4}{c|}{Plural} \\
			\hline
			1 & c. & \foreignlanguage{hebrew}{אֲנִי} ,\foreignlanguage{hebrew}{אָנֹכִי} & \textit{I}  & 1 & c. & \foreignlanguage{hebrew}{נַ֫חְנוּ} ,\foreignlanguage{hebrew}{אֲנַ֫חְנוּ} & \textit{we}\\
			\hline
			\multirow{2}{*}{2} & m. & \foreignlanguage{hebrew}{אַתָּה} & \textit{you} & \multirow{2}{*}{2} & m. & \foreignlanguage{hebrew}{אַתֶּם} & \textit{you}\\
			& f. & \foreignlanguage{hebrew}{אַתְּ} & \textit{you} & & f. & \foreignlanguage{hebrew}{אַתֶּן} ,\foreignlanguage{hebrew}{אַתֵּ֫נָּה} & \textit{you}\\
			\hline
			\multirow{2}{*}{3} & m. & \foreignlanguage{hebrew}{הוּא} & \textit{he} & \multirow{2}{*}{3} & m. & \foreignlanguage{hebrew}{הֵם} ,\foreignlanguage{hebrew}{הֵ֫מָּה} & \textit{they}\\
			& f. & \foreignlanguage{hebrew}{הִיא} & \textit{she} & & f. & \foreignlanguage{hebrew}{הֵ֫נָּה} & \textit{they}\\
			\hline
		\end{tabular}	
	\end{center}
	
	\vspace{0.5cm}
	
	\noindent \textbf{Notes about the Forms of the Independent Personal Pronoun}
	
	% Examples: 1 Sam 24:9
	
	\begin{enumerate}[noitemsep]
		\item The following pausal forms are attested (some of them have a different position of the stress than the context forms): 1 c.\ sg.\ \foreignlanguage{hebrew}{אָנֹ֑כִי} and \foreignlanguage{hebrew}{אָ֑נִי}; 2 m.\ sg.\ \foreignlanguage{hebrew}{אָ֑תָּה} or \foreignlanguage{hebrew}{אַ֔תָּה} (with \textit{zāqēf qāṭōn}); 2 f.\ sg.\ \foreignlanguage{hebrew}{אָ֑תְּ}; 1 c.\ pl.\ \foreignlanguage{hebrew}{אֲנָ֑חְנוּ} or \foreignlanguage{hebrew}{נָ֑חְנוּ}.
		\item In the Torah, the personal pronoun of the 3 f.\ sg.\ is spelled \foreignlanguage{hebrew}{הוא} in the Ketiv with the Qere perpetuum \foreignlanguage{hebrew}{הִיא} in almost all instances. For more information cf.\ Joüon and Muraoka, \textit{A Grammar of Biblical Hebrew}, §§\,16f, 39c.
		\item The existence of two different pronouns for the 1 c.\ sg.\ is remarkable. The form \foreignlanguage{hebrew}{אֲנִי} is more frequent than the form \foreignlanguage{hebrew}{אָנֹכִי} (874 and 359 attestations, respectively). The longer form \foreignlanguage{hebrew}{אָנֹכִי} was phased out in Late Biblical Hebrew. Mishnaic Hebrew only uses \foreignlanguage{hebrew}{אֲנִי}.
		\item The 2 m.\ sg.\ form is five times spelled \foreignlanguage{hebrew}{אַתָּ}.
		\item Rare alternative forms are \foreignlanguage{hebrew}{אַתְּ} for the 2 m.\ sg.\ (four times); \foreignlanguage{hebrew}{אַתִּי}* for the 2 f.\ sg.\ (seven times as Ketiv \foreignlanguage{hebrew}{אתי}); \foreignlanguage{hebrew}{אֲנוּ}* for the 1 c.\ pl.\ (as in Mishnaic Hebrew; only in Jer 42:6 Ketiv).
	\end{enumerate}
	
	\subsection{The Use of the iPP in Nominal Clauses}
	In nominal clauses the iPP may be used in the following ways.
	
	\begin{enumerate}[noitemsep]
		\item The independent personal pronoun is used as the subject in nominal clauses (2 Sam 12:7).
		\item The independent personal pronoun is used as subject in clauses with a participle as predicate (1\,Sam 17:45).
		\item The 3 m.\ sg.\ form \foreignlanguage{hebrew}{הוּא} may be used in verbless clauses to disambiguate the preceding subject and the predicate (2\,Sam 7:28).
		\item As subject of a nominal clause, the independent personal pronoun may be used similarly to the demonstrative pronoun (2\,Kgs 15:12; Lev 10:3).
	\end{enumerate}
	
	\begin{tabular}{>{\raggedleft}p{0.35\linewidth} p{0.55\linewidth}}
		\foreignlanguage{hebrew}{אַתָּה הָאִישׁ} & \textit{You are the man} (2 Sam 12:7) \\
		\foreignlanguage{hebrew}{אַתָּה בָּא אֵלַי בְּחֶרֶב וּבַחֲנִית וּבְכִידוֹן וְאָנֹכִי בָא־אֵלֶיךָ בְּשֵׁם יְהוָה צְבָאוֹת אֱלֹהֵי מַעַרְכוֹת יִשְׂרָאֵל} & \textit{You are coming to me with a sword, with a spear and with a javelin, but I come to you in the name of the Lord of hosts, the God of the ranks of Israel} (1\,Sam 17:45) \\
		\foreignlanguage{hebrew}{אַתָּה־הוּא הָאֱלֹהִים} & \textit{You are God} (2\,Sam 7:28) \\
		\foreignlanguage{hebrew}{הוּא דְבַר־יְהוָה אֲשֶׁר דִּבֶּר אֶל־יֵהוּא} & \textit{This is the word of the Lord that he had spoken to Jehu} (2\,Kgs 15:12) \\
		\foreignlanguage{hebrew}{הוּא אֲשֶׁר־דִּבֶּר יְהוָה} & \textit{This is what the Lord has said~\dots} (Lev 10:3) \\
	\end{tabular}
	
	\subsection{The Use of the iPP in Verbal Clauses}
	The independent personal pronoun is used as the subject in clauses with finite verbal forms as predicate (often for emphasis of the subject):
	
	\vspace{0.5cm}
	
	\begin{tabular}{>{\raggedleft}p{0.35\linewidth} p{0.55\linewidth}}
		\foreignlanguage{hebrew}{כִּי אַתָּה יָדַעְתָּ לְבַדְּךָ אֶת־לְבַב כָּל־בְּנֵי הָאָדָם} & \textit{For \emph{you} alone know the heart of all children of mankind} (1\,Kgs 8:39) \\
	\end{tabular}
	
	
	\subsection{The iPP as Demonstrative Pronoun}
	As attribute to a noun the 3rd person forms may be used as demonstrative pronouns for far deixis. The independent personal pronoun in this function follows the noun it modifies and always has the article.
	
	\vspace{0.5cm}
	
	\begin{tabular}{>{\raggedleft}p{0.35\linewidth} p{0.55\linewidth}}
		\foreignlanguage{hebrew}{הָאִישׁ הַהוּא} & \textit{that man} (Lev 20:4) \\
		\foreignlanguage{hebrew}{הָעִיר הַהִיא} & \textit{that city} (Josh 20:4) \\
		\foreignlanguage{hebrew}{הָאֲנָשִׁים הָהֵמָּה} & \textit{those men} (Num 9:7) \\
	\end{tabular}
	
	\vspace{0.5cm}
	
	This use of the iPP is frequently found in the temporal adverbial phrases \foreignlanguage{hebrew}{בַּיּוֹם הַהוּא} \textit{on that day} and \foreignlanguage{hebrew}{בַּיָּמִים הָהֵם} \textit{in those days}.
	
	\subsection{The iPP in Apposition to an Enclitic Pronoun}
	The independent personal pronoun may be used in apposition to an enclitic personal pronoun for emphasis:   
	
	\vspace{0.5cm}
	
	\begin{tabular}{>{\raggedleft}p{0.35\linewidth} p{0.55\linewidth}}
		\foreignlanguage{hebrew}{בָּרֲכֵנִי גַם־אָנִי אָבִי} & \textit{Bless me, too, O my father} (Gen 27:34) \\
		\foreignlanguage{hebrew}{בִּמְקוֹם אֲשֶׁר לָקְקוּ הַכְּלָבִים אֶת־דַּם נָבוֹת יָלֹקּוּ הַכְּלָבִים אֶת־דָּמְךָ גַּם־אָֽתָּה} &  \textit{In the place where the dogs licked the blood of Naboth the dogs will lick even your blood} (1\,Kgs 21:19) \\
	\end{tabular}
	
	
	
	\section{Nominal Clauses}
	Clauses without a finite verbal form as predicate are called nominal clauses. The subject is typically a noun or pronoun. Predicates can be nouns, adjectives or prepositional phrases.
	
	If the predicate is indefinite, the nominal clause has descriptive or classifying function.
	
	\vspace{0.5cm}
	
	\begin{tabular}{>{\raggedleft}p{0.35\linewidth} p{0.55\linewidth}}
		\foreignlanguage{hebrew}{נַעַר מִצְרִי אָנֹכִי} & \textit{I am a young man from Egypt} (1\,Sam 30:13) \\
		\foreignlanguage{hebrew}{הָרָה אָנֹכִי} & \textit{I am pregnant} (2\,Sam 11:5) \\
	\end{tabular}
	
	\vspace{0.5cm}
	
	
	If the predicate is definite, the nominal clause has identifying function.
	
	\vspace{0.5cm}
	
	\begin{tabular}{>{\raggedleft}p{0.35\linewidth} p{0.55\linewidth}}
		\foreignlanguage{hebrew}{אֲנִי יוֹסֵף} & \textit{I am Joseph} (Gen 45:3) \\
		\foreignlanguage{hebrew}{אָנֹכִי הָרֹאֶה} & \textit{I am the seer} (1 Sam 9:19) \\
		\foreignlanguage{hebrew}{אָנֹכִי אֱלֹהֵי אַבְרָהָם אָבִיךָ} & \textit{I am the God of Abraham, your father} (Gen. 26:24) \\
	\end{tabular}
	
	\vspace{0.5cm}
	
	Prepositional phrases can also function as predicate of a nominal clause.
	
	\vspace{0.5cm}
	
	\begin{tabular}{>{\raggedleft}p{0.35\linewidth} p{0.55\linewidth}}
		\foreignlanguage{hebrew}{וְהִנֵּה אָנֹכִי עִמָּךְ} & \textit{Behold, I am with you} (Gen 28:15) \\
		\foreignlanguage{hebrew}{מֵחָרָן אֲנָחְנוּ} & \textit{We are from Haran} (Gen 29:4) \\
	\end{tabular}
	
	\vspace{0.5cm}
	
	The most common unmarked constituent order of nominal clauses is subject--predicate. The reasons for deviating from this constituent order are diverse.
	
	% Here, additional information is needed.
	
	Nominal clauses are timeless. The default location in time is the present. In a past tense context, however, nominal clauses are located in the past. To make past or future reference explicit, forms of the verb \foreignlanguage{hebrew}{היה} \textit{to be} are used. Temporal adverbs may serve the same function, e.g., \foreignlanguage{hebrew}{וּמָחָר אַתָּה וּבָנֶיךָ עִמִּי} \textit{and tomorrow you and your sons will be with me} (1\,Sam 28:19).
	
	As an additional element, the independent personal pronoun \foreignlanguage{hebrew}{הוּא} may be used in nominal clauses. Different explanations were presented for this use of \foreignlanguage{hebrew}{הוּא}. It possibly functions to disambiguate the preceding subject and the predicate.
	
	% This explanation is taken from BHRG S. 293 (§36.1.1.2.4).
	
	\vspace{0.5cm}
	
	\begin{tabular}{>{\raggedleft}p{0.35\linewidth} p{0.55\linewidth}}
		\foreignlanguage{hebrew}{אַתָּה־הוּא הָאֱלֹהִים} & \textit{You are God} (2\,Sam 7:28) \\
	\end{tabular}
	
	
	
	\section{Adjectives}
	
	\subsection{The Use of the Adjective}
	Adjectives are a subcategory of the noun. Adjectives have have three different functions in Hebrew:
	
	\begin{enumerate}[noitemsep]
		\item As attribute they modify (describe) a noun (2\,Sam 14:2).
		\item They can be used independently like other nouns (the so-called substantivized use) (Qoh 2:14).
		\item They can be used as the predicate of a nominal clause (Jer 8:8).
	\end{enumerate}
	
	\begin{tabular}{>{\raggedleft}p{0.35\linewidth} p{0.55\linewidth}}
		\foreignlanguage{hebrew}{וַיִּקַּח מִשָּׁם אִשָּׁה חֲכָמָה} & \textit{And he took from there a wise woman} (2\,Sam 14:2) \\
		\foreignlanguage{hebrew}{מִי כְּהֶחָכָם} & \textit{Who is like the wise?} (Qoh 8:1) \\
		\foreignlanguage{hebrew}{חֲכָמִים אֲנַחְנוּ} & \textit{We are wise} (Jer 8:8) \\
	\end{tabular}
	
	\vspace{0.5cm}
	
	Adjective phrases can be expanded with the degree particle \foreignlanguage{hebrew}{מְאֹד} \textit{very}, e.g., \foreignlanguage{hebrew}{אִישׁ חָכָם מְאֹד} \textit{a very wise man} (2\,Sam 13:3). Sometimes, the preposition \foreignlanguage{hebrew}{עַד} \textit{up to, until} is added, e.g., \foreignlanguage{hebrew}{חֲרָדָה גְּדֹלָה עַד־מְאֹד} \textit{a very strong trembling} (Gen 27:33). Originally, \foreignlanguage{hebrew}{מְאֹד} is a noun meaning \textit{strength, power}.
	
	\subsection{The Forms of the Adjective}
	As a subcategory of the noun, adjectives have the same endings except for the fact that they are not used in the dual. Adjectives with changeable vowels follow the rules for vowel reduction for nouns (Chapter 4).
	
	\begin{center}
		\begin{tabular}{|l|lr|lr|}
			\hline
			& \multicolumn{2}{c|}{Masculine} & \multicolumn{2}{c|}{Feminine}\\
			\hline
			Singular & ∅ & \foreignlanguage{hebrew}{טוֺב} & \textit{-ā} & \foreignlanguage{hebrew}{טוֺבָה} \\
			Plural & \textit{-īm} & \foreignlanguage{hebrew}{טוֺבִים} & \textit{-ōt} & \foreignlanguage{hebrew}{טוֺבוֺת} \\
			\hline
		\end{tabular}
	\end{center}
	
	\subsection{Adjectives as Attribute or Predicate}
	If used as attribute, adjectives follow the noun they modify and must agree in number and gender with the noun. If the modified noun is definite, the adjective takes the article. The following examples illustrate this.
	
	\vspace{0.5cm}
	
	\begin{tabular}{>{\raggedleft}p{0.35\linewidth} p{0.55\linewidth}}
		\foreignlanguage{hebrew}{אִישׁ גָּדוֹל} & \textit{a great man} (2\,Kgs 5:1) \\
		\foreignlanguage{hebrew}{מְלָכִים גְּדוֹלִים} & \textit{great kings} (Jer 25.14) \\
		\foreignlanguage{hebrew}{אִשָּׁה גְדוֹלָה} & \textit{a wealthy woman} (2\,Kgs 4:8) \\
		\foreignlanguage{hebrew}{מַמְלָכוֹת גְּדֹלוֹת} & \textit{great kingdoms} (Jer 28:8) \\
		\foreignlanguage{hebrew}{הַיָּם הַגָּדוֹל} & \textit{the great sea} (i.e., the Mediterranean Sea) (Josh 23:4) \\
		\foreignlanguage{hebrew}{הַבְּרֵכָה הָעֶלְיוֹנָה} & \textit{the upper pool} (2\,Kgs 18:17) \\
		\foreignlanguage{hebrew}{הֶהָרִים הַגְּבֹהִים} & \textit{the high mountains} (Gen 7:19) \\
		\foreignlanguage{hebrew}{הָאֹתוֹת הַגְּדֹלוֹת} & \textit{the great signs} (Josh 24:17) \\
	\end{tabular}
	
	\vspace{0.5cm}
	
	\noindent \textbf{Note}
	\nopagebreak
	
	\noindent Agreement is also required in the case of fem.\ nouns without a fem.\ ending, e.g., \foreignlanguage{hebrew}{הָעִיר הַגְּדֹלָה} \textit{the great city} (Gen 10:12).
	
	\vspace{0.5cm}
	
	If used as predicate of nominal clauses (see the preceding section), adjectives agree with the subject in gender and number.
	
	\begin{tabular}{>{\raggedleft}p{0.35\linewidth} p{0.55\linewidth}}
		\foreignlanguage{hebrew}{טוֹבָה הָאָרֶץ מְאֹד מְאֹד} & \textit{The land is exceedingly good} (Num 14:7) \\
	\end{tabular}
	
\subsection{Comparative and Superlative}
Unlike English or Greek, Hebrew does not have distinct forms to express the comparative (e.g., \textit{higher}) or the superlative (e.g., \textit{highest}). For the comparative, a prepositional phrase with \foreignlanguage{hebrew}{מִן} is used to indicate the entity with which something is compared; the adjective itself remains unchanged. This is illustrated by the following examples with the adjectives \foreignlanguage{hebrew}{מָתוֺק} \textit{sweet} and \foreignlanguage{hebrew}{צַדִּיק} \textit{righteous}
	
\begin{longtable}{>{\raggedleft}p{0.35\linewidth} p{0.55\linewidth}}
		\foreignlanguage{hebrew}{מַה־מָּתוֹק מִדְּבַשׁ וּמֶה עַז מֵאֲרִי} & \textit{What is sweeter than honey and what is stronger than a lion?} (Judg 14:18) \\
		\foreignlanguage{hebrew}{צַדִּיק אַתָּה מִמֶּנִּי} & \textit{You are more righteous than I} (1\,Sam 24:18) \\
\end{longtable}
	
For the superlative, the adjective is used with the definite article or it is made definite by other means, e.g., a definite dependent noun in a construct chain. 
	
\begin{longtable}{>{\raggedleft}p{0.35\linewidth} p{0.55\linewidth}}
		\foreignlanguage{hebrew}{הָעִיר הַקְּרֹבָה אֶל־הֶחָלָל} & \textit{the town that is nearest to the slain man} (Deut 21:3) \\
		\foreignlanguage{hebrew}{יְהוֹאָחָז קְטֹן בָּנָיו} & \textit{Jehoahaz, his youngest son} (2\,Chr 21:17) \\
\end{longtable}
	
If the two entities that are compared are incommensurable, the particle \textit{too} together with a prepositional phrase with \textit{for} may be used in English.
	
	\vspace{0.5cm}
	
	\begin{tabular}{>{\raggedleft}p{0.35\linewidth} p{0.55\linewidth}}
		\foreignlanguage{hebrew}{קוּם אֱכֹל כִּי רַב מִמְּךָ הַדָּרֶךְ} & \textit{Get up, eat, for the journey is too much for you} (1\,Kgs 19:7) \\
	\end{tabular}
	
	
	\section{The Imperative}
	
	\subsection{The Forms of the Imperative}
	
	\renewcommand\arraystretch{1.4}
	
	\begin{table}[H]
		\begin{center}
			\begin{tabular}{|ll|r|r|}
				\hline
				& & \multicolumn{1}{c|}{fientive} & \multicolumn{1}{c|}{stative} \\
				\hline
				sg.& m. & \foreignlanguage{hebrew}{כְּתֹב} & \foreignlanguage{hebrew}{כְּבַד} \\
				& f. & \foreignlanguage{hebrew}{כִּתְבִי} & \foreignlanguage{hebrew}{כִּבְדִי} \\
				pl. & m. & \foreignlanguage{hebrew}{כִּתְבוּ} & \foreignlanguage{hebrew}{כִּבְדוּ} \\
				& f. & \foreignlanguage{hebrew}{כְּתֹ֫בְנָה} & \foreignlanguage{hebrew}{כְּבַ֫דְנָה} \\
				\hline
			\end{tabular}
		\end{center}
	\end{table}
	
	
	\noindent \textbf{Notes}
	\nopagebreak
	
	\noindent The thematic vowel of the imperative form of a verb is identical to the thematic vowel of the \textit{yiqṭol} form of that verb.
	
	There is no \textit{dageš lene} in the third root consonant in the fem.\ sg.\ and masc.\ pl.\ forms. The \textit{šwa} with the second root consonant it a medial \textit{šwa}.
	
	Pausal forms of the fem.\ sg.\ and masc.\ pl.\ forms have the same thematic vowel as the masc.\ sg.\ form, e.g., \foreignlanguage{hebrew}{כְּתֹ֑בִי}, \foreignlanguage{hebrew}{כְּתֹ֑בוּ}, \foreignlanguage{hebrew}{כְּבָ֑דוּ}.
	
	If the imperative is followed by \textit{maqqef}, the thematic vowel \textit{ḥolem} is shortened to \textit{qameṣ ḥatuf}, e.g., \foreignlanguage{hebrew}{מְלָךְ־עָלֵ֫ינוּ} \textit{Reign over us!} (Judg 9:14).
	
	% Other examples: 1 Sam 11:1; Exod 34:1; Exod 34:4; Exod 34:27
	
	An example of an imperative form of a stative verb is \foreignlanguage{hebrew}{חֲזַק} \textit{be strong!} (Josh 1:6). The pausal form of the m. sg. imperative of the stative verb has a lengthened thematic vowel, e.g., \foreignlanguage{hebrew}{אֱמָ֑ץ} \textit{be bold!} (Josh 1:6).
	
	
	\subsection{The Imperative of Strong Verbs with Gutturals}
	
	\begin{table}[H]
		\begin{center}
			\begin{tabular}{|ll|r|r|r|r|r|}
				\hline
				& & \multicolumn{2}{c|}{I gutt.} & \multicolumn{1}{c|}{I\,ʾ} & \multicolumn{1}{c|}{II gutt.} & \multicolumn{1}{c|}{III gutt.} \\
				\hline
				sg.& m. & \foreignlanguage{hebrew}{עֲזֹב} & \foreignlanguage{hebrew}{חֲזַק} & \foreignlanguage{hebrew}{אֱמֹר} & \foreignlanguage{hebrew}{בְּחַר} & \foreignlanguage{hebrew}{שְׁלַח} \\
				& f. & \foreignlanguage{hebrew}{עִזְבִי} & \foreignlanguage{hebrew}{חִזְקִי} & \foreignlanguage{hebrew}{אִמְרִי} & \foreignlanguage{hebrew}{בַּחֲרִי} & \foreignlanguage{hebrew}{שִׁלְחִי} \\
				pl. & m. & \foreignlanguage{hebrew}{עִזְבוּ} & \foreignlanguage{hebrew}{חִזְקוּ} & \foreignlanguage{hebrew}{אִמְרוּ} & \foreignlanguage{hebrew}{בַּחֲרוּ} & \foreignlanguage{hebrew}{שִׁלְחוּ} \\
				& f. & \foreignlanguage{hebrew}{עֲזֹ֫בְנָה} & \foreignlanguage{hebrew}{חֲזַ֫קְנָה} & \foreignlanguage{hebrew}{אֱמֹ֫רְנָה} & \foreignlanguage{hebrew}{בְּחַ֫רְנָה} & \foreignlanguage{hebrew}{שְׁלַ֫חְנָה} \\
				\hline
			\end{tabular}
		\end{center}
	\end{table}
	
	\pagebreak
	
	\noindent \textbf{Notes}
	\nopagebreak
	
	\noindent Like the regular verb, verbs I gutt.\ and I\,ʾ distinguish between fientive and stative verbs. An example of an imperative form of a I\,ʾ stative verb is \foreignlanguage{hebrew}{אֱמַץ} \textit{be bold!} (Josh 1:7).
	
	Imperative forms of verbs with \foreignlanguage{hebrew}{א} as first root consonant are always strong with a consonantal \foreignlanguage{hebrew}{א} (unlike the weak verbs I\,\textit{ʾ} in the prefix conjugation forms, e.g., \foreignlanguage{hebrew}{יֹאמַר}). The thematic vowel is either \textit{ḥolem} or \textit{pataḥ} depending on the verb type as fientive or stative, e.g., \foreignlanguage{hebrew}{אֱמֹר} \textit{say!}, \foreignlanguage{hebrew}{אֱמַץ} \textit{be bold!}
	
	In the fem.\ sg.\ and masc.\ pl.\ forms, II gutt. verbs have the vowel \textit{pataḥ} with the first root consonant and a composite \textit{šwa} (\textit{ḥaṭef pataḥ}) with the second root consonant instead of a silent \textit{šwa}.
	
	\subsection{The Long Form of the Imperative}
	
	The imperative masc.\ sg.\ may be expanded by attaching the ending \textit{-ā}. This ending is called paragogic \foreignlanguage{hebrew}{ה}. It occurs about 288 times in the Hebrew Bible. It \enquote{has, in general, no appreciable different nuance from the usual form} (JM §\,114m). Illustrative forms are: \foreignlanguage{hebrew}{אָכְלָה} \textit{ʾɔḵlā} \textit{eat!} (sometimes \foreignlanguage{hebrew}{מִכְרָה} \textit{sell!}), \foreignlanguage{hebrew}{שִׁמְעָה} \textit{hear!} (III gutt.).
	
	% Weak verbs: \foreignlanguage{hebrew}{גְּשָׂה} (I\,n), \foreignlanguage{hebrew}{תְּנָה} (I\,n), \foreignlanguage{hebrew}{שְׁבָה} (I\,y), \foreignlanguage{hebrew}{ק֫וּמָה} (II\,w), \foreignlanguage{hebrew}{שִׂ֫ימָה} (II\,)
	
	In pausal forms the ending \textit{-ā} is unaccented and the thematic vowel which is reduced in context forms, is preserved, e.g., \foreignlanguage{hebrew}{שְׁמָ֑עָה} \textit{hear!}.
	
	% Weak verbs: \foreignlanguage{hebrew}{לֵ֑כָה}
	
	\subsection{The Use of the Imperative}
	The imperative is used mainly for commands and wishes.
	
	\begin{longtable}{>{\raggedleft}p{0.35\linewidth} p{0.55\linewidth}}
		\foreignlanguage{hebrew}{שְׁמֹר אֶת־הָאִישׁ הַזֶּה} & \textit{Guard this man} (1\,Kgs 20:39) \\
		\foreignlanguage{hebrew}{וְעַתָּה שְׁמַע לְקוֹל דִּבְרֵי יְהוָה} & \textit{And now, listen to the voice of the words of the Lord} (1\,Sam 15:1) \\
		\foreignlanguage{hebrew}{פִּתְחוּ אֶת־פִּי הַמְּעָרָה} & \textit{Open the mouth of the cave} (Josh 10:22) \\
		\foreignlanguage{hebrew}{קִבְצוּ אֶת־כָּל־יִשְׂרָאֵל הַמִּצְפָּ֫תָה} & \textit{Gather all Israel to Mizpah} (1\,Sam 7:5) \\
	\end{longtable}
	
	Imperatives may be used with an independent personal pronoun (2nd person) as subject, e.g., \foreignlanguage{hebrew}{לֵךְ אַתָּה} \textit{Come!} (Judg 9:14).
	
	The imperative may be used for addressing a superior like a king or God without being impolite.
	
	The imperative is never used with a negative. Instead, either the jussive with \foreignlanguage{hebrew}{אַל} is used (the so-called prohibitive) is used or \foreignlanguage{hebrew}{לֹא} with the prefix conjugation (\textit{yiqṭol} form) (the so-called vetitive). The prohibitive is usually used for specific negative commands in a certain situation, whereas the vetitive typically occurs in general prohibitions in legal texts. Either form, however, can be found in contexts where one would expect the other.
	
	\begin{longtable}{>{\raggedleft}p{0.35\linewidth} p{0.55\linewidth}}
		\foreignlanguage{hebrew}{אַל־תִּשְׁפְּכוּ־דָם} & \textit{Do not shed blood} (Gen 37:22) \\
		\foreignlanguage{hebrew}{אַל־תַּעֲמֹד} & \textit{Don't stop} (1\,Sam 20:38) \\
		\foreignlanguage{hebrew}{לֹא־תִכְרֹת לָהֶם וְלֵאלֹהֵיהֶם בְּרִית} & \textit{Do not make a covenant with them or their gods} (Exod 23:32) \\
	\end{longtable}
	
The three volitive forms, the imperative, the jussive and the cohortative, may be followed by the particle \foreignlanguage{hebrew}{נָא}. The function of \foreignlanguage{hebrew}{נָא} is not completely clear. It may be indicating a polite request. According to a  different interpretation, \foreignlanguage{hebrew}{נָא} is used if the speaker expects some resistance to the request or command or if the request comes unexpectedly or does not flow from the context. It is advisable to leave \foreignlanguage{hebrew}{נָא} untranslated.
	
	% Cf. BHRG, p. 171 and 485; Lettinga (2012), p. 126.
	
	\vspace{0.5cm}
	
	\begin{tabular}{>{\raggedleft}p{0.35\linewidth} p{0.55\linewidth}}
		\foreignlanguage{hebrew}{דְּרָשׁ־נָא כַיּוֹם אֶת־דְּבַר יְהוָה} & \textit{Seek first the word of the Lord} (1\,Kgs 22:5) \\
	\end{tabular}
	
	
\section{Exercises}
	
\subsection{Translation of Sentences}
Translate the following sentences from the Hebrew Bible. Names of persons and geographical names in these sentences: \foreignlanguage{hebrew}{אֶבְיָתָר}, \foreignlanguage{hebrew}{אֲבִימֶלֶךְ}, \foreignlanguage{hebrew}{אַבְרָהָם}, \foreignlanguage{hebrew}{אַחְאָב}, \foreignlanguage{hebrew}{אִיזֶבֶל}, \foreignlanguage{hebrew}{אֵלִיָּהוּ}, \foreignlanguage{hebrew}{בְּאֵר שֶׁבַע}, \foreignlanguage{hebrew}{בֵּית לֶחֶם}, \foreignlanguage{hebrew}{בִּלְעָם}, \foreignlanguage{hebrew}{בָּלָק}, \foreignlanguage{hebrew}{דָּוִד}, \foreignlanguage{hebrew}{זֶבַח}, \foreignlanguage{hebrew}{יְהוֹשֻׁעַ}, \foreignlanguage{hebrew}{יוֹסֵף}, \foreignlanguage{hebrew}{יִשְׂרָאֵל}, \foreignlanguage{hebrew}{יִשַׁי}, \foreignlanguage{hebrew}{מֹשֶׁה}, \foreignlanguage{hebrew}{צַלְמֻנָּע}, \foreignlanguage{hebrew}{שָׁאוּל}, \foreignlanguage{hebrew}{שְׁמוּאֵל}, \foreignlanguage{hebrew}{שֹׁמְרוֹן}, \foreignlanguage{hebrew}{תָּבוֺר}.
	
	
	\vspace{0.5cm}
	
	% #2 was already in Chapter 4.
	
	\selectlanguage{hebrew}
	
	\noindent
	1~~\foreignlanguage{hebrew}{וַיִּכְרְתוּ בְרִית בִּבְאֵר שָׁ֫בַע}\LTRfootnote{\space \foreignlanguage{hebrew}{בְּאֵר שָׁ֫בַע} pausal form of the place name \foreignlanguage{hebrew}{בְּאֵר שֶׁבַע}}  \hspace{0.3cm}
	2~~\foreignlanguage{hebrew}{וַיֹּאמֶר שְׁמוּאֵל אֶל־יִשַׁי לֹא־בָחַר יְהוָה}\LTRfootnote{\space \foreignlanguage{hebrew}{יהוה} read as \foreignlanguage{hebrew}{אֲדֹנָי} (Engl. \textit{Lord})} \foreignlanguage{hebrew}{בָּאֵלֶּה} \hspace{0.3cm}
	3~~\foreignlanguage{hebrew}{וַיִּבְחַר מֹשֶׁה אַנְשֵׁי־חַיִל מִכָּל־יִשְׂרָאֵל}  \hspace{0.3cm}
	4~~\foreignlanguage{hebrew}{וַיֹּאמֶר אֶל־זֶבַח וְאֶל־צַלְמֻנָּע אֵיפֹה}\LTRfootnote{\space \foreignlanguage{hebrew}{אֵיפֹה} \textit{where?}} \foreignlanguage{hebrew}{הָאֲנָשִׁים אֲשֶׁר הֲרַגְתֶּם בְּתָבוֹר} \hspace{0.3cm}
	5~~\foreignlanguage{hebrew}{וַיַּגֵּד}\LTRfootnote{\space \foreignlanguage{hebrew}{וַיַּגֵּד} \textit{and (he) told}} \foreignlanguage{hebrew}{אֶבְיָתָר לְדָוִד כִּי הָרַג שָׁאוּל אֵת כֹּהֲנֵי יְהוָה}\LTRfootnote{\space \foreignlanguage{hebrew}{יהוה} read as \foreignlanguage{hebrew}{אֲדֹנָי} (Engl. \textit{Lord})}  \hspace{0.3cm}
	6~~\foreignlanguage{hebrew}{וַיַּגֵּד}\LTRfootnote{\space \foreignlanguage{hebrew}{וַיַּגֵּד} \textit{and (he) told}} \foreignlanguage{hebrew}{אַחְאָב לְאִיזֶבֶל אֵת כָּל־אֲשֶׁר עָשָׂה אֵלִיָּהוּ וְאֵת כָּל־אֲשֶׁר הָרַג אֶת־כָּל־הַנְּבִיאִים בֶּחָ֫רֶב}\LTRfootnote{\space \foreignlanguage{hebrew}{חָ֫רֶב} pausal form of \foreignlanguage{hebrew}{חֶרֶב}}  \hspace{0.3cm}
	7~~\foreignlanguage{hebrew}{וַיַּהֲרֹג אֶת־אַנְשֵׁי הָעִיר}  \hspace{0.3cm}
	8~~\foreignlanguage{hebrew}{וַיִּזְבַּח אֶת־כָּל־כֹּהֲנֵי הַבָּמוֹת אֲשֶׁר־שָׁם עַל־הַמִּזְבְּחוֹת}  \hspace{0.3cm}
	9~~\foreignlanguage{hebrew}{לֹא־נָפַל דָּבָר מִכֹּל הַדָּבָר הַטּוֹב אֲשֶׁר־דִּבֶּר יְהוָה}\LTRfootnote{\space \foreignlanguage{hebrew}{יהוה} read as \foreignlanguage{hebrew}{אֲדֹנָי} (Engl. \textit{Lord})} \foreignlanguage{hebrew}{אֶל־בֵּית יִשְׂרָאֵל הַכֹּל בָּא} \hspace{0.3cm}
	10~~\foreignlanguage{hebrew}{וַיִּזְכֹּר אֱלֹהִים אֶת־אַבְרָהָם}  \hspace{0.3cm}
	11~~\foreignlanguage{hebrew}{וַיִּזְכֹּר יוֹסֵף אֵת הַחֲלֹמוֹת אֲשֶׁר חָלַם לָהֶם}\LTRfootnote{\space \foreignlanguage{hebrew}{חָלַם לָהֶם} \textit{he had dreamed regarding them}}  \hspace{0.3cm}
	12~~\foreignlanguage{hebrew}{וַיַּעַזְבוּ אֶת־יְהוָה}\LTRfootnote{\space \foreignlanguage{hebrew}{יהוה} read as \foreignlanguage{hebrew}{אֲדֹנָי} (Engl. \textit{Lord})}  \hspace{0.3cm}
	13~~\foreignlanguage{hebrew}{וַיֹּאמֶר יְהוֹשֻׁעַ פִּתְחוּ אֶת־פִּי הַמְּעָרָה}\LTRfootnote{\space \foreignlanguage{hebrew}{מְעָרָה} \textit{cave}} \hspace{0.3cm}
	14~~\foreignlanguage{hebrew}{וַיִּקְבְּרוּ אֶת־הַמֶּלֶךְ בְּשֹׁמְרוֹן}  \hspace{0.3cm}
	15~~\foreignlanguage{hebrew}{וַיֹּאמֶר הַכֹּהֵן נִקְרְבָה הֲלֹם}\LTRfootnote{\space \foreignlanguage{hebrew}{הֲלֹם}  \textit{here}} \foreignlanguage{hebrew}{אֶל־הָאֱלֹהִים}  \hspace{0.3cm}
	16~~\foreignlanguage{hebrew}{וַיִּקְרַב אַהֲרֹן אֶל־הַמִּזְבֵּחַ וַיִּשְׁחַט אֶת־עֵגֶל}\LTRfootnote{\space \foreignlanguage{hebrew}{עֵגֶל} \textit{young bull, ox}} \foreignlanguage{hebrew}{הַחַטָּאת אֲשֶׁר־לוֹ}\LTRfootnote{\space \foreignlanguage{hebrew}{לוֹ} lit. \textit{to him}, i.e., \textit{belonged to him}} \hspace{0.3cm}
	17~~\foreignlanguage{hebrew}{וַיִּשְׂמַח שָׁם שָׁאוּל וְכָל־אַנְשֵׁי יִשְׂרָאֵל עַד־מְאֹד}  \hspace{0.3cm}
	18~~\foreignlanguage{hebrew}{וַיַּעַשׂ שְׁמוּאֵל אֵת אֲשֶׁר דִּבֶּר יְהוָה}\LTRfootnote{\space \foreignlanguage{hebrew}{יהוה} read as \foreignlanguage{hebrew}{אֲדֹנָי} (Engl. \textit{Lord})} \foreignlanguage{hebrew}{וַיָּבֹא בֵּית לָ֫חֶם}\LTRfootnote{\space \foreignlanguage{hebrew}{בֵּית לָחֶם} pausal form of \foreignlanguage{hebrew}{בֵּית לֶחֶם}}  \hspace{0.3cm}
	19~~\foreignlanguage{hebrew}{וַיִּקַּח אֲבִימֶלֶךְ צֹאן וּבָקָר וַעֲבָדִים וּשְׁפָחֹת וַיִּתֵּן לְאַבְרָהָם}  \hspace{0.3cm}
	
	\selectlanguage{english}
	
	
	\chapter{Chapter 8}
	
	\renewcommand\arraystretch{1.4}
	
	\section{Vocabulary}
	
	\subsection{Verbs}
	
	\begin{center}
		
		% For the centering of the separation between the two columns see the documentation of the array package, page 2 
		
		\begin{tabular}{>{\raggedleft}p{0.175\linewidth} p{0.75\linewidth}}
			\foreignlanguage{hebrew}{בטח} & Q.\ \textit{to trust, be confident} \\
			\foreignlanguage{hebrew}{בער} & Q.\ \textit{to burn} (intransitive) \\
			\foreignlanguage{hebrew}{ברך} & Q.\ \textit{to bless} (in the Qal only part.\ pass. \foreignlanguage{hebrew}{בָּרוּךְ} \textit{blessed}) \\
			\foreignlanguage{hebrew}{גאל} & Q.\ \textit{to redeem} \\
			\foreignlanguage{hebrew}{גדל} & Q.\ \textit{to be great, become great, grow up} (stative verb) \\
			\foreignlanguage{hebrew}{חשׁב} & Q.\ \textit{to think, consider, regard, plan, devise} \\
			\foreignlanguage{hebrew}{לכד} & Q.\ \textit{to capture, seize, take} \\
			\foreignlanguage{hebrew}{עמד} & Q.\ \textit{to take one's stand, stand} \\
			\foreignlanguage{hebrew}{פרשׂ} & Q.\ \textit{to spread out, spread} \\
			\foreignlanguage{hebrew}{קבץ} & Q.\ \textit{to gather, collect} \\
			\foreignlanguage{hebrew}{רחץ} & Q.\ \textit{to wash, bathe} \\
			\foreignlanguage{hebrew}{שׂרף} & Q.\ \textit{to burn completely} (transitive) \\
			\foreignlanguage{hebrew}{שׁכן} & Q.\ \textit{to settle down, dwell} \\
		\end{tabular}
	\end{center}
	
	\subsection{Nouns}
	
	\begin{center}
		\begin{longtable}{>{\raggedleft}p{0.175\linewidth} p{0.75\linewidth}}
			\foreignlanguage{hebrew}{אֹיֵב} & \textit{enemy} \\
			\foreignlanguage{hebrew}{אֶרֶז} & \textit{cedar} \\
			\foreignlanguage{hebrew}{גָּמָל} & \textit{camel} (pl. \foreignlanguage{hebrew}{גְּמַלִּים}) \\
			\foreignlanguage{hebrew}{דֶּלֶת} & \textit{door} \\
			\foreignlanguage{hebrew}{חָכְמָה} & \textit{wisdom} (\textit{ḥɔḵmā}) \\
			\foreignlanguage{hebrew}{חֵן} & \textit{favor, grace} (gem.\ noun; with ePP \foreignlanguage{hebrew}{חִנּוֺ}) \\
			\foreignlanguage{hebrew}{חֹשֶׁךְ} & \textit{darkness} \\
			\foreignlanguage{hebrew}{כִּכָּר} & \textit{round loaf, talent, vicinity} (esp. the region of the Jordan) (fem.) \\
			\foreignlanguage{hebrew}{כַּף} & \textit{the hollow, flat of the hand, hand, sole (of the foot)} (gem.\ noun; dual \foreignlanguage{hebrew}{כַּפַּיִם}) \\
			\foreignlanguage{hebrew}{מַרְאֶה} & \textit{sight, appearance, vision} \\ % BDB
			\foreignlanguage{hebrew}{סוּס} & \textit{horse} \\
			\foreignlanguage{hebrew}{רֶכֶב} & \textit{chariotry, chariots} (collective), \textit{chariot} (rare meaning) \\
		\end{longtable}
	\end{center}
	
	\subsection{Prepositions}
	
	\begin{center}
		\begin{longtable}{>{\raggedleft}p{0.175\linewidth} p{0.75\linewidth}}
			\foreignlanguage{hebrew}{אַחֲרֵי} & \textit{after, behind} \\
			\foreignlanguage{hebrew}{לִפְנֵי} & \textit{in front of, before} (local and temporal) (prep. \foreignlanguage{hebrew}{לְ} + \foreignlanguage{hebrew}{פָּנִים}) \\
			\foreignlanguage{hebrew}{נֶגֶד} & \textit{in front of, in sight of, opposite to} \\
			\foreignlanguage{hebrew}{עַד} & \textit{up to, as far as, until} \\
			\foreignlanguage{hebrew}{תַּחַת} & \textit{under, beneath, instead of, in place of} \\
		\end{longtable}
	\end{center}
	
	\subsection{Other Parts of Speech}
	
	\begin{center}
		\begin{tabular}{>{\raggedleft}p{0.175\linewidth} p{0.75\linewidth}}
			\foreignlanguage{hebrew}{לֵאמֹר} & introduction of direct speech (\textit{lē(ʾ)mōr}) \\
			\foreignlanguage{hebrew}{תָּמִיד} & \textit{continuously, always} (adv.), \textit{continuity} (noun) \\
		\end{tabular}
	\end{center}
	
	\vspace{0.5cm}
	
	\section{Nouns with Enclitic Pronouns}
	
	Nouns may be combined with enclitic personal pronouns (ePP) to express the idea of possession. They are attached directly to the noun which may then undergo changes in its form (see below), e.g., \foreignlanguage{hebrew}{בֵּיתִי} \textit{my house}. Most forms of the enclitic personal pronouns show formal similarities with the corresponding independent personal pronouns.
	
	\begin{center}
		\begin{longtable}{llllrrl}
			& & & & ePP & iPP & \\
			\endhead
			\endfoot
			sg. & 1 & c. & & \foreignlanguage{hebrew}{ִ י}\space- & \foreignlanguage{hebrew}{אֲנִי} & \\
			& \multirow{2}{*}{2} & m. & \textit{*-ka} > & \foreignlanguage{hebrew}{ךָ}\space- & \foreignlanguage{hebrew}{אַתָּה} & \\
			& & f. & \textit{*-ki} > & \foreignlanguage{hebrew}{ךְ}\space- & \foreignlanguage{hebrew}{אַתְּ} & < \textit{*ʾatti} \\
			& \multirow{2}{*}{3} & m. & \textit{*-ahu} > & \foreignlanguage{hebrew}{וֺ}\space- & \foreignlanguage{hebrew}{הוּא} & \\
			& & f. & & \foreignlanguage{hebrew}{ָ ה}\space- & \foreignlanguage{hebrew}{הִיא} & \\
			\newpage
			pl. & 1 & c. & & \foreignlanguage{hebrew}{נוּ}\space- & \foreignlanguage{hebrew}{אֲנֲחְנוּ} & \\
			& \multirow{2}{*}{2} & m. & & \foreignlanguage{hebrew}{כֶם}\space- & \foreignlanguage{hebrew}{אַתֶּם} & \\
			& & f. & & \foreignlanguage{hebrew}{כֶן}\space- & \foreignlanguage{hebrew}{אַתֶּן} & \\
			& \multirow{2}{*}{3} & m. & & \foreignlanguage{hebrew}{ָ ם}\space- & \foreignlanguage{hebrew}{הֵם} & \\
			& & f. & & \foreignlanguage{hebrew}{ָ ן}\space- & \foreignlanguage{hebrew}{הֵנָּה} & \\
		\end{longtable}
	\end{center}
	
	Enclitic personal may also express other ideas than possession like the object of an implied action, e.g., \foreignlanguage{hebrew}{חֲמָסִי} \textit{the wrong done to me} (Gen 16:5).
	
	\subsection{Singular Nouns with Enclitic Personal Pronoun}
	
	With singular nouns, the connecting vowels /ā/ and /ē/ are used for the forms with the 2 f.\ sg.\, 3 f.\ sg.\, 1 c.\ pl.\, 3 m./f.\ pl.\  enclitic pronouns. 
	
	\begin{center}
		\begin{longtable}{|lll|rl|rl|}
			\hline
			& & & \multicolumn{2}{c|}{masc.} & \multicolumn{2}{c|}{fem.} \\
			\hline
			\endhead
			\hline
			\endfoot
			1 & c. & sg. & \foreignlanguage{hebrew}{סוּסִי} & \textit{my horse} & \foreignlanguage{hebrew}{סוּסָתִי} & \textit{my mare} \\
			\hline
			\multirow{2}{*}{2} & m. & sg. & \foreignlanguage{hebrew}{סוּסְךָ} & \textit{your horse} & \foreignlanguage{hebrew}{סוּסָתְךָ} & \textit{your mare} \\
			& f. & sg. & \foreignlanguage{hebrew}{סוּסֵךְ} & \textit{your horse} & \foreignlanguage{hebrew}{סוּסָתֵךְ} & \textit{your mare} \\
			\hline
			\multirow{2}{*}{3} & m. & sg. & \foreignlanguage{hebrew}{סוּסוֺ} & \textit{his horse} & \foreignlanguage{hebrew}{סוּסָתוֺ} & \textit{his mare} \\
			& f. & sg. & \foreignlanguage{hebrew}{סוּסָהּ} & \textit{her horse} & \foreignlanguage{hebrew}{סוּסָתָהּ} & \textit{her mare} \\
			\hline
			1 & c. & pl. & \foreignlanguage{hebrew}{סוּסֵ֫נוּ} & \textit{our horse} & \foreignlanguage{hebrew}{סוּסָתֵ֫נוּ} & \textit{our mare} \\
			\hline
			\multirow{2}{*}{2} & m. & pl. & \foreignlanguage{hebrew}{סוּסְכֶם} & \textit{your horse} & \foreignlanguage{hebrew}{סוּסַתְכֶם} & \textit{your mare} \\
			& f. & pl. & \foreignlanguage{hebrew}{סוּסְכֶן} & \textit{your horse} & \foreignlanguage{hebrew}{סוּסַתְכֶן} & \textit{your mare} \\
			\hline
			\multirow{2}{*}{3} & m. & pl. & \foreignlanguage{hebrew}{סוּסָם} & \textit{their horse} & \foreignlanguage{hebrew}{סוּסָתָם} & \textit{their mare} \\
			& f. & pl. & \foreignlanguage{hebrew}{סוּסָן} & \textit{their horse} & \foreignlanguage{hebrew}{סוּסָתָן} & \textit{their mare}  \\
		\end{longtable}	
	\end{center}
	
	\noindent \textbf{Notes about the Form of Enclitic Personal Pronouns on Singular Nouns}
	
	\begin{enumerate}[noitemsep]
		\item Distinct pausal forms are only found with the 2 m.\ sg.\ enclitic pronoun with singular nouns (\foreignlanguage{hebrew}{סוּסֶ֑ךָ} and \foreignlanguage{hebrew}{סוּסָתֶ֑ךָ})  and the 1 c.\ sg.\ enclitic pronoun with plural nouns (\foreignlanguage{hebrew}{סוּסָ֑י}).
		\item The 3 m.\ sg.\ enclitic personal pronoun was originally \textit{-hū}. Together with the connecting vowel /a/ the suffix developed on singular nouns from \textit{--áhū} via elision (loss) of the intervocalic /h/ and contraction of the diphthong to its current form: \textit{-áhū} > \textit{-aw} > \textit{-ō}. The spelling of the 3 m.\ sg.\ enclitic personal pronoun in Hebrew inscriptions \foreignlanguage{hebrew}{ה}– reflects the older form with /h/ before it elision, e.g., \foreignlanguage{hebrew}{בדה}[\foreignlanguage{hebrew}{ע}] \textit{his servant} (Lachish 2:5). The same spelling is sometimes found in the Hebrew Bible, e.g., the Ketiv \foreignlanguage{hebrew}{אהלה} with the Qere \foreignlanguage{hebrew}{אָהֳלוֺ} in Gen 12:8.
		\item With nouns ending in \textit{segol-he} (e.g., \foreignlanguage{hebrew}{שָׂדֶה}) the enclitic pronoun 3 masc.\ sg.\ is \textit{-ēhū}, e.g., \foreignlanguage{hebrew}{שָׂדֵהוּ} \textit{his field}.
		\item With the nouns \foreignlanguage{hebrew}{אָב} (cs.\ st.\ \foreignlanguage{hebrew}{אֲבִי}), \foreignlanguage{hebrew}{אָח} (cs.\ st.\ \foreignlanguage{hebrew}{אֲחִי}), \foreignlanguage{hebrew}{פֶּה} (cs.\ st.\ \foreignlanguage{hebrew}{פִּי}) the 3 m.\ sg.\ enclitic pronoun is either \textit{-hū} or \textit{-w}, e.g., \foreignlanguage{hebrew}{אָבִ֫יהוּ} or \foreignlanguage{hebrew}{אָבִיו} \textit{ʾāḇīw}, \foreignlanguage{hebrew}{אָחִ֫יהוּ} or \foreignlanguage{hebrew}{אָחִיו} \textit{ʾāḥīw}, \foreignlanguage{hebrew}{פִּ֫יהוּ} or \foreignlanguage{hebrew}{פִּיו} \textit{pīw}.
	\end{enumerate}
	
	\subsection{Plural Nouns with Enclitic Personal Pronoun}
	
	With plural nouns with the plural ending \textit{-īm}, the enclitic pronouns are attached to the original form of cs.\ st.\ with subsequent vowel contraction in all forms except the forms with 1 c.\ sg.\ and 2 f.\ sg.\ enclitic pronoun.
	
	\begin{center}
		\begin{longtable}{|lll|rl|rl|}
			\hline
			& & & \multicolumn{2}{c|}{masc.} & \multicolumn{2}{c|}{fem.} \\
			\hline
			1 & c. & sg. & \foreignlanguage{hebrew}{סוּסַי} & \textit{my horses} & \foreignlanguage{hebrew}{סוּסוֺתַי} & \textit{my mares} \\
			\hline
			\multirow{2}{*}{2} & m. & sg. & \foreignlanguage{hebrew}{סוּסֶ֫יךָ} & \textit{your horses} & \foreignlanguage{hebrew}{סוּסוֺתֶ֫יךָ} & \textit{your mares} \\
			& f. & sg. & \foreignlanguage{hebrew}{סוּסַ֫יִךְ} & \textit{your horses} & \foreignlanguage{hebrew}{סוּסוֺתַ֫יִךְ} & \textit{your mares} \\
			\hline
			\multirow{2}{*}{3} & m. & sg. & \foreignlanguage{hebrew}{סוּסָיו} & \textit{his horses} & \foreignlanguage{hebrew}{סוּסוֺתָיו} & \textit{his mares} \\
			& f. & sg. & \foreignlanguage{hebrew}{סוּסֶ֫יהָ} & \textit{her horses} & \foreignlanguage{hebrew}{סוּסוֺתֶ֫יהָ} & \textit{her mares} \\
			\hline
			1 & c. & pl. & \foreignlanguage{hebrew}{סוּסֵ֫ינוּ} & \textit{our horses} & \foreignlanguage{hebrew}{סוּסוֺתֵ֫ינוּ} & \textit{our mares} \\
			\hline
			\multirow{2}{*}{2} & m. & pl. & \foreignlanguage{hebrew}{סוּסֵיכֶם} & \textit{your horses} & \foreignlanguage{hebrew}{סוּסוֺתֵיכֶם} & \textit{your mares} \\
			& f. & pl. & \foreignlanguage{hebrew}{סוּסֵיכֶן} & \textit{your horses} & \foreignlanguage{hebrew}{סוּסוֺתֵיכֶן} & \textit{your mares} \\
			\hline
			\multirow{2}{*}{3} & m. & pl. & \foreignlanguage{hebrew}{סוּסֵיהֶם} & \textit{their horses} &  \foreignlanguage{hebrew}{סוּסוֺתֵיהֶם} & \textit{their mares} \\
			& f. & pl. & \foreignlanguage{hebrew}{סוּסֵיהֶן} & \textit{their horses} & \foreignlanguage{hebrew}{סוּסוֺתֵיהֶן} & \textit{their mares} \\
			\hline
		\end{longtable}	
	\end{center}
	
	\noindent \textbf{Notes about the Form of Enclitic Personal Pronouns on Plural Nouns}
	
	\begin{enumerate}[noitemsep]
		\item The 3 m.\ sg.\ enclitic personal pronoun with plural nouns is pronounced \textit{-āw} (the \foreignlanguage{hebrew}{י} is \emph{not} pronounced), e.g., \foreignlanguage{hebrew}{יָמָיו} \textit{yāmāw} \textit{his days}. This is often reflected in the phonetic spelling without \foreignlanguage{hebrew}{י} in the Ketiv, e.g., \foreignlanguage{hebrew}{ואנשו} \textit{and his men} (with the Qere \foreignlanguage{hebrew}{וַאֲנָשָׁיו}) (1\,Sam 23:4). This phonetic spelling is already attested in a Hebrew inscription from the early 6th century BCE: \foreignlanguage{hebrew}{אנשו} \textit{his men} (Lachish 3:18).
		\item With plural nouns with the ending \textit{-ōt} the 3 m.\ pl.\ suffix may be \textit{-ām}, especially with the noun \foreignlanguage{hebrew}{אָב}, e.g., \foreignlanguage{hebrew}{אֲבוֹתָם} \textit{their fathers} (the form \foreignlanguage{hebrew}{אֲבוֹתֵיהֶם} is only attested in Late Biblical Hebrew), \foreignlanguage{hebrew}{שְׁמוֹתָם} \textit{their names}. With the 3 f.\ pl.\ suffix the form \foreignlanguage{hebrew}{שְׁמוֹתָן} \textit{their names} is attested.
	\end{enumerate}
	
	\subsection{Changes of Nouns with Enclitic Personal Pronouns}
	Enclitic pronouns are attached to the cs.\ st.\ of nouns. This can be clearly seen with feminine nouns with feminine ending.
	
	\begin{center}
		\begin{tabular}{rrrl}
			\multicolumn{1}{c}{abs.\ st.} & \multicolumn{1}{c}{cs.\ st.} & \multicolumn{1}{c}{with sfx.} & \\
			\foreignlanguage{hebrew}{שִׁפְחָה} & \foreignlanguage{hebrew}{שִׁפְחַת} & \foreignlanguage{hebrew}{שִׁפְחָתָהּ} & \textit{her maidservant} \\
			\foreignlanguage{hebrew}{אִשָּׁה} & \foreignlanguage{hebrew}{אֵשֶׁת} & \foreignlanguage{hebrew}{אִשְׁתִּי} & \textit{my wife} \\
		\end{tabular}
	\end{center}
	
	Vowel changes may happen when an enclitic pronoun is attached, e.g., \foreignlanguage{hebrew}{דְּבָרַי} \textit{my words}. For details see Chapter 4.
	
	Enclitic pronouns on nouns ending in \textit{segol-he} are attached to the last root consonant of the noun, \foreignlanguage{hebrew}{שָׂדְךָ} \textit{your field}, \foreignlanguage{hebrew}{שָׂדָהּ} \textit{her field}, \foreignlanguage{hebrew}{מַחֲנֵ֫הוּ} \textit{his army}.
	
	With geminate nouns, the gemination is restored when enclitic pronouns are attached, e.g., \foreignlanguage{hebrew}{עַמּוֺ} \textit{his people}.
	
	For irregular nouns with enclitic pronouns see Chapter 5.
	
	
	\section{The Participle}
	
	\subsection{The Forms of the Participle Active Qal}
	
	The participle in Hebrew is a verbal noun (or verbal adjective) which is inflected for gender and number like the adjective. The participle active Qal has the following forms:
	
	\renewcommand\arraystretch{1.4}
	
	\begin{center}
		\begin{tabular}{|l|r|r|}
			\hline
			& Masculine & Feminine \\
			\hline
			Singular & \foreignlanguage{hebrew}{כֹּתֵב} & \foreignlanguage{hebrew}{כֹּתֶ֫בֶת} \\
			Plural & \foreignlanguage{hebrew}{כֹּתְבִים} & \foreignlanguage{hebrew}{כֹּתְבוֺת} \\
			\hline
		\end{tabular}
	\end{center}
	
	\noindent \textbf{Notes}
	\nopagebreak
	
	\begin{enumerate}[noitemsep]
		\item In the Masoretic Text, the plene spelling of the participle \foreignlanguage{hebrew}{כּוֺתֵב} is much less frequent than the defective spelling \foreignlanguage{hebrew}{כֹּתֵב}.
		\item The vowel /ō/ is unchangeable. This vowel developed via the Canaanite sound change \textit{*kātib > kōtēḇ}.
		\item The f.\ sg.\ form has the alternative form \foreignlanguage{hebrew}{כֹּתְבָה}.
		\item Verbs with gutturals show the expected vowel changes: Verbs II gutt. \foreignlanguage{hebrew}{בֹּחֲרִים}, etc.; verbs III gutt. \foreignlanguage{hebrew}{שֹׁלֵחַ}, \foreignlanguage{hebrew}{שֹׁלַ֫חַת}.
		\item Stative verbs form a participle with the vowel pattern \foreignlanguage{hebrew}{כָּבֵד} with the forms \foreignlanguage{hebrew}{כָּבֵד} (masc.\ sg.), \foreignlanguage{hebrew}{כְּבֵדִים} (masc.\ pl.), \foreignlanguage{hebrew}{כְּבֵדָה} (fem.\ sg.), \foreignlanguage{hebrew}{כְּבֵדוֺת} (fem.\ pl.).
	\end{enumerate}
	
	\subsection{The Forms of the Participle Passive Qal}
	
	\begin{center}
		\begin{tabular}{|l|r|r|}
			\hline
			& Masculine & Feminine \\
			\hline
			Singular & \foreignlanguage{hebrew}{כָּתוּב} & \foreignlanguage{hebrew}{כְּתוּבָה} \\
			Plural & \foreignlanguage{hebrew}{כְּתוּבִים} & \foreignlanguage{hebrew}{כְּתוּבוֺת} \\
			\hline
		\end{tabular}
	\end{center}
	
	\noindent \textbf{Note}
	\nopagebreak
	
	\noindent In forms with an ending, the long vowel /ū/ is frequently spelled defectively, e.g., \foreignlanguage{hebrew}{כְּתֻבִים} \textit{written, inscribed} (Deut 9:10).
	
	
	\subsection{The Use of the Participle}
	Like the adjective, the participle can be used as attribute, as predicate or substantivized. The same rules for the position of the attribute and the agreement between the governing noun and the attribute apply as for the adjective (cf.\ 7.4.3).
	
	The participle active is basically timeless and indicates progressive (on-going) or -- at times -- repeated events. Normally, the participle refers to ongoing events like the English present progressive but it is also used for future events, especially events in the near future.  In a past tense context it can also refer to past events. In narratives, the participle is frequently used in circumstantial clauses (with reference to the past).
	
	The participle of the stative verb with the pattern \foreignlanguage{hebrew}{כָּבֵד} often functions like a normal adjective, e.g., \foreignlanguage{hebrew}{עַמְּךָ הַכָּבֵד הַזֶּה} \textit{this your great people}. But it can also indicate a process, e.g., the participle \foreignlanguage{hebrew}{גָּדֵל} in \foreignlanguage{hebrew}{וַיִּגְדַּל הָאִישׁ וַיֵּלֶךְ הָלוֹךְ וְגָדֵל עַד כִּֽי־גָדַל מְאֹד} \textit{ and the man become wealthier and wealthier, until he was very wealthy} (Gen 26:13).
	
	The participle passive indicates a state that is the the result of a preceding action.
	
	
	\subsubsection{Examples of the Attributive Use}
	
	In the following examples the participle is used as an attribute to a noun.
	
	\begin{longtable}{>{\raggedleft}p{0.35\linewidth} p{0.55\linewidth}}
		\foreignlanguage{hebrew}{עַם בֹּטֵחַ} & \textit{an unsuspecting people} (Judg 18:10) \\
		\foreignlanguage{hebrew}{הָעָם הַיֹּשֵׁב בָּאָרֶץ} & \textit{the people that lives in the land} (Num 13.28) \\
		\foreignlanguage{hebrew}{מִזְבַּח יְהוָה הֶהָרוּס} & \textit{the demolished altar of the Lord} (1\,Kgs 18:30) \\
	\end{longtable}
	
	\subsubsection{Examples of the Predicative Use}
	
	The predicative use of the participle is illustrated by the following examples.
	
	\begin{longtable}{>{\raggedleft}p{0.35\linewidth} p{0.55\linewidth}}
		\foreignlanguage{hebrew}{הִנֵּה־עָם יוֹרֵד מֵרָאשֵׁי הֶהָרִים} & \textit{Look, people are coming down from the mountaintops} (Judg 9:36) \\
		\foreignlanguage{hebrew}{רְאֵה נָא אָנֹכִי יוֹשֵׁב בְּבֵית אֲרָזִים} & \textit{See, I am living in a house of cedar} (2\,Sam 7:2) \\
		\foreignlanguage{hebrew}{בָּרוּךְ יְהוָה אֱלֹהֵי אֲדֹנִי אַבְרָהָם} & \textit{Blessed be the Lord, the God of my master Abraham} (Gen 24.27) \\
		\foreignlanguage{hebrew}{וַיִּרְאוּ וְהִנֵּה דַּלְתוֹת הָעֲלִיָּה נְעֻלוֹת} & \textit{And they saw that the doors of the roof-chamber were locked} (Judg 3:24) \\
		\foreignlanguage{hebrew}{רָאִיתִי אֶת־יְהוָה יֹשֵׁב עַל־כִּסְאוֹ} & \textit{I have seen the Lord sitting on his throne} (1\,Kgs 22:19) \\
	\end{longtable}
	
	
	\subsubsection{Examples of the Substantivized Particple}
	
	Examples of substantivized participles can be found in the following references.
	
	\begin{longtable}{>{\raggedleft}p{0.35\linewidth} p{0.55\linewidth}}
		\foreignlanguage{hebrew}{אָרוּר נֹתֵן אִשָּׁה לְבִנְיָמִן} & \textit{Cursed is the one who gives a wife to Benjamin} (Judg 21:18) \\
		\foreignlanguage{hebrew}{וַיַּעַל עַל־גֹּזֲזֵי צֹאנוֹ} & \textit{And he went up to those who were shearing his sheep} (Gen 38:12) \\
	\end{longtable}
	
	\subsubsection{The Active Participle and its Complements}
	As a verbal adjective, the participle also has properties of the verb in the sense that it allows certain complements to be connected with it in all its three uses (as attribute, as predicate or substantivized). These can be direct objects and/or indirect objects (Judg 21:18), prepositional objects (1\,Kgs 19:5) or local complements (Num 13:29).
	
	% Other example for a participle with DO and IO: Exod 16:29
	
	% The translation of 1 Kgs 19:5 does not sound perfect. It follows BDB.
	
	\begin{longtable}{>{\raggedleft}p{0.35\linewidth} p{0.55\linewidth}}
		\foreignlanguage{hebrew}{אָרוּר נֹתֵן אִשָּׁה לְבִנְיָמִן} & \textit{Cursed is the one who gives a wife to Benjamin} (Judg 21:18) \\
		\foreignlanguage{hebrew}{וְהִנֵּה־זֶה מַלְאָךְ נֹגֵעַ בּוֹ} & \textit{And behold, here an angel touched him} (1\,Kgs 19:5) \\
		\foreignlanguage{hebrew}{עֲמָלֵק יוֹשֵׁב בְּאֶרֶץ הַנֶּגֶב} & \textit{Amalek is living in the land of the Negev} (Num 13:29) \\
	\end{longtable}
	
	% Other examples of indirect objects: 1\,Kgs 5:24 Exod 5:10; Exod 16:19
	
	Complements can be connected to the participle in the construct state. This even applies to prepositional phrases (Isa 9:1). Adverbial adjuncts can also be expressed as dependent nouns of participles (1\,Kgs 18:19).
	
	\begin{longtable}{>{\raggedleft}p{0.35\linewidth} p{0.55\linewidth}}
		\foreignlanguage{hebrew}{גֹּזֲזֵי צֹאנוֹ} & \textit{those who were shearing his sheep} (Gen 38:12) \\
		\foreignlanguage{hebrew}{יֹשְׁבֵי הָאָרֶץ} & \textit{the inhabitants of the land} (Exod 23:31) \\
		\foreignlanguage{hebrew}{יֹצְאֵי הַתֵּבָה} & \textit{those going out from the ark} (Gen 9:10) \\
		\foreignlanguage{hebrew}{יֹשְׁבֵי בְּאֶרֶץ צַלְמָוֶת} & \textit{those living in the land of deep darkness} (Isa 9:1) \\
		\foreignlanguage{hebrew}{אֹכְלֵי שֻׁלְחַן אִיזָ֑בֶל} & \textit{those eating at the Jezebel's table} (1\,Kgs 18:19) \\
	\end{longtable}
	
	\section{Circumstantial Clauses}
	A circumstantial clause gives background information to the main story line of a narrative text which is constituted by a chain of \textit{wayyiqṭol} forms. The circumstantial clause can be a nominal clause (sometimes with the copular verb \foreignlanguage{hebrew}{היה} \textit{to be}), a clause with a participle or a clause with finite verb as predicate.
	
	Circumstantial clauses can interrupt the main story line (Gen 18:1; 2 Kings 8:7; Gen 37:24) or they can occur at the beginning of a narrative (or episode or scene) (1 Kings 13:11). The typical constituent order of circumstantial clauses is subject-predicate.
	
	\begin{longtable}{>{\raggedleft}p{0.35\linewidth} p{0.55\linewidth}}
		\foreignlanguage{hebrew}{וַיֵּרָא אֵלָיו יְהוָה בְּאֵלֹנֵי מַמְרֵא וְהוּא יֹשֵׁב פֶּתַח־הָאֹהֶל כְּחֹם הַיּוֹם} & \textit{And the Lord appeared to him at the terebinths of Mamre as he sat in the entrance of the tent in the heat of the day} (Gen 18:1) \\
		\foreignlanguage{hebrew}{וַיָּבֹא אֱלִישָׁע דַּמֶּשֶׂק וּבֶן־הֲדַד מֶלֶךְ־אֲרָם חֹלֶה} & \textit{And Elisha came to Damascus at a time when Ben-hadad, king of Syria, was sick} (2\,Kgs 8:7) \\
		\foreignlanguage{hebrew}{וַיִּקָּחֻהוּ וַיַּשְׁלִ֫כוּ אֹתוֹ הַבֹּ֫רָה וְהַבּוֹר רֵק אֵין בּוֹ מָיִם} & \textit{And they took him and threw into the cistern; the cistern was empty, there was no water in it} (Gen 37:24) \\
		\foreignlanguage{hebrew}{וְנָבִיא אֶחָד זָקֵן יֹשֵׁב בְּבֵית־אֵל וַיָּבוֹא בְנוֹ} & \textit{There was an old prophet living in Bethel. Now his son came \dots} (1\,Kgs 13:11) \\
	\end{longtable}
	
	The circumstantial clause may even interrupt the clause on the main level of the narrative. In this case circumstantial clause is a parenthesis in Hebrew. In the following example the circumstantial clause is located before the direct speech which is the direct object of the verb \foreignlanguage{hebrew}{וַיֹּאמֶר}.
	
	\vspace{0.5cm}
	
	\begin{tabular}{>{\raggedleft}p{0.35\linewidth} p{0.55\linewidth}}
		\foreignlanguage{hebrew}{וַיֹּאמֶר לִי הַמֶּלֶךְ וְהַשֵּׁגַל יוֹשֶׁבֶת אֶצְלוֹ עַד־מָתַי יִהְיֶה מַהֲלָכֲךָ} & \textit{And the kings asked me with the queen sitting next to him, \enquote{How long will your journey be?}} (Neh 2:6) \\
	\end{tabular}
	
	\vspace{0.5cm}
	
	A clause with the structure \textit{wə}-X-\textit{qatal} that refers to an anterior event can also be classified as a circumstantial clause.
	
	\vspace{0.5cm}
	
	\begin{tabular}{>{\raggedleft}p{0.35\linewidth} p{0.55\linewidth}}
		\foreignlanguage{hebrew}{וַיֹּאמֶר אַבְשָׁלוֹם וְכָל־אִישׁ יִשְׂרָאֵל טוֹבָה עֲצַת חוּשַׁי הָאַרְכִּי מֵעֲצַת אֲחִיתֹפֶל וַיהוָה צִוָּה לְהָפֵר אֶת־עֲצַת אֲחִיתֹפֶל הַטּוֹבָה} & \textit{And Absalom and everyone of Israel said, \enquote{The counsel of Hushai the Archite is better than the counsel of Ahithophel.} But the Lord had ordained to thwart the good counsel of Ahithophel} (2 Sam 17:14) \\
	\end{tabular}
	
	\section{Exercises}
	
	\subsection{Translation of Sentences}
	Translate the following sentences from the Hebrew Bible. Names of persons and geographical names in these sentences: \foreignlanguage{hebrew}{אֱלִישָׁע}, \foreignlanguage{hebrew}{גִלְעָד}, \foreignlanguage{hebrew}{גַּעַל}, \foreignlanguage{hebrew}{דָוִד}, \foreignlanguage{hebrew}{זְבֻל}, \foreignlanguage{hebrew}{יִפְתָּח}, \foreignlanguage{hebrew}{יְרוּשָׁלִַם}, \foreignlanguage{hebrew}{יִשַׁי}, \foreignlanguage{hebrew}{מִצְפָּה}, \foreignlanguage{hebrew}{נַעֲמָן}, \foreignlanguage{hebrew}{סִינַי}, \foreignlanguage{hebrew}{שָׁאוּל}, \foreignlanguage{hebrew}{שְׁלֹמֹה}, \foreignlanguage{hebrew}{שְׁמוּאֵל}
	
	\vspace{0.5cm}
	
	\selectlanguage{hebrew}
	
	\noindent
	1~~\foreignlanguage{hebrew}{בָּרוּךְ־יְהוָה אֱלֹהֵי יִשְׂרָאֵל מִן־הָעוֹלָם וְעַד הָעוֹלָם} \hspace{0.3cm}
	2~~\foreignlanguage{hebrew}{וַיִּלְכְּדוּ אֶת־הָעִיר}  \hspace{0.3cm}
	3~~\foreignlanguage{hebrew}{וַיַּרְא אִשָּׁה רֹחֶצֶת מֵעַל הַגָּג}\LTRfootnote{\space \foreignlanguage{hebrew}{גָּג} \textit{(flat) roof}} \foreignlanguage{hebrew}{וְהָאִשָּׁה טוֹבַת מַרְאֶה מְאֹד} \hspace{0.3cm}
	4~~\foreignlanguage{hebrew}{וְאַל־תַּעֲמֹד בְּכָל־הַכִּכָּר} \hspace{0.3cm}
	5~~\foreignlanguage{hebrew}{וַיָּבֹא אֶל־הָאִישׁ וְהִנֵּה עֹמֵד עַל־הַגְּמַלִּים עַל־הָעָ֑יִן}\LTRfootnote{\space \foreignlanguage{hebrew}{עָ֑יִן} pausal form of \foreignlanguage{hebrew}{עַיִן} \textit{eye, spring}}  \hspace{0.3cm}
	6~~\foreignlanguage{hebrew}{וַתִּקְרְבוּן וַתַּעַמְדוּן תַּחַת הָהָר וְהָהָר בֹּעֵר בָּאֵשׁ עַד־לֵב הַשָּׁמַיִם חֹשֶׁךְ עָנָן וַעֲרָפֶל}\LTRfootnote{\space \foreignlanguage{hebrew}{עֲרָפֶל} \textit{thick darkness}} \hspace{0.3cm}
	7~~\foreignlanguage{hebrew}{וְאַתֶּם אַל־תַּעֲמֹדוּ רִדְפוּ אַחֲרֵי אֹיְבֵיכֶם} \hspace{0.3cm}
	8~~\foreignlanguage{hebrew}{וַיַּעֲמֹד שְׁלֹמֹה לִפְנֵי מִזְבַּח יְהוָה נֶגֶד כָּל־קְהַל יִשְׂרָאֵל וַיִּפְרֹשׂ כַּפָּיו הַשָּׁמָ֑יִם} \hspace{0.3cm}
	9~~\foreignlanguage{hebrew}{בָּרוּךְ אַתָּה בְּנִי דָוִד} \hspace{0.3cm}
	10~~\foreignlanguage{hebrew}{וַיִּשְׁכַּב שְׁמוּאֵל עַד־הַבֹּקֶר וַיִּפְתַּח אֶת־דַּלְתוֹת בֵּית־יְהוָה} \hspace{0.3cm}
	11~~\foreignlanguage{hebrew}{וַיִּשְׁלַח שָׁאוּל אֶל־יִשַׁי לֵאמֹר יַעֲמָד־נָא דָוִד לְפָנַי כִּי־מָצָא}\LTRfootnote{\space \foreignlanguage{hebrew}{מָצָא} \textit{he found}} \foreignlanguage{hebrew}{חֵן בְּעֵינָ֑י}  \hspace{0.3cm}
	12~~\foreignlanguage{hebrew}{וַיִּקַץ}\LTRfootnote{\space \foreignlanguage{hebrew}{וַיִּקַץ} \textit{and he woke up}} \foreignlanguage{hebrew}{שְׁלֹמֹה וְהִנֵּה חֲלוֹם וַיָּבוֹא יְרוּשָׁלִַם}\LTRfootnote{\space \foreignlanguage{hebrew}{יְרוּשָׁלִַם} read as \textit{yərūšāláyim}; cf.\ \foreignlanguage{hebrew}{ירושלים} IQIsa\textsuperscript{a} 30:15 (Isa 37:10)} \foreignlanguage{hebrew}{וַיַּעֲמֹד לִפְנֵי אֲרוֹן בְּרִית־אֲדֹנָי}\LTRfootnote{\space \foreignlanguage{hebrew}{אֲדֹנָי} \textit{the Lord} (literally \textit{my Lord}; surrogate word for the name of God)} \hspace{0.3cm}
	13~~\foreignlanguage{hebrew}{אַשְׁרֵי}\LTRfootnote{\space \foreignlanguage{hebrew}{אַשְׁרֵי} \textit{O the blessedness/happiness of \dots!}} \foreignlanguage{hebrew}{אֲנָשֶׁיךָ אַשְׁרֵי עֲבָדֶיךָ אֵלֶּה}\LTRfootnote{\space A demonstrative pronoun modifying a noun with enclitic pronoun does not get the article.} \foreignlanguage{hebrew}{הָעֹמְדִים לְפָנֶיךָ תָּמִיד הַשֹּׁמְעִים אֶת־חָכְמָתֶ֑ךָ} \hspace{0.3cm}
	14~~\foreignlanguage{hebrew}{וַיָּבֹא נַעֲמָן בְּסוּסָיו וּבְרִכְבּוֹ וַיַּעֲמֹד פֶּתַח־הַבַּיִת לֶאֱלִישָׁע} \hspace{0.3cm}
	15~~\foreignlanguage{hebrew}{וַיִּקְבֹּץ יִפְתָּח אֶת־כָּל־אַנְשֵׁי גִלְעָד} \hspace{0.3cm}
	16~~\foreignlanguage{hebrew}{וַיֹּאמֶר שְׁמוּאֵל קִבְצוּ אֶת־כָּל־יִשְׂרָאֵל הַמִּצְפָּ֫תָה} \hspace{0.3cm}
	17~~\foreignlanguage{hebrew}{וַיָּבֹא דָוִד וַאֲנָשָׁיו אֶל־הָעִיר וְהִנֵּה שְׂרוּפָה בָּאֵשׁ}  \hspace{0.3cm}
	18~~\foreignlanguage{hebrew}{וַיִּשְׁכֹּן כְּבוֹד־יְהוָה עַל־הַר סִינַי}  \hspace{0.3cm}
	19~~\foreignlanguage{hebrew}{וַיַּרְא־גַּעַל אֶת־הָעָם וַיֹּאמֶר אֶל־זְבֻל הִנֵּה־עָם יוֹרֵד מֵרָאשֵׁי הֶהָרִים} \hspace{0.3cm}
	20~~\foreignlanguage{hebrew}{רְאֵה נָא אָנֹכִי יוֹשֵׁב בְּבֵית אֲרָזִים וַאֲרוֹן הָאֱלֹהִים יֹשֵׁב בְּתוֹךְ הַיְרִיעָה}\LTRfootnote{\space \foreignlanguage{hebrew}{יְרִיעָה} \textit{tent curtain}} \hspace{0.3cm}
	21~~\foreignlanguage{hebrew}{בָּרוּךְ אַתָּה בָּעִיר וּבָרוּךְ אַתָּה בַּשָּׂדֶה}  \hspace{0.3cm}
	22~~\foreignlanguage{hebrew}{בָּרוּךְ־יְהוָה אֱלֹהֵי יִשְׂרָאֵל מִן־הָעוֹלָם וְעַד הָעוֹלָם} \hspace{0.3cm}
	
	\selectlanguage{english}
	
	
	\chapter{Chapter 9}
	
	\renewcommand\arraystretch{1.4}
	
	\section{Vocabulary}
	
	\subsection{Verbs}
	
	\begin{center}
		
		% For the centering of the separation between the two columns see the documentation of the array package, page 2 
		% The verb שחט was already included in another chapter
		
		\begin{tabular}{>{\raggedleft}p{0.175\linewidth} p{0.75\linewidth}}
			\foreignlanguage{hebrew}{ברח} & Q.\ \textit{to run away, flee} \\
			\foreignlanguage{hebrew}{דרך} & Q.\ \textit{to tread} (cf.\ \foreignlanguage{hebrew}{דֶּרֶךְ} \textit{way}), with \foreignlanguage{hebrew}{קֶשֶׁת} \textit{to bend the bow} \\
			\foreignlanguage{hebrew}{זעק} & Q.\ \textit{to cry, cry out, call} \\
			\foreignlanguage{hebrew}{עבר} & Q.\ \textit{to pass over, through, by, pass on} \\
			\foreignlanguage{hebrew}{צעק} & Q.\ \textit{to cry, cry out, call} \\
			\foreignlanguage{hebrew}{קרע} & Q.\ \textit{to tear} \\
			\foreignlanguage{hebrew}{שׁחט} & Q.\ \textit{to slaughter} \\
			\foreignlanguage{hebrew}{תפשׂ} & Q.\ \textit{to lay hold of, seize} \\
		\end{tabular}
	\end{center}
	
	\subsection{Nouns}
	
	\begin{center}
		\begin{longtable}{>{\raggedleft}p{0.175\linewidth} p{0.75\linewidth}}
			\foreignlanguage{hebrew}{בֶּגֶד} & \textit{garment, clothing} \\
			\foreignlanguage{hebrew}{בַּעַל} & \textit{owner, lord, Baal} \\
			\foreignlanguage{hebrew}{זָקֵן} & \textit{old, old man} (\foreignlanguage{hebrew}{הַזְּקֵנִים} \textit{the elders}) \\
			\foreignlanguage{hebrew}{חֹק} & \textit{statute} (pl. \foreignlanguage{hebrew}{חֻקִּים}) \\
			\foreignlanguage{hebrew}{יָמִין} & \textit{right side, right hand, south} \\
			\foreignlanguage{hebrew}{כִּסֵּא} & \textit{seat, chair, throne} \\
			\foreignlanguage{hebrew}{מָגֵן} & \textit{shield} (pl.\ \foreignlanguage{hebrew}{מָגִנִּים}) \\
			\foreignlanguage{hebrew}{מַמְלָכָה} & \textit{kingdom, sovereignty, dominion, reign} (cs.\ st.\ \foreignlanguage{hebrew}{מַמְלֶכֶת}) \\
			\foreignlanguage{hebrew}{סֶלַע} & \textit{rock, cliff} \\
			\foreignlanguage{hebrew}{עֵד} & \textit{witness} \\
			\foreignlanguage{hebrew}{עֵדוּת} & \textit{witness, testimony}, pl. \textit{laws, legal provisions} (pl. \foreignlanguage{hebrew}{עֵדֹת}; with ePP also \foreignlanguage{hebrew}{עֵדְוֹתָיו} \textit{ʿēdwōtāw} \textit{his legal provisions}, etc.) \\
			\foreignlanguage{hebrew}{עֶצֶם} & \textit{bone} (m.\ or f.; pl.\ \foreignlanguage{hebrew}{עֲצָמוֺת}) \\
			\foreignlanguage{hebrew}{קֶשֶׁת} & \textit{bow} (fem.)\\
			\foreignlanguage{hebrew}{שְׂמֹאל} & \textit{left side, left, north} (\textit{śəmō(ʾ)l} with silent \foreignlanguage{hebrew}{א}) \\
			\foreignlanguage{hebrew}{שֵׁבֶט} & \textit{rod, staff, scepter, tribe} \\
		\end{longtable}
	\end{center}
	
	\subsection{Prepositions}
	
	\begin{center}
		\begin{tabular}{>{\raggedleft}p{0.175\linewidth} p{0.75\linewidth}}
			\foreignlanguage{hebrew}{אֵת} & \textit{with, close to, beside} \\
			\foreignlanguage{hebrew}{עִם} & \textit{with, close to, beside} \\
		\end{tabular}
	\end{center}
	
	\subsection{Other Parts of Speech}
	
	\begin{center}
		\begin{tabular}{w{r}{3cm}w{l}{12cm}}
			\foreignlanguage{hebrew}{אַחֲרֵי־כֵן} & \textit{afterwards} (adv.) \\
			\foreignlanguage{hebrew}{בַּל} & \textit{not} (negative used in poetry) \\
			\foreignlanguage{hebrew}{בְּלִי} & \textit{not; without} \\
			\foreignlanguage{hebrew}{לְבִלְתִּי} & \textit{not} (negative used with the inf.\,cs.; \foreignlanguage{hebrew}{לְ} + \foreignlanguage{hebrew}{בִּלְתִּי}) \\
			\foreignlanguage{hebrew}{לָכֵן} & \textit{therefore} (\foreignlanguage{hebrew}{לְ} + \foreignlanguage{hebrew}{כֵּן}) \\
		\end{tabular}
	\end{center}
	
	
	\section{Prepositions with Enclitic Pronouns}
	
	Prepositions take enclitic pronouns either in their form with singular nouns or in their form with plural nouns.
	
	The following prepositions take singular noun type enclitic pronouns: \foreignlanguage{hebrew}{לְ}, \foreignlanguage{hebrew}{בְּ}, \foreignlanguage{hebrew}{עִם}, \foreignlanguage{hebrew}{אֵת} (and with \textit{maqqef} \foreignlanguage{hebrew}{אֶת־}). The prepositions \foreignlanguage{hebrew}{אֵת} and \foreignlanguage{hebrew}{עִם} follow the pattern of geminate nouns with enclitic pronouns.
	
	\begin{Center}
		\begin{tabular}{|lll|r|r|r|r|}
			\hline
			& & & \multicolumn{1}{c|}{Prep. \foreignlanguage{hebrew}{לְ}} &  \multicolumn{1}{c|}{Prep. \foreignlanguage{hebrew}{בְּ}} & \multicolumn{1}{c|}{Prep. \foreignlanguage{hebrew}{אֵת}} & \multicolumn{1}{c|}{Prep. \foreignlanguage{hebrew}{עִם}}  \\
			\hline
			1 & c. & sg. & \foreignlanguage{hebrew}{לִי} & \foreignlanguage{hebrew}{בִּי} & \foreignlanguage{hebrew}{אִתִּי} & \foreignlanguage{hebrew}{עִמָּדִי} ,\foreignlanguage{hebrew}{עִמִּי} \\
			\hline
			\multirow{2}{*}{2} & m. & sg. & \foreignlanguage{hebrew}{לְךָ} & \foreignlanguage{hebrew}{בְּךָ} & \foreignlanguage{hebrew}{אִתְּךָ} & \foreignlanguage{hebrew}{עִמְּךָ} \\
			& f. & sg. & \foreignlanguage{hebrew}{לָךְ} & \foreignlanguage{hebrew}{בָּךְ} & \foreignlanguage{hebrew}{אִתָּךְ} & \foreignlanguage{hebrew}{עִמָּךְ} \\
			\hline
			\multirow{2}{*}{3} & m. & sg. & \foreignlanguage{hebrew}{לוֺ} & \foreignlanguage{hebrew}{בּוֺ} & \foreignlanguage{hebrew}{אִתּוֺ} & \foreignlanguage{hebrew}{עִמּוֺ} \\
			& f. & sg. & \foreignlanguage{hebrew}{לָהּ} & \foreignlanguage{hebrew}{בָּהּ} & \foreignlanguage{hebrew}{אִתָּהּ} & \foreignlanguage{hebrew}{עִמָּהּ} \\
			\hline
			1 & c. & pl. & \foreignlanguage{hebrew}{לָ֫נוּ} & \foreignlanguage{hebrew}{בָּ֫נוּ} & \foreignlanguage{hebrew}{אִתָּ֫נוּ} & \foreignlanguage{hebrew}{עִמָּ֫נוּ} \\
			\hline
			\multirow{2}{*}{2} & m. & pl. & \foreignlanguage{hebrew}{לָכֶם} & \foreignlanguage{hebrew}{בָּכֶם} & \foreignlanguage{hebrew}{אִתְּכֶם} & \foreignlanguage{hebrew}{עִמָּכֶם} \\
			& f. & pl. & \foreignlanguage{hebrew}{לָכֶן} & \foreignlanguage{hebrew}{בָּכֶן} & — & — \\
			\hline
			\multirow{2}{*}{3} & m. & pl. & \foreignlanguage{hebrew}{לָהֶם} & \foreignlanguage{hebrew}{בָּם} ,\foreignlanguage{hebrew}{בָּהֶם} & \foreignlanguage{hebrew}{אִתָּם} & \foreignlanguage{hebrew}{עִמָּהֶם} ,\foreignlanguage{hebrew}{עִמָּם} \\
			& f. & pl. & \foreignlanguage{hebrew}{לָהֶן} & \foreignlanguage{hebrew}{בָּהֶן} & — & — \\
			\hline	
		\end{tabular}
	\end{Center}
	
	\vspace{0.5cm}
	
	\noindent \textbf{Notes}
	\nopagebreak
	
	\begin{enumerate}[noitemsep]
		\item The only deviating pausal form is found with the 2 m.\ sg.\ enclitic pronouns; thus \foreignlanguage{hebrew}{לָ֑ךְ}, \foreignlanguage{hebrew}{בָּ֑ךְ}, \foreignlanguage{hebrew}{אִתָּ֑ךְ} and \foreignlanguage{hebrew}{עִמָּ֑ךְ} instead of the context forms \foreignlanguage{hebrew}{לְךָ}, \foreignlanguage{hebrew}{בְּךָ}, \foreignlanguage{hebrew}{אִתְּךָ} and \foreignlanguage{hebrew}{עִמְּךָ}.
		\item With prepositions of the singular noun type different vowels than with the noun are used with the pausal form of the 2 m.\ sg.\ enclitic pronoun, the 2 f.\ sg.\ enclitic pronoun and the 2/3 m./f.\ pl.\ enclitic pronoun.
		\item In poetic texts, the form of the 3 m.\ pl.\ enclitic pronoun with the preposition \foreignlanguage{hebrew}{לְ} is often \foreignlanguage{hebrew}{לָ֫מוֺ} (53 attestations; mainly in Ps, Isa, Job, Lam).
	\end{enumerate}	
	
	\vspace{0.5cm}
	
	The prepositions \foreignlanguage{hebrew}{כְּ} and \foreignlanguage{hebrew}{מִן} have special forms when used with enclitic pronouns. Both prepositions take the enclitic pronouns of the singular noun type. The form of the enclitic pronoun of the 1 c. sg. is \foreignlanguage{hebrew}{נִי}-.
	
	The preposition \foreignlanguage{hebrew}{כְּ} uses a longer form with some enclitic pronouns. With the exception of forms with 2 and 3 pl.\ enclitic pronoun, the preposition \foreignlanguage{hebrew}{מִן} uses reduplicated forms (\textit{*minmin-} > \textit{*mimmin-}) to which the enclitic pronoun is attached
	
	\begin{itemize}[noitemsep]
		\item[--] with no change of consonants as in the case of the 1 c.\ sg.\ and 1 c.\ pl.\ enclitic pronoun forms \foreignlanguage{hebrew}{מִמֶּ֫נִּי} (< \textit{*mimmin-nī}) and \foreignlanguage{hebrew}{מִמֶּ֫נּוּ}
		\item[--] with assimilation of the final /n/ of the reduplicated preposition to the first consonant of the enclitic pronoun as in the 2 m.\ sg.\ form pausal form \foreignlanguage{hebrew}{מִמֶּ֑ךָּ} (<\,\textit{*mimmin-ka}) (the 2 m.\ sg.\ context form \foreignlanguage{hebrew}{מִמְּךָ} and 2 f.\ sg.\ form with enclitic pronoun \foreignlanguage{hebrew}{מִּמֵּךְ} reflect a further development of this type)
		\item[--] with assimilation of the initial /h/ of the original form of the enclitic pronoun to the preceding /n/ as in the 3 m.\ sg.\ and 3 f.\ sg.\ enclitic pronoun forms \foreignlanguage{hebrew}{מִמֶּ֫נּוּ} (< \textit{*mimmin-hu}) and \foreignlanguage{hebrew}{מִמֶּ֫נָּה} (< \textit{*mimmin-ha}).
	\end{itemize}
	
	% The following table may include some rare forms that don't need to be mentioned.
	
	\begin{Center}
		\begin{tabular}{|lll|r|r|}
			\hline
			& & & \multicolumn{1}{c|}{Prep. \foreignlanguage{hebrew}{כְּ}} & \multicolumn{1}{c|}{Prep. \foreignlanguage{hebrew}{מִן}} \\
			\hline
			1 & c. & sg. & \foreignlanguage{hebrew}{כָּמ֫וֺנִי} & \foreignlanguage{hebrew}{מִמֶּ֫נִּי} \\
			\hline
			\multirow{2}{*}{2} & m. & sg. & \foreignlanguage{hebrew}{כָּמ֫וֺךָ} & \foreignlanguage{hebrew}{מִמֶּ֑ךָּ} ,\foreignlanguage{hebrew}{מִמְּךָ} \\
			& f. & sg. & — & \foreignlanguage{hebrew}{מִּמֵּךְ}  \\
			\hline
			\multirow{2}{*}{3} & m. & sg. & \foreignlanguage{hebrew}{כָּמֹ֫הוּ} & \foreignlanguage{hebrew}{מִמֶּ֫נּוּ} \\
			& f. & sg. & \foreignlanguage{hebrew}{כָּמ֫וֺהָ} & \foreignlanguage{hebrew}{מִמֶּ֫נָּה} \\
			\hline
			1 & c. & pl. & \foreignlanguage{hebrew}{כָּמֹ֫נוּ} & \foreignlanguage{hebrew}{מִמֶּ֫נּוּ} \\
			\hline
			\multirow{2}{*}{2} & m. & pl. & \foreignlanguage{hebrew}{כְּמוֹכֶם} ,\foreignlanguage{hebrew}{כָּכֶם} & \foreignlanguage{hebrew}{מִכֶּם} \\
			& f. & pl. & — & — \\
			\hline
			\multirow{2}{*}{3} & m. & pl. & \foreignlanguage{hebrew}{כְּמוֹהֶם} ,\foreignlanguage{hebrew}{כָּהֶם} & \foreignlanguage{hebrew}{מִהֵ֫מּה} ,\foreignlanguage{hebrew}{מֵהֶם} \\
			& f. & pl. & \foreignlanguage{hebrew}{כָּהֵ֑ן} & \foreignlanguage{hebrew}{מֵהֵ֫נָּה} ,\foreignlanguage{hebrew}{מֵהֵן} \\
			\hline
		\end{tabular}
	\end{Center}
	
	\newpage
	
	The object marker \foreignlanguage{hebrew}{אֵת} has the following forms with enclitic pronouns:
	
	\vspace{0.5cm}
	
	\begin{Center}
		\begin{tabular}{|ll|r|r|}
			\hline
			&  & \multicolumn{1}{c|}{ Sg.} & \multicolumn{1}{c|}{ Pl.} \\
			\hline
			1 & c. & \foreignlanguage{hebrew}{אֹתִי} & \foreignlanguage{hebrew}{אֹתָ֫נוּ} \\
			\hline
			\multirow{2}{*}{2} & m. & \foreignlanguage{hebrew}{אֹתְךָ} & \foreignlanguage{hebrew}{אֶתְכֶם} \\
			& f. & \foreignlanguage{hebrew}{אֹתָךְ} & \foreignlanguage{hebrew}{אֶתְכֶן} \\
			\hline
			\multirow{2}{*}{3} & m. & \foreignlanguage{hebrew}{אֹתוֺ} & \foreignlanguage{hebrew}{אֶתְהֶם} ,\foreignlanguage{hebrew}{אֹתָם} \\
			& f. & \foreignlanguage{hebrew}{אֹתָהּ} & \foreignlanguage{hebrew}{אֶתְהֶן} ,\foreignlanguage{hebrew}{אוֺתָן} \\
			\hline
		\end{tabular}
	\end{Center}
	
	\vspace{0.5cm}
	
	\noindent \textbf{Notes}
	\nopagebreak
	
	\begin{enumerate}[noitemsep]
		\item The forms may also be spelled plene; e.g., \foreignlanguage{hebrew}{אוֹתִי}, \foreignlanguage{hebrew}{אוֺתְךָ}.
		\item The pausal form with the 2 m.\ sg.\ enclitic pronoun is \foreignlanguage{hebrew}{אֹתָ֑ךְ}.
	\end{enumerate}
	
	
	The following prepositions take plural noun type enclitic pronouns: \foreignlanguage{hebrew}{אֶל}, \foreignlanguage{hebrew}{עַל}, \foreignlanguage{hebrew}{אַחֲרֵי}, \foreignlanguage{hebrew}{עַד}, \foreignlanguage{hebrew}{תַּ֫חַת} (with the exception of the form \foreignlanguage{hebrew}{תַּחְתָּם} beside the form \foreignlanguage{hebrew}{תַּחְתֵּיהֶם}) and the compound preposition \foreignlanguage{hebrew}{לִפְנֵי}. The forms of the three most frequent prepositions among these are as follows.
	
	\vspace{0.5cm}
	
	\begin{Center}
		\begin{tabular}{|lll|r|r|r|}
			\hline
			& & & \multicolumn{1}{c|}{Prep. \foreignlanguage{hebrew}{אֶל}} & \multicolumn{1}{c|}{Prep.\foreignlanguage{hebrew}{עַל}} & \multicolumn{1}{c|}{Prep. \foreignlanguage{hebrew}{לִפְנֵי}} \\
			\hline
			1 & c. & sg. & \foreignlanguage{hebrew}{אֵלַי} & \foreignlanguage{hebrew}{עָלַי} & \foreignlanguage{hebrew}{לְפָנַי} \\
			\hline
			\multirow{2}{*}{2} & m. & sg. & \foreignlanguage{hebrew}{אֵלֶ֫יךָ} & \foreignlanguage{hebrew}{עָלֶ֫יךָ} & \foreignlanguage{hebrew}{לְפָנֶ֫יךָ} \\
			& f. & sg. & \foreignlanguage{hebrew}{אֵלַ֫יִךְ} & \foreignlanguage{hebrew}{עָלַ֫יִךְ} & \foreignlanguage{hebrew}{לְפָנַ֫יִךְ} \\
			\hline
			\multirow{2}{*}{3} & m. & sg. & \foreignlanguage{hebrew}{אֶלָיו} & \foreignlanguage{hebrew}{עָלָיו} & \foreignlanguage{hebrew}{לְפָנָיו} \\
			& f. & sg. & \foreignlanguage{hebrew}{אֵלֶ֫יהָ} & \foreignlanguage{hebrew}{עָלֶ֫יהָ} & \foreignlanguage{hebrew}{לְפָנֶ֫יהָ} \\
			\hline
			1 & c. & pl. & \foreignlanguage{hebrew}{אֵלֵ֫ינוּ} & \foreignlanguage{hebrew}{עָלֵ֫ינוּ} & \foreignlanguage{hebrew}{לְפָנֵ֫ינוּ} \\
			\hline
			\multirow{2}{*}{2} & m. & pl. & \foreignlanguage{hebrew}{אֲלֵיכֶם} & \foreignlanguage{hebrew}{עֲלֵיכֶם} & \foreignlanguage{hebrew}{לִפְנֵיכֶם} \\
			& f. & pl. & \foreignlanguage{hebrew}{אֲלֵיכֶן} & \foreignlanguage{hebrew}{עֲלֵיכֶן} & \foreignlanguage{hebrew}{לִפְנֵיכֶן} \\
			\hline
			\multirow{2}{*}{3} & m. & pl. & \foreignlanguage{hebrew}{אֲלֵיהֶם} & \foreignlanguage{hebrew}{עֲלֵיהֶם} & \foreignlanguage{hebrew}{לִפְנֵיהֶם}\\
			& f. & pl. & \foreignlanguage{hebrew}{אֲלֵיהֶן} & \foreignlanguage{hebrew}{עֲלֵיהֶן} & \foreignlanguage{hebrew}{לִפְנֵיהֶן} \\
			\hline	
		\end{tabular}
	\end{Center}
	
	\vspace{0.5cm}
	
	\noindent \textbf{Note}
	\nopagebreak
	
	\noindent The forms with 2/3 m./f.\ pl.\ enclitic pronoun are often spelled defectively, e.g., \foreignlanguage{hebrew}{אֲלֵהֶם}.
	
	\section{The Infinitive Construct}
	
	\subsection{The Forms of the Infinitive Construct}
	
	Biblical Hebrew has two infinitives, the infinitive construct and the infinitive absolute. They differ in form and usage. As verbal nouns they render the action of the verb as such, e.g., \textit{reading} in \textit{I like reading}.
	
	The infinitive construct of fientive verbs has the following forms:
	
	
	\begin{center}
		\begin{tabular}{|l|c|c|c|c|c|}
			\hline
			& Reg.\ verb & I\,gutt. & I\,ʾ & II\,gutt. & III\,gutt. \\
			\hline
			inf.\ cs. & \foreignlanguage{hebrew}{כְּתֹב} & \foreignlanguage{hebrew}{עֲזֹב} & \foreignlanguage{hebrew}{אֱסֹר} & \foreignlanguage{hebrew}{בְּחֹר} & \foreignlanguage{hebrew}{שְׁלֹחַ} \\
			\hline
		\end{tabular}
	\end{center}
	
	Stative verbs have the forms \foreignlanguage{hebrew}{כְּבֹד} or \foreignlanguage{hebrew}{כְּבַד} \textit{to be heavy}, \foreignlanguage{hebrew}{קְטֹן} \textit{to be small} and \foreignlanguage{hebrew}{חֲזֹק} \textit{to be strong}.
	
	The inf.\,cs.\ of the stative verb \foreignlanguage{hebrew}{שׁכב} \textit{to lie down} is \foreignlanguage{hebrew}{שְׁכַב} or \foreignlanguage{hebrew}{שְׁכֹב}. The form \foreignlanguage{hebrew}{שְׁכַב} occurs thirteen times with or without enclitic pronoun whereas the form \foreignlanguage{hebrew}{שְׁכֹב} is attested only four times and is used only with enclitic pronouns. Otherwise, the form \foreignlanguage{hebrew}{כְּתַב} for the inf.\,cs.\ of stative verbs is rare.
	
	% More information in JM §49c; LvS §45s, #423
	
	When followed by \textit{maqqef}, the \textit{ḥolem} is shortened to \textit{qameṣ ḥatuf} \foreignlanguage{hebrew}{כְּתָב־} \textit{kətɔḇ}.
	
	% Examples 2 Sam 15:8; 2 Sam 20:9; 1 Kgs 5:20
	
	\vspace{0.5cm}
	
	\noindent \textbf{Note}
	\nopagebreak
	
	\noindent The inf.\,cs.\ with the prefixed prepositions \foreignlanguage{hebrew}{בְּ} and \foreignlanguage{hebrew}{כְּ} is \foreignlanguage{hebrew}{בִּכְתֹב} and \foreignlanguage{hebrew}{כִּכְתֹב}, respectively. But with the preposition \foreignlanguage{hebrew}{לְ} it is \foreignlanguage{hebrew}{לִכְתֹּב} with \textit{dageš lene} in the second root consonant.
	
	\subsection{The Infinitive Construct with Enclitic Pronouns}
	
	With enclitic pronouns the infinitive construct has the following forms.
	
	\begin{center}
		\begin{tabular}{|ll|r|r|}
			\hline
			& & \multicolumn{1}{c|}{Sing. Pron.} & \multicolumn{1}{c|}{Pl. Pron.} \\
			\hline
			1 & c. & \foreignlanguage{hebrew}{כָּתְבִי} / \foreignlanguage{hebrew}{כָּתְבֵנִי} & \foreignlanguage{hebrew}{כָּתְבֵנוּ} \\
			\hline
			\multirow{2}{*}{2} & m. & \foreignlanguage{hebrew}{כָּתְבְּךָ} / \foreignlanguage{hebrew}{כְּתָבְךָ} & \foreignlanguage{hebrew}{כָּתְבְּכֶם} / \foreignlanguage{hebrew}{כְּתָבְכֶם} \\
			& f. & \foreignlanguage{hebrew}{כָּתְבֵךְ} & \\
			\hline
			\multirow{2}{*}{3} & m. & \foreignlanguage{hebrew}{כָּתְבוֺ} & \foreignlanguage{hebrew}{כָּתְבָם} \\
			& f. & \foreignlanguage{hebrew}{כָּתְבָהּ} & \foreignlanguage{hebrew}{} \\
			\hline
		\end{tabular}
	\end{center}
	
	\vspace{0.5cm}
	
	\noindent \textbf{Notes}
	\nopagebreak
	
	\begin{enumerate}[noitemsep]
		\item The \textit{qameṣ} with the first root consonant is a \textit{qameṣ ḥatuf}, e.g., \foreignlanguage{hebrew}{שָׁלְחִי} \textit{šɔlḥī} \textit{my sending}. Likewise, when the \textit{qameṣ} is with the second root consonant, it is a \textit{qameṣ ḥatuf} as well, e.g., \foreignlanguage{hebrew}{אֲכָלְכֵם} \textit{ʾăḵɔlḵæm} \textit{your eating}.
		\item There are two distinct forms of the 1 c.\ sg.\ enclitic pronoun. In the form \foreignlanguage{hebrew}{כָּתְבִי} the enclitic pronoun expresses the subject of the infinitive construct whereas in the form \foreignlanguage{hebrew}{כָּתְבֵנִי} the enclitic pronoun expresses the object, e.g., \foreignlanguage{hebrew}{לְשָׁכְנִי} \textit{that I might dwell} (Exod 29:46), \foreignlanguage{hebrew}{לְלָכְדֵנִי} \textit{to capture me} (Jer 18:22). All other suffixes can either express the subject of the infinitive construct or the object.
		\item Instead of \textit{qameṣ ḥatuf}, the vowel \textit{ḥireq} is sometimes found with the first root consonant, e.g., \foreignlanguage{hebrew}{שִׁבְרִי} \textit{my breaking}. The vowel \textit{ḥireq} is also used for the inf.\,cs.\ with the thematic vowel /a/ \foreignlanguage{hebrew}{כְּתַב}, e.g., \foreignlanguage{hebrew}{שִׁכְבָהּ} \textit{her lying} (Gen 19:33).
	\end{enumerate}
	
	
	\subsection{The Use of the Infinitive Construct}
	
	\subsubsection{The Infinitive Construct as a Verbal Noun}
	
	The infinitive construct can be used just like regular nouns as subject (1\,Sam 15:22) or object of the clause (Deut 2:7). The infinitive construct may be used as it is or it may be preceded by the preposition \foreignlanguage{hebrew}{לְ} (1\,Sam 15:22). It is also used as the dependent noun in a construct chain (Ruth 1:1).
	
	\vspace{0.5cm}
	
	\begin{tabular}{>{\raggedleft}p{0.35\linewidth} p{0.55\linewidth}}
		\foreignlanguage{hebrew}{הִנֵּה שְׁמֹעַ מִזֶּבַח טוֹב לְהַקְשִׁיב מֵחֵלֶב אֵילִים} & \textit{Obeying is better than sacrifice; listening [is better] than the fat of rams} (1\,Sam 15:22) \\
		\foreignlanguage{hebrew}{יָדַע לֶכְתְּךָ אֶת־הַמִּדְבָּר הַגָּדֹל הַזֶּה} & \textit{He knows your walking through this great desert} (Deut 2:7) \\
		\foreignlanguage{hebrew}{וַיְהִי בִּימֵי שְׁפֹט הַשֹּׁפְטִים וַיְהִי רָעָב בָּאָרֶץ} & \textit{In the days of the ruling of the judges
			there was a famine in the land} (Ruth 1:1) \\
	\end{tabular}
	
	\subsubsection{The Infinitive Construct with Prepositions in Adverbial Constructions}
	
	With the preposition \foreignlanguage{hebrew}{לְ} the infinitive construct is frequently used as final or consecutive adjunct (Gen 37:25; 1\,Kgs 16:2) or as modal adjunct \textit{by doing so} (Deut 6:2).
	
	% Other reference for modal use: 1 Sam 12:19
	
	\vspace{0.5cm}
	
	\begin{tabular}{>{\raggedleft}p{0.35\linewidth} p{0.55\linewidth}}
		\foreignlanguage{hebrew}{וַיֵּשְׁבוּ לֶאֱכָל־לֶחֶם} & \textit{And they sat down to eat} (Gen 37.25) \\
		\foreignlanguage{hebrew}{וַתַּחֲטִא אֶת־עַמִּי יִשְׂרָאֵל לְהַכְעִיסֵנִי בְּחַטֹּאתָם} & \textit{And you made my people Israel sin so that they provoked me to anger with their sins} (1\,Kgs 16:2) \\
		\foreignlanguage{hebrew}{לְמַעַן תִּירָא אֶת־יְהוָה אֱלֹהֶיךָ לִשְׁמֹר אֶת־כָּל־חֻקֹּתָיו וּמִצְוֺתָיו} & \textit{\dots \space that you may fear the Lord your God by keeping all his statutes and his commandments} (Deut 6:2) \\
	\end{tabular}
	
	\vspace{0.5cm}
	
	With the prepositions \foreignlanguage{hebrew}{בְּ} and \foreignlanguage{hebrew}{כְּ} the infinitive construct is used frequently as temporal adjunct.
	
	\vspace{0.5cm}
	
	\begin{tabular}{>{\raggedleft}p{0.35\linewidth} p{0.55\linewidth}}
		\foreignlanguage{hebrew}{וַיְהִי בִּבְרֹחַ אֶבְיָתָר בֶּן־אֲחִימֶלֶךְ אֶל־דָּוִד קְעִ֫ילָה אֵפוֹד יָרַד בְּיָדוֹ} & \textit{When Abiathar the son of Ahimelech had fled to David to Keilah, an ephod came down in his hand.} (1\,Sam 23:6) \\
		\foreignlanguage{hebrew}{כִּשְׁמֹעַ עֵשָׂו אֶת־דִּבְרֵי אָבִיו וַיִּצְעַק צְעָקָה גְּדֹלָה וּמָרָה עַד־מְאֹד} & \textit{When Esau heard the words of his father, he cried out with a very loud and bitter cry} (Gen 27:34) \\
	\end{tabular}
	
	\vspace{0.5cm}
	
	With the preposition \foreignlanguage{hebrew}{בְּ} the infinitive construct also may have modal or causal meaning (cf.\ ESV for the causal translation of the infinitive with \foreignlanguage{hebrew}{בְּ} in 1\,Kgs 18:18).
	
	\vspace{0.5cm}
	
	\begin{tabular}{>{\raggedleft}p{0.35\linewidth} p{0.55\linewidth}}
		\foreignlanguage{hebrew}{לֹא עָכַרְתִּי אֶת־יִשְׂרָאֵל כִּי אִם־אַתָּה וּבֵית אָבִיךָ בַּעֲזָבְכֶם אֶת־מִצְוֺת יְהוָה} & \textit{It is not I who have brought trouble on Israel, but you and your father’s House, by forsaking the commandments of the Lord} (1\,Kgs 18:18 NJPS) \\
	\end{tabular}
	
	\vspace{0.5cm}
	
	The preposition \foreignlanguage{hebrew}{מִן} together with the infinitive construct may have (negative) consecutive (1\,Sam 8:7), causal (Deut 7:8) or temporal meaning (Lev 9:22).
	
	\vspace{0.5cm}
	
	\begin{tabular}{>{\raggedleft}p{0.35\linewidth} p{0.55\linewidth}}
		\foreignlanguage{hebrew}{כִּי־אֹתִי מָאֲסוּ מִמְּלֹךְ עֲלֵיהֶם} & \textit{\dots \space for it is me whom they have rejected so that I am not king over them anymore} (1\,Sam 8:7) \\
		\foreignlanguage{hebrew}{כִּי מֵאַהֲבַת יְהוָה אֶתְכֶם וּמִשָּׁמְרוֺ אֶת־הַשְּׁבֻעָה אֲשֶׁר נִשְׁבַּע לַאֲבֹתֵיכֶם הוֹצִיא יְהוָה אֶתְכֶם בְּיָד חֲזָקָה} & \textit{For because the Lord loved you and because he kept his oath that he had sworn to your fathers, he led you out with a strong hand} (Deut 7:8) \\
		\foreignlanguage{hebrew}{וַיֵּרֶד מֵעֲשֹׂת הַֽחַטָּאת וְהָעֹלָה וְהַשְּׁלָמִֽים} & \textit{\dots \space and he stepped down after having sacrificed the sin offering, the burnt offering, and the offering of well-being} (Lev 9:22) \\
	\end{tabular}
	
	\subsubsection{The Infinitive Construct as Complement}
	With some verbs the infinitive construct is used as a complement (usually with the preposition \foreignlanguage{hebrew}{לְ} and at times without). The verbs \foreignlanguage{hebrew}{יכל} \textit{to be able}, \foreignlanguage{hebrew}{חדל} \textit{to cease} and \foreignlanguage{hebrew}{נתן} in the meaning \textit{to allow, permit} can be mentioned here.\footnote{\space The verb \foreignlanguage{hebrew}{יכל} is a stative verb with the the vowel /ō/ as thematic vowel in the suffix conjugation. For the other forms of the verb see Chapter 13.}
	
	\vspace{0.5cm}
	
	\begin{tabular}{>{\raggedleft}p{0.35\linewidth} p{0.55\linewidth}}
		\foreignlanguage{hebrew}{וְלֹא־יָכְלוּ עוֹד לַעֲמֹד לִפְנֵי אוֹיְבֵיהֶם} & \textit{And they could not stand before their enemies} (Judg 2:14) \\
		\foreignlanguage{hebrew}{וַיַּחְדְּלוּ לִבְנֹת הָעִיר} & \textit{And they stopped building the city} (Gen 11:8) \\
		\foreignlanguage{hebrew}{וְלֹא־נָתְנוּ אִישׁ לַעֲבֹר} & \textit{And they did not allow anyone to cross} (Judg 3:28) \\
	\end{tabular}
	
	\vspace{0.5cm}
	
	Other verbs that take infinitives as complements are: \foreignlanguage{hebrew}{אבה}  Q.\ \textit{to be willing}, \foreignlanguage{hebrew}{מאן} Pi.\ \textit{to refuse}, \foreignlanguage{hebrew}{כלה} Pi.\ \textit{to finish}, \foreignlanguage{hebrew}{חלל} Hi.\ \textit{to begin}. These verbs will be introduced in later chapters because they are weak verbs and/or are used in a different \textit{binyan} than Qal.
	
	With some verbs the preposition \foreignlanguage{hebrew}{מִן} can be used instead of \foreignlanguage{hebrew}{לְ}, e.g., \foreignlanguage{hebrew}{כלה} Pi.\ \textit{to finish} and \foreignlanguage{hebrew}{חדל} Q.\ \textit{to cease, stop}.
	
	The use of the infinitive construct with so-called modifying verbs will be discussed in Chapter 23.
	
	
	\section{The Infinitive Absolute}
	
	\subsection{The Form of the Infinitive Absolute}
	
	The infinitive absolute Qal has the following form:
	
	\begin{center}
		\begin{tabular}{lr}
			Inf.\,abs. & \foreignlanguage{hebrew}{כָּתוֹב}
		\end{tabular}
	\end{center}
	
	The inf.\,abs.\ may be spelled defectively, e.g., \foreignlanguage{hebrew}{אָכֹל}. With a guttural as third root consonant the form is \foreignlanguage{hebrew}{שָׁלֹחַ}.
	
	\subsection{The Use of the Infinitive Absolute}
	
	The inf.\,abs.\ is not used with prepositions, enclitic pronouns or as the governing noun in a construct chain. This fact explains the term \emph{infinitive absolute} as opposed to the term \textit{infinitive construct}. The following three main functions of the inf.\,abs.\ can be mentioned.
	
	\subsubsection{The Intensifying Use of the Infinitive Absolute} 
	
	Together with another verb form of the same root the inf.\,abs.\ is used to intensify the verbal content or the meaning of the other verbal form or the nuance in its context or a combination of all these.\footnote{\space Jan P. Lettinga and Heinrich von Siebenthal, \textit{Grammatik des biblischen Hebräisch}, 2nd ed. revised and expanded by  Heinrich von Siebenthal (Gießen: Brunnen; Basel: Immanuel, 2016), 337.} Usually the inf.\,abs.\ is used in the same \textit{binyan} as the other verbal form; but sometimes it can be in a different \textit{binyan}. The following functions can be found frequently:
	
	\begin{enumerate}[noitemsep]
		\item In positive statements, the inf.\,abs.\ can be used to intensify the factuality of the statement (Gen 37:33; Gen 18:18).
		\item In negative statements the inf.\,abs.\ stresses the non-realization of the event (Gen 3:4; Dan 10:3).
		\item  In rhetorical questions the inf.\,abs.\ denies the realization of the event (2 Kgs 18:33).\footnote{\space Christo H.\ J.\ van der Merwe, Jacobus A. Naudé and Jan H.\ Kroeze, \textit{A Biblical Hebrew Reference Grammar}, 2nd ed.\ (London: Bloomsbury, 2017), 179.}
		\item The inf.\,abs.\ intensifies the modal meaning of a verbal form (Num 35:16; Lev 7:24).
	\end{enumerate}
	
	\begin{longtable}{>{\raggedleft}p{0.35\linewidth} p{0.55\linewidth}}
		\foreignlanguage{hebrew}{חַיָּה רָעָה אֲכָלָתְהוּ טָרֹף טֹרַף יוֹסֵֽף} & \textit{A fierce animal has devoured him. Joseph has surely been torn to pieces.} (Gen 37:33) \\
		\foreignlanguage{hebrew}{וְאַבְרָהָם הָיוֹ יִֽהְיֶה לְגוֹי גָּדוֹל וְעָצוּם} & \textit{Abraham will surely become a great and mighty nation.} (Gen 18:18) \\
		\foreignlanguage{hebrew}{לֹֽא־מוֹת תְּמֻתֽוּן} & \textit{You will definitely not die.} (Gen 3:4) \\
		\foreignlanguage{hebrew}{וְסוֹךְ לֹא־סָכְתִּי} & \textit{I did not even anoint myself.} (Dan 10:3) \\
		\foreignlanguage{hebrew}{הַהַצֵּל הִצִּילוּ אֱלֹהֵי הַגּוֹיִם אִישׁ אֶת־אַרְצוֹ מִיַּד מֶלֶךְ אַשּֽׁוּר} & \textit{Did any of the gods of the nations save his land from the hand of the king of Assyria?} (2\,Kgs 18:33) \\
		\foreignlanguage{hebrew}{מוֹת יוּמַת הָרֹצֵֽחַ} & \textit{The murderer must be put to death.} (Num 35:16) \\
		\foreignlanguage{hebrew}{וְחֵלֶב נְבֵלָה וְחֵלֶב טְרֵפָה יֵעָשֶׂה לְכָל־מְלָאכָה וְאָכֹל לֹא תֹאכְלֻֽהוּוּ} & \textit{The fat of an animal that is dying of itself, or of a torn animal may be used for any use, but you must certainly not eat it} (Lev 7:24) \\
	\end{longtable}
	
	\subsubsection{The Infinitive Absolute as Verbal Predicate}
	
	\noindent The inf.\,abs.\ can be used as a replacement for a finite form. It can either continue a preceding finite verbal form in which case it has the same function (Judg 7:19) or it can be used without a preceding verbal form. In this case it often replaces an imperative (Exod 20:8; Jer 2:2) or a different finite verbal form (\foreignlanguage{hebrew}{רָגוֺם} in Num 15:35).
	
	\vspace{0.5cm}
	
	\begin{longtable}{>{\raggedleft}p{0.35\linewidth} p{0.55\linewidth}}
		\foreignlanguage{hebrew}{וַֽיִּתְקְעוּ בַּשּׁוֹפָרוֹת וְנָפוֹץ הַכַּדִּים אֲשֶׁר בְּיָדָֽם} & \textit{They blew the horns and smashed the jars that were in their hands} (Judg 7:19) \\
		\foreignlanguage{hebrew}{זָכוֹר אֶת־יוֹם הַשַׁבָּת לְקַדְּשֽׁוֹ} & \textit{Remember the sabbath day by keeping it holy} (Exod 20:8) \\
		\foreignlanguage{hebrew}{הָלֹךְ וְקָֽרָאתָ בְאָזְנֵי יְרוּשָׁלִַם} & \textit{Go and proclaim in the hearing of Jerusalem} (Jer 2:2) \\
		\foreignlanguage{hebrew}{מוֹת יוּמַת הָאִישׁ רָגוֹם אֹתוֹ בָֽאֲבָנִים כָּל־הָעֵדָה מִחוּץ לַֽמַּחֲנֶֽה} & \textit{The man must be put to death, the whole congregation shall stone him with stones outside of the camp} (Num 15:35) \\
	\end{longtable}
	
	
	
	\subsubsection{Two Consecutive Infinitives Absolute}
	
	When the inf.\,abs.\ is used after a finite form of the same verb together with another coordinated inf.\,abs., the two infinitives express the idea that the two events happened simultaneously (or nearly simultaneously).
	
	\vspace{0.5cm}
	
	\begin{tabular}{>{\raggedleft}p{0.35\linewidth} p{0.55\linewidth}}
		\foreignlanguage{hebrew}{וַיִּרְדֵּהוּ אֶל־כַּפָּיו וַיֵּלֶךְ הָלוֹךְ וְאָכֹל} & \textit{He scraped it [the honey] out into his hands and ate as he went along} (Judg 14:9) \\
		\foreignlanguage{hebrew}{וְכָל־הָעָם אֲשֶׁר־אִתּוֹ חָפוּ אִישׁ רֹאשׁוֹ וְעָלוּ עָלֹה וּבָכֹה} & \textit{\dots and David went up the ascent of the} Mount of \textit{Olives, weeping as he went} (2\,Sam 15:30) \\
	\end{tabular}
	
	\vspace{0.5cm}
	
	\noindent The inf.\,abs.\ \foreignlanguage{hebrew}{הָלֹךְ} may be used together with a finite form and the infinite absolute of another verb to express the idea that an event took place continually or increasingly. (In 2 Sam 5:10 even the finite verb is a form of \foreignlanguage{hebrew}{הלך} leaving the inf.\,abs.\ \foreignlanguage{hebrew}{גָּדוֹל} to express the main meaning \textit{to become greater}.)
	
	\vspace{0.5cm}
	
	\begin{tabular}{>{\raggedleft}p{0.35\linewidth} p{0.55\linewidth}}
		\foreignlanguage{hebrew}{וַיִּסַּע אַבְרָם הָלוֹךְ וְנָסוֹעַ הַנֶּ֫גְבָּה} & \textit{And Abram journeyed by stages to the Negeb} (Gen 12:9) \\
		\foreignlanguage{hebrew}{וַיֵּלֶךְ דָּוִד הָלוֹךְ וְגָדוֹל} & \textit{And David became greater and greater} (2\,Sam 5:10) \\
	\end{tabular}
	
	\vspace{0.5cm}
	
	\section{Exercises}
	
	Translate the following sentences from the Hebrew Bible. Names of persons and geographical names in these sentences: \foreignlanguage{hebrew}{אַבְרָהָם}, \foreignlanguage{hebrew}{אֱדוֹם}, \foreignlanguage{hebrew}{אַחְאָב}, \foreignlanguage{hebrew}{אִיזָבֶל}, \foreignlanguage{hebrew}{אֲשֵׁרָה}, \foreignlanguage{hebrew}{בִּלְהָה}, \foreignlanguage{hebrew}{דָּוִד}, \foreignlanguage{hebrew}{חִירָם}, \foreignlanguage{hebrew}{יְהוּדָה}, \foreignlanguage{hebrew}{יְהוֹשֻׁעַ}, \foreignlanguage{hebrew}{כַּרְמֶל}, \foreignlanguage{hebrew}{מוֹאָב}, \foreignlanguage{hebrew}{סִיחֹן}, \foreignlanguage{hebrew}{עֵשָׂו}, \foreignlanguage{hebrew}{רְאוּבֵן}, \foreignlanguage{hebrew}{שָׁאוּל}, \foreignlanguage{hebrew}{שְׁלֹמֹה}, \foreignlanguage{hebrew}{שְׁמוּאֵל}
	
	\vspace{0.5cm}
	
	\selectlanguage{hebrew}
	\noindent
	1~~\foreignlanguage{hebrew}{וַיֹּאמֶר לָכֵן שְׁמַע דְּבַר־יְהוָה רָאִ֫יתִי}\LTRfootnote{\space \foreignlanguage{hebrew}{רָאִיתִי} \textit{I have seen} (verb \foreignlanguage{hebrew}{ראה} Q. \textit{to see})} \foreignlanguage{hebrew}{אֶת־יְהוָה יֹשֵׁב עַל־כִּסְאוֹ וְכָל־צְבָא הַשָּׁמַיִם עֹמֵד עָלָיו מִימִינוֹ וּמִשְּׂמֹאלוֹ} \hspace{0.3cm}
	2~~\foreignlanguage{hebrew}{וְעַתָּה שְׁלַח קְבֹץ אֵלַי אֶת־כָּל־יִשְׂרָאֵל אֶל־הַר הַכַּרְמֶל וְאֶת־נְבִיאֵי הַבַּעַל אַרְבַּע מֵאוֹת וַחֲמִשִּׁים}\LTRfootnote{\space \foreignlanguage{hebrew}{אַרְבַּע מֵאוֹת וַחֲמִשִּׁים} \textit{four hundred and fifty}} \foreignlanguage{hebrew}{וּנְבִיאֵי הָאֲשֵׁרָה אַרְבַּע מֵאוֹת}\LTRfootnote{\space \foreignlanguage{hebrew}{אַרְבַּע מֵאוֹת} \textit{four hundred}} \foreignlanguage{hebrew}{אֹכְלֵי שֻׁלְחַן אִיזָ֫בֶל. וַיִּשְׁלַח אַחְאָב בְּכָל־בְּנֵי יִשְׂרָאֵל וַיִּקְבֹּץ אֶת־הַנְּבִיאִים אֶל־הַר הַכַּרְמֶֽל}  \hspace{0.3cm}
	3~~\foreignlanguage{hebrew}{בַּיהוָה אֱלֹהֵי־יִשְׂרָאֵל בָּטָח וְאַחֲרָיו לֹא־הָיָה כָמֹהוּ בְּכֹל מַלְכֵי יְהוּדָה וַאֲשֶׁר הָיוּ}\LTRfootnote{\space \foreignlanguage{hebrew}{הָיוּ} \textit{they were}} \foreignlanguage{hebrew}{לְפָנָיו}  \hspace{0.3cm}
	4~~\foreignlanguage{hebrew}{וַיִּשְׁלַח אַבְרָהָם אֶת־יָדוֹ וַיִּקַּח אֶת־הַמַּאֲכֶלֶת}\LTRfootnote{\space \foreignlanguage{hebrew}{מַאֲכֶלֶת} \textit{knife}} \foreignlanguage{hebrew}{לִשְׁחֹט אֶת־בְּנוֹ}  \hspace{0.3cm}
	5~~\foreignlanguage{hebrew}{וַיָּבֹא אֶל־הָאִישׁ וְהִנֵּה עֹמֵד עַל־הַגְּמַלִּים עַל־הָעָֽיִן}  \hspace{0.3cm}
	6~~\foreignlanguage{hebrew}{כִּשְׁמֹעַ עֵשָׂו אֶת־דִּבְרֵי אָבִיו וַיִּצְעַק צְעָקָה}\LTRfootnote{\space \foreignlanguage{hebrew}{צְעָקָה} \textit{cry, outcry}} \foreignlanguage{hebrew}{גְּדֹלָה וּמָרָה}\LTRfootnote{\space \foreignlanguage{hebrew}{מַר} \textit{bitter}} \foreignlanguage{hebrew}{עַד־מְאֹד וַיֹּאמֶר לְאָבִיו בָּרֲכֵנִי}\LTRfootnote{\space \foreignlanguage{hebrew}{בָּרֲכֵנִי} \textit{bless me} (imperative Pi. with enclitic pronoun)} \foreignlanguage{hebrew}{גַם־אָ֫נִי אָבִי} \hspace{0.3cm}
	7~~\foreignlanguage{hebrew}{וַיְהִי בִּשְׁכֹּן יִשְׂרָאֵל בָּאָרֶץ הַהִיא וַיֵּלֶךְ רְאוּבֵן וַיִּשְׁכַּב אֶת־בִּלְהָה פִּילֶגֶשׁ}\LTRfootnote{\space \foreignlanguage{hebrew}{פִּילֶגֶשׁ} \textit{concubine}} \foreignlanguage{hebrew}{אָבִיו וַיִּשְׁמַע יִשְׂרָאֵל} \hspace{0.3cm}
	8~~\foreignlanguage{hebrew}{וְאֵלֶּה הַמְּלָכִים אֲשֶׁר מָלְכוּ בְּאֶרֶץ אֱדוֹם לִפְנֵי מְלָךְ־מֶלֶךְ לִבְנֵי יִשְׂרָאֵל}  \hspace{0.3cm}
	9~~\foreignlanguage{hebrew}{וְלֹא־נָתַן סִיחֹן אֶת־יִשְׂרָאֵל עֲבֹר בִּגְבֻלוֹ}  \hspace{0.3cm}
	10~~\foreignlanguage{hebrew}{וַיְהִי כְּשָׁמְעֲכֶם אֶת־הַקּוֹל מִתּוֹךְ הַחֹשֶׁךְ וְהָהָר בֹּעֵר בָּאֵשׁ וַתִּקְרְבוּן אֵלַי כָּל־רָאשֵׁי שִׁבְטֵיכֶם וְזִקְנֵיכֶם}  \hspace{0.3cm}
	\selectlanguage{english}
	11~~\foreignlanguage{hebrew}{וַיֹּאמֶר יְהוֹשֻׁעַ אֶל־הָעָם עֵדִים אַתֶּם בָּכֶם כִּי־אַתֶּם בְּחַרְתֶּם לָכֶם אֶת־יְהוָה לַעֲבֹד אוֹתוֹ וַיֹּאמְרוּ}\LTRfootnote{\space \foreignlanguage{hebrew}{וַיֹּאמְרוּ} \textit{and they said}} \foreignlanguage{hebrew}{עֵדִים} \hspace{0.3cm}
	12~~\foreignlanguage{hebrew}{וַיְהִי בִּימֵי שְׁפֹט הַשֹּׁפְטִים וַיְהִי רָעָב בָּאָרֶץ וַיֵּלֶךְ אִישׁ מִבֵּית לֶחֶם יְהוּדָה לָגוּר}\LTRfootnote{\space \foreignlanguage{hebrew}{לָגוּר} \textit{to sojourn} (inf.\,cs. with \foreignlanguage{hebrew}{לְ})} \foreignlanguage{hebrew}{בִּשְׂדֵי מוֹאָב הוּא וְאִשְׁתּוֹ וּשְׁנֵי בָנָיו}\foreignlanguage{hebrew}{}\LTRfootnote{\space \foreignlanguage{hebrew}{וּשְׁנֵי בָנָיו} \textit{and his two sons}} \hspace{0.3cm}
	13~~\foreignlanguage{hebrew}{וַיֹּאמֶר שְׁמוּאֵל אֶל־שָׁאוּל אֹתִי שָׁלַח יְהוָה לִמְשָׁחֳךָ לְמֶלֶךְ עַל־עַמּוֹ עַל־יִשְׂרָאֵל וְעַתָּה שְׁמַע לְקוֹל דִּבְרֵי יְהוָה}  \hspace{0.3cm}
	14~~\foreignlanguage{hebrew}{וַיִּזְבַּח אֶת־כָּל־כֹּהֲנֵי הַבָּמוֹת אֲשֶׁר־שָׁם עַל־הַמִּזְבְּחוֹת וַיִּשְׂרֹף אֶת־עַצְמוֹת אָדָם עֲלֵיהֶם וַיָּ֫שָׁב}\LTRfootnote{\space \foreignlanguage{hebrew}{וַיָּ֫שָׁב} \textit{wayyā́šɔḇ} \textit{and he returned}} \foreignlanguage{hebrew}{יְרוּשָׁלִָם}  \hspace{0.3cm}
	15~~\foreignlanguage{hebrew}{וַיְהִי כִּשְׁמֹעַ חִירָם אֶת־דִּבְרֵי שְׁלֹמֹה וַיִּשְׂמַח מְאֹד וַיֹּאמֶר בָּרוּךְ יְהוָה הַיּוֹם אֲשֶׁר נָתַן לְדָוִד בֵּן חָכָם עַל־הָעָם הָרָב הַזֶּה}  \hspace{0.3cm}
	16~~\foreignlanguage{hebrew}{וַיְהִי כִשְׁמֹעַ אַחְאָב אֶת־הַדְּבָרִים הָאֵלֶּה וַיִּקְרַע בְּגָדָיו}  \hspace{0.3cm}
	17~~\foreignlanguage{hebrew}{וְעַתָּה הִנֵּה יָדַעְתִּי כִּי מָלֹךְ תִּמְלוֹךְ וְקָ֫מָה}\LTRfootnote{\space \foreignlanguage{hebrew}{וְקָ֫מָה} \textit{and [it] will be established}} \foreignlanguage{hebrew}{בְּיָדְךָ מַמְלֶכֶת יִשְׂרָאֵל} \hspace{0.3cm}
	18~~\foreignlanguage{hebrew}{שָׁמוֹר תִּשְׁמְרוּן אֶת־מִצְוֹת יְהוָה אֱלֹהֵיכֶם וְעֵדֹתָיו וְחֻקָּיו אֲשֶׁר צִוָּךְ}\LTRfootnote{\space \foreignlanguage{hebrew}{צִוָּךְ} \textit{he commanded you}}  \hspace{0.3cm}
	19~~\foreignlanguage{hebrew}{קָרֹעַ אֶקְרַע אֶת־הַמַּמְלָכָה מֵעָלֶיךָ} \hspace{0.3cm}
	20~~\foreignlanguage{hebrew}{וַיֵּלֶךְ דָּוִד הָלוֹךְ וְגָדוֹל וַיהוָה אֱלֹהֵי צְבָאוֹת עִמּוֹ}  \hspace{0.3cm}
	
	
	\chapter{Chapter 10}
	
	\renewcommand\arraystretch{1.4}
	
	\section{Vocabulary}
	
	\subsection{Verbs}
	
	\begin{center}
		
		% For the centering of the separation between the two columns see the documentation of the array package, page 2 
		
		\begin{tabular}{>{\raggedleft}p{0.175\linewidth} p{0.75\linewidth}}
			\foreignlanguage{hebrew}{דבק} & Q.\ \textit{to cling, cleave, keep close} \\
			\foreignlanguage{hebrew}{זרע} & Q.\ \textit{to sow, scatter seed} \\
			\foreignlanguage{hebrew}{חדל} & Q.\ \textit{to cease, come to an end; to cease, leave off} \\ % BDB
			\foreignlanguage{hebrew}{כתב} & Q.\ \textit{to write} \\ % Already included in Chapter 6; can be deleted
			\foreignlanguage{hebrew}{נטע} & Q.\ \textit{to plant} \\
		\end{tabular}
	\end{center}
	
	\subsection{Nouns}
	
	\begin{center}
		\begin{longtable}{>{\raggedleft}p{0.175\linewidth} p{0.75\linewidth}}
			\foreignlanguage{hebrew}{בוֺר} & \textit{cistern, pit} \\
			\foreignlanguage{hebrew}{גְּבוּרָה} & \textit{strength, might} \\
			\foreignlanguage{hebrew}{חָג} & \textit{festival, festival-gathering, pilgrim-feast} (pl. \foreignlanguage{hebrew}{חַגִּים}; abs.\ st.\ also \foreignlanguage{hebrew}{חַג}) \\ % BDB
			\foreignlanguage{hebrew}{חָדָשׁ} & \textit{new} \\
			\foreignlanguage{hebrew}{חֵלֶק} & \textit{portion, share, possession} \\ % BDB
			\foreignlanguage{hebrew}{נְאֻם} & \textit{utterance, declaration, announcement} \\ % BDB and HALOT
			\foreignlanguage{hebrew}{נַחֲלָה} & \textit{possession, property, inheritance} \\
			\foreignlanguage{hebrew}{צַדִּיק} & \textit{just, righteous} \\
			\foreignlanguage{hebrew}{קָנֶה} & \textit{stalk, reed} \\
			\foreignlanguage{hebrew}{רִיב} & \textit{dispute, lawsuit, quarrel} \\ % HALOT
			\foreignlanguage{hebrew}{שַׂר} & \textit{ruler, official, prince, captain, chieftain, chief} (pl.\ \foreignlanguage{hebrew}{שָׂרִים}) \\ % BDB
			\foreignlanguage{hebrew}{שׁוֺפָר} & \textit{horn} (as a wind instrument) \\
			\foreignlanguage{hebrew}{שַׁעַר} & \textit{gate} \\
		\end{longtable}
	\end{center}
	
	\newpage
	
	\subsection{Other Parts of Speech}
	
	\begin{center}
		\begin{longtable}{>{\raggedleft}p{0.175\linewidth} p{0.75\linewidth}}
			\foreignlanguage{hebrew}{יֵשׁ} & \textit{there is} (particle of existence) \\
			\foreignlanguage{hebrew}{אַיִן} & \textit{there is not} (particle of non-existence) \\
			\foreignlanguage{hebrew}{אֵצֶל} & \textit{in proximity to, beside} (prep.) \\
			\foreignlanguage{hebrew}{הַרְבֵּה} & \textit{greatly, exceedingly, much} (adv.; orig.\ inf.\ abs.\ Hi.\ of \foreignlanguage{hebrew}{רבה}) \\
			\foreignlanguage{hebrew}{לְבַד} & \textit{alone, by itself} (with ePP \foreignlanguage{hebrew}{לְבַדּוֺ}; prep.\ \foreignlanguage{hebrew}{לְ} with noun \foreignlanguage{hebrew}{בַּד} \textit{separation}) \\
			\foreignlanguage{hebrew}{מְעַט} & \textit{a little, fewness, a few} \\
			\foreignlanguage{hebrew}{רַק} & \textit{only, still, but, however, nevertheless, surely} \\
		\end{longtable}
	\end{center}
	
	\section{The waw-Suffix Conjugation or \textit{wəqatalṭí} Form}
	
	\subsection{The Forms of the waw-Suffix Conjugation}
	
	% Action point: check cross reference "cf. 6.2.5" in the following paragraph
	
	The waw-suffix conjugation or \textit{wəqatalṭí} form consists of the forms of the suffix conjugation with an obligatory conjunction \foreignlanguage{hebrew}{וְ} also called \textit{waw consecutive} (cf. 6.2.5). The only difference to the regular suffix conjugation (Chapter 5) is the position of the stress in the forms 2 m.\ sg.\ and 1 c.\ sg.
	
	\begin{center}
		\begin{tabular}{|ll|r|}
			\hline
			sg. & 3 m. & \foreignlanguage{hebrew}{וְכָתַב} \\
			& 3 f. & \foreignlanguage{hebrew}{וְכָתְבָה} \\
			& 2 m. & \foreignlanguage{hebrew}{וְכָתַבְתָּ֫} \\
			& 2 f. & \foreignlanguage{hebrew}{וְכָתַבְתְּ} \\
			& 1 c. & \foreignlanguage{hebrew}{וְכָתַבְתִּ֫י} \\
			pl. & 3 c. & \foreignlanguage{hebrew}{וְכָתְבוּ} \\
			& 2 m. & \foreignlanguage{hebrew}{וּכְתַבְתֶּם} \\
			& 2 f. & \foreignlanguage{hebrew}{וּכְתַבְתֶּן} \\
			& 1 c. & \foreignlanguage{hebrew}{וְכָתַ֫בְנוּ} \\
			\hline
		\end{tabular}
	\end{center} 
	
	% The fact that the shift of stress in the 2 m.\ sg.\ and 1 c.\ sg.\ forms did not result in vowel reduction is an indication that this phenomenon is rather late.
	
	\subsection{The Use of the  waw-Suffix Conjugation}
	
	The \textit{wəqaṭaltí} form typically continues a preceding verbal form or a temporal adverbial expression. It is, however, possible that \textit{wəqaṭaltí} is used without a semantically similar or identical preceding verb (cf.\ \textit{wəqaṭaltí} with modal meaning as in Judg 11:8; Jer 28:13; Ezek 11:17; Ezek 25:13; following a causal clause as in 1 Kgs 20:28; with reference to the past following a \textit{wayyiqṭol} form as in 1 Sam 7:16).
	
	\textit{wəqaṭaltí} often follows \textit{yiqṭol} and has the same meaning as the preceding \textit{yiqṭol}. A possible semantic difference between the two forms may be that \textit{wəqaṭaltí} adds the idea of sequence of events \textit{then \dots \space and then \dots}, whereas \textit{yiqṭol} simply refers to future time. The \textit{wəqaṭaltí} form may then refer to future events (1\,Sam 17:32), to iterative or durative situations in the past (1\,Sam 2:19; 1\,Sam 16:23) or to general truths (Gen 2:24).
	
	\vspace{0.5cm}
	
	\begin{longtable}{>{\raggedleft}p{0.35\linewidth} p{0.55\linewidth}}
		\foreignlanguage{hebrew}{עַבְדְּךָ יֵלֵךְ וְנִלְחַם עִם־הַפְּלִשְׁתִּי הַזֶּה} & \textit{Your servant will go and fight with this Philistine} (1\,Sam 17:32) \\
		\foreignlanguage{hebrew}{וּמְעִיל קָטֹן תַּעֲשֶׂה־לּוֹ אִמּוֹ וְהַעַלְתָה לוֹ מִיָּמִים יָמִימָה בַּעֲלוֹתָהּ אֶת־אִישָׁהּ לִזְבֹּחַ אֶת־זֶבַח הַיָּמִים} & \textit{And his mother used to make him a little robe and bring it up to him every year when she went up with her husband to offer the yearly sacrifice} (1\,Sam 2:19) \\
		\foreignlanguage{hebrew}{וְהָיָה בִּהְיוֹת רוּחַ־אֱלֹהִים אֶל־שָׁאוּל וְלָקַח דָּוִד אֶת־הַכִּנּוֹר וְנִגֵּן בְּיָדוֹ וְרָוַח לְשָׁאוּל וְטוֹב לוֹ וְסָ֫רָה מֵעָלָיו רוּחַ הָרָעָה} & \textit{And whenever the spirit from God came to Saul, David would take the lyre and play it with his hand and Saul would feel himself relieved and it would be well with him and it [i.e., the spirit] would depart from him} (1\,Sam 16:23) \\
		\foreignlanguage{hebrew}{עַל־כֵּן יַעֲזָב־אִישׁ אֶת־אָבִיו וְאֶת־אִמּוֹ וְדָבַק בְּאִשְׁתּוֹ וְהָיוּ לְבָשָׂר אֶחָד} & \textit{Therefore a man will leave his father and his mother and cling to his wife and they shall be one flesh} (Gen 2:24) \\
	\end{longtable}
	
	
	\textit{wəqaṭaltí} can follow a participle (Exod 17:6), an inf.\ cs.\ (Ruth 3:4), or a temporal adverbial expression (Exod 17:4). The verbal form may have modal meaning as in the case of \textit{wəqaṭaltí} following a volitive form.
	
	\vspace{0.5cm}
	
	% Example 1 Kgs 13:31 replaced with Ruth 3:4 (2024-11-25)
	
	\begin{tabular}{>{\raggedleft}p{0.35\linewidth} p{0.55\linewidth}}
		\foreignlanguage{hebrew}{הִנְנִי עֹמֵד לְפָנֶיךָ שָּׁם עַל־הַצּוּר בְּחֹרֵב וְהִכִּיתָ בַצּוּר וְיָצְאוּ מִמֶּנּוּ מַיִם וְשָׁתָה הָעָם} & \textit{I will be standing before you on the rock at Horeb and you shall strike the rock and water will come out from it and the people will drink} (Exod 17:6) \\
		\foreignlanguage{hebrew}{וִיהִי בְשָׁכְבוֹ וְיָדַעַתְּ אֶת־הַמָּקוֹם אֲשֶׁר יִשְׁכַּב־שָׁם} & \textit{When he lies down, note the place where he lies down} (Ruth 3:4) \\
		\foreignlanguage{hebrew}{עוֹד מְעַט וּסְקָלֻנִי} & \textit{A little more and they will stone me} (Exod 17:4 NAB) \\
	\end{tabular}
	
	\vspace{0.5cm}
	
	\textit{wəqaṭaltí} can follow a volitive form (i.e., imperative, jussive, cohortative).  \textit{wəqaṭaltí} has the modal meaning of the preceding verbal form (2\,Kgs 9:1b-2).
	
	\vspace{0.5cm}
	
	\begin{tabular}{>{\raggedleft}p{0.35\linewidth} p{0.55\linewidth}}
		\foreignlanguage{hebrew}{חֲגֹר מָתְנֶיךָ וְקַח פַּךְ הַשֶּׁמֶן הַזֶּה בְּיָדֶךָ וְלֵךְ רָמֹת גִּלְעָד וּבָאתָ שָׁמָּה וּרְאֵה־שָׁם יֵהוּא בֶן־יְהוֹשָׁפָט בֶּן־נִמְשִׁי וּבָאתָ וַהֲקֵמֹתוֹ מִתּוֹך אֶחָיו וְהֵבֵיאתָ אֹתוֹ חֶדֶר בְּחָדֶר} & \textit{Gird up your loins, take this flask of oil in your hand and go to Ramoth-gilead. And look there for Jehu, the son of Jehoshaphat, son of Nimshi and enter and raise him from the midst of his fellows and bring him to an inner chamber} (2\,Kgs 9:1b-2) \\
	\end{tabular}
	
	\vspace{0.5cm}
	
	It is possible, however, that \textit{wəqaṭaltí} is used with volitive meaning without a preceding volitive form (Judg 11:8).
	
	\vspace{0.5cm}
	
	\begin{tabular}{>{\raggedleft}p{0.35\linewidth} p{0.55\linewidth}}
		\foreignlanguage{hebrew}{לָכֵן עַתָּה שַׁבְנוּ אֵלֶיךָ וְהָלַכְתָּ עִמָּנוּ וְנִלְחַמְתָּ בִּבְנֵי עַמּוֹן וְהָיִיתָ לָּנוּ לְרֹאשׁ לְכֹל יֹשְׁבֵי גִלְעָד} & \textit{Therefore, now that we have returned to you, go with us and fight against the Ammonites and become for us the leader, for all inhabitants of Gilead} (Judg 11:8) \\
	\end{tabular}
	
	\vspace{0.5cm}
	
	As the \textit{wəqaṭaltí} form has to appear at the beginning of the clause without any preceding element, \textit{yiqṭol} has to be used whenever an element is placed before the verb (e.g., the negative \foreignlanguage{hebrew}{לֹא} or a constituent of the clause).
	
	\vspace{0.5cm}
	
	\begin{tabular}{>{\raggedleft}p{0.35\linewidth} p{0.55\linewidth}}
		\foreignlanguage{hebrew}{וּזְעַקְתֶּם בַּיּוֹם הַהוּא מִלִּפְנֵי מַלְכְּכֶם אֲשֶׁר בְּחַרְתֶּם לָכֶם וְלֹא־יַעֲנֶה יְהוָה אֶתְכֶם בַּיּוֹם הַהוּא} & \textit{\dots \space and you shall cry out on that day because of your king which you have chosen for yourselves, but the Lord will not answer you on that day} (1\,Sam 8:18) \\
	\end{tabular}
	
	
	\section{Enclitic Pronouns on Verbal Forms}
	
	\subsection{Analytic and Synthetic Constructions}
	Biblical Hebrew has two ways of connecting a pronoun as direct object to a verb. Either the object marker with an enclitic pronoun can be used (2\,Kgs 10:35) or the enclitic pronoun can be attached to the verbal form directly (2\,Kgs 13:9).
	
	% Other good verse pair with the verb \foreignlanguage{hebrew}{קבר}: 2\,Kgs 9:28 (analytic construction) and 2\,Kgs 23:30
	
	\vspace{0.5cm}
	
	\begin{tabular}{>{\raggedleft}p{0.35\linewidth} p{0.55\linewidth}}
		\foreignlanguage{hebrew}{וַיִּקְבְּרוּ אֹתוֹ בְּשֹׁמְרוֹן} & \textit{And they buried him in Samaria} (2\,Kgs 10:35) \\
		\foreignlanguage{hebrew}{וַיִּקְבְּרֻהוּ בְּשֹׁמְרוֹן} & \textit{And they buried him in Samaria} (2\,Kgs 13:9) \\
	\end{tabular}
	
	\vspace{0.5cm}
	
	These two clauses have the same meaning. The choice between the two constructions is purely stylistic. Overall, the synthetic construction with the enclitic pronoun directly attached to the verb is more frequent than the analytic construction with the object marker \foreignlanguage{hebrew}{אֵת}. An important difference between the two constructions is that the analytic construction makes fronting of the pronominal direct object possible (2\,Sam 2:7).
	
	\vspace{0.5cm}
	
	\begin{tabular}{>{\raggedleft}p{0.35\linewidth} p{0.55\linewidth}}
		\foreignlanguage{hebrew}{כִּי־מֵת אֲדֹנֵיכֶם שָׁאוּל וְגַם־אֹתִי מָשְׁחוּ בֵית־יְהוּדָה לְמֶלֶךְ עֲלֵיהֶֽם} & \textit{\dots \space for your lord Saul is dead and the House of Judah have already anointed me king over them} (2\,Sam 2:7 NJPS) \\
	\end{tabular}
	
	% Other example for fronted DO: Jer 2.13
	
	\subsection{The Forms of the Enclitic Personal Pronoun on Verbs}
	The enclitic pronouns on verbs are very similar to their counterparts on the infinitive construct or the noun. If a connecting vowel is used, it is /a/ (\textit{pataḥ}) or /ā/ (\textit{qameṣ}) with the suffix conjugation and /ē/ (\textit{ṣere}) with the prefix conjugation and the imperative.
	
	% Note: PC forms with the connecting vowel /a/ are attested, too (cf. GKC §60d). Action point: Add the adverb "usually" in the preceding sentence?
	
	\vspace{0.5cm}
	
	% For the 2mp, 2fp, 3fp ePP forms with SC and PC verbal forms no example was found with the strong verb (2024-11-08)
	% Examples of ePP 3ms with PC Jer 23:6; Hos 8:3; Hab 1:15; Job 3:4; Job 24:20
	% Examples of ePP 3fs with PC: Jer 36:23; Ps 46:6; Ps 109:19; forms with ePP -āh obviously only with waw-PC forms
	
	\begin{center}
		\begin{tabular}{lllrr}
			& & Suffix & \multicolumn{1}{c}{SC} & \multicolumn{1}{c}{PC} \\
			sg. & 1 c. & \textit{-nī} & \foreignlanguage{hebrew}{שְׁמָרַ֫נִי} & \foreignlanguage{hebrew}{יִשְׁמָעֵ֫נִי} ,\foreignlanguage{hebrew}{יִשְׁמְרֵ֫נִי} \\
			& 2 m. & \textit{-kā} & \foreignlanguage{hebrew}{שְׁמָרְךָ} & \foreignlanguage{hebrew}{יִשְׁאָלְךָ} ,\foreignlanguage{hebrew}{יִשְׁמָרְךָ} \\
			& 2 f. & \textit{-k} & \foreignlanguage{hebrew}{שְׁמָרֵךְ} & \foreignlanguage{hebrew}{יִגְאָלֵךְ} ,\foreignlanguage{hebrew}{יִשְׁמְרֵךְ} \\
			& 3 m. & \textit{-ō}/\textit{-hū} & \foreignlanguage{hebrew}{שְׁמָרוֺ} & \foreignlanguage{hebrew}{יִשְׁכָּחֵ֫הוּ} ,\foreignlanguage{hebrew}{יִשְׁמְרוֺ} \\
			& 3 f. & \textit{-āh}/\textit{-hā} & \foreignlanguage{hebrew}{שְׁמָרָהּ} & \foreignlanguage{hebrew}{יִקְרָעֶ֫הָ} ,\foreignlanguage{hebrew}{יִשְׁמְרֶהָ} \\
			& & & & \foreignlanguage{hebrew}{וַיִּשְׁמְרָהּ} \\
			pl. & 1 c. & \textit{-nū} & \foreignlanguage{hebrew}{שְׁמָרָ֫נוּ} & \foreignlanguage{hebrew}{יִפְגָּעֵ֫נוּ} ,\foreignlanguage{hebrew}{יִשְׁמְרֵ֫נוּ} \\
			& 2 m. & \textit{-kæm} & \foreignlanguage{hebrew}{שְׁמַרְכֶם}* & \foreignlanguage{hebrew}{יִשְׁאַלְכֶם}* ,\foreignlanguage{hebrew}{יִשְׁמָרְכֶם}* \\
			& 2 f. & \textit{*-kæn} & & \\
			& 3 m. & \textit{-m} & \foreignlanguage{hebrew}{שְׁמָרָם} & \foreignlanguage{hebrew}{יִמְאָסֵם} ,\foreignlanguage{hebrew}{יִשְׁמְרֵם} \\
			& 2 f. & \textit{*-n} & & \\
		\end{tabular}
	\end{center}
	
	\vspace{0.5cm}
	
	\noindent \textbf{Notes}
	\nopagebreak
	
	\noindent In PC forms with the thematic vowel /a/ the vowel is preserved as /ā/ (\textit{qameṣ}) before the ePP, while the /ō/ (\textit{ḥolem}) is usually reduced. In the 3 m.\ sg.\ PC form with ePP 2 m.\ sg.\ the \textit{ḥolem} is shortened to \textit{qameṣ ḥatuf}: \foreignlanguage{hebrew}{יִשְׁמָרְךָ} \textit{yišmɔrḵā}.
	
	In verbs with gutturals, the expected changes occur, i.e., \textit{ḥatef šwa} instead of simple \textit{šwa}, e.g., \foreignlanguage{hebrew}{עֲבָדְךָ} \textit{he served you} (Deut 15:18), \foreignlanguage{hebrew}{מְשָׁחֲךָ} \textit{he has anointed you} (1\,Sam 10:1).
	
	The ePP 3 m.\ sg.\ \textit{-ō} is the result of the development from \textit{--áhū} via elision (loss) of the intervocalic /h/ and contraction of the diphthong to its current form: \textit{-áhū} > \textit{-aw} > \textit{-ō}.
	
	Forms of the verb with ePP 2 f.\ pl.\ are not attested in the Hebrew Bible. The ePP would have been \textit{-kæn} \foreignlanguage{hebrew}{כֶן}-.
	
	The enclitic pronoun 3 f.\ pl.\ is \textit{-n} \foreignlanguage{hebrew}{ן}- (with connecting vowel, if needed). Forms are analogous to the forms with ePP 3 m.\ pl.
	
	\vspace{0.5cm}
	
	\noindent With verbal forms other than the 3 m.\ sg.\ the following differences can be observed:
	
	%Examples ‎עֲבָדוּהוּ (Jdg 10:6), ‎‎מְשַׁחְתִּיו (Ps 89:21), ‎דְרַשְׁנֻהוּ (1 Chr 15:13), ‎רְדָפוּךָ (Deut 30:7),
	%‎עֲבַדְתִּיךָ (Gen 30:29), ‎נְתַתִּיהָ (Gen 38:26), ‎שְׁלַחְתִּים (Jer. 14:14), ‎יְדַעְתּוֹ (Deut 22:2), ‎עֲזַבְתָּם (Neh 9:17), ‎נְטַשְׁתַּנִי (Gen 31:28), ‎שְׁלַחְתָּנוּ (Num 13:27), ‎נְתַתִּיהוּ (Ezek 16:19),‎ מְצָאתִים (Jer 2:34), ‎יְלִדְתִּנִי (Jer 15:10), יְלָדַתְנִי (Jer 20:14), ‎מְצָאַתְנוּ (Jdg 6:13), הֲרָגָתְהוּ (Jdg 9:54), ‎יְלָדַתּוּ (Ruth 4:15), הֶעֱלִיתֻנוּ (Num. 20:5)
	
	\begin{enumerate}
		\item In verbal forms ending with a long vowel there is no need for a connecting vowel.
		\begin{enumerate}[noitemsep]
			\item Verbal form 1 c.\ sg.\ SC: \foreignlanguage{hebrew}{מְשַׁחְתִּיו} \textit{I anointed him} (with elision of the intervocalic /h/ \textit{-tīhū > tīw}); \foreignlanguage{hebrew}{עֲבַדְתִּ֫יךָ} \textit{I served you}; \foreignlanguage{hebrew}{נְתַתִּ֫יהָ} \textit{I gave her}; \foreignlanguage{hebrew}{שְׁלַחְתִּים} \textit{I have sent them}
			\item Verbal form 3 c.\ pl.\ SC: \foreignlanguage{hebrew}{עֲבָד֫וּהוּ} \textit{they served him}; \foreignlanguage{hebrew}{רְדָפ֫וּךָ} \textit{they pursued you}
			\item Verbal form 1 c.\ pl.\ SC: \foreignlanguage{hebrew}{דְּרַשְׁנֻ֫הוּ} \textit{we sought him}
			\item Verbal form 3 m.\ pl.\ PC: \foreignlanguage{hebrew}{יִרְגְּמֻ֫הוּ} \textit{they shall stone him}
			\item Verbal form 2 f.\ sg.\ PC: \foreignlanguage{hebrew}{וַתִּזְבָּחִים} \textit{and you sacrificed them}
		\end{enumerate}
		\item In verbal forms 2 f.\ sg.\ SC the original suffix \textit{-tī} is preserved: \foreignlanguage{hebrew}{יְלִדְתִּ֫נִי} \textit{you bore me}, \foreignlanguage{hebrew}{נְתַתִּ֫יהוּ} \textit{you gave it}, \foreignlanguage{hebrew}{מְצָאתִים} \textit{you found them}
		\item In verbal forms 3 f.\ sg.\ SC the original suffix /t/ is preserved
		\begin{enumerate}[noitemsep]
			\item ePP 1 c.\ sg.: \foreignlanguage{hebrew}{יְלָדַ֫תְנִי} \textit{she gave birth to me}
			\item ePP 3 m.\ sg.: \foreignlanguage{hebrew}{הֲרָגָ֫תְהוּ} \textit{she killed him}; \foreignlanguage{hebrew}{יְלָדַ֫תּוּ} \textit{she gave birth to him} (with assimilation of the /h/ to the /t/ \textit{*-athu > -attū})
			\item ePP 3 f.\ sg.: \foreignlanguage{hebrew}{אֲחָזַ֫תָּה} \textit{it has taken hold of her} (with assimilation of the /h/ to the /t/ \textit{*-athā > -attā})
			\item ePP 1 c.\ pl.: \foreignlanguage{hebrew}{מְצָאַ֫תְנוּ} \textit{it found us}
			\item ePP 3 m.\ pl.: \foreignlanguage{hebrew}{גְּנָבָ֫תַם} \textit{she stole them}
		\end{enumerate}
		\item In verbal forms 2 m.\ sg.\ SC with the suffix \textit{-tā} and an ePP 3 m.\ sg.\ contraction happens, e.g., \foreignlanguage{hebrew}{יְדַעְתּוֹ} \textit{you know him} \textit{< -tō < *-taw < *-tahū} (with elision of the intervocalic /h/ and following contraction). Other enclitic pronouns simply follow the suffix \textit{-tā}, e.g., \foreignlanguage{hebrew}{נְטַשְׁתַּ֫נִי} \textit{you have permitted me}, \foreignlanguage{hebrew}{עֲזַבְתָּם} \textit{you forsook them}, \foreignlanguage{hebrew}{שְׁלַחְתָּ֫נוּ} \textit{you sent them}.
		\item In verbal forms 2 m.\ pl. SC the suffix is \textit{-tū} instead of \textit{-tæm}: \foreignlanguage{hebrew}{הֶעֱלִיתֻ֫נוּ} \textit{you have brought us up} (2 m.\ pl.\ SC Hiphil \foreignlanguage{hebrew}{עלה} with ePP 1 c.\ pl.).
	\end{enumerate}
	
	
	
	
	\subsection{The Enclitic Personal Pronoun with Energic Nun}
	In a number of cases an additional \foreignlanguage{hebrew}{נ} is used between the verbal form and the enclitic pronoun. This \foreignlanguage{hebrew}{נ} is called \textit{energic Nun} and is considered a residue of an early energic mode (cf.\ Chapter 6). It is almost always used in PC forms without suffix. Instances of the energic Nun with the jussive, the \textit{wayyiqṭol} form or the imperative are infrequent. The energic Nun does not change the meaning of the form when compared to the forms without. The following forms of the ePP with energic Nun are attested.
	
	\vspace{0.25cm}
	
	\begin{center}
		\begin{tabular}{llr}
			sg. & 3 m. & \foreignlanguage{hebrew}{יִשְׁמְרֶ֫נּוּ} \\
			& 3 f. & \foreignlanguage{hebrew}{יִשְׁמְרֶ֫נָּה} \\
			& 2 m. & \foreignlanguage{hebrew}{יִשְׁמְרֶ֫ךָּ} \\
			& 1 c. & \foreignlanguage{hebrew}{יִשְׁמְרֶ֫נִּי} \\
			pl. & 1 c. & \foreignlanguage{hebrew}{יִשְׁמְרֶ֫נּוּ} \\
		\end{tabular}
	\end{center}
	
	\vspace{0.5cm}
	
	\noindent \textbf{Notes}
	\nopagebreak
	
	\noindent The gemination of the /n/ in the forms with 3 m./f.\ sg. ePP is the result of assimilation of the /h/ of the ePP to the preceding /n/: \textit{-ænnū < *-ænhū} and \textit{-ænnā < *-ænhā}, respectively.
	
	The gemination of the /n/ in the forms with 2 m.\ sg. ePP is the result of assimilation of the /n/ to the following /k/ of the ePP: \textit{-ækkā < *-ænkā}.
	
	In forms of the PC with the thematic vowel /a/, the vowel is preserved as \textit{qameṣ}, e.g., \foreignlanguage{hebrew}{יִשְׁמָעֶ֫ךָּ} \textit{he will hear you}.
	
	
	
	\section{Existential Clauses}
	To make statements about the existence or non-existence of somebody or something Hebrew uses the particles \foreignlanguage{hebrew}{יֵשׁ}  \textit{there is} and \foreignlanguage{hebrew}{אַיִן} \textit{there is not} which is used in the cs.\ st.\ \foreignlanguage{hebrew}{אֵין} most of the time.
	
	
	\begin{longtable}{>{\raggedleft}p{0.35\linewidth} p{0.55\linewidth}}
		\foreignlanguage{hebrew}{וַיַּרְא יַעֲקֹב כִּי יֶשׁ־שֶׁבֶר בְּמִצְרָיִם} & \textit{And Jacob saw that there was grain in Egypt} (Gen 42:1) \\
		\foreignlanguage{hebrew}{יָבֹא־נָא אֵלַי וְיֵדַע כִּי יֵשׁ נָבִיא בְּיִשְׂרָאֵל} & \textit{Let him come to me, that he may know that there is a prophet in Israel} (2\,Kgs 5:8) \\
		\foreignlanguage{hebrew}{וַיִּקָּחֻהוּ וַיַּשְׁלִכוּ אֹתוֹ הַבֹּ֫רָה וְהַבּוֹר רֵק אֵין בּוֹ מָיִם} & \textit{And they took him and threw him into the cistern; the cistern was empty, there was no water in it} (Gen 37:24) \\
		\foreignlanguage{hebrew}{וַיִּפֶן כֹּה וָכֹה וַיַּרְא כִּי אֵין אִישׁ} & \textit{And he turned this way and that way and saw that there was no one} (Exod 2:12) \\
	\end{longtable}
	
	\vspace{0.5cm}
	
	With the preposition \foreignlanguage{hebrew}{לְ} the particle \foreignlanguage{hebrew}{יֵשׁ} and its negative counterpart \foreignlanguage{hebrew}{אַיִן} express the idea that somebody possesses something or something is available to somebody.
	
	\vspace{0.25cm}
	
	\begin{center}
		\begin{tabular}{>{\raggedleft}p{0.35\linewidth} p{0.55\linewidth}}
			\foreignlanguage{hebrew}{וַיֹּאמֶר עֵשָׂו יֶשׁ־לִי רָב אָחִי} & \textit{And Esau said, \enquote{I have plenty, my brother}} (Gen 33:9) \\
			\foreignlanguage{hebrew}{וְיֵדְעוּ כָּל־הָאָרֶץ כִּי יֵשׁ אֱלֹהִים לְיִשְׂרָאֵל} & \textit{\dots \space so that all the earth may know that Israel has a God} (1\,Sam 17:46) \\
			\foreignlanguage{hebrew}{וַתְּהִי שָׂרַי עֲקָרָה אֵין לָהּ וָלָד} & \textit{Sarai was barren; she did not have a child} (Gen 11:30) \\
			\foreignlanguage{hebrew}{וַתֹּאמֶר אֵין לְשִׁפְחָתְךָ כֹל בַּבַּיִת כִּי אִם־אָסוּךְ שָׁמֶן} & \textit{And she said, \enquote{Your maidservant does not have anything in the house except a flask of oil}} (2\,Kgs 4:2) \\
		\end{tabular}
	\end{center}
	
	
	\section{Particles with Enclitic Pronouns}
	A number of particles can be used with enclitic pronouns. In some forms, an additional \foreignlanguage{hebrew}{נ} is found which is similar to the energic Nun with verbal forms.
	
	\vspace{0.5cm}
	
	\begin{center}
		\begin{longtable}{llrrrr}
			& & \multicolumn{1}{c}{\foreignlanguage{hebrew}{הִנֵּה}} & \multicolumn{1}{c}{\foreignlanguage{hebrew}{יֵשׁ}} & \multicolumn{1}{c}{\foreignlanguage{hebrew}{אַיִן}} & \multicolumn{1}{c}{\foreignlanguage{hebrew}{עוֺד}} \\
			\endhead
			\endfoot
			sg. & 1 c. & \foreignlanguage{hebrew}{הִנֶּ֫נִּי}/\foreignlanguage{hebrew}{הִנְנִי} & \foreignlanguage{hebrew}{} & \foreignlanguage{hebrew}{אֵינֶ֫נִּי} & \foreignlanguage{hebrew}{עוֺדֶ֫נִּי}/\foreignlanguage{hebrew}{עוֺדִי} \\
			& 2 m. & \foreignlanguage{hebrew}{הִנְּךָ} & \foreignlanguage{hebrew}{יֶשְׁךָ} & \foreignlanguage{hebrew}{אֵינְךָ} & \foreignlanguage{hebrew}{עוֺדְךָ} \\
			& 2 f. & \foreignlanguage{hebrew}{הִנָּךְ} & & \foreignlanguage{hebrew}{אֵינֵךְ} & \foreignlanguage{hebrew}{עוֺדָךְ} \\
			& 3 m. & \foreignlanguage{hebrew}{הִנּוֺ} & \foreignlanguage{hebrew}{יֶשְׁנוֹ} & \foreignlanguage{hebrew}{אֵינֶ֫נּוּ} & \foreignlanguage{hebrew}{עוֺדֶ֫נּוּ} \\
			& 3 f. & & & \foreignlanguage{hebrew}{אֵינֶ֫נָּה} & \foreignlanguage{hebrew}{עוֺדֶ֫נָּה}/\foreignlanguage{hebrew}{עוֺדָהּ} \\
			\newpage
			pl. & 1 c. & \foreignlanguage{hebrew}{הִנֶּ֫נּוּ}/\foreignlanguage{hebrew}{הִנְּנוּ} & & \foreignlanguage{hebrew}{אֵינֶ֫נּוּ} & \\
			& 2 m. & \foreignlanguage{hebrew}{הִנְּכֶם} & \foreignlanguage{hebrew}{יֶשְׁכֶם} & \foreignlanguage{hebrew}{אֵינְכֶם} & \\
			& 3 m. & \foreignlanguage{hebrew}{הִנָּם} & & \foreignlanguage{hebrew}{אֵינָם} & \foreignlanguage{hebrew}{עוֺדָם} \\
		\end{longtable}
	\end{center}
	
	\noindent \textbf{Notes}
	\nopagebreak
	
	\noindent Distinct pausal forms are \foreignlanguage{hebrew}{הִנֵּ֑נִי} (ePP 1 c.\ sg.), \foreignlanguage{hebrew}{הִנֶּ֑ךָּ} (ePP 2 m.\ sg.) \foreignlanguage{hebrew}{הִנֵּ֑נוּ} (ePP 1 c.\ pl.).
	
	Alternative forms are \foreignlanguage{hebrew}{הִנֵּ֫הוּ}  (Jer 18:3 Ketiv) instead of \foreignlanguage{hebrew}{הִנּוֺ} and \foreignlanguage{hebrew}{יִשְׁכֶם} (Deut 13:14) instead of \foreignlanguage{hebrew}{יֶשְׁכֶם}.
	
	\section{Negative Nominal Clauses}
	In order to form negative nominal clauses, Hebrew uses the negative \foreignlanguage{hebrew}{אַיִן} or, more frequently, its cs.\ st.\ form \foreignlanguage{hebrew}{אֵין}. The negative \foreignlanguage{hebrew}{לֹא}, however, is also used for this purpose.
	
	\vspace{0.5cm}
	
	% The translation of Deut 4.22 needs to be improved; 2 Sam 18:20, too.
	
	\begin{tabular}{>{\raggedleft}p{0.35\linewidth} p{0.55\linewidth}}
		\foreignlanguage{hebrew}{אַל־תַּעֲלוּ כִּי אֵין יְהוָה בְּקִרְבְּכֶם} & \textit{Do not go up, for the \textsc{lord} is not in your midst} (Num 14:42) \\
		\foreignlanguage{hebrew}{כִּי אָנֹכִי מֵת בָּאָרֶץ הַזֹּאת אֵינֶנִּי עֹבֵר אֶת־הַיַּרְדֵּן} & \textit{For I am about to die in this land; I won't cross the Jordan} (Deut 4:22) \\
		\foreignlanguage{hebrew}{לֹא־עֵת הֵאָסֵף הַמִּקְנֶה} & \textit{It is not the time for the livestock to be gathered} (Gen 29:7) \\
		\foreignlanguage{hebrew}{לֹא אִישׁ בְּשֹׂרָה אַתָּה הַיּוֹם הַזֶּה} & \textit{You are not a man of news today} (2\,Sam 18:20) \\
	\end{tabular}
	
	
	\section{Exercises}
	
	\subsection{Translation of Sentences}
	
	Translate the following sentences from the Hebrew Bible. Names of persons and geographical names in these sentences: \foreignlanguage{hebrew}{אַבְרָהָם}, \foreignlanguage{hebrew}{אַבְשָׁלוֹם}, \foreignlanguage{hebrew}{אַחְאָב}, \foreignlanguage{hebrew}{בֵּית־עֵקֶד}, \foreignlanguage{hebrew}{דָּוִד}, \foreignlanguage{hebrew}{יֵהוּא}, \foreignlanguage{hebrew}{יְהוֹאָחָז}, \foreignlanguage{hebrew}{יְהוֹשֻׁעַ}, \foreignlanguage{hebrew}{יוֹאָשׁ}, \foreignlanguage{hebrew}{יוֹסֵף}, \foreignlanguage{hebrew}{יַעֲקֹב}, \foreignlanguage{hebrew}{יִשַׁי}, \foreignlanguage{hebrew}{מֹשֶׁה}, \foreignlanguage{hebrew}{סֻכּוֹת}, \foreignlanguage{hebrew}{רְאוּבֵן}.
	
	\vspace{0.5cm}
	
	\selectlanguage{hebrew}
	\noindent
	1~~\foreignlanguage{hebrew}{וַיִּקְבֹּץ יֵהוּא אֶת־כָּל־הָעָם וַיֹּאמֶר אֲלֵהֶם אַחְאָב עָבַד אֶת־הַבַּעַל מְעָט יֵהוּא יַעַבְדֶנּוּ הַרְבֵּה}  \hspace{0.5cm}
	2~~\foreignlanguage{hebrew}{וַיְהִי אַחֲרֵי קָבְרוֹ אֹתוֹ וַיֹּאמֶר אֶל־בָּנָיו לֵאמֹר בְּמוֹתִי וּקְבַרְתֶּם אֹתִי בַּקֶּבֶר אֲשֶׁר אִישׁ הָאֱלֹהִים קָבוּר בּוֹ אֵצֶל עַצְמֹתָיו הַנִּ֫יחוּ}\LTRfootnote{\space \foreignlanguage{hebrew}{הַנִּיחוּ} \textit{put!} (impv.)} \foreignlanguage{hebrew}{אֶת־עַצְמֹתָי} \hspace{0.5cm}
	3~~\foreignlanguage{hebrew}{וַיִּשְׁכַּב יְהוֹאָחָז עִם־אֲבֹתָיו וַיִּקְבְּרֻהוּ בְּשֹׁמְרוֹן וַיִּמְלֹךְ יוֹאָשׁ בְּנוֹ תַּחְתָּיו}  \hspace{0.5cm}
	4~~\foreignlanguage{hebrew}{וַיִּקְרָא יְהוָה אֶל־שְׁמוּאֵל וַיֹּאמֶר הִנֵּ֑נִי}  \hspace{0.5cm}
	5~~\foreignlanguage{hebrew}{וַיִּשְׁלַח אַבְשָׁלוֹם מְרַגְּלִים}\LTRfootnote{\space \foreignlanguage{hebrew}{מְרַגְּלִים} \textit{spies} (participle Piel m.\ sg.)} \foreignlanguage{hebrew}{בְּכָל־שִׁבְטֵי יִשְׂרָאֵל לֵאמֹר כְּשָׁמְעֲכֶם אֶת־קוֹל הַשֹּׁפָר וַאֲמַרְתֶּם מָלַךְ אַבְשָׁלוֹם בְּחֶבְרוֹן} \hspace{0.5cm}
	6~~\foreignlanguage{hebrew}{וַיֹּאמֶר הִנֵּה שָׁמַעְתִּי כִּי יֶשׁ־שֶׁבֶר}\LTRfootnote{\space \foreignlanguage{hebrew}{שֶׁבֶר} \textit{grain}} \foreignlanguage{hebrew}{בְּמִצְרָיִם} \hspace{0.5cm}
	7~~\foreignlanguage{hebrew}{וַיָּ֫שָׁב}\LTRfootnote{\space \foreignlanguage{hebrew}{וַיָּ֫שָׁב} \textit{wayyā́šɔḇ} \textit{and he returned} (verb \foreignlanguage{hebrew}{שׁוב})} \foreignlanguage{hebrew}{רְאוּבֵן אֶל־הַבּוֹר וְהִנֵּה אֵין־יוֹסֵף בַּבּוֹר וַיִּקְרַע אֶת־בְּגָדָיו} \hspace{0.5cm}
	8~~\foreignlanguage{hebrew}{וַיֹּאמֶר אַבְרָהָם כִּי}\LTRfootnote{\space \foreignlanguage{hebrew}{כִּי} here functioning as introduction of the direct speech; not to be translated (cf.\ Greek \foreignlanguage{greek}{ὅτι}).} \foreignlanguage{hebrew}{אָמַרְתִּי רַק אֵין־יִרְאַת}\LTRfootnote{\space \foreignlanguage{hebrew}{יִרְאָה} \textit{fear, reverence}} \foreignlanguage{hebrew}{אֱלֹהִים בַּמָּקוֹם הַזֶּה וַהֲרָגוּנִי עַל־דְּבַר}\LTRfootnote{\space \foreignlanguage{hebrew}{עַל־דְּבַר} \textit{because of}} \foreignlanguage{hebrew}{אִשְׁתִּי} \hspace{0.1cm}
	9~~\foreignlanguage{hebrew}{כָּל־מָקוֹם אֲשֶׁר תִּדְרֹךְ כַּף־רַגְלְכֶם בּוֹ – לָכֶם נְתַתִּיו כַּאֲשֶׁר דִּבַּרְתִּי אֶל־מֹשֶׁה}  \hspace{0.5cm}
	10~~\foreignlanguage{hebrew}{וַיִּלְכָּד־נַעַר מֵאַנְשֵׁי סֻכּוֹת וַיִּשְׁאָלֵהוּ וַיִּכְתֹּב אֵלָיו אֶת־שָׂרֵי סֻכּוֹת וְאֶת־זְקֵנֶיהָ שִׁבְעִים וְשִׁבְעָה אִישׁ}\LTRfootnote{\space \foreignlanguage{hebrew}{שִׁבְעִים וְשִׁבְעָה אִישׁ} \textit{seventy-seven men} (The counted noun follows the numeral here in the singular.)} \hspace{0.5cm}
	11~~\foreignlanguage{hebrew}{וַיְהִי בַבֹּקֶר וַיֵּצֵא וַיַּעֲמֹד וַיֹּאמֶר אֶל־כָּל־הָעָם צַדִּקִים אַתֶּם הִנֵּה אֲנִי קָשַׁרְתִּי}\LTRfootnote{\space \foreignlanguage{hebrew}{קָשַׁרְתִּי} \textit{I have conspired}} \foreignlanguage{hebrew}{עַל־אֲדֹנִי וָאֶהְרְגֵהוּ} \hspace{0.1cm}
	12~~\foreignlanguage{hebrew}{וַיַּעַשׂ לָהֶם כֵּן וַיַּצֵּל}\LTRfootnote{\space \foreignlanguage{hebrew}{וַיַּצֵּל} \textit{and he rescued}} \foreignlanguage{hebrew}{אוֹתָם מִיַּד בְּנֵי־יִשְׂרָאֵל וְלֹא הֲרָגוּם} \hspace{0.5cm}
	13~~\foreignlanguage{hebrew}{וְעַתָּה כֹּה־תֹאמַר}\LTRfootnote{\space \foreignlanguage{hebrew}{תֹאמַר} \textit{you shall say}} \foreignlanguage{hebrew}{לְעַבְדִּי לְדָוִד כֹּה אָמַר יְהוָה צְבָאוֹת אֲנִי לְקַחְתִּיךָ מִן־הַנָּוֶה}\LTRfootnote{\space \foreignlanguage{hebrew}{נָוֶה} \textit{grazing place}} \foreignlanguage{hebrew}{מֵאַחַר הַצֹּאן לִהְיוֹת נָגִיד}\LTRfootnote{\space \foreignlanguage{hebrew}{לִהְיוֹת נָגִיד} \textit{so that [you] become ruler}} \foreignlanguage{hebrew}{עַל־עַמִּי עַל־יִשְׂרָאֵל} \hspace{0.5cm}
	14~~\foreignlanguage{hebrew}{וַיֹּאמֶר יְהוָה אֶל־יְהוֹשֻׁעַ אַל־תִּירָא}\LTRfootnote{\space \foreignlanguage{hebrew}{אַל־תִּירָא} \textit{do not be afraid} (often with the prep.\ \foreignlanguage{hebrew}{מִן} to express the object of the verb)} \foreignlanguage{hebrew}{ מֵהֶם כִּי בְיָדְךָ נְתַתִּים לֹא־יַעֲמֹד אִישׁ מֵהֶם בְּפָנֶיךָ} \hspace{0.5cm}
	15~~\foreignlanguage{hebrew}{וַיֹּאמֶר תִּפְשׂוּם חַיִּים וַיִּתְפְּשׂוּם חַיִּים וַיִּשְׁחָטוּם אֶל־בּוֹר בֵּית־עֵקֶד אַרְבָּעִים וּשְׁנַיִם אִישׁ}\LTRfootnote{\space \foreignlanguage{hebrew}{אַרְבָּעִים וּשְׁנַיִם אִישׁ} \textit{forty men}} \foreignlanguage{hebrew}{וְלֹא־הִשְׁאִיר}\LTRfootnote{\space \foreignlanguage{hebrew}{הִשְׁאִיר} \textit{he spared, left over}} \foreignlanguage{hebrew}{אִישׁ מֵהֶם} \hspace{0.5cm}
	16~~\foreignlanguage{hebrew}{וַיֹּאמֶר אֵין־לָנוּ חֵלֶק בְּדָוִד וְלֹא נַחֲלָה־לָנוּ בְּבֶן־יִשַׁי }  \hspace{0.3cm}
	17~~\foreignlanguage{hebrew}{וְהַלֵּוִי}\LTRfootnote{\space \foreignlanguage{hebrew}{לֵוִי} \textit{Levite} (cf.\ the proper noun \foreignlanguage{hebrew}{לֵוִי})} \foreignlanguage{hebrew}{אֲשֶׁר־בִּשְׁעָרֶיךָ לֹא תַעַזְבֶנּוּ כִּי אֵין לוֹ חֵלֶק וְנַחֲלָה עִמָּךְ} \hspace{0.5cm}
	18~~\foreignlanguage{hebrew}{וַיִּיקַץ יַעֲקֹב}\LTRfootnote{\space \foreignlanguage{hebrew}{וַיִּיקַץ יַעֲקֹב} \textit{And Jacob woke up}} \foreignlanguage{hebrew}{מִשְּׁנָתוֹ}\LTRfootnote{\space \foreignlanguage{hebrew}{שֵׁנָה} \textit{sleep}} \foreignlanguage{hebrew}{וַיֹּאמֶר אָכֵן}\LTRfootnote{\space \foreignlanguage{hebrew}{אָכֵן} \textit{surely, truly}} \foreignlanguage{hebrew}{יֵשׁ יְהוָה בַּמָּקוֹם הַזֶּה וְאָנֹכִי לֹא יָדָעְתִּי} \hspace{0.5cm}
	19~~\foreignlanguage{hebrew}{וְלֹא אִתְּכֶם לְבַדְּכֶם אָנֹכִי כֹּרֵת אֶת־הַבְּרִית הַזֹּאת וְאֶת־הָאָלָה}\LTRfootnote{\space \foreignlanguage{hebrew}{אָלָה} \textit{curse}} \foreignlanguage{hebrew}{הַזֹּֽאת. כִּי אֶת־אֲשֶׁר יֶשְׁנוֹ פֹּה עִמָּנוּ עֹמֵד הַיּוֹם לִפְנֵי יְהוָה אֱלֹהֵינוּ וְאֵת אֲשֶׁר אֵינֶנּוּ פֹּה עִמָּנוּ הַיּֽוֹם.} \hspace{0.5cm}
	20~~\foreignlanguage{hebrew}{הִנֵּה יָמִים בָּאִים}\LTRfootnote{\space \foreignlanguage{hebrew}{בָּאִים} \textit{are coming} (part. m.\ pl.\ Q.\ \foreignlanguage{hebrew}{בוא})} \foreignlanguage{hebrew}{נְאֻם־יְהוָה וְכָרַתִּי אֶת־בֵּית יִשְׂרָאֵל וְאֶת־בֵּית יְהוּדָה בְּרִית חֲדָשָֽׁה. לֹא כַבְּרִית אֲשֶׁר כָּרַתִּי אֶת־אֲבוֹתָם בְּיוֹם הֶחֱזִיקִי בְיָדָם לְהוֹצִיאָם}\LTRfootnote{\space \foreignlanguage{hebrew}{בְּיוֹם הֶחֱזִיקִי בְיָדָם לְהוֹצִיאָם} lit. \textit{on the day when I took them by their hand to bring them out}} \foreignlanguage{hebrew}{מֵאֶרֶץ מִצְרָיִם \space \dots \space כִּי זֹאת הַבְּרִית אֲשֶׁר אֶכְרֹת אֶת־בֵּית יִשְׂרָאֵל אַחֲרֵי הַיָּמִים הָהֵם נְאֻם־יְהוָה נָתַתִּי אֶת־תּוֹרָתִי בְּקִרְבָּם וְעַל־לִבָּם אֶכְתֲּבֶנָּה וְהָיִיתִי לָהֶם לֵאלֹהִים וְהֵמָּה יִהְיוּ־לִי לְעָֽם.}\LTRfootnote{\space \foreignlanguage{hebrew}{וְהָיִיתִי לָהֶם לֵאלֹהִים וְהֵמָּה יִהְיוּ־לִי לְעָם} \hspace{0.2cm} \textit{\dots \space and I shall be their God and they shall be my people}} \hspace{0.3cm}
	\selectlanguage{english}
	
	
	\chapter{Chapter 11}
	
	\renewcommand\arraystretch{1.4}
	
	\section{Vocabulary}
	
	\subsection{Verbs}
	
	\begin{center}
		
		% For the centering of the separation between the two columns see the documentation of the array package, page 2 
		
		\begin{tabular}{>{\raggedleft}p{0.175\linewidth} p{0.75\linewidth}}
			\foreignlanguage{hebrew}{אבד} & Q.\ \textit{to perish} \\
			\foreignlanguage{hebrew}{אהב} & Q.\ \textit{to love} (stative verb, pausal form SC \foreignlanguage{hebrew}{אָהֵ֑ב}) \\
			\foreignlanguage{hebrew}{אחז} & Q.\ \textit{to grasp, take hold, take possession} \\
			\foreignlanguage{hebrew}{אסף} & Q.\ \textit{to gather, remove} \\
			\foreignlanguage{hebrew}{אסר} & Q.\ \textit{to tie, bind, imprison} \\
			\foreignlanguage{hebrew}{מכר} & Q.\ \textit{to sell} \\
			\foreignlanguage{hebrew}{משׁל} & Q.\ \textit{to rule, reign} (with prepositional object with \foreignlanguage{hebrew}{בְּ}) \\
			\foreignlanguage{hebrew}{נגע} & Q.\ \textit{to touch, reach, strike} (usually with prep. object with \foreignlanguage{hebrew}{בְּ}) \\
			\foreignlanguage{hebrew}{נגשׁ} & Q.\ \textit{to draw near, approach} (in Qal only PC, impv., inf.\ cs.) \\
			\foreignlanguage{hebrew}{נסע} & Q.\ \textit{to pull up or out, set out, journey} \\
			\foreignlanguage{hebrew}{נצר} & Q.\ \textit{to watch, guard, keep} \\
			\foreignlanguage{hebrew}{שׁכח} & Q.\ \textit{to forget} \\
		\end{tabular}
	\end{center}
	
	\subsection{Nouns}
	
	\begin{center}
		\begin{longtable}{>{\raggedleft}p{0.175\linewidth} p{0.75\linewidth}}
			\foreignlanguage{hebrew}{אֶבֶן} & \textit{stone} (f., pl.\ \foreignlanguage{hebrew}{אֲבָנִים})\\
			\foreignlanguage{hebrew}{אֲחֻזָּה} & \textit{possession, property} \\
			\foreignlanguage{hebrew}{גּוֺי} & \textit{nation, people} (pl.\ \foreignlanguage{hebrew}{גּוֺיִם})\\
			\foreignlanguage{hebrew}{זֶבַח} & \textit{sacrifice} \\
			\foreignlanguage{hebrew}{מִשְׁכָּן} & \textit{dwelling-place, abode} \\
			\foreignlanguage{hebrew}{מִשְׁפָּחָה} & \textit{extended family, clan} \\ % HALOT
			\foreignlanguage{hebrew}{נֶגֶב} & \textit{south-country, Negev, south} \\
			\foreignlanguage{hebrew}{נֶגַע} & \textit{stroke, plague, blow, mark} \\
			\foreignlanguage{hebrew}{עֵץ} & \textit{tree, wood}, sometimes \textit{trees} (coll.) (pl.\ \foreignlanguage{hebrew}{עֵצִים}) \\
			\foreignlanguage{hebrew}{שַׁעַר} & \textit{gate} \\ % Already included in Chapter 10; can be deleted
			\foreignlanguage{hebrew}{תּוֺרָה} & \textit{instruction} \\
		\end{longtable}
	\end{center}
	
	\subsection{Other Parts of Speech}
	
	\begin{center}
		\begin{tabular}{>{\raggedleft}p{0.175\linewidth} p{0.75\linewidth}}
			\foreignlanguage{hebrew}{אַף} & \textit{also, even} (adv.) \\
		\end{tabular}
	\end{center}
	
	
	\section{Weak Verbs in Biblical Hebrew}
	
	\noindent Strong verbs are verbs that have three root consonants in all \emph{all} their forms. Weak verbs, on the other hand, are verbs that do not have three root consonants in all their forms, either because one of the three root consonants is weak and prone to be lost or because they do not have more than two root consonants.
	
	The weak root consonant can be found in first position in the root or in third position. Verbal roots with two weak root consonants are called \emph{doubly weak} verbs. Verbal roots with a vocalic element in the middle as weakness -- the II\,w/y verbs in the table below -- are verbs with only two root consonants. These verbal roots are also called \emph{hollow roots}. The geminate verb with a doubled second root consonant is a case apart. 
	
	The classes of weak verbs are identified by different names in Hebrew grammars. The Hebrew designation of weak verbal roots is based on the root \foreignlanguage{hebrew}{פעל} \textit{to do, make}. The \foreignlanguage{hebrew}{פ} represents the first, the \foreignlanguage{hebrew}{ע} the second and the \foreignlanguage{hebrew}{ל} the third root consonant. In this textbook the romanized designations are used.
	
	% Reason: The romanized designations agree in form with designations like II\.gutt.
	
	\vspace{0.5cm}
	
	\renewcommand\arraystretch{1.4}
	
	\begin{Center}
		\begin{tabular}{|r|ll|}
			\hline
			Hebrew & \multicolumn{2}{c|}{Romanized} \\
			\hline
			\foreignlanguage{hebrew}{פ״א} & I-\textit{ʾālep̄} & I\,\textit{ʾ} \\
			\foreignlanguage{hebrew}{פ״י} & I-\textit{yōd} & I\,\textit{y}  \\
			\foreignlanguage{hebrew}{פ״נ} & I-\textit{nūn} & I\,\textit{n}  \\
			\foreignlanguage{hebrew}{ע״ו/י} & II-\textit{wāw}/\textit{yōd} & II\,\textit{w/y}\\
			\foreignlanguage{hebrew}{ע״ע} & II geminate & II gem.  \\
			\foreignlanguage{hebrew}{ל״א} & III-\textit{ʾālep̄} & III\,\textit{ʾ}  \\
			\foreignlanguage{hebrew}{ל״ה} & III-\textit{hē} & III\,\textit{y}  \\
			\hline
		\end{tabular}
	\end{Center}
	
	
	\section{Weak Verbs: I \textit{n} Verbs}
	
	I\,\textit{n} verbs have /n/ as the first root consonant. The consonant /n/ is weak because it is assimilated to the following consonant if this follows the /n/ immediately, e.g., \textit{-nC- > -CC-}. The following forms of I\,\textit{n} verbs in the Qal stem are weak.
	
	\begin{enumerate}[noitemsep]
		\item The forms of the prefix conjugation including its related forms, the jussive and the cohortative, because of the assimilation of the /n/ to the following second root consonant: \textit{*yinC\textsubscript{2}VC\textsubscript{3} > yiC\textsubscript{2}C\textsubscript{2}VC\textsubscript{3}}, e.g., \textit{*yinpōl > yippōl} \foreignlanguage{hebrew}{יִפֹּל} \textit{he will fall}, \textit{*yingaš > yiggaš} \foreignlanguage{hebrew}{יִגַּשׁ} \textit{he will approach}. In these forms the /n/ does not appear as a consonant anymore, but only in the \textit{dageš forte} in the second root consonant. The assimilation, however, of the /n/ does \emph{not} happen, when the second root consonant is a guttural, e.g., \foreignlanguage{hebrew}{יִנְחַל} \textit{he will inherit}.
		\item The imperative and the inf.\ cs.\ of I\,\textit{n} verbs with /a/ as the thematic vowel in the prefix conjugation and the imperative and inf.\ cs.\ of the verb \foreignlanguage{hebrew}{נתן} \textit{to give} are weak forms. In these forms the /n/ is lost completely.
	\end{enumerate}
	
	The very frequent verb \foreignlanguage{hebrew}{נתן} \textit{to give} has the exceptional vowel pattern \textit{a--i} in the prefix conjugation Qal: \textit{*yattin > yittēn} \foreignlanguage{hebrew}{יִתֵּן} \textit{he will give} (with attenuation of \textit{a > i} in closed unstressed syllable).
	
	The very frequent verb \foreignlanguage{hebrew}{לקח} \textit{to take} behaves like a I\,\textit{n} verb. The prefix conjugation forms, the imperative and the inf.\ cs.\ are weak.
	
	
	\renewcommand\arraystretch{1.4}
	
	% The following forms are not attested in the Hebrew Bible:
	% verb נפל PC 2fs, 1cp, 2mp, 2fp
	% verb נתן PC 2fp
	% verb לקח PC 2fp
	
	
	\begin{center}
		\begin{longtable}{|lll|r|r|r|r|}
			\hline
			\multicolumn{3}{|c|}{} & \foreignlanguage{hebrew}{נפל} & \foreignlanguage{hebrew}{נגשׁ} & \foreignlanguage{hebrew}{נתן} &  \foreignlanguage{hebrew}{לקח} \\
			\hline
			SC & sg. & 3 m. & \foreignlanguage{hebrew}{נָפַל} & & \foreignlanguage{hebrew}{נָתַן} & \foreignlanguage{hebrew}{לָקַח} \\
			& & 3 f. & \foreignlanguage{hebrew}{נָפְלָה} & & \foreignlanguage{hebrew}{נָתְנָה} & \foreignlanguage{hebrew}{לָקְחָה} \\
			& & 2 m. & \foreignlanguage{hebrew}{נָפַלְתָּ} & & \foreignlanguage{hebrew}{נָתַתָּה} & \foreignlanguage{hebrew}{לָקַחְתָּ} \\
			& & 2 f. & \foreignlanguage{hebrew}{נָפַלְתְּ} & & \foreignlanguage{hebrew}{נָתַתְּ} & \foreignlanguage{hebrew}{לָקַחַתְּ} \\
			& & 1 c. & \foreignlanguage{hebrew}{נָפַלְתִּי} & & \foreignlanguage{hebrew}{נָתַתִּי} & \foreignlanguage{hebrew}{לָקַחְתִּי} \\
			& pl. & 3 c. & \foreignlanguage{hebrew}{נָפְלוּ} & & \foreignlanguage{hebrew}{נָתְנוּ} & \foreignlanguage{hebrew}{לָקְחוּ} \\
			& & 2 m. & \foreignlanguage{hebrew}{נְפַלְתֶּם} & & \foreignlanguage{hebrew}{נְתַתֶּם} & \foreignlanguage{hebrew}{לְקַחְתֶּם} \\
			& & 2 f. & \foreignlanguage{hebrew}{נְפַלְתֶּן} & & \foreignlanguage{hebrew}{נְתַּתֶּן} & \foreignlanguage{hebrew}{לְקַחְתֶּן} \\
			& & 1 c. & \foreignlanguage{hebrew}{נָפַלְנוּ} & & \foreignlanguage{hebrew}{נָתַנּוּ} & \foreignlanguage{hebrew}{לָקַחְנוּ} \\
			\hline
			PC & sg. & 3 m. & \foreignlanguage{hebrew}{יִפֹּל} & \foreignlanguage{hebrew}{יִגַּשׁ} & \foreignlanguage{hebrew}{יִתֵּן} & \foreignlanguage{hebrew}{יִקַּח} \\
			& & 3 f. & \foreignlanguage{hebrew}{תִּפֹּל} & \foreignlanguage{hebrew}{תִּגַּשׁ} & \foreignlanguage{hebrew}{תִּתֵּן} &  \foreignlanguage{hebrew}{תִּקַּח} \\
			& & 2 m. & \foreignlanguage{hebrew}{תִּפֹּל} & \foreignlanguage{hebrew}{תִּגַּשׁ} & \foreignlanguage{hebrew}{תִּתֵּן} &  \foreignlanguage{hebrew}{תִּקַּח} \\
			& & 2 f. & \foreignlanguage{hebrew}{תִּפְּלִי} & \foreignlanguage{hebrew}{תִּגְּשִׁי} & \foreignlanguage{hebrew}{תִּתְּנִי} & \foreignlanguage{hebrew}{תִּקְחִי} \\
			& & 1 c. & \foreignlanguage{hebrew}{אֶפֹּל} & \foreignlanguage{hebrew}{אֶגַּשׁ} & \foreignlanguage{hebrew}{אֶתֵּן} & \foreignlanguage{hebrew}{אֶקַּח} \\
			& pl. & 3 m. & \foreignlanguage{hebrew}{יִפְּלוּ} & \foreignlanguage{hebrew}{יִגְּשׁוּ} & \foreignlanguage{hebrew}{יִתְּנוּ} & \foreignlanguage{hebrew}{יִקְחוּ} \\
			& & 3 f. & \foreignlanguage{hebrew}{תִּפֹּ֫לְנָה} & \foreignlanguage{hebrew}{תִּגַּ֫שְׂנָה} & \foreignlanguage{hebrew}{תִּתֵּ֫נָּה} & \foreignlanguage{hebrew}{תִּקַּ֫חְנָה} \\
			& & 2 m. & \foreignlanguage{hebrew}{תִּפְּלוּ} & \foreignlanguage{hebrew}{תִּגְּשׁוּ} & \foreignlanguage{hebrew}{תִּתְּנוּ} & \foreignlanguage{hebrew}{תִּקְחוּ} \\
			& & 2 f. & \foreignlanguage{hebrew}{תִּפֹּ֫לְנָה} & \foreignlanguage{hebrew}{תִּגַּ֫שְׂנָה} & \foreignlanguage{hebrew}{תִּתֵּ֫נָּה} & \foreignlanguage{hebrew}{תִּקַּ֫חְנָה} \\
			& & 1 c. & \foreignlanguage{hebrew}{נִפֹּל} & \foreignlanguage{hebrew}{נִגַּשׁ} & \foreignlanguage{hebrew}{נִתֵּן} & \foreignlanguage{hebrew}{נִקַּח} \\
			\hline
			Jussive & sg. & 3 m. & \foreignlanguage{hebrew}{יִפֹּל} & \foreignlanguage{hebrew}{יִגַּשׁ} & \foreignlanguage{hebrew}{יִתֵּן} & \foreignlanguage{hebrew}{יִקַּח} \\
			\textit{waw}-PC & sg. & 3 m. & \foreignlanguage{hebrew}{וַיִּפֹּל} & \foreignlanguage{hebrew}{וַיִּגַּשׁ} & \foreignlanguage{hebrew}{וַיִּתֵּן} & \foreignlanguage{hebrew}{וַיִּקַּח} \\
			\hline
			\pagebreak
			\hline
			Impv.\ & sg. & m. & \foreignlanguage{hebrew}{נְפֹל} & \foreignlanguage{hebrew}{גַּשׁ} & \foreignlanguage{hebrew}{תֵּן} & \foreignlanguage{hebrew}{קַח} \\
			& & f. & \foreignlanguage{hebrew}{נִפְלִי} & \foreignlanguage{hebrew}{גְּשִׁי} & \foreignlanguage{hebrew}{תְּנִי} & \foreignlanguage{hebrew}{קְחִי} \\
			& pl. & m. & \foreignlanguage{hebrew}{נִפְלוּ} & \foreignlanguage{hebrew}{גְּשׁוּ} & \foreignlanguage{hebrew}{תְּנוּ} & \foreignlanguage{hebrew}{קְחוּ} \\
			& & f. & \foreignlanguage{hebrew}{נְפֹ֫לְנָה} & \foreignlanguage{hebrew}{גַּ֫שְׁנָה} & & \\
			\hline
			Inf.\ cs.\ & & & \foreignlanguage{hebrew}{נְפֹל} & \foreignlanguage{hebrew}{גֶּשֶׁת} & \foreignlanguage{hebrew}{תֵּת} & \foreignlanguage{hebrew}{קַחַת} \\
			Inf.\ abs.\  & & & \foreignlanguage{hebrew}{נָפוֺל} & & \foreignlanguage{hebrew}{נָתוֹן} & \foreignlanguage{hebrew}{לָקוֺחַ} \\
			\hline
			Part.\ & sg. & m. & \foreignlanguage{hebrew}{נֹפֵל} & & \foreignlanguage{hebrew}{נֹתֵן} & \foreignlanguage{hebrew}{לֹקֵחַ} \\
			\hline
		\end{longtable}	
	\end{center}
	
	\noindent \textbf{Notes}
	\nopagebreak
	
	\noindent I\,\textit{n} verbs have either the \textit{i--o} or the \textit{i--a} vowel pattern without a semantic difference between fientive and stative meaning.
	
	The verb \foreignlanguage{hebrew}{לקח} looses the \textit{dageš forte} in the second root consonant in context forms of the prefix conjugation with vocalic suffix, e.g., \foreignlanguage{hebrew}{יִקְחוּ} \textit{they will take}.
	
	The \textit{dageš lene} in the 2 f.\ sg.\ SC form \foreignlanguage{hebrew}{לָקַחַתְּ} distinguishes this form from the inf.\ cs.\ with the preposition \foreignlanguage{hebrew}{לְ} \foreignlanguage{hebrew}{לָקַחַת} \textit{to take}.
	
	In pausal forms with vocalic suffix the thematic vowel is preserved and stressed, e.g., \foreignlanguage{hebrew}{יִפֹּ֑לוּ}, \foreignlanguage{hebrew}{יִתֵּ֑נוּ} and \foreignlanguage{hebrew}{יִקָּ֑חוּ}, \foreignlanguage{hebrew}{קָ֑חוּ} (impv.).
	
	Verbs that are both I\,\textit{n} and III gutt. often have strong inf.\ cs.\ forms, e.g., \foreignlanguage{hebrew}{נְסֹעַ}. Some verbs have both forms, e.g., \foreignlanguage{hebrew}{נְגֹעַ} and less frequently \foreignlanguage{hebrew}{גַּ֫עַת}.
	
	The inf.\,cs. \foreignlanguage{hebrew}{תֵּת} of the verb \foreignlanguage{hebrew}{נתן} is the result of the assimilation of the third root consonant /n/ to the ending \textit{-t}: \textit{*tint > > *titt > tēt} \foreignlanguage{hebrew}{תֵּת}. A strong form of the inf.\,cs.\ \foreignlanguage{hebrew}{נְתֹן} (or \foreignlanguage{hebrew}{נְתָן־}) is attested only in Gen 38:9 and Num 20:21 whereas the weak form \foreignlanguage{hebrew}{תֵּת} is attested more than 150 times. With enclitic pronoun, the inf.\ cs.\ of the verb \foreignlanguage{hebrew}{נתן} has forms like \foreignlanguage{hebrew}{תִּתּוֺ}, \foreignlanguage{hebrew}{תִּתֵּ֫נוּ}, etc. with restoration of the gemination.
	
	The imperative m.\ sg.\ of the verb \foreignlanguage{hebrew}{נתן} is often the long form \foreignlanguage{hebrew}{תְּנָה} \textit{give!}.
	
	The verb \foreignlanguage{hebrew}{נגשׁ} \textit{to draw near, approach} is a so-called suppletive verb. In the Qal it is only used in the PC, imperative and inf.\ cs.; in the SC and the participle it is used in the Niphal with no difference in meaning. It is included in the table because it is the most frequent verb I\,\textit{n} with /a/ as thematic vowel in the PC Qal by far.
	
	
	\section{Weak Verbs: I ʾ Verbs}
	
	A number of verbs I\,ʾ have weak forms in the prefix conjugation. The five I\,ʾ verbs that only have weak PC forms are the verbs: \foreignlanguage{hebrew}{אמר} \textit{to say}, \foreignlanguage{hebrew}{אכל} \textit{to eat}, \foreignlanguage{hebrew}{אבד} \textit{to perish} and the two doubly weak verbs  \foreignlanguage{hebrew}{אבה} \textit{to be willing} (almost always with negation) and \foreignlanguage{hebrew}{אפה} \textit{to bake} (I\,ʾ and III\,\textit{y}).\footnote{\space Mnemonic saying: The baker says, \enquote{the one who does not want to eat will perish.}} Therefore the rules of III\,\textit{y} verbs apply to \foreignlanguage{hebrew}{אבה} and \foreignlanguage{hebrew}{אפה} as well (cf.\ Chapter 12).
	
	The weakness is due to \foreignlanguage{hebrew}{א} becoming silent at the end of the prefix syllable. As a result, the prefix vowel was lengthened /a/ > /ā/ and then changed to /ō/ according to the Canaanite sound shift. Subsequently, the thematic vowel became /a/ through dissimilation to avoid two consecutive dark vowels /ō/ and /u/. This development started in the 1 c.\ sg.\ form to avoid two consecutive consonants /ʾ/:
	
	\begin{center}
		\textit{*ʾaʾmur > *ʾāmur > *ʾōmur > ʾōmar} \foreignlanguage{hebrew}{אֹמַר} \textit{I will say}
	\end{center}
	
	From this form the loss of the consonantal \foreignlanguage{hebrew}{א} spread to the rest of the PC forms. PC forms are the only weak forms of I\,ʾ verbs in the Qal; all other forms are strong.
	
	
	\renewcommand\arraystretch{1.4}
	
	% Not attested forms (not a complete list)
	% verb אכל impv. fs
	% Two verbs are included in this table because of horror vacui.
	
	\begin{center}
		\begin{longtable}{|lll|r|r|}
			\hline
			\multicolumn{3}{|c|}{} & \foreignlanguage{hebrew}{אמר} & \foreignlanguage{hebrew}{אכל} \\
			\hline
			SC & sg. & 3 m. & \foreignlanguage{hebrew}{אָמַר} & \foreignlanguage{hebrew}{אָכַל}  \\
			& sg. & 3 f. & \foreignlanguage{hebrew}{אָמְרָה} & \foreignlanguage{hebrew}{אָכְלָה} \\
			& sg. & 2 m. & \foreignlanguage{hebrew}{אָמַרְתָּ} & \foreignlanguage{hebrew}{אָכַלְתָּ} \\
			& sg. & 2 f. & \foreignlanguage{hebrew}{אָמַרְתְּ} & \foreignlanguage{hebrew}{אָכַלְתְּ} \\
			& sg. & 1 c. & \foreignlanguage{hebrew}{אָמַרְתִּי} & \foreignlanguage{hebrew}{אָכַלְתִּי} \\
			& pl. & 3 c. & \foreignlanguage{hebrew}{אָמְרוּ} & \foreignlanguage{hebrew}{אָכְלוּ} \\
			& pl. & 2 m. & \foreignlanguage{hebrew}{אֲמַרְתֶּם} & \foreignlanguage{hebrew}{אֲכַלְתֶּם} \\
			& pl. & 2 f. & \foreignlanguage{hebrew}{אֲמַרְתֶּן} & \foreignlanguage{hebrew}{אֲכַלְתֶּן} \\
			& pl. & 1 c. & \foreignlanguage{hebrew}{אָמַרְנוּ} & \foreignlanguage{hebrew}{אָכַלְנוּ} \\
			\hline
			PC & sg. & 3 m. & \foreignlanguage{hebrew}{יֹאמַר} &  \foreignlanguage{hebrew}{יֹאכַל} \\
			& & 3 f. & \foreignlanguage{hebrew}{תֹּאמַר} &  \foreignlanguage{hebrew}{תֹּאכַל} \\
			& & 2 m. & \foreignlanguage{hebrew}{תֹּאמַר} & \foreignlanguage{hebrew}{תֹּאכַל} \\
			& & 2 f. & \foreignlanguage{hebrew}{תֹּאמְרִי} & \foreignlanguage{hebrew}{תֹּאכְלִי} \\
			& & 1 c. & \foreignlanguage{hebrew}{אֹמַר} & \foreignlanguage{hebrew}{אֹכַל} \\
			& pl. & 3 pl. & \foreignlanguage{hebrew}{יֹאמְרוּ} &  \foreignlanguage{hebrew}{יֹאכְלוּ} \\
			& & 3 f. & \foreignlanguage{hebrew}{תֹּאמַ֫רְנָה} &  \foreignlanguage{hebrew}{תֹּאכַ֫לְנָה} \\
			& & 2 m. & \foreignlanguage{hebrew}{תֹּאמְרוּ} & \foreignlanguage{hebrew}{תֹּאכְלוּ} \\
			& & 2 f. & \foreignlanguage{hebrew}{תֹּאמַ֫רְנָה} & \foreignlanguage{hebrew}{תֹּאכַ֫לְנָה} \\
			& & 1 c. & \foreignlanguage{hebrew}{נֹאמַר} & \foreignlanguage{hebrew}{נֹאכַל} \\
			\hline
			Jussive & sg. & 3 m. & \foreignlanguage{hebrew}{יֹאמַר} & \foreignlanguage{hebrew}{יֹאכַל} \\
			\textit{waw-PC} & sg. & 3 m. & \foreignlanguage{hebrew}{וַיֹּ֫אמֶר} & \foreignlanguage{hebrew}{וַיֹּ֫אכַל} \\
			\hline
			\pagebreak
			\hline
			Impv. & sg. & m. & \foreignlanguage{hebrew}{אֱמֹר} & \foreignlanguage{hebrew}{אֱכֹל} \\
			& & f. & \foreignlanguage{hebrew}{אִמְרִי} & \foreignlanguage{hebrew}{אִכְלִי} \\
			& pl. & m. & \foreignlanguage{hebrew}{אִמְרוּ} & \foreignlanguage{hebrew}{אִכְלוּ} \\
			\hline
			Inf.\ cs. & & & \foreignlanguage{hebrew}{אֱמֹר} & \foreignlanguage{hebrew}{אֱכֹל} \\
			Inf.\ abs. & & & \foreignlanguage{hebrew}{אָמוֺר} & \foreignlanguage{hebrew}{אָכוֺל} \\
			\hline
			Part. & sg. & m. & \foreignlanguage{hebrew}{אֹמֵר} & \foreignlanguage{hebrew}{אֹכֵל} \\
			\hline
		\end{longtable}
	\end{center}
	
	\noindent \textbf{Notes}
	\nopagebreak
	
	\noindent In the 1 c.\ sg.\ form \foreignlanguage{hebrew}{אֹמַר} the root consonant \foreignlanguage{hebrew}{א} is never preserved in writing. In the other forms it is almost always preserved in writing which makes the identification of these forms easy.
	
	The 3 m./f.\ sg.\ and 2 m.\ sg.\ \textit{wayyiqṭol} forms of the verbs \foreignlanguage{hebrew}{אמר} and \foreignlanguage{hebrew}{אכל} without suffix are stressed on the penultimate syllable, e.g., \foreignlanguage{hebrew}{וַיֹּ֫אמֶר}, \foreignlanguage{hebrew}{וַתֹּ֫אמֶר}, \foreignlanguage{hebrew}{וַנֹּ֫אמֶר}, \foreignlanguage{hebrew}{וַיֹּ֫אכַל}, \foreignlanguage{hebrew}{וַתֹּ֫אכַל}. Pausal forms 3 m./f.\ sg., 2 m.\ sg.\ and 1 c.\ sg.\ forms, however, are stressed on the last syllable, e.g., \foreignlanguage{hebrew}{וַיֹּאמַ֑ר}, \foreignlanguage{hebrew}{וַיֹּאכַ֑ל}, \foreignlanguage{hebrew}{וָאֹמַ֑ר}, \foreignlanguage{hebrew}{וָאֹכַ֑ל} including the pausal form \foreignlanguage{hebrew}{וָאֹכֵ֑ל}
	
	In the impv.\ m.\ sg.\ and the inf.\ cs., the \foreignlanguage{hebrew}{א} takes \textit{ḥatef segol}. In the impv.\ the original vowel \textit{ō < u} is preserved as \textit{segol} because there is no need for dissimilation as in the prefix conjugation.
	
	The verbs \foreignlanguage{hebrew}{אהב}, \foreignlanguage{hebrew}{אחז} and \foreignlanguage{hebrew}{אסף} have strong forms with preservation of the \foreignlanguage{hebrew}{א} as consonant as well as weak forms with elision of the \foreignlanguage{hebrew}{א}, e.g., \foreignlanguage{hebrew}{וַיֶּאֱחֹז}, \foreignlanguage{hebrew}{יֹאחֵז}, \foreignlanguage{hebrew}{וַיֹּ֫אחֶז}, \foreignlanguage{hebrew}{יֶאֱסֹף}, \foreignlanguage{hebrew}{וַיֶּאֱסֹף}, \foreignlanguage{hebrew}{וַיֹּ֫סֶף} (!), \foreignlanguage{hebrew}{אֹסְפָה} (cohortative), \foreignlanguage{hebrew}{וַיֶּאֱהַב}, \foreignlanguage{hebrew}{וָאֹהַב}.
	
	The introduction of direct speech \foreignlanguage{hebrew}{לֵאמֹר} is originally the inf.\ cs.\ of the verb \foreignlanguage{hebrew}{אמר} with the preposition \foreignlanguage{hebrew}{לְ}.
	
	
	
\section{Questions in Biblical Hebrew (Part 1)}
	
\subsection{Yes-No Questions}
	
Yes-no questions are usually marked with the syllable \foreignlanguage{hebrew}{הֲ} \textit{hă} which is prefixed to the first word of the question. This prefixed \foreignlanguage{hebrew}{הֲ} \textit{hă} is called \textit{interrogative He}. In addition to this, it can be assumed that questions were pronounced with a different intonation than statements.
	
The interrogative He is vocalized in different ways:
	
	\begin{enumerate}[noitemsep]
		\item Before non-gutturals with a full vowel it is \foreignlanguage{hebrew}{הֲ} \textit{hă-} (Gen 4:9)
		\item Before non-gutturals with \textit{šwa} it is \foreignlanguage{hebrew}{הַ} \textit{ha-} (1\,Kgs 21:20). Sometimes a \textit{dageš forte} is added (Gen 37:32).
		\item Before gutturals with any vowel except \textit{qameṣ} or \textit{ḥatef qameṣ} it is \foreignlanguage{hebrew}{הַ} \textit{ha-} (Judg 13:11)
		\item Before gutturals with \textit{qameṣ} or \textit{ḥatef qameṣ} it is \foreignlanguage{hebrew}{הֶ} \textit{hæ-} (Gen 24:5)
	\end{enumerate}
	
	\begin{longtable}{>{\raggedleft}p{0.35\linewidth} p{0.55\linewidth}}
		\foreignlanguage{hebrew}{הֲשֹׁמֵר אָחִי אָנֹ֫כִי} & \textit{Am I the keeper of my brother?} (Gen 4:9) \\
		\foreignlanguage{hebrew}{הַֽמְצָאתַנִי אֹיְבִי} & \textit{Have you found me, my enemy?} (1\,Kgs 21:20) \\
		\foreignlanguage{hebrew}{הַכְּתֹנֶת בִּנְךָ הִיא אִם־לֹא} & \textit{Is it the robe of your son or not?} (Gen 37:20 Qere) \\
		\foreignlanguage{hebrew}{הַאַתָּה הָאִישׁ אֲשֶׁר־דִּבַּרְתָּ אֶל־הָאִשָּׁה} & \textit{Are you the man who spoke to the woman?} (Judg 13:11) \\
		\foreignlanguage{hebrew}{הֶהָשֵׁב אָשִׁיב אֶת־בִּנְךָ אֶל־הָאָרֶץ אֲשֶׁר־יָצָאתָ מִשָּׁם} & \textit{Shall I then bring your son back to the land that you left?} (Gen 24:5) \\
	\end{longtable}
	
	Unmarked yes-no question are also attested in Biblical. It can be assumed that in spoken language they were discernible by their intonation. In a written text, the context helps to identify unmarked questions.
	
	\vspace{0.5cm}
	
	\begin{tabular}{>{\raggedleft}p{0.35\linewidth} p{0.55\linewidth}}
		\foreignlanguage{hebrew}{וַיֹּאמֶר הַמֶּלֶךְ שָׁלוֹם לַנַּעַר לְאַבְשָׁלוֹם} & \textit{And the king said, \enquote{Is it well with the young man Absalom?}} (2\,Sam 18:29) \\
		\foreignlanguage{hebrew}{וַיֹּאמֶר נָתָן אֲדֹנִי הַמֶּלֶךְ אַתָּה אָמַרְתָּ אֲדֹנִיָּהוּ יִמְלֹךְ אַחֲרָי וְהוּא יֵשֵׁב עַל־כִּסְאִֽי} & \textit{And Nathan said, \enquote{My lord, the king, have you said \enquote*{Adonijah shall become king after me and he shall sit on my throne}?}} (1\,Kgs 1:24) \\
	\end{tabular}
	
	% https://www.grammarbank.com/question-types.html
	
	\vspace{0.5cm}
	
	Biblical Hebrew does not have a word for \textit{yes}. For a positive answer to a yes-no question either a word from the question (Gen 24:58) or another corresponding word (2\,Kgs 18:7--8) is used. For a negative answer, Hebrew uses \foreignlanguage{hebrew}{לֹא} \textit{no} (Judg 12:5).
	
	% König § 351h
	
	\vspace{0.5cm}
	
	\begin{tabular}{>{\raggedleft}p{0.35\linewidth} p{0.55\linewidth}}
		\foreignlanguage{hebrew}{וַיֹּאמְרוּ אֵלֶיהָ הֲתֵלְכִי עִם־הָאִישׁ הַזֶּה וַתֹּאמֶר אֵלֵךְ} & \textit{And they said to her, \enquote{Will you go with this man?} And she said, \enquote{Yes.}} (Gen 24:58) \\
		\foreignlanguage{hebrew}{וַיֹּאמֶר הַאַתָּה זֶה אֲדֹנִי אֵלִיָּהוּ וַיֹּאמֶר לוֹ אָ֑נִי} & \textit{And he said, \enquote{Is it you,  my lord Elijah?} And he said to him, \enquote{Yes.}} (2\,Kgs 18:7--8) \\
		\foreignlanguage{hebrew}{וַיֹּאמְרוּ לוֹ אַנְשֵׁי־גִלְעָד הַאֶפְרָתִי אַתָּה וַיֹּאמֶר לֹא} & \textit{And the people from Gilead said to him, \enquote{Are you an Ephraimite?}. And he said, \enquote{No.}} (Judg 12:5) \\
	\end{tabular}
	
	
	
	\subsection{Choice Questions}
	The interrogative He is also used for introducing choice questions; the second option is introduced with \foreignlanguage{hebrew}{אִם}.
	
	\vspace{0.5cm}
	
	\begin{tabular}{>{\raggedleft}p{0.35\linewidth} p{0.55\linewidth}}
		\foreignlanguage{hebrew}{הֲלָנוּ אַתָּה אִם־לְצָרֵינוּ} & \textit{Are you for us, or for our enemies?} (Josh 5:13) \\
		\foreignlanguage{hebrew}{הַאֵלֵךְ עַל־רָמֹת גִּלְעָד לַמִּלְחָמָה אִם־אֶחְדָּל} & \textit{Shall I go to Ramoth-gilead for battle or shall I refrain?} (1\, Kgs 22:6) \\
	\end{tabular}
	
	\vspace{0.5cm}
	
	\noindent \textbf{Note}
	\nopagebreak
	
	\noindent The word \foreignlanguage{hebrew}{אִם} is most frequently used as conditional conjunction \textit{if} (cf. Chapter 15).
	
	% https://www.grammarbank.com/question-types.html
	
	\subsection{Indirect Questions}
	The interrogative He also introduces indirect (or embedded) questions (Exod 4:18), but \foreignlanguage{hebrew}{אִם} is used for this, as well (2\,Kgs 1:2). For indirect choice questions \foreignlanguage{hebrew}{הֲ \dots \space אִם} is used (Gen 27:21).
	
	\vspace{0.5cm}
	
	\begin{tabular}{>{\raggedleft}p{0.35\linewidth} p{0.55\linewidth}}
		\foreignlanguage{hebrew}{וְאֶרְאֶה הַעוֹדָם חַיִּים} & \textit{\dots \space I want to see whether they are still alive} (Exod 4:18) \\
		\foreignlanguage{hebrew}{לְכוּ דִרְשׁוּ בְּבַעַל זְבוּב אֱלֹהֵי עֶקְרוֹן אִם־אֶחְיֶה מֵחֳלִי זֶה} & \textit{Go, inquire of Baal-zebub, the god of Ekron, whether I will recover from this sickness} (2\,Kgs 1:2) \\
		\foreignlanguage{hebrew}{לָדַעַת הֽ͏ַהִצְלִיחַ יְהוָה דַּרְכּוֹ אִם־לֹא} & \textit{\dots \space to find out whether God had made his journey successful or not} (Gen 24:21) \\
	\end{tabular}
	
	% Other example for indirect choice quesitons: Gen 27:211; Gen 37:32
	
	\section{Exercises}
	
	\subsection{Translation of Verbal Forms}
	Translate the following verbal forms. Identify the gender (masc., fem., comm.) and number (sg., pl.) of the forms of which the English translation is ambiguous (i.e., \textit{you}, \textit{they}). Mark the stressed syllable if stress is not on the last syllable.
	
	\hspace{0.5cm}
	
	\selectlanguage{hebrew}
	
	\noindent
	1~~\foreignlanguage{hebrew}{וַיֹּאכְלוּ}  \hspace{0.3cm}
	2~~\foreignlanguage{hebrew}{אֱכֹל}  \hspace{0.3cm}
	3~~\foreignlanguage{hebrew}{אִכְלוּ}  \hspace{0.3cm}
	4~~\foreignlanguage{hebrew}{אָמַרְתִּי}  \hspace{0.3cm}
	5~~\foreignlanguage{hebrew}{אֹמַר}  \hspace{0.3cm}
	6~~\foreignlanguage{hebrew}{אֹמֵר}  \hspace{0.3cm}
	7~~\foreignlanguage{hebrew}{תִּגְּשׁוּ}  \hspace{0.3cm}
	8~~\foreignlanguage{hebrew}{גַּשׁ}  \hspace{0.3cm}
	9~~\foreignlanguage{hebrew}{וָאֶפֹּל}  \hspace{0.3cm}
	10~~\foreignlanguage{hebrew}{נָפַלְתִּי}  \hspace{0.3cm}
	11~~\foreignlanguage{hebrew}{וַיֶּאֱחֹז}  \hspace{0.3cm}
	12~~\foreignlanguage{hebrew}{וַיִּגַּע}  \hspace{0.3cm}
	
	\selectlanguage{english}
	
	\subsection{Translation of Sentences}
	
	Translate the following sentences from the Hebrew Bible. Names of persons and geographical names in these sentences: \foreignlanguage{hebrew}{אַבְרָהָם}, \foreignlanguage{hebrew}{אַבְרָם}, \foreignlanguage{hebrew}{אַבְשָׁלֹם}, \foreignlanguage{hebrew}{אַהֲרֹן}, \foreignlanguage{hebrew}{אֱלִישָׁע}, \foreignlanguage{hebrew}{אַמְנוֹן}, \foreignlanguage{hebrew}{בְּאֵר שֶׁבַע}, \foreignlanguage{hebrew}{גִּלְגָּל}, \foreignlanguage{hebrew}{זִלְפָּה}, \foreignlanguage{hebrew}{יְהוּדָה}, \foreignlanguage{hebrew}{יְהוֹשֻׁעַ}, \foreignlanguage{hebrew}{יוֹסֵף}, \foreignlanguage{hebrew}{יְפֻנֶּה}, \foreignlanguage{hebrew}{יִצְחָק}, \foreignlanguage{hebrew}{יְרִיחוֺ}, \foreignlanguage{hebrew}{כָּלֵב}, \foreignlanguage{hebrew}{לֵאָה}, \foreignlanguage{hebrew}{לָבָן}, \foreignlanguage{hebrew}{מֹשֶׁה}, \foreignlanguage{hebrew}{קָדֵשׁ בַּרְנֵעַ}, \foreignlanguage{hebrew}{תָּמָר}
	
	\vspace{0.5cm}
	
	\selectlanguage{hebrew}
	\noindent
	1~~\foreignlanguage{hebrew}{וַיֹּאמְרוּ}\LTRfootnote{\space \foreignlanguage{hebrew}{וַיֹּאמְרוּ} \textit{and they said} (pl.\ of \foreignlanguage{hebrew}{וַיֹּאמֶר})} \foreignlanguage{hebrew}{לוֹ אֶחָיו הֲמָלֹךְ תִּמְלֹךְ עָלֵינוּ אִם־מָשׁוֹל תִּמְשֹׁל בָּנוּ} \hspace{0.3cm}
	2~~\foreignlanguage{hebrew}{וַיִּשְׁאַל לָהֶם לְשָׁלוֹם וַיֹּאמֶר הֲשָׁלוֹם אֲבִיכֶם הַזָּקֵן אֲשֶׁר אֲמַרְתֶּם. הַעוֹדֶנּוּ חָי}  \hspace{0.3cm}
	3~~\foreignlanguage{hebrew}{וַיֹּאמֶר לוֹ אַמְנוֹן אֶת־תָּמָר אֲחוֹת אַבְשָׁלֹם אָחִי אֲנִי אֹהֵב}  \hspace{0.3cm}
	4~~\foreignlanguage{hebrew}{וַיֶּאֱסֹף לָבָן אֶת־כָּל־אַנְשֵׁי הַמָּקוֹם וַיַּעַשׂ מִשְׁתֶּה. וַיְהִי בָעֶרֶב וַיִּקַּח אֶת־לֵאָה בִתּוֹ וַיָּבֵא}\LTRfootnote{\space \foreignlanguage{hebrew}{וַיָּבֵא} \textit{and he brought her in} (Hiphil of \foreignlanguage{hebrew}{בוא})} \foreignlanguage{hebrew}{אֹתָהּ אֵלָיו וַיָּבֹא אֵלֶיהָ. וַיִּתֵּן לָבָן לָהּ אֶת־זִלְפָּה שִׁפְחָתוֹ לְלֵאָה בִתּוֹ שִׁפְחָה. וַיְהִי בַבֹּקֶר וְהִנֵּה־הִוא}\LTRfootnote{\space \foreignlanguage{hebrew}{הִוא} to be read as \foreignlanguage{hebrew}{הִיא}} \foreignlanguage{hebrew}{לֵאָה} \hspace{0.3cm}
	5~~\foreignlanguage{hebrew}{וַיֵּלֶךְ מֹשֶׁה וְאַהֲרֹן וַיַּאַסְפוּ אֶת־כָּל־זִקְנֵי בְּנֵי יִשְׂרָאֵֽל. וַיְדַבֵּר אַהֲרֹן אֵת כָּל־הַדְּבָרִים אֲשֶׁר־דִּבֶּר יְהוָה אֶל־מֹשֶׁה וַיַּעַשׂ הָאֹתֹת לְעֵינֵי הָעָֽם.}  \hspace{0.3cm}
	6~~\foreignlanguage{hebrew}{וַיֹּאמֶר יוֹסֵף אֶל־אֶחָיו גְּשׁוּ־נָא אֵלַי וַיִּגָּ֫שׁוּ}\LTRfootnote{\space \foreignlanguage{hebrew}{וַיִּגָּ֫שׁוּ} pausal form; context form \foreignlanguage{hebrew}{וַיִּגְּשׁוּ}} \foreignlanguage{hebrew}{וַיֹּאמֶר אֲנִי יוֹסֵף אֲחִיכֶם אֲשֶׁר־מְכַרְתֶּם אֹתִי מִצְרָ֫יְמָה} \hspace{0.3cm}
	7~~\foreignlanguage{hebrew}{וַיֹּאמֶר יְהוֹשֻׁעַ אֶל־בְּנֵי יִשְׂרָאֵל גֹּ֫שׁוּ הֵ֫נָּה}\LTRfootnote{\space \foreignlanguage{hebrew}{הֵ֫נָּה} \textit{here, hither}; the stress of the preceding word \foreignlanguage{hebrew}{גֹּ֫שׁוּ} is moved to the syllable with the thematic vowel to avoid the clash of two stressed syllable.} \foreignlanguage{hebrew}{וְשִׁמְעוּ אֶת־דִּבְרֵי יְהוָה אֱלֹהֵיכֶם} \hspace{0.3cm}
	8~~\foreignlanguage{hebrew}{וַיִּגְּשׁוּ בְנֵי־יְהוּדָה אֶל־יְהוֹשֻׁעַ בַּגִּלְגָּל וַיֹּאמֶר אֵלָיו כָּלֵב בֶּן־יְפֻנֶּה הַקְּנִזִּי}\LTRfootnote{\space \foreignlanguage{hebrew}{קְנִזִּי} \textit{Kenizzite}} \foreignlanguage{hebrew}{אַתָּה יָדַעְתָּ אֶת־הַדָּבָר אֲשֶׁר־דִּבֶּר יְהוָה אֶל־מֹשֶׁה אִישׁ־הָאֱלֹהִים עַל אֹדוֹתַי}\LTRfootnote{\space \foreignlanguage{hebrew}{עַל אֹדוֺת} \textit{because of, concerning}} \foreignlanguage{hebrew}{וְעַל אֹדוֹתֶיךָ בְּקָדֵשׁ בַּרְנֵעַ} \hspace{0.3cm}
	9~~\foreignlanguage{hebrew}{‎וַיִּגְּשׁוּ בְנֵי־הַנְּבִיאִים אֲשֶׁר־בִּירִיחוֹ אֶל־אֱלִישָׁע וַיֹּאמְרוּ אֵלָיו הֲיָדַעְתָּ כִּי הַיּוֹם יְהוָה לֹקֵחַ אֶת־אֲדֹנֶיךָ מֵעַל רֹאשֶׁךָ וַיֹּאמֶר גַּם־אֲנִי יָדַעְתִּי הֶחֱשׁוּ}\LTRfootnote{\space \foreignlanguage{hebrew}{הֶחֱשׁוּ} \textit{be quiet}} \hspace{0.3cm}
	10~~\foreignlanguage{hebrew}{וַיִּסַּע אַבְרָם הָלוֹךְ וְנָסוֹעַ הַנֶּ֫גְבָּה}  \hspace{0.3cm}
	11~~\foreignlanguage{hebrew}{וַיִּסַּע יִשְׂרָאֵל וְכָל־אֲשֶׁר־לוֹ וַיָּבֹא בְּאֵרָה שָּׁבַע}\LTRfootnote{\space \foreignlanguage{hebrew}{בְּאֵר שָׁ֑בַע} pausal form of \foreignlanguage{hebrew}{בְּאֵר שֶׁבַע}} \foreignlanguage{hebrew}{וַיִּזְבַּח זְבָחִים לֵאלֹהֵי אָבִיו יִצְחָק} \hspace{0.3cm}
	12~~\foreignlanguage{hebrew}{בְּנִי תּוֹרָתִי אַל־תִּשְׁכָּח וּמִצְוֹתַי יִצֹּר לִבֶּךָ}  \hspace{0.3cm}
	13~~\foreignlanguage{hebrew}{וַיַּרְא אֱלֹהִים אֶת־כָּל־אֲשֶׁר עָשָׂה וְהִנֵּה־טוֹב מְאֹד}  \hspace{0.3cm}
	14~~\foreignlanguage{hebrew}{יְהוָה אֱלֹהֵי הַשָּׁמַיִם אֲשֶׁר לְקָחַנִי מִבֵּית אָבִי וּמֵאֶרֶץ מוֹלַדְתִּי}\LTRfootnote{\space \foreignlanguage{hebrew}{מוֺלֶדֶת} \textit{kindred, offspring, birth}} \foreignlanguage{hebrew}{וַאֲשֶׁר נִשְׁבַּע־לִי}\LTRfootnote{\space \foreignlanguage{hebrew}{נִשְׁבַּע} \textit{he swore}} \foreignlanguage{hebrew}{לֵאמֹר לְזַרְעֲךָ אֶתֵּן אֶת־הָאָרֶץ הַזֹּאת הוּא יִשְׁלַח מַלְאָכוֹ לְפָנֶיךָ וְלָקַחְתָּ אִשָּׁה לִבְנִי מִשָּׁם} \hspace{0.3cm}
	15~~\foreignlanguage{hebrew}{אֲנִי יְהוָה אֱלֹהֵי אַבְרָהָם אָבִיךָ וֵאלֹהֵי יִצְחָק הָאָרֶץ אֲשֶׁר אַתָּה שֹׁכֵב עָלֶיהָ, לְךָ אֶתְּנֶנָּה וּלְזַרְעֶךָ}  \hspace{0.3cm}
	\selectlanguage{english}
	
	
	\chapter{Chapter 12}
	
	\renewcommand\arraystretch{1.4}
	
	\section{Vocabulary}
	
	\subsection{Verbs}
	
	\begin{center}
		
		% For the centering of the separation between the two columns see the documentation of the array package, page 2 
		
		\begin{tabular}{>{\raggedleft}p{0.175\linewidth} p{0.75\linewidth}}
			\foreignlanguage{hebrew}{אבה} & Q.\ \textit{to be willing, consent} (almost always with a negative) \\
			\foreignlanguage{hebrew}{בכה} & Q.\ \textit{to weep, weep for} \\ % HALOT
			\foreignlanguage{hebrew}{בנה} & Q.\ \textit{to build} \\
			\foreignlanguage{hebrew}{גלה} & Q.\ \textit{to uncover, remove, go into exile} \\
			\foreignlanguage{hebrew}{היה} & Q.\ \textit{to be, become} \\
			\foreignlanguage{hebrew}{חטא} & Q.\ \textit{to sin} \\
			\foreignlanguage{hebrew}{חיה} & Q.\ \textit{to be alive, stay alive, revive, recover, return to life} \\ % HALOT
			\foreignlanguage{hebrew}{חנה} & Q.\ \textit{to encamp} \\
			\foreignlanguage{hebrew}{חרה} & Q.\ \textit{to burn, kindle} (of anger \foreignlanguage{hebrew}{אַף}) \\
			\foreignlanguage{hebrew}{מלא} & Q.\ \textit{to be full} (stative verb; SC \foreignlanguage{hebrew}{מָלֵא}; with noun to indicate the substance) \\
			\foreignlanguage{hebrew}{מצא} & Q.\ \textit{to find} \\
			\foreignlanguage{hebrew}{נטה} & Q.\ \textit{to stretch out, spread out, extend, bend, turn, incline} \\
			\foreignlanguage{hebrew}{נשׂא} & Q.\ \textit{to lift, carry, take} \\
			\foreignlanguage{hebrew}{עלה} & Q.\ \textit{to go up, ascend} \\
			\foreignlanguage{hebrew}{ענה} & Q.\ \textit{to answer, respond} \\
			\foreignlanguage{hebrew}{עשׂה} & Q.\ \textit{to do, to make} \\
			\foreignlanguage{hebrew}{קרא} & Q.\ \textit{to call, proclaim, read} \\
			\foreignlanguage{hebrew}{שׁתה} & Q.\ \textit{to drink} \\
		\end{tabular}
	\end{center}
	
	\subsection{Nouns}
	
	\begin{center}
		\begin{longtable}{>{\raggedleft}p{0.175\linewidth} p{0.75\linewidth}}
			\foreignlanguage{hebrew}{אָדוֺן} & \textit{lord} (with ref. to God \foreignlanguage{hebrew}{אֲדֹנָי} \textit{my Lord}) \\
			\foreignlanguage{hebrew}{חֹדֶשׁ} & \textit{new moon, month} \\
			\foreignlanguage{hebrew}{מִלְחָמָה} & \textit{battle, war} \\
			\foreignlanguage{hebrew}{מַעֲשֶׂה} & \textit{deed, work} \\
			\foreignlanguage{hebrew}{מִשְׁפָט} & \textit{judgment; justice; ordinance; decision; custom} \\
			\foreignlanguage{hebrew}{נֶפֶשׁ} & \textit{soul, self, life, living being, person, desire, throat} (f.) \\
			\foreignlanguage{hebrew}{סֵפֶר} & \textit{something written, letter, scroll} \\ % HALOT
			\foreignlanguage{hebrew}{סֹפֵר} & \textit{secretary, scribe} \\
			\foreignlanguage{hebrew}{עֹלָה} & \textit{(whole) burnt offering} \\
			\foreignlanguage{hebrew}{עֵת} & \textit{time} (f.; gem. noun; with enclitic pronoun \foreignlanguage{hebrew}{עִתּוֺ}) \\
			\foreignlanguage{hebrew}{פְּלִשְׁתִּים} & \textit{Philistines} (sg.\ \foreignlanguage{hebrew}{פְּלִשְׁתִּי}) \\
			\foreignlanguage{hebrew}{צַר} & \textit{enemy} (pl. \foreignlanguage{hebrew}{צָרִים}), pl.\ cs.\ st.\ \foreignlanguage{hebrew}{צָרֵי} (gem. noun) \\
		\end{longtable}
	\end{center}
	
	\subsection{Other Parts of Speech}
	
	\begin{center}
		\begin{tabular}{>{\raggedleft}p{0.175\linewidth} p{0.75\linewidth}}
			\foreignlanguage{hebrew}{אוֺ} & \textit{or} \\
			\foreignlanguage{hebrew}{אֵי} & \textit{where?} \\
			\foreignlanguage{hebrew}{אַיֵּה} & \textit{where?} \\
			\foreignlanguage{hebrew}{אַךְ} & \textit{yea, surely, only} \\ % HALOT
			\foreignlanguage{hebrew}{אָ֫נָה} & \textit{where to?} (interrogative adverb \foreignlanguage{hebrew}{אָן} \textit{where to?} with directional He) \\
			\foreignlanguage{hebrew}{לָ֫מָּה} & \textit{why?} (preposition \foreignlanguage{hebrew}{לְ} + \foreignlanguage{hebrew}{מָה}) \\
			\foreignlanguage{hebrew}{לְמַ֫עַן} & \textit{in order to, so that} (conj.); \textit{on account of, for the sake of} (prep.) (\foreignlanguage{hebrew}{לְ} + \foreignlanguage{hebrew}{מַעַן}) \\
			\foreignlanguage{hebrew}{מֵאַ֫יִן} & \textit{whence? where from?} (prep.\ \foreignlanguage{hebrew}{מִן} + the interrogative adv.\ \foreignlanguage{hebrew}{אַיִן} \textit{where?}) \\
			\foreignlanguage{hebrew}{מָדוּעַ} & \textit{why?} \\
			\foreignlanguage{hebrew}{מָה} & \textit{what?} \\
			\foreignlanguage{hebrew}{מִי} & \textit{who?} \\
			\foreignlanguage{hebrew}{מָתַי} & \textit{when?} \\
			\foreignlanguage{hebrew}{פֹּה} & \textit{here} (adv.) \\
		\end{tabular}
	\end{center}
	
	
	
	\section{Weak Verbs: III \textit{y} Verbs}
	
	Originally, verbs of the category  III\,\textit{y} were verbs with the semivowels /y/ or /w/ as third root consonant. Because of contraction of the semivowel and the vowels preceding and/or following the semivowel, forms of III\,\textit{y} verbs are often characterized by a long vowel after the second root consonant. The only form, in which the original root consonant /y/ is found regularly, is the participle passive.\footnote{\space A number of times, the semivowel may occur in other forms as well, e.g., \foreignlanguage{hebrew}{יִבְכָּיוּן} \textit{they weep} (Isa 33:7); \foreignlanguage{hebrew}{יִשְׁלָ֫יוּ} \textit{may they be at ease} (Ps 122:6) from the verb \foreignlanguage{hebrew}{שׁלה}; compare the adjective \foreignlanguage{hebrew}{שָׁלֵו} \textit{quiet, at ease} with a consonantal \foreignlanguage{hebrew}{ו}.}
	
	% As this vowel is spelled with \foreignlanguage{hebrew}{ה} in 3 m.\ sg.\ SC and the suffix-less forms of the PC, these verbs are called \foreignlanguage{hebrew}{ל״ה} in Hebrew.
	
	\newpage
	
	\begin{center}
		\begin{longtable}{|lll|r|r|r|}
			\hline
			\multicolumn{3}{|c|}{} & \foreignlanguage{hebrew}{גלה} & \foreignlanguage{hebrew}{עשׂה} & \foreignlanguage{hebrew}{היה} \\
			\hline
			SC & sg. & 3 m. & \foreignlanguage{hebrew}{גָּלָה} & \foreignlanguage{hebrew}{עָשָׂה} & \foreignlanguage{hebrew}{הָיָה} \\
			& & 3 f. & \foreignlanguage{hebrew}{גָּלְתָה} & \foreignlanguage{hebrew}{עָשְׂתָה} & \foreignlanguage{hebrew}{הָיְתָה} \\
			& & 2 m. & \foreignlanguage{hebrew}{גָּלִ֫יתָ} & \foreignlanguage{hebrew}{עָשִׂ֫יתָּ} & \foreignlanguage{hebrew}{הָיִ֫יתָ} \\
			& & 2 f. & \foreignlanguage{hebrew}{גָּלִית} & \foreignlanguage{hebrew}{עָשִׂית} & \foreignlanguage{hebrew}{הָיִית} \\
			& & 1 c. & \foreignlanguage{hebrew}{גָּלִ֫יתִי} & \foreignlanguage{hebrew}{עָשִׂ֫יתִי} & \foreignlanguage{hebrew}{הָיִ֫יתִי} \\
			& pl. & 3 c. & \foreignlanguage{hebrew}{גָּלוּ} & \foreignlanguage{hebrew}{עָשׂוּ} & \foreignlanguage{hebrew}{הָיוּ} \\
			& & 2 m. & \foreignlanguage{hebrew}{גְּלִיתֶם} & \foreignlanguage{hebrew}{עֲשִׂיתֶם} & \foreignlanguage{hebrew}{הֱיִיתֶם} \\
			& & 2 f. & \foreignlanguage{hebrew}{גְּלִיתֶן} & \foreignlanguage{hebrew}{עֲשִׂיתֶן} & \foreignlanguage{hebrew}{הֲיִיתֶן} \\
			& & 1 c. & \foreignlanguage{hebrew}{גָּלִ֫ינוּ} & \foreignlanguage{hebrew}{עָשִׂ֫ינוּ} & \foreignlanguage{hebrew}{הָיִ֫ינוּ} \\
			\hline
			PC & sg. & 3 m. & \foreignlanguage{hebrew}{יִגְלֶה} & \foreignlanguage{hebrew}{יַעֲשֶׂה} & \foreignlanguage{hebrew}{יִהְיֶה} \\
			& & 3 f. & \foreignlanguage{hebrew}{תִּגְלֶה} & \foreignlanguage{hebrew}{תַּעֲשֶׂה} & \foreignlanguage{hebrew}{תִּהְיֶה} \\
			& & 2 m. & \foreignlanguage{hebrew}{תִּגְלֶה} & \foreignlanguage{hebrew}{תַּעֲשֶׂה} & \foreignlanguage{hebrew}{תִּהְיֶה} \\
			& & 2 f. & \foreignlanguage{hebrew}{תִּגְלִי} & \foreignlanguage{hebrew}{תַּעֲשִׂי} & \foreignlanguage{hebrew}{תִּהְיִי} \\
			& & 1 c. & \foreignlanguage{hebrew}{אֶגְלֶה} & \foreignlanguage{hebrew}{אֶעֱשֶׂה} & \foreignlanguage{hebrew}{אֶהְיֶה} \\
			& pl. & 3 m. & \foreignlanguage{hebrew}{יִגְלוּ} & \foreignlanguage{hebrew}{יַעֲשׂוּ} & \foreignlanguage{hebrew}{יִהְיוּ} \\
			& & 3 f. & \foreignlanguage{hebrew}{תִּגְלֶ֫ינָה} & \foreignlanguage{hebrew}{תַּעֲשֶׂ֫ינָה} & \foreignlanguage{hebrew}{תִּהְיֶ֫ינָה} \\
			& & 2 m. & \foreignlanguage{hebrew}{תִּגְלוּ} & \foreignlanguage{hebrew}{תַּעֲשׂוּ} & \foreignlanguage{hebrew}{תִּהְיוּ} \\
			& & 2 f. & \foreignlanguage{hebrew}{תִּגְלֶ֫ינָה} & \foreignlanguage{hebrew}{תַּעֲשֶׂ֫ינָה} & \foreignlanguage{hebrew}{תִּהְיֶ֫ינָה} \\
			& & 1 c. & \foreignlanguage{hebrew}{נִגְלֶה} & \foreignlanguage{hebrew}{נַעֲשֶׂה} & \foreignlanguage{hebrew}{נִהְיֶה} \\
			\hline
			Jussive & sg. & 3 m. & \foreignlanguage{hebrew}{יִ֫גֶל} & \foreignlanguage{hebrew}{יַ֫עַשׂ} & \foreignlanguage{hebrew}{יְהִי} \\
			\textit{wayyiqṭol} & sg. & 3 m. & \foreignlanguage{hebrew}{וַיִּ֫גֶל} & \foreignlanguage{hebrew}{וַיַּ֫עַשׂ} &  \foreignlanguage{hebrew}{וַיְהִי} \\
			\hline
			Impv.\ & sg. & m. & \foreignlanguage{hebrew}{גְּלֵה} & \foreignlanguage{hebrew}{עֲשֵׂה} & \foreignlanguage{hebrew}{הֱיֵה} \\
			& & f. & \foreignlanguage{hebrew}{גְּלִי} & \foreignlanguage{hebrew}{עֲשִׂי} & \foreignlanguage{hebrew}{הֲיִי} \\
			& pl. & m. & \foreignlanguage{hebrew}{גְּלוּ} & \foreignlanguage{hebrew}{עֲשׂוּ} & \foreignlanguage{hebrew}{הֱיוּ} \\
			& & f. & \foreignlanguage{hebrew}{גְּלֶ֫ינָה} & & \\
			\hline
			Inf.\ cs.\ & & & \foreignlanguage{hebrew}{גְּלוֺת} & \foreignlanguage{hebrew}{עֲשׂוֺת} & \foreignlanguage{hebrew}{הֱיוֺת} \\
			Inf.\ abs.\  & & & \foreignlanguage{hebrew}{גָּלֹה} & \foreignlanguage{hebrew}{עָשֹׂה} & \foreignlanguage{hebrew}{הָיֹה} \\
			\hline
			Part.\ act.\ & sg. & m. & \foreignlanguage{hebrew}{גֹּלֶה} & \foreignlanguage{hebrew}{עֹשֶׂה} & \\
			Part.\ pass.\ & sg. & m. & \foreignlanguage{hebrew}{גָלוּי} & \foreignlanguage{hebrew}{עָשׂוּי} & \\
			\hline
		\end{longtable}	
	\end{center}
	
	\noindent \textbf{Notes}
	\nopagebreak	
	
	\noindent Verbs III\,\textit{y} were originally roots with the semivowels /y/ or /w/ as third root consonant. The diphthongs and triphthongs of the early forms of these root were simplified to long vowels, e.g., \textit{-aya > -ā} in the 3 m.\ sg.\ form \textit{galaya > galā} (cf. the original form of the strong verb \textit{*kataba}).
	
	The 3 f.\ sg.\ SC form \foreignlanguage{hebrew}{גָּלְתָה} combines the original suffix \textit{-āt} and the suffix \textit{-ā} of the strong verb. The resulting form \foreignlanguage{hebrew}{גָּלְתָה} has the same syllable structure as the corresponding form of the strong verb \foreignlanguage{hebrew}{כָּתְבָה}.
	
	The 3 c.\ pl.\ SC form \foreignlanguage{hebrew}{גָּלוּ} is the result of the contraction of a diphthong \textit{-ay-} and the 3 c.\ pl.\ morpheme /ū/: \textit{galayū > gālū}.
	
	Suffix conjugation forms  of III\,\textit{y} verbs in the Qal binyan are characterized by a long vowel \textit{-ā} following the second root consonant in the 3 m.\ sg.\ form (e.g., \foreignlanguage{hebrew}{גָּלָה}) or by the long vowel \textit{-ī} before consonantal suffixes (e.g., \foreignlanguage{hebrew}{גָּלִ֫יתָ}).
	
	Prefix conjugation forms  of III\,\textit{y} verbs in the Qal binyan are characterized by the long final vowel \textit{-ǣ} in the forms without suffix. In the imperative m.\ sg.\ the final vowel is \textit{-ē}.
	
	III\,\textit{y} verbs have distinct jussive and \textit{wayyiqṭol} forms. The 3 m./f. sg., 2 m.\ sg.\ jussive and \textit{wayyiqṭol} forms do not have the long final vowel \textit{-ǣ}. The resulting consonant cluster at the end of the word is frequently resolved by insertion of a helping vowel similar to segolate nouns, e.g., \foreignlanguage{hebrew}{וַיִּ֫גֶל}, \foreignlanguage{hebrew}{וַיַּ֫עַשׂ} \textit{and he did}. The prefix vowel is stressed except with the verbs \foreignlanguage{hebrew}{היה} and \foreignlanguage{hebrew}{חיה}. Important forms of some frequent verbs are the following:
	
	\begin{itemize}[noitemsep]
		\item[--] Verb \foreignlanguage{hebrew}{ראה}: jussive \foreignlanguage{hebrew}{יֵ֫רֶא}, \textit{wayyiqṭol} 3 m.\ sg.\ \foreignlanguage{hebrew}{וַיַּרְא} (\textit{wayyar(ʾ)} with preservation of the \foreignlanguage{hebrew}{א} in the spelling although it is not pronounced), 3 f.\ sg.\ \foreignlanguage{hebrew}{וַתֵּ֫רֶא}, 1 c.\ sg.\ \foreignlanguage{hebrew}{וָאֵ֫רֶא}. The \textit{wayyiqṭol} forms with enclitic pronoun have consonantal \foreignlanguage{hebrew}{א}, e.g., \foreignlanguage{hebrew}{וַיִּרְאֵהוּ} \textit{and he saw him} (Judg 19:3)
		\item[--] Verb \foreignlanguage{hebrew}{בכה}: \foreignlanguage{hebrew}{וַיֵּבְךְ} \foreignlanguage{hebrew}{וַתֵּבְךְ}
		\item[--] Verb \foreignlanguage{hebrew}{שׁתה}: \foreignlanguage{hebrew}{וַיֵּשְׁתְּ}
	\end{itemize}
	
	Short jussive and \textit{wayyiqṭol} forms without the long final vowel \textit{-ǣ} are called \textit{apocopate forms} (Latin \textit{forma apocopata} or in the pl.\ \textit{formae apocopatae} from the Latin noun \textit{apocope} which comes from the Greek noun \foreignlanguage{greek}{ἀποκοπή} \textit{a cutting off}).
	
	At times, the long final vowel \textit{-ǣ} is preserved in jussive and \textit{wayyiqṭol} forms, e.g., \foreignlanguage{hebrew}{יִרְאֶה} \textit{let him see}, \foreignlanguage{hebrew}{וַיִּבְנֶה} \textit{and he built}, \foreignlanguage{hebrew}{וַיַּעֲשֶׂה} \textit{and he made}.
	
	% Long forms of the jussive of ראה are found in Job 3:9 and 2 Kgs 6:17.
	
	Instead of distinct forms for the cohortative, III\,\textit{y} verbs use the regular prefix conjugation forms 1 c.\ sg./pl.\ as volitive forms, e.g., \foreignlanguage{hebrew}{נִבְנֶה} \textit{let us build} (Gen 11:4).
	
	The forms of the participle active are \foreignlanguage{hebrew}{גֹּלֶה} (m.\ sg.), \foreignlanguage{hebrew}{גֹּלָה} (f.\ sg.), \foreignlanguage{hebrew}{גֹּלִים} (m.\ pl.), \foreignlanguage{hebrew}{גֹּלוֺת} (fem.\ pl.). The forms of the participle passive are \foreignlanguage{hebrew}{גָּלוּי} (m.\ sg.), \foreignlanguage{hebrew}{גְּלוּיָה} (f.\ sg.), \foreignlanguage{hebrew}{גְּלוּיִים} (m.\ pl.), \foreignlanguage{hebrew}{גְּלוּיוֺת} (fem.\ pl.).
	
	Verbs III\,\textit{y} with gutturals  have the same vowel differences when compared to verbs without gutturals as strong verbs, e.g., \foreignlanguage{hebrew}{עֲלִיתֶם} \textit{you went up} (2 m.\ sg.\ SC), \foreignlanguage{hebrew}{יַעֲלוּ} \textit{they shall go up} (3 m.\ pl.\ jussive), \foreignlanguage{hebrew}{עֲלֵה} \textit{go up!} (impv.\ m.\ sg.). As a stative verb, the verb \foreignlanguage{hebrew}{חרה} has prefix conjugation forms with the prefix vowel \textit{segol}, e.g., \foreignlanguage{hebrew}{יֶחֱרֶה}. The jussive and \textit{wayyiqṭol} 3 m.\ sg.\ are \foreignlanguage{hebrew}{יִ֫חַר} and \foreignlanguage{hebrew}{וַיִּ֫חַר}.
	
	% Action point: translations of all forms in the previous paragraph
	
	The very frequent verb \foreignlanguage{hebrew}{היה} \textit{to be} has \textit{ḥatef segol} with the first root consonant in the 2 m.\ pl.\ SC form: \foreignlanguage{hebrew}{הֱיִיתֶם} (the 2 f.\ pl.\ form is not attested in Biblical Hebrew). With the conjunction \foreignlanguage{hebrew}{וְ} the form is \foreignlanguage{hebrew}{וִהְיִיתֶם}. In the PC the verb \foreignlanguage{hebrew}{היה} has the prefix vowel /i/ in all forms except the 1 c.\ sg.\ despite the guttural \foreignlanguage{hebrew}{ה}. The first root consonant \foreignlanguage{hebrew}{ה} always has silent \textit{šwa}. In the \textit{wayyiqṭol} form 3 m.\ sg.\ the \textit{dageš forte} in the prefix consonant \foreignlanguage{hebrew}{י} is lost: \foreignlanguage{hebrew}{וַיְהִי}. The inf.\ cs.\ with the proclitic prepositions is \foreignlanguage{hebrew}{לִהְיוֺת} and \foreignlanguage{hebrew}{בִּהְיוֺת}. The inf.\ abs.\ may also be \foreignlanguage{hebrew}{הָיוֺ}. There is one occurrence of a participle in Exod 9:3: \foreignlanguage{hebrew}{הוֺיָה} (f.\ sg.).
	
	The verb \foreignlanguage{hebrew}{חיה} \textit{to live} has forms corresponding to those of the verb \foreignlanguage{hebrew}{היה}, e.g., \foreignlanguage{hebrew}{יִחְיֶה}, \foreignlanguage{hebrew}{תִּחְיֶה}, \foreignlanguage{hebrew}{וִחְיִיתֶם}, \foreignlanguage{hebrew}{וַיְחִי}, etc. The 3 m.\ sg.\ SC form may also be \foreignlanguage{hebrew}{חַי}, as pausal form \foreignlanguage{hebrew}{חָ֑י} and with consecutive \textit{waw} \foreignlanguage{hebrew}{וָחַי}.
	
	The verb \foreignlanguage{hebrew}{נטה} \textit{to stretch out} is doubly weak being I\, \textit{n} and III\,\textit{y}. It has prefix conjugation forms like \foreignlanguage{hebrew}{יִטֶּה}, jussive forms like \foreignlanguage{hebrew}{יֵט} (3 m.\ sg.) and \foreignlanguage{hebrew}{תֵּט} (2 f.\ sg.)  and \textit{wayyiqṭol} forms like \foreignlanguage{hebrew}{וַיֵּט} (3 m.\ sg.) and \foreignlanguage{hebrew}{וַתֵּט} (3 f.\ sg.) with only \emph{one} root consonant.
	
	
\section{Weak Verbs: III\,ʾ Verbs}
	
III\,ʾ verbs are characterized by the loss of the /ʾ/ as a consonant at the end of syllables. The silent (or: quiescent) \foreignlanguage{hebrew}{א} is usually retained in the spelling of the forms which makes the identification of forms of III\,ʾ verbs easy.
	
	\begin{center}
		\begin{longtable}{|lll|r|r|r|}
			\hline
			\multicolumn{3}{|c|}{} & \multicolumn{1}{c|}{Fientive} & \multicolumn{1}{c|}{Stative} &  \\
			\cline{4-5}
			\multicolumn{3}{|c|}{} & \foreignlanguage{hebrew}{מצא} & \foreignlanguage{hebrew}{שׂנא} & \foreignlanguage{hebrew}{נשׂא} \\
			\hline
			\endhead
			\hline
			\endfoot
			SC & sg. & 3 m.& \foreignlanguage{hebrew}{מָצָא} & \foreignlanguage{hebrew}{שָׂנֵא} & \foreignlanguage{hebrew}{נָשָׂא} \\
			& & 3 f. & \foreignlanguage{hebrew}{מָצְאָה} & \foreignlanguage{hebrew}{שָׂנְאָה} & \foreignlanguage{hebrew}{נָשְׂאָה} \\
			& & 2 m. & \foreignlanguage{hebrew}{מָצָאתָ} & \foreignlanguage{hebrew}{שָׂנֵאתָ} & \foreignlanguage{hebrew}{נָשָׂאתָ} \\
			& & 2 f. & \foreignlanguage{hebrew}{מָצָאת} & \foreignlanguage{hebrew}{שָׂנֵאת} & \foreignlanguage{hebrew}{נָשָׂאת} \\
			& & 1 c. & \foreignlanguage{hebrew}{מָצָאתִי} & \foreignlanguage{hebrew}{שָׂנֵאתִי} & \foreignlanguage{hebrew}{נָשָׂאתִי} \\
			& pl. & 3 c. & \foreignlanguage{hebrew}{מָצְאוּ} & \foreignlanguage{hebrew}{שָׂנְאוּ} & \foreignlanguage{hebrew}{נָשׂאוּ}\\
			& & 2 m. & \foreignlanguage{hebrew}{מְצָאתֶם} & \foreignlanguage{hebrew}{שְׂנֵאתֶם} & \foreignlanguage{hebrew}{נְשָׂאתֶם} \\
			& & 2 f. & \foreignlanguage{hebrew}{מְצָאתֶן} & \foreignlanguage{hebrew}{שְׂנֵאתֶן} & \foreignlanguage{hebrew}{נְשָׂאתֶן}\\
			& & 1 c. & \foreignlanguage{hebrew}{מָצָאנוּ} & \foreignlanguage{hebrew}{שָׂנֵאנוּ} & \foreignlanguage{hebrew}{נָשָׂאנוּ} \\
			\hline
			PC & sg. & 3 m. & \foreignlanguage{hebrew}{יִמְצָא} & \foreignlanguage{hebrew}{יִשְׂנָא} & \foreignlanguage{hebrew}{יִשָּׂא} \\
			& & 3 f. & \foreignlanguage{hebrew}{תִּמְצָא} & \foreignlanguage{hebrew}{תִּשְׂנָא} & \foreignlanguage{hebrew}{תִּשָּׂא} \\
			& & 2 m. & \foreignlanguage{hebrew}{תִּמְצָא} & \foreignlanguage{hebrew}{תִּשְׂנָא} & \foreignlanguage{hebrew}{תִּשָּׂא} \\
			& & 2 f. & \foreignlanguage{hebrew}{תִּמְצְאִי} & \foreignlanguage{hebrew}{תִּשְׂנְאִי} & \foreignlanguage{hebrew}{תִּשְׂאִי} \\
			& & 1 c. & \foreignlanguage{hebrew}{אֶמְצָא} & \foreignlanguage{hebrew}{אֶשְׂנָא} & \foreignlanguage{hebrew}{אֶשָּׂא} \\
			\hline
			\pagebreak
			\hline
			& pl. & 3 m. & \foreignlanguage{hebrew}{יִמְצְאוּ} & \foreignlanguage{hebrew}{יִשְׂנְאוּ} & \foreignlanguage{hebrew}{יִשְׂאוּ} \\
			& & 3 f. & \foreignlanguage{hebrew}{תִּמְצֶ֫אנָה} & \foreignlanguage{hebrew}{תִּשְׂנֶאנָה} & \foreignlanguage{hebrew}{תִּשֶּׂ֫אנָה} \\
			& & 2 m. & \foreignlanguage{hebrew}{תִּמְצְאוּ} & \foreignlanguage{hebrew}{תִּשְׂנְאוּ} & \foreignlanguage{hebrew}{תִּשְׂאוּ} \\
			& & 2 f. & \foreignlanguage{hebrew}{תִּמְצֶ֫אנָה} & \foreignlanguage{hebrew}{תִּשְׂנֶאנָה} & \foreignlanguage{hebrew}{תִּשֶּׂ֫אנָה} \\
			& & 1 c. & \foreignlanguage{hebrew}{נִמְצָא} & \foreignlanguage{hebrew}{נִשְׂנָא} & \foreignlanguage{hebrew}{נִשָּׂא} \\
			\hline
			Impv.\ & sg. & m. & \foreignlanguage{hebrew}{מְצָא} & \foreignlanguage{hebrew}{שְׂנָא} & \foreignlanguage{hebrew}{שָׂא} \\
			& & f. & \foreignlanguage{hebrew}{מִצְאִי} & \foreignlanguage{hebrew}{שִׂנְאִי} & \foreignlanguage{hebrew}{שְׂאִי} \\
			& pl. & m. & \foreignlanguage{hebrew}{מִצְאוּ} & \foreignlanguage{hebrew}{שִׂנְאוּ} &  \foreignlanguage{hebrew}{שְׂאוּ} \\
			& & f. & \foreignlanguage{hebrew}{מְצֶאןָ} & & \\
			\hline
			Inf.\ cs.\ & & & \foreignlanguage{hebrew}{מְצֹא} & \foreignlanguage{hebrew}{שְׂנֹא} & \foreignlanguage{hebrew}{שְׂאֵת}/\foreignlanguage{hebrew}{שֵׂאת} \\
			Inf.\ abs.\  & & & \foreignlanguage{hebrew}{מָצוֹא} & \foreignlanguage{hebrew}{שָׂנוֺא} & \\
			\hline
			Part.\ act.\ & sg. & m. & \foreignlanguage{hebrew}{מֹצֵא} & \foreignlanguage{hebrew}{שֹׂנֵא} & \foreignlanguage{hebrew}{נֹשֵׂא} \\
			Part.\ pass.\ & sg. & m. & & \foreignlanguage{hebrew}{שָׂנוּא} & \foreignlanguage{hebrew}{נָשׂוּא} \\
		\end{longtable}	
	\end{center}
	
	
	\noindent \textbf{Notes}
	\nopagebreak
	
	\noindent While in the suffix conjugation fientive and stative verbs are distinguished, there is no such distinction in the prefix conjugation. In the prefix conjugation, III\,ʾ verbs always have the vowel pattern \textit{i--a}.
	
	In forms with vocalic suffix or with enclitic pronouns the \foreignlanguage{hebrew}{א} is preserved as a consonant, e.g., \foreignlanguage{hebrew}{מָצְאוּ} \textit{they found}, \foreignlanguage{hebrew}{תִּמְצְאוּ} \textit{you will find}, \foreignlanguage{hebrew}{מְצָאוֹ} \textit{he found him}, \foreignlanguage{hebrew}{יִמְצָאֲךָ} \textit{he will find you}. Therefore, these forms are strong forms.
	
	In the 3/2 f.\ pl.\ PC forms of III\,ʾ verbs the vocalization follows the corresponding form of III\,\textit{y} verbs. The \foreignlanguage{hebrew}{א} is usually retained in the spelling. At times, other forms of III\,ʾ verbs follow the lead of III\,\textit{y} verbs as well.
	
	The very frequent verb \foreignlanguage{hebrew}{נשׂא} \textit{to lift, carry} is doubly weak (I\,\textit{n} and III\,ʾ) with elision of the third root consonant /ʾ/ at the end of syllables and assimilation of the first root consonant /n/ in the prefix conjugation and complete loss of the /n/ in the imperative and the inf.\ cs.
	
	III\,ʾ verbs do not have distinct jussive or \textit{wayyiqṭol} forms.
	
	The verb \foreignlanguage{hebrew}{חטא} \textit{to sin} as a verb I\,gutt.\ and III\,ʾ has prefix conjugation forms with \textit{segol} as prefix vowel, e.g., \foreignlanguage{hebrew}{יֶחֱטָא}. % Quickly browsing through Ges18 did not show more I gutt and III ʾ verbs that are used in the Qal (2025-02-05)
	
	
	\section{WH-Questions}
	
	\subsection{The Interrogative Pronouns \foreignlanguage{hebrew}{מִי} \textit{Who?} and \foreignlanguage{hebrew}{מָה} \textit{What?}}
	
	Biblical Hebrew has two interrogative pronouns proper, \foreignlanguage{hebrew}{מִי} \textit{who?} and \foreignlanguage{hebrew}{מָה} \textit{what?}. The forms of \foreignlanguage{hebrew}{מָה} differ depending on the initial consonant of the following word.
	
	\begin{itemize}[noitemsep]
		\item[--] Before words beginning with a non-guttural it is \foreignlanguage{hebrew}{מַה־ ּ} (with \textit{dageš forte} in the first consonant of the following word), e.g., \foreignlanguage{hebrew}{מַה־שְּׁמֶ֫ךָ} \textit{What is your name?} (Gen 32:28)
		\item[--] Before the gutturals \foreignlanguage{hebrew}{א}, \foreignlanguage{hebrew}{ר} (always) and \foreignlanguage{hebrew}{ע} (almost always) it is \foreignlanguage{hebrew}{מָה} without \textit{dageš forte}, e.g., \foreignlanguage{hebrew}{מָה אֹמַר אֲלֵהֶם} \textit{What shall I say to them?} (Exod 3:13), \foreignlanguage{hebrew}{מָה רָאוּ בְּבֵיתֶךָ} \textit{What have they seen in your house?} (2\,Kgs 20:15)
		\item[--] Before the gutturals \foreignlanguage{hebrew}{ה} and \foreignlanguage{hebrew}{ח} it is \foreignlanguage{hebrew}{מַה}, e.g., \foreignlanguage{hebrew}{מַה־הוּא} \textit{What is it?} (Exod 16:15), \foreignlanguage{hebrew}{מַה חַטָּאתִי} \textit{What is my sin?} (Gen 31:36) (except before the article)
		\item[--] Before the definite article it is \foreignlanguage{hebrew}{מָה}, e.g., \foreignlanguage{hebrew}{מָה הָאֲבָנִים הָאֵלֶּה} \textit{What are these stones?} (Josh 4:21)
		\item[--] Before gutturals with \textit{qameṣ} or \textit{qameṣ ḥatuf} it is \foreignlanguage{hebrew}{מֶה}, e.g., \foreignlanguage{hebrew}{מֶה עָשִׂיתָ} \textit{What have you done?} (Gen 4:10), \foreignlanguage{hebrew}{מֶה־הָיָה הַדָּבָר בְּנִי} \textit{What is the matter, my son?} (1\,Sam 4:16). At times, this may be the case before gutturals with other vowels as well, e.g., \foreignlanguage{hebrew}{מֶה עָשִׂיתִי מֶה־עֲוֺנִי וּמֶה־חַטָּאתִי} \textit{What have I done? What is my guilt and what is my sin?} (1\,Sam 20:1). This happens rarely even before non-gutturals (2\,Kgs 1:7).
	\end{itemize} 
	
	Both \foreignlanguage{hebrew}{מִי} and \foreignlanguage{hebrew}{מָה} can be used together with prepositions. When \foreignlanguage{hebrew}{מָה} is used with the prepositions \foreignlanguage{hebrew}{בְּ} or \foreignlanguage{hebrew}{כְּ}, the forms are \foreignlanguage{hebrew}{בַּמָּה}, \foreignlanguage{hebrew}{בַּמֶּה}, and \foreignlanguage{hebrew}{כַּמָּה}, and \foreignlanguage{hebrew}{כַּמֶּה}; forms with \textit{pataḥ} and following \textit{dageš forte} do not occur. The following examples illustrate this use.
	
	\vspace{0.5cm}
	
	\begin{tabular}{>{\raggedleft}p{0.35\linewidth} p{0.55\linewidth}}
		\foreignlanguage{hebrew}{לְמִי־אַתָּה} & \textit{To whom do you belong?} (1\,Sam 30:13) (Ps 78:25) \\
		\foreignlanguage{hebrew}{לְמִי אֲנִי אֶעֱבֹד} & \textit{Whom should I serve?} (2\,Sam 16:19) \\
		\foreignlanguage{hebrew}{בַּמָּה} & \textit{With what?}, i.e., \textit{How [are you going to entice him]?} (1\,Kgs 22:21) \\
		\foreignlanguage{hebrew}{כַּמָּה יְמֵי שְׁנֵי חַיֶּֽיךָ} & lit. \textit{How many are the days of the years of your life?} (Gen 47:8) \\
	\end{tabular}
	
	\vspace{0.5cm}
	
	The interrogative pronouns \foreignlanguage{hebrew}{מִי} and \foreignlanguage{hebrew}{מָה} can be used in construct chains, e.g., \foreignlanguage{hebrew}{בַּת־מִי אַתְּ} \textit{Whose daughter are you?} (Gen 24:47) \foreignlanguage{hebrew}{בֶּן־מִי אַתָּה הַנָּעַר} \textit{Whose son are you, young man?} (1\,Sam 17:58)
	
	The interrogative pronoun \foreignlanguage{hebrew}{מָה} preceded by the preposition \foreignlanguage{hebrew}{לְ} has the meaning \textit{why?} (see the next paragraph).
	
	If \foreignlanguage{hebrew}{מִי} is the object of the interrogative clause it is always used with the object marker \foreignlanguage{hebrew}{אֵת}, e.g., \foreignlanguage{hebrew}{אֶת־מִי אֶשְׁלַח} \textit{Whom shall I send?} (Isa 6:8).
	
	Beside \textit{what?} the interrogative pronoun \foreignlanguage{hebrew}{מָה} has other meanings like \textit{how?}, \textit{why?}. It may also be used in exclamations, e.g., \foreignlanguage{hebrew}{מַה־טֹּבוּ אֹהָלֶיךָ יַעֲקֹב מִשְׁכְּנֹתֶיךָ יִשְׂרָאֵל} \textit{How fair are your tents, O Jacob, your dwellings, O Israel!} (Num 24:5).
	
	\subsection{Interrogative adverbs}
	
	Biblical Hebrew has the following frequent interrogative adverbs:
	
	\begin{itemize}[noitemsep]
		\item[--] Local: \foreignlanguage{hebrew}{אַיֵּה} \textit{where?}, \foreignlanguage{hebrew}{אֵי} \textit{where?}, \foreignlanguage{hebrew}{אָ֫נָה} \textit{where to?} (interrogative adverb \foreignlanguage{hebrew}{אָן} \textit{where to?} with directional He), \foreignlanguage{hebrew}{מֵאַיִן} \textit{from where?}
		\item[--] Temporal: \foreignlanguage{hebrew}{מָתַי} \textit{when?}
		\item[--] Causal: \foreignlanguage{hebrew}{לָ֫מָּה} \textit{why?} (preposition \foreignlanguage{hebrew}{לְ} + \foreignlanguage{hebrew}{מָה}), \foreignlanguage{hebrew}{מָדוּעַ} \textit{why?}
	\end{itemize}
	
	\noindent
	
	\noindent \foreignlanguage{hebrew}{אָ֫נָה} may be used together with the preposition \foreignlanguage{hebrew}{עַד} in a temporal sense \textit{How long?}, e.g., \foreignlanguage{hebrew}{עַד־אָנָה יְנַאֲצֻנִי הָעָם הַזֶּה} \textit{How long will this people despise me?} (Num 14:11).
	
	The interrogative adverb \foreignlanguage{hebrew}{לָ֫מָּה} \textit{why?} is \foreignlanguage{hebrew}{לָמָ֫ה} before gutturals.
	
	
	
	\section{Exercises}
	
	\subsection{Translation of Verbal Forms}
	Translate the following verbal forms. Identify the gender (masc., fem., comm.) and number (sg., pl.) the forms of which the English translation is ambiguous (i.e., \textit{you}, \textit{they}). Mark the stressed syllable if stress is not on the last syllable.
	
	\hspace{0.5cm}
	
	\selectlanguage{hebrew}
	
	\noindent
	1~~\foreignlanguage{hebrew}{בָּנוּ}  \hspace{0.3cm}
	2~~\foreignlanguage{hebrew}{אֶגְלֶה}  \hspace{0.3cm}
	3~~\foreignlanguage{hebrew}{מָצָאתָ}  \hspace{0.3cm}
	4~~\foreignlanguage{hebrew}{לֹא אָבִיתִי}  \hspace{0.3cm}
	5~~\foreignlanguage{hebrew}{תֶּחֶטְאוּ}  \hspace{0.3cm}
	6~~\foreignlanguage{hebrew}{תִּחְיוּ}  \hspace{0.3cm}
	7~~\foreignlanguage{hebrew}{וַיַּחֲנוּ}  \hspace{0.3cm}
	8~~\foreignlanguage{hebrew}{מִלְאוּ}  \hspace{0.3cm}
	9~~\foreignlanguage{hebrew}{וַיֵּט}  \hspace{0.3cm}
	10~~\foreignlanguage{hebrew}{וַיִּשְׂאוּ}  \hspace{0.3cm}
	11~~\foreignlanguage{hebrew}{וַתַּעַל}  \hspace{0.3cm}
	12~~\foreignlanguage{hebrew}{עֲנִיתֶם}  \hspace{0.3cm}
	13~~\foreignlanguage{hebrew}{קֹרֵא}  \hspace{0.3cm}
	14~~\foreignlanguage{hebrew}{שְׁתֵה}  \hspace{0.3cm}
	15~~\foreignlanguage{hebrew}{נִשְׁתֶּה}  \hspace{0.3cm}
	
	\selectlanguage{english}
	
	
	\subsection{Translation of Sentences}
	Translate the following sentences from the Hebrew Bible. Names of persons and geographical names in these sentences: \foreignlanguage{hebrew}{אָכִישׁ}, \foreignlanguage{hebrew}{אֱלִישָׁע}, \foreignlanguage{hebrew}{בֵּית־אֵל}, \foreignlanguage{hebrew}{הֶבֶל}, \foreignlanguage{hebrew}{חִלְקִיָּהוּ}, \foreignlanguage{hebrew}{יָבֵשׁ גִּלְעָד}, \foreignlanguage{hebrew}{יְהוֹשֻׁעַ}, \foreignlanguage{hebrew}{יוֹסֵף}, \foreignlanguage{hebrew}{יְרִיחוֺ}, \foreignlanguage{hebrew}{נָחָשׁ}, \foreignlanguage{hebrew}{עָכָן}, \foreignlanguage{hebrew}{קַיִן}, \foreignlanguage{hebrew}{שְׁמוּאֵל}, \foreignlanguage{hebrew}{שָׁפָן},
	
	\vspace{0.5cm}
	
	\selectlanguage{hebrew}
	
	\noindent
	1~~\foreignlanguage{hebrew}{וַיַּעַשׂ לָהֶם מִשְׁתֶּה וַיֹּאכְלוּ וַיִּשְׁתּוּ}  \hspace{0.3cm}
	2~~\foreignlanguage{hebrew}{וַיֹּאמֶר יְהוָה אֶל־קַיִן אֵי הֶבֶל אָחִיךָ וַיֹּאמֶר לֹא יָדַעְתִּי הֲשֹׁמֵר אָחִי אָנֹכִי}  \hspace{0.3cm}
	3~~\foreignlanguage{hebrew}{וַיֹּאמֶר שְׁמוּאֵל אֶל־כָּל־הָעָם הַרְּאִיתֶם אֲשֶׁר בָּחַר־בּוֹ יְהוָה כִּי אֵין כָּמֹהוּ בְּכָל־הָעָם}  \hspace{0.3cm}
	4~~\foreignlanguage{hebrew}{וַיֵּצְאוּ}\LTRfootnote{\space \foreignlanguage{hebrew}{וַיֵּצְאוּ} \textit{and they went out}} \foreignlanguage{hebrew}{בְנֵי־הַנְּבִיאִים אֲשֶׁר־בֵּית־אֵל אֶל־אֱלִישָׁע וַיֹּאמְרוּ אֵלָיו הֲיָדַעְתָּ כִּי הַיּוֹם יְהוָה לֹקֵחַ אֶת־אֲדֹנֶיךָ מֵעַל רֹאשֶׁךָ וַיֹּאמֶר גַּם־אֲנִי יָדַעְתִּי הֶחֱשׁוּ}\LTRfootnote{\space \foreignlanguage{hebrew}{הֶחֱשׁוּ} \textit{be quiet!} (impv.)} \hspace{0.3cm}
	5~~\foreignlanguage{hebrew}{וַיַּעַן עָכָן אֶת־יְהוֹשֻׁעַ וַיֹּאמַר אָמְנָה}\LTRfootnote{\space \foreignlanguage{hebrew}{אָמְנָה} \textit{ʾɔmnā} \textit{truly, indeed}} \foreignlanguage{hebrew}{אָנֹכִי חָטָאתִי לַיהוָה אֱלֹהֵי יִשְׂרָאֵל וְכָזֹאת וְכָזֹאת עָשִׂיתִי} \hspace{0.3cm}
	6~~\foreignlanguage{hebrew}{וַיַּעַן אָכִישׁ וַיֹּאמֶר אֶל־דָּוִד יָדַעְתִּי כִּי טוֹב אַתָּה בְּעֵינַי כְּמַלְאַךְ אֱלֹהִים אַךְ שָׂרֵי פְלִשְׁתִּים אָמְרוּ לֹא־יַעֲלֶה עִמָּנוּ בַּמִּלְחָמָה}  \hspace{0.3cm}
	7~~\foreignlanguage{hebrew}{וַיַּעַל נָחָשׁ הָעַמּוֹנִי}\LTRfootnote{\space \foreignlanguage{hebrew}{עַמּוֺנִי} \textit{Ammonite}} \foreignlanguage{hebrew}{\foreignlanguage{hebrew}{וַיִּחַן עַל־יָבֵשׁ גִּלְעָד וַיֹּאמְרוּ כָּל־אַנְשֵׁי יָבֵישׁ אֶל־נָחָשׁ כְּרָת־לָנוּ בְרִית וְנַעַבְדֶךָּ}}  \hspace{0.3cm}
	8~~\foreignlanguage{hebrew}{וַיַּעַל יוֹסֵף לִקְבֹּר אֶת־אָבִיו וַיַּעֲלוּ אִתּוֹ כָּל־עַבְדֵי פַרְעֹה זִקְנֵי בֵיתוֹ וְכֹל זִקְנֵי אֶרֶץ־מִצְרָיִם}  \hspace{0.3cm}
	9~~\foreignlanguage{hebrew}{וַיֹּאמֶר חִלְקִיָּהוּ הַכֹּהֵן הַגָּדוֹל עַל־שָׁפָן הַסֹּפֵר סֵפֶר הַתּוֹרָה מָצָאתִי בְּבֵית יְהוָה וַיִּתֵּן חִלְקִיָּה אֶת־הַסֵּפֶר אֶל־שָׁפָן וַיִּקְרָאֵהוּ}  \hspace{0.3cm}
	10~~\foreignlanguage{hebrew}{וַיְהִי בִּהְיוֹת יְהוֹשֻׁעַ בִּירִיחוֹ וַיִּשָּׂא עֵינָיו וַיַּרְא וְהִנֵּה־אִישׁ עֹמֵד לְנֶגְדּוֹ וְחַרְבּוֹ שְׁלוּפָה}\LTRfootnote{\space \foreignlanguage{hebrew}{שׁלף} Q.\ \textit{to draw out; to draw off}} \foreignlanguage{hebrew}{בְּיָדוֹ וַיֵּלֶךְ יְהוֹשֻׁעַ אֵלָיו וַיֹּאמֶר לוֹ הֲלָנוּ אַתָּה אִם־לְצָרֵינוּ} \hspace{0.3cm}
	
	\selectlanguage{english}
	
	% Include Gen 2:24
	
	
	\chapter{Chapter 13}
	
	\renewcommand\arraystretch{1.4}
	
	\section{Vocabulary}
	
	\subsection{Verbs}
	
	\begin{center}
		
		% For the centering of the separation between the two columns see the documentation of the array package, page 2 
		
		\begin{longtable}{>{\raggedleft}p{0.175\linewidth} p{0.75\linewidth}}
			\foreignlanguage{hebrew}{הלך} & Q.\ \textit{to go} \\
			\foreignlanguage{hebrew}{טמא} & Q.\ \textit{to be unclean, become unclean} (stative verb, SC \foreignlanguage{hebrew}{טָמֵא}) \\
			\foreignlanguage{hebrew}{יטב} & Q.\ \textit{to be good, well, glad, pleasant} \\ % BDB
			\foreignlanguage{hebrew}{יצא} & Q.\ \textit{to go out, come out} \\
			\foreignlanguage{hebrew}{יצק} & Q.\ \textit{to pour out} \\
			\foreignlanguage{hebrew}{ירא} & Q.\ \textit{to fear, be afraid} (stative verb, SC \foreignlanguage{hebrew}{יָרֵא}; with direct object or prep. \foreignlanguage{hebrew}{מִן} or \foreignlanguage{hebrew}{מִפְּנֵי} indicate the object) \\
			\foreignlanguage{hebrew}{ירשׁ} & Q.\ \textit{to take possession of, inherit, dispossess} \\
			\foreignlanguage{hebrew}{רעה} & Q.\ \textit{to pasture, tend, graze} (part.\ \foreignlanguage{hebrew}{רֹעֶה} \textit{shepherd}) \\
			\foreignlanguage{hebrew}{שׂנא} & Q.\ \textit{to hate} (stative verb, PC \foreignlanguage{hebrew}{שָׁנֵא}) \\
		\end{longtable}
	\end{center}
	
	\subsection{Nouns}
	
	\begin{center}
		\begin{longtable}{>{\raggedleft}p{0.175\linewidth} p{0.75\linewidth}}
			\foreignlanguage{hebrew}{אֹזֶן} & \textit{ear} (du. \foreignlanguage{hebrew}{אָזְנַיִם} \textit{ʾɔznáyim}) \\
			\foreignlanguage{hebrew}{אֵל} & \textit{God, god} \\
			\foreignlanguage{hebrew}{אַמָּה} & \textit{cubit} \\
			\foreignlanguage{hebrew}{אַף} & \textit{nose, anger, nostrils} (gem. noun; dual \foreignlanguage{hebrew}{אַפַּיִם}) \\
			\foreignlanguage{hebrew}{אַפַּיִם} & \textit{face} (dual of \foreignlanguage{hebrew}{אַף}) \\
			\foreignlanguage{hebrew}{בְּאֵר} & \textit{well} (of underground water), \textit{watering place} (fem.) \\	% HALOT
			\foreignlanguage{hebrew}{בָּשָׂר} & \textit{flesh, meat} \\
			\foreignlanguage{hebrew}{גֶּפֶן} & \textit{vine} \\
			\foreignlanguage{hebrew}{דַּעַת} & \textit{knowledge} (inf.\ cs.\ Q. \foreignlanguage{hebrew}{ידע}) \\
			\foreignlanguage{hebrew}{זֶרַע} & \textit{seed, sowing; offspring, descendants} \\ % BDB
			\foreignlanguage{hebrew}{חֶסֶד} & \textit{joint obligation, loyalty, faithfulness, goodness, graciousness} \\ % HALOT
			\foreignlanguage{hebrew}{טָמֵא} & \textit{unclean} (adj., m.\ pl.\ \foreignlanguage{hebrew}{טְמֵאִים}) \\
			\foreignlanguage{hebrew}{כֶּרֶם} & \textit{vineyard} \\
			\foreignlanguage{hebrew}{מוֺעֵד} & \textit{appointed time, appointed place, meeting} (\foreignlanguage{hebrew}{אֹהֶל מוֺעֵד} \textit{tent of meeting}) \\
			\foreignlanguage{hebrew}{מַטֶּה} & \textit{staff, tribe} \\
			\foreignlanguage{hebrew}{קָצֶה} & \textit{edge, end, extremity} \\	% HALOT
		\end{longtable}
	\end{center}
	
	\subsection{Other Parts of Speech}
	
	\begin{center}
		\begin{longtable}{>{\raggedleft}p{0.175\linewidth} p{0.75\linewidth}}
			\foreignlanguage{hebrew}{אָז} & \textit{then, at that time} (often with a PC form with past tense meaning) \\
			\foreignlanguage{hebrew}{אַחַר} & \textit{behind, afterwards} (adv.), \textit{behind, after} (prep.) \\
			\foreignlanguage{hebrew}{בֵּין} & \textit{between} (prep., usually \foreignlanguage{hebrew}{בֵּין ... וּבֵין} or \foreignlanguage{hebrew}{ בֵּין ... לְ}) \\
			\foreignlanguage{hebrew}{לִקְרַאת} & \textit{toward, against} (prep.; \foreignlanguage{hebrew}{לְ} + inf.\ cs.\ of \foreignlanguage{hebrew}{קרא}\textsubscript{2} \textit{ to encounter}) \\
			\foreignlanguage{hebrew}{עַל־כֵּן} & \textit{therefore} \\
		\end{longtable}
	\end{center}
	
	\section{Weak Verbs: I \textit{y} Verbs}
	
	In the suffix conjugation I\,\textit{y} verbs have strong forms with three root consonants. Forms of the prefix conjugation are always weak, and so are most forms of the imperative and the infinitive construct. The infinitive absolute and the participles are strong.
	
	In the prefix conjugation, two groups of I\,\textit{y} verbs need to be differentiated. There are I\,\textit{y} verbs with the vowel pattern \textit{i--a} and verbs with the vowel pattern \textit{a--i}. The verb \foreignlanguage{hebrew}{יכל} \textit{to be able} is a case apart.
	
	The group of I\,\textit{y} comprises two groups of verbs; first, verbs with /y/ as first root consonant originally. and, second, verbs with /w/ as first root consonant originally. In early Northwest Semitic initial /w/ became /y/. As a consequence, the verbs with an original first root consonant /w/  have /y/ as first root consonant in the suffix conjugation Qal.
	
	% Previous version of the final sentence: In general, verbs with /w/ as original root consonant follow the pattern of verbs with /y/ as first root consonant (changed on 2025-03-23).
	
	\renewcommand\arraystretch{1.4}
	
	\begin{center}
		\begin{longtable}{|lll|r|r|r|r|}
			\hline
			\multicolumn{3}{|c|}{} & \foreignlanguage{hebrew}{ירשׁ} & \foreignlanguage{hebrew}{ישׁב} & \foreignlanguage{hebrew}{הלך} & \foreignlanguage{hebrew}{יכל} \\
			\hline
			\endhead
			\hline
			\endfoot
			%		\hline
			SC & sg. & 3 m. & \foreignlanguage{hebrew}{יָרַשׁ} & \foreignlanguage{hebrew}{יָשַׁב} & \foreignlanguage{hebrew}{הָלַךְ} & \foreignlanguage{hebrew}{יָכֹל} \\
			& & 3 f. & \foreignlanguage{hebrew}{יָרְשָׁה} & \foreignlanguage{hebrew}{יָשְׁבָה} & \foreignlanguage{hebrew}{הָלְכָה} & \foreignlanguage{hebrew}{יָכְלָה} \\
			& & 2 m. & \foreignlanguage{hebrew}{יָרַשְׁתָּ} & \foreignlanguage{hebrew}{יָשַׁבְתָּ} & \foreignlanguage{hebrew}{הָלַכֽתָּ} &  \\
			& & 2 f. & \foreignlanguage{hebrew}{יָרַשְׁתְּ} & \foreignlanguage{hebrew}{יָשַׁבְתְּ} & \foreignlanguage{hebrew}{הָלַכְתְּ} &  \\
			& & 1 c. & \foreignlanguage{hebrew}{יָרַשְׁתִּי} & \foreignlanguage{hebrew}{יָשַׁבְתִּי} & \foreignlanguage{hebrew}{הָלַכְתִּי} & \foreignlanguage{hebrew}{יָכֹלְתִּי} \\
			\hline
			\pagebreak
			\hline
			& pl. & 3 c. & \foreignlanguage{hebrew}{יָרְשׁוּ} & \foreignlanguage{hebrew}{יָשְׁבוּ} & \foreignlanguage{hebrew}{הָלְכוּ} & \foreignlanguage{hebrew}{יָכְלוּ} \\
			& & 2 m. & \foreignlanguage{hebrew}{יֽרַשְׁתֶּם} & \foreignlanguage{hebrew}{יְשַׁבְתֶּם} & \foreignlanguage{hebrew}{הֲלַכְתֶּם} &  \\
			& & 2 f. & \foreignlanguage{hebrew}{יְרַשְׁתֶּן} & \foreignlanguage{hebrew}{יְשַׁבְתֶּן} & \foreignlanguage{hebrew}{הֲלַכְתֶּן} &  \\
			& & 1 c. & \foreignlanguage{hebrew}{יָרַשְׁנוּ} & \foreignlanguage{hebrew}{יָשַׁבְנוּ} & \foreignlanguage{hebrew}{הָלַכְנוּ} &  \\
			\hline
			PC & sg. & 3 m. & \foreignlanguage{hebrew}{יִירַשׁ} & \foreignlanguage{hebrew}{יֵשֵׁב} & \foreignlanguage{hebrew}{יֵלֵךְ} & \foreignlanguage{hebrew}{יוּכַל} \\
			& & 3 f. & \foreignlanguage{hebrew}{תִּירַשׁ} & \foreignlanguage{hebrew}{תֵּשֵׁב} & \foreignlanguage{hebrew}{תֵּלֵךְ} & \foreignlanguage{hebrew}{תּוּכַל} \\
			& & 2 m. & \foreignlanguage{hebrew}{תִּירַשׁ} & \foreignlanguage{hebrew}{תֵּשֵׁב} & \foreignlanguage{hebrew}{תֵּלֵךְ} & \foreignlanguage{hebrew}{תּוּכַל} \\
			& & 2 f. & \foreignlanguage{hebrew}{תִּירְשִׁי} & \foreignlanguage{hebrew}{תֵּשְׁבִי} & \foreignlanguage{hebrew}{תְּלְכִי} & \foreignlanguage{hebrew}{תּוּכְלִי} \\
			& & 1 c. & \foreignlanguage{hebrew}{אִירַשׁ} & \foreignlanguage{hebrew}{אֵשֵׁב} & \foreignlanguage{hebrew}{אֵלֵךְ} & \foreignlanguage{hebrew}{אוּכַל} \\
			& pl. & 3 m. & \foreignlanguage{hebrew}{יִירְשׂוּ} & \foreignlanguage{hebrew}{יֵשְׁבוּ} & \foreignlanguage{hebrew}{יֵלְכוּ} & \foreignlanguage{hebrew}{יוּכְלוּ} \\
			& & 3 f. & \foreignlanguage{hebrew}{תִּירַ֫שְׁנָה} & \foreignlanguage{hebrew}{תֵּשַׁ֫בְנָה} & \foreignlanguage{hebrew}{תֵּלַכְנָה} &  \\
			& & 2 m. & \foreignlanguage{hebrew}{תִּירְשׁוּ} & \foreignlanguage{hebrew}{תֵּשְׁבוּ} & \foreignlanguage{hebrew}{תֵּלְכוּ} & \foreignlanguage{hebrew}{תּוּכְלוּ} \\
			& & 2 f. & \foreignlanguage{hebrew}{תִּירַ֫שְׁנָה} & \foreignlanguage{hebrew}{תֵּשַׁ֫בְנָה} & \foreignlanguage{hebrew}{תֵּלַכְנָה} &  \\
			& & 1 c. & \foreignlanguage{hebrew}{נִירַשׁ} & \foreignlanguage{hebrew}{נֵשֵׁב} & \foreignlanguage{hebrew}{נֵלֵךְ} & \foreignlanguage{hebrew}{נוּכַל} \\
			\hline
			Jussive & sg. & 3 m. & \foreignlanguage{hebrew}{יִירַשׁ} & \foreignlanguage{hebrew}{יֵשֵׁב} & \foreignlanguage{hebrew}{יֵלֵךְ} & \\
			\textit{wayyiqṭol} & sg. & 3 m. & \foreignlanguage{hebrew}{וַיִּירַשׁ} & \foreignlanguage{hebrew}{וַיֵּשֶׁב} & \foreignlanguage{hebrew}{וַיֵּלֶךְ} &  \foreignlanguage{hebrew}{וַיוּכַל} \\
			\hline
			Impv.\ & sg. & m. & \foreignlanguage{hebrew}{רֵשׁ} & \foreignlanguage{hebrew}{שֵׁב} & \foreignlanguage{hebrew}{לֵךְ} & \\
			& & f. & & \foreignlanguage{hebrew}{שְׁבִי} & \foreignlanguage{hebrew}{לְכִי} & \\
			& pl. & m. & \foreignlanguage{hebrew}{רְשׁוּ} & \foreignlanguage{hebrew}{שְׁבוּ} & \foreignlanguage{hebrew}{לְכוּ} & \\
			& & f. & & & & \\
			\hline
			Inf.\ cs.\ & & & \foreignlanguage{hebrew}{רֶשֶׁת} & \foreignlanguage{hebrew}{שֶׁבֶת} & \foreignlanguage{hebrew}{לֶכֶת} & \foreignlanguage{hebrew}{יְכֹלֶת} \\
			Inf.\ abs.\  & & & & \foreignlanguage{hebrew}{יָשׁוֺב} & \foreignlanguage{hebrew}{הָלוֺךְ} & \foreignlanguage{hebrew}{יָכוֺל} \\
			\hline
			Part.\ act.\ & sg. & m. & \foreignlanguage{hebrew}{יֹרֵשׁ} & \foreignlanguage{hebrew}{יֹשֵׁב} & \foreignlanguage{hebrew}{הֹלֵךְ} & \\
			% Part.\ pass.\ & sg. & m. & & & & \\
		\end{longtable}	
	\end{center}
	
	\noindent \textbf{Notes}
	\nopagebreak
	
	\noindent In the \textit{i--a} verbs -- represented by the verb \foreignlanguage{hebrew}{ירשׁ} in the first column -- the development of the prefix conjugation form was \textit{yiyraš > yīraš} \foreignlanguage{hebrew}{יִירַשׁ} with contraction of \textit{iy > ī}.
	
	The group with the \textit{a--i} vowel pattern -- represented by the verb \foreignlanguage{hebrew}{ישׁב} in the second column -- contains the following six verbs: \foreignlanguage{hebrew}{ישׁב} \textit{to sit}, \foreignlanguage{hebrew}{ידע} \textit{to know}, \foreignlanguage{hebrew}{יצא} \textit{to go out},  \foreignlanguage{hebrew}{ירד} \textit{to go down}, \foreignlanguage{hebrew}{ילד} \textit{to bear, beget} and the verb \foreignlanguage{hebrew}{הלך} \textit{to go}.
	
	The explanation of the actual vowel pattern \textit{ē--ē} of the \textit{a--i} verbs \foreignlanguage{hebrew}{יֵשֵׁב} is disputed.\footnote{\space The same vowel pattern in its original form \textit{a--i} is found in the prefix conjugation forms of the verb \foreignlanguage{hebrew}{נתן}.} Based on an original form \textit{*yašib} with only two root consonants and an open prefix syllable the form either developed along the path \textit{*yašib > *yašēb > yēšēb} with assimilation of the prefix vowel to the thematic vowel or along the path \textit{*yašib > *yayšib > yēšēb} with addition of the semivowel /y/ to the bring the form closer to the regular verb with three root consonants and subsequent contraction of the diphthong \textit{ay > ē}.
	
	In both types of I\,\textit{y} verbs the prefix vowel is unchangeable while the thematic vowel is reduced to \textit{šwa} in context forms with vocalic suffixes, e.g., \foreignlanguage{hebrew}{תֵּשְׁבוּ}, \foreignlanguage{hebrew}{יִירְשׁוּ}. In pausal forms the thematic vowel is preserved, e.g., \foreignlanguage{hebrew}{יֵשֵׁ֑בוּ}, \foreignlanguage{hebrew}{יִבָ֑שׁוּ} (with defective spelling of the prefix).
	
	The prefix vowel \textit{ū} of the verb \foreignlanguage{hebrew}{יכל} is the result of the contraction of the diphthong \textit{iw > ū}: \textit{*yiwkal > yūkal} \foreignlanguage{hebrew}{יוּכַל}.\footnote{\space At face value, the prefix conjugation forms of \foreignlanguage{hebrew}{יכל} look like Hophal forms of I\,\textit{y} or II\,\textit{wy}/II\,gem.\ verbs, but they are indeed Qal forms.}
	
	In the prefix conjugation forms of the verb \foreignlanguage{hebrew}{ידע} the thematic vowel is changed to /a/ because of the the guttural \foreignlanguage{hebrew}{ע}: \foreignlanguage{hebrew}{יֵדַע}, \foreignlanguage{hebrew}{תֵּדַע}, \foreignlanguage{hebrew}{תֵּדַע}, \foreignlanguage{hebrew}{תֵּדְעִי} etc. In pausal forms the thematic vowel is lengthened to \textit{qameṣ}, e.g., \foreignlanguage{hebrew}{יֵדָ֑ע}, \foreignlanguage{hebrew}{תֵּדָ֑עִי}. The same vowel /a/ is also found in the forms of the impv.\ \foreignlanguage{hebrew}{דַּע}, etc., and the inf.\ cs.\ \foreignlanguage{hebrew}{דַּעַת}.
	
	In the frequent doubly weak verbs \foreignlanguage{hebrew}{יצא} \textit{to go out} and \foreignlanguage{hebrew}{ירא} \textit{to fear} the /ʾ/ becomes silent at the end of the syllable, e.g., \foreignlanguage{hebrew}{יֵצֵא} \textit{yēṣē(ʾ)} and \foreignlanguage{hebrew}{יִירָא} \textit{yīrā(ʾ)}. In forms with vocalic suffix the \foreignlanguage{hebrew}{א} functions as consonant, e.g., \foreignlanguage{hebrew}{יֵצְאוּ} and \foreignlanguage{hebrew}{תִּירְאוּ}. In 2/3 f.\ pl.\ forms the vowels follow the pattern of III\,\textit{y} verbs, e.g., \foreignlanguage{hebrew}{תֵּצֶ֫אנָה} (2 f.\ pl.) \foreignlanguage{hebrew}{וַתִּירֶ֫אןָ} (3 f.\ pl.) (cf.\ \foreignlanguage{hebrew}{תִּגְלֶ֫ינָה}). The forms of the inf.\ cs.\ are \foreignlanguage{hebrew}{צֵאת} and \foreignlanguage{hebrew}{יְרֹא} and \foreignlanguage{hebrew}{יִרְאָה}, respectively.
	
	The verb \foreignlanguage{hebrew}{ירא} \textit{to fear} is a stative verb with suffix conjugation forms like \foreignlanguage{hebrew}{יָרֵא} (3 m.\ sg.), \foreignlanguage{hebrew}{יָרֵאתִי} (1 c.\ sg.), \foreignlanguage{hebrew}{יְרֵאתֶם} (2 pl.). The noun \foreignlanguage{hebrew}{יִרְאָה} functions as inf.\ cs.\ besides the rare \foreignlanguage{hebrew}{יְרֹא}.
	
	In the 2/3 f.\ pl.\ form \foreignlanguage{hebrew}{תֵּשַׁ֫בְנָה} the vowel /a/ between the second and third root consonant is due to Philippi's Law (\textit{i > a} in stressed closed syllable).
	
	In \textit{wayyiqṭol} forms of \textit{a--i} type verbs without suffix (except in 1 c.\ sg.\ forms) the stress is moved to the prefix syllable and the thematic vowel is changed to /æ/, e.g., \foreignlanguage{hebrew}{וַיֵּ֫שֶׁב}, \foreignlanguage{hebrew}{וַיֵּ֫רֶד}, \foreignlanguage{hebrew}{וַיֵּ֫לֶךְ}, but in the 1 c.\ sg.\ the stress is on the last syllable \foreignlanguage{hebrew}{וָאֵרֵד}, \foreignlanguage{hebrew}{וָאֵשֵׁב}. In pausal forms stress does not move, e.g., \foreignlanguage{hebrew}{וַיֵּרַ֑ד} \foreignlanguage{hebrew}{וַיֵּלַ֑ךְ}. As retraction of stress in verbal forms only happens when the penultimate syllable is open and the final syllable is closed, stress does not move in forms of the verbs \foreignlanguage{hebrew}{יצא} and \foreignlanguage{hebrew}{ירא} due the quiescent /ʾ/ which results in an open final syllable.
	
	Suffix-less jussive forms of I\,\textit{y} verbs of the \textit{a--i} type are frequently stressed on the last syllable, e.g., \foreignlanguage{hebrew}{וְיֵרֵד אֵלָ֑י} \textit{and let him come to me} (1\,Sam 17:8) but forms with stress on the penultimate syllable are also attested, e.g., \foreignlanguage{hebrew}{תֵּ֫רֶד אֵשׁ מִן־הַשָּׁמַיִם} \textit{let fire come down from heaven} (2\,Kgs 1:12).
	
	Verbs I\,\textit{y} with \foreignlanguage{hebrew}{צ} as second root consonant have irregular prefix conjugation Qal forms with doubling of the \foreignlanguage{hebrew}{צ} as well as forms with a single /ṣ/, e.g., \foreignlanguage{hebrew}{וַיִּצֹק} \textit{and he poured} (Gen 28:18), \foreignlanguage{hebrew}{אֶצֹּק} \textit{I shall pour} (Isa 44:3) from the verb \foreignlanguage{hebrew}{יצק}.\footnote{\space At face value forms like \foreignlanguage{hebrew}{אֶצֹּק} look like forms of I\,\textit{n} verbs, although they are indeed forms of I\,\textit{y} verbs.}
	
	% Is it necessary to mention this last point? Does it need to be expanded? What does Jenni present?
	
	
	\section{Numerals (I): Numbers 1--10}
	
	\subsection{The Forms of Cardinal Numbers 1--10}
	The cardinal numbers from 1 to 10 have the following forms:\footnote{\space In Modern Hebrew the numerals for \textit{four} \foreignlanguage{hebrew}{אַרְבַּע} and \textit{eight} \foreignlanguage{hebrew}{שׁׂמוֺנֶה} are stressed on the penultimate syllable.}
	
	%Only the numerals 1--10 in the fourth column (fem. nouns, abs. st.) need to be memorized. These numerals are used in Modern Hebrew for counting from 1--10. The other numerals are then easy to recognize.
	%  The masc.\ form \foreignlanguage{hebrew}{אֶחָד} may used instead of \foreignlanguage{hebrew}{אַחַת} in Modern Hebrew.
	
	\vspace{0.5cm}
	
	\begin{Center}
		\begin{tabular}{|r|rr|rr|}
			\hline
			\multicolumn{1}{|c}{} & \multicolumn{2}{|c|}{With masc. nouns} & \multicolumn{2}{c|}{With fem. nouns} \\
			& abs. st. & cs.\ st. & abs. st. & cs.\ st. \\
			\hline
			1 & \foreignlanguage{hebrew}{אֶחָד} & \foreignlanguage{hebrew}{אַחַד} & \foreignlanguage{hebrew}{אַחַת} & \foreignlanguage{hebrew}{אַחַת} \\
			2 & \foreignlanguage{hebrew}{שְׁנַ֫יִם} & \foreignlanguage{hebrew}{שְׁנֵי} & \foreignlanguage{hebrew}{שְׁתַּ֫יִם} & \foreignlanguage{hebrew}{שְׁתֵּי} \\
			3 & \foreignlanguage{hebrew}{שְׁלֹשָׁה} & \foreignlanguage{hebrew}{שְׁלֹשֶׁת} & \foreignlanguage{hebrew}{שָׁלֹשׁ} & \foreignlanguage{hebrew}{שְׁלֹשׁ} \\
			4 & \foreignlanguage{hebrew}{אַרְבָּעָה} & \foreignlanguage{hebrew}{אַרְבַּ֫עַת} & \foreignlanguage{hebrew}{אַרְבַּע} & \foreignlanguage{hebrew}{אַרְבַּע} \\
			5 & \foreignlanguage{hebrew}{חֲמִשָּׁה} & \foreignlanguage{hebrew}{חֲמֵ֫שֶׁת} & \foreignlanguage{hebrew}{חָמֵשׁ} & \foreignlanguage{hebrew}{חֲמֵשׁ} \\
			6 & \foreignlanguage{hebrew}{שִּׁשָּׁה} & \foreignlanguage{hebrew}{שֵׁ֫שֶׁת} & \foreignlanguage{hebrew}{שֵׁשׁ} & \foreignlanguage{hebrew}{שֵׁשׁ} \\
			7 & \foreignlanguage{hebrew}{שִׁבְעָה} & \foreignlanguage{hebrew}{שִׁבְעַת} & \foreignlanguage{hebrew}{שֶׁ֫בַע} & \foreignlanguage{hebrew}{שְׁבַע} \\
			8 & \foreignlanguage{hebrew}{שְׁמֹנָה} & \foreignlanguage{hebrew}{שְׁמֹנַת} & \foreignlanguage{hebrew}{שְׁמֹנֶה} & \foreignlanguage{hebrew}{שְׁמֹנֶה} \\
			9 & \foreignlanguage{hebrew}{תִּשְׁעָה} & \foreignlanguage{hebrew}{תִּשְׁעַת} & \foreignlanguage{hebrew}{תֵּ֫שַׁע} & \foreignlanguage{hebrew}{תְּשַׁע} \\
			10 & \foreignlanguage{hebrew}{עֲשָׂרָה} & \foreignlanguage{hebrew}{עֲשֶׂ֫רֶת} & \foreignlanguage{hebrew}{עֶ֫שֶׂר} & \foreignlanguage{hebrew}{עֶ֫שֶׂר}  \\
			\hline
		\end{tabular}
	\end{Center}
	
	\vspace{0.5cm}
	
	\subsection{The Use of the Cardinal Numbers 1--10}
	
	The cardinal number for \textit{one} \foreignlanguage{hebrew}{אֶחָד} and \foreignlanguage{hebrew}{אַחַת}, respectively, is an adjective. It has different uses in Biblical Hebrew:
	
	% The use of  \foreignlanguage{hebrew}{אֶחָד} in the plural (Gen Gen 27:44; 29:20) and the use with the article could be included
	
	\begin{itemize}[noitemsep]
		\item[--] When it modifies a noun, it follows the noun and agrees with it in gender and number, e.g., \foreignlanguage{hebrew}{אִישׁ־אֶחָד} \textit{one man} (in \foreignlanguage{hebrew}{אֲנַחְנוּ בְּנֵי אִישׁ־אֶחָד} \textit{We are the sons of one man} Gen 42:13), \foreignlanguage{hebrew}{בְּיוֹם אֶחָד} \textit{on one day} (1\,Sam 2:34), \foreignlanguage{hebrew}{אֶבֶן אַחַת} \textit{one stone} (Josh 4:5).
		\item[--] In cs.\ st.\ followed by a dependent noun, noun phrase or prepositional phrase with \foreignlanguage{hebrew}{מִן}, e.g., \foreignlanguage{hebrew}{אַחַד הֶהָרִים} \textit{one of the mountains}, \foreignlanguage{hebrew}{אַחַת עָרֶיךָ} \textit{one of your towns}, \foreignlanguage{hebrew}{אַחַד מֵהֶם} \textit{one of them}
		\item[--] Like an indefinite article, e.g., \foreignlanguage{hebrew}{אִישׁ אֶחָד} \textit{a man} (1\,Sam 1:1), \foreignlanguage{hebrew}{אִשָּׁה אַחַת} \textit{a woman} (2\,Kgs 4:1)
		\item[--] Like an ordinal number \textit{first}, e.g., \foreignlanguage{hebrew}{שֵׁם הָאַחַת עָרְפָּה וְשֵׁם הַשֵּׁנִית רוּת} \textit{The name of the first one was Orpah and the name of the second Ruth} (Ruth 1:4)
	\end{itemize}
	
	The cardinal number for \textit{two} agrees with the noun it modifies in number and gender, i.e., \foreignlanguage{hebrew}{שְׁנַ֫יִם} is used with masculine nouns and \foreignlanguage{hebrew}{שְׁתַּ֫יִם} is used with feminine nouns.
	
	The cardinal numbers 3--10 are characterized by gender inversion. This means that masculine nouns are used with number words with feminine ending whereas feminine nouns are used with masculine number words.
	
	
	The numerals 2--10 may be used in three different ways when they are used as modifiers of nouns:
	
	\begin{itemize}[noitemsep]
		\item[--] In construct state before the counted noun, e.g., \foreignlanguage{hebrew}{שְׁנֵי בָנִים} \textit{two sons}, \foreignlanguage{hebrew}{שְׁתֵּי בָנוֹת} \textit{two daughters}, \foreignlanguage{hebrew}{שְׁלֹשֶׁת בָּנָיו} \textit{his three sons}, \foreignlanguage{hebrew}{שְׁלֹשׁ סְאִים} \textit{three seahs} (sg.\ \foreignlanguage{hebrew}{שְׂאָה}, f.)
		\item[--] In absolute state before the counted noun,  e.g., \foreignlanguage{hebrew}{שְׁנַיִם־אֲנָשִׁים} \textit{two men}, \foreignlanguage{hebrew}{שְׁתַּיִם נָשִׁים} \textit{two women}, \foreignlanguage{hebrew}{שְׁלֹשָׁה בָנִים} \textit{three sons}, \foreignlanguage{hebrew}{שָׁלֹשׁ עָרִים} \textit{three cities}
		\item[--] In absolute state following the counted noun (especially in lists), e.g., \foreignlanguage{hebrew}{עַמֻּדִים שְׁנַיִם} \textit{two pillars}, \foreignlanguage{hebrew}{עָרִים שְׁתָּ֑יִם} \textit{two cities}, \foreignlanguage{hebrew}{יָמִים שְׁלוֹשָׁה} \textit{three days}, \foreignlanguage{hebrew}{עָרִים שָׁלֹשׁ} \textit{three cities}
	\end{itemize}
	
	Numerals may be used with enclitic pronouns, e.g., \foreignlanguage{hebrew}{שְׁנֵיהֶם} \textit{the two of them}, \foreignlanguage{hebrew}{שְׁלָשְׁתָּם} \textit{the three of them}, \foreignlanguage{hebrew}{אַרְבַּעְתָּם} \textit{the four of them}.
	
	\subsection{Ordinal Numbers 1--10}
	
	\begin{center}
		\begin{tabular}{lrr}
			& \multicolumn{1}{c}{masc.} & \multicolumn{1}{c}{fem.} \\
			1st & \foreignlanguage{hebrew}{רִאשׁוֺן} & \foreignlanguage{hebrew}{רִאשׁוֺנָה} \\
			2nd & \foreignlanguage{hebrew}{שֵׁנִי} & \foreignlanguage{hebrew}{שֵׁנִית} \\
			3rd & \foreignlanguage{hebrew}{שְׁלִישִׁי} & \foreignlanguage{hebrew}{שְׁלִישִׁית} \\
			4th & \foreignlanguage{hebrew}{רֽבִיעִי} & \foreignlanguage{hebrew}{רֽבִיעִית} \\
			5th & \foreignlanguage{hebrew}{חֲמִישִׁי} & \foreignlanguage{hebrew}{חֲמִישִׁית} \\
			6th & \foreignlanguage{hebrew}{שִׁשִּׁי} & \foreignlanguage{hebrew}{שִׁשִּׁית} \\
			7th & \foreignlanguage{hebrew}{שְׁבִיעִי} & \foreignlanguage{hebrew}{שְׁבִיעִית} \\
			8th & \foreignlanguage{hebrew}{שְׁמִינִי} & \foreignlanguage{hebrew}{שְׁמִינִית} \\
			9th & \foreignlanguage{hebrew}{תְּשִׁיעִי} & \foreignlanguage{hebrew}{תְּשִׁיעִית} \\
			10th & \foreignlanguage{hebrew}{עֲשִׂירִי} & \foreignlanguage{hebrew}{עֲשִׂירִית} \\
		\end{tabular}
	\end{center}
	
	Ordinal numbers are adjectives. As such, they follow the noun they modify and agree with it in number and gender. Dedicated forms of cardinal numbers exist only up to \textit{10th}. For values above 10th cardinal numbers are used. With years and months, cardinal numbers are often used instead of ordinal numbers.
	
	
	\section{Adjectives with the Ending \textit{-ī}}
	Adjectives with the ending \textit{-ī} often indicate the belonging to an ethnic group, the origin from a country, area or city. This ending is therefore called \textit{gentilic ending} (from Latin \textit{gens} \textit{tribe, people}). The ending \textit{-ī} can also be used on regular adjectives like \foreignlanguage{hebrew}{נָכְרִי} \textit{foreign}, \foreignlanguage{hebrew}{נָקִי} \textit{free from guilt, innocent}. The different forms of \foreignlanguage{hebrew}{עִבְרִי} \textit{Hebrew} are as follows:
	
	\vspace{0.25cm}
	
	\begin{center}
		\begin{tabular}{lrr}
			& masc. & fem. \\
			sg. & \foreignlanguage{hebrew}{עִבְרִי} & \foreignlanguage{hebrew}{עִבְרִיָּה} \\
			pl. & \foreignlanguage{hebrew}{עִבְרִים} & \foreignlanguage{hebrew}{עִבְרִיּוֺת} \\
		\end{tabular}
	\end{center}
	
	\vspace{0.25cm}
	
	\noindent \textbf{Notes}
	\nopagebreak
	
	\noindent The plural form may have the ending \textit{-iyyīm}, e.g., \foreignlanguage{hebrew}{עִבְרִיִּים} (only once in Exod 3:18), \foreignlanguage{hebrew}{נְקִיִּם}.
	
	The fem.\ sg.\ ending may also be \textit{-īt}, e.g., \foreignlanguage{hebrew}{שִׁפְחָה מִצְרִית} \textit{an Egyptean maidservant} (Gen 16:1), \foreignlanguage{hebrew}{אֲרָמִית} (adv., \textit{in Aramaic}), \foreignlanguage{hebrew}{יְהוּדִית} (adv., \textit{in Judean}).
	
	The ordinal numbers \textit{2nd} to \textit{10th} belong to this category of adjectives. 
	
	
	\section{Exercises}
	
	\subsection{Translation of Verbal Forms}
	
	Translate the following verbal forms. Identify the gender (masc., fem., comm.) and number (sg., pl.) the forms of which the English translation is ambiguous (i.e., \textit{you}, \textit{they}). Mark the stressed syllable if stress is not on the last syllable.
	
	\hspace{0.5cm}
	
	\selectlanguage{hebrew}
	
	\noindent
	1~~\foreignlanguage{hebrew}{אֵלֵךְ}  \hspace{0.3cm}
	2~~\foreignlanguage{hebrew}{נֵלֵךְ}  \hspace{0.3cm}
	3~~\foreignlanguage{hebrew}{תֵּלְכוּ}  \hspace{0.3cm}
	4~~\foreignlanguage{hebrew}{לְכוּ}  \hspace{0.3cm}
	5~~\foreignlanguage{hebrew}{תִּטְמָא}  \hspace{0.3cm}
	6~~\foreignlanguage{hebrew}{יָצָאתִי}  \hspace{0.3cm}
	7~~\foreignlanguage{hebrew}{יָצְאוּ}  \hspace{0.3cm}
	8~~\foreignlanguage{hebrew}{יֵצְאוּ}  \hspace{0.3cm}
	9~~\foreignlanguage{hebrew}{יֹצְאִים}  \hspace{0.3cm}
	10~~\foreignlanguage{hebrew}{יָרַשְׁנוּ}  \hspace{0.3cm}
	11~~\foreignlanguage{hebrew}{תִּירְשׁוּ}  \hspace{0.3cm}
	12~~\foreignlanguage{hebrew}{וַיִּירְשׁוּ}  \hspace{0.3cm}
	13~~\foreignlanguage{hebrew}{יָרֵאתִי}  \hspace{0.3cm}
	14~~\foreignlanguage{hebrew}{תִּירְאוּ}  \hspace{0.3cm}
	15~~\foreignlanguage{hebrew}{תִּרְאוּ}  \hspace{0.3cm}
	16~~\foreignlanguage{hebrew}{יָשְׁבוּ}  \hspace{0.3cm}
	17~~\foreignlanguage{hebrew}{וַיֵּשְׁבוּ}  \hspace{0.3cm}
	18~~\foreignlanguage{hebrew}{שְׁבָה}  \hspace{0.3cm}
	19~~\foreignlanguage{hebrew}{שְׁבוּ}  \hspace{0.3cm}
	20~~\foreignlanguage{hebrew}{יָרַדְתָּ}  \hspace{0.3cm}
	
	\selectlanguage{english}
	
	
	\subsection{Translation of Sentences}
	
	Translate the following sentences from the Hebrew Bible. Names of persons and geographical names in these sentences: \foreignlanguage{hebrew}{אֱדוֹם}, \foreignlanguage{hebrew}{אַהֲרֹן}, \foreignlanguage{hebrew}{בֵּית־אֵל}, \foreignlanguage{hebrew}{בֵּית־שֶׁמֶשׁ}, \foreignlanguage{hebrew}{דָּוִד}, \foreignlanguage{hebrew}{יְהוֹשֻׁעַ}, \foreignlanguage{hebrew}{יְרוּשָׁלִַם}, \foreignlanguage{hebrew}{לוֹט}, \foreignlanguage{hebrew}{מִרְיָם}, \foreignlanguage{hebrew}{מֹשֶׁה}, \foreignlanguage{hebrew}{נוּן}, \foreignlanguage{hebrew}{סִינַי}, \foreignlanguage{hebrew}{צ֫וֺעַר}, \foreignlanguage{hebrew}{קָדֵשׁ}, \foreignlanguage{hebrew}{קַיִן}, \foreignlanguage{hebrew}{רָמֹת גִּלְעָד}, \foreignlanguage{hebrew}{שָׁאוּל}, \foreignlanguage{hebrew}{שִׁמְעִי}
	
	\vspace{0.5cm}
	
	\selectlanguage{hebrew}
	\noindent
	1~~\foreignlanguage{hebrew}{וַיֹּאמֶר יְהוֹשֻׁעַ אֶל־הַכֹּהֲנִים לֵאמֹר שְׂאוּ אֶת־אֲרוֹן הַבְּרִית וְעִבְרוּ לִפְנֵי הָעָם וַיִּשְׂאוּ אֶת־אֲרוֹן הַבְּרִית וַיֵּלְכוּ לִפְנֵי הָעָם}  \hspace{0.3cm}
	2~~\foreignlanguage{hebrew}{וַיִּקְרָא יְהוֹשֻׁעַ בִּן־נוּן אֶל־הַכֹּהֲנִים וַיֹּאמֶר אֲלֵהֶם שְׂאוּ אֶת־אֲרוֹן הַבְּרִית וְשִׁבְעָה כֹהֲנִים יִשְׂאוּ שִׁבְעָה שׁוֹפְרוֹת יוֹבְלִים}\LTRfootnote{\space \foreignlanguage{hebrew}{יוֺבֵל} \textit{ram}} \foreignlanguage{hebrew}{לִפְנֵי אֲרוֹן יְהוָה} \hspace{0.3cm}
	3~~\foreignlanguage{hebrew}{וַיֹּאמֶר יְהוָה פִּתְאֹם}\LTRfootnote{\space \foreignlanguage{hebrew}{פִּתְאֹם} \textit{suddenly, surprisingly}} \foreignlanguage{hebrew}{אֶל־מֹשֶׁה וְאֶל־אַהֲרֹן וְאֶל־מִרְיָם צְאוּ שְׁלָשְׁתְּכֶם אֶל־אֹהֶל מוֹעֵד וַיֵּצְאוּ שְׁלָשְׁתָּם} \hspace{0.3cm}
	4~~\foreignlanguage{hebrew}{וַיֵּרֶד יְהוָה עַל־הַר סִינַי אֶל־רֹאשׁ הָהָר וַיִּקְרָא יְהוָה לְמֹשֶׁה אֶל־רֹאשׁ הָהָר וַיַּעַל מֹשֶׁה}  \hspace{0.3cm}
	5~~\foreignlanguage{hebrew}{וַיַּעַל לוֹט מִצּ֫וֹעַר וַיֵּשֶׁב בָּהָר וּשְׁתֵּי בְנֹתָיו עִמּוֹ כִּי יָרֵא לָשֶׁבֶת בְּצ֫וֹעַר וַיֵּשֶׁב בַּמְּעָרָה}\LTRfootnote{\space \foreignlanguage{hebrew}{מְעָרָה} \textit{cave}} \foreignlanguage{hebrew}{‎הוּא וּשְׁתֵּי בְנֹתָיו}  \hspace{0.3cm}
	6~~\foreignlanguage{hebrew}{וַיָּבֹא הָעָם בֵּית־אֵל וַיֵּשְׁבוּ שָׁם עַד־הָעֶרֶב לִפְנֵי הָאֱלֹהִים וַיִּשְׂאוּ קוֹלָם וַיִּבְכּוּ בְּכִי}\LTRfootnote{\space \foreignlanguage{hebrew}{בְּכִי} \textit{weeping} (noun)} \foreignlanguage{hebrew}{גָדוֹל}  \hspace{0.3cm}
	7~~\foreignlanguage{hebrew}{וַיֹּאמֶר אֵלָיו אַל־תִּירָא כִּי לֹא תִמְצָאֲךָ יַד שָׁאוּל אָבִי וְאַתָּה תִּמְלֹךְ עַל־יִשְׂרָאֵל וְאָנֹכִי אֶהְיֶה־לְּךָ לְמִשְׁנֶה}\LTRfootnote{\space \foreignlanguage{hebrew}{מִשְׁנֶה} here: \textit{second in command}} \foreignlanguage{hebrew}{וְגַם־שָׁאוּל אָבִי יֹדֵעַ כֵּן} \hspace{0.3cm}
	8~~\foreignlanguage{hebrew}{וַיִּשְׁלַח הַמֶּלֶךְ וַיִּקְרָא לְשִׁמְעִי וַיֹּאמֶר לוֹ בְּנֵה־לְךָ בַיִת בִּירוּשָׁלִַם}\LTRfootnote{\space \foreignlanguage{hebrew}{ירושלם} to be read \textit{yərūšāláyim}} \foreignlanguage{hebrew}{וְיָשַׁבְתָּ שָׁם וְלֹא־תֵצֵא מִשָּׁם אָ֫נֶה וָאָ֫נָה}\LTRfootnote{\space \foreignlanguage{hebrew}{אָ֫נֶה וָאָ֫נָה} \textit{hither and tither} (\textit{HALOT})} \hspace{0.3cm}
	9~~\foreignlanguage{hebrew}{וַיֹּאמְרוּ הָ֫בָה}\LTRfootnote{\space \foreignlanguage{hebrew}{הָ֫בָה} \textit{come! come on!} (interjection; actually impv. of \foreignlanguage{hebrew}{יהב}).} \foreignlanguage{hebrew}{נִבְנֶה־לָּנוּ עִיר וּמִגְדָּל וְרֹאשׁוֹ בַשָּׁמַיִם וְנַעֲשֶׂה־לָּנוּ שֵׁם ‎וַיֵּרֶד יְהוָה לִרְאֹת אֶת־הָעִיר וְאֶת־הַמִּגְדָּל אֲשֶׁר בָּנוּ בְּנֵי הָאָדָם} \hspace{0.3cm}
	10~~\foreignlanguage{hebrew}{וָאֶשְׁמַע אֶת־קוֹל אֲדֹנָי אֹמֵר אֶת־מִי אֶשְׁלַח וּמִי יֵלֶךְ־לָנוּ וָאֹמַר הִנְנִי שְׁלָחֵנִי}  \hspace{0.3cm}
	11~~\foreignlanguage{hebrew}{וַיֹּאמֶר אֲלֵהֶם הַאֵלֵךְ עַל־רָמֹת גִּלְעָד לַמִּלְחָמָה אִם־אֶחְדָּל}  \hspace{0.3cm}
	12~~ \foreignlanguage{hebrew}{וַיֵּדַע קַיִן אֶת־אִשְׁתּוֹ וַתַּהַר}\LTRfootnote{\space \foreignlanguage{hebrew}{הרה} \textit{to be pregnant, become pregnant}} \foreignlanguage{hebrew}{וַתֵּלֶד אֶת־חֲנוֹךְ וַיְהִי בֹּנֶה עִיר וַיִּקְרָא שֵׁם הָעִיר כְּשֵׁם בְּנוֹ חֲנוֹךְ}  \hspace{0.3cm}
	13~~\foreignlanguage{hebrew}{וְכֹל אֲשֶׁר־יִגַּע־בּוֹ הַטָּמֵא יִטְמָא וְהַנֶּפֶשׁ הַנֹּגַעַת תִּטְמָא עַד־הָעָרֶב}  \hspace{0.3cm}
	14~~\foreignlanguage{hebrew}{כִּי־אַתָּה יְהוָה צְבָאוֹת אֱלֹהֵי יִשְׂרָאֵל גָּלִיתָה אֶת־אֹזֶן עַבְדְּךָ לֵאמֹר בַּיִת אֶבְנֶה־לָּךְ עַל־כֵּן מָצָא עַבְדְּךָ אֶת־לִבּוֹ לְהִתְפַּלֵּל}\LTRfootnote{\space \foreignlanguage{hebrew}{לְהִתְפַּלֵּל} \textit{to pray}} \foreignlanguage{hebrew}{אֵלֶיךָ אֶת־הַתְּפִלָּה}\LTRfootnote{\space \foreignlanguage{hebrew}{תְּפִלָּה} \textit{prayer}} \foreignlanguage{hebrew}{הַזֹּאת} \hspace{0.3cm}
	15~~\foreignlanguage{hebrew}{וַיִּשְׁלַח מֹשֶׁה מַלְאָכִים מִקָּדֵשׁ אֶל־מֶלֶךְ אֱדוֹם כֹּה אָמַר אָחִיךָ יִשְׂרָאֵל אַתָּה יָדַעְתָּ אֵת כָּל־הַתְּלָאָה}\LTRfootnote{\space \foreignlanguage{hebrew}{הַתְּלָאָה} \textit{tribulation, hardship}} \foreignlanguage{hebrew}{אֲשֶׁר מְצָאָֽתְנוּ. וַיֵּרְדוּ אֲבֹתֵינוּ מִצְרַיְמָה וַנֵּשֶׁב בְּמִצְרַיִם יָמִים רַבִּים וַיָּרֵעוּ}\LTRfootnote{\space \foreignlanguage{hebrew}{וַיָּרֵעוּ} \textit{and they treated badly}} \foreignlanguage{hebrew}{לָנוּ מִצְרַיִם וְלַאֲבֹתֵֽינוּ. וַנִּצְעַק אֶל־יְהוָה וַיִּשְׁמַע קֹלֵנוּ וַיִּשְׁלַח מַלְאָךְ וַיֹּצִאֵנוּ}\LTRfootnote{\space \foreignlanguage{hebrew}{וַיֹּצִאֵנוּ} \textit{and he lead us out} (\foreignlanguage{hebrew}{יצא} Hiphil)} \foreignlanguage{hebrew}{מִמִּצְרָיִם וְהִנֵּה אֲנַחְנוּ בְקָדֵשׁ עִיר קְצֵה גְבוּלֶֽךָ. נַעְבְּרָה־נָּא בְאַרְצֶךָ לֹא נַעֲבֹר בְּשָׂדֶה וּבְכֶרֶם וְלֹא נִשְׁתֶּה מֵי בְאֵר. דֶּרֶךְ הַמֶּלֶךְ נֵלֵךְ. לֹא נִטֶּה יָמִין וּשְׂמֹאול עַד אֲשֶֽר־נַעֲבֹר גְּבוּלֶֽךָ. וַיֹּאמֶר אֵלָיו אֱדוֹם לֹא תַעֲבֹר בִּי פֶּן־}\LTRfootnote{\space \foreignlanguage{hebrew}{פֶּן} \textit{lest, so that not}} \foreignlanguage{hebrew}{בַּחֶרֶב אֵצֵא לִקְרָאתֶֽךָ.} \hspace{0.3cm}
	16~~\foreignlanguage{hebrew}{וַיִּחַר־אַף יְהוָה בְּיִשְׂרָאֵל וַיִּתְּנֵם בְּיַד־שֹׁסִים}\LTRfootnote{\space \foreignlanguage{hebrew}{שׁסה} \textit{to plunder}} \hspace{0.3cm}
	17~~\foreignlanguage{hebrew}{וְהַמֶּלֶךְ דָּוִד שָׁמַע אֵת כָּל־הַדְּבָרִים הָאֵלֶּה וַיִּחַר לוֹ מְאֹד}  \hspace{0.3cm}
	18~~\foreignlanguage{hebrew}{וַיֹּאמְרוּ אַנְשֵׁי בֵית־שֶׁמֶשׁ מִי יוּכַל לַעֲמֹד לִפְנֵי יְהוָה הָאֱלֹהִים הַקָּדוֹשׁ הַזֶּה}  \hspace{0.3cm}
	
	\selectlanguage{english}
	
	
	
	\chapter{Chapter 14}
	
	\renewcommand\arraystretch{1.4}
	
	\section{Vocabulary}
	
	\subsection{Verbs}
	
	\begin{center}
		
		% For the centering of the separation between the two columns see the documentation of the array package, page 2 
		
		\begin{tabular}{>{\raggedleft}p{0.175\linewidth} p{0.75\linewidth}}
			\foreignlanguage{hebrew}{בוא} & Q.\ \textit{to come, to enter} \\
			\foreignlanguage{hebrew}{בושׁ} & Q.\ \textit{to be ashamed} \\
			\foreignlanguage{hebrew}{בין} & Q.\ \textit{to discern, understand} \\
			\foreignlanguage{hebrew}{גור} & Q.\ \textit{to sojourn, dwell as alien and dependent} \\ % BDB + HALOT
			\foreignlanguage{hebrew}{מאס} & Q.\ \textit{to reject, despise} \\
			\foreignlanguage{hebrew}{מות} & Q.\ \textit{to die} (SC 3 m.\ sg.\ \foreignlanguage{hebrew}{מֵת}; PC 3 m.\ sg.\ \foreignlanguage{hebrew}{יָמוּת}; part. m.\ sg.\ \foreignlanguage{hebrew}{מֵת}) \\
			\foreignlanguage{hebrew}{נוס} & Q.\ \textit{to flee, escape} \\
			\foreignlanguage{hebrew}{סור} & Q.\ \textit{to turn aside} \\
			\foreignlanguage{hebrew}{קום} & Q.\ \textit{to arise, stand up} \\
			\foreignlanguage{hebrew}{רוץ} & Q.\ \textit{to run} \\
			\foreignlanguage{hebrew}{שׂים} & Q.\ \textit{to put, to make into, to give} \\
			\foreignlanguage{hebrew}{שׁוב} & Q.\ \textit{to turn back, return} \\
			\foreignlanguage{hebrew}{שׁיר} & Q.\ \textit{to sing} (PC \foreignlanguage{hebrew}{יָשִׁיר}) \\
			\foreignlanguage{hebrew}{שׁית} & Q.\ \textit{to set, put} (PC \foreignlanguage{hebrew}{יָשִׁית}) \\
		\end{tabular}
	\end{center}
	
	\subsection{Nouns}
	
	\begin{center}
		\begin{longtable}{>{\raggedleft}p{0.175\linewidth} p{0.75\linewidth}}
			\foreignlanguage{hebrew}{אֲדָמָה} & \textit{ground, land} \\
			\foreignlanguage{hebrew}{אַיִל} & \textit{ram} \\
			\foreignlanguage{hebrew}{בְּהֵמָה} & \textit{beast, animal, cattle} \\
			\foreignlanguage{hebrew}{גֵּר} & \textit{sojourner} \\
			\foreignlanguage{hebrew}{חָצֵר} & \textit{court, enclosure} \\ % BDB
			\foreignlanguage{hebrew}{מִנְחָה} & \textit{gift, tribute, offering} \\
			\foreignlanguage{hebrew}{מִסְפָּר} & \textit{number} \\ % once also "tale" (Judg 7.15)
			\foreignlanguage{hebrew}{פָּרָשׁ} & \textit{horseman, charioteer; team of horses, horses for a chariot} (pl.\ \foreignlanguage{hebrew}{פָּרָשִׁים}) \\ % HALOT
			\foreignlanguage{hebrew}{קֶרֶב} & \textit{inward part, midst} (with ePP \foreignlanguage{hebrew}{קִרְבּוֺ}) \\
			\foreignlanguage{hebrew}{רֵעַ} & \textit{friend, companion, fellow} \\
			\foreignlanguage{hebrew}{רָשָׁע} & \textit{wicked, guilty} \\
			\foreignlanguage{hebrew}{שֶׁמֶן} & \textit{oil, fat, fatness} \\
			\foreignlanguage{hebrew}{שֹׁפֵט} & \textit{judge, ruler, governor} (part.\ act.\ Qal) \\
		\end{longtable}
	\end{center}
	
	\subsection{Other Parts of Speech}
	
	\begin{center}
		\begin{tabular}{>{\raggedleft}p{0.175\linewidth} p{0.75\linewidth}}
			\foreignlanguage{hebrew}{סָבִיב} & \textit{around, round about} (adv. and prep., as a noun \textit{surroundings}) \\
			\foreignlanguage{hebrew}{סְבִיבוֺת} & \textit{around, round about} (prep.; pl.\ of \foreignlanguage{hebrew}{סָבִיב}, with ePP \foreignlanguage{hebrew}{סְבִיבֹתֵיהֶם}) \\
		\end{tabular}
	\end{center}
	
	\section{Weak Verbs: II \textit{w/y} Verbs}
	
	The verbs of the type II\,\textit{w/y} have roots with only two consonants. Another term for II\,\textit{w/y} verbs is \textit{hollow roots.}
	
	II\,\textit{w/y} verbs are characterized by 
	
	\begin{itemize}[noitemsep]
		\item[--] a vocalic element between the two radicals in almost all forms
		\item[--] vocalic suffixes on SC, PC and imperative are \emph{not} stressed
		\item[--] open prefix syllables in the prefix conjugation Qal
	\end{itemize}
	
	Unlike other verbs, II\,\textit{w/y} verbs are listed in the lexicon with the infinitive construct (without vowel pointing; except BDB); e.g., \foreignlanguage{hebrew}{קום}, \foreignlanguage{hebrew}{שׂים}, \foreignlanguage{hebrew}{בוא}, \foreignlanguage{hebrew}{מות}, \foreignlanguage{hebrew}{בושׁ}.
	
	% The following table includes all possible forms of the verb קום; not attested are at least the 3/2 fem. pl. PC forms.
	% Of the other verbs only actually attested forms are included.
	
	\begin{center}
		\begin{longtable}{|lll|rrr|rr|}
			\hline
			& & & \multicolumn{3}{c|}{Fientive Verbs} & \multicolumn{2}{c|}{Stative Verbs} \\
			\hline
			& & & \multicolumn{1}{c}{\foreignlanguage{hebrew}{קום}} & \multicolumn{1}{c}{\foreignlanguage{hebrew}{שׂים}} & \multicolumn{1}{c|}{\foreignlanguage{hebrew}{בוא}} & \multicolumn{1}{c}{\foreignlanguage{hebrew}{מות}} & \multicolumn{1}{c|}{\foreignlanguage{hebrew}{בושׁ}} \\
			\hline
			SC & sg. & 3 m. & \foreignlanguage{hebrew}{קָם} & \foreignlanguage{hebrew}{שָׂם} & \foreignlanguage{hebrew}{בָּא} & \foreignlanguage{hebrew}{מֵת} & \foreignlanguage{hebrew}{בּוֺשׁ} \\
			& & 3 f. & \foreignlanguage{hebrew}{קָ֫מָה} & \foreignlanguage{hebrew}{שָׂ֫מָה} & \foreignlanguage{hebrew}{בָּ֫אָה} & \foreignlanguage{hebrew}{מֵ֫תָה} & \foreignlanguage{hebrew}{בּ֫וֺשָׁה} \\
			& & 2 m. & \foreignlanguage{hebrew}{קַ֫מְתָּ} & \foreignlanguage{hebrew}{שַׂ֫מְתָּ} & \foreignlanguage{hebrew}{בָּ֫אתָ} &  & \\
			& & 2 f. & \foreignlanguage{hebrew}{קַמְתְּ} & \foreignlanguage{hebrew}{שַׂמְתְּ} & \foreignlanguage{hebrew}{בָּאת} &  & \foreignlanguage{hebrew}{בֹּשְׁתְּ} \\
			& & 1 c. & \foreignlanguage{hebrew}{קַ֫מְתִּי} & \foreignlanguage{hebrew}{שַׂ֫מְתִּי} & \foreignlanguage{hebrew}{בָּ֫אתִי} &  & \foreignlanguage{hebrew}{בֹּ֫שְׁתִּי} \\
			& pl. & 3 c. & \foreignlanguage{hebrew}{קָ֫מוּ} & \foreignlanguage{hebrew}{שָׂ֫מוּ} & \foreignlanguage{hebrew}{בָּ֫אוּ} & \foreignlanguage{hebrew}{מֵ֫תוּ} & \foreignlanguage{hebrew}{בֹּ֫שׁוּ} \\
			& & 2 m. & \foreignlanguage{hebrew}{קַמְתֶּם} & \foreignlanguage{hebrew}{שַׂמְתֶּם} & \foreignlanguage{hebrew}{בָּאתֶם} & & \\
			& & 2 f. & \foreignlanguage{hebrew}{קַמְתֶּן} & & & & \\
			& & 1 c. & \foreignlanguage{hebrew}{קַ֫מְנוּ} & \foreignlanguage{hebrew}{שַׂ֫מְנוּ} & \foreignlanguage{hebrew}{בָּ֫אנוּ} & \foreignlanguage{hebrew}{מַ֫תְנוּ} & \foreignlanguage{hebrew}{בֹּ֫שְׁנוּ} \\
			\hline
			\pagebreak
			\hline
			PC & sg. & 3 m. & \foreignlanguage{hebrew}{יָקוּם} & \foreignlanguage{hebrew}{יָשִׂים} & \foreignlanguage{hebrew}{יָבֹא} & \foreignlanguage{hebrew}{יָמוּת} & \foreignlanguage{hebrew}{יֵבוֺשׁ} \\
			& & 3 f. & \foreignlanguage{hebrew}{תָּקוּם} & \foreignlanguage{hebrew}{תָּשִׂים} & \foreignlanguage{hebrew}{תָּבֹא} & \foreignlanguage{hebrew}{תָּמוּת} & \foreignlanguage{hebrew}{תֵּבוֺשׁ} \\
			& & 2 m. & \foreignlanguage{hebrew}{תָּקוּם} & \foreignlanguage{hebrew}{תָּשִׂים} & \foreignlanguage{hebrew}{תָּבֹא} & \foreignlanguage{hebrew}{תָּמוּת} & \\
			& & 2 f. & \foreignlanguage{hebrew}{תָּק֫וּמִי} & \foreignlanguage{hebrew}{תָּשִׂ֫ימִי} & \foreignlanguage{hebrew}{תָּב֫וֹאִי} & \foreignlanguage{hebrew}{תָּמ֫וּתִי} & \foreignlanguage{hebrew}{תֵּב֫וֹשִׁי} \\
			& & 1 c. & \foreignlanguage{hebrew}{אָקוּם} & \foreignlanguage{hebrew}{אָשִׂים} & \foreignlanguage{hebrew}{אָבֹוא} & \foreignlanguage{hebrew}{אָמוּת} & \foreignlanguage{hebrew}{אֵבוֹשׁ} \\
			& pl. & 3 m. & \foreignlanguage{hebrew}{יָק֫וּמוּ} & \foreignlanguage{hebrew}{יָשִׂ֫ימוּ} & \foreignlanguage{hebrew}{יָבֹ֫אוּ} & \foreignlanguage{hebrew}{יָמ֫וּתוּ} & \foreignlanguage{hebrew}{יֵבֹ֫שׁוּ} \\
			& & 3 f. & \foreignlanguage{hebrew}{תָּקֹ֫מְנָה} & & \foreignlanguage{hebrew}{תָּבֹ֫אנָה} & \foreignlanguage{hebrew}{תְּמוּתֶ֫נָה} & \foreignlanguage{hebrew}{} \\
			& & 2 m. & \foreignlanguage{hebrew}{תָּק֫וּמוּ} & \foreignlanguage{hebrew}{תָּשִׂ֫ימוּ} & \foreignlanguage{hebrew}{תָּבֹ֫אוּ} & \foreignlanguage{hebrew}{תָּמ֫וּתוּ} & \foreignlanguage{hebrew}{תֵּבֹ֫שׁוּ} \\
			& & 2 f. & \foreignlanguage{hebrew}{תָּקֹ֫מְנָה} & & & & \\
			& & 1 c. & \foreignlanguage{hebrew}{נָקוּם} & \foreignlanguage{hebrew}{נָשִׂים} & \foreignlanguage{hebrew}{נָבוֹא} & \foreignlanguage{hebrew}{נָמוּת} & \foreignlanguage{hebrew}{נֵבוֹשׁ} \\
			\hline
			Jussive & sg. & 3 m. & \foreignlanguage{hebrew}{יָקֹם} & \foreignlanguage{hebrew}{יָשֵׂם} & \foreignlanguage{hebrew}{יָבֹא} & \foreignlanguage{hebrew}{יָמֹת} & \\
			\textit{wayyiqṭol} & sg. & 3 m. & \foreignlanguage{hebrew}{וַיָּ֫קָם} & \foreignlanguage{hebrew}{וַיָּ֫שֶׂם} & \foreignlanguage{hebrew}{וַיָּבֹא} & \foreignlanguage{hebrew}{וַיָּ֫מָת} & \\
			\hline
			Impv. & sg. & m. & \foreignlanguage{hebrew}{קוּם} & \foreignlanguage{hebrew}{שִׂים} & \foreignlanguage{hebrew}{בֹּא} & \foreignlanguage{hebrew}{מֻת} & \\
			& & f. & \foreignlanguage{hebrew}{ק֫וּמִי} & \foreignlanguage{hebrew}{שִׂ֫ימִי} & \foreignlanguage{hebrew}{בֹּ֫אִי} &  & \foreignlanguage{hebrew}{בּ֫וֺשִׁי} \\
			& pl. & m. & \foreignlanguage{hebrew}{ק֫וּמוּ} & \foreignlanguage{hebrew}{שִׂ֫ימוּ} & \foreignlanguage{hebrew}{בֹּ֫אוּ} & & \foreignlanguage{hebrew}{בּ֫וֺשׁוּ} \\
			& & f. & \foreignlanguage{hebrew}{קֹ֫מְנָה} & & & & \\
			\hline
			Inf.\,cs. & & & \foreignlanguage{hebrew}{קוּם} & \foreignlanguage{hebrew}{שׂוּם} & \foreignlanguage{hebrew}{בֹּא} & \foreignlanguage{hebrew}{מוּת} & \foreignlanguage{hebrew}{בּוֺשׁ} \\
			Inf.\,abs. & & & \foreignlanguage{hebrew}{קוֺם} & \foreignlanguage{hebrew}{שׂוֺם} & \foreignlanguage{hebrew}{בֹּא} & \foreignlanguage{hebrew}{מוֺת} & \foreignlanguage{hebrew}{בּוֺשׁ} \\
			\hline
			Part. act. & sg. & m. & \foreignlanguage{hebrew}{קָם} & \foreignlanguage{hebrew}{שָׂם} & \foreignlanguage{hebrew}{בָּא} & \foreignlanguage{hebrew}{מֵת} & \foreignlanguage{hebrew}{בוֺשׁ} \\
			Part. pass. & sg. & m. & \foreignlanguage{hebrew}{מוּל} & \foreignlanguage{hebrew}{שִׂים} & --- & --- & --- \\
			\hline
		\end{longtable}
	\end{center}
	
	
	\noindent \textbf{Notes}
	\nopagebreak
	
	\noindent The very frequent verb \foreignlanguage{hebrew}{בוא} is doubly weak with elision of the /ʾ/ at the end of a syllable (cf.\ the III\,ʾ verbs).
	
	In the suffix conjugation, three types of II\,\textit{w/y} verbs can be distinguished:
	
	\begin{itemize}[noitemsep]
		\item[--] verbs with \textit{ā} or \textit{a} as stem vowel (e.g., \foreignlanguage{hebrew}{קום} \textit{to arise, stand up}, \foreignlanguage{hebrew}{שׂים} \textit{to put}, \foreignlanguage{hebrew}{בוא} \textit{to come} and most other II\,\textit{w/y})
		\item[--] verbs with \textit{ē} (< *\textit{i}) as stem vowel (in Biblical Hebrew only the verb \foreignlanguage{hebrew}{מות} \textit{to die})
		\item[--] verbs with \textit{ō} as stem vowel (only a few stative verbs, e.g. \foreignlanguage{hebrew}{בושׁ} \textit{to be ashamed}, \foreignlanguage{hebrew}{טוב} \textit{to be good})
	\end{itemize}
	
	In the prefix conjugation, there are three different types:
	
	\begin{itemize}[noitemsep]
		\item[--] verbs with the vowel pattern \textit{a--u} based on the primitive forms \textit{yaqṭulu} with the prefix vowel /a/ and the thematic vowel /u/: \textit{*yaqūm > yāqūm} \foreignlanguage{hebrew}{יָקוּם} (the majority of II\,\textit{wy} verbs including the verb \foreignlanguage{hebrew}{מות})
		\item[--] verbs with the vowel pattern\textit{ a--i}: \textit{*yaśīm > yāśīm} from \foreignlanguage{hebrew}{יָשִׂים} (cf.\ the verbs \foreignlanguage{hebrew}{ישׁב} and \foreignlanguage{hebrew}{נתן} with the same vowel pattern in the prefix conjugation)
		\item[--] verbs with the vowel pattern \textit{i--a} for stative verbs with the change of the thematic vowel \textit{ā > ō}: \textit{*yibāš > yēbōš} \foreignlanguage{hebrew}{יֵבוֹשׁ} \textit{he will be ashamed}
	\end{itemize}
	
	The original prefix vowels have been changed only slightly within the parameters of Biblical Hebrew (\textit{a > ā} and \textit{i > ē}, respectively).
	
	In 3/2 f.\ pl.\ forms there are forms without and with connecting vowel, e.g., \foreignlanguage{hebrew}{תָּשֹׁ֫בְנָה} without connecting vowel, \foreignlanguage{hebrew}{תְּשֻׁבֶ֫ינָה} and \foreignlanguage{hebrew}{תְּבֹאֶ֫ינָה} with connecting vowel and reduction of the prefix vowel).
	
	II\,\textit{w/y} verbs have distinct jussive forms in forms without vocalic suffix, e.g., \foreignlanguage{hebrew}{יָקֹם} (but also \foreignlanguage{hebrew}{יָקֻם} or \foreignlanguage{hebrew}{וְיָ֫קָם}),  \foreignlanguage{hebrew}{יָשֹׁב}, \foreignlanguage{hebrew}{יָשֵׂם}.
	
	\textit{wayyiqṭol} forms of II\,w/y verbs without vocalic suffix are based on the corresponding jussive forms. In context forms, the stress is retracted to the penultimate syllable and the thematic vowel is shortened, e.g., \foreignlanguage{hebrew}{וַיָּ֫קָם} \textit{wayyā́qɔm},  \foreignlanguage{hebrew}{וַתָּ֫שָׁב} \textit{wattā́šɔḇ}, \foreignlanguage{hebrew}{וַיָּ֫שֶׂם} \textit{wayyā́śæm}. In pausal forms, however, the stress is on the last syllable, e.g., \foreignlanguage{hebrew}{וַיָּשֹׁ֑ב} \textit{wayyāšōḇ}. Retraction of stress does not happen in 1 c.\ sg.\ forms, e.g., \foreignlanguage{hebrew}{וָאָקוּם} and \foreignlanguage{hebrew}{וָאָשִׂים}.
	
	The 3 m.\ sg.\ \textit{wayyiqṭol} form of the verb \foreignlanguage{hebrew}{סור} is \foreignlanguage{hebrew}{וַיָּ֫סַר} with \textit{pataḥ} because of the final consonant \foreignlanguage{hebrew}{ר}.
	
	The impv. m.\ sg.\ may be expanded by the ending \textit{-ā}, e.g., \foreignlanguage{hebrew}{ק֫וּמָה} \textit{arise!} (Num 10:35)
	
	Forms of the verb \foreignlanguage{hebrew}{בוא} are frequently spelled defectively. The 3/2 m.\ pl.\ forms with the ending \textit{-ū} of the verbs \foreignlanguage{hebrew}{בוא} and \foreignlanguage{hebrew}{בושׁ} are usually spelled defectively, e.g., \foreignlanguage{hebrew}{יֵבֹ֫שׁוּ}, \foreignlanguage{hebrew}{וַיָּבֹ֫אוּ} to avoid the use of two vowel letters. Defective spelling of 3/2 m.\ pl.\ forms of the verbs \foreignlanguage{hebrew}{קום} and \foreignlanguage{hebrew}{שׁוב} is also common, e.g., \foreignlanguage{hebrew}{וַיָּשֻׁ֫בוּ}.
	
	The inf.\,cs. of the verb \foreignlanguage{hebrew}{שִׂים} is \foreignlanguage{hebrew}{שׂוּם} in most cases, but the form \foreignlanguage{hebrew}{שִׂים} is attested as well.
	
	\section{Numerals (II): Numbers above 10}
	
	% This section relies heavily on Lettinga/von Siebenthal, especially the tables.
	
	\subsection{Numbers 11--19}
	The numbers 11--19 consist of the words for the units in the absolute state with masculine nouns and of the words for the units in the cs.\ st.\ with fem.\ nouns, respectively, and special forms of the word for ten \foreignlanguage{hebrew}{עֶשֶׂר}. Gender inversion occurs for the numbers 13--19.
	
	\begin{center}
		\begin{longtable}{lrr}
			& \multicolumn{1}{c}{With masc.\ nouns} & \multicolumn{1}{c}{With fem.\ nouns} \\
			\endhead
			\endfoot
			11 & \foreignlanguage{hebrew}{עַשְׁתֵּי עָשָׂר} / \foreignlanguage{hebrew}{אַחַד עָשָׂר} & \foreignlanguage{hebrew}{עַשְׁתֵּי עֶשְׂרֵה} / \foreignlanguage{hebrew}{אַחַת עֶשְׂרֵה} \\
			12 &  \foreignlanguage{hebrew}{שְׁנֵי עָשָׂר} / \foreignlanguage{hebrew}{שְׁנֵים עָשָׂר} &  \foreignlanguage{hebrew}{שְׁתֵּי עֶשְׂרֵה } /   \foreignlanguage{hebrew}{שְׁתֵּים עֶשְׂרֵה} \\
			13 &  \foreignlanguage{hebrew}{שְׂלֹשָׁה עָשָׂר} &  \foreignlanguage{hebrew}{שְׁלֹשׁ עֶשְׂרֵה} \\
			14 &  \foreignlanguage{hebrew}{אַרְבָּעָה עָשָׂר} &  \foreignlanguage{hebrew}{אַרְבַּע עֶשְׂרֵה} \\
			15 &  \foreignlanguage{hebrew}{חֲמִשָּׁה עָשָׂר} &  \foreignlanguage{hebrew}{חֲמֵשׁ עֶשְׂרֵה} \\
			16 &  \foreignlanguage{hebrew}{שִׁשָּׁה עָשָׂר} &  \foreignlanguage{hebrew}{שֵׁשׁ עֶשְׂרֵה} \\
			17 &  \foreignlanguage{hebrew}{שִׁבְעָה עָשָׂר} &  \foreignlanguage{hebrew}{שְׁבַע עֶשְׂרֵה} \\
			18 &  \foreignlanguage{hebrew}{שְׁמֹנָה עָשָׂר} &  \foreignlanguage{hebrew}{שְׁמֹנֶה עֶשְׂרֵה} \\
			19 &  \foreignlanguage{hebrew}{תִּשְׁעָה עָשָׂר} &  \foreignlanguage{hebrew}{תְּשַׁע עֶשְׂרֵה} \\
		\end{longtable}
	\end{center}
	
	\noindent \textbf{Note}
	\nopagebreak
	
	\noindent Rarely, the cs.\ st.\ form is used for the units with masc.\ nouns, e.g., \foreignlanguage{hebrew}{חֲמֵשֶׁת עָשָׂר} \textit{fifteen}.
	
	\subsection{Numbers 20--90}
	The numeral for \textit{twenty} is the plural form of \foreignlanguage{hebrew}{עֶשֶׂר}. The numerals for \textit{thirty}, \textit{forty} \dots \space \textit{ninety} are the plural forms of the respective words for the units.
	
	\begin{center}
		\begin{tabular}{lr}
			20 & \foreignlanguage{hebrew}{עֶשְׂרִים} \\
			30 & \foreignlanguage{hebrew}{שְׂלֹשִׁים} \\
			40 & \foreignlanguage{hebrew}{אַרְבָּעִים} \\
			50 & \foreignlanguage{hebrew}{חֲמִשִּׁים} \\
			60 & \foreignlanguage{hebrew}{שִׁשִּׁים} \\
			70 & \foreignlanguage{hebrew}{שִׁבְעִים} \\
			80 & \foreignlanguage{hebrew}{שְׁמֹנִים} \\
			90 & \foreignlanguage{hebrew}{תִּשְׁעִים} \\
		\end{tabular}
	\end{center}
	
	\subsection{Numerals for 100 and above}
	
	The numeral for \textit{one hundred} is the fem.\ noun \foreignlanguage{hebrew}{מֵאָה}, the numeral for \textit{one thousand                                                                                                                                        } is the masc.\ noun \foreignlanguage{hebrew}{אֶלֶף}. Both numerals are used in the dual for \textit{two hundred} and \textit{two thousand}, respectively.
	
	\begin{Center}
		\begin{longtable}{rl}
			\foreignlanguage{hebrew}{מֵאָה} & 100 \\
			\foreignlanguage{hebrew}{מָאתַיִם} & 200 \\
			\foreignlanguage{hebrew}{שְׁלֹשׁ מֵאוֺת ... תְּשַׁע מֵאוֺת} & 300 \dots \space 900 \\
			\foreignlanguage{hebrew}{אֶ֫לֶף} & 1,000 \\
			\foreignlanguage{hebrew}{אַלְפַּיִם} & 2,000 \\
			\foreignlanguage{hebrew}{שְׁלֹשֶׁת אַלָפִים ... תִּשְׁעַת אֲלָפִים} & 3,000 \dots \space 9,000 \\
			\foreignlanguage{hebrew}{רִבּוֺ} / \foreignlanguage{hebrew}{רִבֹּוא} or \foreignlanguage{hebrew}{רְבָבָה} or \foreignlanguage{hebrew}{עֲשֶׂרֶת אֲלָפִים} & 10,000 \\
			\foreignlanguage{hebrew}{שְׁתֵּי רִבּוֺא} / \foreignlanguage{hebrew}{שְׁתֶּי רִבּוֺת} or \foreignlanguage{hebrew}{עֶשְׂרִים אֶלֶף} & 20,000 \\
			\foreignlanguage{hebrew}{שְׁלֹשִׁים אֲלָפִים} & 30,000 \\
			\foreignlanguage{hebrew}{מְאַת אֶלֶף} / \foreignlanguage{hebrew}{מֵאָה אֶלֶף} & 100,000 \\
			\foreignlanguage{hebrew}{מָאתַיִם אֶלֶף וגו׳} & 200,000 etc. \\
			\foreignlanguage{hebrew}{אֶלֶף אֲלָפִים} & 1,000,000 \\
		\end{longtable}
	\end{Center}
	
	
	\subsection{Notes of the Use of Numerals above Ten}
	
	% Summarized from Lettinga/von Siebenthal, §40 (343-363)
	
	\begin{itemize}[noitemsep]
		\item[--] Usually the numeral precedes the counted noun, e.g., \foreignlanguage{hebrew}{שְׁתֵּי־עֶשְׂרֵה אֲבָנִים} \textit{twelve stones} (Josh 4:8). % Example from LvS §40k (353)
		\item[--] The counted noun is in the plural except for nouns that are used with numerals frequently, e.g., \foreignlanguage{hebrew}{יוֺם} \textit{day}, \foreignlanguage{hebrew}{שָׁנָה} \textit{year}, \foreignlanguage{hebrew}{אִישׁ} \textit{man}, \foreignlanguage{hebrew}{נֶפֶשׁ} \textit{soul, person}, \foreignlanguage{hebrew}{שֵׁבֶט} \textit{tribe}, but also with \foreignlanguage{hebrew}{אַמָּה} \textit{cubit}, \foreignlanguage{hebrew}{שֶׁקֶל} \textit{sheqel}, e.g., \foreignlanguage{hebrew}{אַרְבָּעִים יוֹם וְאַרְבָּעִים לָיְלָה} \textit{forty days and forty nights} (Gen 7:12).
		\item[--] Units and tens are connected with the conjunction \foreignlanguage{hebrew}{וְ}, e.g., \foreignlanguage{hebrew}{עֶשְׂרִים וּשְׁתַּיִם שָׁנָה} \textit{twenty-two years} (1\,Kgs 16:29).
		\item[--] When units, tens and hundreds are used together, the higher numerals usually come first, e.g., \foreignlanguage{hebrew}{חֲמִשָּׁה וְאַרְבָּעִים אֶלֶף וְשֵׁשׁ מֵאוֹת וַחֲמִשִּׁים} \textit{forty-five thousand, six hundred and fifty} (Num 1:25). Deviating sequences of the numerals, however, are attested as well, e.g., \foreignlanguage{hebrew}{שְׁתַּיִם וּשְׁמוֹנִים שָׁנָה וּשְׁבַע מֵאוֹת שָׁנָה} \textit{seven hundred and eighty-two years} (Gen 5:26), \foreignlanguage{hebrew}{אֶלֶף וּשְׁבַע מֵאוֹת וַחֲמִשָּׁה וְשִׁבְעִים שֶׁקֶל} \textit{one thousand and seven hundred and seventy-five sheqel} (Exod 38:25).\footnote{\space In the genealogies in Gen 5 and Gen 10 the smaller numerals always come before the higher numerals.}
		\item[--] For ordinal numbers above \textit{tenth} cardinal numbers are used.
		\item[--] For indication of days and years, cardinal numbers may be used instead of ordinal numbers for \textit{second} to \textit{tenth}.
	\end{itemize}
	
	
	\section{Exercises}
	
	\subsection{Translation of Verbal Forms}
	Translate the following verbal forms. Identify the gender (masc., fem., comm.) and number (sg., pl.) of forms of which the English translation is ambiguous (i.e., \textit{you}, \textit{they}). Mark the stressed syllable if stress is not on the last syllable.
	
	\hspace{0.5cm}
	
	\selectlanguage{hebrew}
	
	\noindent
	1~~\foreignlanguage{hebrew}{יֵבֹשׁוּ}  \hspace{0.3cm}
	2~~\foreignlanguage{hebrew}{בֹּשְׁנוּ}  \hspace{0.3cm}
	3~~\foreignlanguage{hebrew}{וַיָּבִינוּ}  \hspace{0.3cm}
	4~~\foreignlanguage{hebrew}{בֹּאִי}  \hspace{0.3cm}
	5~~\foreignlanguage{hebrew}{גַּרְתִּי}  \hspace{0.3cm}
	6~~\foreignlanguage{hebrew}{גָּר}  \hspace{0.3cm}
	7~~\foreignlanguage{hebrew}{תָּמוּת}  \hspace{0.3cm}
	8~~\foreignlanguage{hebrew}{וַיָּנֻסוּ}  \hspace{0.3cm}
	9~~\foreignlanguage{hebrew}{סוּרָה}  \hspace{0.3cm}
	10~~\foreignlanguage{hebrew}{סוּר}  \hspace{0.3cm}
	11~~\foreignlanguage{hebrew}{קָמוּ}  \hspace{0.3cm}
	12~~\foreignlanguage{hebrew}{וַיָּקָם}  \hspace{0.3cm}
	13~~\foreignlanguage{hebrew}{רָצִים}  \hspace{0.3cm}
	14~~\foreignlanguage{hebrew}{רַצְתָּה}  \hspace{0.3cm}
	15~~\foreignlanguage{hebrew}{שִׂימוּ}  \hspace{0.3cm}
	16~~\foreignlanguage{hebrew}{שַׂמְתֶּם}  \hspace{0.3cm}
	17~~\foreignlanguage{hebrew}{וַיָּשָׁב}  \hspace{0.3cm}
	18~~\foreignlanguage{hebrew}{וָאָשׁוּב}  \hspace{0.3cm}
	
	\selectlanguage{english}
	
	\subsection{Numerals}
	Read \href{https://www.stepbible.org/?q=version=WLC@reference=Ezra.2&options=VNUGH}{Ezra 2:2b--17} and record the number of the descendants of each person.
	
	\subsection{Translation of Sentences}
	Translate the following sentences from the Hebrew Bible. Names of persons and geographical names in these sentences: \foreignlanguage{hebrew}{אַבְנֵר}, \foreignlanguage{hebrew}{אֱדוֹם}, \foreignlanguage{hebrew}{אוּרִיָּה}, \foreignlanguage{hebrew}{אָחָז}, \foreignlanguage{hebrew}{אֵלִיָּהוּ}, \foreignlanguage{hebrew}{אֲחִיטוּב}, \foreignlanguage{hebrew}{אֲחִימֶלֶךְ}, \foreignlanguage{hebrew}{אֱלִימֶלֶךְ}, \foreignlanguage{hebrew}{בֵּית לֶחֶם}, \foreignlanguage{hebrew}{דָּוִד}, \foreignlanguage{hebrew}{דַּמֶּשֶׂק}, \foreignlanguage{hebrew}{יְהוּדָה}, \foreignlanguage{hebrew}{יְהוֹשֻׁעַ}, \foreignlanguage{hebrew}{כִּלְיוֺן}, \foreignlanguage{hebrew}{מוֹאָב}, \foreignlanguage{hebrew}{מַחְלוֹן}, \foreignlanguage{hebrew}{מֹשֶׁה}, \foreignlanguage{hebrew}{נָעֳמִי}, \foreignlanguage{hebrew}{נֵר}, \foreignlanguage{hebrew}{עָרְפָּה}, \foreignlanguage{hebrew}{רוּת}, \foreignlanguage{hebrew}{שָׁאוּל}, \foreignlanguage{hebrew}{שְׁמוּאֵל}.
	
	\vspace{0.5cm}
	
	\selectlanguage{hebrew}
	\noindent
	1~~\foreignlanguage{hebrew}{וַיִּשְׁלַח הַמֶּלֶךְ לִקְרֹא אֶת־אֲחִימֶלֶךְ בֶּן־אֲחִיטוּב הַכֹּהֵן וְאֵת כָּל־בֵּית אָבִיו הַכֹּהֲנִים אֲשֶׁר בְּנֹב וַיָּבֹאוּ כֻלָּם אֶל־הַמֶּלֶךְ}  \hspace{0.3cm}
	2~~\foreignlanguage{hebrew}{וַיַּרְא דָּוִד כִּי עֲבָדָיו מִתְלַחֲשִׁים}\LTRfootnote{\space \foreignlanguage{hebrew}{מִתְלַחֲשִׁים} \textit{were whispering to another} (part.\ m.\ pl.\ Hitpa. \foreignlanguage{hebrew}{לחשׁ})} \foreignlanguage{hebrew}{וַיָּבֶן דָּוִד כִּי מֵת הַיָּלֶד וַיֹּאמֶר דָּוִד אֶל־עֲבָדָיו הֲמֵת הַיֶּלֶד וַיֹּאמְרוּ מֵת} \hspace{0.3cm}
	3~~\foreignlanguage{hebrew}{וַיִּבֶן אוּרִיָּה הַכֹּהֵן אֶת־הַמִּזְבֵּחַ כְּכֹל אֲשֶׁר־שָׁלַח הַמֶּלֶךְ אָחָז מִדַּמֶּשֶׂק כֵּן עָשָׂה אוּרִיָּה הַכֹּהֵן עַד־בּוֹא הַמֶּלֶךְ־אָחָז מִדַּמָּ֑שֶׂק}\LTRfootnote{\space \foreignlanguage{hebrew}{מִדַּמָּ֑שֶׂק} pausal form of \foreignlanguage{hebrew}{דַּמֶּשֶׂק}}  \hspace{0.3cm}
	4~~\foreignlanguage{hebrew}{וַיָּקָם דָּוִד וַיָּבֹא אֶל־הַמָּקוֹם אֲשֶׁר חָנָה־שָׁם שָׁאוּל וַיַּרְא דָּוִד אֶת־הַמָּקוֹם אֲשֶׁר שָׁכַב־שָׁם שָׁאוּל וְאַבְנֵר בֶּן־נֵר שַׂר־צְבָאוֹ וְשָׁאוּל שֹׁכֵב בַּמַּעְגָּל}\LTRfootnote{\space \foreignlanguage{hebrew}{מַעְגָּל} \textit{ring of wagons}} \foreignlanguage{hebrew}{וְהָעָם חֹנִים סְבִיבֹתָיו} \hspace{0.3cm}
	5~~\foreignlanguage{hebrew}{וַיִּשְׁמַע יְהוָה בְּקוֹל אֵלִיָּהוּ וַתָּשָׁב נֶפֶשׁ־הַיֶּלֶד עַל־קִרְבּוֹ וַיֶּ֑חִי}  \hspace{0.3cm}
	6~~\foreignlanguage{hebrew}{וְלָכֵן כֹּה־אָמַר יְהוָה הַמִּטָּה}\LTRfootnote{\space \foreignlanguage{hebrew}{מִטָּה} \textit{couch, bed} (root \foreignlanguage{hebrew}{נטה})} \foreignlanguage{hebrew}{אֲשֶׁר־עָלִיתָ שָּׁם לֹא־תֵרֵד מִמֶּנָּה כִּי מוֹת תָּמוּת} \hspace{0.3cm}
	7~~\foreignlanguage{hebrew}{וַיֹּאמֶר שְׁמוּאֵל אֶל־שָׁאוּל לֹא אָשׁוּב עִמָּךְ כִּי מָאַסְתָּה אֶת־דְּבַר יְהוָה וַיִּמְאָסְךָ יְהוָה מִהְיוֹת מֶלֶךְ עַל־יִשְׂרָאֵֽל}  \hspace{0.3cm}
	8~~\foreignlanguage{hebrew}{וַיְהִי אַחֲרֵי הַדְּבָרִים הָאֵלֶּה וַיָּמָת יְהוֹשֻׁעַ בִּן־נוּן עֶבֶד יְהוָה בֶּן־מֵאָה וָעֶשֶׂר שָׁנִים}  \hspace{0.3cm}
	9~~\foreignlanguage{hebrew}{וַיֹּאמֶר יְהוָה אֶל־מֹשֶׁה נְטֵה אֶת־יָדְךָ עַל־הַיָּם וְיָשֻׁבוּ הַמַּיִם עַל־מִצְרַיִם עַל־רִכְבּוֹ וְעַל־פָּרָשָׁיו}  \hspace{0.3cm}
	10~~\foreignlanguage{hebrew}{וְעַתָּה קְחוּ לָכֶם שְׁנֵי עָשָׂר אִישׁ מִשִּׁבְטֵי יִשְׂרָאֵל אִישׁ־אֶחָד אִישׁ־אֶחָד לַשָּׁ֫בֶט}\LTRfootnote{\space \foreignlanguage{hebrew}{שָׁ֫בֶט} pausal form of \foreignlanguage{hebrew}{שֶׁבֵט}}  \hspace{0.3cm}
	11~~\foreignlanguage{hebrew}{וַיְצַו}\LTRfootnote{\space \foreignlanguage{hebrew}{וַיְצַו} \textit{and he commanded} (normally with direct object but also with prepositional object with \foreignlanguage{hebrew}{עַל})} \foreignlanguage{hebrew}{יְהוָה אֱלֹהִים עַל־הָאָדָם לֵאמֹר מִכֹּל עֵץ־הַגָּן}\LTRfootnote{\space \foreignlanguage{hebrew}{גַּן} \textit{garden}} \foreignlanguage{hebrew}{אָכֹל תֹּאכֵל. וּמֵעֵץ הַדַּעַת טוֹב וָרָע לֹא תֹאכַל מִמֶּנּוּ כִּי בְּיוֹם אֲכָלְךָ מִמֶּנּוּ מוֹת תָּמוּת}  \hspace{0.3cm}
	12~~\foreignlanguage{hebrew}{וַיְהִי בִּימֵי שְׁפֹט הַשֹּׁפְטִים וַיְהִי רָעָב בָּאָרֶץ וַיֵּלֶךְ אִישׁ מִבֵּית לֶחֶם יְהוּדָה לָגוּר בִּשְׂדֵי מוֹאָב הוּא וְאִשְׁתּוֹ וּשְׁנֵי בָנָיו. וְשֵׁם הָאִישׁ אֱלִימֶלֶךְ וְשֵׁם אִשְׁתּוֹ נָעֳמִי וְשֵׁם שְׁנֵי־בָנָיו מַחְלוֹן וְכִלְיוֹן אֶפְרָתִים}\LTRfootnote{\space \foreignlanguage{hebrew}{אֶפְרָתִי} \textit{person from} \foreignlanguage{hebrew}{אֶפְרָ֫תָה} (which is \foreignlanguage{hebrew}{בֵּית לֶחֶם} here)} \foreignlanguage{hebrew}{מִבֵּית לֶחֶם יְהוּדָה וַיָּבֹאוּ שְׂדֵי־מוֹאָב וַיִּהְיוּ־שָׁם. וַיָּמָת אֱלִימֶלֶךְ אִישׁ נָעֳמִי וַתִּשָּׁאֵר}\LTRfootnote{\space \foreignlanguage{hebrew}{וַתִּשָּׁאֵר} \textit{and she was left behind} (Niphal \foreignlanguage{hebrew}{שׂאר})} \foreignlanguage{hebrew}{הִיא וּשְׁנֵי בָנֶיהָ. וַיִּשְׂאוּ לָהֶם נָשִׁים מֹאֲבִיּוֹת}\LTRfootnote{\space \foreignlanguage{hebrew}{מוֺאָבִי} \textit{Moabite}} \foreignlanguage{hebrew}{שֵׁם הָאַחַת עָרְפָּה וְשֵׁם הַשֵּׁנִית רוּת וַיֵּשְׁבוּ שָׁם כְּעֶשֶׂר שָׁנִים. וַיָּמוּתוּ גַם־שְׁנֵיהֶם מַחְלוֹן וְכִלְיוֹן וַתִּשָּׁאֵר}\LTRfootnote{\space \foreignlanguage{hebrew}{וַתִּשָּׁאֵר} \textit{and she was left behind}} \foreignlanguage{hebrew}{הָאִשָּׁה מִשְּׁנֵי יְלָדֶיהָ וּמֵאִישָׁהּ. וַתָּקָם הִיא וְכַלֹּתֶיהָ}\LTRfootnote{\space \foreignlanguage{hebrew}{כַּלָּה} \textit{daughter in law, bride}} \foreignlanguage{hebrew}{וַתָּשָׁב מִשְּׂדֵי מוֹאָב כִּי שָׁמְעָה בִּשְׂדֵה מוֹאָב כִּי־פָקַד יְהוָה אֶת־עַמּוֹ לָתֵת לָהֶם לָחֶם. וַתֵּצֵא מִן־הַמָּקוֹם אֲשֶׁר הָיְתָה־שָׁמָּה וּשְׁתֵּי כַלֹּתֶיהָ}\LTRfootnote{\space \foreignlanguage{hebrew}{כַּלָּה} \textit{daughter in law, bride}} \foreignlanguage{hebrew}{עִמָּהּ וַתֵּלַכְנָה בַדֶּרֶךְ לָשׁוּב אֶל־אֶרֶץ יְהוּדָה.}
	
	\selectlanguage{english}
	
	
	
	\chapter{Chapter 15}
	
	\renewcommand\arraystretch{1.4}
	
	\section{Vocabulary}
	
	\subsection{Verbs}
	
	\begin{center}
		
		% For the centering of the separation between the two columns see the documentation of the array package, page 2 
		
		\begin{tabular}{>{\raggedleft}p{0.175\linewidth} p{0.75\linewidth}}
			\foreignlanguage{hebrew}{חנן} & Q.\ \textit{to show favor, be gracious} \\
			\foreignlanguage{hebrew}{מדד} & Q.\ \textit{to measure} \\
			\foreignlanguage{hebrew}{לבשׁ} & Q.\ \textit{to put on} (a garment), \textit{to clothe oneself, clothe} (stative verb; SC \foreignlanguage{hebrew}{לָבֵשׁ} and \foreignlanguage{hebrew}{לָבַשׁ}, PC \foreignlanguage{hebrew}{יִלְבַּשׁ}) \\
			\foreignlanguage{hebrew}{לין} & Q.\ \textit{to spent the night, stay overnight; to stay, dwell} (cf.\ \foreignlanguage{hebrew}{לַיְלָה}) \\
			\foreignlanguage{hebrew}{מדד} & Q.\ \textit{to measure} \\
			\foreignlanguage{hebrew}{משׁח} & Q.\ \textit{to anoint, to smear} \\
			\foreignlanguage{hebrew}{סבב} & Q.\ \textit{to turn about, go around, surround} \\
			\foreignlanguage{hebrew}{עזר} & Q.\ \textit{to help} \\
			\foreignlanguage{hebrew}{שׁדד} & Q.\ \textit{to devastate, despoil, deal violently with} \\
			\foreignlanguage{hebrew}{שׁמם} & Q.\ \textit{to uninhabited, be deserted; to shudder, be appalled} \\
			\foreignlanguage{hebrew}{תמם} & Q.\ \textit{to be complete, finished} \\
			\foreignlanguage{hebrew}{תקע} & Q.\ \textit{to give a blow, blast; clap; drive in} \\ % BDB
		\end{tabular}
	\end{center}
	
	\subsection{Nouns}
	
	\begin{center}
		\begin{longtable}{>{\raggedleft}p{0.175\linewidth} p{0.75\linewidth}}
			\foreignlanguage{hebrew}{חוּץ} & \textit{outside, street} (m.; pl.\ \foreignlanguage{hebrew}{חוּצוֺת}) \\
			\foreignlanguage{hebrew}{חֲצִי} & \textit{half, middle} \\
			\foreignlanguage{hebrew}{מִדָּה} & \textit{measured length; measurement} \\ % HALOT
			\foreignlanguage{hebrew}{מָוֶת} & \textit{death} \\
			\foreignlanguage{hebrew}{נֵדֶר} & \textit{vow} \\
			\foreignlanguage{hebrew}{עָוֹן} & \textit{iniquity, guilt} (pl.\ \foreignlanguage{hebrew}{עֲוֹנוֺת})\\ % BDB
			\foreignlanguage{hebrew}{פַּעַם} & \textit{foot; time; step, pace} (fem.) \\
			\foreignlanguage{hebrew}{פֶּתַח} & \textit{opening, doorway, entrance} \\
			\foreignlanguage{hebrew}{קֶדֶם} & \textit{front; east; [what was] before, earlier; prehistoric times, primeval} \\ % HALOT
			& \textit{times} \\
		\end{longtable}
	\end{center}
	
	\subsection{Other Parts of Speech}
	
	\begin{center}
		\begin{tabular}{>{\raggedleft}p{0.175\linewidth} p{0.75\linewidth}}
			\foreignlanguage{hebrew}{אִם} & \textit{if} (conj.) \\
		\end{tabular}
	\end{center}
	
	\section{Weak Verbs: Geminate Verbs}
	
	Geminate verbs are originally monosyllabic verbs with a doubled second root consonant. A number of the Qal forms of II\,gem.\ verbs are strong. The majority of forms, however, is weak.
	
	\begin{center}
		\begin{longtable}{|lll|r|r|}
			\hline
			& & & fientive & stative \\
			\hline
			SC & sg. & 3 m. & \foreignlanguage{hebrew}{סָבַב} & \foreignlanguage{hebrew}{קַל} \\
			& & 3 f. & \foreignlanguage{hebrew}{סָבְבָה} & \foreignlanguage{hebrew}{קַ֫לָּה} \\
			& & 2 m. & \foreignlanguage{hebrew}{סַבּ֫וֺתָ} & \foreignlanguage{hebrew}{קַלּ֫וֺתָ} \\
			& & 2 f. & \foreignlanguage{hebrew}{סַבּוֺת} & \foreignlanguage{hebrew}{קַלּוֺת} \\
			& & 1 c. & \foreignlanguage{hebrew}{סַבּ֫וֺתִי} & \foreignlanguage{hebrew}{קַלּ֫וֺתִי} \\
			& pl. & 3 c. & \foreignlanguage{hebrew}{סָבְְבוּ} & \foreignlanguage{hebrew}{קַ֫לּוּ} \\
			& & 2 m. & \foreignlanguage{hebrew}{סַבּוֺתֶם} & \foreignlanguage{hebrew}{קַלּוֺתֶם} \\
			& & 2 f. & \foreignlanguage{hebrew}{סַבּוֺתֶן} & \foreignlanguage{hebrew}{קַלּוֺתֶן} \\
			& & 1 c. & \foreignlanguage{hebrew}{סַבּ֫וֺנוּ} & \foreignlanguage{hebrew}{קַלּ֫וֺנוּ} \\
			\hline
			PC & sg. & 3 m. & \foreignlanguage{hebrew}{יָסֹב} & \foreignlanguage{hebrew}{יֵקַל} \\
			& & 3 f. & \foreignlanguage{hebrew}{תָּסֹב} & \foreignlanguage{hebrew}{תֵּקַל} \\
			& & 2 m. & \foreignlanguage{hebrew}{תָּסֹב} & \foreignlanguage{hebrew}{תֵּקַל} \\
			& & 2 f. & \foreignlanguage{hebrew}{תָּסֹ֫בִּי} & \foreignlanguage{hebrew}{תֵּקַ֫לִּי} \\
			& & 1 c. & \foreignlanguage{hebrew}{אָסֹב} & \foreignlanguage{hebrew}{אֵקַל} \\
			& pl. & 3 m. & \foreignlanguage{hebrew}{יָסֹ֫בּוּ} & \foreignlanguage{hebrew}{יֵקַ֫לּוּ} \\
			& & 3 f. & \foreignlanguage{hebrew}{תְּסֻבֶּ֫ינָה} & \foreignlanguage{hebrew}{תְּקַלֶּ֫ינָה} \\
			& & 2 m. & \foreignlanguage{hebrew}{תָּסֹ֫בּוּ} & \foreignlanguage{hebrew}{תֵּקַ֫לּוּ} \\
			& & 2 f. & \foreignlanguage{hebrew}{תְּסֻבֶּ֫ינָה} & \foreignlanguage{hebrew}{תְּקַלֶּ֫ינָה} \\
			& & 1 c. & \foreignlanguage{hebrew}{נָסֹב} & \foreignlanguage{hebrew}{נֵקַל} \\
			\hline
			Jussive & sg. & 3 m. & \foreignlanguage{hebrew}{יָסֹב} & \\
			\textit{wayyiqṭol} & sg. & 3 m. & \foreignlanguage{hebrew}{וַיָּ֫סָב} & \\
			\hline
			\pagebreak
			\hline
			Impv. & sg. & m. & \foreignlanguage{hebrew}{סֹב} & \\
			& & f. &\foreignlanguage{hebrew}{סֹ֫בִּי} & \\
			& pl. & m. & \foreignlanguage{hebrew}{סֹ֫בּוּ} & \\
			& & f. & \foreignlanguage{hebrew}{סֻבֶּ֫ינָה} & \\
			\hline
			Inf.\,cs. & & & \foreignlanguage{hebrew}{סֹב} & \\
			Inf.\,abs. & & & \foreignlanguage{hebrew}{סָבוֺב} & \\
			\hline
			Part. act. & sg. & m. & \foreignlanguage{hebrew}{סֹבֵב} & \\
			Part pass. & sg. & m. & \foreignlanguage{hebrew}{סָבוּב} & \\
			\hline
		\end{longtable}
	\end{center}
	
	
	\noindent \textbf{Notes}
	\nopagebreak
	
	\noindent Vocalic suffixes are not stressed, e.g., \foreignlanguage{hebrew}{וַיָּסֹ֫בּוּ} \textit{and they marched around [the city]} (Josh 6:14), \foreignlanguage{hebrew}{תַּ֫מּוּ} \textit{they were complete} (Josh 4:1); but in strong forms they are stressed, e.g., \foreignlanguage{hebrew}{סָבְְבוּ} \textit{they marched around} (Josh 6:15).
	
	Fientive II\,gem.\ verbs have strong forms in the suffix conjugation Qal 3 m./f.\ and 3 c.\ pl.\ with dissociated second root consonant (e.g., \foreignlanguage{hebrew}{סָבַב}, \foreignlanguage{hebrew}{סָבְבוּ}) while other forms are weak with a doubled second root consonant and a connecting vowel before the consonantal suffix (e.g., \foreignlanguage{hebrew}{סַבּ֫וֺתָ}).
	
	Stative II\,gem.\ verbs normally have weak forms with doubled second root consonant. There are, however, some strong suffix conjugation forms of stative II\,gem.\ verbs with dissociated second root consonant.
	
	Instead of weak forms with connecting vowel between the second root consonant and the suffix there are also forms without connecting vowel, e.g., the 1 c.\ pl.\ form \foreignlanguage{hebrew}{תַּ֫מְנוּ} \textit{we have been consumed} (Jer 44:18).
	
	Prefix conjugation forms of II\,gem.\ verbs are characterized by an open prefix syllable. The vowel is preserved if the following syllable is stressed, e.g., \foreignlanguage{hebrew}{יָסֹ֫בּוּ}.
	
	As in geminate nouns, the gemination of the second root consonant is lost in forms without an ending. When an ending is added, the gemination returns.
	
	Fientive verbs have the vowel pattern \textit{a--u} of the fientive type of the strong verb \textit{*yaqṭul(u)} with preservation of the vowel quality of the prefix vowel and slight adjustment of the vowel quality of the thematic vowel: \textit{*yasubbu > *yasubb > yāsōḇ} \foreignlanguage{hebrew}{יָסֹב}.
	
	Stative  II\,gem.\ verbs have the vowel pattern \textit{i--a} \textit{*yikbad(u)} with change of the prefix vowel from /i/ to /ē/: \textit{*yiqallu > *yiqall > yēqal} \foreignlanguage{hebrew}{יֵקַל}.
	
	In \textit{wayyiqṭol} forms without suffix, retraction of stress is attested, e.g., \foreignlanguage{hebrew}{וַנָּ֫סָב} \textit{and we went around [Mount Seir]} (Deut 2:1); \foreignlanguage{hebrew}{וַיָּ֫הָם} \textit{and he brought into confusion} (Exod 14:24); but cf.\ \foreignlanguage{hebrew}{וַתֵּקַל} \textit{and she became insignificant} (Gen 16:4) with stress on the last syllable.
	
	Alternative forms of the prefix conjugation have a geminated first root consonant and a simple second root consonant, e.g., 3 m.\ sg.\ \foreignlanguage{hebrew}{יִסֹּב} (instead of \foreignlanguage{hebrew}{יָסֹב}), 3 m.\ pl.\ \foreignlanguage{hebrew}{יִסְּבוּ} (instead of \foreignlanguage{hebrew}{יָסֹבּוּ}), 3 f.\ pl.\ \foreignlanguage{hebrew}{תִּסֹּ֫בְנָה} (instead of \foreignlanguage{hebrew}{תְּסֻבֶּ֫ינָה}). As this is normal in Aramaic, these forms are called \textit{aramaizing forms}.
	
	Participle forms of fientive verbs are strong. Stative verbs may have an adjective instead of a dedicated participle forms, e.g., \foreignlanguage{hebrew}{קַל} \textit{light, swift} (pl.\ \foreignlanguage{hebrew}{קַלִּים}) from the verb \foreignlanguage{hebrew}{קלל}.
	
	The vowel pattern \textit{a--i} in the prefix conjugation is preserved with a few infrequent verbs, e.g., \foreignlanguage{hebrew}{יָגֵן} \textit{he will defend} from the verb \foreignlanguage{hebrew}{גנן} (Isa 31:5).\footnote{\space At face value, a form like this looks like a Hiphil form. In Rabbinic Hebrew Qal forms of the verb  \foreignlanguage{hebrew}{גנן} were indeed reanalyzed as Hiphil forms.}
	
	% Check II gem verbs forms without connecting vowel
	
	In a few instances, the gemination of the second root consonant is lost, e.g., \foreignlanguage{hebrew}{נָבֹ֫זָה} \textit{let us plunder} (1\,Sam 14:36), \foreignlanguage{hebrew}{נָבְלָה} \textit{let us confuse} (Gen 11:7) from the verbs \foreignlanguage{hebrew}{בזז} and \foreignlanguage{hebrew}{בלל}, respectively.
	
	
	\section{Conditional Clauses}
	
	% The terms "subject" and "contingent" are taken from Siebenthal, Heinrich von, \textit{Ancient Greek Grammar for the Study of the New Testament} (Oxford [e.a.]: Peter Lang, 2019), 522 and Hadumod Bussmann, Routledge Dictionary of Language and Linguistics, 1996, p. 93 (entry conditional clause)
	
	A conditional clause indicates a condition that another action, event or state is subject to or contingent on. The condition and the contingent action, event or state may be simply be two clauses next to each other. Sometimes they are juxtaposed without the conjunction \foreignlanguage{hebrew}{וְ} (Neh 1:8) but normally they are joined by means of the conjunction \foreignlanguage{hebrew}{וְ} (1\,Sam 16:2; Gen 44:22).
	
	\vspace{0.5cm}
	
	% Accents in Neh 1:8 added
	
	\begin{tabular}{>{\raggedleft}p{0.35\linewidth} p{0.55\linewidth}}
		\foreignlanguage{hebrew}{אַתֶּם תִּמְעָ֔לוּ אֲנִי אָפִיץ אֶתְכֶם בָּעַמִּים} & \textit{If you will act unfaithfully, I will disperse you among the nations.} (Neh 1:8) \\
		\foreignlanguage{hebrew}{וַיֹּאמֶר שְׁמוּאֵל אֵיךְ אֵלֵךְ וְשָׁמַע שָׁאוּל וַהֲרָגָנִי} & \textit{And Samuel Said, \enquote{How can I go? If Saul hears about this, he will kill me.}} (1\,Sam 16:2) \\
		\foreignlanguage{hebrew}{וַנֹּאמֶר אֶל־אֲדֹנִי לֹא־יוּכַל הַנַּעַר לַעֲזֹב אֶת־אָבִיו וְעָזַב אֶת־אָבִיו וָמֵת} & \textit{And we said to my lord, \enquote{The boy cannot leave his father. If he leaves his father, [his father] will die.}} (Gen 44:22) \\
	\end{tabular}
	
	\vspace{0.5cm}
	
	Most commonly, however, conditional clauses are introduced by the conjunctions \foreignlanguage{hebrew}{אִם} \textit{if} and less frequently by \foreignlanguage{hebrew}{כִּי} \textit{if}. With the conjunction \foreignlanguage{hebrew}{אִם}:
	
	\vspace{0.5cm}
	
	% Examples from Gesenius, 18th ed.
	
	\begin{tabular}{>{\raggedleft}p{0.35\linewidth} p{0.55\linewidth}}
		\foreignlanguage{hebrew}{אִם־טוֹב בְּעֵינֶיךָ לָבוֹא אִתִּי בָבֶל בֹּא וְאָשִׂים אֶת־עֵינִי עָלֶיךָ וְאִם־רַע בְּעֵינֶיךָ לָבוֹא־אִתִּי בָבֶל חֲדָֽל} & \textit{If it is good in your sight to come with me to Babylon, then come and I will take care of you. But if it is bad in your sight to come with me to Babylon, then don't come} (Jer 40:4) \\
		\foreignlanguage{hebrew}{אִם־נָא מָצָאתִי חֵן בְּעֵינֶיךָ אַל־נָא תַעֲבֹר מֵעַל עַבְדֶּֽךָ} & \textit{If I have found favor in your sight, do not pass by your servant} (Gen 18:3) \\
		\foreignlanguage{hebrew}{אִם־אֶמְצָא בִסְדֹם חֲמִשִּׁים צַדִּיקִם בְּתוֹךְ הָעִיר וְנָשָׂ֫אתִי לְכָל־הַמָּקוֹם בַּעֲבוּרָם} & \textit{If I find in Sodom fifty righteous within the city, I will forgive the whole place for their sake} (Gen 18:26) \\
	\end{tabular}
	
	\vspace{0.5cm}
	
	\noindent Examples with the conjunction \foreignlanguage{hebrew}{כִּי}:
	
	% Examples form BDB
	
	\vspace{0.5cm}
	
	\begin{longtable}{>{\raggedleft}p{0.35\linewidth} p{0.55\linewidth}}
		
		\foreignlanguage{hebrew}{כִּי־תִמְצָא אִישׁ לֹא תְבָרְכֶנּוּ} & \textit{If you meet anyone, do not greet him} (2\,Kgs 4:29) \\
		\foreignlanguage{hebrew}{כִּי תֹאמַר בִּלְבָבְךָ רַבִּים הַגּוֹיִם הָאֵלֶּה מִמֶּנִּי אֵיכָה אוּכַל לְהוֹרִישָָם לֹא תִירָא מֵהֶם} & \textit{If you say in your heart, \enquote{These nation are more numerous than we; how could I dispossess them?} do not be afraid of them \dots} (Deut 7:17--18) \\
	\end{longtable}
	
	\vspace{0.5cm}
	
	\noindent \textbf{Note}
	\nopagebreak
	
	\noindent In these examples the use of the conjunction \foreignlanguage{hebrew}{כִּי} comes close to its use as temporal conjunction with the meaning \textit{when}.
	
	\vspace{0.5cm}
	
	The main clause may simply follow the \textit{if}-clause (Exod 21:7) or it may be connected to the \textit{if}-clause my means of the conjunction \foreignlanguage{hebrew}{וְ}, the so-called \textit{waw} of apodosis (or \textit{waw apodoseos}) (Exod 21:35).\footnote{\space The technical term for the \textit{if}-clause is \emph{protasis} (from Greek \foreignlanguage{greek}{πρότασις}); the  main clause is called \emph{apodosis} (from Greek \foreignlanguage{greek}{ἀπόδοσις}).} At times, it may be difficult to identify the beginning of the main clause, but logic and the accents in a pointed text make the identification certain in most cases. In the example Exod 21:35 the main clause begins with \foreignlanguage{hebrew}{וּמָכְרוּ} \textit{and they shall sell}. (The conjunction \foreignlanguage{hebrew}{וְ} before the conditional conjunction \foreignlanguage{hebrew}{כִּי} connects this law with the preceding context.)
	
	% Other possible example for a main clause without waw apodoseos is Ex 21:2
	
	\vspace{0.5cm}
	
	\begin{tabular}{>{\raggedleft}p{0.35\linewidth} p{0.55\linewidth}}
		\foreignlanguage{hebrew}{וְכִי־יִמְכֹּר אִישׁ אֶת־בִּתּוֹ לְאָמָה לֹא תֵצֵא כְּצֵאת הָעֲבָדִים} & \textit{And if someone sells his daughter as a slave, she shall not go out as the [male] slaves do} (Exod 21:7) \\
		\foreignlanguage{hebrew}{וְכִי־יִגֹּף שׁוֺר־אִישׁ אֶת־שׁוֹר רֵעֵהוּ וָמֵת וּמָכְרוּ אֶת־הַשּׁוֹר הַחַי וְחָצוּ אֶת־כַּסְפּוֹ וְגַם אֶת־הַמֵּת יֶחֱצוּן} & \textit{If the ox of a man injures the ox of his neighbor and it dies, they shall sell the live ox and divide its price and they shall also divide the dead one} (Exod 21:35) \\
	\end{tabular}
	
	\vspace{0.5cm}
	
	\section{Accents}
	
	A fully pointed biblical text does not only include vowel signs and other reading signs as the \textit{šwa} or the \textit{meteg} but also accents. The original purpose of the accents was to regulate the musical recitation of the biblical text (JM §\,15\,\textit{e}).
	
	The accents can be divided into disjunctive accents and conjunctive accents. Whereas disjunctive accents separate words and phrases from each other, conjunctive accents join them to the following element or elements in a clause.
	
	The majority of accents are on the stressed syllable of a word while a number of accents are on the last syllable of a word (\emph{postpositive} accents) or on the first syllable of a word (\emph{prepositive accents}). Therefore, postpositive and prepositive accents do not necessarily indicate the stressed syllable of the word. An exception is the disjunctive postpositive accent \textit{pašta} which is repeated on the stressed syllable if it is the penultimate syllable, e.g., \foreignlanguage{hebrew}{דָּבָר֙}, \foreignlanguage{hebrew}{מֶ֙לֶךְ֙}.
	
	In this chapter the four most important disjunctive accents are introduced and then used in the Hebrew reading (Section 15.6).
	
	\vspace{0.5cm}
	
	% The names of the accents follow the JM §15g-h
	
	\begin{tabular}{rll}
		\foreignlanguage{hebrew}{דָּבָֽר} & \textit{silluq} & with the last word of the verse before \foreignlanguage{hebrew}{׃} \textit{sof pasuq} (\textit{end of the} \\
		& & \textit{verse} \\
		\foreignlanguage{hebrew}{דָּבָ֑ר} & \textit{atnaḥ} & divides the verse into two parts \\
		\foreignlanguage{hebrew}{דָּבָר֒} & \textit{segolta} & postpositive \\
		\foreignlanguage{hebrew}{דָּבָ֔ר} & \textit{zaqef qaṭon} & \\
	\end{tabular}
	
	\vspace{0.5cm}
	
	Accents serve two important functions for the study of the Hebrew Bible. First, most accents indicate the position of stress which is helpful for the correct pronunciation especially of unfamiliar words. Second, accents help to identify grammatical relations between words. For example, in Exod 21:35 in Section 3 the \textit{if}-clause ends with \foreignlanguage{hebrew}{וָמֵ֑ת} with \textit{atnaḥ}, thus marking the beginning of the main clause. In 1\,Kgs 1:19 the \textit{atnaḥ} separates between the objects of the verbal predicates \foreignlanguage{hebrew}{וַיִּקְרָא} and \foreignlanguage{hebrew}{קָרָא}, respectively.
	
	\vspace{0.5cm}
	
	\begin{tabular}{>{\raggedleft}p{0.35\linewidth} p{0.55\linewidth}}
		\foreignlanguage{hebrew}{וַיִּזְבַּח שׁוֹר וּמְרִיא־וְצֹאן לָרֹב֒ וַיִּקְרָא לְכָל־בְּנֵי הַמֶּ֔לֶךְ וּלְאֶבְיָתָר הַכֹּ֔הֵן וּלְיֹאָב שַׂר הַצָּבָ֑א וְלִשְׁלֹמֹה עַבְדְּךָ לֹא קָרָֽא׃} & \textit{And he sacrificed oxen, fattened calves and small cattle in abundance and invited all the sons of the king and Abiathar the priest and Joab the commander of the army; but he did not invite your servant Solomon} (1\,Kgs 1:19) \\
	\end{tabular}
	
	\vspace{0.5cm}
	
	Pausal forms are used with \textit{silluq} and \textit{atnaḥ}; but they can occur with other disjunctive accents as well. 
	
	
	\section{Exercises}
	
	\subsection{Translation of Verbal Forms}
	Translate the following verbal forms. Identify the gender (masc., fem., comm.) and number (sg., pl.) of forms of which the English translation is ambiguous (i.e., \textit{you}, \textit{they}). Mark the stressed syllable if stress is not on the last syllable.
	
	\hspace{0.5cm}
	
	\selectlanguage{hebrew}
	
	\noindent
	1~~\foreignlanguage{hebrew}{וַיָּמֹדּוּ}  \hspace{0.3cm}
	2~~\foreignlanguage{hebrew}{תָּמֹדּוּ}  \hspace{0.3cm}
	3~~\foreignlanguage{hebrew}{וַיָּסֹבּוּ}  \hspace{0.3cm}
	4~~\foreignlanguage{hebrew}{וַיִּסֹּב}  \hspace{0.3cm}
	5~~\foreignlanguage{hebrew}{סֹבּוּ}  \hspace{0.3cm}
	6~~\foreignlanguage{hebrew}{וְסַבֹּתֶם}  \hspace{0.3cm}
	7~~\foreignlanguage{hebrew}{סָבַב}  \hspace{0.3cm}
	8~~\foreignlanguage{hebrew}{תְסֻבֶּינָה}  \hspace{0.3cm}
	9~~\foreignlanguage{hebrew}{וַנָּסָב}  \hspace{0.3cm}
	10~~\foreignlanguage{hebrew}{שֹׁדֵד}  \hspace{0.3cm}
	11~~\foreignlanguage{hebrew}{יִשֹּׁם}  \hspace{0.3cm}
	12~~\foreignlanguage{hebrew}{יָשֹׁמּוּ}  \hspace{0.3cm}
	13~~\foreignlanguage{hebrew}{יִתַּמּוּ}  \hspace{0.3cm}
	14~~\foreignlanguage{hebrew}{תַּמּוּ}  \hspace{0.3cm}
	15~~\foreignlanguage{hebrew}{תַּם}  \hspace{0.3cm}
	16~~\foreignlanguage{hebrew}{יָחֹן}  \hspace{0.3cm}
	17~~\foreignlanguage{hebrew}{וַיָּחָן}  \hspace{0.3cm}
	18~~\foreignlanguage{hebrew}{אָחֹן}  \hspace{0.3cm}
	19~~\foreignlanguage{hebrew}{וְחַנֹּתִי}  \hspace{0.3cm}
	20~~\foreignlanguage{hebrew}{חָנֵּנִי}  \hspace{0.3cm}
	
	\selectlanguage{english}
	
	\subsection{Translation of Sentences}
	Translate the following sentences from the Hebrew Bible. Names of persons and geographical names in these sentences: \foreignlanguage{hebrew}{בָּרָק}, \foreignlanguage{hebrew}{יְהוֹשֻׁעַ},  \foreignlanguage{hebrew}{יַעֲקֹב}, \foreignlanguage{hebrew}{יַרְדֵּן}, \foreignlanguage{hebrew}{יְרִיחוֹ}, \foreignlanguage{hebrew}{מִצְפָּה}, \foreignlanguage{hebrew}{שְׁמוּאֵל}.
	
	\vspace{0.5cm}
	
	% Action point: horizontal space (hspace) reduced for sentences 1 and 2 to improve line breaking.
	
	\selectlanguage{hebrew}
	\noindent
	1~~\foreignlanguage{hebrew}{שְׁמַע־יְהוָה וְחָנֵּ֫נִי}\LTRfootnote{\space Verbal form \foreignlanguage{hebrew}{חֹן}* with enclitic person pronoun} \foreignlanguage{hebrew}{יְהוָה הֱיֵה־עֹזֵר לִי} \hspace{0.1cm}
	2~~\foreignlanguage{hebrew}{וַיְהִי כַּאֲשֶׁר־תַּמּוּ כָל־הַגּוֹי לַעֲבוֹר אֶת־הַיַּרְדֵּן וַיֹּאמֶר יְהוָה אֶל־יְהוֹשֻׁעַ לֵאמֹר קְחוּ לָכֶם מִן־הָעָם שְׁנֵים עָשָׂר אֲנָשִׁים אִישׁ־אֶחָד אִישׁ־אֶחָד מִשָּׁ֫בֶט}\LTRfootnote{\space \foreignlanguage{hebrew}{שָׁ֫בֶט} pausal form of \foreignlanguage{hebrew}{שֵׁבֶט}}  \hspace{0.2cm}
	3~~\foreignlanguage{hebrew}{וַיֹּאמֶר יְהוָה אֶל־יְהוֹשֻׁעַ רְאֵה נָתַתִּי בְיָדְךָ אֶת־יְרִיחוֹ וְאֶת־מַלְכָּהּ גִּבּוֹרֵי הֶחָֽיִל}\LTRfootnote{\space \foreignlanguage{hebrew}{חָֽיִל} pausal form of \foreignlanguage{hebrew}{חַיִל}} \foreignlanguage{hebrew}{ וְסַבֹּתֶם אֶת־הָעִיר כֹּל אַנְשֵׁי הַמִּלְחָמָה הַקֵּיף}\LTRfootnote{\space \foreignlanguage{hebrew}{הַקֵּיף} \textit{go around!} (inf.\ abs.\ Hiphil of \foreignlanguage{hebrew}{נקף} used as impv.)} \foreignlanguage{hebrew}{אֶת־הָעִיר פַּעַם אֶחָת}\LTRfootnote{\space \foreignlanguage{hebrew}{אֶחָת} pausal form of \foreignlanguage{hebrew}{אַחַת}} \foreignlanguage{hebrew}{כֹּה תַעֲשֶׂה שֵׁשֶׁת יָמִים. וְשִׁבְעָה כֹהֲנִים יִשְׂאוּ שִׁבְעָה שׁוֹפְרוֹת הַיּוֹבְלִים}\LTRfootnote{\space \foreignlanguage{hebrew}{יוֺבֵל} \textit{ram}} \foreignlanguage{hebrew}{לִפְנֵי הָאָרוֹן וּבַיּוֹם הַשְּׁבִיעִי תָּסֹבּוּ אֶת־הָעִיר שֶׁבַע פְּעָמִים וְהַכֹּהֲנִים יִתְקְעוּ בַּשּׁוֹפָרוֹת.}
	4~~\foreignlanguage{hebrew}{וַיֹּאמֶר אֵלֶיהָ בָּרָק אִם־תֵּלְכִי עִמִּי וְהָלָכְתִּי וְאִם־לֹא תֵלְכִי עִמִּי לֹא אֵלֵךְ}  \hspace{0.3cm}
	5~~\foreignlanguage{hebrew}{וְעַתָּה אִם־שָׁמוֹעַ תִּשְׁמְעוּ בְּקֹלִי וּשְׁמַרְתֶּם אֶת־בְּרִיתִי וִהְיִיתֶם לִי סְגֻלָּה}\LTRfootnote{\space \foreignlanguage{hebrew}{סְגֻלָּה} \textit{(personal) property}} \foreignlanguage{hebrew}{מִכָּל־הָעַמִּים כִּי־לִי כָּל־הָאָרֶץ וְאַתֶּם תִּהְיוּ־לִי מַמְלֶכֶת כֹּהֲנִים וְגוֹי קָדוֹשׁ אֵלֶּה הַדְּבָרִים אֲשֶׁר תְּדַבֵּר}\LTRfootnote{\space \foreignlanguage{hebrew}{תְּדַבֵּר} \textit{you shall speak}} \foreignlanguage{hebrew}{אֶל־בְּנֵי יִשְׂרָאֵל} \hspace{0.3cm}
	6~~\foreignlanguage{hebrew}{וַיֹּאמֶר אִישׁ־הָאֱלֹהִים אֶל־הַמֶּלֶךְ אִם־תִּתֶּן־לִי אֶת־חֲצִי בֵיתֶךָ לֹא אָבֹא עִמָּךְ וְלֹא־אֹכַל לֶחֶם וְלֹא אֶשְׁתֶּה־מַּיִם בַּמָּקוֹם הַזֶּה}  \hspace{0.3cm}
	7~~\foreignlanguage{hebrew}{וַיִּדַּר}\LTRfootnote{\space \foreignlanguage{hebrew}{וַיִּדַּר} \textit{and he performed a vow} (\foreignlanguage{hebrew}{נדר}; often with the internal object \foreignlanguage{hebrew}{נֵדֶר})} \foreignlanguage{hebrew}{יַעֲקֹב נֶדֶר לֵאמֹר אִם־יִהְיֶה אֱלֹהִים עִמָּדִי וּשְׁמָרַנִי בַּדֶּרֶךְ הַזֶּה אֲשֶׁר אָנֹכִי הוֹלֵךְ וְנָתַן־לִי לֶחֶם לֶאֱכֹל וּבֶגֶד לִלְבֹּשׁ וְשַׁבְתִּי בְשָׁלוֹם אֶל־בֵּית אָבִי וְהָיָה יְהוָה לִי לֵאלֹהִים} \hspace{0.3cm}
	8~~\foreignlanguage{hebrew}{וַיִּקַּח שְׁמוּאֵל אֶבֶן אַחַת וַיָּשֶׂם בֵּין־הַמִּצְפָּה וּבֵין הַשֵּׁן}\LTRfootnote{\space \foreignlanguage{hebrew}{הַשֵּׁן} name of a place} \foreignlanguage{hebrew}{וַיִּקְרָא אֶת־שְׁמָהּ אֶבֶן הָעָזֶר}\LTRfootnote{\space \foreignlanguage{hebrew}{עֵזֶר} \textit{help} (noun)} \foreignlanguage{hebrew}{וַיֹּאמַר עַד־הֵנָּה}\LTRfootnote{\space \foreignlanguage{hebrew}{עַד־הֵנָּה} \textit{until now}} \foreignlanguage{hebrew}{עֲזָרָ֫נוּ יְהוָה} \hspace{0.3cm}
	9~~\foreignlanguage{hebrew}{לֹא אֶת־אֲבֹתֵינוּ כָּרַת יְהוָה אֶת־הַבְּרִית הַזֹּאת כִּי אִתָּנוּ אֲנַחְנוּ אֵלֶּה פֹה הַיּוֹם כֻּלָּנוּ חַיִּים}  \hspace{0.3cm}
	10~~\foreignlanguage{hebrew}{וַיָּבֹא־שָׁם אֶל־הַמְּעָרָה}\LTRfootnote{\space \foreignlanguage{hebrew}{מְעָרָה} \textit{cave}} \foreignlanguage{hebrew}{וַיָּלֶן שָׁם וְהִנֵּה דְבַר־יְהוָה אֵלָיו וַיֹּאמֶר לוֹ מַה־לְּךָ פֹה אֵלִיָּהוּ}
	\selectlanguage{english}
	
	
	\section{Hebrew Reading: Genesis 12:4--9}
	Translate Gen 12:4--9 with the help of notes below the text. Names of persons and places are easily identifiable.
	
	\vspace{0.5cm}
	
	\selectlanguage{hebrew}
	\noindent
	\textsuperscript{4}~\foreignlanguage{hebrew}{וַיֵּלֶךְ אַבְרָם כַּאֲשֶׁר דִּבֶּר אֵלָיו יְהוָ֔ה וַיֵּלֶךְ אִתּוֹ לוֹ֑ט וְאַבְרָם בֶּן־חָמֵשׁ שָׁנִים וְשִׁבְעִים שָׁנָ֔ה בְּצֵאתוֹ מֵחָרָֽן׃} \hspace{0.3cm}
	\textsuperscript{5}~\foreignlanguage{hebrew}{וַיִּקַּח אַבְרָם אֶת־שָׂרַי אִשְׁתּוֹ וְאֶת־לוֹט בֶּן־אָחִיו וְאֶת־כָּל־רְכוּשָׁם אֲשֶׁר רָכָ֔שׁוּ וְאֶת־הַנֶּפֶשׁ אֲשֶׁר־עָשׂוּ בְחָרָ֑ן וַיֵּצְאוּ לָלֶכֶת אַרְצָה כְּנַ֔עַן וַיָּבֹאוּ אַרְצָה כְּנָֽעַן׃} \hspace{0.3cm}
	\textsuperscript{6}~\foreignlanguage{hebrew}{וַיַּעֲבֹר אַבְרָם בָּאָ֔רֶץ עַד מְקוֹם שְׁכֶם עַד אֵלוֹן מוֹרֶ֑ה וְהַכְּנַעֲנִי אָז בָּאָֽרֶץ׃} \hspace{0.3cm}
	\textsuperscript{7}~\foreignlanguage{hebrew}{וַיֵּרָא יְהוָה אֶל־אַבְרָ֔ם וַיֹּאמֶר לְזַרְעֲךָ֔ אֶתֵּן אֶת־הָאָרֶץ הַזֹּ֑את וַיִּבֶן שָׁם מִזְבֵּ֔חַ לַיהוָה הַנִּרְאֶה אֵלָֽיו׃} \hspace{0.3cm}
	\textsuperscript{8}~\foreignlanguage{hebrew}{וַיַּעְתֵּק מִשָּׁם הָהָ֫רָה מִקֶּדֶם לְבֵית־אֵל וַיֵּט אָהֳלֹ֑ה בֵּית־אֵל מִיָּם וְהָעַי מִקֶּ֔דֶם וַיִּבֶן־שָׁם מִזְבֵּחַ לַיהוָ֔ה וַיִּקְרָא בְּשֵׁם יְהוָֽה׃} \hspace{0.3cm}
	\textsuperscript{9}~\foreignlanguage{hebrew}{וַיִּסַּע אַבְרָ֔ם הָלוֹךְ וְנָסוֹעַ הַנֶּֽגְבָּה׃} \hspace{0.3cm}
	\selectlanguage{english}
	
	\vspace{1cm}
	
	\hspace*{-0.5cm}\begin{tabular}{lrl}
		12:5 & \foreignlanguage{hebrew}{רְכוּשׁ} & \textit{property, goods} (movable possessions of all kind) (BDB) \\
		& \foreignlanguage{hebrew}{רכשׁ} & Q. \textit{collect, acquire} \\
		12:6 & \foreignlanguage{hebrew}{אֵלוֹן} & \textit{tall tree, terebinth} \\
		12:7 & \foreignlanguage{hebrew}{וַיֵּרָא} & \textit{and he appeared} (Niphal \foreignlanguage{hebrew}{ראה}) \\
		& \foreignlanguage{hebrew}{הַנִּרְאֶה} & \textit{the one who had appeared} (Niphal \foreignlanguage{hebrew}{ראה}) \\
		12:8 & \foreignlanguage{hebrew}{וַיַּעְתֵּק} & \textit{and he moved forward} \\
		& \foreignlanguage{hebrew}{מִקֶּדֶם לְ} & \textit{east of} \\
		& \foreignlanguage{hebrew}{אָהֳלֹה} & Ketiv -- Qere \foreignlanguage{hebrew}{אָהֳלוֺ} \\
		& \foreignlanguage{hebrew}{מִיָּם} & \textit{on the west} (\foreignlanguage{hebrew}{יָם} meaning \textit{west} here) \\
		& \foreignlanguage{hebrew}{מִקֶּדֶם} & \textit{on the east} (\foreignlanguage{hebrew}{קֶדֶם} meaning \textit{east} here) \\
	\end{tabular}
	
	
	
	\chapter{Chapter 16}
	
	\renewcommand\arraystretch{1.4}
	
	\section{Vocabulary}
	
	\subsection{Verbs}
	
	\begin{center}
		
		% For the centering of the separation between the two columns see the documentation of the array package, page 2 
		
		\begin{tabular}{>{\raggedleft}p{0.175\linewidth} p{0.75\linewidth}}
			\foreignlanguage{hebrew}{לחם} & Ni.\ \textit{to fight, do battle} (with prep. objects with \foreignlanguage{hebrew}{בְּ}, \foreignlanguage{hebrew}{אֵת} (\textit{with}), \foreignlanguage{hebrew}{עִם}, \foreignlanguage{hebrew}{אֶל} or \foreignlanguage{hebrew}{עַל}) \\
			\foreignlanguage{hebrew}{מלט} & Ni.\ \textit{to escape, be delivered} \\
			\foreignlanguage{hebrew}{שׁבע} & Ni.\ \textit{to swear, take an oath} \\
		\end{tabular}
	\end{center}
	
	\subsection{Nouns}
	
	\begin{longtable}{>{\raggedleft}p{0.175\linewidth} p{0.75\linewidth}}
		\foreignlanguage{hebrew}{אוֺצָר} & \textit{treasure, supplies} \\ % HALOT
		\foreignlanguage{hebrew}{אַחֵר} & \textit{another, different} (m.\ pl.\ \foreignlanguage{hebrew}{אֲחֵרִים}, f.\ sg.\ \foreignlanguage{hebrew}{אַחֶרֶת}, f.\ pl.\ \foreignlanguage{hebrew}{אֲחֵרוֺת}) \\
		\foreignlanguage{hebrew}{אֱמוּנָה} & \textit{steadfastness; trustworthiness, faithfulness} \\ % HALOT
		\foreignlanguage{hebrew}{אֱמֹרִי} & \textit{Amorite} \\
		\foreignlanguage{hebrew}{אֱמֶת} & \textit{firmness, trustworthiness; faithfulness; truth} (fem.; with ePP \foreignlanguage{hebrew}{אֲמִתּוֺ} \textit{ʾămittō < *ʾămintō} from the root \foreignlanguage{hebrew}{אמן}, cf.\ \foreignlanguage{hebrew}{אָמֵן} \textit{surely!}) \\
		\foreignlanguage{hebrew}{גִּבְעָה} & \textit{hill} \\
		\foreignlanguage{hebrew}{דוֺר} & \textit{generation} (pl. \foreignlanguage{hebrew}{דֹּרוֺת}) \\
		\foreignlanguage{hebrew}{יַיִן} & \textit{wine} \\
		\foreignlanguage{hebrew}{מְלָאכָה} & \textit{occupation, work} (\textit{məlā(ʾ)ḵā}; cs.\ st.\ \foreignlanguage{hebrew}{מְלֶ֫אכֶת}) \\
		\foreignlanguage{hebrew}{נָהָר} & \textit{river, stream} (pl.\ \foreignlanguage{hebrew}{נְהָרוֺת}; \foreignlanguage{hebrew}{נְהַר פְּרָת} \textit{Euphrates}) \\
		\foreignlanguage{hebrew}{נַחַל} & \textit{river valley, wadi; stream} \\
		\foreignlanguage{hebrew}{עֲבֹדָה} & \textit{work, service} \\
		\foreignlanguage{hebrew}{עֵבֶר} & \textit{region across or beyond; side} \\ % BDB
		\foreignlanguage{hebrew}{עֵדָה} & \textit{congregation} \\
		\foreignlanguage{hebrew}{צְדָקָה} & \textit{righteousness} \\
		\foreignlanguage{hebrew}{צָפוֺן} & \textit{north} \\
		\foreignlanguage{hebrew}{רֹב} & \textit{multitude, abundance, greatness} (cf. \foreignlanguage{hebrew}{רַב}) \\
		\foreignlanguage{hebrew}{שָׂפָה} & \textit{lip, speech, edge} \\
		\foreignlanguage{hebrew}{תָּמִים} & \textit{complete; without fault; perfect; impeccable; honest, devout} (adj.) \\ % HALOT
	\end{longtable}
	
	
	\subsection{Other Parts of Speech}
	
	\begin{center}
		\begin{tabular}{>{\raggedleft}p{0.175\linewidth} p{0.75\linewidth}}
			\foreignlanguage{hebrew}{בַּעֲבוּר} & \textit{because of, for the sake of} (prep.): \textit{in order that} (conj.) (prep. \foreignlanguage{hebrew}{בְּ} + noun \foreignlanguage{hebrew}{עֲבוּר}) \\
		\end{tabular}
	\end{center}
	
	\section{The Niphal Binyan}
	
	\subsection{The Forms of the Strong Verb in the Niphal}
	
	The Niphal is the fourth most frequent binyan behind the Qal, the Hiphil (Chapter 17) and the Piel (Chapter 19). It is attested about 4140 times in the Hebrew Bible. The Niphal has mainly reflexive-passive meaning. Morphologically, it is characterized by the additional element /n/ that is prefixed as the syllable \textit{ni-} in the suffix conjugation and participle forms. In the other forms, it is placed as \textit{-n} immediately before the first root consonant and subsequently assimilated to it: \textit{\nobreakdashes-nC\textsubscript{1}- > \nobreakdashes-C\textsubscript{1}C\textsubscript{1}-}. The suffixes and prefixes of the SC, PC and imperative indicating person, number and gender are identical to the suffixes and prefixes of the corresponding forms in the Qal.
	
	\begin{center}
		\begin{longtable}{|lll|r|}
			\hline
			SC &sg. & 3 m. & \foreignlanguage{hebrew}{נִכְתַּב} \\
			& & 3 f. & \foreignlanguage{hebrew}{נִכְתְּבָה} \\
			& & 2 m. & \foreignlanguage{hebrew}{נִכְתַּ֫בְתָּ} \\
			& & 2 f. & \foreignlanguage{hebrew}{נִכְתַּבְתְּ} \\
			& & 1 c. & \foreignlanguage{hebrew}{נִכְתַּ֫בְתִּי} \\
			\hline
			& pl. & 3 c. & \foreignlanguage{hebrew}{נִכְתְּבוּ} \\
			& & 2 m. & \foreignlanguage{hebrew}{נִכְתַּבְתֶּם} \\
			& & 2 f. & \foreignlanguage{hebrew}{נִכְתַּבְתֶּן} \\
			& & 1 c. & \foreignlanguage{hebrew}{נִכְתַּ֫בְנוּ} \\
			\hline
			PC & sg. & 3 m. & \foreignlanguage{hebrew}{יִכָּתֵב} \\
			& & 3 f. & \foreignlanguage{hebrew}{תִּכָּתֵב} \\
			& & 2 m. & \foreignlanguage{hebrew}{תִּכָּתֵב} \\
			& & 2 f. & \foreignlanguage{hebrew}{תִּכָּתְבִי} \\
			& & 1 c. & \foreignlanguage{hebrew}{אֶכָּתֵב} \\
			\hline
			\pagebreak
			\hline
			& pl. & 3 c. & \foreignlanguage{hebrew}{יִכָּתְבוּ} \\
			& & 2 f. & \foreignlanguage{hebrew}{תִּכָּתַ֫בְנָה} \\
			& & 2 m. & \foreignlanguage{hebrew}{תִּכָּתְבוּ} \\
			& & 2 f. & \foreignlanguage{hebrew}{תִּכָּתַ֫בְנָה} \\
			& & 1 c. & \foreignlanguage{hebrew}{נִכָּתֵב} \\
			\hline
			Jussive & sg. & 3 m. & \foreignlanguage{hebrew}{יִכָּתֵב} \\
			\textit{wayyiqṭol} & sg. & 3 m. & \foreignlanguage{hebrew}{וַיִּכָּתֵב} \\
			\hline
			Impv. & sg. & 2 m. & \foreignlanguage{hebrew}{הִכָּתֵב} \\
			& & 2 f. & \foreignlanguage{hebrew}{הִכָּתְבִי} \\
			\hline
			& pl. & 2 m. & \foreignlanguage{hebrew}{הִכָּתְבוּ} \\
			& & 2 f. & \foreignlanguage{hebrew}{הִכָּתַ֫בְנָה} \\
			\hline
			Inf.\ cs.\ & & & \foreignlanguage{hebrew}{הִכָּתֵב} \\
			Inf.\ abs. & & & \foreignlanguage{hebrew}{הִכָּתֹב} \\
			& & & \foreignlanguage{hebrew}{נִכְתֹּב} \\
			& & & \foreignlanguage{hebrew}{הִכָּתֵב} \\
			\hline
			Part. & m. & sg. & \foreignlanguage{hebrew}{נִכְתָּב} \\
			&  & pl. & \foreignlanguage{hebrew}{נִכְתָּבִים} \\
			& f. & sg. & \foreignlanguage{hebrew}{נִכְתֶּ֫בֶת} \\
			&  & pl. & \foreignlanguage{hebrew}{נִכְתָּבוֺת} \\
			\hline
		\end{longtable}
	\end{center}
	
	\noindent \textbf{Notes}
	\nopagebreak
	
	\noindent The forms of the Niphal have the following main characteristics:
	
	\begin{enumerate}[noitemsep]
		\item[--] In the suffix conjugation and the participle, the syllable \textit{ni-} is prefixed to the verbal form, e.g., \foreignlanguage{hebrew}{נִשְׁבַּר} \textit{it has been broken} (SC); \foreignlanguage{hebrew}{נִשְׁבָּר} \textit{broken} (part.).
		\item[--] In the prefix conjugation forms the /n/ of the Ni.\ is assimilated to the first root consonant of the verb: \textit{*yinkatib > yikkātēb} \foreignlanguage{hebrew}{יִכָּתֵב}.
		\item[--] In the imperative, the infinitive construct and some forms of the infinitive absolute a prosthetic consonant /h/ is added to the form in order to preserve the syllable structure and the assimilated /n/ as a sign of the Niphal.
	\end{enumerate}
	
	
	The prefix vowel of the 1 c.\ sg.\ suffix conjugation can also be /i/ instead of /æ/, e.g., \foreignlanguage{hebrew}{אִשָּׁבֵעַ} \textit{I will swear}. The prefix vowel of the cohortative 1 c.\ sg.\ is always /i/, e.g., \foreignlanguage{hebrew}{אִמָּלְטָה} \textit{let me escape}.
	
	In \textit{wayyiqṭol} and imperative forms without suffix, retraction of stress may occur, e.g., \foreignlanguage{hebrew}{וַיִּלָּ֫חֶם} \textit{and he fought}, \foreignlanguage{hebrew}{הִשָּׁ֫מֶר} \textit{be careful!}.
	
	In the participle f.\ sg., forms with the ending /-ā/ are attested as well, e.g., \foreignlanguage{hebrew}{נִשְׁבָּרָה}.
	
	In pausal forms, reduced vowels are preserved as full vowels, e.g., \foreignlanguage{hebrew}{נִלְקָ֑חָה} \textit{it was taken}, \foreignlanguage{hebrew}{נִלְחָ֑מוּ} \textit{they fought}, \foreignlanguage{hebrew}{יִפָּרֵ֑דוּ} \textit{they will separate from each other}.
	
	At times, the vowel between R\textsubscript{1} and R\textsubscript{2} in pausal forms is /a/ instead of /ē/, e.g., \foreignlanguage{hebrew}{וַיִּגָּמַ֑ל} \textit{and he was weaned} (Gen 21:8).
	
	In the suffix conjugation, forms with paragogic Nun may occur, e.g., \foreignlanguage{hebrew}{יִכָּרֵתוּן} \textit{they will be cut off}.
	
	The verb \foreignlanguage{hebrew}{לקח}, which behaves like a I\,\textit{n} verb in the Qal, has strong forms in all forms of the Ni., e.g., \foreignlanguage{hebrew}{נִלְקַח} \textit{it was taken}, \foreignlanguage{hebrew}{אֶלָּקַח} \textit{I will be taken}.
	
	Verbs I\,\textit{r} do not have gemination of the first root consonant in the prefix conjugation, the imperative and the infinitives. Instead of \textit{ḥireq} these verbs have \textit{ṣere} in the prefix syllable of the prefix conjugation, the imperative and the infinitive construct, e.g., \foreignlanguage{hebrew}{וַיֵּרָדַם} \textit{and he fell asleep} (Jonah 1:5).
	
	The primitive form of the suffix conjugation of the strong verb *\textit{naqṭala} is not preserved in the strong verb. Instead the form changed to *\textit{niqṭala} which explains the forms of the I\,gutt.\ verbs (see below). The primitive form *\textit{naqṭala} becomes relevant in the Niphal forms of I\,\textit{y}, II\,\textit{w/y} and II\,gem.\ verbs. The same applies to the participle.
	
	% The preceding paragraph is based on Blau, Phonology and Morphology, p. 228 except for the application of the law on attenuation a > i.
	
	% Another form that may be worth mentioning: הֵרָגְעִי (Jer 47:6)
	
	
	\subsection{The Meaning of Verbs in the Niphal}
	
	Originally, the Niphal binyan had reflexive-middle meaning but assumed passive meaning over time when the original Qal passive, of which only vestiges are left in Biblical Hebrew, was being phased out. As a consequence, the Niphal often serves as a passive to the Qal (and sometimes the Hiphil).
	
	\begin{enumerate}[noitemsep]
		\item Passive: \foreignlanguage{hebrew}{לקח} Qal \textit{take}, Ni.\ \textit{to be taken} (The subject is acted upon.)
		\item Reflexive meaning: \foreignlanguage{hebrew}{שׁמר} Qal \textit{keep}, Ni.\ \textit{keep oneself, be on one's guard} (The subject affects itself with the verbal action.)
		\item Middle meaning: \foreignlanguage{hebrew}{שׁאל} Qal \textit{ask}, Ni.\ \textit{ask for oneself}; \foreignlanguage{hebrew}{נקם} Qal \textit{avenge}, Ni.\ \textit{avenge oneself} (The subject acts in its own interest.)
		\item Reciprocal: \foreignlanguage{hebrew}{שׁפט} Qal \textit{judge}, Ni.\ \textit{enter into controversy} (Two or more entities are involved in the \enquote{action} of the verb.)
		\item Tolerative: \foreignlanguage{hebrew}{דרשׁ} Qal \textit{seek, inquire}, Ni.\ \textit{to let oneself be inquired of}
		\item No obvious Niphal meaning: \foreignlanguage{hebrew}{שׁבע} Ni.\ \textit{to take an oath} (no Qal equivalent)
	\end{enumerate}
	
	If the meaning of a verb in the Qal (or sometimes the Hiphil) is known, these semantic categories should be considered for  the translation of the verb in the Niphal. A lexicon will help in further specifying the meaning.
	
	
	\subsection{Guttural Verbs in the Niphal}
	Niphal forms of guttural verbs show the expected differences with the regular verb:
	
	\begin{itemize}[noitemsep]
		\item[--] change of vowel quality to /a/ or a vowel closer to /a/
		\item[--] avoidance of silent \textit{šwa} in the prefix syllable \textit{ni-} of the suffix conjugation and use of \textit{ḥatef šwa} instead
		\item[--] use of \textit{ḥatef šwa} instead vocal \textit{šwa}
		\item[--] loss of the gemination of R\textsubscript{1} in the prefix conjugation, the impv. and the infinitives and change of the preceding vowel form /i/ to /ē/ (this applies to all gutturals including \foreignlanguage{hebrew}{ה} and \foreignlanguage{hebrew}{ח})
	\end{itemize}
	
	% The use of the verb שבע instead of שלח should be considered because שלח is only attested once in the Niphal. On the other hand, שלח is a useful verb because it is used in paradigms for the Qal and the Piel. The verbs שחט and עזב have only three and nine occurrences, respectively. Are their more frequent gutturals that could be used?
	
	\begin{center}
		\begin{longtable}{|lll|r|r|r|}
			\hline
			& & & I gutt. & II gutt. & III gutt. \\
			\hline
			\endhead
			\hline
			\endfoot
			SC & sg. & 3 m. & \foreignlanguage{hebrew}{נֶעֱזַב} & \foreignlanguage{hebrew}{נִשְׁחַט} & \foreignlanguage{hebrew}{נִשְׁלַח} \\
			& & 3 f. & \foreignlanguage{hebrew}{נֶעֶזְבָה} & \foreignlanguage{hebrew}{נִשְׁחֲטָה} & \foreignlanguage{hebrew}{נִשְׁלְחָה} \\
			& & 2 m. & \foreignlanguage{hebrew}{נֶעֱזַבְתָּ} & \foreignlanguage{hebrew}{נִשְׂחַטְתָּ} & \foreignlanguage{hebrew}{נִשְׁלַחְתָּ} \\
			& & 2 f. & \foreignlanguage{hebrew}{נֶעֱזַבְתְּ} & \foreignlanguage{hebrew}{נִשְׁחַטְתְּ} & \foreignlanguage{hebrew}{נִשְׁלַחְתְּ} \\
			& & 1 c. & \foreignlanguage{hebrew}{נֶעֱזַבְתִּי} & \foreignlanguage{hebrew}{נִשְׁחַטְתִּי} & \foreignlanguage{hebrew}{נִשְׁלַחְתִּי} \\
			\hline
			& pl. & 3 c. & \foreignlanguage{hebrew}{נֶעֶזְבוּ} & \foreignlanguage{hebrew}{נִשְׁחֲטוּ} & \foreignlanguage{hebrew}{נִשְׁלְחוּ} \\
			& & 2 m. & \foreignlanguage{hebrew}{נֶעֱזַבְתֶּם} & \foreignlanguage{hebrew}{נִשְׁחַטְתֶּם} & \foreignlanguage{hebrew}{נִשְׁלַחְתֶּם} \\
			& & 2 f. & \foreignlanguage{hebrew}{נֶעֱזַבְתֶּן} & \foreignlanguage{hebrew}{נִשְׁחַטְתֶּן} & \foreignlanguage{hebrew}{נִשְׁלַחְתֶּן} \\
			& & 1 c. & \foreignlanguage{hebrew}{נֶעֱזַ֫בְנוּ} & \foreignlanguage{hebrew}{נִשְׁחַ֫טְנוּ} & \foreignlanguage{hebrew}{נִשְׁלַ֫חְנוּ} \\
			\hline
			PC & sg. & 3 m. & \foreignlanguage{hebrew}{יֵעָזֵב} & \foreignlanguage{hebrew}{יִשָּׁחֵט} & \foreignlanguage{hebrew}{יִשָּׁלַח} \\
			& & 3 f. & \foreignlanguage{hebrew}{תֵּעָזֵב} & \foreignlanguage{hebrew}{תִּשָּׁחֵט} & \foreignlanguage{hebrew}{תִּשָּׁלַח} \\
			& & 2 m. & \foreignlanguage{hebrew}{תֵּעָזֵב} & \foreignlanguage{hebrew}{תִּשָּׁחֵט} & \foreignlanguage{hebrew}{תִּשָּׁלַח} \\
			& & 2 f. & \foreignlanguage{hebrew}{תֵּעָֽזְבִי} & \foreignlanguage{hebrew}{תִּשָּׁחֲטִי} & \foreignlanguage{hebrew}{תִּשָּֽׁלְחִי} \\
			& & 1 c. & \foreignlanguage{hebrew}{אֵעָזֵב} & \foreignlanguage{hebrew}{אֶשָּׁחֵט} & \foreignlanguage{hebrew}{אֶשָּׁלַח} \\
			& pl. & 3 m. & \foreignlanguage{hebrew}{יֵעָזְבוּ} & \foreignlanguage{hebrew}{יִשָּׁחֲטוּ} & \foreignlanguage{hebrew}{יִשָּׁלְחוּ} \\
			& & 3 m. & \foreignlanguage{hebrew}{תֵּעָזַ֫בְנָה} & \foreignlanguage{hebrew}{תִּשָׁחַ֫טְנָה} & \foreignlanguage{hebrew}{תִּשָּׁלַ֫חְנָה} \\
			& & 2 m. & \foreignlanguage{hebrew}{תֵּעָֽזְבוּ} & \foreignlanguage{hebrew}{תִּשָּׁחֲטוּ} & \foreignlanguage{hebrew}{תִּשָּֽׁלְחוּ} \\
			& & 2 f. & \foreignlanguage{hebrew}{תֵּעָזַ֫בְנָה} & \foreignlanguage{hebrew}{תִּשָׁחַ֫טְנָה} & \foreignlanguage{hebrew}{תִּשָּׁלַ֫חְנָה} \\
			& & 1 c. & \foreignlanguage{hebrew}{נֵעָזֵב} & \foreignlanguage{hebrew}{נִשָּׁחֵט} & \foreignlanguage{hebrew}{נִשָּׁלַח} \\
			\hline
			Jussive & sg. & 3 m. & \foreignlanguage{hebrew}{יֵעָזֵב} & \foreignlanguage{hebrew}{יִשָּׁחֵט} & \foreignlanguage{hebrew}{יִשָּׁלַח} \\
			\textit{wayyiqṭol} & sg  & 3 m. & \foreignlanguage{hebrew}{וַיֵּעָזֵב} & \foreignlanguage{hebrew}{וַיִּשָּׁחֵט} & \foreignlanguage{hebrew}{וַיִּשָּׁלַח} \\
			\hline
			Impv. & sg. & 2 m. & \foreignlanguage{hebrew}{הֵעָזֵב} & \foreignlanguage{hebrew}{הִשָּׁחֵט} & \foreignlanguage{hebrew}{הִשָּׁלַח} \\
			& & 2 f. & \foreignlanguage{hebrew}{הֵעָֽזְבִי} & \foreignlanguage{hebrew}{הִשָּׁחֲטִי} & \foreignlanguage{hebrew}{הִשָּֽׁלְחִי} \\
			& pl. & 2 m. & \foreignlanguage{hebrew}{הֵעָֽזְבוּ} & \foreignlanguage{hebrew}{הִשָּׁחֲטוּ} & \foreignlanguage{hebrew}{הִשָּֽׁלְחוּ} \\
			& & 2 f. & \foreignlanguage{hebrew}{הֵעָזַ֫בְנָה} & \foreignlanguage{hebrew}{הִשָּׁחַטְנָה} & \foreignlanguage{hebrew}{הִשָּׁלַחְנָה} \\
			\pagebreak
			\hline
			Inf.\ cs. & & & \foreignlanguage{hebrew}{הֵעָזֵב} & \foreignlanguage{hebrew}{הִשָּׁחֵט} & \foreignlanguage{hebrew}{הִשָּׁלַח} \\
			Inf.\ abs. & & & \foreignlanguage{hebrew}{הֵעָזֹב} & \foreignlanguage{hebrew}{הִשָּׁחֹט} & \foreignlanguage{hebrew}{הִשָּׁלֵחַ} \\
			& & & \foreignlanguage{hebrew}{נַעֲזֹב} & \foreignlanguage{hebrew}{נִשְׁחֹט} & \foreignlanguage{hebrew}{נִשְׁלֹחַ} \\
			\hline
			Part. & m. & sg. & \foreignlanguage{hebrew}{נֶעֱזָב} & \foreignlanguage{hebrew}{נִשְׁחָט} & \foreignlanguage{hebrew}{נִשְׁלָח} \\
			& & pl. & \foreignlanguage{hebrew}{נֶעֱזָבִים} & \foreignlanguage{hebrew}{נִשְׁחָטִים} & \foreignlanguage{hebrew}{נִשְׁלָחִים} \\
			& f. & sg. & \foreignlanguage{hebrew}{נֶעֱזֱ֫בֶת} & \foreignlanguage{hebrew}{נִשְׁחֶ֫טֶת} & \foreignlanguage{hebrew}{נִשְׂלַ֫חַת} \\
			& & pl. & \foreignlanguage{hebrew}{נֶעֱזָבוֹת} & \foreignlanguage{hebrew}{נִשְׁחָטוֺת} & \foreignlanguage{hebrew}{נִשְׁלָחוֺת} \\
		\end{longtable}
	\end{center}
	
	
	% Cf. Zech 11:9 for a form of the participle fem. sg. of a verb II gutt. 
	
	\noindent \textbf{Notes}
	\nopagebreak
	
	\noindent Retraction of stress may occur in prefix conjugation and imperative forms without suffix, e.g., \foreignlanguage{hebrew}{וַיֵּאָ֫סֶף} \textit{and he was gathered} (Gen 25:8), \foreignlanguage{hebrew}{וַיִּנָּ֫חֶם} \textit{and he regretted} (Gen 6:6).
	
	The original vowel /ē/ between R\textsubscript{2} and R\textsubscript{3} is restored in pausal forms of the prefix conjugation without suffix, e.g., \foreignlanguage{hebrew}{תִּשָּׁבֵ֑עַ} \textit{you shall swear} (Deut 6:13).
	
	
	
	\section{Oaths in Biblical Hebrew}
	
	Oaths and curses are frequently introduced either with the conditional conjunction \foreignlanguage{hebrew}{אִם} for a negative statement (\enquote{cer­tainly not}) (1\,Sam 3:14) or with \foreignlanguage{hebrew}{אִם לֹא} or \foreignlanguage{hebrew}{כִּי} for a positive statement (\enquote{cer­tainly}) (Josh 14:9; 1\,Kgs 1:13). This use of \foreignlanguage{hebrew}{אִם}, \foreignlanguage{hebrew}{אִם לֹא} or \foreignlanguage{hebrew}{כִּי} is conditioned by the self-imprecation \foreignlanguage{hebrew}{כֹּה־יַעֲשֶׂה־לִּי אֱלֹהִים וְכֹה יוֹסִף} \textit{May God do so to me and add more also} (as, e.g., in 2\,Kgs 6:31) which means that the person who takes an oath is willing to accept divine punishment in the case of non-realization of the oath.\footnote{\space The verbal form \foreignlanguage{hebrew}{יוֹסִף} is a Hiphil jussive form of the verb \foreignlanguage{hebrew}{יסף} meaning \textit{to do again, continue doing something}.} The use of \foreignlanguage{hebrew}{אִם} or \foreignlanguage{hebrew}{אִם לֹא} in oaths or curses without the self-imprecation depends on its suppression or omission although it is implied. As oaths are also taken by God (cf.\ 1\,Sam 3:14 above) \enquote{the conscious­ness of the real meaning of the formula was lost at an early period, and \foreignlanguage{hebrew}{אִם לֹא} simply came to express verily, \foreignlanguage{hebrew}{אִם} verily not} (GKC §\,149\,b).\footnote{\space A different explanation is offered in JM §\,165.}
	
	\vspace{0.5cm}
	
	\begin{longtable}{>{\raggedleft}p{0.35\linewidth} p{0.55\linewidth}}
		\foreignlanguage{hebrew}{וְלָכֵן נִשְׁבַּעְתִּי לְבֵית עֵלִי אִֽם־יִתְכַּפֵּר עֲוֺן בֵּית־עֵלִי בְּזֶבַח וּבְמִנְחָה עַד־עוֹלָם} & \textit{Therefore I have sworn to the house of Eli that the iniquity of the house of Eli shall not be atoned for by sacrifice or offering forever} (1\,Sam 3:14) \\
		\foreignlanguage{hebrew}{וַיִּשָּׁבַע מֹשֶׁה בַּיּוֹם הַהוּא לֵאמֹר אִם־לֹא הָאָרֶץ אֲשֶׁר דָּרְכָה רַגְלְךָ בָּהּ לְךָ תִהְיֶה לְנַחֲלָה וּלְבָנֶיךָ עַד־עוֹלָם} & \textit{And Moses swore on that day,\enquote{The land on which your foot treads shall be belong to you as inheritance and to your sons forever}} (Josh 14:9) \\
		\foreignlanguage{hebrew}{הֲלֹא־אַתָּה אֲדֹנִי הַמֶּלֶךְ נִשְׁבַּעְתָּ לַאֲמָתְךָ לֵאמֹר כִּי־שְׁלֹמֹה בְנֵךְ יִמְלֹךְ אַחֲרַי וְהוּא יֵשֵׁב עַל־כִּסְאִי} & \textit{Did you not, my lord the king, swear to your maidservant, \enquote{Your son Solomon shall reign as king after me and he shall sit on my throne?}} (1\,Kgs 1:13) \\
		\foreignlanguage{hebrew}{וַיֹּאמֶר כֹּה־יַעֲשֶׂה־לִּי אֱלֹהִים וְכֹה יוֹסִף אִם־יַעֲמֹד רֹאשׁ אֱלִישָׁע בֶּן־שָׁפָט עָלָיו הַיּוֹם} & \textit{And he said, \enquote{May God do so to me and more also, if the head of Elisha son of Shaphan remains on him today}} (2\,Kgs 6:31) \\
	\end{longtable}
	
	\vspace{0.5cm}
	
	At times, the oath can be expressed as simple statements without \foreignlanguage{hebrew}{אִם} or \foreignlanguage{hebrew}{אִם לֹא} (Judg 21:1).
	
	\vspace{0.5cm}
	
	\begin{tabular}{>{\raggedleft}p{0.35\linewidth} p{0.55\linewidth}}
		\foreignlanguage{hebrew}{וְאִישׁ יִשְׂרָאֵל נִשְׁבַּע בַּמִּצְפָּה לֵאמֹר אִישׁ מִמֶּנּוּ לֹא־יִתֵּן בִּתּוֹ לְבִנְיָמִן לְאִשָּׁה} & \textit{Now, the men of Israel had sworn in Mizpah, \enquote{No one from us shall give his daughter to Benjamin as a wife}} (Judg 21:1) \\
	\end{tabular}
	
	\vspace{0.5cm}
	
	An exclamatory formula with a form of \foreignlanguage{hebrew}{חַי} meaning \textit{life} in this context may be used at the beginning of an oath. If followed by \foreignlanguage{hebrew}{יהוה} (pronounced \textit{ʾădōnāy}) or the independent personal pronoun \foreignlanguage{hebrew}{אֲנִי} \textit{I}, the form is \foreignlanguage{hebrew}{חַי} (1\,Sam 19:6; Num 14:28). Followed by other nouns the form is \foreignlanguage{hebrew}{חֵי} (1\,Sam 17:55; Gen 42:15).
	
	\vspace{0.5cm}
	
	\begin{longtable}{>{\raggedleft}p{0.35\linewidth} p{0.55\linewidth}}
		\foreignlanguage{hebrew}{וַיִּשָּׁבַע שָׁאוּל חַי־יְהוָה אִם־יוּמָת} & \textit{And Saul swore, \enquote{As the Lord lives, he shall not be put to death}} (1\,Sam 19:6) \\
		\foreignlanguage{hebrew}{אֱמֹר אֲלֵהֶם חַי־אָנִי נְאֻם־יְהוָה אִם־לֹא כַּאֲשֶׁר דִּבַּרְתֶּם בְּאָזְנָי כֵּן אֶעֱשֶׂה לָכֶם} & \textit{Say to them, \enquote{As I live, says the Lord, I will certainly do to you as you have spoken in my hearing}} (Num 14:28) \\
		\foreignlanguage{hebrew}{וַיֹּאמֶר אַבְנֵר חֵי־נַפְשְׁךָ הַמֶּלֶךְ אִם־יָדָֽעְתִּי} & \textit{And Abner said, \enquote{By your life, O king, I do not know}} (1\,Sam 17:55) \\
		\foreignlanguage{hebrew}{בְּזֹאת תִּבָּחֵנוּ חֵי פַרְעֹה אִם־תֵּצְאוּ מִזֶּה כִּי אִם־בְּבוֹא אֲחִיכֶם הַקָּטֹן הֵנָּה} & \textit{By this you shall be tested: As Pharaoh lives, you shall not go out from here unless your youngest brother comes here} (Gen 42:15) \\
	\end{longtable}
	
	\vspace{0.5cm}
	
	Another way of introducing an oath is the interjection \foreignlanguage{hebrew}{חָלִ֫ילָה} \textit{far be it!} (literally \textit{to the profane!}). Different constructions following \foreignlanguage{hebrew}{חָלִ֫ילָה} are possible; the most common one is the construction with the preposition \foreignlanguage{hebrew}{לְ} and an enclitic personal pronoun and a following inf.\ cs.\ with the preposition \foreignlanguage{hebrew}{מִן} (Gen 44:17).
	
	\vspace{0.5cm}
	
	\begin{tabular}{>{\raggedleft}p{0.35\linewidth} p{0.55\linewidth}}
		\foreignlanguage{hebrew}{וַיֹּאמֶר חָלִילָה לִּי מֵעֲשׂוֹת זֹאת} & \textit{And he said, \enquote{Far be it from me that I should do this}}, i.e., \textit{I will not do this} (Gen 44:17) \\
	\end{tabular}
	
	\vspace{0.5cm}
	
	The deity by whom someone takes an oaths is expressed by a prepositional phrase with \foreignlanguage{hebrew}{בְּ} (1\,Sam 28:10).
	
	\vspace{0.5cm}
	
	\begin{tabular}{>{\raggedleft}p{0.35\linewidth} p{0.55\linewidth}}
		\foreignlanguage{hebrew}{וַיִּשָּׁבַע לָהּ שָׁאוּל בַּיהוָה לֵאמֹר חַי־יְהוָה אִם־יִקְּרֵךְ עָוֺן בַּדָּבָר הַזֶּה} & \textit{And Saul swore to her by the Lord, \enquote{As the Lord lives, no punishment shall happen to you because of this matter}} (1\,Sam 28:10) \\
	\end{tabular}
	
	
	\section{Exercises}
	
	\subsection{Translation of Verbal Forms}
	Translate the following verbal forms. Identify the gender (masc., fem., comm.) and number (sg., pl.) of forms of which the English translation is ambiguous (i.e., \textit{you}, \textit{they}). Mark the stressed syllable if stress is not on the last syllable.
	
	\hspace{0.5cm}
	
	\selectlanguage{hebrew}
	
	\noindent
	1~~\foreignlanguage{hebrew}{נִלְחַם}  \hspace{0.3cm}
	2~~\foreignlanguage{hebrew}{נִלְחֲמוּ}  \hspace{0.3cm}
	3~~\foreignlanguage{hebrew}{נִלְחָמוּ}  \hspace{0.3cm}
	4~~\foreignlanguage{hebrew}{נִלְחַמְתִּי}  \hspace{0.3cm}
	5~~\foreignlanguage{hebrew}{יִלָּחֵם}  \hspace{0.3cm}
	6~~\foreignlanguage{hebrew}{וַיִּלָּחֶם}  \hspace{0.3cm}
	7~~\foreignlanguage{hebrew}{תִלָּחֲמוּ}  \hspace{0.3cm}
	8~~\foreignlanguage{hebrew}{נִלָּחֵם}  \hspace{0.3cm}
	9~~\foreignlanguage{hebrew}{תִלָּחֲמוּן}  \hspace{0.3cm}
	10~~\foreignlanguage{hebrew}{הִלָּחֲמוּ}  \hspace{0.3cm}
	11~~\foreignlanguage{hebrew}{נִמְלַטְתִּי}  \hspace{0.3cm}
	12~~\foreignlanguage{hebrew}{נִמְלְטוּ }  \hspace{0.3cm}
	13~~\foreignlanguage{hebrew}{אִמָּלֵט}  \hspace{0.3cm}
	14~~\foreignlanguage{hebrew}{תִּמָּלֵט}  \hspace{0.3cm}
	15~~\foreignlanguage{hebrew}{נִשְׁבַּעְתָּ}  \hspace{0.3cm}
	16~~\foreignlanguage{hebrew}{נִשְׁבַּעְתֶּם}  \hspace{0.3cm}
	17~~\foreignlanguage{hebrew}{נִשְׁבַּעְנוּ}  \hspace{0.3cm}
	18~~\foreignlanguage{hebrew}{נִשְׁבַּעְתִּי}  \hspace{0.3cm}
	19~~\foreignlanguage{hebrew}{תִּשָּׁבְעוּ}  \hspace{0.3cm}
	20~~\foreignlanguage{hebrew}{וַיִּשָּׁבַע}  \hspace{0.3cm}
	
	\selectlanguage{english}
	
	
	
	
	\subsection{Translation of Sentences}
	Translate the following sentences from the Hebrew Bible. Names of persons and geographical names in these sentences: \foreignlanguage{hebrew}{אַבְנֵר}, \foreignlanguage{hebrew}{אוּרִיָּה}, \foreignlanguage{hebrew}{אֲחַזְיָהוּ},  \foreignlanguage{hebrew}{אֵלִיָּהוּ}, \foreignlanguage{hebrew}{בִּנְיָמִין}, \foreignlanguage{hebrew}{גַּת}, \foreignlanguage{hebrew}{דָּוִד}, \foreignlanguage{hebrew}{חֲזָאֵל}, \foreignlanguage{hebrew}{יְהוֹאָשׁ}, \foreignlanguage{hebrew}{יְהוּדָה}, \foreignlanguage{hebrew}{יְהוֺרָם}, \foreignlanguage{hebrew}{יְהוֹשָׁפָט}, \foreignlanguage{hebrew}{יוֹאָב}, \foreignlanguage{hebrew}{יְפֻנֶּה}, \foreignlanguage{hebrew}{כָּלֵב}, \foreignlanguage{hebrew}{קִישׁוֹן}, \foreignlanguage{hebrew}{שָׁאוּל}, \foreignlanguage{hebrew}{רְחַבְעָם}, \foreignlanguage{hebrew}{שְׁלֹמֹה}, \foreignlanguage{hebrew}{שְׁמַעְיָה}.
	
	\vspace{0.5cm}
	
	\selectlanguage{hebrew}
	\noindent
	1~~\foreignlanguage{hebrew}{וַיִּשְׁמַע יְהוָה אֶת־קוֹל דִּבְרֵיכֶם וַיִּקְצֹף}\LTRfootnote{\space \foreignlanguage{hebrew}{וַיִּקְצֹף} verb \foreignlanguage{hebrew}{קצף} Q.\ \textit{to be angry, to become angry}} \foreignlanguage{hebrew}{וַיִּשָּׁבַע לֵאמֹר אִם־יִרְאֶה אִישׁ בָּאֲנָשִׁים הָאֵלֶּה הַדּוֹר הָרָע הַזֶּה אֵת הָאָרֶץ הַטּוֹבָה אֲשֶׁר נִשְׁבַּעְתִּי לָתֵת לַאֲבֹתֵיכֶם זוּלָתִי}\LTRfootnote{\space \foreignlanguage{hebrew}{זוּלָתִי} \textit{except}} \foreignlanguage{hebrew}{כָּלֵב בֶּן־יְפֻנֶּה הוּא יִרְאֶנָּה וְלוֹ־אֶתֵּן אֶת־הָאָרֶץ אֲשֶׁר דָּרַךְ־בָּהּ וּלְבָנָיו} \hspace{0.3cm}
	2~~\foreignlanguage{hebrew}{וַיֹּאמֶר לַאֲנָשָׁיו חָלִילָה לִּי מֵיהוָה אִם־אֶעֱשֶׂה אֶת־הַדָּבָר הַזֶּה לַאדֹנִי לִמְשִׁיחַ}\LTRfootnote{\space \foreignlanguage{hebrew}{מָשִׁיחַ} \textit{anointed}} \foreignlanguage{hebrew}{יְהוָה לִשְׁלֹחַ יָדִי בּוֹ כִּי־מְשִׁיחַ יְהוָה הוּא} \hspace{0.3cm}
	3~~\foreignlanguage{hebrew}{וַיֹּאמֶר אוּרִיָּה אֶל־דָּוִד הָאָרוֹן וְיִשְׂרָאֵל וִיהוּדָה יֹשְׁבִים בַּסֻּכּוֹת}\LTRfootnote{\space \foreignlanguage{hebrew}{סֻכָּה} \textit{hut, booth} (as a temporary shelter)} \foreignlanguage{hebrew}{וַאדֹנִי יוֹאָב וְעַבְדֵי אֲדֹנִי עַל־פְּנֵי הַשָּׂדֶה חֹנִים וַאֲנִי אָבוֹא}\LTRfootnote{\space \foreignlanguage{hebrew}{וַאֲנִי אָבוֹא} should be translated as a rhetorical question \textit{Shall I then \dots}} \foreignlanguage{hebrew}{אֶל־בֵּיתִי לֶאֱכֹל וְלִשְׁתּוֹת וְלִשְׁכַּב עִם־אִשְׁתִּי חַיֶּךָ וְחֵי נַפְשֶׁךָ אִם־אֶעֱשֶׂה אֶת־הַדָּבָר הַזֶּה} \hspace{0.3cm}
	4~~\foreignlanguage{hebrew}{וַיֹּאמֶר אֵלִיָּהוּ לָהֶם תִּפְשׂוּ אֶת־נְבִיאֵי הַבַּעַל אִישׁ אַל־יִמָּלֵט מֵהֶם וַיִּתְפְּשׂוּם וַיּוֹרִדֵם}\LTRfootnote{\space \foreignlanguage{hebrew}{וַיּוֹרִדֵם} \textit{and he brought them down} (Hiphil of \foreignlanguage{hebrew}{ירד})} \foreignlanguage{hebrew}{אֵלִיָּהוּ אֶל־נַחַל קִישׁוֹן וַיִּשְׁחָטֵם שָׁם} \hspace{0.3cm}
	5~~\foreignlanguage{hebrew}{וְכִרְאוֹת שָׁאוּל אֶת־דָּוִד יֹצֵא לִקְרַאת הַפְּלִשְׁתִּי אָמַר אֶל־אַבְנֵר שַׂר הַצָּבָא בֶּן־מִי־זֶה}\LTRfootnote{\space \foreignlanguage{hebrew}{זֶה} often follows an interrogative pronoun or adverb \enquote{without any notable change in meaning} (JM §\,143\,\textit{g})} \foreignlanguage{hebrew}{הַנַּעַר אַבְנֵר וַיֹּאמֶר אַבְנֵר חֵי־נַפְשְׁךָ הַמֶּלֶךְ}\LTRfootnote{\space \foreignlanguage{hebrew}{הַמֶּלֶךְ} \textit{O king}. Nouns used in a direct address are used with the article.} \foreignlanguage{hebrew}{אִם־יָדָעְתִּי׃} \hspace{0.3cm}
	6~~\foreignlanguage{hebrew}{וַיִּתֵּן יְהוָה לְיִשְׂרָאֵל אֶת־כָּל־הָאָרֶץ אֲשֶׁר נִשְׁבַּע לָתֵת לַאֲבוֹתָם וַיִּרָשׁוּהָ וַיֵּשְׁבוּ בָהּ}  \hspace{0.3cm}
	7~~\foreignlanguage{hebrew}{וְעַתָּה יְראוּ אֶת־יְהוָה וְעִבְדוּ אֹתוֹ בְּתָמִים}\LTRfootnote{\space \foreignlanguage{hebrew}{בְּתָמִים} \textit{in integrity}} \foreignlanguage{hebrew}{וּבֶאֱמֶת וְהָסִ֫ירוּ}\LTRfootnote{\space \foreignlanguage{hebrew}{הָסִ֫ירוּ} \textit{remove!} (impv. of \foreignlanguage{hebrew}{סור} Hiphil)} \foreignlanguage{hebrew}{אֶת־אֱלֹהִים אֲשֶׁר עָבְדוּ אֲבוֹתֵיכֶם בְּעֵבֶר הַנָּהָר וּבְמִצְרַיִם וְעִבְדוּ אֶת־יְהוָה וְאִם רַע בְּעֵינֵיכֶם לַעֲבֹד אֶת־יְהוָה בַּחֲרוּ לָכֶם הַיּוֹם אֶת־מִי תַעֲבֹדוּן אִם אֶת־אֱלֹהִים אֲשֶׁר־עָבְדוּ אֲבוֹתֵיכֶם אֲשֶׁר מֵעֵבֶר הַנָּהָר וְאִם אֶת־אֱלֹהֵי הָאֱמֹרִי אֲשֶׁר אַתֶּם יֹשְׁבִים בְּאַרְצָם וְאָנֹכִי וּבֵיתִי נַעֲבֹד אֶת־יְהוָה וַיַּעַן הָעָם וַיֹּאמֶר חָלִילָה לָּנוּ מֵעֲזֹב אֶת־יְהוָה לַעֲבֹד אֱלֹהִים אֲחֵרִים}  \hspace{0.3cm}
	8~~\foreignlanguage{hebrew}{וַיְהִי דְּבַר הֽ͏ָאֱלֹהִים אֶל־שְׁמַעְיָה אִישׁ־הָאֱלֹהִים לֵאמֹר אֱמֹר אֶל־רְחַבְעָם בֶּן־שְׁלֹמֹה מֶלֶךְ יְהוּדָה וְאֶל־כָּל־בֵּית יְהוּדָה וּבִנְיָמִין וְיֶתֶר הָעָם לֵאמֹר כֹּה אָמַר יְהוָה לֹא־תַעֲלוּ וְלֹא־תִלָּחֲמוּן עִם־אֲחֵיכֶם בְּנֵי־יִשְׂרָאֵל. שׁוּבוּ אִישׁ לְבֵיתוֹ כִּי מֵאִתִּי נִהְיָה}\LTRfootnote{\space \foreignlanguage{hebrew}{נִהְיָה} \textit{it happened} (\foreignlanguage{hebrew}{היה} Niphal)} \foreignlanguage{hebrew}{הַדָּבָר הַזֶּה וַיִּשְׁמְעוּ אֶת־דְּבַר יְהוָה וַיָּשֻׁבוּ לָלֶכֶת כִּדְבַר יְהוָה} \hspace{0.3cm}
	9~~\foreignlanguage{hebrew}{אָז יַעֲלֶה חֲזָאֵל מֶלֶךְ אֲרָם וַיִּלָּחֶם עַל־גַּת וַֽיִּלְכְּדָהּ וַיָּשֶׂם חֲזָאֵל פָּנָיו לַעֲלוֹת עַל־יְרוּשָׁלָֽ͏ִם׃ וַיִּקַּח יְהוֹאָשׁ מֶֽלֶךְ־יְהוּדָה אֵת כָּל־הַקֳּדָשִׁים אֲשֶׁר־הִקְדִּישׁוּ}\LTRfootnote{\space \foreignlanguage{hebrew}{הִקְדִּישׁוּ} \textit{they dedicated, they consecrated}} \foreignlanguage{hebrew}{יְהוֹשָׁפָט וִיהוֹרָם וַאֲחַזְיָהוּ אֲבֹתָיו מַלְכֵי יְהוּדָה וְאֶת־קֳדָשָׁיו וְאֵת כָּל־הַזָּהָב הַנִּמְצָא בְּאֹצְרוֹת בֵּית־יְהוָה וּבֵית הַמֶּלֶךְ וַיִּשְׁלַח לַֽחֲזָאֵל מֶלֶךְ אֲרָם וַיַּעַל מֵעַל יְרוּשָׁלָֽ͏ִם׃} \hspace{0.3cm}
	10~~\foreignlanguage{hebrew}{וַיֹּאמֶר חָלִילָה לִּי מֵעֲשׂוֹת זֹאת. הָאִישׁ אֲשֶׁר נִמְצָא הַגָּבִיעַ}\LTRfootnote{\space \foreignlanguage{hebrew}{גָּבִיעַ} \textit{cup, bowl}} \foreignlanguage{hebrew}{בְּיָדוֹ הוּא יִהְיֶה־לִּי עָבֶד וְאַתֶּם עֲלוּ לְשָׁלוֹם אֶל־אֲבִיכֶם} \hspace{0.3cm}
	\selectlanguage{english}
	
	
	
	\section{Hebrew Reading: Genesis 12:10--14}
	Translate Gen 12:10--14 with the help of notes below the text. Names of persons and places are easily identifiable; if not, they are included in the notes.
	
	\vspace{0.5cm}
	
	\selectlanguage{hebrew}
	
	\noindent
	\textsuperscript{10}~\foreignlanguage{hebrew}{וַיְהִי רָעָב בָּאָ֑רֶץ וַיֵּרֶד אַבְרָם מִצְרַ֙יְמָה֙  לָגוּר שָׁ֔ם כִּי־כָבֵד הָרָעָב בָּאָֽרֶץ׃} \hspace{0.3cm}
	\textsuperscript{11}~\foreignlanguage{hebrew}{וַיְהִי כַּאֲשֶׁר הִקְרִיב לָבוֹא מִצְרָ֑יְמָה וַיֹּאמֶר אֶל־שָׂרַי אִשְׁתּ֔וֹ הִנֵּה־נָא יָדַ֔עְתִּי כִּי אִשָּׁה יְפַת־מַרְאֶה אָֽתְּ׃} \hspace{0.3cm}
	\textsuperscript{12}~\foreignlanguage{hebrew}{וְהָיָה כִּי־יִרְאוּ אֹתָךְ הַמִּצְרִ֔ים וְאָמְרוּ אִשְׁתּוֹ זֹ֑את וְהָרְגוּ אֹתִי וְאֹתָךְ יְחַיּֽוּ׃} \hspace{0.3cm}
	\textsuperscript{13}~\foreignlanguage{hebrew}{אִמְרִי־נָא אֲחֹתִי אָ֑תְּ לְמַעַן יִיטַב־לִי בַעֲבוּרֵ֔ךְ וְחָיְתָה נַפְשִׁי בִּגְלָלֵֽךְ׃} \hspace{0.3cm}
	\textsuperscript{14}~\foreignlanguage{hebrew}{וַיְהִי כְּבוֹא אַבְרָם מִצְרָ֑יְמָה וַיִּרְאוּ הַמִּצְרִים אֶת־הָאִשָּׁ֔ה כִּֽי־יָפָה הִיא מְאֹֽד׃} \hspace{0.3cm}
	\vspace{1cm}
	
	\selectlanguage{english}
	
	\hspace*{-0.5cm}\begin{longtable}{p{0.075\linewidth} p{0.1\linewidth}p{0.725\linewidth}}
		12:10 & \foreignlanguage{hebrew}{מִצְרַ֙יְמָה֙} & The accent here is called Pašta. It is put on the last syllable of the word. If the last syllable is not stressed, it is repeated on the stressed syllable. \\
		&  \foreignlanguage{hebrew}{כָּבֵד} & \textit{heavy, oppressing, severe} \\
		12:11 & \foreignlanguage{hebrew}{הִקְרִיב} & \textit{he drew near, approached} (the intransitive meaning of the Hiphil form here is unusual) \\
		& \foreignlanguage{hebrew}{יָפֶה} & \textit{beautiful, fair} (adj.) \\
		12:12 &  \foreignlanguage{hebrew}{מִצְרִי} & \textit{Egyptian} \\
		& \foreignlanguage{hebrew}{יְחַיּוּ} & \textit{they will let live} (Piel of the verb \foreignlanguage{hebrew}{חיה}) \\
		12:13 & \foreignlanguage{hebrew}{בִּגְלַל} & \textit{because of} (prep.)
	\end{longtable}
	
	
	\chapter{Chapter 17}
	
	\renewcommand\arraystretch{1.4}
	
	\section{Vocabulary}
	
	\subsection{Verbs}
	
	\begin{center}
		
		% For the centering of the separation between the two columns see the documentation of the array package, page 2 
		
		\begin{tabular}{>{\raggedleft}p{0.175\linewidth} p{0.75\linewidth}}
			\foreignlanguage{hebrew}{אמן} & Hi.\ \textit{to believe, think; to believe in; to have trust in}; Ni.\ \textit{to be reliable, faithful; to be permanent, endure} \\ % HALOT
			\foreignlanguage{hebrew}{חרם} & Hi.\ \textit{to put under a ban} \\ % HALOT
			\foreignlanguage{hebrew}{כעס} & Hi.\ \textit{to provoke to anger} \\
			\foreignlanguage{hebrew}{כרת} & Q.\ \textit{to cut}, Hi.\ \textit{cut off, exterminate} \\ % BDB and HALOT
			\foreignlanguage{hebrew}{כשׁל} & Q.\ \textit{to stumble, stagger}; Hi.\ \textit{to cause to stumble, stagger} \\
			\foreignlanguage{hebrew}{סגר} & Q.\ \textit{to shut, close}; Hi.\ \textit{to deliver, hand over; to shut} \\
			\foreignlanguage{hebrew}{צלח} & Q.\ \textit{to force entry to; to succeed, be successful}; Hi.\ \textit{to be successful; to make something a success} \\ % HALOT (BDB splits the verb into two roots)
			\foreignlanguage{hebrew}{שׂכל} & Hi.\ \textit{to understand; to have insight; to make wise, insightful; to achieve success} \\
			\foreignlanguage{hebrew}{שׁאר} & Ni.\ \textit{be left over, remain}; Hi.\ \textit{to leave over, spare} \\ % BDB
			\foreignlanguage{hebrew}{שׁבת} & Q.\ \textit{to cease; to desist; to rest}; Hi.\ \textit{to cause to cease, put an end to; exterminate, destroy; to remove} \\
			\foreignlanguage{hebrew}{שׁחת} & Hi.\ \textit{to ruin, destroy; annihilate, exterminate} \\ % HALOT
			\foreignlanguage{hebrew}{שׁכם} & Hi.\ \textit{to do something early; to rise early} \\
			\foreignlanguage{hebrew}{שׁלך} & Hi.\ \textit{to throw} \\
			\foreignlanguage{hebrew}{שׁמד} & Hi.\ \textit{to annihilate, exterminate; destroy} \\
		\end{tabular}
	\end{center}
	
	\subsection{Nouns}
	
	\begin{center}
		\begin{longtable}{>{\raggedleft}p{0.175\linewidth} p{0.75\linewidth}}
			\foreignlanguage{hebrew}{אוֺר} & \textit{light} \\
			\foreignlanguage{hebrew}{בְּכֹר} & \textit{firstborn} \\
			\foreignlanguage{hebrew}{חוֺמָה} & \textit{city wall, wall} \\ % HALOT
			\foreignlanguage{hebrew}{חֵמָה} & \textit{heat; rage, wrath; poison} \\ % BDB
			\foreignlanguage{hebrew}{חֲמוֺר} & \textit{donkey, he-ass} \\
			\foreignlanguage{hebrew}{כֹּחַ} & \textit{strength, power} \\
			\foreignlanguage{hebrew}{נְחֹשֶׁת} & \textit{copper, bronze} \\
			\foreignlanguage{hebrew}{נָשִׂיא} & \textit{chief, prince} \\ % BDB "a chief prince" without comma
			\foreignlanguage{hebrew}{פַּר} & \textit{bull, steer} (gem.\ noun; pl. \foreignlanguage{hebrew}{פָרִים}) \\ % HALOT
			\foreignlanguage{hebrew}{שְׁאֵרִית} & \textit{remainder; remnant} (fem.) \\ % HALOT
			\foreignlanguage{hebrew}{שְׁכֶם} & \textit{shoulder, back} \\
			\foreignlanguage{hebrew}{שֶׁקֶר} & \textit{breach of faith, lie} \\ % HALOT
		\end{longtable}
	\end{center}
	
	%\subsection{Other Parts of Speech}
	%
	%\begin{center}
	%	\begin{tabular}{>{\raggedleft}p{0.175\linewidth} p{0.75\linewidth}}
		%		\foreignlanguage{hebrew}{לְמַעַן} & \textit{in order to, so that} (conj.); \textit{on account of, for the sake of} (prep.) \\
		%	\end{tabular}
	%\end{center}
	
	
	\section{The Hiphil Binyan}
	
	\subsection{The Forms of the Strong Verb in the Hiphil}
	
	With about 9300 occurrences, the Hiphil is the second most frequent binyan behind the Qal. The Hiphil is characterized by the additional element /h/ that is prefixed as the syllable \textit{hi-} in the suffix conjugation or \textit{ha-} in the imperative, infinitive construct and infinitive absolute. In the prefix conjugation and the participle the /h/ is elided between two vowels \textit{CVha- > Ca}. The suffixes and prefixes of the SC, PC and imperative indicating person, number and gender are identical to the suffixes and prefixes of the corresponding forms in the Qal and the Niphal.
	
	\begin{center}
		\begin{longtable}{|lll|r|}
			\hline
			SC & sg. & 3 m. & \foreignlanguage{hebrew}{הִכְתִּיב} \\
			& & 3 f. & \foreignlanguage{hebrew}{הִכְתִּ֫יבָה} \\
			& & 2 m. & \foreignlanguage{hebrew}{הִכְתַּ֫בְתָּ} \\
			& & 2 f. & \foreignlanguage{hebrew}{הִכְתַּבְתְּ} \\
			& & 1 c. & \foreignlanguage{hebrew}{הִבְתַּ֫בְתִּי} \\
			\hline
			& pl. & 3 c. & \foreignlanguage{hebrew}{הִכְתִּ֫יבוּ} \\
			& & 2 m. & \foreignlanguage{hebrew}{הִכְתַּבְתֶּם} \\
			& & 2 f. & \foreignlanguage{hebrew}{הִכְתַּבְתֶּן} \\
			& & 1 c. & \foreignlanguage{hebrew}{הִכְתַּ֫בְנוּ} \\
			\hline
			PC & sg. & 3 m. & \foreignlanguage{hebrew}{יַכְתִּיב} \\
			& & 3 f. & \foreignlanguage{hebrew}{תַּכְתִּיב} \\
			& & 2 m. & \foreignlanguage{hebrew}{תַּכְתִּיב} \\
			& & 2 f. & \foreignlanguage{hebrew}{תַּכְתִּ֫יבִי} \\
			& & 1 c. & \foreignlanguage{hebrew}{אַכְתִּיב} \\
			\hline
			\pagebreak
			\hline
			& pl. & 3 c. & \foreignlanguage{hebrew}{יַכְתִּ֫יבוּ} \\
			& & 2 f. & \foreignlanguage{hebrew}{תַּכְתֵּ֫בְנָה} \\
			& & 2 m. & \foreignlanguage{hebrew}{תַּכְתִּ֫יבוּ} \\
			& & 2 f. & \foreignlanguage{hebrew}{תַּכְתֵּ֫בְנָה} \\
			& & 1 c. & \foreignlanguage{hebrew}{נַכְתִּיב} \\
			\hline
			Jussive & sg. & 3 m. & \foreignlanguage{hebrew}{יַכְתֵּב} \\
			\textit{wayyiqṭol} & sg. & 3 m. & \foreignlanguage{hebrew}{וַיַּכְתֵּב} \\
			\hline
			Impv. & sg. & 2 m. & \foreignlanguage{hebrew}{הַכְתֵּב} \\
			& & 2 f. & \foreignlanguage{hebrew}{הַכְתִּ֫יבִי} \\
			\hline
			& pl. & 2 m. & \foreignlanguage{hebrew}{הַכְתִּ֫יבוּ} \\
			& & 2 f. & \foreignlanguage{hebrew}{הַכְתֵּ֫בְנָה} \\
			\hline
			Inf.\ cs. & & & \foreignlanguage{hebrew}{הַכְתִּיב} \\
			Inf.\ abs. & & & \foreignlanguage{hebrew}{הַכְתֵּב} \\
			\hline
			Part. & m. & sg. & \foreignlanguage{hebrew}{מַכְתִּיב} \\
			& & pl.& \foreignlanguage{hebrew}{מַכְתִּיבִים} \\
			& f. & sg. & \foreignlanguage{hebrew}{מַכְתֶּבֶת} \\
			& & pl. & \foreignlanguage{hebrew}{מַכְתִּיבוֺת} \\
			\hline
		\end{longtable}
	\end{center}
	
	\noindent \textbf{Notes}
	\nopagebreak
	
	\noindent Vocalic suffixes are not stressed in Hiphil forms. Compare \foreignlanguage{hebrew}{הִכְתִּ֫יבוּ} with the corresponding Qal form \foreignlanguage{hebrew}{כָּתְבוּ}.
	
	In \textit{wayyiqṭol} forms and jussive forms without a suffix the thematic vowel is /ē/, e.g., \foreignlanguage{hebrew}{וַיַּשְׁלֵךְ} \textit{and he threw} and \foreignlanguage{hebrew}{יַשְׁלֵךְ} \textit{let him throw}. With enclitic pronouns, however, these forms have the thematic vowel /ī/, e.g., \foreignlanguage{hebrew}{וַיַּשְׁלִיכֵ֫הוּ} \textit{and he threw it [the staff]}.
	
	In the imperative the /h/ of the Hiphil at the beginning of the forms is preserved. This applies to the infinitive construct and infinitive absolute as well.
	
	The \textit{ṣere} of the imperative m.\ sg.\ correspondents to the same vowel in \textit{wayyiqṭol} forms and jussive forms. With enclitic pronoun, however, this form has the thematic vowel /ī/, e.g., \foreignlanguage{hebrew}{הַשְׁלִיכֵ֫הוּ} \textit{throw it [the staff]}.
	
	The longer forms of the imperative Hiphil is \foreignlanguage{hebrew}{הַכְתִּ֫יבָה}, e.g., \foreignlanguage{hebrew}{הַאֲזִ֫ינָה} \textit{listen!} (Job 33:1).
	
	The explanations about the prefix syllable of the prefix conjugation forms are to be applied to the prefix of the participles, too: \textit{mVhaktīb > maktīb > maḵtīḇ} \foreignlanguage{hebrew}{מַכְתִּיב}.
	
	The participle f.\ sg.\ is usually \foreignlanguage{hebrew}{מַכְתֶּבֶת}; forms with the feminine ending \textit{-ā}  are attested, too, e.g., \foreignlanguage{hebrew}{מַבְכִּירָה} \textit{a woman giving birth for the first time} (Jer 4:31).
	
	The primitive form of the suffix conjugation may have been \textit{*hiqṭil}. The long vowel /ī/ in forms without a suffix or forms with vocalic suffix is due to analogy with II w/y verbs. The vowel /a/ in the second syllable of suffix conjugation forms with consonantal suffixes is the result of the sound change /i/ > /a/ in stressed closed syllables (so-called Philippi's Law). A different primitive form of the suffix conjugation \textit{*haqṭal} may be assumed for some categories of weak verbs.\footnote{\space Some scholars assume \textit{*haqṭal} to be the primitive form of the suffix conjugation forms of both strong and weak verbs. In this case the prefix vowel \textit{hi-} is the result of attenuation /a/ > /i/ in an unstressed closed syllable. The long vowel /ī/ in forms without a suffix or forms with vocalic suffix is due to analogy with imperfect forms where its presence can be accounted for by analogy with II \textit{w/y} verbs.}
	
	The primitive form of the prefix conjugation was \textit{*yahaqṭil} or \textit{*yuhaqṭil}. The /h/ of the Hiphil was elided between vowels; the prefix vowel /a/ of the /*-ha-/ syllable is preserved as the vowel of the prefix syllable. The long vowel /ī/ is due to analogy with II w/y verbs.
	
	% Possible additional Note:
	% Loss of the /h/ of the Inf. cs. after preposition in 2 Sam 19:19 (best example for explanation); Deut 1:33; Ex 13:21; 1 Sam 2:33; 2 Kgs 9:15; 2 Kgs 19:25
	
	
	\subsection{The Meaning of Verbs in the Hiphil}
	
	The Hiphil has the following main meanings:
	
	\begin{itemize}[noitemsep]
		\item[--] Causative (someone causes somebody to do something), e.g., \foreignlanguage{hebrew}{מלך} Qal \textit{to rule as king}, Hi.\ \textit{to make somebody rule as king}; \foreignlanguage{hebrew}{ראה} Qal \textit{to see}, Hi.\ \textit{to show something to somebody}
		\item[--] Denominative (the verb is derived from a noun): noun \foreignlanguage{hebrew}{אֹזֶן} \textit{ear} → verb \foreignlanguage{hebrew}{אזן} Hi.\ \textit{to listen, hear}; noun \foreignlanguage{hebrew}{שֹׁ֫רֶשׁ} \textit{root} → verb \foreignlanguage{hebrew}{שׁרשׁ} Hi.\ \textit{to put down roots}
		\item[--] Declarative: \foreignlanguage{hebrew}{רשׁע} Qal \textit{to be guilty}; Hi.\ \textit{to declare guilty} (cf. \foreignlanguage{hebrew}{רָשָׁע} \textit{guilty})
		\item[--] Intransitive (frequently from adjectives or stative verbs): \foreignlanguage{hebrew}{שׁמן} Hi.\ \textit{to make fat} (caus\-ative) or \textit{to become fat} (intransitive; from \foreignlanguage{hebrew}{שָׁמֵן} \textit{fat, robust})
		\item[--] Without obvious Hiphil meaning: \foreignlanguage{hebrew}{שׁלךְ} Hi.\ \textit{to throw}
		\item[--] Sometimes the Hiphil functions as a causative to the Niphal: \foreignlanguage{hebrew}{שׁבע} Ni.\ \textit{to take an oath}, Hi.\ \textit{to make somebody take an oath}; \foreignlanguage{hebrew}{נצל} Ni.\ \textit{tear oneself away, deliver oneself; be delivered}, Hi.\  \textit{take away; rescue}
	\end{itemize}
	
	If the meaning of a verb in the Qal is known, these semantic categories should be considered for the translation of the verb in the Hiphil. A lexicon will help in further specifying the meaning of Hiphil forms.
	
	\subsection{Guttural Verbs in the Hiphil Binyan}
	
	\subsubsection{Verbs I gutt.}
	
	In the suffix conjugation, the prefix vowel is changed to \textit{səgol}: \foreignlanguage{hebrew}{הֶעֱמִיד}, etc.\ (cf.\ the primitive form \textit{*hiqṭil}).
	
	The prefix syllable is usually closed by a \textit{ḥaṭef} vowel: suffix conjugation \foreignlanguage{hebrew}{הֶעֱמַ֫דְתָּ}, prefix conjugation \foreignlanguage{hebrew}{תַּעֲמִ֫ידוּ}, etc.
	
	
	
	\subsubsection{Verbs II gutt.}
	
	Verbs II gutt.\ don't show any differences to the regular strong verbs without gutturals because the second root consonant R\textsubscript{2} is always at the beginning of a syllable.
	
	
	
	\subsubsection{Verbs III gutt.}
	
	In forms of the suffix conjugation without suffix and in the prefix conjugation the long vowel /ī/ is followed by a furtive \textit{pátaḥ}, e.g., \foreignlanguage{hebrew}{הִשְׁבִּיעַ} \textit{he made somebody swear}. The same applies to the /ē/ of the infinitive absolute, e.g., \foreignlanguage{hebrew}{הַשְׁבֵּעַ}.
	
	In \textit{wayyiqṭol} forms without suffix and forms of the jussive and the imperative without suffix the vowel /ē/ is replaced by /a/, e.g., \foreignlanguage{hebrew}{יַבְטַח} (\foreignlanguage{hebrew}{וְאַל־יַבְטַח אֶתְכֶם חִזְקִיָּהוּ אֶל־יְהוָה} \textit{Don't let Hezeqiah make you trust in the Lord}, 2\,Kgs 18:30), \foreignlanguage{hebrew}{וַיַּבְטַח} (\foreignlanguage{hebrew}{וַיַּבְטַח אֶתְכֶם עַל־שָֽׁקֶר} \textit{and he made you trust in a lie}, Jer 29:31). This vowel change is also attested in verbs III\,\textit{r}, e.g., \foreignlanguage{hebrew}{וַתֹּתַ֑ר} \textit{and she [Ruth] had left over} (3 f.\ sg.\ \textit{wayyiqṭol} of the verb \foreignlanguage{hebrew}{יתר} Hi., Ruth 2:14), but preservation of the \textit{ṣere} is attested as well, e.g., \foreignlanguage{hebrew}{וַיַּבְעֵר} \textit{and he set to fire} (Judg 15:5).
	
	In the participle f.\ sg.\ the vowel sequence of the ending is \textit{a--a}, e.g., \foreignlanguage{hebrew}{מַגַּ֫עַת} (from the root \foreignlanguage{hebrew}{נגע}; 2\,Chr 3:11).
	
	
	\section{Hypothetical Conditional Clauses with \foreignlanguage{hebrew}{לוּ} or \foreignlanguage{hebrew}{לוּלֵא}}
	
	For hypothetical conditional clauses -- the contents of both the \textit{if}-clause and the main clause are unreal -- the conjunctions \foreignlanguage{hebrew}{לוּ} \textit{if} and \foreignlanguage{hebrew}{לוּלֵא} or -- with a different spelling and actually more frequent -- \foreignlanguage{hebrew}{לוּלֵי} \textit{if not} are used (\foreignlanguage{hebrew}{לוּלֵא} is dissimilated from \textit{lūlō(ʾ)}, i.e., \foreignlanguage{hebrew}{לוּ} + \foreignlanguage{hebrew}{לֹא}).
	
	\begin{longtable}{>{\raggedleft}p{0.35\linewidth} p{0.55\linewidth}}
		\foreignlanguage{hebrew}{וַיֹּאמַר אַחַי בְּנֵי־אִמִּי הֵם חַי־יְהוָה לוּ הַחֲיִתֶם אוֹתָם לֹא הָרַגְתִּי אֶתְכֶם} & \textit{And he said, \enquote{They are my brothers, the sons of my mother. As the Lord lives, if you had spared their their lives I would not kill you.}} (Judg 8:19) \\
		\foreignlanguage{hebrew}{וַיֹּאמֶר הָאִישׁ אֶל־יוֹאָב וְלוּ אָנֹכִי שֹׁקֵל עַל־כַּפַּי אֶלֶף כֶּסֶף לֹא־אֶשְׁלַח יָדִי אֶל־בֶּן־הַמֶּלֶךְ} & \textit{And the man said to Joab, \enquote{Even if I felt the weight of a thousand shekels in my hands I would not stretch out my hand against the king's son}} (2\,Sam 18:12 Qere Aleppo Codex) \\
		\foreignlanguage{hebrew}{וַיֹּאמֶר לָהֶם לוּלֵא חֲרַשְׁתֶּם בְּעֶגְלָתִי לֹא מְצָאתֶם חִידָתִי} & \textit{And he said to them, \enquote{If you had not plowed with my heifer you would not have solved my riddle}} (Judg 14:18) \\
	\end{longtable}
	
	% The Hebrew text of 2 Sam 18:20 follows the Qere of the Aleppo Codex; the Qere of the Leningrad Codex reads ולוא (BHS ולו)
	% Other possible example: Judg 13:23
	
	\noindent \textbf{Note}
	\nopagebreak
	
	\noindent The conjunction \foreignlanguage{hebrew}{לוּ} is sometimes spelled \foreignlanguage{hebrew}{לוּא} or even \foreignlanguage{hebrew}{לא} in the Ketiv.
	
	Related to its use in hypothetical conditional clauses is the use of the suffix conjugation or \textit{qaṭal} form in hypothetical main clauses (Gen 18:12; Gen 21:7).
	
	\vspace{0.5cm}
	
	\begin{tabular}{>{\raggedleft}p{0.35\linewidth} p{0.55\linewidth}}
		\foreignlanguage{hebrew}{וַתִּצְחַק שָׂרָה בְּקִרְבָּהּ לֵאמֹר אַחֲרֵי בְלֹתִי הָיְתָה־לִּי עֶדְנָה}  & \textit{And Sarah laughed to herself, \enquote{Shall I have pleasure after I am worn out?}} (Gen 18:12) \\
		\foreignlanguage{hebrew}{וַתֹּאמֶר מִי מִלֵּל לְאַבְרָהָם הֵינִיקָה בָנִים שָׂרָה} & \textit{And she said, \enquote{Who would have said to Abraham that Sarah has nursed children?}} (Gen 21:7) \\
	\end{tabular}
	
	
	\section{Wishes with \foreignlanguage{hebrew}{לוּ}}
	
	As a particle, \foreignlanguage{hebrew}{לוּ} also introduces wishes (Gen 17:18). At times, these wishes are hypothetical (Num 14:2).
	
	\vspace{0.5cm}
	
	\begin{tabular}{>{\raggedleft}p{0.35\linewidth} p{0.55\linewidth}}
		\foreignlanguage{hebrew}{וַיֹּאמֶר אַבְרָהָם אֶל־הָאֱלֹהִים לוּ יִשְׁמָעֵאל יִחְיֶה לְפָנֶיךָ} & \textit{And Abraham said to God, \enquote{O that Ishmael might live before you!}} (Gen 17:18) \\
		\foreignlanguage{hebrew}{וַיֹּאמְרוּ אֲלֵהֶם כָּל־הָעֵדָה לוּ־מַתְנוּ בְּאֶרֶץ מִצְרַיִם אוֹ בַּמִּדְבָּר הַזֶּה לוּ־מָֽתְנוּ} & \textit{And the whole congregation said to them, \enquote{If only we had died in Egypt or if only we had died in this wilderness!}} (Num 14:2) \\
	\end{tabular}
	
	% Other possible example: Gen 30:34 (with jussive)
	
	
	
	\section{Exercises}
	
	\subsection{Translation of Verbal Forms}
	
	Translate the following verbal forms. Identify the gender (masc., fem., comm.) and number (sg., pl.) of forms of which the English translation is ambiguous (i.e., \textit{you}, \textit{they}). Mark the stressed syllable if stress is not on the last syllable.
	
	\hspace{0.5cm}
	
	\selectlanguage{hebrew}
	
	\noindent
	1~~\foreignlanguage{hebrew}{הִשְׁכִּימוּ}  \hspace{0.3cm}
	2~~\foreignlanguage{hebrew}{יַאֲמִינוּ}  \hspace{0.3cm}
	3~~\foreignlanguage{hebrew}{מַשְׁחִתִים}  \hspace{0.3cm}
	4~~\foreignlanguage{hebrew}{הֶאֱמִין}  \hspace{0.3cm}
	5~~\foreignlanguage{hebrew}{אַשְׁלִיךְ}  \hspace{0.3cm}
	6~~\foreignlanguage{hebrew}{וַיַּשְׁכֵּם}  \hspace{0.3cm}
	7~~\foreignlanguage{hebrew}{הֶאֱמִינוּ}  \hspace{0.3cm}
	8~~\foreignlanguage{hebrew}{הַאֲמִינוּ}  \hspace{0.3cm}
	9~~\foreignlanguage{hebrew}{תַּשְׁלִיךְ}  \hspace{0.3cm}
	10~~\foreignlanguage{hebrew}{הִשְׁלִיכָה}  \hspace{0.3cm}
	11~~\foreignlanguage{hebrew}{הַשְׁלֵךְ}  \hspace{0.3cm}
	12~~\foreignlanguage{hebrew}{וַיַּשְׁלִיכוּ}  \hspace{0.3cm}
	13~~\foreignlanguage{hebrew}{הִשְׁלַכְתִּי}  \hspace{0.3cm}
	14~~\foreignlanguage{hebrew}{הֶאֱמַנְתֶּם}  \hspace{0.3cm}
	15~~\foreignlanguage{hebrew}{הִכְרַתִּי}  \hspace{0.3cm}
	
	\selectlanguage{english}
	
	
	\subsection{Translation of Sentences}
	
	Translate the following sentences from the Hebrew Bible. Names of persons and geographical names in these sentences: \foreignlanguage{hebrew}{אַבְנֵר}, \foreignlanguage{hebrew}{אֶלְעָזָר}, \foreignlanguage{hebrew}{בֵּית־חוֹרֹן}, \foreignlanguage{hebrew}{בַּעַל פְּעוֺר} (name of a place and of a deity), \foreignlanguage{hebrew}{בַּעְשָׁא}, \foreignlanguage{hebrew}{דָּוִד}, \foreignlanguage{hebrew}{זִמְרִי}, \foreignlanguage{hebrew}{חֲנַנְיָה}, \foreignlanguage{hebrew}{יֵהוּא}, \foreignlanguage{hebrew}{יְהוֹשֻׁעַ}, \foreignlanguage{hebrew}{יְהוֹשָׁפָט}, \foreignlanguage{hebrew}{יִרְמְיָה}, \foreignlanguage{hebrew}{לוֹט}, \foreignlanguage{hebrew}{מֹשֶׁה}, \foreignlanguage{hebrew}{עֲזֵקָה}, \foreignlanguage{hebrew}{שָׁאוּל}, \foreignlanguage{hebrew}{תְּקוֹעַ}
	
	\vspace{0.5cm}
	
	\selectlanguage{hebrew}
	\noindent
	1~~\foreignlanguage{hebrew}{וַיְהִי בְּנֻסָם מִפְּנֵי יִשְׂרָאֵל –~הֵם בְּמוֹרַד}\LTRfootnote{\space \foreignlanguage{hebrew}{מוֺרָד} \textit{descent, slope} (from the root \foreignlanguage{hebrew}{ירד} with prefixed \foreignlanguage{hebrew}{מ})} \foreignlanguage{hebrew}{בֵּית־חוֹרֹן~– וַיהוָה הִשְׁלִיךְ עֲלֵיהֶם אֲבָנִים גְּדֹלוֹת מִן־הַשָּׁמַיִם עַד־עֲזֵקָה וַיָּמֻתוּ. רַבִּים אֲשֶׁר־מֵתוּ בְּאַבְנֵי הַבָּרָד}\LTRfootnote{\space \foreignlanguage{hebrew}{בָּרָד} \textit{hail}} \foreignlanguage{hebrew}{מֵאֲשֶׁר הָרְגוּ בְּנֵי יִשְׂרָאֵל בֶּחָרֶב} \hspace{0.3cm}
	2~~\foreignlanguage{hebrew}{עֵינֵיכֶם הָרֹאֹת אֵת אֲשֶׁר־עָשָׂה יְהוָה בְּבַעַל פְּעוֹר כִּי}\LTRfootnote{\space \foreignlanguage{hebrew}{כִּי} here \textit{that} (introducing a second object clause to \foreignlanguage{hebrew}{הָרֹאֹת})} \foreignlanguage{hebrew}{כָל־הָאִישׁ אֲשֶׁר הָלַךְ אַחֲרֵי בַֽעַל־פְּעוֹר הִשְׁמִידוֹ יְהוָה אֱלֹהֶיךָ מִקִּרְבֶּךָ}  \hspace{0.3cm}
	3~~\foreignlanguage{hebrew}{וַיֹּאמֶר יִרְמְיָה הַנָּבִיא אֶל־חֲנַנְיָה הַנָּבִיא שְׁמַע־נָא חֲנַנְיָה לֹא־שְׁלָחֲךָ יְהוָה וְאַתָּה הִבְטַחְתָּ אֶת־הָעָם הַזֶּה עַל־שָֽׁקֶר}\LTRfootnote{\space \foreignlanguage{hebrew}{שָֽׁקֶר} pausal form of \foreignlanguage{hebrew}{שֶׁקֶר}} \hspace{0.3cm}
	4~~\foreignlanguage{hebrew}{וְאַנְשֵׁי־קֹדֶשׁ תִּהְיוּן לִי. וּבָשָׂר בַּשָּׂדֶה – טְרֵפָה}\LTRfootnote{\space \foreignlanguage{hebrew}{טְרֵפָה} \textit{animal torn by wild beasts} (apposition to \foreignlanguage{hebrew}{בָּשָׂר בַּשָׁדֶה})} \foreignlanguage{hebrew}{– לֹא תֹאכֵלוּ. לַכֶּלֶב}\LTRfootnote{\space \foreignlanguage{hebrew}{כֶּלֶב} \textit{dog}} \foreignlanguage{hebrew}{תַּשְׁלִכוּן אֹתוֹ} \hspace{0.3cm}
	5~~\foreignlanguage{hebrew}{וְעַתָּה הִנֵּה יָדַעְתִּי כִּי מָלֹךְ תִּמְלוֹךְ וְקָמָה בְּיָדְךָ מַמְלֶכֶת יִשְׂרָאֵל וְעַתָּה הִשָּׁבְעָה לִּי בַּיהוָה אִם־תַּכְרִית אֶת־זַרְעִי אַחֲרָי וְאִם־תַּשְׁמִיד אֶת־שְׁמִי מִבֵּית אָבִי וַיִּשָּׁבַע דָּוִד לְשָׁאוּל וַיֵּלֶךְ שָׁאוּל אֶל־בֵּיתוֹ וְדָוִד וַֽאֲנָשָׁיו עָלוּ עַל־הַמְּצוּדָה}\LTRfootnote{\space \foreignlanguage{hebrew}{מְצוּדָה} \textit{mountain stronghold}} \hspace{0.3cm}
	6~~\foreignlanguage{hebrew}{וַיֵּצֵא לוֹט וַיְדַבֵּר אֶל־חֲתָנָיו}\LTRfootnote{\space \foreignlanguage{hebrew}{חָתָן} \textit{son in law}} \foreignlanguage{hebrew}{לֹקְחֵי בְנֹתָיו וַיֹּאמֶר קוּמוּ צְּאוּ מִן־הַמָּקוֹם הַזֶּה כִּי־מַשְׁחִית יְהוָה אֶת־הָעִיר וַיְהִי כִמְצַחֵק}\LTRfootnote{\space \foreignlanguage{hebrew}{מְצַחֵק} \textit{somebody who is joking} (participle Pi.)} \foreignlanguage{hebrew}{בְּעֵינֵי חֲתָנָיו} \hspace{0.3cm}
	7~~\foreignlanguage{hebrew}{וַיַּעַשׂ מֹשֶׁה כַּאֲשֶׁר צִוָּה}\LTRfootnote{\space \foreignlanguage{hebrew}{צִוָּה} \textit{he instructed} (Pi.)} \foreignlanguage{hebrew}{יְהוָה אֹתוֹ וַיִּקַּח אֶת־יְהוֹשֻׁעַ וַיַּעֲמִדֵהוּ לִפְנֵי אֶלְעָזָר הַכֹּהֵן וְלִפְנֵי כָּל־הָעֵדָה וַיִּסְמֹךְ}\LTRfootnote{\space \foreignlanguage{hebrew}{וַיִּסְמֹךְ} with \foreignlanguage{hebrew}{יָד} and \foreignlanguage{hebrew}{עַל} \textit{to lay hand on} (in consecration)} \foreignlanguage{hebrew}{אֶת־יָדָיו עָלָיו וַיְצַוֵּהוּ}\LTRfootnote{\space \foreignlanguage{hebrew}{וַיְצַוֵּהוּ} \textit{and he gave him instructions}} \foreignlanguage{hebrew}{כַּאֲשֶׁר דִּבֶּר יְהוָה בְּיַד־מֹשֶׁה} \hspace{0.3cm}
	8~~\foreignlanguage{hebrew}{וַיַּשְׁמֵד יֵהוּא אֶת־הַבַּעַל מִיִּשְׂרָאֵל}  \hspace{0.3cm}
	9~~\foreignlanguage{hebrew}{וַיַּשְׁמֵד זִמְרִי אֵת כָּל־בֵּית בַּעְשָׁא כִּדְבַר יְהוָה אֲשֶׁר דִּבֶּר אֶל־בַּעְשָׁא בְּיַד יֵהוּא הַנָּבִיא}  \hspace{0.3cm}
	10~~\foreignlanguage{hebrew}{וַיֹּאמֶר דָּוִד אֶל־אַבְנֵר הֲלוֹא־אִישׁ אַתָּה וּמִי כָמוֹךָ בְּיִשְׂרָאֵל וְלָמָּה לֹא שָׁמַרְתָּ אֶל־}\LTRfootnote{\space \foreignlanguage{hebrew}{אֶל} here for \foreignlanguage{hebrew}{עַל} (cf. the continuation); the verb \foreignlanguage{hebrew}{שׁמר} normally governs a direct object.} \foreignlanguage{hebrew}{אֲדֹנֶיךָ הַמֶּלֶךְ כִּי־בָא אַחַד הָעָם לְהַשְׁחִית אֶת־הַמֶּלֶךְ אֲדֹנֶיךָ. לֹא־טוֹב הַדָּבָר הַזֶּה אֲשֶׁר עָשִׂיתָ. חַי־יְהוָה כִּי בְנֵי־מָוֶת אַתֶּם אֲשֶׁר לֹא־שְׁמַרְתֶּם עַל־אֲדֹנֵיכֶם עַל־מְשִׁיחַ}\LTRfootnote{\space \foreignlanguage{hebrew}{מָשִׁיחַ} \textit{anointed one}} \foreignlanguage{hebrew}{יְהוָה} \hspace{0.3cm}
	11~~\foreignlanguage{hebrew}{וַיַּעַן מֹשֶׁה וַיֹּאמֶר וְהֵן}\LTRfootnote{\space \foreignlanguage{hebrew}{הֵן} \textit{if}; the main clause is missing.} \foreignlanguage{hebrew}{לֹא־יַאֲמִינוּ לִי וְלֹא יִשְׁמְעוּ בְּקֹלִי כִּי יֹאמְרוּ לֹא־נִרְאָה}\LTRfootnote{\space \foreignlanguage{hebrew}{נִרְאָה} \textit{he appeared} (Ni. of \foreignlanguage{hebrew}{ראה})} \foreignlanguage{hebrew}{אֵלֶיךָ יְהוָה. וַיֹּאמֶר אֵלָיו יְהוָה מַה־זֶּה בְיָדֶךָ וַיֹּאמֶר מַטֶּה. וַיֹּאמֶר הַשְׁלִיכֵ֫הוּ אַ֫רְצָה. וַיַּשְׁלִיכֵ֫הוּ אַ֫רְצָה וַיְהִי לְנָחָשׁ}\LTRfootnote{\space \foreignlanguage{hebrew}{נָחָשׁ} \textit{snake}} \foreignlanguage{hebrew}{וַיָּנָס מֹשֶׁה מִפָּנָיו.} \hspace{0.3cm}
	12~~\foreignlanguage{hebrew}{וַיַּשְׁכִּימוּ בַבֹּקֶר וַיֵּצְאוּ לְמִדְבַּר תְּקוֹעַ וּבְצֵאתָם עָמַד יְהוֹשָׁפָט וַיֹּאמֶר שְׁמָעוּנִי יְהוּדָה וְיֹשְׁבֵי יְרוּשָׁלִַם הַאֲמִינוּ בַּיהוָה אֱלֹהֵיכֶם וְתֵאָמֵנוּ הַאֲמִינוּ בִנְבִיאָיו וְהַצְלִיחוּ}\LTRfootnote{\space \foreignlanguage{hebrew}{צלח} Hi. \textit{to be successful}}
	
	\selectlanguage{english}
	
	
	
	\section{Hebrew Reading: Genesis 12:15--20}
	Translate Gen 12:15--20 with the help of notes below the text. Names of persons and places are easily identifiable.
	
	\vspace{0.5cm}
	
	\selectlanguage{hebrew}
	\noindent
	\textsuperscript{15}~\foreignlanguage{hebrew}{וַיִּרְאוּ אֹתָהּ שָׂרֵי פַרְעֹ֔ה וַיְהַלְלוּ אֹתָהּ אֶל־פַּרְעֹ֑ה וַתֻּקַּח הָאִשָּׁה בֵּית פַּרְעֹֽה׃} \hspace{0.3cm}
	\textsuperscript{16}~\foreignlanguage{hebrew}{וּלְאַבְרָם הֵיטִיב בַּעֲבוּרָ֑הּ וַיְהִי־לוֹ צֹאן־וּבָקָר וַחֲמֹרִ֔ים וַעֲבָדִים וּשְׁפָחֹ֔ת וַאֲתֹנֹת וּגְמַלִּֽים׃} \hspace{0.3cm}
	\textsuperscript{17}~\foreignlanguage{hebrew}{יְנַגַּע יְהוָה אֶת־פַּרְעֹה נְגָעִים גְּדֹלִים וְאֶת־בֵּית֑וֹ עַל־דְּבַר שָׂרַי אֵשֶׁת אַבְרָֽם׃} \hspace{0.3cm}
	\textsuperscript{18}~\foreignlanguage{hebrew}{וַיִּקְרָא פַרְעֹה לְאַבְרָ֔ם וַיֹּאמֶר מַה־זֹּאת עָשִׂיתָ לִּ֑י לָמָּה לֹא־הִגַּדְתָּ לִּ֔י כִּי אִשְׁתְּךָ הִֽוא׃} \hspace{0.3cm}
	\textsuperscript{19}~\foreignlanguage{hebrew}{לָמָה אָמַרְתָּ אֲחֹתִי הִ֔וא וָאֶקַּח אֹתָהּ לִי לְאִשָּׁ֑ה וְעַתָּה הִנֵּה אִשְׁתְּךָ קַח וָלֵֽךְ׃} \hspace{0.3cm}
	\textsuperscript{20}~\foreignlanguage{hebrew}{וַיְצַו עָלָיו פַּרְעֹה אֲנָשִׁ֑ים וֽ͏ַיְשַׁלְּחוּ אֹתוֹ וְאֶת־אִשְׁתּוֹ וְאֶת־כָּל־אֲשֶׁר־לֽוֹ׃} \hspace{0.3cm}
	\selectlanguage{english}
	
	
	
	\hspace*{-0.5cm}\begin{longtable}{p{0.075\linewidth} p{0.1\linewidth}p{0.725\linewidth}}
		12:15 & \foreignlanguage{hebrew}{וַיְהַלְלוּ} & \textit{and they praised her} (Pi. of \foreignlanguage{hebrew}{הלל}) \\
		& \foreignlanguage{hebrew}{וַתֻּקַּח} & \textit{she was taken} (Qal passive of \foreignlanguage{hebrew}{לקח}) \\
		12:16 & \foreignlanguage{hebrew}{הֵיטִיב} & \textit{he did well, he dealt well} (Hi. of \foreignlanguage{hebrew}{יטב}) \\
		& \foreignlanguage{hebrew}{אָתוֺן} & \textit{she-ass} \\
		12:17 & \foreignlanguage{hebrew}{יְנַגַּע} & \textit{he afflicted} (Pi. of \foreignlanguage{hebrew}{נגע}; cf. \foreignlanguage{hebrew}{נֶגַע}) \\
		& \foreignlanguage{hebrew}{עַל־דְּבַר} & \textit{because of} \\
		12:18 & \foreignlanguage{hebrew}{מַה־זֹּאת} & The demonstrative pronouns \foreignlanguage{hebrew}{זֶה} and \foreignlanguage{hebrew}{זֹאת} often follow an interrogative pronoun or adverb \enquote{without any notable change in meaning} (JM §\,143\,\textit{g}) \\
		& \foreignlanguage{hebrew}{הִגַּדְתָּ} & \textit{you told} (verb \foreignlanguage{hebrew}{נגד} Hi.) \\
		& \foreignlanguage{hebrew}{הִֽוא} & read \foreignlanguage{hebrew}{הִיא} \\
		12:19 & \foreignlanguage{hebrew}{הִֽוא} & read \foreignlanguage{hebrew}{הִיא} \\
		12:20 & \foreignlanguage{hebrew}{וַיְצַו} & \textit{and he gave instructions} \\
		& \foreignlanguage{hebrew}{וֽ͏ַיְשַׁלְּחוּ} & \textit{and he sent  off} (Pi. of \foreignlanguage{hebrew}{שלח}) \\
	\end{longtable}
	
	
	
	\chapter{Chapter 18}
	
	\renewcommand\arraystretch{1.4}
	
	\section{Vocabulary}
	
	\subsection{Verbs}
	
	\begin{center}
		
		% For the centering of the separation between the two columns see the documentation of the array package, page 2 
		
		\begin{tabular}{>{\raggedleft}p{0.175\linewidth} p{0.75\linewidth}}
			\foreignlanguage{hebrew}{סתר} & Hi.\ \textit{to hide, conceal} (transitive); Ni. \textit{to hide oneself; to be hid, concealed} \\ % HALOT
			\foreignlanguage{hebrew}{עזר} & Q.\ \textit{to help, assist}  \\
			\foreignlanguage{hebrew}{ערך} & Q.\ \textit{to arrange, set in order; to draw up a battle formation}  \\
			\foreignlanguage{hebrew}{פנה} & Q.\ \textit{to turn (turn to one side, head in a particular direction; turn round; turn away and go on further)} \\ % HALOT
			\foreignlanguage{hebrew}{פקד} & Q.\ \textit{to visit, attend to, muster, appoint}; Hi. \textit{to set (over), make overseer; commit, entrust; deposit}  \\
			\foreignlanguage{hebrew}{רכב} & Q.\ \textit{to mount and ride, ride}; Hi.\ \textit{to cause ride, to make to ride} \\ % BDB
		\end{tabular}
	\end{center}
	
	\subsection{Nouns}
	
	\begin{center}
		\begin{longtable}{>{\raggedleft}p{0.175\linewidth} p{0.75\linewidth}}
			\foreignlanguage{hebrew}{גּוֺרָל} & \textit{lot; allocation by lot} \\ % HALOT
			\foreignlanguage{hebrew}{חַיָּה} & \textit{animals} (usually collective) \\
			\foreignlanguage{hebrew}{יְשׁוּעָה} & \textit{help, salvation} \\ % HALOT
			\foreignlanguage{hebrew}{כָּנָף} & \textit{wing; edge, extremity; skirt} (of a garment) (fem.) \\
			\foreignlanguage{hebrew}{לָשׁוֺן} & \textit{tongue; language} (m.; pl.\ \foreignlanguage{hebrew}{לְשֹׁנוֺת}) \\
			\foreignlanguage{hebrew}{מִגְרָשׁ} & \textit{pastureland (belonging to a city); outskirts} \\ % HALOT
			\foreignlanguage{hebrew}{מִקְנֶה} & \textit{livestock as property, cattle} \\
			\foreignlanguage{hebrew}{עַמּוּד} & \textit{pillar; column} (cf. \foreignlanguage{hebrew}{עמד}) \\
			\foreignlanguage{hebrew}{עָפָר} & \textit{dry earth; dust} \\ % BDB
			\foreignlanguage{hebrew}{פְּרִי} & \textit{fruit} \\
			\foreignlanguage{hebrew}{צֶדֶק} & \textit{rightness; righteousness} \\ % BDB
			\foreignlanguage{hebrew}{שֶׂה} & \textit{a small livestock beast} (\textit{sheep} or \textit{goat}; unitary noun for the collective noun \foreignlanguage{hebrew}{צֹאן}) \\
			\foreignlanguage{hebrew}{שָׂעִיר} & \textit{billy-goat, buck} \\
			\foreignlanguage{hebrew}{שַׁבָּת} & \textit{Sabbath} (f.\ or m.; pl.\ \foreignlanguage{hebrew}{שַׁבָּתוֺת}) \\
			\foreignlanguage{hebrew}{שׁוֺר} & \textit{a single beast, bovid} (\textit{bull}, etc.; unitary noun for the collective noun \foreignlanguage{hebrew}{בָּקָר}) \\
			\foreignlanguage{hebrew}{תוֺעֵבָה} & \textit{abomination, abhorrence} \\
		\end{longtable}
	\end{center}
	
	\subsection{Other Parts of Speech}
	
	\begin{center}
		\begin{tabular}{>{\raggedleft}p{0.175\linewidth} p{0.75\linewidth}}
			\foreignlanguage{hebrew}{מַ֫עְלָה} & \textit{upwards, above} (adv.) (noun \foreignlanguage{hebrew}{מַעַל} \textit{higher part}) \\ % Jenni, Lehrbuch, includes this and the following item in lesson 20
			\foreignlanguage{hebrew}{מִמַּעַל} & \textit{above} (adv.) (\foreignlanguage{hebrew}{מִמַּעַל לְ} prep. on top of, above) \\
		\end{tabular}
	\end{center}
	
	
	\section{The Hophal Binyan}
	
	\subsection{The Forms of the Strong Verb in the Hophal}
	
	The Hophal is the passive form to the Hiphil; e.g., \foreignlanguage{hebrew}{הִשְׁלַ֫כְתִּי} \textit{I threw} (Hi.) and \foreignlanguage{hebrew}{הָשְׁלַ֫כְתִּי} \textit{I was thrown} (Ho.). There are only about 380 occurrences of the Hophal in the Hebrew Bible with most forms from weak verbs. In Modern Hebrew, the Hophal is called Huphal.
	
	The Hophal is characterized by the prefix syllable \textit{*hu-}. The vowel sequence \textit{u--a} is characteristic of passive verbal forms. In Tiberian Hebrew the prefix vowel can be either /ɔ/ (< /u/) (\textit{qameṣ ḥatuf}) or the original /u/ (\textit{qibbuṣ}). In the participle /u/ is more frequent than /ɔ/ because of the preceding labial sound /m/.
	
	\begin{center}
		\begin{longtable}{|lll|r|}
			\hline
			SC & sg. & 3 m. & \foreignlanguage{hebrew}{הֻכְתַּב}/\foreignlanguage{hebrew}{הָכְתַּב} \\
			& & 3 f. & \foreignlanguage{hebrew}{הָכְתְּבָה} \\
			& & 2 m. & \foreignlanguage{hebrew}{הָכְתַּ֫בְתָּ} \\
			& & 2 f. & \foreignlanguage{hebrew}{הָכְתַּבְתְּ} \\
			& & 1 c. & \foreignlanguage{hebrew}{הָכְתַּ֫בְתִּי} \\
			\hline
			& pl. & 3 c. & \foreignlanguage{hebrew}{הָכְתְּבוּ} \\
			& & 2 m. & \foreignlanguage{hebrew}{הָכְתַּבְתֶּם} \\
			& & 2 f. & \foreignlanguage{hebrew}{הָכְתַּבְתֶּן} \\
			& & 1 c. & \foreignlanguage{hebrew}{הָכְתַּ֫בְנוּ} \\
			\hline
			PC & sg. & 3 m. & \foreignlanguage{hebrew}{יָכְתַּב} \\
			& & 3 f. & \foreignlanguage{hebrew}{תָּכְתַּב} \\
			& & 2 m. & \foreignlanguage{hebrew}{תָּכְתַּב} \\
			& & 2 f. & \foreignlanguage{hebrew}{תָּכְתְּבִי} \\
			& & 1 c. & \foreignlanguage{hebrew}{אָכְתַּב} \\
			\hline
			& pl. & 3 c. & \foreignlanguage{hebrew}{יָכְתְּבוּ} \\
			& & 2 f. & \foreignlanguage{hebrew}{תָּכְתַּ֫בְנָה} \\
			& & 2 m. & \foreignlanguage{hebrew}{תָּכְתְּבוּ} \\
			& & 2 f. & \foreignlanguage{hebrew}{תָּכְתַּ֫בְנָה} \\
			& & 1 c. & \foreignlanguage{hebrew}{נָכְתַּב} \\
			\hline
			Jussive  & sg. & 3 m. & \foreignlanguage{hebrew}{יָכְתַּב} \\
			\textit{wayyiqṭol}  & sg. & 3 m. & \foreignlanguage{hebrew}{וַיָּכְתַּב} \\
			\hline
			Impv. & sg. & 2 m. & \foreignlanguage{hebrew}{הָכְתַּב} \\
			& & 2 f. & \foreignlanguage{hebrew}{הָכְתְּבִי} \\
			\hline
			& pl. & 2 m. & \foreignlanguage{hebrew}{הָכְתְּבוּ} \\
			& & 2 f. & \foreignlanguage{hebrew}{הָכְתַּ֫בְנוּ} \\
			\hline
			Inf.\ cs. & & & \foreignlanguage{hebrew}{הָכְתַּב} \\
			Inf.\ abs. & &  & \foreignlanguage{hebrew}{הָכְתֵּב} \\
			\hline
			Part. & m. & sg. & \foreignlanguage{hebrew}{מֻכְתָּב} \\
			& & pl. & \foreignlanguage{hebrew}{מֻכְתָּבִים}\\
			& f. & sg. & \foreignlanguage{hebrew}{מֻכְתֶּבֶת} \\
			&  & pl. & \foreignlanguage{hebrew}{מֻכְתָּבוֺת} \\
			\hline
		\end{longtable}
	\end{center}
	
	\noindent \textbf{Notes}
	\nopagebreak
	
	\noindent In the suffix conjugation the primitive form was \textit{*huqṭila}. The thematic vowel changed from \textit{*i} to \textit{a} due to analogy to the prefix conjugation \textit{yuqṭal} and/or Philippi's Law (\textit{i > a} in stressed closed syllable).
	
	In the prefix conjugation the primitive form was \textit{*yuhuqṭala} resulting in \textit{yuqṭal} or \textit{yɔqṭal} in Tiberian Hebrew after the loss of the short final vowel and the elision of the intervocalic /h/.
	
	The imperative Ho.\ is attested only twice in the Hebrew Bible: \foreignlanguage{hebrew}{הָפְנוּ} \textit{be turned back}  (from the root \foreignlanguage{hebrew}{פנה}, Jer 49:8), \foreignlanguage{hebrew}{הָשְׁכְּבָה} \textit{put yourself to bed} (Ezek 32:19).
	
	% Include a note about pausal forms (see the exercises)
	
	\subsection{Guttural Verbs in the Hophal}
	
	\subsubsection{Verbs I gutt.}
	
	The prefix vowel of I\,gutt.\ verbs is always /ɔ/ (\textit{qameṣ ḥatuf}). There are forms with \textit{ḥatef qameṣ} and forms with silent \textit{šwa} with the first root consonant, e.g., \foreignlanguage{hebrew}{הָהְפַּךְ} \textit{it has been turned upon} (Job 30:15) \foreignlanguage{hebrew}{יָעֳמַד}, \textit{it shall be placed} (Lev 16:10)
	
	\subsubsection{Verbs II gutt.\ and III\,gutt.}
	
	The attested forms do not show differences to the regular strong verb.
	
	
	\section{Wishes with \foreignlanguage{hebrew}{מִי יִתֵּן}}
	
	The interrogative pronoun \foreignlanguage{hebrew}{מִי} \textit{who?} and the verbal form \foreignlanguage{hebrew}{יִתֵּן} (3 m.\ sg.\ prefix conjugation of \foreignlanguage{hebrew}{נתן}) are used idiomatically with the meaning \textit{Would that it may be so!}\ to express wishes.
	
	\vspace{0.5cm}
	
	\begin{tabular}{>{\raggedleft}p{0.35\linewidth} p{0.55\linewidth}}
		\foreignlanguage{hebrew}{וַיֹּאמְרוּ אֲלֵהֶם בְּנֵי יִשְׂרָאֵל מִי־יִתֵּן מוּתֵנוּ בְיַד־יְהוָה בְּאֶרֶץ מִצְרַיִם} & \textit{And the Israelites said to them, \enquote{Would that we had died through the hand of the Lord in the land of Egypt.}} (Exod 16:3) \\
		\foreignlanguage{hebrew}{בַּבֹּקֶר תֹּאמַר מִי־יִתֵּן עֶרֶב וּבָעֶרֶב תֹּאמַר מִי־יִתֵּן בֹּקֶר} & \textit{In the morning you will say, \enquote{If only it were evening}, and in the evening you will say, \enquote{If only it were morning}} (Deut 28:67) \\
	\end{tabular}
	
	
	\section{The Forms \foreignlanguage{hebrew}{וַיְהִי} and \foreignlanguage{hebrew}{וְהָיָה} as Discourse Markers}
	
	The forms \foreignlanguage{hebrew}{וַיְהִי} (3 m.\ sg.\ \textit{wayyiqṭol}) and \foreignlanguage{hebrew}{וְהָיָה} (3 m.\ sg.\ \textit{wəqaṭaltí}) of the verb \foreignlanguage{hebrew}{היה} may be used as ordinary verbs in normal verbal clauses with a subject and a predicate.
	
	\vspace{0.5cm}
	
	\begin{tabular}{>{\raggedleft}p{0.35\linewidth} p{0.55\linewidth}}
		\foreignlanguage{hebrew}{וַיְהִי רָעָב בָּאָרֶץ} & \textit{And there was a famine in the land} (Gen 12:10) \\
		\foreignlanguage{hebrew}{וַיֹּאמֶר יִשְׂרָאֵל אֶל־יוֹסֵף הִנֵּה אָנֹכִי מֵת וְהָיָה אֱלֹהִים עִמָּכֶם} & \textit{And Israel said to Joseph, \enquote{I am about to die, but God will be with you}} (Gen 48:21) \\
	\end{tabular}
	
	\vspace{0.5cm}
	
	In many cases, however, the forms \foreignlanguage{hebrew}{וַיְהִי} and \foreignlanguage{hebrew}{וְהָיָה} are not used as normal verb forms but as discourse markers to indicate the time reference for the following state of affairs as past with \foreignlanguage{hebrew}{וַיְהִי} or as future with \foreignlanguage{hebrew}{וְהָיָה}. Usually these forms are followed by a temporal adverbial phrase or a temporal clause (Gen 26:32; Hos 1:5). There are quite many cases of \foreignlanguage{hebrew}{וְהָיָה} in a future time discourse before a normal main clause (Gen 4:14).
	
When used as discourse markers, \foreignlanguage{hebrew}{וַיְהִי} and \foreignlanguage{hebrew}{וְהָיָה} should not be translated.
	
% Other examples: Josh 2:19; for more information Bartelmus, HYH, pp. 211ff.
	

	\begin{longtable}{>{\raggedleft}p{0.35\linewidth} p{0.55\linewidth}}
		\foreignlanguage{hebrew}{וַיְהִי בַּיּוֹם הַהוּא וַיָּבֹאוּ עַבְדֵי יִצְחָק} & \textit{On that day the servants of Isaac came} (Gen 26:32) \\
		\foreignlanguage{hebrew}{וְהָיָה בַּיּוֹם הַהוּא וְשָׁבַרְתִּי אֶת־קֶשֶׁת יִשְׂרָאֵל בְּעֵמֶק יִזְרְעֶאל} & \textit{On that day I will break the bow of Israel in the plains of Jezreel} (Hos 1:5) \\
		\foreignlanguage{hebrew}{וְהָיָה כָל־מֹצְאִי יַהַרְגֵנִי} & \textit{And whoever finds me will kill me} (Gen 4:14) \\
	\end{longtable}
	
In line with the possible meanings of \textit{waw}-SC (\textit{wəqaṭaltí}) forms, \foreignlanguage{hebrew}{וְהָיָה} may be used as a discourse marker for the past time with iterative meaning (Exod 33:9; Judg 19:30).
	
	% Action : Look for a better example than Exod 33:9
	
	\begin{longtable}{>{\raggedleft}p{0.35\linewidth} p{0.55\linewidth}}
		\foreignlanguage{hebrew}{וְהָיָה כְּבֹא מֹשֶׁה הָאֹהֱלָה יֵרֵד עַמּוּד הֶעָנָן וְעָמַד פֶּתַח הָאֹהֶל וְדִבֶּר עִם־מֹשֶה} & \textit{When Moses entered the tent, the pillar of cloud would come down and stand at the entrance of the tent and he [the Lord] would speak with Moses} (Exod 33:9) \\
		\foreignlanguage{hebrew}{וְהָיָה כָל־הָרֹאֶה וְאָמַר לֹא־נִהְיְתָה וְלֹא־נִרְאֲתָה כָּזֹאת לְמִיּוֹם עֲלוֹת בְּנֵי־יִשְׂרָאֵל מֵאֶרֶץ מִצְרַיִם עַד הַיּוֹם הַזֶּה} & \textit{Anybody who saw it, would say \enquote{Something like this has not happened or been seen since the day the Israelites went up from the land of Egypt until this day}} (Judg 19:30) \\
	\end{longtable}
	
	
	
	\section{Exercises}
	
	\subsection{Parsing of Verbal Forms}
	
	Parse the following verbal forms by identifying person, number, gender --~if applicable~-- and the conjugation (PC, SC, jussive, \textit{wayyiqṭol}, impv., inf.\ cs., inf.\ abs., part.) and the binyan. Translate the finite forms. Mark the stressed syllable if stress is not on the last syllable.
	
	\hspace{0.5cm}
	
	\selectlanguage{hebrew}
	
	\noindent
	1~~\foreignlanguage{hebrew}{יָחֳרָם}  \hspace{0.3cm}
	2~~\foreignlanguage{hebrew}{הָפְקַד}  \hspace{0.3cm}
	3~~\foreignlanguage{hebrew}{יָעֳמַד}  \hspace{0.3cm}
	4~~\foreignlanguage{hebrew}{מֻשְׁלָךְ}  \hspace{0.3cm}
	5~~\foreignlanguage{hebrew}{מֻשְׁלֶכֶת}  \hspace{0.3cm}
	6~~\foreignlanguage{hebrew}{מָעֳמָד}  \hspace{0.3cm}
	7~~\foreignlanguage{hebrew}{מֻשְׁכָּב}  \hspace{0.3cm}
	8~~\foreignlanguage{hebrew}{הָשְׁלַכְתִּי}  \hspace{0.3cm}
	9~~\foreignlanguage{hebrew}{מָשְׁחָת}  \hspace{0.3cm}
	10~~\foreignlanguage{hebrew}{יֻשְׁלָ֑כוּ}  \hspace{0.3cm}
	
	\selectlanguage{english}
	
	
	\subsection{Translation of Sentences}
	
	Translate the following sentences from the Hebrew Bible. Names of persons and geographical names in these sentences: \foreignlanguage{hebrew}{אַבְשָׁלוֹם}, \foreignlanguage{hebrew}{אַחְאָב}, \foreignlanguage{hebrew}{אֱלִישָׁע}, \foreignlanguage{hebrew}{בֵּית־עֵקֶד}, \foreignlanguage{hebrew}{גֶּלְגָּל}, \foreignlanguage{hebrew}{דָוִד}, \foreignlanguage{hebrew}{יְהוּדָה}, \foreignlanguage{hebrew}{יְהוֹרָם}, \foreignlanguage{hebrew}{יְהוֹשָׁפָט}, \foreignlanguage{hebrew}{יוֺסֵף}, \foreignlanguage{hebrew}{יוֺרָם}, \foreignlanguage{hebrew}{יַעֲקֹב}, \foreignlanguage{hebrew}{יַרְדֵּן}, \foreignlanguage{hebrew}{צִיּוֺן}, \foreignlanguage{hebrew}{שָׁאוּל}.
	
	\vspace{0.5cm}
	
	\selectlanguage{hebrew}
	\noindent
	1~~\foreignlanguage{hebrew}{וְהַשָּׂעִיר אֲשֶׁר עָלָה עָלָיו הַגּוֹרָל} \enquote{\foreignlanguage{hebrew}{לַעֲזָאזֵל}}\LTRfootnote{\space \foreignlanguage{hebrew}{עֲזָאזֵל} \textit{Azazel}, name of a demon in the wilderness (\textit{HALOT}); \foreignlanguage{hebrew}{לַעֲזָאזֵל} is what was written on the \foreignlanguage{hebrew}{גּוֺרָל}.} \foreignlanguage{hebrew}{יָעֳמַד־חַי לִפְנֵי יְהוָה} \hspace{0.3cm}
	2~~\foreignlanguage{hebrew}{וַיָּבֹא אֱלִישָׁע הַבָּ֫יְתָה וְהִנֵּה הַנַּעַר מֵת}\LTRfootnote{\space \foreignlanguage{hebrew}{מֵת} \textit{dead} (part. of \foreignlanguage{hebrew}{מות}; cf. 17.2)} \foreignlanguage{hebrew}{מֻשְׁכָּב עַל־מִטָּתוֹ}\LTRfootnote{\space \foreignlanguage{hebrew}{מִטָּה} \textit{couch, bed}} \hspace{0.3cm}
	3~~\foreignlanguage{hebrew}{וַיִּרְגַּז}\LTRfootnote{\space \foreignlanguage{hebrew}{רגז} Qal \textit{to tremble} (with emotion)} \foreignlanguage{hebrew}{הַמֶּלֶךְ וַיַּעַל עַל־עֲלִיַּת}\LTRfootnote{\space \foreignlanguage{hebrew}{עֲלִיָּה} \textit{upper room, room in an upper story}} \foreignlanguage{hebrew}{הַשַּׁעַר וַיֵּבְךְּ וְכֹה אָמַר בְּלֶכְתּוֹ בְּנִי אַבְשָׁלוֹם בְּנִי בְנִי אַבְשָׁלוֹם מִי־יִתֵּן מוּתִי אֲנִי תַחְתֶּיךָ אַבְשָׁלוֹם בְּנִי בְנִי} \hspace{0.3cm}
	4~~\foreignlanguage{hebrew}{מִי יִתֵּן מִצִּיּוֹן יְשׁוּעַת יִשְׂרָאֵל. בְּשׁוּב}\LTRfootnote{\space \foreignlanguage{hebrew}{שׁוב} with the direct object \foreignlanguage{hebrew}{שְׁבוּת} \textit{to turn someone's fortune, bring about change} (\textit{HALOT})} \foreignlanguage{hebrew}{יְהוָה שְׁבוּת עַמּוֹ, יָגֵל}\LTRfootnote{\space \foreignlanguage{hebrew}{גיל} Qal \textit{to shout in exultation, rejoice}} \foreignlanguage{hebrew}{יַעֲקֹב, יִשְׂמַח יִשְׂרָאֵל.} \hspace{0.3cm}
	5~~\foreignlanguage{hebrew}{וַיֹּאמֶר תִּפְשׂוּם חַיִּים וַיִּתְפְּשׂוּם חַיִּים וַיִּשְׁחָטוּם אֶל־בּוֹר בֵּית־עֵקֶד אַרְבָּעִים וּשְׁנַיִם אִישׁ וְלֹא־הִשְׁאִיר אִישׁ מֵהֶם}  \hspace{0.3cm}
	6~~\foreignlanguage{hebrew}{וַיָּשָׁב הַמֶּלֶךְ וַיָּבֹא עַד־הַיַּרְדֵּן וִיהוּדָה בָּא הַגִּלְגָּ֗לָה לָלֶכֶת לִקְרַאת הַמֶּלֶךְ לְהַעֲבִיר אֶת־הַמֶּלֶךְ אֶת־הַיַּרְדֵּן}  \hspace{0.3cm}
	7~~\foreignlanguage{hebrew}{לֹא־תַסְגִּיר עֶבֶד אֶל־אֲדֹנָיו אֲשֶׁר־יִנָּצֵל אֵלֶיךָ מֵעִם אֲדֹנָיו. עִמְּךָ יֵשֵׁב בְּקִרְבְּךָ בַּמָּקוֹם אֲשֶׁר־יִבְחַר בְּאַחַד שְׁעָרֶיךָ בַּטּוֹב לוֹ. לֹא תּוֹנֶנּוּ}\LTRfootnote{\space \foreignlanguage{hebrew}{לֹא תּוֹנֶנּוּ} \textit{You shall not oppress him} (verb \foreignlanguage{hebrew}{ינה} Hi.)} \hspace{0.3cm}
	8~~\foreignlanguage{hebrew}{וַיַּחֲרִימוּ אֶת־כָּל־אֲשֶׁר בָּעִיר מֵאִישׁ וְעַד־אִשָּׁה מִנַּעַר וְעַד־זָקֵן וְעַד שׁוֹר וָשֶׂה וַחֲמוֹר לְפִי־חָרֶב}  \hspace{0.3cm}
	9~~\foreignlanguage{hebrew}{שׁוּבָה יִשְׂרָאֵל עַד יְהוָה אֱלֹהֶיךָ כִּי כָשַׁלְתָּ בַּעֲוֹנֶךָ}  \hspace{0.3cm}
	10~~\foreignlanguage{hebrew}{וַיֵּצֵא דָוִד, בְּכֹל אֲשֶׁר יִשְׁלָחֶנּוּ שָׁאוּל יַשְׂכִּיל וַיְשִׂמֵהוּ שָׁאוּל עַל אַנְשֵׁי הַמִּלְחָמָה וַיִּיטַב בְּעֵינֵי כָל־הָעָם וְגַם בְּעֵינֵי עַבְדֵי שָׁאוּל}  \hspace{0.3cm}
	11~~\foreignlanguage{hebrew}{וַיְהִי יְהוָה אֶת־יוֹסֵף וַיְהִי אִישׁ מַצְלִיחַ וַיְהִי בְּבֵית אֲדֹנָיו הַמִּצְרִי וַיַּרְא אֲדֹנָיו כִּי יְהוָה אִתּוֹ וְכֹל אֲשֶׁר־הוּא עֹשֶׂה יְהוָה מַצְלִיחַ בְּיָדוֹ}  \hspace{0.3cm}
	12~~\foreignlanguage{hebrew}{וּבִשְׁנַת חָמֵשׁ לְיוֹרָם בֶּן־אַחְאָב מֶלֶךְ יִשְׂרָאֵל וִיהוֹשָׁפָט מֶלֶךְ יְהוּדָה מָלַךְ יְהוֹרָם בֶּן־יְהוֹשָׁפָט מֶלֶךְ יְהוּדָה. בֶּן־שְׁלֹשִׁים וּשְׁתַּיִם שָׁנָה הָיָה בְמָלְכוֹ וּשְׁמֹנֶה שָׁנִים מָלַךְ בִּירוּשָׁלָֽ͏ִם. וַיֵּלֶךְ בְּדֶרֶךְ מַלְכֵי יִשְׂרָאֵל כַּאֲשֶׁר עָשׂוּ בֵּית אַחְאָב כִּי בַּת־אַחְאָב הָיְתָה־לּוֹ לְאִשָּׁה וַיַּעַשׂ הָרַע בְּעֵינֵי יְהוָה. וְלֹא־אָבָה יְהוָה לְהַשְׁחִית אֶת־יְהוּדָה לְמַעַן דָּוִד עַבְדּוֹ כַּאֲשֶׁר אָמַר־לוֹ לָתֵת לוֹ נִיר}\LTRfootnote{\space \foreignlanguage{hebrew}{נִיר} \textit{light, lamp}} \foreignlanguage{hebrew}{לְבָנָיו כָּל־הַיָּמִים.} \hspace{0.3cm}
	
	\selectlanguage{english}
	
	
	
	\section{Hebrew Reading: Genesis 13:1--7}
	Translate Gen 13:1--7 with the help of notes below the text. Names of persons and places are easily identifiable.
	
	\vspace{0.5cm}
	
	\selectlanguage{hebrew}
	\noindent
	\textsuperscript{1}~\foreignlanguage{hebrew}{וַיַּעַל אַבְרָם מִמִּצְרַיִם הוּא וְאִשְׁתּוֹ וְכָל־אֲשֶׁר־לוֹ וְלוֹט עִמּוֹ הַנֶּֽגְבָּה׃} \hspace{0.3cm}
	\textsuperscript{2}~\foreignlanguage{hebrew}{וְאַבְרָם כָּבֵד מְאֹ֑ד בַּמִּקְנֶה בַּכֶּסֶף וּבַזָּהָֽב׃} \hspace{0.3cm}
	\textsuperscript{3}~\foreignlanguage{hebrew}{וַיֵּלֶךְ לְמַסָּעָ֔יו מִנֶּגֶב וְעַד־בֵּית־אֵ֑ל עַד־הַמָּקוֹם אֲשֶׁר־הָיָה שָׁם אָהֳלוֹ בַּתְּחִלָּ֔ה בֵּין בֵּית־אֵל וּבֵין הָעָֽי׃} \hspace{0.3cm}
	\textsuperscript{4}~\foreignlanguage{hebrew}{אֶל־מְקוֹם הַמִּזְבֵּ֔חַ אֲשֶׁר־עָשָׂה שָׁם בָּרִאשֹׁנָ֑ה וַיִּקְרָא שָׁם אַבְרָם בְּשֵׁם יְהוָֽה׃} \hspace{0.3cm}
	\textsuperscript{5}~\foreignlanguage{hebrew}{וְגַם־לְל֔וֹט הַהֹלֵךְ אֶת־אַבְרָ֑ם הָיָה צֹאן־וּבָקָר וְאֹהָלִֽים׃} \hspace{0.3cm}
	\textsuperscript{6}~\foreignlanguage{hebrew}{וְלֹא־נָשָׂא אֹתָם הָאָרֶץ לָשֶׁבֶת יַחְדָּ֑ו כִּי־הָיָה רְכוּשָׁם רָ֔ב וְלֹא יָכְלוּ לָשֶׁבֶת יַחְדָּֽו׃} \hspace{0.3cm}
	\textsuperscript{7}~\foreignlanguage{hebrew}{וַיְהִי־רִיב בֵּין רֹעֵי מִקְנֵה־אַבְרָ֔ם וּבֵין רֹעֵי מִקְנֵה־ל֑וֹט וְהַכְּנַעֲנִי וְהַפְּרִזִּ֔י אָז יֹשֵׁב בָּאָֽרֶץ׃} \hspace{0.3cm}
	\selectlanguage{english}
	
	
	
	\hspace*{-0.5cm}\begin{longtable}{p{0.075\linewidth} p{0.1\linewidth}p{0.725\linewidth}}
		13:2 & \foreignlanguage{hebrew}{כָּבֵד}  & \textit{heavy, weighty, rich} \\
		13:3 & \foreignlanguage{hebrew}{לְמַסָּעָיו} & \textit{by stages} (\foreignlanguage{hebrew}{מַסַּע} \textit{breaking camp; departure})  \\
		& \foreignlanguage{hebrew}{תְּחִלָּה} & \textit{beginning} \\
		& \foreignlanguage{hebrew}{בָּרִאשֹׁנָה} & \textit{before, formerly} \\
		13:6 & \foreignlanguage{hebrew}{וְלֹא־נָשָׂא ... הָאָרֶץ} & lack of agreement in gender between the predicate and the subject \\
		& \foreignlanguage{hebrew}{רְכוּשׁ} & \textit{possession; property; goods} \\
		13:7 & \foreignlanguage{hebrew}{רֹעֶה} & \textit{shepherd} (participle of \foreignlanguage{hebrew}{רעה}) \\
	\end{longtable}
	
	
	
	
	\chapter{Chapter 19}
	
	\renewcommand\arraystretch{1.4}
	
	\section{Vocabulary}
	
	\subsection{Verbs}
	
	\begin{center}
		
		% For the centering of the separation between the two columns see the documentation of the array package, page 2 
		
		\begin{longtable}{>{\raggedleft}p{0.175\linewidth} p{0.75\linewidth}}
			\foreignlanguage{hebrew}{ארר} & Q.\ \textit{to curse} \\
			\foreignlanguage{hebrew}{בקשׁ} & Pi.\ \textit{to seek}  \\ % BDB
			\foreignlanguage{hebrew}{ברך} & Pi.\ \textit{to bless}  \\ % HALOT
			\foreignlanguage{hebrew}{גדל} & Q.\ \textit{to be great, become great, grow up} (stative verb; PC \foreignlanguage{hebrew}{יִגְדַּל}); Hi.\ \textit{to make great; to make great things}; Pi.\ \textit{to cause to grow; make great, powerful} \\
			\foreignlanguage{hebrew}{גרשׁ} & Pi.\ \textit{to drive out} \\
			\foreignlanguage{hebrew}{דבר} & Pi.\ \textit{to speak} \\
			\foreignlanguage{hebrew}{כבס} & Pi.\ \textit{to wash} (normally clothes) (SC \foreignlanguage{hebrew}{כִּבֶּס})  \\
			\foreignlanguage{hebrew}{למד} & Q.\ \textit{to learn}; Pi.\ \textit{to teach}  \\
			\foreignlanguage{hebrew}{מאן} & Pi.\ \textit{to refuse}  \\ % Only 46 occurrences, but mainly in Gen--Jer
			\foreignlanguage{hebrew}{מהר} & Pi.\ \textit{to hasten; to do something hastily}  \\
			\foreignlanguage{hebrew}{ספר} & Q.\ \textit{to count}; Pi.\ \textit{to report; tell; to make known, announce}  \\
			\foreignlanguage{hebrew}{פתח} & Q.\ \textit{to open}; Pi.\ \textit{to let loose; to untie; to liberate}  \\
			\foreignlanguage{hebrew}{קדשׁ} & Q.\ \textit{to be holy} (stative verb; SC pausal form \foreignlanguage{hebrew}{קָדֵ֑שׁוּ}; PC \foreignlanguage{hebrew}{יִקְדַּשׁ}); Pi.\ \textit{to sanctify, make holy; to declare holy; to dedicate}; Hi.\ \textit{to set apart, devote, consecrate; regard} or \textit{treat as sacred} \\
			\foreignlanguage{hebrew}{קטר} & Hi.\ \textit{to cause to go up in smoke}; Pi.\ \textit{to make a sacrifice go up in}  \\
			& \textit{smoke} \\ % HALOT
			\foreignlanguage{hebrew}{קלל} & Q.\ \textit{to be small, insignificant; to be swift}; Pi.\ \textit{to curse}  \\
			\foreignlanguage{hebrew}{שׁחת} & Hi./Pi.\ \textit{to ruin, destroy; annihilate, exterminate} \\ % HALOT
			\foreignlanguage{hebrew}{שׁלח} & Q.\ \textit{to send}; Pi. \textit{to send off; to send out, forth; to let go, set free}  \\ % BDB
			\foreignlanguage{hebrew}{שׁלם} & Pi.\ \textit{to make whole} or \textit{good, restore; make good}, i.e., \textit{pay} (vows); \textit{to requite, recompense, reward} \\ % BDB
			\foreignlanguage{hebrew}{שׁפך} & Q.\ \textit{to pour, pour out}  \\
			\foreignlanguage{hebrew}{שׁרת} & Pi.\ \textit{to serve} \\ % BDB
		\end{longtable}
	\end{center}
	
	\subsection{Nouns}
	
	\begin{center}
		\begin{longtable}{w{r}{3cm}w{l}{12cm}}
			\foreignlanguage{hebrew}{אֹרֶךְ} & \textit{length} \\
			\foreignlanguage{hebrew}{בְּרָכָה} & \textit{blessing} (cs.\ st.\ \foreignlanguage{hebrew}{בִּרְכַּת}, but with ePP \foreignlanguage{hebrew}{בִּרְכָתִי}) \\
			\foreignlanguage{hebrew}{חֻקָּה} & \textit{statute} \\
			\foreignlanguage{hebrew}{טָהוֺר} & \textit{pure; ceremionally clean} (adj.) \\
			\foreignlanguage{hebrew}{יֶתֶר} & \textit{rest, remainder} \\
			\foreignlanguage{hebrew}{כֶּבֶשׂ} & \textit{young ram} \\ % HALOT
			\foreignlanguage{hebrew}{עוֺר} & \textit{skin; animal skin; leather} \\
			\foreignlanguage{hebrew}{קְטֹרֶת} & \textit{smoke, odor of (burning) sacrifice; incense} (fem.) \\ % HALOT
			\foreignlanguage{hebrew}{רֹחַב} & \textit{breadth, width} \\
			\foreignlanguage{hebrew}{רָחוֺק} & \textit{distant, far} (adj.) \\
			\foreignlanguage{hebrew}{שִׂמְחָה} & \textit{joy, gladness} \\ % BDB
		\end{longtable}
	\end{center}
	
	\subsection{Other Parts of Speech}
	
	\begin{center}
		\begin{tabular}{w{r}{3cm}w{l}{12cm}}
			\foreignlanguage{hebrew}{מַהֵר} & \textit{quickly, speedily} (adv.; inf.\ abs.\ of \foreignlanguage{hebrew}{מהר} Pi.\ \textit{to hasten}) \\
			\foreignlanguage{hebrew}{מָחָר} & \textit{tomorrow} (adv.) \\
		\end{tabular}
	\end{center}
	
	
	\section{The Piel Binyan}
	
	\subsection{The Forms of the Strong Verb in the Piel}
	
	The Piel is characterized by the gemination (doubling) of the second root consonant R\textsubscript{2}.
	
	
	\begin{center}
		\begin{longtable}{|lll|r|}
			\hline
			SC & sg. & 3 m. & \foreignlanguage{hebrew}{כִּתֵּב} \\
			& & 3 f. & \foreignlanguage{hebrew}{כִּתְּבָה} \\
			& & 2 m. & \foreignlanguage{hebrew}{כִּתַּ֫בְתָּ} \\
			& & 2 f. & \foreignlanguage{hebrew}{כִּתַּבְתְּ} \\
			& & 1 c. & \foreignlanguage{hebrew}{כִּתַּ֫בְתִּי} \\
			\hline
			\pagebreak
			\hline
			& pl. & 3 c. & \foreignlanguage{hebrew}{כִּתְּבוּ} \\
			& & 2 m. & \foreignlanguage{hebrew}{כִּתַּבְתֶּם} \\
			& & 2 f. & \foreignlanguage{hebrew}{כִּתַּבְתֶּן} \\
			& & 1 c. & \foreignlanguage{hebrew}{כִּתַּבְנוּ} \\
			\hline
			PC & sg. & 3 m. & \foreignlanguage{hebrew}{יְכַתֵּב} \\
			& & 3 f. & \foreignlanguage{hebrew}{תְּכַתֵּב} \\
			& & 2 m. & \foreignlanguage{hebrew}{תְּכַתֵּב} \\
			& & 2 f. & \foreignlanguage{hebrew}{תְּכַתְּבִי} \\
			& & 1 c. & \foreignlanguage{hebrew}{אֲכַתֵּב} \\
			& pl. & 3 c. & \foreignlanguage{hebrew}{יְכַתְּבוּ} \\
			& & 3 f. & \foreignlanguage{hebrew}{תְּכַתֵּבְנָה} \\
			& & 2 m. & \foreignlanguage{hebrew}{תְּכַתְּבוּ} \\
			& & 2 f. & \foreignlanguage{hebrew}{תְּכַתֵּבְנָה} \\
			& & 1 c. & \foreignlanguage{hebrew}{נְכַתֵּב} \\
			\hline
			Jussive & sg. & 3 m. & \foreignlanguage{hebrew}{יְכַתֵּב} \\
			\textit{wayyiqṭol} & sg. & 3 m. & \foreignlanguage{hebrew}{וַיְכַתֵּב} \\
			\hline
			Impv. & sg. & 2 m. & \foreignlanguage{hebrew}{כַּתֵּב} \\
			& & & \foreignlanguage{hebrew}{כַּתְּבָה} \\
			& & 2 f. & \foreignlanguage{hebrew}{כַּתְּבִי} \\
			& pl. & 2 m. & \foreignlanguage{hebrew}{כַּתְּבוּ} \\
			& & 2 f. & \foreignlanguage{hebrew}{כַּתֵּבְנָה} \\
			\hline
			Inf.\ cs. & & & \foreignlanguage{hebrew}{כַּתֵּב} \\
			Inf.\ abs. & & & \foreignlanguage{hebrew}{כַּתֹּב}/\foreignlanguage{hebrew}{כַּתֵּב} \\
			\hline
			Part. & m. & sg. & \foreignlanguage{hebrew}{מְכַתֵּב} \\
			& & pl. & \foreignlanguage{hebrew}{מְכַתְּבִים} \\
			& f. & sg. & \foreignlanguage{hebrew}{מְכַתֵּבָה}/\foreignlanguage{hebrew}{מְכַתֶּבֶת} \\
			& & pl. & \foreignlanguage{hebrew}{מְכַתְּבוֺת} \\
			\hline
		\end{longtable}
	\end{center}
	
	\vspace{0.5cm}
	
	\noindent \textbf{Notes}
	\nopagebreak
	
	\noindent In the second syllable of 3 m.\ sg.\ suffix conjugation forms three different vowels are attested:
	
	\begin{itemize}[noitemsep]
		\item[--] \textit{ṣere} is the most common vowel, e.g., \foreignlanguage{hebrew}{בִּקֵּשׁ} Pi.\ \textit{he sought} (in pausal forms one only finds \textit{ṣere}).
		\item[--] \textit{pataḥ} is less common, e.g., \foreignlanguage{hebrew}{קִדַּשׁ} Pi.\ \textit{he sanctified}.
		\item[--] \textit{səgol} with the three verbs \foreignlanguage{hebrew}{דִּבֶּר} Pi.\ \textit{he spoke}, \foreignlanguage{hebrew}{כִּבֶּס} Pi.\ \textit{he washed} (clothes) and \foreignlanguage{hebrew}{כִּפֶּר} Pi.\ \textit{he appeased, made atonement}.
	\end{itemize}
	
	In \textit{wayyiqṭol} forms 3 m.\ sg./pl.\ the prefix consonant /y/ looses its gemination because of the vocal \textit{šwa}, e.g., \foreignlanguage{hebrew}{וַיְדַבֵּר} \textit{and he spoke}.
	
	Geminate verbs have strong Piel forms with consonants in all three slots, e.g., \foreignlanguage{hebrew}{קִלֵּל} (SC) \textit{he cursed} (2\,Sam 19:22), \foreignlanguage{hebrew}{לְקַלֵּל} (inf.\ cs.) \textit{to curse} (Josh 24:9).
	
	If R\textsubscript{2} is a sibilant (i.e., \foreignlanguage{hebrew}{ס}, \foreignlanguage{hebrew}{צ}, \foreignlanguage{hebrew}{שׂ} and \foreignlanguage{hebrew}{שׁ}) or \foreignlanguage{hebrew}{מ}, \foreignlanguage{hebrew}{נ}, \foreignlanguage{hebrew}{ק}, \foreignlanguage{hebrew}{י}, and \foreignlanguage{hebrew}{ל} and requires a vocal \textit{šwa}, the gemination of R\textsubscript{2} is frequently lost. The following examples illustrate this phenomenon:
	
\vspace{0.25cm}
	
	\begin{center}
		\begin{tabular}{lrl}
			suffix conj. & \foreignlanguage{hebrew}{בִּקְשׁוּ} & 3 m.\ pl. \\
			prefix conj & \foreignlanguage{hebrew}{תְּבַקְשִׁי} & 2 f.\ sg. \\
			\textit{wayyiqṭol} & \foreignlanguage{hebrew}{וַיְבַקְשׁוּ} & 3 m.\ pl. \\
			impv. & \foreignlanguage{hebrew}{בַּקְשׁוּ} & m.\ pl. \\
			inf.\ cs. & \foreignlanguage{hebrew}{לְבַקְשֵׁ֫נִי} & with enclitic pronoun \\
			participle & \foreignlanguage{hebrew}{מְבַקְשִׁים} & m.\ pl. \\
		\end{tabular}
	\end{center}

\vspace{0.25cm}

The primitive form of the suffix conjugation form was possibly \textit{*kattaba}. In the 3 m.\ sg.\ form the vowel \textit{ṣere} of the second syllable is due to analogy to the prefix conjugation form \foreignlanguage{hebrew}{יְכַתֵּב}.\footnote{\space Alternatively the primitive vowel may have been \textit{kittibu > kittib > kittēḇ} with the vowel change \textit{*i > a} in forms with consonantal suffix due to Philippi's law (\textit{*i > a} in closed stressed syllable), e.g., \textit{kittaḇtæm} \foreignlanguage{hebrew}{כִּתַּבְתֱּם}.}
	
The primitive form of the prefix conjugation form was \textit{*yukattib > *yakattib} which developed to \foreignlanguage{hebrew}{יְכַּתֵּב} in Tiberian Masoretic Hebrew. The same primitive vowels were present in the forms of imperative, the inf.\ cs.\ and participle.
	
	
	\subsection{The Meanings of Verbs in the Piel}
	
	Piel verbs have the following main meanings compared to the Qal:
	
	\vspace{0.25cm}
	
	\begin{itemize}[noitemsep]
		\item[--] \textit{Factitive:} The Piel verb expresses that a state is brought about: \foreignlanguage{hebrew}{גדל} Q.\ \textit{to be, become great}; \foreignlanguage{hebrew}{גדל} Pi.\ \textit{to make great}. The factitive meaning is the most common one.
		\item[--] \textit{Intensive:} The Piel verb expresses an intensified action compared to the Qal: \foreignlanguage{hebrew}{שׁבר} Q.\ \textit{to break}; \foreignlanguage{hebrew}{שׁבר} Pi.\ \textit{to shatter, break}. (Sometimes, the difference in meaning may be difficult to discern.)
		\item[--] \textit{Iterative:} A repeated action or an action that affects a number of objects is expressed: \foreignlanguage{hebrew}{קבר} Q.\ \textit{to bury}; \foreignlanguage{hebrew}{קבר} Pi.\ \textit{to bury many}.
		\item[--] \textit{Declarative-estimative:} The Piel verb expresses that the object is in a certain state: \foreignlanguage{hebrew}{קדשׁ} Qal \textit{to be holy}; \foreignlanguage{hebrew}{קדשׁ} Pi.\ \textit{to declare holy}. (JM §\,52\,\textit{d}: \enquote{This may be subsumed under factitive in the sense that, whilst the factitive denotes the generation of a state or quality actually and physically, the declarative-estimative does so mentally or verbally.})
		\item[--] \textit{Denominative:} The verb expresses an event that is derived from a noun: \foreignlanguage{hebrew}{כֹּהֵן} \textit{priest}; \foreignlanguage{hebrew}{כהן} Pi.\ \textit{to serve as priest}. Denominative verbs in the Piel may have privative meaning: \foreignlanguage{hebrew}{דֶּשֶׁן} \textit{fat}; \foreignlanguage{hebrew}{דשׁן} Pi.\ \textit{to take away, clear the fat ashes} (besides the meaning \textit{to make fat}).
		\item[--] \textit{Without obvious Piel meaning:} With a number of verbs, the Piel seems to be used without a particular Piel meaning, e.g.,  \foreignlanguage{hebrew}{ארשׂ} Pi.\ \textit{to betroth}, \foreignlanguage{hebrew}{צוה} Pi.\ \textit{to command}.
	\end{itemize}
	
	\vspace{0.5cm}
	
	\noindent If the meaning of a verb in the Qal is known, these semantic categories should be considered for the translation of the verb in the Piel. A lexicon will help in further specifying the meaning.
	
	\subsection{Verbs with Gutturals or /r/ in the Piel Binyan}
	
	\subsubsection{Verbs with Gutturals or /r/ as R\textsubscript{2}}
	
	In Biblical Hebrew as it has been transmitted by the Masoretes, the guttural consonants \foreignlanguage{hebrew}{א}~/ʾ/, \foreignlanguage{hebrew}{ה} /h/, \foreignlanguage{hebrew}{ח} /ḥ/ and \foreignlanguage{hebrew}{ע} /ʿ/ and the consonant \foreignlanguage{hebrew}{ר} /r/ are not geminated. At earlier stages of the Hebrew language, however, they could be doubled, but this possibility was lost in stages over time. Depending on when it was lost, Hebrew did either tolerate short syllables in now open unstressed syllables after the loss of the gemination or lengthened the vowel in the now open syllable.
	
	\vspace{0.5cm}
	
	\noindent The following verbs with gutturals as R\textsubscript{2} preserve the short vowels /i/ or /a/:
	
	\begin{itemize}[noitemsep]
		\item[--] Almost always: verbs with  \foreignlanguage{hebrew}{ח} /ḥ/, e.g., \foreignlanguage{hebrew}{רִחַמְתִּיךְ} \textit{I have compassion on you}, \foreignlanguage{hebrew}{יְרַחֵם} \textit{he will have compassion}
		\item[--] Mostly: verbs with \foreignlanguage{hebrew}{ה} /h/, e.g., \foreignlanguage{hebrew}{נִהַגְתָּ} \textit{you led}, \textit{} \foreignlanguage{hebrew}{יְנַהֵג} \textit{he will lead}
		\item[--] Frequently: verbs with \foreignlanguage{hebrew}{ע} /ʿ/, e.g., \foreignlanguage{hebrew}{בִּעַרְתִּי} \textit{I have removed}, \foreignlanguage{hebrew}{יְבַעֵר} \textit{he will remove}
	\end{itemize}
	
	\noindent The following verbs with gutturals or \foreignlanguage{hebrew}{ר} /r/ lengthen the vowels /i/ and /a/ to /ē/ and /ā/, respectively:
	
	\begin{itemize}[noitemsep]
		\item[--] Always: verbs with \foreignlanguage{hebrew}{ר} /r/, e.g., \foreignlanguage{hebrew}{בֵּרַכְתִּי} \textit{I have blessed}, \foreignlanguage{hebrew}{תְּבָרֵךְ} \textit{you bless}
		\item[--] Relatively frequently: verbs with \foreignlanguage{hebrew}{א} /ʾ/, e.g., \foreignlanguage{hebrew}{מֵאֵן} \textit{he refused} (suffix conj.), \foreignlanguage{hebrew}{בֵּאֵר} \textit{he explained} (suffix conj.), but \foreignlanguage{hebrew}{בַּאֵר} (inf.\ abs.) and \foreignlanguage{hebrew}{נִאֲצוּ} \textit{they disregarded}
	\end{itemize}
	
	\noindent Where the strong verb without gutturals requires a vocal \textit{šwa} with R\textsubscript{2} (forms with vocalic suffixes or participles with plural ending) verbs with a guttural as R\textsubscript{2} have \textit{ḥaṭef pataḥ}, e.g., \foreignlanguage{hebrew}{נִאֲפָה} \textit{she committed adultery} as opposed to \foreignlanguage{hebrew}{דִּבְּרָה} \textit{she spoke}.
	
	When the vowel between R\textsubscript{1} and R\textsubscript{2} is lengthened, stress moves to the penultimate syllable in \textit{wayyiqṭol} forms, e.g., \foreignlanguage{hebrew}{וַיְבָ֫רֶךְ} \textit{and he blessed}. This is limited to context forms; retraction of stress does not occur in pausal forms.
	
	
	\subsubsection{Verbs with Gutturals as R\textsubscript{1} or R\textsubscript{3}}
	
	Verbs with gutturals as R\textsubscript{1} do not have different vowels than the regular strong verb.
	
	When R\textsubscript{3} is a guttural (except with weak verbs III\,\textit{ʾ}), the vowel /e/ is replaced with /a/ between R\textsubscript{2} and R\textsubscript{3} in forms of the suffix conjugation, the prefix conjugation and the imperative, e.g., \foreignlanguage{hebrew}{שִׁלַּח} \textit{he let go}, \foreignlanguage{hebrew}{וַיְשַׁלַּח} \textit{and he sent out}. In pausal forms, however, \textit{ṣere} is used (with furtive \textit{pataḥ}), e.g., \foreignlanguage{hebrew}{יְשַׁלֵּ֑חַ} \textit{he will set free}.
	
	
	
	\section{Left Dislocation of Clause Constituents}
	
	Left dislocation is the placement of a constituent outside and in front of the clause in which it is represented by a pronominal element, the so-called resumptive pronominal element. As a result, the dislocated constituent is separated from the clause. In the following example from Gen 3:12, the subject of the clause \foreignlanguage{hebrew}{הָאִשָּׁה אֲשֶׁר נָתַתָּה עִמָּדִי} is dislocated and taken up by the resumptive pronoun \foreignlanguage{hebrew}{הִיא} in the clause itself.
	
	\vspace{0.5cm}
	
	\begin{tabular}{>{\raggedleft}p{0.35\linewidth} p{0.55\linewidth}}
		\foreignlanguage{hebrew}{הָאִשָּׁה אֲשֶׁר נָתַתָּה עִמָּדִי הִיא נָתְנָה־לִּי מִן־הָעֵץ} & \textit{The woman that you put at my side -- she gave me from the tree} (Gen 3:12 Qere) \\
	\end{tabular}
	
	% Only the PP "at my side" is taken from the NJPS.
	
	\vspace{0.5cm}
	
	The dislocated element can have any syntactic function. In Gen 3:12 it is the subject (see above). In the case of Gen 28:13 it is the direct object, whereas in 2\,Kgs 1:6 it is a local complement.
	
	\vspace{0.5cm}
	
	\begin{tabular}{>{\raggedleft}p{0.35\linewidth} p{0.55\linewidth}}
		\foreignlanguage{hebrew}{הָאָרֶץ אֲשֶׁר אַתָּה שֹׁכֵב עָלֶיהָ לְךָ אֶתְּנֶנָּה וּלְזַרְעֶךָ} & \textit{The land you lie on -- I will give it to you and to your offspring} (Gen 28:13) \\
		\foreignlanguage{hebrew}{לָכֵן הַמִּטָּה אֲשֶׁר־עָלִיתָ שָּׁם לֹא־תֵרֵד מִמֶּנָּה כִּי־מוֹת תָּמוּת} & \textit{Therefore, the bed you went up -- you will not go down from it for you will surely die} (2\,Kgs 1:6) \\
	\end{tabular}
	
	\vspace{0.5cm}
	
	The dislocated element may be marked for its syntactic function in the clause itself even in its position before the clause. In Gen 13:15 the dislocated element is marked as direct object; in Gen 2:17 it is a prepositional phrase.
	
	\vspace{0.5cm}
	
	\begin{tabular}{>{\raggedleft}p{0.35\linewidth} p{0.55\linewidth}}
		\foreignlanguage{hebrew}{כִּי אֶת־כָּל־הָאָרֶץ אֲשֶׁר־אַתָּה רֹאֶה לְךָ אֶתְּנֶנָּה וּלְזַרְעֲךָ עַד־עוֹלָם} & \textit{For the entire land that you see -- I will give it to you and to your offspring forever} (Gen 13:15) \\
		\foreignlanguage{hebrew}{וּמֵעֵץ הַדַּעַת טוֹב וָרָע לֹא תֹאכַל מִמֶּנּוּ} & \textit{But from the tree of the knowledge of good and evil -- you shall not eat from it} (Gen 2:17) \\
	\end{tabular}
	
	\vspace{0.5cm}
	
	The dislocated element may even be followed by a \textit{wayyiqṭol} form.
	
	\vspace{0.5cm}
	
	\begin{tabular}{>{\raggedleft}p{0.35\linewidth} p{0.55\linewidth}}
		\foreignlanguage{hebrew}{וּבְנֵי יִשְׂרָאֵל הַיֹּשְׁבִים בְּעָרֵי יְהוּדָה וַיִּמְלֹךְ עֲלֵיהֶם רְחַבְעָם} & \textit{And the Israelites that lived in the cities of Juda -- Rehabeam became king over them} (1\,Kgs 12:17) \\
	\end{tabular}
	
	\vspace{0.5cm}
	
	Left dislocation may occur in all sorts of clauses. In Ps 11:4 it is a nominal clause.
	
	\vspace{0.5cm}
	
	\begin{tabular}{>{\raggedleft}p{0.35\linewidth} p{0.55\linewidth}}
		\foreignlanguage{hebrew}{ יְהוָה בַּשָּׁמַיִם כִּסְאוֹ} & \textit{The Lord -- his throne is in heaven} (Ps 11:4) \\
	\end{tabular}
	
	\vspace{0.5cm}
	
	Left dislocation is used to mark the dislocated element as salient for the clause, usually as the topic about which the predication is made. 
	
	The traditional term for left dislocation is the Latin expression \textit{casus pendens} because the dislocated element seems to be suspended in front of the clause. The English term \textit{left dislocation} is based on the left-to-right writing direction of English or modern transliterations of languages; it does not agree with the writing direction of Hebrew.
	
	
	\section{Exercises}
	
	\subsection{Translation of Verbal Forms}
	
	Translate the following verbal forms. Identify the gender (masc., fem., comm.) and number (sg., pl.) of forms of which the English translation is ambiguous (i.e., \textit{you}, \textit{they}). Mark the stressed syllable if stress is not on the last syllable.
	
	\hspace{0.5cm}
	
	\selectlanguage{hebrew}
	
	\noindent
	1~~\foreignlanguage{hebrew}{דִּבַּרְתָּ}  \hspace{0.3cm}
	2~~\foreignlanguage{hebrew}{תְּדַבֶּר}  \hspace{0.3cm}
	3~~\foreignlanguage{hebrew}{דִּבְּרוּ}  \hspace{0.3cm}
	4~~\foreignlanguage{hebrew}{יְדַבֵּ֑רוּ}  \hspace{0.3cm}
	5~~\foreignlanguage{hebrew}{דִּבַּרְתִּי}  \hspace{0.3cm}
	6~~\foreignlanguage{hebrew}{תְּדַבְּרוּן}  \hspace{0.3cm}
	7~~\foreignlanguage{hebrew}{מִהַרְתֶּן}  \hspace{0.3cm}
	8~~\foreignlanguage{hebrew}{אֲדַבֵּר}  \hspace{0.3cm}
	9~~\foreignlanguage{hebrew}{לְדַבֵּר}  \hspace{0.3cm}
	10~~\foreignlanguage{hebrew}{דַּבְּרוּ}  \hspace{0.3cm}
	11~~\foreignlanguage{hebrew}{מְדַבֵּר}  \hspace{0.3cm}
	12~~\foreignlanguage{hebrew}{דִּבַּרְנוּ}  \hspace{0.3cm}
	13~~\foreignlanguage{hebrew}{יְדַבֵּר}  \hspace{0.3cm}
	14~~\foreignlanguage{hebrew}{לִמַּדְתְּ}  \hspace{0.3cm}
	15~~\foreignlanguage{hebrew}{‎יְשָׁרְתוּ}  \hspace{0.3cm}
	16~~\foreignlanguage{hebrew}{תְּדַבֵּרְנָה}  \hspace{0.3cm}
	17~~\foreignlanguage{hebrew}{‎תְּבָרֲכוּ}  \hspace{0.3cm}
	18~~\foreignlanguage{hebrew}{‎וַיְדַבֵּ֑רוּ}  \hspace{0.3cm}
	19~~\foreignlanguage{hebrew}{וַיְבָרֶךְ}  \hspace{0.3cm}
	19~~\foreignlanguage{hebrew}{וַיְבָרֶךְ}  \hspace{0.3cm}
	19~~\foreignlanguage{hebrew}{וַיְבַקְשׁוּ}  \hspace{0.3cm}
	
	\selectlanguage{english}
	
	
	\subsection{Translation of Sentences}
	
	Translate the following sentences from the Hebrew Bible. Names of persons and geographical names in these sentences: \foreignlanguage{hebrew}{אַבְרָהָם}, \foreignlanguage{hebrew}{אַהֲרֹן}, \foreignlanguage{hebrew}{בֵּית־אֵל}, \foreignlanguage{hebrew}{בִּנְיָמִין}, \foreignlanguage{hebrew}{זִיף}, \foreignlanguage{hebrew}{יוֺנָתָן}, \foreignlanguage{hebrew}{מִדְיָן}, \foreignlanguage{hebrew}{מִכְמָשׂ}, \foreignlanguage{hebrew}{מֹשֶׁה}, \foreignlanguage{hebrew}{מִצְרַיִם}, \foreignlanguage{hebrew}{שָׂרָה}, \foreignlanguage{hebrew}{שָׂרַי}, \foreignlanguage{hebrew}{שָׁאוּל}, \foreignlanguage{hebrew}{שִׁמְשׁוֹן}
	
	\vspace{0.5cm}
	
	\selectlanguage{hebrew}
	\noindent 
	1~~\foreignlanguage{hebrew}{וַיִּשְׁמַע פַּרְעֹה אֶת־הַדָּבָר הַזֶּ֔ה וַיְבַקֵּשׁ לַהֲרֹג אֶת־מֹשֶׁ֑ה וַיִּבְרַח מֹשֶׁה מִפְּנֵי פַרְעֹ֔ה וַיֵּשֶׁב בְּאֶרֶץ־מִדְיָן וַיֵּשֶׁב עַל־הַבְּאֵֽר׃}  \hspace{0.3cm}
	2~~\foreignlanguage{hebrew}{וַיֹּאמֶר יְהוָה אֶל־מֹשֶׁה בְּמִדְיָ֔ן לֵךְ שֻׁב מִצְרָ֑יִם כִּי־מֵ֙ת֙וּ כָּל־הָאֲנָשִׁ֔ים הַֽמְבַקְשִׁים אֶת־נַפְשֶֽׁךָ׃}  \hspace{0.3cm}
	3~~\foreignlanguage{hebrew}{וַיֹּאמְרוּ אִישׁ אֶל־רֵעֵ֔הוּ}\LTRfootnote{\space \foreignlanguage{hebrew}{אִישׁ אֶל־רֵעֵ֔הוּ} \textit{one to another} (typical reciprocal use of \foreignlanguage{hebrew}{אִישׁ} and \foreignlanguage{hebrew}{רֵעַ} with plural verb forms).} \foreignlanguage{hebrew}{מִי עָשָׂה הַדָּבָר הַזֶּ֑ה וַיִּדְרְשׁוּ וַיְבַקְשׁוּ וַיֹּאמְר֔וּ גִּדְעוֹן בֶּן־יוֹאָ֔שׁ עָשָׂה הַדָּבָר הַזֶּֽה׃}  \hspace{0.3cm}
	4~~\foreignlanguage{hebrew}{וַיָּקָם שָׁאוּל וַיֵּרֶד אֶל־מִדְבַּר־זִ֔יף וְאִתּוֹ שְׁלֹשֶׁת־אֲלָפִים אִישׁ בְּחוּרֵי יִשְׂרָאֵ֑ל לְבַקֵּשׁ אֶת־דָּוִד בְּמִדְבַּר־זִיף׃}  \hspace{0.3cm}
	5~~\foreignlanguage{hebrew}{וַיֹּאמֶר אֱלֹהִים אֶל־אַבְרָהָ֔ם שָׂרַי אִשְׁתְּךָ֔ לֹא־תִקְרָא אֶת־שְׁמָהּ שָׂרָ֑י כִּי}\LTRfootnote{\space \foreignlanguage{hebrew}{כִּי} \textit{but} (following a negative)} \foreignlanguage{hebrew}{שָׂרָה שְׁמָֽהּ׃ וּבֵרַכְתִּי אֹתָ֔הּ וְגַם נָתַתִּי מִמֶּנָּה לְךָ בֵּ֑ן וּבֵֽרַכְתִּ֙יהָ֙ וְהָֽיְתָה לְגוֹיִ֔ם מַלְכֵי עַמִּים מִמֶּנָּה יִהְיֽוּ׃} \hspace{0.3cm}
	6~~\foreignlanguage{hebrew}{וַתֵּלֶד הָאִשָּׁה בֵּ֔ן וַתִּקְרָא אֶת־שְׁמוֹ שִׁמְשׁ֑וֹן וַיִּגְדַּל הַנַּ֔עַר וַיְבָרְכֵהוּ יְהוָֽה׃}  \hspace{0.3cm}
	7~~\foreignlanguage{hebrew}{וַיֹּאמֶר יְהוָה אֶל־מֹשֶׁה לֵךְ אֶל־הָעָ֔ם וְקִדַּשְׁתָּם הַיּוֹם וּמָחָ֑ר וְכִבְּסוּ שִׂמְלֹתָם׃}\LTRfootnote{\space \foreignlanguage{hebrew}{שִׂמְלָה} \textit{outer garment, cloak, mantle}} \hspace{0.3cm}
	8~~\foreignlanguage{hebrew}{וַיֵּרֶד מֹשֶׁה מִן־הָהָר אֶל־הָעָ֑ם וַיְקַדֵּשׁ אֶת־הָעָ֔ם וַיְכַבְּסוּ שִׂמְלֹתָם׃}\LTRfootnote{\space \foreignlanguage{hebrew}{שִׂמְלָה} \textit{outer garment, cloak, mantle}} \hspace{0.3cm}
	9~~\foreignlanguage{hebrew}{וְאַחַר בָּאוּ מֹשֶׁה וְאַהֲרֹ֔ן וַיֹּאמְרוּ אֶל־פַּרְעֹ֑ה כֹּה־אָמַר יְהוָה אֱלֹהֵי יִשְׂרָאֵ֔ל שַׁלַּח אֶת־עַמִּ֔י וְיָחֹגּוּ}\LTRfootnote{\space \foreignlanguage{hebrew}{חגג} \textit{to celebrate a pilgrim's feast}} \foreignlanguage{hebrew}{לִי בַּמִּדְבָּֽר׃ וַיֹּאמֶר פַּרְעֹ֔ה מִי יְהוָה אֲשֶׁר}\LTRfootnote{\space \foreignlanguage{hebrew}{אֲשֶׁר} here \textit{that} (introducing a consecutive clause)} \foreignlanguage{hebrew}{אֶשְׁמַע בְּקֹל֔וֹ לְשַׁלַּח אֶת־יִשְׂרָאֵ֑ל לֹא יָדַ֙עְתִּ֙י אֶת־יְהוָ֔ה וְגַם אֶת־יִשְׂרָאֵל לֹא אֲשַׁלֵּֽחַ׃} \hspace{0.3cm}
	10~~\foreignlanguage{hebrew}{וַיִּבְחַר־לוֹ שָׁאוּל שְׁלֹשֶׁת אֲלָפִים מִיִּשְׂרָאֵל֒ וַיִּהְיוּ עִם־שָׁאוּל אַלְפַּיִם בְּמִכְמָשׂ וּבְהַר בֵּית־אֵ֔ל וְאֶלֶף הָיוּ עִם־יוֹנָתָ֔ן בְּגִבְעַת בִּנְיָמִ֑ין וְיֶתֶר הָעָ֔ם שִׁלַּח אִישׁ לְאֹהָלָיו׃} \hspace{0.3cm}
	
	\selectlanguage{english}
	
	
	
	\section{Hebrew Reading: Genesis 12:1--3}
	Translate Gen 12:1--3 with the help of notes below the text.
	
	\vspace{0.5cm}
	
	\selectlanguage{hebrew}
	\noindent
	\textsuperscript{1}~\foreignlanguage{hebrew}{וַיֹּאמֶר יְהוָה אֶל־אַבְרָ֔ם לֶךְ־לְךָ מֵאַרְצְךָ וּמִמּוֹלַדְתְּךָ וּמִבֵּית אָבִ֑יךָ אֶל־הָאָרֶץ אֲשֶׁר אַרְאֶֽךָּ׃} \hspace{0.3cm}
	\textsuperscript{2}~\foreignlanguage{hebrew}{ וְאֶֽעֶשְׂךָ לְגוֹי גָּד֔וֹל וַאֲבָרֶכְךָ֔ וַאֲגַדְּלָה שְׁמֶ֑ךָ וֶהְיֵה בְּרָכָֽה׃} \hspace{0.3cm}
	\textsuperscript{3}~\foreignlanguage{hebrew}{וַאֲבָֽרֲכָה מְבָרְכֶ֔יךָ וּמְקַלֶּלְךָ אָאֹ֑ר וְנִבְרְכוּ בְךָ֔ כֹּל מִשְׁפְּחֹת הָאֲדָמָֽה׃} \hspace{0.3cm}
	\selectlanguage{english}
	
	\vspace{0.5cm}
	
	\hspace*{-0.5cm}\begin{longtable}{p{0.075\linewidth} p{0.1\linewidth}p{0.725\linewidth}}
		12:1 & \foreignlanguage{hebrew}{מוֺלֶדֶת}  & \textit{relations, the relatives} \\
		& \foreignlanguage{hebrew}{אַרְאֶֽךָּ} & \textit{I will show you} (1 c.\ sg.\ PC Hi. of \foreignlanguage{hebrew}{ראה} with enclitic pronoun 2 m.\ sg.) \\
		12:3 & \foreignlanguage{hebrew}{וְנִבְרְכוּ} & It is preferable to interpret \foreignlanguage{hebrew}{וְנִבְרְכוּ} as a reflexive verbal form, e.g., \textit{to ask blessings for oneself by mentioning someone blessed as an example of divine blessing} \\
	\end{longtable}
	
	\section{Hebrew Reading: Genesis 13:8--13}
	Translate Gen 13:8--13 with the help of notes below the text. Names of persons and places are easily identifiable.
	
	\vspace{0.5cm}
	
	\selectlanguage{hebrew}
	\noindent
	\textsuperscript{8}~\foreignlanguage{hebrew}{וַיֹּאמֶר אַבְרָם אֶל־לוֹט אַל־נָא תְהִי מְרִיבָה בֵּינִי וּבֵינֶ֔יךָ וּבֵין רֹעַי וּבֵין רֹעֶ֑יךָ כִּי־אֲנָשִׁים אַחִים אֲנָֽחְנוּ׃} \hspace{0.3cm}
	\textsuperscript{9}~\foreignlanguage{hebrew}{הֲלֹא כָל־הָאָ֙רֶ֙ץ לְפָנֶ֔יךָ הִפָּרֶד נָא מֵעָלָ֑י אִם־הַשְּׂמֹאל וְאֵימִ֔נָה וְאִם־הַיָּמִין וְאַשְׂמְאִֽילָה׃} \hspace{0.3cm}
	\textsuperscript{10}~\foreignlanguage{hebrew}{וַיִּשָּׂא־לוֹט אֶת־עֵינָיו וַיַּרְא אֶת־כָּל־כִּכַּר הַיַּרְדֵּ֔ן כִּי כֻלָּהּ מַשְׁקֶ֑ה לִפְנֵי ׀ שַׁחֵת יְהוָה אֶת־סְדֹם וְאֶת־עֲמֹרָ֔ה כְּגַן־יְהוָה כְּאֶרֶץ מִצְרַ֔יִם בֹּאֲכָה צֹֽעַר׃} \hspace{0.3cm}
	\textsuperscript{11}~\foreignlanguage{hebrew}{וַיִּבְחַר־לוֹ לוֹט אֵת כָּל־כִּכַּר הַיַּרְדֵּ֔ן וַיִּסַּע לוֹט מִקֶּ֑דֶם וַיִּפָּרְדוּ אִישׁ מֵעַל אָחִֽיו׃} \hspace{0.3cm}
	\textsuperscript{12}~\foreignlanguage{hebrew}{אַבְרָם יָשַׁב בְּאֶרֶץ־כְּנָ֑עַן וְלוֹט יָשַׁב בְּעָרֵי הַכִּכָּ֔ר וַיֶּאֱהַל עַד־סְדֹֽם׃} \hspace{0.3cm}
	\textsuperscript{13}~\foreignlanguage{hebrew}{וְאַנְשֵׁי סְדֹ֔ם רָעִים וְחַטָּאִ֑ים לַיהוָה מְאֹֽד׃} \hspace{0.3cm}
	\selectlanguage{english}
	
	
	\hspace*{-0.5cm}\begin{longtable}{p{0.075\linewidth} p{0.1\linewidth}p{0.725\linewidth}}
		13:8 & \foreignlanguage{hebrew}{מְרִיבָה}  & \textit{quarrel} \\
		13:9 & \foreignlanguage{hebrew}{פרד} Ni. & \textit{to divide, separate} (intransitive) \\
		& \foreignlanguage{hebrew}{אֵימִ֫נָה} & \textit{let me go to the right} (1 c.\ sg.\ cohortative Hi. \foreignlanguage{hebrew}{ימן}) \\
		& \foreignlanguage{hebrew}{אַשְׂמְאִ֫ילָה} & \textit{let me go to the left} (1 c.\ sg.\ cohortative Hi. \foreignlanguage{hebrew}{שׂמאל}) \\
		13:10 & \foreignlanguage{hebrew}{הַיַּרְדֵּן} & \textit{Jordan} \\
		& \foreignlanguage{hebrew}{מַשְׁקֶה} & \textit{well watered} \\
		& \foreignlanguage{hebrew}{גַּן} & \textit{garden} (geminate noun) \\
		& \foreignlanguage{hebrew}{בֹּאֲכָה} & \textit{as far as} (petrified inf.\ cs.\ of \foreignlanguage{hebrew}{בוא} with enclitic pronoun) \\
		& \foreignlanguage{hebrew}{צֹעַר} & \textit{Zoar} \\
		13:11 & \foreignlanguage{hebrew}{מִקֶּדֶם} & here \textit{to the east} \\
		13:12 & \foreignlanguage{hebrew}{אהל} Qal & \textit{to obtain grazing rights} (\textit{HALOT}); traditionally \textit{to tent} (BDB) \\
		13:13 & \foreignlanguage{hebrew}{חַטָּא} Qal & \textit{sinful, sinner} \\
	\end{longtable}
	
	
	\chapter{Chapter 20}
	
	\renewcommand\arraystretch{1.4}
	
	\section{Vocabulary}
	
	\subsection{Verbs}
	
	% For the centering of the separation between the two columns see the documentation of the array package, page 2 
	
	\begin{longtable}{>{\raggedleft}p{0.175\linewidth} p{0.75\linewidth}}
		\foreignlanguage{hebrew}{הלך} & Q.\ \textit{to go}; Hitpa.\ \textit{to walk; to walk about; to move to and fro} \\ % BDB
		\foreignlanguage{hebrew}{הלל} & Pi.\ \textit{to praise}; Hitpa.\ \textit{to boast} \\
		\foreignlanguage{hebrew}{הפך} & Q.\ \textit{to turn; to overthrow; to change} \\ % HALOT
		\foreignlanguage{hebrew}{חזק} & Q.\ \textit{to be firm, strong} (stative verb, PC \foreignlanguage{hebrew}{יֶחֱזַק}, but SC \foreignlanguage{hebrew}{חָזַק}); Pi.\ \textit{to make firm, strong; repair}; Hi.\ \textit{to seize, grasp; to keep hold of}; Hitpa.\ \textit{to show oneself courageous; to prove oneself strong} \\
		\foreignlanguage{hebrew}{טהר} & Q.\ \textit{to be clean} (stative verb, SC \foreignlanguage{hebrew}{טָהֵר}, PC \foreignlanguage{hebrew}{יִטְהַר}); Pi.\ \textit{to cleanse; to pronounce clean} \\
		\foreignlanguage{hebrew}{כבד} & Q.\ \textit{to be heavy; to be insensitive, dull; to be honored} (stative verb, SC \foreignlanguage{hebrew}{כָּבֵד}, PC \foreignlanguage{hebrew}{יִכְבַּד}); Pi.\ \textit{to make honorable, honor}; Ni.\ \textit{to be honored, enjoy honor} \\
		\foreignlanguage{hebrew}{פלל} & Hitpa.\ \textit{to pray} \\
		\foreignlanguage{hebrew}{שׂבע} & Q.\ \textit{be sated, to eat one's fill, satisfy oneself with} (stative verb, SC pausal form \foreignlanguage{hebrew}{שָׂבֵ֑עוּ}, PC \foreignlanguage{hebrew}{יִשְׂבַּע}) (with noun phrase) \\ % BDB, HALOT
	\end{longtable}
	
	
	\subsection{Nouns}
	
	\begin{longtable}{>{\raggedleft}p{0.175\linewidth} p{0.75\linewidth}}
		\foreignlanguage{hebrew}{הָמוֺן} & \textit{sound, murmur, roar; crowd; abundance} \\
		\foreignlanguage{hebrew}{זָכָר} & \textit{male} (adj.) \\
		\foreignlanguage{hebrew}{זְרוֺעַ} & \textit{arm, forearm} (fem.) \\
		\foreignlanguage{hebrew}{חֵלֶב} & \textit{fat; best, choice part} \\ % HALOT
		\foreignlanguage{hebrew}{חָלָל} & \textit{slain, pierced} \\ % HALOT
		\foreignlanguage{hebrew}{כְּרוּב} & \textit{cherub} \\
		\foreignlanguage{hebrew}{מַלְכוּת} & \textit{royalty; royal power; reign; kingdom} \\
		\foreignlanguage{hebrew}{עֵצָה} & \textit{counsel, advice; plan} \\ % BDB, HALOT
		\foreignlanguage{hebrew}{פֵּאָה} & \textit{corner; side} \\
		\foreignlanguage{hebrew}{פֶּשַׁע} & \textit{transgression, crime} \\ % BDB, HALOT
		\foreignlanguage{hebrew}{צוּר} & \textit{rock; rocky hill, mountain} \\ % HALOT
		\foreignlanguage{hebrew}{רָעָה} & \textit{evil; distress; misery; disaster, calamity} (f.\ sg.\ of the adj.\ \foreignlanguage{hebrew}{רַע} \textit{evil, bad}) \\
		\foreignlanguage{hebrew}{שִׁיר} & \textit{song} \\
		\foreignlanguage{hebrew}{שֶׁקֶל} & \textit{weight; shekel} (measurement of weight; about 11\,g) \\ % Hebrew Wikipedia; WiBiLex ca. 11.3 g (average; range of weights from 11-13 g)
		\foreignlanguage{hebrew}{תְּהִלָּה} & \textit{glory, praise; song of praise} \\
		\foreignlanguage{hebrew}{תְּפִלָּה} & \textit{prayer} \\
	\end{longtable}
	
	\subsection{Other Parts of Speech}
	
	\begin{longtable}{>{\raggedleft}p{0.175\linewidth} p{0.75\linewidth}}
		\foreignlanguage{hebrew}{בַּעַד} & \textit{behind; through, out of; round about; for the benefit of, for} (prep.; cs.\ st.\ \foreignlanguage{hebrew}{בְּעַד}, with ePP \foreignlanguage{hebrew}{בַּעֲדוֺ}) \\ % HALOT
		\foreignlanguage{hebrew}{הֵן} & \textit{behold} (points to the word it precedes; cf. \foreignlanguage{hebrew}{הִנֵּה}); \textit{if} (conditional conj.); \textit{if} (for indirect questions) \\ % HALOT
		\foreignlanguage{hebrew}{פֹּה} & \textit{here, at this place; to here} (adv.) \\ % HALOT
		\foreignlanguage{hebrew}{פֶן} & \textit{so that not, lest} (conj.) \\
	\end{longtable}
	
	
	\section{The Pual and Hitpael Binyanim}
	
	\subsection{The Forms of the Strong Verb in the Pual and Hitpael}
	
	The Pual and the Hitpael are the passive and the reflexive counterparts of the Piel. As the Piel, the Pual and the Hitpael are characterized by the gemination (doubling) of the second root consonant.
	
	As a passive form, the Pual is marked by the vowel sequence \textit{u--a}, e.g., \foreignlanguage{hebrew}{נֻתּ͏ַץ} \textit{it was torn down} (Judg 6:26), \foreignlanguage{hebrew}{וְכֻבּ͏ַס} \textit{and it shall be washed} (Lev 15:17) (except in forms with vocal \textit{šwa} with the second root consonant due to suffixation).
	
	The Hitpael as a reflexive form is marked by a prefixed /t/ before the first root consonant.
	
	\newpage
	
	\begin{center}
		\begin{longtable}{|lll|r|r|}
			\hline
			& & & \multicolumn{1}{c|}{Pual} & \multicolumn{1}{c|}{Hitpael} \\
			\hline
			\endhead
			\hline
			\endfoot
			SC & sg. & 3 m. & \foreignlanguage{hebrew}{כּ͏ֻתּ͏ַב} & \foreignlanguage{hebrew}{הִתְכַּתֵּב} \\
			& & 3 f. & \foreignlanguage{hebrew}{כּ͏ֻתְּ͏בָה} & \foreignlanguage{hebrew}{הִתְכַּתְּבָה} \\
			& & 2 m. & \foreignlanguage{hebrew}{כּ͏ֻתּ͏ַבְתּ͏ָ} & \foreignlanguage{hebrew}{הִתְכַּתַּבְתָּ} \\
			& & 2 f. & \foreignlanguage{hebrew}{כּ͏ֻתּ͏ַבְתְּ͏} & \foreignlanguage{hebrew}{הִתְכַּתַּבְתְּ} \\
			& & 1 c. & \foreignlanguage{hebrew}{כּ͏ֻתּ͏ַ֫בְתִּ͏י} & \foreignlanguage{hebrew}{הִתְכַּתַּ֫בְתִּי} \\
			\hline
			& pl. & 3 c. & \foreignlanguage{hebrew}{כּ͏ֻתְּ͏בוּ͏} & \foreignlanguage{hebrew}{הִתְכַּתְּבוּ} \\
			& & 2 m. & \foreignlanguage{hebrew}{כּ͏ֻתּ͏ַבְתּ͏ֶם} & \foreignlanguage{hebrew}{הִתְכַּתַּבְתֶּם} \\
			& & 2 f. & \foreignlanguage{hebrew}{כּ͏ֻתּ͏ַבְתּ͏ֶן} & \foreignlanguage{hebrew}{הִתְכַּתַּבְתֶּן} \\
			& & 1 c. & \foreignlanguage{hebrew}{כּ͏ֻתּ͏ַ֫בְנוּ͏} & \foreignlanguage{hebrew}{הִתְכַּתַּ֫בְנוּ} \\
			\hline
			PC & sg. & 3 m. & \foreignlanguage{hebrew}{יְכֻתּ͏ַב} & \foreignlanguage{hebrew}{יִתְכַּתֵּב} \\
			& & 3 f. & \foreignlanguage{hebrew}{תְּ͏כֻתּ͏ַב} & \foreignlanguage{hebrew}{תִּתְכַּתֵּב} \\
			& & 2 m. & \foreignlanguage{hebrew}{תְּ͏כֻתּ͏ַב} & \foreignlanguage{hebrew}{תִּתְכַּתֵּב} \\
			& & 2 f. & \foreignlanguage{hebrew}{תְּ͏כֻתְּ͏בִי} & \foreignlanguage{hebrew}{תִּתְכַּתְּבִי} \\
			& & 1 c. & \foreignlanguage{hebrew}{אֲכֻתּ͏ַב} & \foreignlanguage{hebrew}{אֶתְכַּתֵּב} \\
			& pl. & 3 c. & \foreignlanguage{hebrew}{יְכֻתְּ͏בוּ͏} & \foreignlanguage{hebrew}{יִתְכַּתְּבוּ} \\
			& & 3 f. & \foreignlanguage{hebrew}{תְּ͏כֻתּ͏ַ֫בְנָה} & \foreignlanguage{hebrew}{תִּתְכַּתֵּ֫בְנָה} \\
			& & 2 m. & \foreignlanguage{hebrew}{תְּ͏כֻתְּ͏בוּ͏} & \foreignlanguage{hebrew}{תִּתְכַּתְּבוּ} \\
			& & 2 f. & \foreignlanguage{hebrew}{תְּ͏כֻתּ͏ַ֫בְנָה} & \foreignlanguage{hebrew}{תִּתְכַּתֵּ֫בְנָה} \\
			& & 1 c. & \foreignlanguage{hebrew}{נְכֻתּ͏ַב} & \foreignlanguage{hebrew}{נִתְכַּתֵּב} \\
			\hline
			Jussive & sg. & 3 m. & \foreignlanguage{hebrew}{יְכֻתּ͏ַב} & \foreignlanguage{hebrew}{יִתְכַּתֵּב} \\
			\textit{wayyiqṭol} & sg. & 3 m. & \foreignlanguage{hebrew}{וַיְכֻתַּב} & \foreignlanguage{hebrew}{וַיִּתְכַּתֵּב} \\
			\hline
			Impv. & sg. & 2 m. & & \foreignlanguage{hebrew}{הִתְכַּתֵּב} \\
			& & & & \foreignlanguage{hebrew}{הִתְכַּתְּבָה} \\
			& & 2 f. & & \foreignlanguage{hebrew}{הִתְכַּתְּבִי} \\
			& pl. & 2 m. &  & \foreignlanguage{hebrew}{הִתְכַּתְּבוּ} \\
			% & & 2 f. & & \foreignlanguage{hebrew}{הִתְכַּתֵּבְנָה} \\
			\hline
			Inf.\ cs. & & &  & \foreignlanguage{hebrew}{הִתְכַּתֵּב} \\
			Inf.\ abs. & & &  & \foreignlanguage{hebrew}{הִתְכַּתֵּב} \\
			\pagebreak
			\hline
			Part. & m. & sg. & \foreignlanguage{hebrew}{מְכֻתּ͏ָב} & \foreignlanguage{hebrew}{מִתְכַּתֵּב} \\
			& & pl. & \foreignlanguage{hebrew}{מְכֻתּ͏ָבִים} & \foreignlanguage{hebrew}{מִתְכַּתְּבִים} \\
			& f. & sg. & \foreignlanguage{hebrew}{מְכֻתּ͏ֶבֶת} & \foreignlanguage{hebrew}{מִתְכַּתֶּבֶת} \\
			& & & \foreignlanguage{hebrew}{מְכֻתּ͏ָבָה} & \foreignlanguage{hebrew}{מִתְכַתֵּבָה} \\
			& & pl. & \foreignlanguage{hebrew}{מְכַתָּבוֺת} & \foreignlanguage{hebrew}{מִתְכַּתְּבוֺת} \\
		\end{longtable}
	\end{center}
	
	\vspace{0.5cm}
	
	\noindent \textbf{Notes Pertaining to Pual Forms}
	
	\noindent In \textit{wayyiqṭol} forms 3 m.\ sg./pl.\ the prefix consonant /y/ looses its gemination because of the \textit{dageš forte}, e.g., \foreignlanguage{hebrew}{וַיְבֻקּ͏ַשׁ͏} \textit{and [the matter] was investigated} (Esth 2:32).
	
	Pual imperative forms are not attested in Biblical Hebrew. The only attestation of an infinitive is \foreignlanguage{hebrew}{עֻנּ͏וֹתוֹ} (inf.\ cs. with enclitic pronoun) \textit{his being afflicted} in Ps 132:1  (from the verb \foreignlanguage{hebrew}{ענה} Pu.\ \textit{to be afflicted}; Pi.\ \textit{to humble, mishandle, afflict}).
	
	As in the Piel, gemination of the second root consonant may be lost in certain consonants with vocal \textit{šwa}. In Biblical Hebrew only three forms can be found: \foreignlanguage{hebrew}{תְבֻקְשׁ͏ִי} (PC 2  f.\ sg.\ Ezek 26:21), \foreignlanguage{hebrew}{מְחֻלְלֵי} (part.\ m.\ pl.\ cs.\ st.\ Ezek 32:26), \foreignlanguage{hebrew}{מְקֻצְעֹת} (part. f.\ pl.\ cs.\ st.\ Exod 26:23; 36:28).
	
	In pausal forms, the \textit{pataḥ} is lengthened to \textit{qameṣ}, e.g., \foreignlanguage{hebrew}{שֻׁלָּח} \textit{he was sent} (Obad 1:1). Forms with vocal \textit{šwa} with the second root consonant in the context forms have the original vowel /a/ as a \textit{qameṣ}, e.g., \foreignlanguage{hebrew}{חֻבָּ֑שׁוּ} \textit{it was bound up} (Isa 1:6).
	
	Geminate verbs have strong Pual forms with consonants in all three slots, e.g., \foreignlanguage{hebrew}{יְקֻלָּֽל} \textit{he shall be considered cursed} (Isa 65:20; pausal form).
	
	\vspace{0.5cm}
	
	\noindent \textbf{Notes pertaining to Hitpael Forms}
	
	\noindent As in the Piel, gemination of R\textsubscript{2} may be lost in forms with certain consonants as R\textsubscript{2} with vocal \textit{šwa}, e.g., \foreignlanguage{hebrew}{יִתְהַלְלוּ} \textit{they shall boast}, \foreignlanguage{hebrew}{הַלְלוּ־יָהּ} \textit{Praise Yah!}.\footnote{In the Piel, this occurs with sibilants (i.e., \foreignlanguage{hebrew}{ס}, \foreignlanguage{hebrew}{צ}, \foreignlanguage{hebrew}{שׂ} and \foreignlanguage{hebrew}{שׁ}) or the consonants \foreignlanguage{hebrew}{מ}, \foreignlanguage{hebrew}{נ}, \foreignlanguage{hebrew}{ק}, \foreignlanguage{hebrew}{י}, and \foreignlanguage{hebrew}{ל} as R\textsubscript{2}.} Preservation of the gemination, however, appears to be frequent.
	
	Seven verbs have the vowel /a/ as thematic vowel between R\textsubscript{2} and R\textsubscript{3}, e.g., \foreignlanguage{hebrew}{הִתְאַנַּף} \textit{he became angry} (so-called Hitpaal; cf. JM §\,53\,\textit{b}).
	
	Pausal forms have /a/ in intermediate pause and /ā/ in major pause instead of the /ē/ of the context forms, e.g., \foreignlanguage{hebrew}{יִתְיַצַּ֔ב} and \foreignlanguage{hebrew}{יִתְיַצָּ֑ב}.
	
	\vspace{0.5cm}
	
	\subsection{Hitpael Forms with Sibilants and Dentals as R\textsubscript{1}}
	
	When R\textsubscript{1} is a sibilant, i.e., \foreignlanguage{hebrew}{ס}, \foreignlanguage{hebrew}{צ}, \foreignlanguage{hebrew}{שׂ} or \foreignlanguage{hebrew}{שׁ}, the prefix /t/ and the sibilant change position (\textit{metathesis}), e.g., \textit{*mitsattēr > mistattēr} \foreignlanguage{hebrew}{מִסְתַּתֵּר} \textit{[he is] hiding himself} (1\,Sam 23:19), \textit{wāʾætšammərā > wāʾæštammərā} \foreignlanguage{hebrew}{וָאֶשְׁתַּמְּרָה} \textit{and I kept myself} (2\,Sam 22:24).
	
	In the case of the emphatic sibilant \foreignlanguage{hebrew}{צ} as R\textsubscript{1}, the /t/ of the prefix and R\textsubscript{1} do not only change position but the /t/ is assimilated to the emphatic sibilant \foreignlanguage{hebrew}{צ} becoming emphatic itself (/t/ > /ṭ/), e.g., \textit{*hitṣayyadnū  > *hiṣṭayyadnū} \foreignlanguage{hebrew}{הִצְטַיַּדְנוּ} \textit{we took [something] as provision with us} (Josh 9:12), \foreignlanguage{hebrew}{נִצְטַדָּ֑ק} \textit{how shall we justify ourselves?} (Gen 44:16; pausal form).
	
	In the case of the root \foreignlanguage{hebrew}{זכה} total assimilation of the prefix /t/ is attested: \textit{*hitzakkû > *hiztakkû > *hizdakkû > hizzakkû} \foreignlanguage{hebrew}{הִזַּּכּוּ} \textit{make yourselves clean} (Isa 1:16; the only example of this in the Hebrew Bible).
	
	With dentals, the prefix /t/ of the Hitpael is usually assimilated to the dental R\textsubscript{1} of the root, e.g., \textit{*mitdabbēr > middabbēr} \foreignlanguage{hebrew}{מִדַּבֵּר} \textit{someone speaking} (Num 7:89), \textit{*hitṭaharnū > hiṭṭaharnū} \foreignlanguage{hebrew}{הִטַּהַרְנוּ} \textit{we have cleansed ourselves} (Josh 22:17).
	
	With /t/ as R\textsubscript{1}, only one \foreignlanguage{hebrew}{ת} is written, e.g., \foreignlanguage{hebrew}{תִּתַּמָּ֑ם} \textit{tittammām} \textit{you deal with integrity with [the blameless]} (2\,Sam 22:26 par.\ Ps 18:26).
	
	Assimilation of the /t/ to a following /n/ or even /k/ is attested, as well.
	
	
	\subsection{Verbs with Gutturals or /r/ in the Pual and Hitpael Binyanim}
	
	\subsubsection{Verbs with Gutturals or /r/ as R\textsubscript{2}}
	
	In the same way as in the Piel, the short vowel before R\textsubscript{2} may either be preserved despite being in an open unstressed syllable or it is lengthened /a/ > /ā/. The same tendencies for the preservation of the short vowel or the lengthening of the vowel can be observed in the Pual and the Hitpael (cf. Chapter 19). The following examples illustrate this:
	
	\vspace{0.25cm}
	
	\noindent With preservation of /a/ as a short vowel:
	
	\begin{itemize}[noitemsep]
		\item[--] suffix conj.: \foreignlanguage{hebrew}{רֻחַ֫צְתְּ}
		\item[--] prefix conj.: \foreignlanguage{hebrew}{יְרֻחַם}
		\item[--] inf.\ cs.: \foreignlanguage{hebrew}{לְהִתְנַחֵם}, \foreignlanguage{hebrew}{הִתְנַחֵל}
		\item[--] imperative: \foreignlanguage{hebrew}{הִטַּהֲרוּ} (m.\ pl., verb \foreignlanguage{hebrew}{טהר})
		\item[--] participle: \foreignlanguage{hebrew}{מִתְנַחֵם} (m.\ sg.)
	\end{itemize}
	
	\noindent \textbf{Note}
	\nopagebreak
	
	\noindent In pausal forms, the vowel /a/ between R\textsubscript{1} and R\textsubscript{2} may change to /æ/ because of the \textit{qameṣ} with the guttural as R\textsubscript{2}, e.g., \foreignlanguage{hebrew}{יִתְנֶחָ֑ם}, \foreignlanguage{hebrew}{וְהִטֶּהָ֑רוּ} (compare the similar vowel change with the definite article and the interrogative He).
	
	\vspace{0.5cm}
	
	\noindent With lengthening of the short vowel to /ā/:
	
	\begin{itemize}[noitemsep]
		\item[--] suffix conj.: \foreignlanguage{hebrew}{מֹעֲכוּ} (3 c.\ pl.), \foreignlanguage{hebrew}{וְהִתְבָּרֲכוּ} (3 m.\ pl.\ with waw cons.)
		\item[--] prefix conj.: \foreignlanguage{hebrew}{יְגֹעֲשׁוּ} (3 m.\ pl.), \foreignlanguage{hebrew}{יִתְפָּאֵר} (3 m.\ sg.)
		\item[--] \textit{wayyiqṭol}: \foreignlanguage{hebrew}{וַיְגֹאֲלוּ}, \foreignlanguage{hebrew}{וַיִּתְגָּעֲשׁוּ} (3 m.\ pl.)
		\item[--] participle: \foreignlanguage{hebrew}{מְגֹאָל}, \foreignlanguage{hebrew}{מִתְבָּרֵךְ} (m.\ sg.)
	\end{itemize}
	
	\noindent Where the strong verb without gutturals requires a vocal \textit{šwa} with R\textsubscript{2}, verbs with a guttural as R\textsubscript{2} have \textit{ḥaṭef pataḥ}, e.g., \foreignlanguage{hebrew}{וַיִּתְגָּעֲשׁוּ}.
	
	\vspace{0.25cm}
	
	\noindent Verbs with a guttural as R\textsubscript{1} do not have forms that deviate from the strong verb without gutturals.
	
	In verbs with the gutturals \foreignlanguage{hebrew}{ה}, \foreignlanguage{hebrew}{ח} and \foreignlanguage{hebrew}{ע} as R\textsubscript{3} the vowel /ē/ between R\textsubscript{2} and R\textsubscript{3} is usually replaced by /a/, e.g., \foreignlanguage{hebrew}{יִתְנַגַּח} \textit{he will engage in thrusting with someone}. In the inf.\ cs.\ and the participle the /ē/ may be preserved and a furtive \textit{pataḥ} is inserted, e.g., \foreignlanguage{hebrew}{הִשְׁתַּגֵּעַ}, \foreignlanguage{hebrew}{מִשְׁתַּגֵּעַ} \textit{someone behaving like a mad man}.
	
	
	
	\section{Accents (Continued)}
	
	% A cross-reference to chapter 15 would be useful (action point)
	
	The accents of the Masoretic Text were already briefly introduced in chapter 15. Here all the accents of the two accentuation systems for the three poetic books Psalms, Job, Proverbs (mnemonic word \foreignlanguage{hebrew}{אֱמֶת} for \foreignlanguage{hebrew}{אִיּוֺב מְשָׁלִים תְּהִלִּים} \textit{Job, Proverbs, Psalms}) and the other 21 books are presented. It is not necessary to memorize the names and functions -- conjunctive or disjunctive -- of all the accents. The following tables help to identify accents and their functions.
	
	% The names of the accents follow the JM §g-h as much as possible
	
	
	\subsection*{Disjunctive Accents in the 21 Books}
	
	\begin{longtable}{>{\raggedleft}p{0.15\linewidth} p{0.75\linewidth}}
		\foreignlanguage{hebrew}{דָּבָֽר׃} & \textit{silluq} (always with \foreignlanguage{hebrew}{׃} \textit{sof pasuq}) \\
		\foreignlanguage{hebrew}{דָּבָ֑ר} & \textit{atnaḥ} \\
		\foreignlanguage{hebrew}{דָּבָר֒} & \textit{segolta} (postpositive) \\
		\foreignlanguage{hebrew}{דָּבָ֓ר ׀} & \textit{šalšelet} (with the divider \textit{paseq} \foreignlanguage{hebrew}{׀}) \\
		\foreignlanguage{hebrew}{דָּבָ֔ר} & \textit{zaqef qaṭon} \\
		\foreignlanguage{hebrew}{דָּבָ֕ר} & \textit{zaqef gadol} \\
		\foreignlanguage{hebrew}{דָּבָ֖ר} & \textit{tip̄ḥa} \\
		\foreignlanguage{hebrew}{דָּבָ֗ר} & \textit{rəḇiaʿ} \\
		\foreignlanguage{hebrew}{דָּבָר֘} & \textit{zarqa} (postpositive) \\
		\foreignlanguage{hebrew}{מֶ֨לֶךְ֨}/\foreignlanguage{hebrew}{דָּבָ֨ר} & \textit{pašta} (postpositive) \\
		\foreignlanguage{hebrew}{מֶ֚לֶךְ} & \textit{yətiḇ} (prepositive) \\
		\foreignlanguage{hebrew}{דָּבָ֛ר} & \textit{təḇir} \\
		\foreignlanguage{hebrew}{דָּבָ֜ר} & \textit{gereš} \\
		\foreignlanguage{hebrew}{דָּבָ֞ר} & \textit{geršayim} \\
		\foreignlanguage{hebrew}{דָּבָ֡ר} & \textit{pazer} \\
		\foreignlanguage{hebrew}{דָּבָ֟ר} & \textit{pazer gadol} or \textit{qarne p̄ara} \\
		\foreignlanguage{hebrew}{דָּ֠בָר} & \textit{təliša gədola} (prepositive) \\
		\foreignlanguage{hebrew}{דָּבָ֣ר ׀} & \textit{ləgarme} (with the divider \textit{paseq} \foreignlanguage{hebrew}{׀}) \\
	\end{longtable}
	
	
	\subsection*{Conjunctive Accents in the 21 Books}
	
	\begin{longtable}{>{\raggedleft}p{0.15\linewidth} p{0.75\linewidth}}
		\foreignlanguage{hebrew}{דָּבָ֣ר} & \textit{munaḥ}  \\
		\foreignlanguage{hebrew}{דָּבָ֤ר} & \textit{məhuppaḵ} \\
		\foreignlanguage{hebrew}{דָּבָ֥ר} & \textit{merəḵa} \\
		\foreignlanguage{hebrew}{דָּבָ֦ר} & \textit{merəḵa kəp̄ula} \\
		\foreignlanguage{hebrew}{דָּבָ֧ר} & \textit{darga} \\
		\foreignlanguage{hebrew}{דָּבָ֨ר} & \textit{azla} \\
		\foreignlanguage{hebrew}{דָּבָר֩} & \textit{təliša qətanna} (postpositive) \\
		\foreignlanguage{hebrew}{דָּבָ֪ר} & \textit{galgal} or \textit{yeraḥ} \\
		\foreignlanguage{hebrew}{דָּבָּ֖ר} & \textit{mayəla} \\
	\end{longtable}
	
	\vspace{0.25cm} 
	
	\noindent \textbf{Note}
	\nopagebreak
	
	\noindent
	The accent \textit{mayəla} \foreignlanguage{hebrew}{דָּבָּ֖ר} looks like \textit{tip̄ḥa} \foreignlanguage{hebrew}{דָּבָ֖ר}. It is used to mark secondary stress in words or groups with \textit{silluq} or \textit{atnaḥ}.
	
	
	\subsection*{Disjunctive Accents in the Three Poetic Books}
	
	\begin{longtable}{>{\raggedleft}p{0.15\linewidth} p{0.75\linewidth}}
		\foreignlanguage{hebrew}{דָּבָֽר׃} & \textit{silluq} (always with \foreignlanguage{hebrew}{׃} \textit{Sof pasuq}) \\
		\foreignlanguage{hebrew}{דָּ֫בָּ֥ר} & \textit{ʿole wəyored} \\
		\foreignlanguage{hebrew}{דָּבָ֑ר} & \textit{atnaḥ} \\
		\foreignlanguage{hebrew}{דָּבָ֗ר} & \textit{rəḇiaʿ gadol} \\
		\foreignlanguage{hebrew}{דָּ֜בָ֗ר} & \textit{rəḇiaʿ mugraš} \\
		\foreignlanguage{hebrew}{דָּבָ֓ר ׀} & \textit{šalšelet gədola} (with the divider \textit{paseq} \foreignlanguage{hebrew}{׀}) \\
		\foreignlanguage{hebrew}{דָּבָר֘} & \textit{ṣinnor} or \textit{zarqa} (postpositive) \\
		\foreignlanguage{hebrew}{דָּבָ֗ר} & \textit{rəḇiaʿ qatan} (if followed by \textit{ʿole wəyored}) \\
		\foreignlanguage{hebrew}{דָּ֭בָר} & \textit{dəhi} (prepositive) \\
		\foreignlanguage{hebrew}{דָּבָ֡ר} & \textit{pazer qatan}  \\
		\foreignlanguage{hebrew}{דָּבָ֤ר ׀} & \textit{məhuppaḵ ləgarme} \\
		\foreignlanguage{hebrew}{דָּבָ֨ר ׀} & \textit{azla ləgarme}   \\
	\end{longtable}
	
	
	\subsection*{Conjunctive Accents in the Three Poetic Books}
	
	\begin{longtable}{>{\raggedleft}p{0.15\linewidth} p{0.75\linewidth}}
		\foreignlanguage{hebrew}{דָּבָ֣ר} & \textit{munaḥ}  \\
		\foreignlanguage{hebrew}{דָּבָ֥ר} & \textit{merəḵa} \\
		\foreignlanguage{hebrew}{דָּבָ֬ר} & \textit{ʿilluy} \\
		\foreignlanguage{hebrew}{דָּבָ֖ר} & \textit{tarḥa} \\
		\foreignlanguage{hebrew}{דָּבָ֪ר} & \textit{galgal} or \textit{yeraḥ} \\
		\foreignlanguage{hebrew}{דָּבָ֤ר} & \textit{məhuppaḵ} \\
		\foreignlanguage{hebrew}{דָּבָ֨ר} & \textit{azla} \\
		\foreignlanguage{hebrew}{דָּבָ֓ר} & \textit{šalšelet qətanna} \\
		\foreignlanguage{hebrew}{דָּ֮בָ֤ר}/\foreignlanguage{hebrew}{דָּ֮בָ֥ר} & \textit{ṣinnorit} \\
	\end{longtable}
	
	\vspace{0.25cm} 
	
	\noindent \textbf{Note}
	\nopagebreak
	
	\noindent The accent \textit{ṣinnorit} is used on open syllables before \textit{merəḵa} or \textit{məhuppaḵ}.
	
	
	\section{Exercises}
	
	\subsection{Translation of Verbal Forms}
	
	Translate the following verbal forms. Identify the gender (masc., fem., comm.) and number (sg., pl.) of forms of which the English translation is ambiguous (i.e., \textit{you}, \textit{they}). Mark the stressed syllable if stress is not on the last syllable.
	
	\hspace{0.5cm}
	
	\selectlanguage{hebrew}
	
	\noindent
	1~~\foreignlanguage{hebrew}{שֻׁלְּחוּ}  \hspace{0.3cm}
	2~~\foreignlanguage{hebrew}{תְּבֹרַךְ}  \hspace{0.3cm}
	3~~\foreignlanguage{hebrew}{יְבֻקַּשׁ}  \hspace{0.3cm}
	4~~\foreignlanguage{hebrew}{תְּבֻקְשִׁי}  \hspace{0.3cm}
	5~~\foreignlanguage{hebrew}{יְשֻׁלַּח}  \hspace{0.3cm}
	6~~\foreignlanguage{hebrew}{תְּקֻלַּל}  \hspace{0.3cm}
	7~~\foreignlanguage{hebrew}{הִתְהַלֵּךְ}  \hspace{0.3cm}
	8~~\foreignlanguage{hebrew}{הִתְהַלַּכְתִּי}  \hspace{0.3cm}
	9~~\foreignlanguage{hebrew}{הִתְהַלַּכְנוּ}  \hspace{0.3cm}
	10~~\foreignlanguage{hebrew}{הִתְהַלְּכוּ}  \hspace{0.3cm}
	11~~\foreignlanguage{hebrew}{וַיִּתְהַלְּכוּ}  \hspace{0.3cm}
	12~~\foreignlanguage{hebrew}{יִתְפַּלֵּל}  \hspace{0.3cm}
	13~~\foreignlanguage{hebrew}{וַתִּתְפַּלֵּל}  \hspace{0.3cm}
	14~~\foreignlanguage{hebrew}{הִתְפַּלָּלְתִּי}  \hspace{0.3cm}
	15~~\foreignlanguage{hebrew}{‎הִתְקַדְּשׁוּ}  \hspace{0.3cm}
	\selectlanguage{english}
	
	
	\subsection{Translation of Sentences}
	
	Translate the following sentences from the Hebrew Bible. Names of persons and geographical names in these sentences: \foreignlanguage{hebrew}{אַבְשָׁלוֺם}, \foreignlanguage{hebrew}{אַהֲרֹן}, \foreignlanguage{hebrew}{אֻזִּיָּהוּ}, \foreignlanguage{hebrew}{אַחְאָב}, \foreignlanguage{hebrew}{אִיזֶבֶל}, \foreignlanguage{hebrew}{אֱלִישָׁע}, \foreignlanguage{hebrew}{אֲמִינוֺן}, \foreignlanguage{hebrew}{אֲרָם}, \foreignlanguage{hebrew}{חִזְקִיָּהוּ}, \foreignlanguage{hebrew}{חֲנוֺךְ}, \foreignlanguage{hebrew}{מִצְפָּה}, \foreignlanguage{hebrew}{נָבוֺת}, \foreignlanguage{hebrew}{עֲזַרְיָהוּ}, \foreignlanguage{hebrew}{שְׁמוּאֵל}, \foreignlanguage{hebrew}{תָּמָר}
	
	\vspace{0.5cm}
	
	\selectlanguage{hebrew}
	\noindent 
	1~~\foreignlanguage{hebrew}{יְהִ֤י שֵׁ֣ם יְהוָ֣ה מְבֹרָ֑ךְ מֵֽ֝עַתָּ֗ה וְעַד־עוֹלָֽם׃}  \hspace{0.3cm}
	2~~\foreignlanguage{hebrew}{וַיָּבֹא֙  [אֻזִּיָּהוּ הַמֶּלֶךְ] אֶל־הֵיכַ֣ל יְהוָ֔ה לְהַקְטִ֖יר עַל־מִזְבַּ֥ח הַקְּטֹֽרֶת׃ וַיָּבֹ֥א אַחֲרָ֖יו עֲזַרְיָ֣הוּ הַכֹּהֵ֑ן וְעִמּ֞וֹ כֹּהֲנִ֧ים ׀ לַיהוָ֛ה שְׁמוֹנִ֖ים בְּנֵי־חָֽיִל׃ וַיַּעַמְד֞וּ עַל־עֻזִּיָּ֣הוּ הַמֶּ֗לֶךְ וַיֹּ֤אמְרוּ לוֹ֙ לֹא־לְךָ֣ עֻזִּיָּ֗הוּ לְהַקְטִיר֙ לַֽיהוָ֔ה כִּ֣י לַכֹּהֲנִ֧ים בְּנֵי־אַהֲרֹ֛ן הַמְקֻדָּשִׁ֖ים לְהַקְטִ֑יר צֵ֤א מִן־הַמִּקְדָּשׁ֙ כִּ֣י מָעַ֔לְתָּ}\LTRfootnote{\space \foreignlanguage{hebrew}{מעל} Q.\ \textit{to be untrue, violate one's legal obligation}} \hspace{0.3cm}
	3~~\foreignlanguage{hebrew}{וַֽיִּשְׁלְח֖וּ אֶל־אִיזֶ֣בֶל לֵאמֹ֑ר סֻקַּ֥ל}\LTRfootnote{\space \foreignlanguage{hebrew}{סקל} Q.\ \textit{to stone}; Pu. \textit{to be stoned}} \foreignlanguage{hebrew}{נָב֖וֹת וַיָּמֹֽת׃ וַֽיְהִי֙ כִּשְׁמֹ֣עַ אִיזֶ֔בֶל כִּֽי־סֻקַּ֥ל נָב֖וֹת וַיָּמֹ֑ת וַתֹּ֨אמֶר אִיזֶ֜בֶל אֶל־אַחְאָ֗ב ק֣וּם רֵ֞שׁ אֶת־כֶּ֣רֶם ׀ נָב֣וֹת הַיִּזְרְעֵאלִ֗י אֲשֶׁ֤ר מֵאֵן֙ לָתֶת־לְךָ֣ בְכֶ֔סֶף כִּ֣י אֵ֥ין נָב֛וֹת חַ֖י כִּי־מֵֽת׃ וַיְהִ֛י כִּשְׁמֹ֥עַ אַחְאָ֖ב כִּ֣י מֵ֣ת נָב֑וֹת וַיָּ֣קָם אַחְאָ֗ב לָרֶ֛דֶת אֶל־כֶּ֛רֶם נָב֥וֹת הַיִּזְרְעֵאלִ֖י לְרִשְׁתּֽוֹ׃} \hspace{0.3cm}
	4~~\foreignlanguage{hebrew}{וּֽבְאַהֲרֹ֗ן הִתְאַנַּ֧ף}\LTRfootnote{\space \foreignlanguage{hebrew}{אנף} Hitpa.\ \textit{to be, become angry}} \foreignlanguage{hebrew}{יְהוָ֛ה מְאֹ֖ד לְהַשְׁמִיד֑וֹ וָֽאֶתְפַּלֵּ֛ל גַּם־בְּעַ֥ד אַהֲרֹ֖ן בָּעֵ֥ת הַהִֽיא} \hspace{0.3cm}
	5~~\foreignlanguage{hebrew}{הִֽתְחַזְּק֞וּ וִֽהְי֤וּ לֽ͏ַאֲנָשִׁים֙ פְּלִשְׁתִּ֔ים}\LTRfootnote{\space \foreignlanguage{hebrew}{פְּלִשְׁתִּים} vocative (direct address)} \foreignlanguage{hebrew}{פֶּ֚ן תַּעַבְד֣וּ לָעִבְרִ֔ים כַּאֲשֶׁ֥ר עָבְד֖וּ לָכֶ֑ם וִהְיִיתֶ֥ם לַאֲנָשִׁ֖ים וְנִלְחַמְתֶּֽם׃} \hspace{0.3cm}
	6~~\foreignlanguage{hebrew}{וַיֹּ֣אמֶר שְׁמוּאֵ֔ל קִבְצ֥וּ אֶת־כָּל־יִשְׂרָאֵ֖ל הַמִּצְפָּ֑תָה וְאֶתְפַּלֵּ֥ל בַּעַדְכֶ֖ם אֶל־יְהוָֽה׃}  \hspace{0.3cm}
	7~~\foreignlanguage{hebrew}{וַיֹּאמְר֨וּ כָל־הָעָ֜ם אֶל־שְׁמוּאֵ֗ל הִתְפַּלֵּ֧ל בְּעַד־עֲבָדֶ֛יךָ אֶל־יְהוָ֥ה אֱלֹהֶ֖יךָ וְאַל־נָמ֑וּת כִּֽי־יָסַ֤פְנוּ}\LTRfootnote{\space \foreignlanguage{hebrew}{יסף} Q.\ \textit{to add}} \foreignlanguage{hebrew}{עַל־כָּל־חַטֹּאתֵ֙ינוּ֙ רָעָ֔ה לִשְׁאֹ֥ל לָ֖נוּ מֶֽלֶךְ׃} \hspace{0.3cm}
	8~~\foreignlanguage{hebrew}{וַיִּתְפַּלֵּ֤ל אֱלִישָׁע֙ וַיֹּאמַ֔ר יְהוָ֕ה פְּקַח־נָ֥א}\LTRfootnote{\space \foreignlanguage{hebrew}{פקח} Q.\ \textit{to open}} \foreignlanguage{hebrew}{אֶת־עֵינָ֖יו וְיִרְאֶ֑ה וַיִּפְקַ֤ח יְהוָה֙ אֶת־עֵינֵ֣י הַנַּ֔עַר וַיַּ֗רְא וְהִנֵּ֨ה הָהָ֜ר מָלֵ֨א סוּסִ֥ים וְרֶ֛כֶב אֵ֖שׁ סְבִיבֹ֥ת אֱלִישָֽׁע׃} \hspace{0.3cm}
	9~~\foreignlanguage{hebrew}{וַיִּגַּ֤שׁ הַנָּבִיא֙ אֶל־מֶ֣לֶךְ יִשְׂרָאֵ֔ל וַיֹּ֤אמֶר לוֹ֙ לֵ֣ךְ הִתְחַזַּ֔ק וְדַ֥ע וּרְאֵ֖ה אֵ֣ת אֲשֶֽׁר־תַּעֲשֶׂ֑ה כִּ֚י לִתְשׁוּבַ֣ת}\LTRfootnote{\space \foreignlanguage{hebrew}{תְּשׁוּבָה} \textit{return} (noun)} \foreignlanguage{hebrew}{הַשָּׁנָ֔ה מֶ֥לֶךְ אֲרָ֖ם עֹלֶ֥ה עָלֶֽיךָ׃} \hspace{0.3cm}
	10~~\foreignlanguage{hebrew}{וַיִּתְפַּלֵּ֨ל חִזְקִיָּ֜הוּ לִפְנֵ֣י יְהוָה֮ וַיֹּאמַר֒ יְהוָ֞ה אֱלֹהֵ֤י יִשְׂרָאֵל֙ יֹשֵׁ֣ב הַכְּרֻבִ֔ים אַתָּה־ה֤וּא הָֽאֱלֹהִים֙ לְבַדְּךָ֔ לְכֹ֖ל מַמְלְכ֣וֹת הָאָ֑רֶץ אַתָּ֣ה עָשִׂ֔יתָ אֶת־הַשָּׁמַ֖יִם וְאֶת־הָאָֽרֶץ׃}  \hspace{0.3cm}
	11~~\foreignlanguage{hebrew}{וַיִּתְהַלֵּ֥ךְ חֲנ֖וֹךְ אֶת־הֽ͏ָאֱלֹהִ֑ים וְאֵינֶ֕נּוּ כִּֽי־לָקַ֥ח אֹת֖וֹ אֱלֹהִֽים׃}  \hspace{0.3cm}
	12~~\foreignlanguage{hebrew}{וַיֹּ֨אמֶר אֵלֶ֜יהָ אַבְשָׁל֣וֹם אָחִ֗יהָ הַאֲמִינ֣וֹן}\LTRfootnote{\space \foreignlanguage{hebrew}{אֲמִינוֺן} This form of the name \foreignlanguage{hebrew}{אַמְנוֺן} is found only here in 2\,Sam 13:20.} \foreignlanguage{hebrew}{אָחִיךְ֮ הָיָ֣ה עִמָּךְ֒ וְעַתָּ֞ה אֲחוֹתִ֤י הַחֲרִ֙ישִׁי֙}\LTRfootnote{\space \foreignlanguage{hebrew}{חרשׁ} Hi.\ \textit{to be silent}} \foreignlanguage{hebrew}{אָחִ֣יךְ ה֔וּא אַל־תָּשִׁ֥יתִי אֶת־לִבֵּ֖ךְ לַדָּבָ֣ר הַזֶּ֑ה וַתֵּ֤שֶׁב תָּמָר֙ וְשֹׁ֣מֵמָ֔ה בֵּ֖ית אַבְשָׁל֥וֹם אָחִֽיהָ׃} \hspace{0.3cm}
	
	\selectlanguage{english}
	
	
	\section{Hebrew Reading: Genesis 13:14--18}
	Translate Gen 13:14--18 with the help of notes below the text. Names of persons and places are easily identifiable.
	
	\vspace{0.5cm}
	
	\selectlanguage{hebrew}
	
	\noindent
	\textsuperscript{14}~\foreignlanguage{hebrew}{וֽ͏ַיהוָ֞ה אָמַ֣ר אֶל־אַבְרָ֗ם אַחֲרֵי֙ הִפָּֽרֶד־ל֣וֹט מֽ͏ֵעִמּ֔וֹ שָׂ֣א נָ֤א עֵינֶ֙יךָ֙ וּרְאֵ֔ה מִן־הַמָּק֖וֹם אֲשֶׁר־אַתָּ֣ה שָׁ֑ם צָפֹ֥נָה וָנֶ֖גְבָּה וָקֵ֥דְמָה וָיָֽמָּה׃} \hspace{0.3cm}
	\textsuperscript{15}~\foreignlanguage{hebrew}{כִּ֧י אֶת־כָּל־הָאָ֛רֶץ אֲשֶׁר־אַתָּ֥ה רֹאֶ֖ה לְךָ֣ אֶתְּנֶ֑נָּה וּֽלְזַרְעֲךָ֖ עַד־עוֹלָֽם׃} \hspace{0.3cm}
	\textsuperscript{16}~\foreignlanguage{hebrew}{וְשַׂמְתִּ֥י אֶֽת־זַרְעֲךָ֖ כַּעֲפַ֣ר הָאָ֑רֶץ אֲשֶׁ֣ר ׀ אִם־יוּכַ֣ל אִ֗ישׁ לִמְנוֹת֙ אֶת־עֲפַ֣ר הָאָ֔רֶץ גַּֽם־זַרְעֲךָ֖ יִמָּנֶֽה׃} \hspace{0.3cm}
	\textsuperscript{17}~\foreignlanguage{hebrew}{ק֚וּם הִתְהַלֵּ֣ךְ בָּאָ֔רֶץ לְאָרְכָּ֖הּ וּלְרָחְבָּ֑הּ כִּ֥י לְךָ֖ אֶתְּנֶֽנָּה} \hspace{0.3cm}
	\textsuperscript{18}~\foreignlanguage{hebrew}{וַיֶּאֱהַ֣ל אַבְרָ֗ם וַיָּבֹ֛א וַיֵּ֛שֶׁב בְּאֵלֹנֵ֥י מַמְרֵ֖א אֲשֶׁ֣ר בְּחֶבְר֑וֹן וַיִּֽבֶן־שָׁ֥ם מִזְבֵּ֖חַ לֽ͏ַיהוָֽה׃} \hspace{0.3cm}
	\selectlanguage{english}
	
	
	\hspace*{-0.5cm}\begin{longtable}{p{0.075\linewidth} p{0.1\linewidth}p{0.725\linewidth}}
		13:14 & \foreignlanguage{hebrew}{פרד} & Ni.\ \textit{to divide, separate} (intrans.) \\
		13:16 & \foreignlanguage{hebrew}{אֲשֶׁר} & here \textit{so that} \\
		& \foreignlanguage{hebrew}{מנה} & Q.\ \textit{to count, number} \\
		& \foreignlanguage{hebrew}{יִמָּנֶה} & \textit{it will be counted} (Ni. of \foreignlanguage{hebrew}{מנה}) \\
		13:18  & \foreignlanguage{hebrew}{אהל} & \textit{to obtain grazing rights} (\textit{HALOT}); traditionally \textit{to tent} (BDB) \\
		& \foreignlanguage{hebrew}{אֵלוֺן} & \textit{tall tree, terebinth} \\
		& \foreignlanguage{hebrew}{מַמְרֵא} & place name \textit{Mamre} \\
	\end{longtable}
	
	
	
	\chapter{Chapter 21}
	
	\renewcommand\arraystretch{1.4}
	
	\section{Vocabulary}
	
	\subsection{Verbs}
	
	
	% For the centering of the separation between the two columns see the documentation of the array package, page 2 
	
	% Not included:
	%\foreignlanguage{hebrew}{נדח}
	%\foreignlanguage{hebrew}{נכר} & Hi. \textit{to recognize; to acknowledge; to investigate} \\ % HALOT
	
	\begin{longtable}{>{\raggedleft}p{0.175\linewidth} p{0.75\linewidth}}
		\foreignlanguage{hebrew}{נבט} & Hi.\ \textit{to look} \\
		\foreignlanguage{hebrew}{נגד} & Hi.\ \textit{to tell, inform, report} \\
		\foreignlanguage{hebrew}{נחל} & Q.\ \textit{to get a possession; to take a possession; to inherit} \\ % Hi. only 17 occurrences to give as a possession; to give as an inheritance
		\foreignlanguage{hebrew}{נחם} & Ni.\ \textit{to regret; to be sorry; to console oneself}; Pi.\ \textit{to comfort} \\ % HALOT
		\foreignlanguage{hebrew}{נצב} & Ni.\ \textit{to take one's stand, to stand} \\ % BDB, HALOT. Only 21 occurrences in the Hi. to place; to set up; to cause to stand; to fix, establish
		\foreignlanguage{hebrew}{נצל} & Hi.\ \textit{to take away, snatch away; to rescue; to deliver from} \\ % BDB
		\foreignlanguage{hebrew}{נשׂג} & Hi.\ \textit{to reach; to overtake} \\ % BDB
		\foreignlanguage{hebrew}{קנה} & Q.\ \textit{to buy; to acquire; to create} \\
	\end{longtable}
	
	
	\subsection{Nouns}
	
	\begin{longtable}{>{\raggedleft}p{0.175\linewidth} p{0.75\linewidth}}
		\foreignlanguage{hebrew}{אָוֶן} & \textit{disaster; sin, injustice; deception, nothingness; idolatry} \\ % HALOT
		\foreignlanguage{hebrew}{בַּרְזֶל} & \textit{iron} \\
		\foreignlanguage{hebrew}{הֵיכָל} & \textit{palace; temple} \\ % Action point: move to chapter 20 (2025-05-20: why?)
		\foreignlanguage{hebrew}{חָמָס} & \textit{violence, wrong} \\ % included in the Hebrew reading Gen 16:1-6
		\foreignlanguage{hebrew}{מִזְרָח} & \textit{sunrise; east} \\
		\foreignlanguage{hebrew}{מִקְדָּשׁ} & \textit{sacred place, sanctuary} \\ % BDB
		\foreignlanguage{hebrew}{קִיר} & \textit{wall} (pl. \foreignlanguage{hebrew}{קִירוֺת}) \\
		\foreignlanguage{hebrew}{קֵץ} & \textit{end} (gem. noun; with ePP \foreignlanguage{hebrew}{קִצּוֺ}) \\ %  Gen 16:1-6
		\foreignlanguage{hebrew}{קָרְבָּן} & \textit{offering, gift} (only in Lev, Num, Ezek) \\
		\foreignlanguage{hebrew}{קָרוֺב} & \textit{near} \\
		\foreignlanguage{hebrew}{קֶרֶן} & \textit{horn} \\
		\foreignlanguage{hebrew}{שֻׁלְחָן} & \textit{table} \\
		\foreignlanguage{hebrew}{שָׁלָל} & \textit{spoil, plunder, booty} \\
	\end{longtable}
	
	
	\subsection{Other Parts of Speech}
	
	\begin{longtable}{>{\raggedleft}p{0.175\linewidth} p{0.75\linewidth}}
		\foreignlanguage{hebrew}{יַעַן} ,\foreignlanguage{hebrew}{יַעַן אֲשֶׁר}  & \textit{because} \\
		\foreignlanguage{hebrew}{עַל אֲשֶׁר} & \textit{because} \\
	\end{longtable}
	
	
	
	\section{Verbs I \textit{n} and Verbs I ʾ: Derived Binyanim}
	
	\subsection{General Overview of the Forms of Verbs I \textit{n}}
	
	When the first root consonant /n/ is immediately followed by the second root consonant, the /n/ is assimilated to the following consonant. The result are weak forms. In the Qal, this could be observed in the forms of the prefix conjugation. The assimilation of the first root consonant /n/ happens in the Niphal in the suffix conjugation and the participle. In the Hiphil and Hophal it happens in \emph{all} forms. The Piel, Pual and Hitpael forms of I\,\textit{n} verbs are always strong as the first root consonant /n/ is never immediately followed by a consonant.
	
	\begin{center}
		\begin{longtable}{|lll|r|r|r|}
			\hline
			& & & \multicolumn{1}{c|}{Niphal} & \multicolumn{1}{c|}{Hiphil} & \multicolumn{1}{c|}{Hophal} \\
			\cline{4-6}
			& & & \multicolumn{1}{c|}{\foreignlanguage{hebrew}{נגף}} & \multicolumn{1}{c|}{\foreignlanguage{hebrew}{נגד}} & \multicolumn{1}{c|}{\foreignlanguage{hebrew}{נגד}} \\
			\hline
			\endhead
			\hline
			\endfoot
			SC & sg. & 3 m. & \foreignlanguage{hebrew}{נִגַּף} & \foreignlanguage{hebrew}{הִגִּיד} & \foreignlanguage{hebrew}{הֻגַּד} \\
			& & 3 f. & \foreignlanguage{hebrew}{נִגְּפָה} & \foreignlanguage{hebrew}{הִגִּ֫ידָה} & \foreignlanguage{hebrew}{הֻגְּדָה} \\
			& & 2 m. & \foreignlanguage{hebrew}{נִגַּ֫פְתָּ}  & \foreignlanguage{hebrew}{הִגַּ֫דְתָּ} & \foreignlanguage{hebrew}{הֻגַּדְתָּ} \\
			& & 2 f. & \foreignlanguage{hebrew}{נִגַּפְתְּ} & \foreignlanguage{hebrew}{הִגַּדְתְּ} & \foreignlanguage{hebrew}{הֻגַּדְתְּ} \\
			& & 1 c. & \foreignlanguage{hebrew}{נִגַּ֫פְתִּי} & \foreignlanguage{hebrew}{הִגַּ֫דְתִּי} & \foreignlanguage{hebrew}{הֻגַּדְתִּי} \\
			\hline
			& pl. & 3 c. & \foreignlanguage{hebrew}{נִגְּפוּ} & \foreignlanguage{hebrew}{הִגִּ֫ידוּ} & \foreignlanguage{hebrew}{הֻגְּדוּ} \\
			& & 2 m. & \foreignlanguage{hebrew}{נִגַּפְתֶּם} & \foreignlanguage{hebrew}{הִגַּדְתֶּם} & \foreignlanguage{hebrew}{הֻגַּדְתֶּם} \\
			& & 2 f. & \foreignlanguage{hebrew}{נִגַּפְתֶּן} & \foreignlanguage{hebrew}{הִגַּדְתֶּן} & \foreignlanguage{hebrew}{הֻגַּדְתֶּן} \\
			& & 1 c. & \foreignlanguage{hebrew}{נִגַּ֫פְנוּ} & \foreignlanguage{hebrew}{הִגַּ֫דְנוּ} & \foreignlanguage{hebrew}{הֻגַּדְנוּ} \\
			\hline
			PC & sg. & 3 m. & \foreignlanguage{hebrew}{יִנָּגֵף} & \foreignlanguage{hebrew}{יַגִּיד} & \foreignlanguage{hebrew}{יֻגַּד} \\
			& & 3 f. & \foreignlanguage{hebrew}{תִּנָּגֵף} & \foreignlanguage{hebrew}{תַּגִּיד} & \foreignlanguage{hebrew}{תֻּגַּד} \\
			& & 2 m. & \foreignlanguage{hebrew}{תִּנָּגֵף} & \foreignlanguage{hebrew}{תַּגִּיד} & \foreignlanguage{hebrew}{תֻּגַּד} \\
			& & 2 f. & \foreignlanguage{hebrew}{תִּנָּֽגְפִי} & \foreignlanguage{hebrew}{תַּגִּ֫ידִי} & \foreignlanguage{hebrew}{תֻּגְּדִי} \\
			& & 1 c. & \foreignlanguage{hebrew}{אִנָּגֵף} & \foreignlanguage{hebrew}{אַגִּיד} & \foreignlanguage{hebrew}{אֻגַּד} \\
			\pagebreak
			\hline
			& pl. & 3 c. & \foreignlanguage{hebrew}{יִנָּֽגְפוּ} & \foreignlanguage{hebrew}{יַגִּ֫ידוּ} & \foreignlanguage{hebrew}{יֻגְּדוּ} \\
			& & 3 f. & \foreignlanguage{hebrew}{תִּנָּגַ֫פְנָה} & \foreignlanguage{hebrew}{תַּגֵּ֫דְנָה} & \foreignlanguage{hebrew}{תֻּגַּדְנָה} \\
			& & 2 m. & \foreignlanguage{hebrew}{תִּנָּֽגְפוּ} & \foreignlanguage{hebrew}{תַּגִּ֫ידוּ}  & \foreignlanguage{hebrew}{תֻּגְּדוּ} \\
			& & 2 f. & \foreignlanguage{hebrew}{תִּנָּגַ֫פְנָה} & \foreignlanguage{hebrew}{תַּגֵּ֫דְנָה} & \foreignlanguage{hebrew}{תֻּגַּדְנָה} \\
			& & 1 c. & \foreignlanguage{hebrew}{נִנָָּגֵף} & \foreignlanguage{hebrew}{נַגִּיד} & \foreignlanguage{hebrew}{נֻגַּד} \\
			\hline
			Jussive & sg. & 3 m. & \foreignlanguage{hebrew}{יִנָּגֵף} & \foreignlanguage{hebrew}{יַגֵּד} & \foreignlanguage{hebrew}{יֻגַּד}  \\
			\textit{wayyiqṭol} & sg. & 3 m. & \foreignlanguage{hebrew}{וַיִּנָּ֫גֶף} & \foreignlanguage{hebrew}{וַיַּגֵּד} & \foreignlanguage{hebrew}{וַיֻּגַּד}  \\
			\hline
			Impv. & sg. & m. & \foreignlanguage{hebrew}{הִנָּגֵף} & \foreignlanguage{hebrew}{הַגֵּד} &  \\
			& & & & \foreignlanguage{hebrew}{הַגִּ֫ידָה} & \foreignlanguage{hebrew}{} \\
			& & f. & \foreignlanguage{hebrew}{הִנָּֽגְפִי} & \foreignlanguage{hebrew}{הַגִּ֫ידִי} & \\
			& pl. & m. & \foreignlanguage{hebrew}{הִנָּֽגְפוּ} & \foreignlanguage{hebrew}{הַגִּ֫ידוּ} & \\
			& & f. & \foreignlanguage{hebrew}{הִנָּֽגְפוּ} & \foreignlanguage{hebrew}{הַגֵּ֫דְנָה} & \\
			\hline
			Inf.\ cs.\ & & & \foreignlanguage{hebrew}{הִנָּגֵף} & \foreignlanguage{hebrew}{הַגִּיד} & \foreignlanguage{hebrew}{הֻגַּד} \\
			Inf.\ abs.\ & & & \foreignlanguage{hebrew}{הִנָּזֵר} & \foreignlanguage{hebrew}{הַגֵּד} & \foreignlanguage{hebrew}{הֻגֵּד} \\
			& & & \foreignlanguage{hebrew}{נִגּוֺף} & &  \\
			\hline
			Part. & sg. & m. & \foreignlanguage{hebrew}{נִגָּף} & \foreignlanguage{hebrew}{מַגִּיד} & \foreignlanguage{hebrew}{מֻגָּד} \\
			& & f. & \foreignlanguage{hebrew}{נִגֶּ֫פֶת} & \foreignlanguage{hebrew}{מַגֶּ֫דֶת} & \foreignlanguage{hebrew}{מֻגֶּדֶת} \\
			& pl. & m. & \foreignlanguage{hebrew}{נִגָּפִים} & \foreignlanguage{hebrew}{מַגִּידִים} & \foreignlanguage{hebrew}{מֻגָּדִים} \\
			& & f. & \foreignlanguage{hebrew}{נִגָּפוֺת} & \foreignlanguage{hebrew}{מַגִּידוֺת} & \foreignlanguage{hebrew}{מֻגָּדוֺת} \\
		\end{longtable}
	\end{center}
	
	\noindent \textbf{Notes}
	\nopagebreak
	
	\noindent The endings marking person, gender and number are the same as the endings of the corresponding forms of the strong verb including the reduction of vowels.
	
	In the suffix conjugation and participle forms of the Niphal, the first root consonant /n/ is assimilated to the second root consonant: \textit{ningap̄ > niggap̄} \foreignlanguage{hebrew}{נִגַּף}.
	
	In the prefix conjugation, the /n/ infix of the Niphal merges with the first root consonant: \textit{yin-nāgēp̄} \foreignlanguage{hebrew}{יִנָּגֵף}.
	
	As with the strong verb, a prosthetic /h/ is added to the form to preserve the structure of the imperative and infinitives in the Niphal.
	
	For the Niphal infinitive absolute only forms that are attested in the Hebrew Bible are mentioned.
	
	In all forms of the Hiphil, the first root consonant /n/ closes the prefix syllable and is assimilated to the second root consonant: \textit{*hingida > *higgid > higgīd} \foreignlanguage{hebrew}{הִגִּיד}.
	
	In the Hophal, the first root consonant /n/ is assimilated to the second root consonant, e.g., \textit{*hungad > huggad} \foreignlanguage{hebrew}{הֻגַּד}. The prefix vowel is always /u/.
	
	\subsection{Verbs I \textit{n} and II Guttural: Derived Binyanim}
	
	In the Niphal, the /n/ of the root is assimilated to the second root consonant. The result are weak forms, e.g., \textit{ninḥam > niḥḥam > niḥam} (with loss of the gemination and preservation of the vowel /i/).
	
	In the Hiphil, however, the /n/ is not assimilated to the second root consonant resulting in strong forms.
	
	The forms of the verbs \foreignlanguage{hebrew}{נחם} and \foreignlanguage{hebrew}{נחל} in the following table illustrate this.
	
	\begin{center}
		\begin{tabular}{lllrrr}
			& & & Niphal & Hiphil & Hophal \\
			SC & sg. & 3 m. & \foreignlanguage{hebrew}{נִחַם} & & \\
			& & 1 c. & \foreignlanguage{hebrew}{נִחַ֫מְתִּי} & \foreignlanguage{hebrew}{הִנְחַלְתִּי} & \foreignlanguage{hebrew}{הָנְחַלְתִּי} \\
			PC & sg. & 3 m. & \foreignlanguage{hebrew}{יִנָּחֵם} & \foreignlanguage{hebrew}{יַנְחִיל} & \\
			\textit{wayyiqṭol} & sg. & 3 m. & \foreignlanguage{hebrew}{וַיִּנָּ֫חֶם} & & \\
			Inf.\ cs. & & & \foreignlanguage{hebrew}{הִנָּחֵם} & \foreignlanguage{hebrew}{הַנְחִיל} & \\
			Part. & & m.\ sg.\ & \foreignlanguage{hebrew}{נִחָם} & \foreignlanguage{hebrew}{מַנְחִיל} & \\
		\end{tabular}
	\end{center} 
	
	
	\subsection{Verb I ʾ: Exceptional Forms in the Derived Binyanim}
	
	The forms of I\,ʾ verbs in the Piel, Pual and Hitpael are always strong. In the Niphal, Hiphil and Hophal binyanim they are usually strong displaying only the regular vowel changes of guttural verbs, e.g. PC Ni.\ \foreignlanguage{hebrew}{יֵאָכֵל} \textit{it will be eaten} with loss of the gemination (cf. \foreignlanguage{hebrew}{יִכָּתֵב}). In rare cases, exceptions to this occur with forms with quiescent \foreignlanguage{hebrew}{א}. The following examples illustrate this:
	
	\begin{itemize}[noitemsep]
		\item[--] Niphal: \foreignlanguage{hebrew}{נֹאחֲזוּ} \textit{they have settled} (Josh 22:9)  and \foreignlanguage{hebrew}{וְנֹאחֲזוּ} \textit{and they will settle} (Num 32:30); cp. the strong form \foreignlanguage{hebrew}{נֶאֱחַז} \textit{it was caught} (Gen 22:13)
		\item[--] Hiphil: \foreignlanguage{hebrew}{אֹב֫ידָה} \textit{let me destroy} (verb \foreignlanguage{hebrew}{אבד}; Jer 46:8); cp. the strong form \foreignlanguage{hebrew}{וְהַאֲבַדְתָּ} \textit{and you shall destroy} (Deut 7:24)
	\end{itemize}
	
	\section{Causal Constructions}
	
	Causal constructions indicate the cause of or the reason for a state of affairs. They can be expressed as subordinate clauses or as prepositional phrases.
	
	Frequently used causal conjunctions in Biblical Hebrew are: \foreignlanguage{hebrew}{כִּי} \textit{because, for}, \foreignlanguage{hebrew}{יַעַן}, \foreignlanguage{hebrew}{יַעַן אֲשֶׁר} \textit{because}, \foreignlanguage{hebrew}{עַל אֲשֶׁר} \textit{because}. The causal subordinate clause may precede the main clause (Gen 3:14; 1\,Kgs 20:28) or it may follow it (Gen 5:24; Exod 32:35).
	
	\vspace{0.5cm}
	
	\begin{longtable}{>{\raggedleft}p{0.35\linewidth} p{0.55\linewidth}}
		\foreignlanguage{hebrew}{כִּי עָשִׂיתָ זֹּאת אָרוּר אַתָּה מִכָּל־הַבְּהֵמָה וּמִכֹּל חַיַּת הַשָּׂדֶה} & \textit{Because you have done this, you are cursed among all the cattle and all the animals of the field} (Gen 3:14) \\
		\foreignlanguage{hebrew}{וַיִּתְהַלֵּךְ חֲנוֹךְ אֶת־הָאֱלֹהִים וְאֵינֶנּוּ כִּי־לָקַח אֹתוֹ אֱלֹהִים} & \textit{Enoch walked with God and he was no more because God had taken him} (Gen 5:24) \\
		\foreignlanguage{hebrew}{{כֹּה־אָמַר יְהוָה יַעַן אֲשֶׁר אָמְרוּ אֲרָם אֱלֹהֵי הָרִים יְהוָה וְלֹא־אֱלֹהֵי עֲמָקִים הוּא וְנָתַתִּי אֶת־כָּל־הֶהָמוֹן הַגָּדוֹל הַזֶּה בְּיָדֶךָ וִידַעְתֶּם כִּֽי־אֲנִי יְהוָה}} & \textit{Thus says the Lord: Because the Arameans said that the Lord is a god of mountains and not a god of plains, I will give this entire great multitude in your hand so that you know that I am the Lord} (1\,Kgs 20:28) \\
		\foreignlanguage{hebrew}{וַיִּגֹּף יְהוָה אֶת־הָעָם עַל אֲשֶׁר עָשׂוּ אֶת־הָעֵגֶל אֲשֶׁר עָשָׂה אַהֲרֹן} & \textit{The Lord struck the people because they had made the calf that Aaron had made} (Exod 32:35) \\ % Other options: 2 Sam 3:30; 2 Sam 6:8
	\end{longtable}
	
	\vspace{0.5cm}
	
	Infrequent causal prepositions are: \foreignlanguage{hebrew}{אֲשֶׁר}, \foreignlanguage{hebrew}{בַּאֲשֶׁר}, \foreignlanguage{hebrew}{כַּאֲשֶׁר}, \foreignlanguage{hebrew}{מֵאֲשֶׁר}, \foreignlanguage{hebrew}{יַעַן כִּי}, \foreignlanguage{hebrew}{עַל־דְּבַר אֲשֶׁר}, \foreignlanguage{hebrew}{עַל כִּי}, \foreignlanguage{hebrew}{עֵקֶב}, \foreignlanguage{hebrew}{עֵקֶב אֲשֶׁר}, \foreignlanguage{hebrew}{עֵקֶב כִּי} and \foreignlanguage{hebrew}{תַּחַת אֲשֶׁר}.
	
	% Frequency: \foreignlanguage{hebrew}{עֵקֶב}, \foreignlanguage{hebrew}{עֵקֶב אֲשֶׁר}, \foreignlanguage{hebrew}{עֵקֶב כִּי} 15 times; \foreignlanguage{hebrew}{תַּחַת אֲשֶׁר} 13 times.
	
	\vspace{0.5cm}
	
	\noindent The following prepositions are frequently used with causal meaning:
	
	\begin{longtable}{>{\raggedleft}p{0.1\linewidth} p{0.8\linewidth}}
		\foreignlanguage{hebrew}{בְּ} & \textit{because of}, e.g., \foreignlanguage{hebrew}{בְּצִדְקָתִי} \textit{because of my righteousness} (Deut 9:4), \foreignlanguage{hebrew}{בְּחֶטְאוֹ} \textit{because of his [own] sin} (Deut 24:16) \\
		\foreignlanguage{hebrew}{לְמַעַן} & \textit{because of, for the sake of, on account of}, e.g., \foreignlanguage{hebrew}{לְמַעַן דָּוִד עַבְדִּי וּלְמַעַן יְרוּשָׁלַ͏ִם אֲשֶׁר בָּחָֽרְתִּי} \textit{because of David, my servant, and because of Jerusalem that I have chosen} (1\,Kgs 11:13), \foreignlanguage{hebrew}{לְמַעַן בֵּית־יְהוָה אֱלֹהֵינוּ} \textit{for the sake of the house of the Lord our God} (Ps 122:9) \\
		\foreignlanguage{hebrew}{מִן} & \textit{because of}, e.g., \foreignlanguage{hebrew}{מִמַּרְאֵה עֵינֶיךָ אֲשֶׁר תִּרְאֶה} lit.\ \textit{because of the sight of your eyes that you will see} (Deut 28:34), \foreignlanguage{hebrew}{מֵרָעַת יֹשְׁבֵי־בָהּ} \textit{because of the evil of those who dwell in it} (Jer 12:4) \\
		\foreignlanguage{hebrew}{מִפְּנֵי} & \textit{because of}, e.g., \foreignlanguage{hebrew}{מִפְּנֵי הָרָעָב} \textit{because of the famine} (Gen 47:13), \foreignlanguage{hebrew}{מִפְּנֵי רָעַת עַמִּי יִשְׂרָאֵל} \textit{because the evil of my people Israel} (Jer 7:12) \\
		\foreignlanguage{hebrew}{עַל} & \textit{because of}, e.g., \foreignlanguage{hebrew}{עַל כָּל־הָרָעָה אֲשֶׁר עָשָׂה} \textit{because of all the evil that [the people] has done} (Deut 31:18), \foreignlanguage{hebrew}{עַל־הַשְּׁבוּעָה אֲשֶׁר־נִשְׁבַּעְנוּ לָהֶם} \textit{because of the oath that we swore to them} (Josh 9:20) \\
	\end{longtable}
	
	
	\section{Impersonal Constructions (Vague Subject)}
	
	% JM §155b-i; JM §128b
	
	The vague subject (as \textit{one} in the English clause \textit{One never knows}) is used when the identification of the agent of the action is impossible or irrelevant. In Biblical Hebrew, the vague subject is expressed often by 3 c./m.\ pl.\ verbal forms (2 Sam 19:9), 3 m.\ sg.\ verbal forms (Num 35:30) or passive verbal forms (Gen 27:42). In English, passive verbal forms are often employed to translate these impersonal constructions.
	
	\vspace{0.5cm}
	
	% Other possible Example: 1 Sam 27:4
	
	\begin{longtable}{>{\raggedleft}p{0.35\linewidth} p{0.55\linewidth}}
		\foreignlanguage{hebrew}{וּלְכָל־הָעָם הִגִּידוּ לֵאמֹר הִנֵּה הַמֶּלֶךְ יוֹשֵׁב בַּשַּׁעַר} &  \textit{And all the people were told, \enquote{Behold, the king is sitting in the gate}} (2 Sam 19:9) \\
		\foreignlanguage{hebrew}{כָּל־מַכֵּה־נֶפֶשׁ לְפִי עֵדִים יִרְצַח אֶת־הָרֹצֵחַ} & \textit{Everyone who kills a person -- one shall kill the murderer on the evidence of witnesses} (Num 35:30) \\
		\foreignlanguage{hebrew}{וַיֻּגַּד לְמֶלֶךְ מִצְרַיִם כִּי בָרַח הָעָם} &  \textit{And it was reported to the king of Egypt that the people had fled} (Exod 14:5)\\
	\end{longtable}
	
	
	
	\section{Exercises}
	
	\subsection{Translation of Verbal Forms}
	
	Translate the following verbal forms. Identify the gender (masc., fem., comm.) and number (sg., pl.) of forms of which the English translation is ambiguous (i.e., \textit{you}, \textit{they}). Mark the stressed syllable if stress is not on the last syllable.
	
	\hspace{0.5cm}
	
	\selectlanguage{hebrew}
	
	\noindent
	1~~\foreignlanguage{hebrew}{הִבִּיטוּ}  \hspace{0.3cm}
	2~~\foreignlanguage{hebrew}{הִבַּטְתֶּם}  \hspace{0.3cm}
	3~~\foreignlanguage{hebrew}{אַבִּיט}  \hspace{0.3cm}
	4~~\foreignlanguage{hebrew}{יַבִּיטוּ}  \hspace{0.3cm}
	5~~\foreignlanguage{hebrew}{הִגַּדְתִּי}  \hspace{0.3cm}
	6~~\foreignlanguage{hebrew}{הִגִּידוּ}  \hspace{0.3cm}
	7~~\foreignlanguage{hebrew}{הִגַּדְתָּ}  \hspace{0.3cm}
	8~~\foreignlanguage{hebrew}{הִגִּידָה}  \hspace{0.3cm}
	9~~\foreignlanguage{hebrew}{וַיַּגֵּד}  \hspace{0.3cm}
	10~~\foreignlanguage{hebrew}{תַּגִּידִי}  \hspace{0.3cm}
	11~~\foreignlanguage{hebrew}{תַּגִּידוּ}  \hspace{0.3cm}
	12~~\foreignlanguage{hebrew}{אַגִּיד}  \hspace{0.3cm}
	13~~\foreignlanguage{hebrew}{נַגִּיד}  \hspace{0.3cm}
	14~~\foreignlanguage{hebrew}{וַיַּגִּדוּ}  \hspace{0.3cm}
	15~~\foreignlanguage{hebrew}{‎הַגִּידָה}  \hspace{0.3cm}
	16~~\foreignlanguage{hebrew}{הַגֵּד}  \hspace{0.3cm}
	17~~\foreignlanguage{hebrew}{‎נִחַמְתִּי}  \hspace{0.3cm}
	18~~\foreignlanguage{hebrew}{‎יְנַחֵמוּן}  \hspace{0.3cm}
	19~~\foreignlanguage{hebrew}{וַיְנַחֵם}  \hspace{0.3cm}
	20~~\foreignlanguage{hebrew}{הִצִּילוּ}  \hspace{0.3cm}
	21~~\foreignlanguage{hebrew}{הִצַּלְנוּ}  \hspace{0.3cm}
	22~~\foreignlanguage{hebrew}{וַיַּצֵּל}  \hspace{0.3cm}
	23~~\foreignlanguage{hebrew}{הַצִּילוּ}  \hspace{0.3cm}
	24~~\foreignlanguage{hebrew}{יַשִּׂיגוּ}  \hspace{0.3cm}
	
	\selectlanguage{english}
	
	
	\subsection{Translation of Sentences}
	
	Translate the following sentences from the Hebrew Bible. Names of persons and geographical names in these sentences: \foreignlanguage{hebrew}{אַבְרָהָם}, \foreignlanguage{hebrew}{אַבְרָם}, \foreignlanguage{hebrew}{בַּת־שֶׁבַע}, \foreignlanguage{hebrew}{גֹּשֶׁן}, \foreignlanguage{hebrew}{דָּוִד}, \foreignlanguage{hebrew}{יְהוֹשֻׁעַ}, \foreignlanguage{hebrew}{יוֺסֵף}, \foreignlanguage{hebrew}{יַעֲקֹב}, \foreignlanguage{hebrew}{יִצְחָק}, \foreignlanguage{hebrew}{כְּנַעַן}, \foreignlanguage{hebrew}{לָבָן}, \foreignlanguage{hebrew}{מִלְכָּה}, \foreignlanguage{hebrew}{מֹשֶׁה}, \foreignlanguage{hebrew}{נָחוֹר}, \foreignlanguage{hebrew}{שָׁאוּל}, \foreignlanguage{hebrew}{שְׁמוּאֵל}, \foreignlanguage{hebrew}{שִׁמְשׁוֹן}, \foreignlanguage{hebrew}{תִּמְנָה}
	
	\vspace{0.5cm}
	
	\selectlanguage{hebrew}
	\noindent 
	1~~\foreignlanguage{hebrew}{וַיֹּ֗אמֶר אָנֹכִי֙ אֱלֹהֵ֣י אָבִ֔יךָ אֱלֹהֵ֧י אַבְרָהָ֛ם אֱלֹהֵ֥י יִצְחָ֖ק וֵאלֹהֵ֣י יַעֲקֹ֑ב וַיַּסְתֵּ֤ר מֹשֶׁה֙ פָּנָ֔יו כִּ֣י יָרֵ֔א מֵהַבִּ֖יט אֶל־הָאֱלֹהִֽים׃}  \hspace{0.3cm}
	2~~\foreignlanguage{hebrew}{וַיִּקְרָ֤א פַרְעֹה֙ לְאַבְרָ֔ם וַיֹּ֕אמֶר מַה־זֹּ֖את}\LTRfootnote{\space \foreignlanguage{hebrew}{מַה־זֹּ֖את} The demonstrative pronouns \foreignlanguage{hebrew}{זֶה} and \foreignlanguage{hebrew}{זֹאת} often follow an interrogative pronoun or adverb \enquote{without any notable change in meaning} (JM §\,143\,\textit{g})} \foreignlanguage{hebrew}{עָשִׂ֣יתָ לִּ֑י לָ֚מָּה לֹא־הִגַּ֣דְתָּ לִּ֔י כִּ֥י אִשְׁתְּךָ֖ הִֽוא׃}\LTRfootnote{\space \foreignlanguage{hebrew}{הִֽוא} to be read as \foreignlanguage{hebrew}{הִיא}} \hspace{0.3cm}
	3~~\foreignlanguage{hebrew}{וַיֹּ֨אמֶר יוֹסֵ֤ף אֶל־אֶחָיו֙ וְאֶל־בֵּ֣ית אָבִ֔יו אֶעֱלֶ֖ה וְאַגִּ֣ידָה לְפַרְעֹ֑ה וְאֹֽמְרָ֣ה אֵלָ֔יו אַחַ֧י וּבֵית־אָבִ֛י אֲשֶׁ֥ר בְּאֶֽרֶץ־כְּנַ֖עַן בָּ֥אוּ אֵלָֽי׃}  \hspace{0.3cm}
	4~~\foreignlanguage{hebrew}{וַיָּבֹ֣א יוֹסֵף֮ וַיַּגֵּ֣ד לְפַרְעֹה֒ וַיֹּ֗אמֶר אָבִ֨י וְאַחַ֜י וְצֹאנָ֤ם וּבְקָרָם֙ וְכָל־אֲשֶׁ֣ר לָהֶ֔ם בָּ֖אוּ מֵאֶ֣רֶץ כְּנָ֑עַן וְהִנָּ֖ם בְּאֶ֥רֶץ גֹּֽשֶׁן׃}  \hspace{0.3cm}
	5~~\foreignlanguage{hebrew}{וַיֵּ֥רֶד שִׁמְשׁ֖וֹן תִּמְנָ֑תָה וַיַּ֥רְא אִשָּׁ֛ה בְּתִמְנָ֖תָה מִבְּנ֥וֹת פְּלִשְׁתִּֽים׃ וַיַּ֗עַל וַיַּגֵּד֙ לְאָבִ֣יו וּלְאִמּ֔וֹ וַיֹּ֗אמֶר אִשָּׁ֛ה רָאִ֥יתִי בְתִמְנָ֖תָה מִבְּנ֣וֹת פְּלִשְׁתִּ֑ים וְעַתָּ֕ה קְחוּ־אוֹתָ֥הּ לִּ֖י לְאִשָּֽׁה׃ וַיֹּ֨אמֶר ל֜וֹ אָבִ֣יו וְאִמּ֗וֹ הַאֵין֩ בִּבְנ֨וֹת אַחֶ֤יךָ וּבְכָל־עַמִּי֙ אִשָּׁ֔ה כִּֽי־אַתָּ֤ה הוֹלֵךְ֙ לָקַ֣חַת אִשָּׁ֔ה מִפְּלִשְׁתִּ֖ים הָעֲרֵלִ֑ים}\LTRfootnote{\space \foreignlanguage{hebrew}{עָרֵל} \textit{uncircumcised} (adj.)} \foreignlanguage{hebrew}{וַיֹּ֨אמֶר שִׁמְשׁ֤וֹן אֶל־אָבִיו֙ אוֹתָ֣הּ קַֽח־לִ֔י כִּֽי־הִ֖יא יָשְׁרָ֥ה בְעֵינָֽי׃} \hspace{0.3cm}
	6~~\foreignlanguage{hebrew}{וַיְהִ֛י בַּיּ֥וֹם הַשְּׁבִיעִ֖י וַיָּ֣מָת הַיָּ֑לֶד וַיִּֽרְאוּ֩ עַבְדֵ֨י דָוִ֜ד לְהַגִּ֥יד ל֣וֹ ׀ כִּי־מֵ֣ת הַיֶּ֗לֶד כִּ֤י אָֽמְרוּ֙ הִנֵּה֩ בִהְי֨וֹת הַיֶּ֜לֶד חַ֗י דִּבַּ֤רְנוּ אֵלָיו֙ וְלֹא־שָׁמַ֣ע בְּקוֹלֵ֔נוּ וְאֵ֨יךְ נֹאמַ֥ר אֵלָ֛יו מֵ֥ת הַיֶּ֖לֶד וְעָשָׂ֥ה רָעָֽה׃ וַיַּ֣רְא דָּוִ֗ד כִּ֤י עֲבָדָיו֙ מִֽתְלַחֲשִׁ֔ים}\LTRfootnote{\space \foreignlanguage{hebrew}{לחשׁ} Hitpa. \textit{to whisper one to another}} \foreignlanguage{hebrew}{וַיָּ֥בֶן דָּוִ֖ד כִּ֣י מֵ֣ת הַיָּ֑לֶד וַיֹּ֨אמֶר דָּוִ֧ד אֶל־עֲבָדָ֛יו הֲמֵ֥ת הַיֶּ֖לֶד וַיֹּ֥אמְרוּ מֵֽת׃} \hspace{0.3cm}
	7~~\foreignlanguage{hebrew}{וַיְהִ֗י אַחֲרֵי֙ הַדְּבָרִ֣ים הָאֵ֔לֶּה וַיֻּגַּ֥ד לְאַבְרָהָ֖ם לֵאמֹ֑ר הִ֠נֵּה יָלְדָ֨ה מִלְכָּ֥ה גַם־הִ֛וא}\LTRfootnote{\space \foreignlanguage{hebrew}{הִֽוא} to be read as \foreignlanguage{hebrew}{הִיא}} \foreignlanguage{hebrew}{בָּנִ֖ים לְנָח֥וֹר אָחִֽיךָ׃} \hspace{0.3cm}
	8~~\foreignlanguage{hebrew}{וַיֻּגַּ֥ד לְלָבָ֖ן בַּיּ֣וֹם הַשְּׁלִישִׁ֑י כִּ֥י בָרַ֖ח יַעֲקֹֽב׃}  \hspace{0.3cm}
	9~~\foreignlanguage{hebrew}{וַֽיְהִי֙ דְּבַר־יְהוָ֔ה אֶל־שְׁמוּאֵ֖ל לֵאמֹֽר׃ נִחַ֗מְתִּי כִּֽי־הִמְלַ֤כְתִּי אֶת־שָׁאוּל֙ לְמֶ֔לֶךְ כִּֽי־שָׁב֙ מֵאַֽחֲרַ֔י וְאֶת־דְּבָרַ֖י לֹ֣א הֵקִ֑ים}\LTRfootnote{\space \foreignlanguage{hebrew}{הֵקִ֑ים} \textit{he kept; he carried out}} \foreignlanguage{hebrew}{וַיִּ֙חַר֙ לִשְׁמוּאֵ֔ל וַיִּזְעַ֥ק אֶל־יְהוָ֖ה כָּל־הַלָּֽיְלָה׃} \hspace{0.3cm}
	10~~\foreignlanguage{hebrew}{וַיַּ֣רְא יְהוָ֔ה כִּ֥י רַבָּ֛ה רָעַ֥ת הָאָדָ֖ם בָּאָ֑רֶץ וְכָל־יֵ֙צֶר֙ מַחְשְׁבֹ֣ת}\LTRfootnote{\space \foreignlanguage{hebrew}{יֵצֶר} \textit{inclination, striving}, \foreignlanguage{hebrew}{מַחֲשָׁבָה} \textit{thought, intent}} \foreignlanguage{hebrew}{לִבּ֔וֹ רַ֥ק רַ֖ע כָּל־הַיּֽוֹם׃ וַיִּנָּ֣חֶם יְהוָ֔ה כִּֽי־עָשָׂ֥ה אֶת־הֽ͏ָאָדָ֖ם בָּאָ֑רֶץ וַיִּתְעַצֵּ֖ב}\LTRfootnote{\space \foreignlanguage{hebrew}{עצב} Hitpa.\ \textit{to be deeply worried} (\textit{HALOT})} \foreignlanguage{hebrew}{אֶל־לִבּֽוֹ׃} \hspace{0.3cm}
	11~~\foreignlanguage{hebrew}{וַיֹּ֨אמֶר מֹשֶׁ֤ה אֶל־יְהוֹשֻׁ֙עַ֙ בְּחַר־לָ֣נוּ אֲנָשִׁ֔ים וְצֵ֖א הִלָּחֵ֣ם בַּעֲמָלֵ֑ק מָחָ֗ר אָנֹכִ֤י נִצָּב֙ עַל־רֹ֣אשׁ הַגִּבְעָ֔ה וּמַטֵּ֥ה הָאֱלֹהִ֖ים בְּיָדִֽי׃}  \hspace{0.3cm}
	12~~\foreignlanguage{hebrew}{וַיֹּאמֶר֮ דָּוִד֒ יְהוָ֗ה אֲשֶׁ֨ר הִצִּלַ֜נִי מִיַּ֤ד הָֽאֲרִי֙}\LTRfootnote{\space \foreignlanguage{hebrew}{אֲרִי} \textit{lion}} \foreignlanguage{hebrew}{וּמִיַּ֣ד הַדֹּ֔ב}\LTRfootnote{\space \foreignlanguage{hebrew}{דֹּב} \textit{bear}} \foreignlanguage{hebrew}{ה֣וּא יַצִּילֵ֔נִי מִיַּ֥ד הַפְּלִשְׁתִּ֖י הַזֶּ֑ה וַיֹּ֨אמֶר שָׁא֤וּל אֶל־דָּוִד֙ לֵ֔ךְ וַֽיהוָ֖ה יִהְיֶ֥ה עִמָּֽךְ׃} \hspace{0.3cm}
	13~~\foreignlanguage{hebrew}{וַיִּשְׁאַ֨ל דָּוִ֤ד בַּֽיהוָה֙ לֵאמֹ֔ר אֶרְדֹּ֛ף אַחֲרֵ֥י הַגְּדוּד}\LTRfootnote{\space \foreignlanguage{hebrew}{גְּדוּד} \textit{marauding band}}\foreignlanguage{hebrew}{־הַזֶּ֖ה הַֽאַשִּׂגֶ֑נּוּ וַיֹּ֤אמֶר לוֹ֙ רְדֹ֔ף כִּֽי־הַשֵּׂ֥ג תַּשִּׂ֖יג וְהַצֵּ֥ל תַּצִּֽיל׃} \hspace{0.3cm}
	14~~\foreignlanguage{hebrew}{וַיְנַחֵ֣ם דָּוִ֗ד אֵ֚ת בַּת־שֶׁ֣בַע אִשְׁתּ֔וֹ וַיָּבֹ֥א אֵלֶ֖יהָ וַיִּשְׁכַּ֣ב עִמָּ֑הּ וַתֵּ֣לֶד בֵּ֗ן וַתִּקְרָ֤א אֶת־שְׁמוֹ֙ שְׁלֹמֹ֔ה וַיהוָ֖ה אֲהֵבֽוֹ׃}  \hspace{0.3cm}
	
	\selectlanguage{english}
	
	
	\section{Hebrew Reading: Genesis 16:1--6}
	Translate Gen 16:1--6 with the help of notes below the text.
	
	\vspace{0.5cm}
	
	\selectlanguage{hebrew}
	\noindent
	\textsuperscript{1}~\foreignlanguage{hebrew}{וְשָׂרַי֙ אֵ֣שֶׁת אַבְרָ֔ם לֹ֥א יָלְדָ֖ה ל֑וֹ וְלָ֛הּ שִׁפְחָ֥ה מִצְרִ֖ית וּשְׁמָ֥הּ הָגָֽר׃} \hspace{0.3cm}
	\textsuperscript{2}~\foreignlanguage{hebrew}{וַתֹּ֨אמֶר שָׂרַ֜י אֶל־אַבְרָ֗ם הִנֵּה־נָ֞א עֲצָרַ֤נִי יְהוָה֙ מִלֶּ֔דֶת בֹּא־נָא֙ אֶל־שִׁפְחָתִ֔י אוּלַ֥י אִבָּנֶ֖ה מִמֶּ֑נָּה וַיִּשְׁמַ֥ע אַבְרָ֖ם לְק֥וֹל שָׂרָֽי׃} \hspace{0.3cm}
	\textsuperscript{3}~\foreignlanguage{hebrew}{וַתִּקַּ֞ח שָׂרַ֣י אֵֽשֶׁת־אַבְרָ֗ם אֶת־הָגָ֤ר הַמִּצְרִית֙ שִׁפְחָתָ֔הּ מִקֵּץ֙ עֶ֣שֶׂר שָׁנִ֔ים לְשֶׁ֥בֶת אַבְרָ֖ם בְּאֶ֣רֶץ כְּנָ֑עַן וַתִּתֵּ֥ן אֹתָ֛הּ לְאַבְרָ֥ם אִישָׁ֖הּ ל֥וֹ לְאִשָּֽׁה׃} \hspace{0.3cm}
	\textsuperscript{4}~\foreignlanguage{hebrew}{וַיָּבֹ֥א אֶל־הָגָ֖ר וַתַּ֑הַר וַתֵּ֙רֶא֙ כִּ֣י הָרָ֔תָה וַתֵּקַ֥ל גְּבִרְתָּ֖הּ בְּעֵינֶֽיהָ׃} \hspace{0.3cm}
	\textsuperscript{5}~\foreignlanguage{hebrew}{וַתֹּ֨אמֶר שָׂרַ֣י אֶל־אַבְרָם֮ חֲמָסִ֣י עָלֶיךָ֒ אָנֹכִ֗י נָתַ֤תִּי שִׁפְחָתִי֙ בְּחֵיקֶ֔ךָ וַתֵּ֙רֶא֙ כִּ֣י הָרָ֔תָה וָאֵקַ֖ל בְּעֵינֶ֑יהָ יִשְׁפֹּ֥ט יְהוָ֖ה בֵּינִ֥י וּבֵינֶֽיׄךָ׃} \hspace{0.3cm}
	\textsuperscript{6}~\foreignlanguage{hebrew}{וַיֹּ֨אמֶר אַבְרָ֜ם אֶל־שָׂרַ֗י הִנֵּ֤ה שִׁפְחָתֵךְ֙ בְּיָדֵ֔ךְ עֲשִׂי־לָ֖הּ הַטּ֣וֹב בְּעֵינָ֑יִךְ וַתְּעַנֶּ֣הָ שָׂרַ֔י וַתִּבְרַ֖ח מִפָּנֶֽיהָ׃} \hspace{0.3cm}
	\selectlanguage{english}
	
	\vspace{0.25cm}
	
	\hspace*{-0.5cm}\begin{longtable}{p{0.075\linewidth} p{0.1\linewidth}p{0.725\linewidth}}
		16:1 & \foreignlanguage{hebrew}{מִצְרִי} & \textit{Egyptian} \\
		16:2 & \foreignlanguage{hebrew}{עצר} & Q. \textit{to restrain; to retain} \\
		& \foreignlanguage{hebrew}{אוּלַי} & \textit{perhaps} \\
		& \foreignlanguage{hebrew}{אִבָּנָה} & 1 c.\ sg.\ PC Ni.\ \foreignlanguage{hebrew}{בנה} \\
		16:4 & \foreignlanguage{hebrew}{הרה} & Q. \textit{to conceive, become pregnant} \\
		& \foreignlanguage{hebrew}{גְּבִירָה} & \textit{mistress; lady} (title of the queen mother) (cs.\ st.\ \foreignlanguage{hebrew}{גְּבֶרֶת}) \\
		16:5 & \foreignlanguage{hebrew}{חֵיק} & \textit{lap; fold} (of the garment) \\ % HALOT
		& \foreignlanguage{hebrew}{בֵּינֶיךָ} & The form \foreignlanguage{hebrew}{בֵּינֶךָ} (pausal form of \foreignlanguage{hebrew}{בֵּינְךָ}) is expected here (cf. 1\,Sam 20:42). The sign above the second \textit{yod} of \foreignlanguage{hebrew}{וּבֵינֶֽיׄךָ} in the Masoretic Text is called \textit{extraordinary point} (from Latin \textit{punctum extrordinarium}, pl.\ \textit{puncta extraordinaria}). In manuscripts from Qumran dots above letters indicated that the letters had to be erased (cf. Emanuel Tov, \textit{Textual Criticism of the Hebrew Bible}, 4th ed., 2022, p. 43).  \\
		16:6 & \foreignlanguage{hebrew}{וַתְּעַנֶּ֣הָ} & 3 f.\ sg.\ \textit{wayyiqṭol} \foreignlanguage{hebrew}{ענה} Pi. \textit{to oppress} with ePP 3 f.\ sg. \\
	\end{longtable}
	
	
	\chapter{Chapter 22}
	
	\renewcommand\arraystretch{1.4}
	
	\section{Vocabulary}
	
	
	\subsection{Verbs}
	
	
	% For the centering of the separation between the two columns see the documentation of the array package, page 2 
	
	% Not included:
	%\foreignlanguage{hebrew}{נדח}
	%\foreignlanguage{hebrew}{נכר} & Hi. \textit{to recognize; to acknowledge; to investigate} \\ % HALOT
	
	\begin{longtable}{>{\raggedleft}p{0.175\linewidth} p{0.75\linewidth}}
		\foreignlanguage{hebrew}{זנה} & Q.\ \textit{to commit fornication; to be unfaithful} (in a relationship with God) \\ % HALOT
		\foreignlanguage{hebrew}{חזה} & Q.\ \textit{to see, behold} (almost always poetic) \\ % BDB
		\foreignlanguage{hebrew}{חפץ} & Q.\ \textit{to take pleasure in, desire; to delight in} (freq.\ with prep.\ obj.\ with \foreignlanguage{hebrew}{בְּ}) \\ % HALOT
		\foreignlanguage{hebrew}{טמא} & Q.\ \textit{to be unclean, become unclean} (stative verb, SC \foreignlanguage{hebrew}{טָמֵא}); Pi.\ \textit{to defile; to desecrate} \\ % HALOT for Piel
		\foreignlanguage{hebrew}{כלה} & Q.\ \textit{to be complete, at an end, finished, accomplished, spent}; Pi.\ \textit{to complete, bring to an end, finish; to accomplish; to cause to cease} \\ % BDB
		\foreignlanguage{hebrew}{כסה} & Pi.\ \textit{to cover} \\
		\foreignlanguage{hebrew}{נבא} & Ni.\ \textit{to be in a prophetic trance, behave like a} \foreignlanguage{hebrew}{נָבִיא}; Hitpa.\ \textit{to exhibit the behavior of a} \foreignlanguage{hebrew}{נָבִיא} \\ % HALOT
		\foreignlanguage{hebrew}{נטה} & Q.\ \textit{to stretch out, spread out, extend, bend, turn, incline}; Hi.\ \textit{to turn, incline} \\
		\foreignlanguage{hebrew}{נכה} & Hi.\ \textit{to smite, to strike; to strike dead} \\
		\foreignlanguage{hebrew}{ענה}\textsubscript{2} & Q. \textit{to be bowed down; to be downcast, afflicted}; Pi.\ \textit{to oppress; to mishandle, humble, afflict} \\ % mostly BDB
		\foreignlanguage{hebrew}{פדה} & Q.\ \textit{to buy out, redeem, ransom} \\ % HALOT, BDB
		\foreignlanguage{hebrew}{פלא} & Ni.\ \textit{to be surpassing, extraordinary} \\ % BDB
		\foreignlanguage{hebrew}{צוה} & Pi.\ \textit{to give an order, to command; to order, to instruct; to charge, to commission} (cf. the noun \foreignlanguage{hebrew}{מִצְוָה}) \\
		\foreignlanguage{hebrew}{ראה} & Q.\ \textit{to see}; Ni.\ \textit{to appear, become visible; to present oneself}; Hi.\ \textit{to show, let someone see something} \\ % HALOT
		\foreignlanguage{hebrew}{רבה} & Q.\ \textit{to become numerous, increase; to become great}; Hi.\ \textit{to make numerous; to make great} \\ % HALOT
		\foreignlanguage{hebrew}{רצה} & Q.\ \textit{to be pleased with; to accept favorably} \\
		\foreignlanguage{hebrew}{שׁקה} & Hi.\ \textit{to give to drink; to cause to drink water} (causative counterpart to the verb \foreignlanguage{hebrew}{שׁתה} Q. \textit{to drink}) \\ % BDB
	\end{longtable}
	
	\newpage
	
	\subsection{Nouns}
	
	\begin{longtable}{>{\raggedleft}p{0.175\linewidth} p{0.75\linewidth}}
		\foreignlanguage{hebrew}{בֶּטֶן} & \textit{belly; internal organs} (fem.) \\ % HALOT
		\foreignlanguage{hebrew}{חֶרְפָּה} & \textit{disgrace, shame; reviling, taunt} \\ % HALOT
		\foreignlanguage{hebrew}{מַכָּה} & \textit{blow; wound; plague} \\
		\foreignlanguage{hebrew}{עוֺף} & \textit{birds, fowl; winged insects} (everything that flies) (coll.) \\ % BDB, HALOT
		\foreignlanguage{hebrew}{עָנִי} & \textit{poor; afflicted; humble} \\
		\foreignlanguage{hebrew}{רָצוֺן} & \textit{goodwill; favor; acceptance} \\
		\foreignlanguage{hebrew}{תְּרוּמָה} & \textit{contribution, offering} \\
	\end{longtable}
	
	
	\subsection{Other Parts of Speech}
	
	\begin{longtable}{>{\raggedleft}p{0.175\linewidth} p{0.75\linewidth}}
		\foreignlanguage{hebrew}{טֶרֶם} & \textit{not yet} (adv.); \textit{before} (conj.) \\
		\foreignlanguage{hebrew}{בְּטֶרֶם} & \textit{before} (conj.\ or prep.) \\
	\end{longtable}
	
	
	
	\section{Verbs III \textit{y} and Verbs III ʾ: Derived Binyanim}
	
	\subsection{Preliminary Remarks}
	
	Weak verbs of the type  III\,\textit{y} and III\,ʾ have a weak third root consonant, either in the form of a vocalic element in the case of III\,\textit{y} verbs or in the form of a quiescent \foreignlanguage{hebrew}{א} if it is at the end of the syllable. The main features of all the derived binyanim, however, are located at the beginning of the verbal form in the Niphal, Hiphil, Hophal and Hitpael or in the form of gemination in the second root consonant in the Piel, Pual and Hitpael with the exception of the long vowel /ī/ in a number of Hiphil forms. As a consequence the main features of the derived binyanim do not interact with the weak element of III\,\textit{y} and III\,ʾ verbs.
	
	
	\subsection{General Overview of the Forms of Verbs III \textit{y}}
	
	Verbs of the III\,\textit{y} type are characterized by the long final vowels /ā/ (\textit{qameṣ}), /ǣ/ (\textit{səgol}) /ē/ (\textit{ṣere}) in many forms. These vowels remain the same in all binyanim. Before consonantal suffixes in the suffix conjugation the forms vary between the vowels /ī/ (\textit{ḥireq}) and /ē/ (\textit{ṣere}). These vowels are usually spelled with the vowel letter \foreignlanguage{hebrew}{י}. Vocalic suffixes always follow the second root consonant directly.
	
	\newpage
	
	\begin{center}
		\begin{longtable}{|lll|r|r|r|r|}
			\hline
			& & & \multicolumn{1}{c|}{Qal} & \multicolumn{1}{c|}{Niphal} & \multicolumn{1}{c|}{Hiphil} & \multicolumn{1}{c|}{Hophal} \\
			\hline
			\endhead
			\hline
			\endfoot
			SC & sg. & 3 m. & \foreignlanguage{hebrew}{גָּלָה} & \foreignlanguage{hebrew}{נִגְלָה} & \foreignlanguage{hebrew}{הִגְלָה} & \foreignlanguage{hebrew}{הָגְלָה} \\
			& & 3 f. & \foreignlanguage{hebrew}{גָּלְתָה} & \foreignlanguage{hebrew}{נִגְלְתָה} & \foreignlanguage{hebrew}{הִגְלְתָה} & \foreignlanguage{hebrew}{הָגְלְתָה} \\
			& & 2 m. & \foreignlanguage{hebrew}{גָּלִ֫יתָ}  & \foreignlanguage{hebrew}{נִגְלֵ֫יתָ} & \foreignlanguage{hebrew}{הִגְלִ֫יתָ} & \foreignlanguage{hebrew}{הָגְלֵיתָ} \\
			& & 2 f. & \foreignlanguage{hebrew}{גָּלִית} & \foreignlanguage{hebrew}{נִגְלֵית} & \foreignlanguage{hebrew}{הִגְלִית} & \foreignlanguage{hebrew}{הָגְלֵית} \\
			& & 1 c. & \foreignlanguage{hebrew}{גָּלִ֫יתִי} & \foreignlanguage{hebrew}{נִגְלֵ֫יתִי} & \foreignlanguage{hebrew}{הִגְלִ֫יתִי} & \foreignlanguage{hebrew}{הָגְלֵיתִי} \\
			\hline
			& pl. & 3 c. & \foreignlanguage{hebrew}{גָּלוּ} & \foreignlanguage{hebrew}{נִגְלוּ} & \foreignlanguage{hebrew}{הִגְלוּ} & \foreignlanguage{hebrew}{הָגְלוּ} \\
			& & 2 m. & \foreignlanguage{hebrew}{גְּלִיתֶם} & \foreignlanguage{hebrew}{נִגְלֵיתֶם} & \foreignlanguage{hebrew}{הִגְלִיתֶם} & \foreignlanguage{hebrew}{הָגְלֵיתֶם} \\
			& & 2 f. & \foreignlanguage{hebrew}{גְּלִיתֶן} & \foreignlanguage{hebrew}{נִגְלֵיתֶן} & \foreignlanguage{hebrew}{הִגְלִיתֶן} & \foreignlanguage{hebrew}{הָגְלֵיתֶן} \\
			& & 1 c. & \foreignlanguage{hebrew}{גָּלִ֫ינוּ} & \foreignlanguage{hebrew}{נִגְלֵ֫ינוּ} & \foreignlanguage{hebrew}{הִגְלִ֫ינוּ} & \foreignlanguage{hebrew}{הָגְלֵינוּ} \\
			\hline
			PC & sg. & 3 m. & \foreignlanguage{hebrew}{יִגְלֶה} & \foreignlanguage{hebrew}{יִגָּלֶה} & \foreignlanguage{hebrew}{יַגְלֶה} & \foreignlanguage{hebrew}{יָגְלֶה} \\
			& & 3 f. & \foreignlanguage{hebrew}{תִּגְלֶה} & \foreignlanguage{hebrew}{תִּגָּלֶה} & \foreignlanguage{hebrew}{תַּגְלֶה} & \foreignlanguage{hebrew}{תָּגְלֶה} \\
			& & 2 m. & \foreignlanguage{hebrew}{תִּגְלֶה} & \foreignlanguage{hebrew}{תִּגָּלֶה} & \foreignlanguage{hebrew}{תַּגְלֶה} & \foreignlanguage{hebrew}{תָּגְלֶה} \\
			& & 2 f. & \foreignlanguage{hebrew}{תִּגְלִי} & \foreignlanguage{hebrew}{תִּגָּלִי} & \foreignlanguage{hebrew}{תַּגְלִי} & \foreignlanguage{hebrew}{תָּגְלִי} \\
			& & 1 c. & \foreignlanguage{hebrew}{אֶגְלֶה} & \foreignlanguage{hebrew}{אִגָּלֶה} & \foreignlanguage{hebrew}{אַגְלֶה} & \foreignlanguage{hebrew}{אָגְלֶה} \\
			\hline
			& pl. & 3 c. & \foreignlanguage{hebrew}{יִגְלוּ} & \foreignlanguage{hebrew}{יִגָּלוּ} & \foreignlanguage{hebrew}{יַגְלוּ} & \foreignlanguage{hebrew}{יָגְלוּ} \\
			& & 3 f. & \foreignlanguage{hebrew}{תִּגְלֶ֫ינָה} & \foreignlanguage{hebrew}{תִּגָּלֶ֫ינָה} & \foreignlanguage{hebrew}{תַּגְלֶ֫ינָה} & \foreignlanguage{hebrew}{תָּגְלֶינָה} \\
			& & 2 m. & \foreignlanguage{hebrew}{תִּגְלוּ} & \foreignlanguage{hebrew}{תִּגָּלוּ}  & \foreignlanguage{hebrew}{תַּגְלוּ} & \foreignlanguage{hebrew}{תָּגְלוּ} \\
			& & 2 f. & \foreignlanguage{hebrew}{תִּגְלֶ֫ינָה} & \foreignlanguage{hebrew}{תִּגָּלֶ֫ינָה} & \foreignlanguage{hebrew}{תַּגְלֶ֫ינָה} & \foreignlanguage{hebrew}{תָּגְלֶינָה} \\
			& & 1 c. & \foreignlanguage{hebrew}{נִגְלֶה} & \foreignlanguage{hebrew}{נִגָּלֶה} & \foreignlanguage{hebrew}{נַגְלֶה} & \foreignlanguage{hebrew}{נָגְלֶה} \\
			\hline
			Juss. & sg. & 3 m. & \foreignlanguage{hebrew}{יִ֫גֶל} & \foreignlanguage{hebrew}{} & \foreignlanguage{hebrew}{} & \foreignlanguage{hebrew}{} \\
			\textit{waw}-PC & sg. & 3 m. & \foreignlanguage{hebrew}{וַיִּ֫גֶל} & \foreignlanguage{hebrew}{וַיִּגָּל} & \foreignlanguage{hebrew}{וַיֶּ֫גֶל} & \\
			\hline
			Impv. & sg. & m. & \foreignlanguage{hebrew}{גְּלֵה} & \foreignlanguage{hebrew}{הִגָּלֵה} & \foreignlanguage{hebrew}{הַגְלֵה} &  \\
			& & f. & \foreignlanguage{hebrew}{גְּלִי} & \foreignlanguage{hebrew}{הִגָּלִי} & \foreignlanguage{hebrew}{הַגְלִי} &  \\
			& pl. & m. & \foreignlanguage{hebrew}{גְּלוּ} & \foreignlanguage{hebrew}{הִגָּלוּ} & \foreignlanguage{hebrew}{הַגְלוּ} &  \\
			& & f. & \foreignlanguage{hebrew}{גְּלֶ֫ינָה} & \foreignlanguage{hebrew}{הִגָּלֶ֫ינָה} & \foreignlanguage{hebrew}{הַגְלֶ֫ינָה} &  \\
			\hline
			Inf.\ cs.\ & & & \foreignlanguage{hebrew}{גְּלוֺת} & \foreignlanguage{hebrew}{הִגָּלוֺת} & \foreignlanguage{hebrew}{הַגְלוֺת} & \\
			Inf.\ abs.\ & & & \foreignlanguage{hebrew}{גָּלֹה} & \foreignlanguage{hebrew}{הִגָּלֵּה} & \foreignlanguage{hebrew}{הַגְלֵה} & \foreignlanguage{hebrew}{הָגְלֵה} \\
			& & & & \foreignlanguage{hebrew}{נִגְלֹה} & &  \\
			\hline
			\pagebreak
			\hline
			Part. & sg. & m. & \foreignlanguage{hebrew}{גֹּלֶה} & \foreignlanguage{hebrew}{נִגְלֶה} & \foreignlanguage{hebrew}{מַגְלֶה} & \foreignlanguage{hebrew}{מָגְלֶה} \\
			& & f. & \foreignlanguage{hebrew}{גֹּלָה} & \foreignlanguage{hebrew}{נִגְלָה} & \foreignlanguage{hebrew}{מַגְלָה} & \foreignlanguage{hebrew}{} \\
			& pl. & m. & \foreignlanguage{hebrew}{גֹּלִים} & \foreignlanguage{hebrew}{נִגְלִים} & \foreignlanguage{hebrew}{מַגְלִים} & \foreignlanguage{hebrew}{} \\
			& & f. & \foreignlanguage{hebrew}{גֹּלוֺת} & \foreignlanguage{hebrew}{נִגְלוֺת} & \foreignlanguage{hebrew}{מַגְלוֺת} & \foreignlanguage{hebrew}{} \\
			
		\end{longtable}
	\end{center}
	
	\begin{center}
		\begin{longtable}{|lll|r|r|r|r|}
			\hline
			& & & \multicolumn{1}{c|}{Qal} & \multicolumn{1}{c|}{Piel} & \multicolumn{1}{c|}{Pual} & \multicolumn{1}{c|}{Hitpael} \\
			\hline
			\endhead
			\hline
			\endfoot
			SC & sg. & 3 m. & \foreignlanguage{hebrew}{גָּלָה} & \foreignlanguage{hebrew}{גִּלָּה} & \foreignlanguage{hebrew}{גֻּלָּה} & \foreignlanguage{hebrew}{הִתְגַּלָּה} \\
			& & 3 f. & \foreignlanguage{hebrew}{גָּלְתָה} & \foreignlanguage{hebrew}{גִּלְּתָה} & \foreignlanguage{hebrew}{גֻּלְּתָה} & \foreignlanguage{hebrew}{הִתְגַּלְּתָה} \\
			& & 2 m. & \foreignlanguage{hebrew}{גָּלִ֫יתָ}  & \foreignlanguage{hebrew}{גִּלִּיתָ} & \foreignlanguage{hebrew}{גֻּלֵּיתָ} & \foreignlanguage{hebrew}{הִתְגַּלִּיתָ} \\
			& & 2 f. & \foreignlanguage{hebrew}{גָּלִית} & \foreignlanguage{hebrew}{גִּלִּית} & \foreignlanguage{hebrew}{גֻּלֵּית} & \foreignlanguage{hebrew}{הִתְגַּלִּית} \\
			& & 1 c. & \foreignlanguage{hebrew}{גָּלִ֫יתִי} & \foreignlanguage{hebrew}{גִּלִּיתִי} & \foreignlanguage{hebrew}{גֻּלֵּיתִי} & \foreignlanguage{hebrew}{הִתְגַּלֵּיתִי} \\
			\hline
			& pl. & 3 c. & \foreignlanguage{hebrew}{גָּלוּ} & \foreignlanguage{hebrew}{גִּלּוּ} & \foreignlanguage{hebrew}{גֻּלּוּ} & \foreignlanguage{hebrew}{הִתְגַּלּוּ} \\
			& & 2 m. & \foreignlanguage{hebrew}{גְּלִיתֶם} & \foreignlanguage{hebrew}{גִּלִּיתֶם} & \foreignlanguage{hebrew}{גֻּלֵּיתֶם} & \foreignlanguage{hebrew}{הִתְגַּלִּיתֶם} \\
			& & 2 f. & \foreignlanguage{hebrew}{גְּלִיתֶן} & \foreignlanguage{hebrew}{גִּלִּיתֶן} & \foreignlanguage{hebrew}{גֻּלֵּיתֶן} & \foreignlanguage{hebrew}{הִתְגַּלִּיתֶן} \\
			& & 1 c. & \foreignlanguage{hebrew}{גָּלִ֫ינוּ} & \foreignlanguage{hebrew}{גִּלִּינוּ} & \foreignlanguage{hebrew}{גֻּלֵּינוּ} & \foreignlanguage{hebrew}{הִתְגַּלִּינוּ} \\
			\hline
			PC & sg. & 3 m. & \foreignlanguage{hebrew}{יִגְלֶה} & \foreignlanguage{hebrew}{יְגַלֶּה} & \foreignlanguage{hebrew}{יְגֻלֶּה} & \foreignlanguage{hebrew}{יִתְגַּלֶּה} \\
			& & 3 f. & \foreignlanguage{hebrew}{תִּגְלֶה} & \foreignlanguage{hebrew}{תְּגַלֶּה} & \foreignlanguage{hebrew}{תְּגֻלֶּה} & \foreignlanguage{hebrew}{תִּתְגַּלֶּה} \\
			& & 2 m. & \foreignlanguage{hebrew}{תִּגְלֶה} & \foreignlanguage{hebrew}{תְּגַלֶּה} & \foreignlanguage{hebrew}{תְּגֻלֶּה} & \foreignlanguage{hebrew}{תִּתְגַּלֶּה} \\
			& & 2 f. & \foreignlanguage{hebrew}{תִּגְלִי} & \foreignlanguage{hebrew}{תְּגַּלִּי} & \foreignlanguage{hebrew}{תְּגֻלִּי} & \foreignlanguage{hebrew}{תִּתְגַּלִּי} \\
			& & 1 c. & \foreignlanguage{hebrew}{אֶגְלֶה} & \foreignlanguage{hebrew}{אֲגַלֶּה} & \foreignlanguage{hebrew}{אֲגֻלֶּה} & \foreignlanguage{hebrew}{אֶתְגַּלֶּה} \\
			\hline
			& pl. & 3 c. & \foreignlanguage{hebrew}{יִגְלוּ} & \foreignlanguage{hebrew}{יְגַלּוּ} & \foreignlanguage{hebrew}{יְגֻלּוּ} & \foreignlanguage{hebrew}{יִתְגַּלּוּ} \\
			& & 3 f. & \foreignlanguage{hebrew}{תִּגְלֶ֫ינָה} & \foreignlanguage{hebrew}{תְּגַלֶּינָה} & \foreignlanguage{hebrew}{תְּגֻלֶּינָה} & \foreignlanguage{hebrew}{תִּתְגַּלֶּינָה} \\
			& & 2 m. & \foreignlanguage{hebrew}{תִּגְלוּ} & \foreignlanguage{hebrew}{תְּגַלּוּ} & \foreignlanguage{hebrew}{תְּגֻלּוּ} & \foreignlanguage{hebrew}{תִּתְגַּלּוּ} \\
			& & 2 f. & \foreignlanguage{hebrew}{תִּגְלֶ֫ינָה} & \foreignlanguage{hebrew}{תְּגַלֶּינָה} & \foreignlanguage{hebrew}{תְּגֻלֶּינָה} & \foreignlanguage{hebrew}{תִּתְגַּלֶּינָה} \\
			& & 1 c. & \foreignlanguage{hebrew}{נִגְלֶה} & \foreignlanguage{hebrew}{נְגַלֶּה} & \foreignlanguage{hebrew}{נְגֻלֶּה} & \foreignlanguage{hebrew}{נִתְגַּלֶּה} \\
			\hline
			Juss. & sg. & 3 m. & \foreignlanguage{hebrew}{יִ֫גֶל} & \foreignlanguage{hebrew}{יְגַל} & \foreignlanguage{hebrew}{} & \foreignlanguage{hebrew}{} \\
			\textit{waw}-PC & sg. & 3 m. & \foreignlanguage{hebrew}{וַיִּ֫גֶל} & \foreignlanguage{hebrew}{וַיְגַל} & \foreignlanguage{hebrew}{} & \foreignlanguage{hebrew}{וַיִּתְגַּל} \\
			\pagebreak
			\hline
			Impv. & sg. & m. & \foreignlanguage{hebrew}{גְּלֵה} & \foreignlanguage{hebrew}{גַּל} & & \foreignlanguage{hebrew}{הִתְגַּלֵּה} \\
			& & f. & \foreignlanguage{hebrew}{גְּלִי} & & & \foreignlanguage{hebrew}{הִתְגַּלִּי} \\
			& pl. & m. & \foreignlanguage{hebrew}{גְּלוּ} & \foreignlanguage{hebrew}{גַּלּוּ} & & \foreignlanguage{hebrew}{הִתְגַּלּוּ}  \\
			& & f. & \foreignlanguage{hebrew}{גְּלֶ֫ינָה} & & &  \\
			\hline
			Inf.\ cs.\ & & & \foreignlanguage{hebrew}{גְּלוֺת} & \foreignlanguage{hebrew}{גַּלּוֺת} & \foreignlanguage{hebrew}{גֻּלּוֺת} & \foreignlanguage{hebrew}{הִתְגַּלּוֺת}  \\
			Inf.\ abs.\ & & & \foreignlanguage{hebrew}{גָּלֹה} & \foreignlanguage{hebrew}{גַּלֵּה} & \foreignlanguage{hebrew}{גֻּלֹּה} & \foreignlanguage{hebrew}{הִתגַּלֵּה}  \\
			& & & & \foreignlanguage{hebrew}{גַּלֹּה} & &  \\
			\hline
			Part. & sg. & m. & \foreignlanguage{hebrew}{גֹּלֶה} & \foreignlanguage{hebrew}{מְגַלֶּה} & \foreignlanguage{hebrew}{מְגֻלֶּה} & \foreignlanguage{hebrew}{מִתְגַּלֶּה}  \\
			& & f. & \foreignlanguage{hebrew}{גֹּלָה} & \foreignlanguage{hebrew}{מְגַלָּה} & \foreignlanguage{hebrew}{} & \foreignlanguage{hebrew}{}  \\
			& pl. & m. & \foreignlanguage{hebrew}{גֹּלִים} & \foreignlanguage{hebrew}{מְגַלִּים} & \foreignlanguage{hebrew}{} & \foreignlanguage{hebrew}{}  \\
			& & f. & \foreignlanguage{hebrew}{גֹּלוֺת} & \foreignlanguage{hebrew}{מְגַלּוֺת} & \foreignlanguage{hebrew}{} & \foreignlanguage{hebrew}{}  \\
		\end{longtable}
	\end{center}
	
	\noindent \textbf{Notes}
	\nopagebreak
	
	\noindent Variation between the vowels /ī/ (\textit{ḥireq}) and /ē/ (\textit{ṣere}) after the second root consonant in suffix conjugation forms is attested even within one and the same verse, e.g., \foreignlanguage{hebrew}{וְהִכֵּיתִי} and \foreignlanguage{hebrew}{וְהִכִּיתָ} in 1\,Sam 23:2 (\foreignlanguage{hebrew}{וַיִּשְׁאַל דָּוִד בַּיהוָה לֵאמֹר הַאֵלֵךְ וְהִכֵּיתִי בַּפְּלִשְׁתִּים הָאֵלֶּה וַיֹּאמֶר יְהוָה אֶל־דָּוִד לֵךְ וְהִכִּיתָ בַפְּלִשְׁתִּים וְהוֹשַׁעְתָּ אֶת־קְעִילָה} \textit{David asked the Lord, \enquote{Shall I go and attack these Philistines?} And the Lord said to David, \enquote{Go and attack the Philistines and save Keilah}}.
	
	Verbs that are both I gutt.\ and III\,\textit{y} (e.g., \foreignlanguage{hebrew}{עשׂה}) have the same vowel changes before the first root consonant as the strong verb, e.g., \foreignlanguage{hebrew}{נַעֲשָׂה} Ni.\ \textit{it was done} (with \textit{pataḥ} in the prefix), \foreignlanguage{hebrew}{נֶעֶשׂתָה} Ni.\ \textit{it was done} (with \textit{segol} in the prefix), \foreignlanguage{hebrew}{יֵעָשֶׂה} Ni.\ \textit{it is done} etc. The same applies to the verb \foreignlanguage{hebrew}{ראה} in the Niphal, e.g., \foreignlanguage{hebrew}{יֵרָאֶה} \textit{he shall appear} (Exod 23:17).
	
	As in the Qal, III\,\textit{y} verbs may loose the long final vowel in jussive and \textit{wayyiqṭol} forms without an ending, e.g., \foreignlanguage{hebrew}{וַיְגַל} \textit{and he uncovered} (Pi.), \foreignlanguage{hebrew}{וַתְּגַל} \textit{and she uncovered} (Pi.), \foreignlanguage{hebrew}{וַיִּתְכַּס} \textit{and he covered himself} (Hitpa.).
	
	The verb \foreignlanguage{hebrew}{עלה} is used both in the Qal and in the Hiphil (558 and 250 times in the Hebrew Bible, respectively). Because of the guttural as first root consonant, prefix conjugation forms of the Qal and the Hiphil are identical, e.g., \foreignlanguage{hebrew}{יַעֲלֶה} Q.\ \textit{he will go up} or Hi.\ \textit{he will bring up}, \foreignlanguage{hebrew}{וַיַּ֫עַל} Q.\ \textit{he went up} or Hi.\ \textit{he brought up}. As the verb \foreignlanguage{hebrew}{עלה} is intransitive in the Qal and transitive in the Hiphil, the presence or absence of a direct object makes it possible to distinguish between these forms. (Ellipsis of direct objects is frequent in Biblical Hebrew.)
	
	The verb \foreignlanguage{hebrew}{היה} has suffix conjugation and participle forms in the Ni.\ with the prefix vowel /i/, e.g., \foreignlanguage{hebrew}{נִהְיְתָה} (SC 3 f.\ sg.) \textit{it happened}, \foreignlanguage{hebrew}{נִהְיָה} (part.\ f.\ sg.)
	
	The verbs \foreignlanguage{hebrew}{נטה} Hi.\ \textit{to turn, incline} and \foreignlanguage{hebrew}{נכה} Hi.\ \textit{to smite, strike} are doubly weak: I\,\textit{n} and III\,\textit{y}. The rules regarding sound changes of both weak verb categories I\,\textit{n} and III\,\textit{y} are applied. As a result, Hiphil forms of these two verbs show only R\textsubscript{2} while R\textsubscript{1} /n/ is only present in the gemination of R\textsubscript{2} in non-apocopate forms. The following characteristic Hiphil forms are attested:
	
	\begin{center}
		\begin{tabular}{lrr}
			% & \multicolumn{1}{c}{\foreignlanguage{hebrew}{נטה}} & \multicolumn{1}{c}{\foreignlanguage{hebrew}{נכה}} \\
			SC & \foreignlanguage{hebrew}{הִטָּה} & \foreignlanguage{hebrew}{הִכָּה} \\
			PC & \foreignlanguage{hebrew}{יַטֶּה} & \foreignlanguage{hebrew}{יַכֶּה} \\
			Juss. & \foreignlanguage{hebrew}{תַּט} & \foreignlanguage{hebrew}{יַךְ} \\
			\textit{waw}-PC & \foreignlanguage{hebrew}{וָאַט} ,\foreignlanguage{hebrew}{וַיַּט} & \foreignlanguage{hebrew}{וַיָּ֑ךְ} ,\foreignlanguage{hebrew}{וַיַּךְ} \\
			Impv. & \foreignlanguage{hebrew}{הַט} ,\foreignlanguage{hebrew}{הַטֵּה} & \foreignlanguage{hebrew}{הַךְ} ,\foreignlanguage{hebrew}{הַכֵּה} \\
			Inf.\ cs. & \foreignlanguage{hebrew}{הַטּוֺת} & \foreignlanguage{hebrew}{הַכּוֺת} \\
			Part. & \foreignlanguage{hebrew}{מַטֶּה} & \foreignlanguage{hebrew}{מַכֶּה} \\
		\end{tabular}
	\end{center}
	
	The verb \foreignlanguage{hebrew}{נכה} is also attested in the Pual, e.g., \foreignlanguage{hebrew}{הֻכָּה} (SC), \foreignlanguage{hebrew}{מֻכֶּה} (part.).
	
	As in the strong verb, assimilation of R\textsubscript{1} to R\textsubscript{2} does not happen in the case of a guttural as R\textsubscript{2}, e.g., \foreignlanguage{hebrew}{יַנְחֵנִי} \textit{he leads me} (Ps 23:3; verb \foreignlanguage{hebrew}{נחה} Hi.\ with ePP).
	
	% Only roots with \foreignlanguage{hebrew}{ח} and  \foreignlanguage{hebrew}{ע} as R\textsubscript{2} are attested in Biblical Hebrew.
	
	The infinitive absolute \foreignlanguage{hebrew}{הַרְבֵּה} of the verb \foreignlanguage{hebrew}{רבה} Hi.\ is used as an adverb with the meaning \textit{greatly, exceedingly}. The form \foreignlanguage{hebrew}{הַרְבָּה} is used as a substitute for the infinitive absolute, but the form \foreignlanguage{hebrew}{הַרְבֵּה} is used as infinitive absolute as well.
	
	
	\subsection{General Overview of the Forms of Verbs III ʾ}
	As in the Qal, the weak consonant \foreignlanguage{hebrew}{א} becomes silent at the end of a syllable and, therefore, is not a consonant anymore. However, it is preserved in writing in most cases. When vocalic suffixes or enclitic pronouns are added, the \foreignlanguage{hebrew}{א} is retained as a consonant.\footnote{\space In the tables below, the verb \foreignlanguage{hebrew}{מלא} is used for the Piel, Pual and Hitpael forms because it is actually used in these binyanim. In forms where the \foreignlanguage{hebrew}{ל} has vocal \textit{šwa} it looses the \textit{dageš forte}.}
	
	% For the ṣere in the 1 c. sg. Hitpa. form cf. Jer 17:16 and GKC p. 529
	% For the most forms, the tables follow Lettinga/von Siebenthal, pp. 450-453 except for the choice of the verb מלא for the Piel, Pual, Hitpael because this verb is actually attested in all three binyanim.
	% For the form \foreignlanguage{hebrew}{נִמְצֵאת} (part. fem. sg.) cf. Deut 30:11; 1 Chr 14:2.
	% For the form \foreignlanguage{hebrew}{נִמְצְאִים} (part. masc. pl.) cf. Josh 10:17; 1 Sam 19:10; 1 Kgs 22:12; 2 Chr 5:11.
	% For the form \foreignlanguage{hebrew}{נִמְצָאוֺת} (part. fem. pl.) cf. Gen 19:15; Josh 3.5.
	
	
	\begin{center}
		\begin{longtable}{|lll|r|r|r|r|}
			\hline
			& & & \multicolumn{1}{c|}{Qal} & \multicolumn{1}{c|}{Niphal} & \multicolumn{1}{c|}{Hiphil} & \multicolumn{1}{c|}{Hophal} \\
			\hline
			\endhead
			\hline
			\endfoot
			SC & sg. & 3 m. & \foreignlanguage{hebrew}{מָצָא} & \foreignlanguage{hebrew}{נִמְצָא} & \foreignlanguage{hebrew}{הִמְצִיא} & \foreignlanguage{hebrew}{הָמְצָא} \\
			& & 3 f. & \foreignlanguage{hebrew}{מָצְאָה} & \foreignlanguage{hebrew}{נִמְצְאָה} & \foreignlanguage{hebrew}{הִמְצִ֫יאָה} & \foreignlanguage{hebrew}{הָמְצְאָה}  \\
			& & 2 m. & \foreignlanguage{hebrew}{מָצָאתָ}  & \foreignlanguage{hebrew}{נִמְְצֵ֫אתָ} & \foreignlanguage{hebrew}{הִמְצֵ֫אתָ} & \foreignlanguage{hebrew}{הָמְצֵאתָ}  \\
			& & 2 f. & \foreignlanguage{hebrew}{מָצָאת} & \foreignlanguage{hebrew}{נִמְצֵאת} & \foreignlanguage{hebrew}{הִמְצֵאת} & \foreignlanguage{hebrew}{הָמְצֵאת} \\
			& & 1 c. & \foreignlanguage{hebrew}{מָצָאתִי} & \foreignlanguage{hebrew}{נִמְצֵ֫אתִי} & \foreignlanguage{hebrew}{הִמְצֵ֫אתִי} & \foreignlanguage{hebrew}{הָמְצֵאתִי} \\
			\hline
			& pl. & 3 c. & \foreignlanguage{hebrew}{מָצְאוּ} & \foreignlanguage{hebrew}{נִמְצְאוּ} & \foreignlanguage{hebrew}{הִמְצִ֫יאוּ} & \foreignlanguage{hebrew}{הָמְצְאוּ} \\
			& & 2 m. & \foreignlanguage{hebrew}{מְצָאתֶם} & \foreignlanguage{hebrew}{נִמְצֵאתֶם} & \foreignlanguage{hebrew}{הִמְצָאתֶם} & \foreignlanguage{hebrew}{הָמְצֵאתֶם}  \\
			& & 2 f. & \foreignlanguage{hebrew}{מְצָאתֶן} & \foreignlanguage{hebrew}{נִמְצֵאתֶן} & \foreignlanguage{hebrew}{הִמְצֵאתֶן} & \foreignlanguage{hebrew}{הָמְצֵאתֶן} \\
			& & 1 c. & \foreignlanguage{hebrew}{מָצָאנוּ} & \foreignlanguage{hebrew}{נִמְצֵ֫אנוּ} & \foreignlanguage{hebrew}{הִמְצֵ֫אנוּ} & \foreignlanguage{hebrew}{הָמְצֵאנוּ} \\
			\pagebreak
			\hline
			PC & sg. & 3 m. & \foreignlanguage{hebrew}{יִמְצָא} & \foreignlanguage{hebrew}{יִמָּצֵא} & \foreignlanguage{hebrew}{יַמְצִיא} & \foreignlanguage{hebrew}{יָמְצָא} \\
			& & 3 f. & \foreignlanguage{hebrew}{תִּמְצָא} & \foreignlanguage{hebrew}{תִּמָּצֵא} & \foreignlanguage{hebrew}{תַּמְצִיא} & \\
			& & 2 m. & \foreignlanguage{hebrew}{תִּמְצָא} & \foreignlanguage{hebrew}{תִּמָּצֵא} & \foreignlanguage{hebrew}{תַּמְצִיא} &  \\
			& & 2 f. & \foreignlanguage{hebrew}{תִּמְצְאִי} & \foreignlanguage{hebrew}{תִּמָּֽצְאִי} & \foreignlanguage{hebrew}{תַּמְצִ֫יאִי} & \foreignlanguage{hebrew}{תָּמְצְאִי} \\
			& & 1 c. & \foreignlanguage{hebrew}{אֶמְצָא} & \foreignlanguage{hebrew}{אִמָּצֵא} & \foreignlanguage{hebrew}{אַמְצִיא} &  \\
			\hline
			& pl. & 3 c. & \foreignlanguage{hebrew}{יִמְצְאוּ} & \foreignlanguage{hebrew}{יִמָּֽצְאוּ} & \foreignlanguage{hebrew}{יַמְצִ֫יאוּ} & \foreignlanguage{hebrew}{יָמְצְאוּ} \\
			& & 3 f. & \foreignlanguage{hebrew}{תִּמְצֶ֫אנָה} & \foreignlanguage{hebrew}{תִּמָּצֶ֫אנָה} & \foreignlanguage{hebrew}{תַּמְצֶ֫אנָה} &  \\
			& & 2 m. & \foreignlanguage{hebrew}{תִּמְצְאוּ} & \foreignlanguage{hebrew}{תִּמָּֽצְאוּ}  & \foreignlanguage{hebrew}{תַּמְצִ֫יאוּ} & \foreignlanguage{hebrew}{תָּמְצְאוּ} \\
			& & 2 f. & \foreignlanguage{hebrew}{תִּמְצֶ֫אנָה} & \foreignlanguage{hebrew}{תִּמָּצֶ֫אנָה} & \foreignlanguage{hebrew}{תַּמְצֶ֫אנָה} &  \\
			& & 1 c. & \foreignlanguage{hebrew}{נִמְצָא} & \foreignlanguage{hebrew}{נִמָּצֵא} & \foreignlanguage{hebrew}{נַמְצִיא} &  \\
			\hline
			Juss. & sg. & 3 m. & \foreignlanguage{hebrew}{יִמְצָא} & \foreignlanguage{hebrew}{יִמָּצֵא} & \foreignlanguage{hebrew}{יַמְצִיא} & \foreignlanguage{hebrew}{} \\
			\textit{waw}-PC & sg. & 3 m. & \foreignlanguage{hebrew}{וַיִּמְצָא} & \foreignlanguage{hebrew}{וַיִּמָּצֵא} & \foreignlanguage{hebrew}{וַיַּמְצִיא} & \foreignlanguage{hebrew}{} \\
			\hline
			Impv. & sg. & m. & \foreignlanguage{hebrew}{מְצָא} & \foreignlanguage{hebrew}{הִמָּצֵא} & \foreignlanguage{hebrew}{הַמְצֵא} &  \\
			& & & & & \foreignlanguage{hebrew}{הַמְצִ֫יאָה} & \\
			& & f. & \foreignlanguage{hebrew}{מִצְאִי} & \foreignlanguage{hebrew}{הִמָּֽצְאִי} & \foreignlanguage{hebrew}{הַמְצִ֫יאִי} &  \\
			& pl. & m. & \foreignlanguage{hebrew}{מִצְאוּ} & \foreignlanguage{hebrew}{הִמָּֽצְאוּ} & \foreignlanguage{hebrew}{הַמְצִ֫יאוּ} &  \\
			& & f. & \foreignlanguage{hebrew}{מְצֶאןָ} & \foreignlanguage{hebrew}{הִמָּצֶ֫אנָה} & \foreignlanguage{hebrew}{הַמְצֶ֫אנָה} &   \\
			\hline
			Inf.\ cs.\ & & & \foreignlanguage{hebrew}{מְצֹא} & \foreignlanguage{hebrew}{הִמָּצֵא} & \foreignlanguage{hebrew}{הַמְצִיא} &  \\
			Inf.\ abs.\ & & & \foreignlanguage{hebrew}{מָצוֹא} & \foreignlanguage{hebrew}{הִמָּצֵא} & \foreignlanguage{hebrew}{הַמְצֵא} &  \\
			& & & & \foreignlanguage{hebrew}{נִמְצֹא} & &  \\
			\hline
			Part. & sg. & m. & \foreignlanguage{hebrew}{מֹצֵא} & \foreignlanguage{hebrew}{נִמְצָא} & \foreignlanguage{hebrew}{מַמְצִיא} & \foreignlanguage{hebrew}{מָמְצָא} \\
			& & f. & \foreignlanguage{hebrew}{מֹצֵאת} & \foreignlanguage{hebrew}{נִמְצֵאת} & \foreignlanguage{hebrew}{} & \\
			& pl. & m. & \foreignlanguage{hebrew}{מֹצְאִים} & \foreignlanguage{hebrew}{נִמְצְאִים} & \foreignlanguage{hebrew}{} &  \\
			& & f. & \foreignlanguage{hebrew}{מֹצְאוֺת} & \foreignlanguage{hebrew}{נִמְצָאוֺת} & \foreignlanguage{hebrew}{} & \\
		\end{longtable}
	\end{center}
	
	\newpage
	
	\begin{center}
		\begin{longtable}{|lll|r|r|r|r|}
			\hline
			& & & \multicolumn{1}{c|}{Qal} & \multicolumn{1}{c|}{Piel} & \multicolumn{1}{c|}{Pual} & \multicolumn{1}{c|}{Hitpael} \\
			\hline
			\endhead
			\hline
			\endfoot
			SC & sg. & 3 m. & \foreignlanguage{hebrew}{מָצָא} & \foreignlanguage{hebrew}{מִלֵּא} & \foreignlanguage{hebrew}{מֻלָּא} & \foreignlanguage{hebrew}{הִתֽמַלֵּא} \\
			& & 3 f. & \foreignlanguage{hebrew}{מָצְאָה} & \foreignlanguage{hebrew}{מִלְאָה} & \foreignlanguage{hebrew}{מֻלְאָה} & \foreignlanguage{hebrew}{הִתְמַלְאָה} \\
			& & 2 m. & \foreignlanguage{hebrew}{מָצָאתָ} & \foreignlanguage{hebrew}{מִלֵּאתָ} & & \foreignlanguage{hebrew}{הִתְמַלֵּאתָ} \\
			& & 2 f. & \foreignlanguage{hebrew}{מָצָאת} & \foreignlanguage{hebrew}{מִלֵּאת} & & \foreignlanguage{hebrew}{הִתְמַלֵּאת} \\
			& & 1 c. & \foreignlanguage{hebrew}{מָצָאתִי} & \foreignlanguage{hebrew}{מִלֵּאתִי} & & \foreignlanguage{hebrew}{הֶתמַלֵּאתִי} \\
			\hline
			& pl. & 3 c. & \foreignlanguage{hebrew}{מָצְאוּ} & \foreignlanguage{hebrew}{מִלְאוּ} & \foreignlanguage{hebrew}{מֻלְאוּ} & \foreignlanguage{hebrew}{הִתְמַלְאוּ} \\
			& & 2 m. & \foreignlanguage{hebrew}{מְצָאתֶם} & \foreignlanguage{hebrew}{מִלֵּאתֶם} & & \foreignlanguage{hebrew}{הִתְמַלֵּאתֶם} \\
			& & 2 f. & \foreignlanguage{hebrew}{מְצָאתֶן} & \foreignlanguage{hebrew}{מִלֵּאתֶן} & & \foreignlanguage{hebrew}{הִתְמַלֵּאתֶן} \\
			& & 1 c. & \foreignlanguage{hebrew}{מָצָאנוּ} & \foreignlanguage{hebrew}{מִלֵּאנוּ} & & \foreignlanguage{hebrew}{הִתְמַלֵּאנוּ} \\
			\hline
			PC & sg. & 3 m. & \foreignlanguage{hebrew}{יִמְצָא} & \foreignlanguage{hebrew}{יְמַלֵּא} & \foreignlanguage{hebrew}{יְמֻלָּא} & \foreignlanguage{hebrew}{יִתְמַלֵּא} \\
			& & 3 f. & \foreignlanguage{hebrew}{תִּמְצָא} & \foreignlanguage{hebrew}{תְּמַלֵּא} & \foreignlanguage{hebrew}{תְּמֻלָּא} & \foreignlanguage{hebrew}{תִּתְמַלֵּא} \\
			& & 2 m. & \foreignlanguage{hebrew}{תִּמְצָא} & \foreignlanguage{hebrew}{תְּמַלֵּא} & \foreignlanguage{hebrew}{תְּמֻלָּא} & \foreignlanguage{hebrew}{תִּתְמַלֵּא} \\
			& & 2 f. & \foreignlanguage{hebrew}{תִּמְצְאִי} & \foreignlanguage{hebrew}{תְּמַלְאִי} & \foreignlanguage{hebrew}{תְּמֻלְאִי} & \foreignlanguage{hebrew}{תִּתְמַלְאִי} \\
			& & 1 c. & \foreignlanguage{hebrew}{אֶמְצָא} & \foreignlanguage{hebrew}{אֲמַלֵּא} & \foreignlanguage{hebrew}{אֲמֻלָּא} & \foreignlanguage{hebrew}{אֶתְמַלֵּא} \\
			\hline
			& pl. & 3 c. & \foreignlanguage{hebrew}{יִמְצְאוּ} & \foreignlanguage{hebrew}{יְמַלְאוּ} & \foreignlanguage{hebrew}{יְמֻלְאוּ} & \foreignlanguage{hebrew}{יִתְמַלְאוּ} \\
			& & 3 f. & \foreignlanguage{hebrew}{תִּמְצֶ֫אנָה} & \foreignlanguage{hebrew}{תְּמַלֶּאנָה} & \foreignlanguage{hebrew}{תְּמֻלֶּאנָה} & \foreignlanguage{hebrew}{תִּתְמַלֶּאנָה} \\
			& & 2 m. & \foreignlanguage{hebrew}{תִּמְצְאוּ} & \foreignlanguage{hebrew}{תְּמַלְאוּ} & \foreignlanguage{hebrew}{תְּמֻלְאוּ} & \foreignlanguage{hebrew}{תִּתְמַלְאוּ} \\
			& & 2 f. & \foreignlanguage{hebrew}{תִּמְצֶ֫אנָה} & \foreignlanguage{hebrew}{תְּמַלֶּאנָה} & \foreignlanguage{hebrew}{תְּמֻלֶּאנָה} & \foreignlanguage{hebrew}{תִּתְמַלֶּאנָה} \\
			& & 1 c. & \foreignlanguage{hebrew}{נִמְצָא} & \foreignlanguage{hebrew}{נְמַלֵּא} & \foreignlanguage{hebrew}{נְמֻלָּא} & \foreignlanguage{hebrew}{נִתְמַלֵּא} \\
			\hline
			Juss. & sg. & 3 m. & \foreignlanguage{hebrew}{יִמְצָא} & \foreignlanguage{hebrew}{יְמַלֵּא} & \foreignlanguage{hebrew}{יְמֻלָּא} & \foreignlanguage{hebrew}{יִתְמַלֵּא} \\
			\textit{waw}-PC & sg. & 3 m. & \foreignlanguage{hebrew}{וַיִּמְצָא} & \foreignlanguage{hebrew}{וַיְמַלֵּא} & \foreignlanguage{hebrew}{וַיְמֻלָּא} & \foreignlanguage{hebrew}{וַיִּתְמַלֵּא} \\
			\hline
			Impv. & sg. & m. & \foreignlanguage{hebrew}{מְצָא} & \foreignlanguage{hebrew}{מַלֵּא} & & \foreignlanguage{hebrew}{הִתְמַלֵּא} \\
			& & & & \foreignlanguage{hebrew}{מַלְאָה} & & \\
			& & f. & \foreignlanguage{hebrew}{מִצְאִי} & \foreignlanguage{hebrew}{מַלְאִי} & & \foreignlanguage{hebrew}{הִתְמַלְאִי} \\
			& pl. & m. & \foreignlanguage{hebrew}{מִצְאוּ} & \foreignlanguage{hebrew}{מַלְאוּ} & & \foreignlanguage{hebrew}{הִתְמַלְאוּ}  \\
			& & f. & \foreignlanguage{hebrew}{מְצֶאןָ} & \foreignlanguage{hebrew}{מַלֶּאנָה} & & \foreignlanguage{hebrew}{הִתְמַלֶּאנָה}  \\
			\hline
			Inf.\ cs.\ & & & \foreignlanguage{hebrew}{מְצֹא} & \foreignlanguage{hebrew}{מַלֵּא} & & \foreignlanguage{hebrew}{הִתְמַלֵּא} \\
			Inf.\ abs.\ & & & \foreignlanguage{hebrew}{מָצוֹא} & \foreignlanguage{hebrew}{מַלֵּא} & &  \\
			& & & & \foreignlanguage{hebrew}{מַלֹּא} & &  \\
			\hline
			\pagebreak
			\hline
			Part. & sg. & m. & \foreignlanguage{hebrew}{מֹצֵא} & \foreignlanguage{hebrew}{מְמַלֵּא} & \foreignlanguage{hebrew}{מְמֻלָּא} & \foreignlanguage{hebrew}{מִתְמַלֵּא}  \\
			& & f. & \foreignlanguage{hebrew}{מֹצֵאת} & & \foreignlanguage{hebrew}{} & \foreignlanguage{hebrew}{}  \\
			& pl. & m. & \foreignlanguage{hebrew}{מֹצְאִים} & \foreignlanguage{hebrew}{מְמַלְאִים} & \foreignlanguage{hebrew}{} & \foreignlanguage{hebrew}{}  \\
			& & f. & \foreignlanguage{hebrew}{מֹצְאוֺת} & \foreignlanguage{hebrew}{} & \foreignlanguage{hebrew}{} & \foreignlanguage{hebrew}{}  \\
		\end{longtable}
	\end{center}
	
	\noindent \textbf{Notes}
	\nopagebreak
	
	\noindent In the suffix conjugation of the derived binyanim, the vowel between the second root consonant and a consonantal suffix is always \textit{ṣere}, whereas the Qal has \textit{qameṣ}.
	
	The 3/2 f.\ pl.\ forms of the prefix conjugation with the vowel /ǣ/ are due to analogy to III\,\textit{y} verbs, e.g., \foreignlanguage{hebrew}{תִּמָּצֶאינָה} \textit{they will be found}. Sometimes, forms of III\,ʾ verbs follow the pattern of III\,\textit{y} completely, e.g., \foreignlanguage{hebrew}{נִבֵּיתָ} \textit{you have prophesied} (verb \foreignlanguage{hebrew}{נבא}; Jer 26:9) as opposed to the expected form \foreignlanguage{hebrew}{נִבֵּאתָ} (Jer 20:6; 28:6).
	
	\section{Temporal Subordinate Clauses}
	
	Temporal subordinate clauses locate in time a state of affair in relation to another state of affairs. The event can be either anterior, simultaneous or posterior to the other event. Frequently used temporal conjunctions are: \foreignlanguage{hebrew}{אַחֲרֵי אֲשֶׁר} \textit{after}, \foreignlanguage{hebrew}{כַּאֲשֶׁר} \textit{when, after}, \foreignlanguage{hebrew}{בְּטֶרֶם} \textit{before}, \foreignlanguage{hebrew}{עַד} \textit{until, when}, \foreignlanguage{hebrew}{עַד אֲשֶׁר} \textit{until}, \foreignlanguage{hebrew}{כִּי} \textit{when}.
	
	The temporal subordinate clause either precedes the main clause (Josh 9:16) or it follows the main clause (2\,Sam 19:31).
	
	\vspace{0.25cm}
	
	\begin{longtable}{>{\raggedleft}p{0.35\linewidth} p{0.55\linewidth}}
		\foreignlanguage{hebrew}{וַיְהִי מִקְצֵה שְׁלֹשֶׁת יָמִים אַחֲרֵי אֲשֶׁר־כָּרְתוּ לָהֶם בְּרִית וַיִּשְׁמְעוּ כִּי־קְרֹבִים הֵם אֵלָיו וּבְקִרְבּוֹ הֵם יֹשְׁבִים} & \textit{At the end of three days after they had made a covenant with them, they heard that they were from close by and that they were living among them} (Josh 9:16) \\
		\foreignlanguage{hebrew}{וַיֹּאמֶר מְפִיבֹשֶׁת אֶל־הַמֶּלֶךְ גַּם אֶת־הַכֹּל יִקָּח אַחֲרֵי אֲשֶׁר־בָּא אֲדֹנִי הַמֶּלֶך בְּשָׁלוֹם אֶל־בֵּיתוֹ} & \textit{Mephibosheth said to the king, \enquote{He may take it all after my lord the king has come to his house in peace}} (2\,Sam 19:31) \\
		\foreignlanguage{hebrew}{וְהָיָה טֶרֶם־יִקְרָ֫אוּ וַאֲנִי אֶעֱנֶה} & \textit{Before they call, I will answer} (Isa 65:24) \\
		\foreignlanguage{hebrew}{שְׁאַל מָה אֶעֱשֶׂה־לָּךְ בְּטֶרֶם אֶלָּקַח מֵעִמָּךְ} & \textit{Ask what I shall do for you, before I am taken from you} (2\,Kgs 2:9) \\
		\foreignlanguage{hebrew}{וַיְהִי כַּאֲשֶׁר קָרַב אֶל־הַמַּחֲנֶה וַיַּרְא אֶת־הָעֵגֶל וּמְחֹלֹת} & \textit{When he came near the camp, he saw the calf and the dancing} (Exod 32:19) \\
		\foreignlanguage{hebrew}{וַיֵּלְכוּ וַיָּבֹאוּ הָהָ֫רָה וַיֵּשְׁבוּ שָׁם שְׁלֹשֶׁת יָמִים עַד־שָׁבוּ הָרֹדְפִים} & \textit{They left and came to the hills and stayed there for three days, until the pursuers returned} (Josh 2:22) \\
		\foreignlanguage{hebrew}{וְלֹא־הֶאֱמַנְתִּי לַדְּבָרִים עַד אֲשֶׁר־בָּאתִי וַתִּרְאֶינָה עֵינַי} & \textit{I did not believe the words, until I came and my eyes saw it} (1\,Kgs 10:7) \\
		\foreignlanguage{hebrew}{כִּי נַעַר יִשְׂרָאֵל וָאֹהֲבֵהוּ} & \textit{When Israel was a child, I loved him} (Hos 11:1) \\
	\end{longtable}
	
	% Infrequent temporal conjunctions are \foreignlanguage{hebrew}{עַד אֲשֶׁר אִם}, \foreignlanguage{hebrew}{עַד כִּי}, \foreignlanguage{hebrew}{עַד אִם}
	
	Similar to temporal subordinate clauses with a finite verb as predicate are constructions with a preposition and an infinitive construct.
	
	\vspace{0.25cm}
	
	\begin{tabular}{>{\raggedleft}p{0.35\linewidth} p{0.55\linewidth}}
		\foreignlanguage{hebrew}{אָמוּתָה הַפָּעַם אַחֲרֵי רְאוֹתִי אֶת־פָּנֶיךָ} & \textit{I can die now after I have seen your face} (Gen 46:30) \\
		\foreignlanguage{hebrew}{וְאֵלֶּה הַמְּלָכִים אֲשֶׁר מָלְכוּ בְּאֶרֶץ אֱדוֹם לִפְנֵי מְלָךְ־מֶלֶךְ לִבְנֵי יִשְׂרָאֵל} & \textit{These are the kings that ruled in the land of Edom before a king of the Israelites ruled as king} (Gen 36:31) \\
		\foreignlanguage{hebrew}{כִּי לֹא־יֹאכַל הָעָם עַד־בֹּאוֹ} & \textit{\dots  \space for the people do not eat before he [Samuel] comes} (1\,Sam 9:13) \\
	\end{tabular}
	
	\vspace{0.5cm}
	
	For the use of the inf.\ cs.\ with the prepositions \foreignlanguage{hebrew}{כְּ} and \foreignlanguage{hebrew}{בְּ} as temporal adverbial expressions see section 9.3.3
	
	
	
	\section{Exercises}
	
	\subsection{Translation of Verbal Forms}
	
	Translate the following verbal forms. Identify the gender (masc., fem., comm.) and number (sg., pl.) of forms of which the English translation is ambiguous (i.e., \textit{you}, \textit{they}). Mark the stressed syllable if stress is not on the last syllable.
	
	\hspace{0.5cm}
	
	\selectlanguage{hebrew}
	
	\noindent
	1~~\foreignlanguage{hebrew}{כִּלָּה}  \hspace{0.3cm}
	2~~\foreignlanguage{hebrew}{כִּלּוּ}  \hspace{0.3cm}
	3~~\foreignlanguage{hebrew}{כִּלִּיתֶם}  \hspace{0.3cm}
	4~~\foreignlanguage{hebrew}{כִּלִּיתִי}  \hspace{0.3cm}
	5~~\foreignlanguage{hebrew}{אֲכַלֶּה}  \hspace{0.3cm}
	6~~\foreignlanguage{hebrew}{יְכַלּוּ}  \hspace{0.3cm}
	7~~\foreignlanguage{hebrew}{כִּסְּתָה}  \hspace{0.3cm}
	8~~\foreignlanguage{hebrew}{תְּכַסִּי}  \hspace{0.3cm}
	9~~\foreignlanguage{hebrew}{נִבָּא}  \hspace{0.3cm}
	10~~\foreignlanguage{hebrew}{נִבְּאוּ}  \hspace{0.3cm}
	11~~\foreignlanguage{hebrew}{נִבֵּאתָ}  \hspace{0.3cm}
	12~~\foreignlanguage{hebrew}{תִּנָּבֵא}  \hspace{0.3cm}
	13~~\foreignlanguage{hebrew}{הִכִּיתָ}  \hspace{0.3cm}
	14~~\foreignlanguage{hebrew}{הִכָּה}  \hspace{0.3cm}
	15~~\foreignlanguage{hebrew}{‎הִכּוּ}  \hspace{0.3cm}
	16~~\foreignlanguage{hebrew}{הִכֵּיתִי}  \hspace{0.3cm}
	17~~\foreignlanguage{hebrew}{‎הִכִּיתֶם}  \hspace{0.3cm}
	18~~\foreignlanguage{hebrew}{‎יַכֶּה}  \hspace{0.3cm}
	19~~\foreignlanguage{hebrew}{יַכּוּ}  \hspace{0.3cm}
	20~~\foreignlanguage{hebrew}{אַכֶּה}  \hspace{0.3cm}
	21~~\foreignlanguage{hebrew}{וַיַּכּוּ}  \hspace{0.3cm}
	22~~\foreignlanguage{hebrew}{וַנַּךְ}  \hspace{0.3cm}
	23~~\foreignlanguage{hebrew}{נִרְאוּ}  \hspace{0.3cm}
	24~~\foreignlanguage{hebrew}{יֵרָאוּ}  \hspace{0.3cm}
	25~~\foreignlanguage{hebrew}{הֶרְאָה}  \hspace{0.3cm}
	26~~\foreignlanguage{hebrew}{צִוִּיתִי}  \hspace{0.3cm}
	27~~\foreignlanguage{hebrew}{צִוִּיתָ}  \hspace{0.3cm}
	28~~\foreignlanguage{hebrew}{וַיְצַו}  \hspace{0.3cm}
	29~~\foreignlanguage{hebrew}{הֶעֱלוּ}  \hspace{0.3cm}
	30~~\foreignlanguage{hebrew}{הֶעֱלֵיתִי}  \hspace{0.3cm}
	\selectlanguage{english}
	
	
	\subsection{Translation of Sentences}
	
	Translate the following sentences from the Hebrew Bible. Names of persons and geographical names in these sentences: \foreignlanguage{hebrew}{אַבְרָהָם}, \foreignlanguage{hebrew}{אַהֲרֹן}, \foreignlanguage{hebrew}{אֶלְיָקִים}, \foreignlanguage{hebrew}{אָמוֺץ}, \foreignlanguage{hebrew}{דָּוִד}, \foreignlanguage{hebrew}{חִזְקִיָּהוּ}, \foreignlanguage{hebrew}{יְרוּשָׁלִַם}, \foreignlanguage{hebrew}{יְשַׁעְיָהוּ}, \foreignlanguage{hebrew}{מֹשֶׁה}, \foreignlanguage{hebrew}{עֹבֵד אֱדֹם}, \foreignlanguage{hebrew}{שָׁאוּל}, \foreignlanguage{hebrew}{שֶׁבְנָא}, \foreignlanguage{hebrew}{שְׁלֹמֹה}, \foreignlanguage{hebrew}{שְׁמוּאֵל}
	
	\vspace{0.5cm}
	
	\selectlanguage{hebrew}
	\noindent 
	1~~\foreignlanguage{hebrew}{וַיַּ֥עַשׂ מֹשֶׁ֖ה וְאַהֲרֹ֑ן כַּאֲשֶׁ֨ר צִוָּ֧ה יְהוָ֛ה אֹתָ֖ם כֵּ֥ן עָשֽׂוּ׃}  \hspace{0.3cm}
	2~~\foreignlanguage{hebrew}{וַיַּגֵּ֤ד מֹשֶׁה֙ לְאַֽהֲרֹ֔ן אֵ֛ת כָּל־דִּבְרֵ֥י יְהוָ֖ה אֲשֶׁ֣ר שְׁלָח֑וֹ וְאֵ֥ת כָּל־הָאֹתֹ֖ת אֲשֶׁ֥ר צִוָּֽהוּ׃}  \hspace{0.3cm}
	3~~\foreignlanguage{hebrew}{וַיִּ֧בֶן שְׁלֹמֹ֛ה אֶת־הַבַּ֖יִת וַיְכַלֵּֽהוּ׃}  \hspace{0.3cm}
	4~~\foreignlanguage{hebrew}{וַיְכַ֤ל אֱלֹהִים֙ בַּיּ֣וֹם הַשְּׁבִיעִ֔י מְלַאכְתּ֖וֹ אֲשֶׁ֣ר עָשָׂ֑ה וַיִּשְׁבֹּת֙ בַּיּ֣וֹם הַשְּׁבִיעִ֔י מִכָּל־מְלַאכְתּ֖וֹ אֲשֶׁ֥ר עָשָֽׂה׃}  \hspace{0.3cm}
	5~~\foreignlanguage{hebrew}{וַיִּשְׁלַ֨ח שָׁא֣וּל מַלְאָכִים֮ לָקַ֣חַת אֶת־דָּוִד֒ וַיַּ֗רְא}\LTRfootnote{\space Instead of \foreignlanguage{hebrew}{וַיַּרְא} the form \foreignlanguage{hebrew}{וַיּרְאוּ} is expected.} \foreignlanguage{hebrew}{אֶֽת־לַהֲקַ֤ת}\LTRfootnote{\space \foreignlanguage{hebrew}{לַהֲקָה} \textit{band, company}} \foreignlanguage{hebrew}{הַנְּבִיאִים֙ נִבְּאִ֔ים וּשְׁמוּאֵ֕ל עֹמֵ֥ד נִצָּ֖ב עֲלֵיהֶ֑ם וַתְּהִ֞י עַֽל־מַלְאֲכֵ֤י שָׁאוּל֙ ר֣וּחַ אֱלֹהִ֔ים וַיִּֽתְנַבְּא֖וּ גַּם־הֵֽמָּה׃} \hspace{0.3cm}
	6~~\foreignlanguage{hebrew}{וַיִּלָּחֲמ֤וּ בְנֵֽי־יְהוּדָה֙ בִּיר֣וּשָׁלִַ֔ם וַיִּלְכְּד֣וּ אוֹתָ֔הּ וַיַּכּ֖וּהָ לְפִי־חָ֑רֶב וְאֶת־הָעִ֖יר שִׁלְּח֥וּ בָאֵֽשׁ׃} \hspace{0.3cm}
	7~~\foreignlanguage{hebrew}{כָּל־הַמִּצְוָ֗ה אֲשֶׁ֨ר אָנֹכִ֧י מְצַוְּךָ֛ הַיּ֖וֹם תִּשְׁמְר֣וּן לַעֲשׂ֑וֹת לְמַ֨עַן תִּֽחְי֜וּן וּרְבִיתֶ֗ם וּבָאתֶם֙ וִֽירִשְׁתֶּ֣ם אֶת־הָאָ֔רֶץ אֲשֶׁר־נִשְׁבַּ֥ע יְהוָ֖ה לַאֲבֹתֵיכֶֽם׃}  \hspace{0.3cm}
	8~~\foreignlanguage{hebrew}{וְעַתָּ֖ה מַמְלַכְתְּךָ֣ לֹא־תָק֑וּם בִּקֵּשׁ֩ יְהוָ֨ה ל֜וֹ אִ֣ישׁ כִּלְבָב֗וֹ וַיְצַוֵּ֨הוּ יְהוָ֤ה לְנָגִיד֙ עַל־עַמּ֔וֹ כִּ֚י לֹ֣א שָׁמַ֔רְתָּ אֵ֥ת אֲשֶֽׁר־צִוְּךָ֖ יְהוָֽה׃}  \hspace{0.3cm}
	9~~\foreignlanguage{hebrew}{וַיֵּרָ֨א אֵלָ֤יו יְהוָה֙ בַּלַּ֣יְלָה הַה֔וּא וַיֹּ֕אמֶר אָנֹכִ֕י אֱלֹהֵ֖י אַבְרָהָ֣ם אָבִ֑יךָ אַל־תִּירָא֙ כִּֽי־אִתְּךָ֣ אָנֹ֔כִי וּבֵֽרַכְתִּ֙יךָ֙ וְהִרְבֵּיתִ֣י אֶֽת־זַרְעֲךָ֔ בַּעֲב֖וּר אַבְרָהָ֥ם עַבְדִּֽי׃ וַיִּ֧בֶן שָׁ֣ם מִזְבֵּ֗חַ וַיִּקְרָא֙ בְּשֵׁ֣ם יְהוָ֔ה וַיֶּט־שָׁ֖ם אָהֳל֑וֹ וַיִּכְרוּ}\LTRfootnote{\space \foreignlanguage{hebrew}{כרה} Q. \textit{to hollow out, dig}}\foreignlanguage{hebrew}{־שָׁ֥ם עַבְדֵי־יִצְחָ֖ק בְּאֵֽר׃} \hspace{0.3cm}
	10~~\foreignlanguage{hebrew}{וַיַּ֥עַל מֹשֶׁ֖ה אֶל־הָהָ֑ר וַיְכַ֥ס הֶעָנָ֖ן אֶת־הָהָֽר׃ וַיִּשְׁכֹּ֤ן כְּבוֹד־יְהוָה֙ עַל־הַ֣ר סִינַ֔י וַיְכַסֵּ֥הוּ הֶעָנָ֖ן שֵׁ֣שֶׁת יָמִ֑ים וַיִּקְרָ֧א אֶל־מֹשֶׁ֛ה בַּיּ֥וֹם הַשְּׁבִיעִ֖י מִתּ֥וֹךְ הֶעָנָֽן׃ וּמַרְאֵה֙ כְּב֣וֹד יְהוָ֔ה כְּאֵ֥שׁ אֹכֶ֖לֶת בְּרֹ֣אשׁ הָהָ֑ר לְעֵינֵ֖י בְּנֵ֥י יִשְׂרָאֵֽל׃ וַיָּבֹ֥א מֹשֶׁ֛ה בְּת֥וֹךְ הֶעָנָ֖ן וַיַּ֣עַל אֶל־הָהָ֑ר וַיְהִ֤י מֹשֶׁה֙ בָּהָ֔ר אַרְבָּעִ֣ים י֔וֹם וְאַרְבָּעִ֖ים לָֽיְלָה׃}  \hspace{0.3cm}
	11~~\foreignlanguage{hebrew}{וַיְהִ֗י כִּשְׁמֹ֙עַ֙ הַמֶּ֣לֶךְ חִזְקִיָּ֔הוּ וַיִּקְרַ֖ע אֶת־בְּגָדָ֑יו וַיִּתְכַּ֣ס בַּשָּׂ֔ק}\LTRfootnote{\space \foreignlanguage{hebrew}{שַׂק} \textit{sack, sackcloth} (gem.\ noun)} \foreignlanguage{hebrew}{וַיָּבֹ֖א בֵּ֥ית יְהוָֽה׃ וַ֠יִּשְׁלַח אֶת־אֶלְיָקִ֨ים אֲשֶׁר־עַל־הַבַּ֜יִת וְשֶׁבְנָ֣א הַסֹּפֵ֗ר וְאֵת֙ זִקְנֵ֣י הַכֹּֽהֲנִ֔ים מִתְכַּסִּ֖ים בַּשַּׂקִּ֑ים אֶל־יְשַֽׁעְיָ֥הוּ הַנָּבִ֖יא בֶּן־אָמֽוֹץ׃} \hspace{0.3cm}
	12~~\foreignlanguage{hebrew}{כִּ֣י כֹה֩ אָמַ֨ר יְהוָ֜ה אֱלֹהֵ֤י יִשְׂרָאֵל֙ אֵלַ֔י קַ֠ח אֶת־כּ֨וֹס}\LTRfootnote{\space \foreignlanguage{hebrew}{כּוֺס} \textit{cup}} \foreignlanguage{hebrew}{כִּ֣י כֹה֩ אָמַ֨ר יְהוָ֜ה אֱלֹהֵ֤י יִשְׂרָאֵל֙ אֵלַ֔י קַ֠ח אֶת־כּ֨וֹס הַיַּ֧יִן הַחֵמָ֛ה הַזֹּ֖את מִיָּדִ֑י וְהִשְׁקִיתָ֤ה אֹתוֹ֙ אֶת־כָּל־הַגּוֹיִ֔ם אֲשֶׁ֧ר אָנֹכִ֛י שֹׁלֵ֥חַ אוֹתְךָ֖ אֲלֵיהֶֽם׃  ...  וָאֶקַּ֥ח אֶת־הַכּ֖וֹס מִיַּ֣ד יְהוָ֑ה וָֽאַשְׁקֶה֙ אֶת־כָּל־הַגּוֹיִ֔ם אֲשֶׁר־שְׁלָחַ֥נִי יְהוָ֖ה אֲלֵיהֶֽם׃} \hspace{0.3cm}
	13~~\foreignlanguage{hebrew}{וַיֹּ֣אמֶר ׀ אֶל־בְּנֵ֣י יִשְׂרָאֵ֗ל כֹּֽה־אָמַ֤ר יְהוָה֙ אֱלֹהֵ֣י יִשְׂרָאֵ֔ל אָנֹכִ֛י הֶעֱלֵ֥יתִי אֶת־יִשְׂרָאֵ֖ל מִמִּצְרָ֑יִם וָאַצִּ֤יל אֶתְכֶם֙ מִיַּ֣ד מִצְרַ֔יִם וּמִיַּד֙ כָּל־הַמַּמְלָכ֔וֹת הַלֹּחֲצִ֖ים}\LTRfootnote{\space \foreignlanguage{hebrew}{לחץ} Q. \textit{to oppress}} \foreignlanguage{hebrew}{אֶתְכֶֽם׃} \hspace{0.3cm}
	14~~\foreignlanguage{hebrew}{וַיֻּגַּ֗ד לַמֶּ֣לֶךְ דָּוִד֮ לֵאמֹר֒ בֵּרַ֣ךְ יְהוָ֗ה אֶת־בֵּ֨ית עֹבֵ֤ד אֱדֹם֙ וְאֶת־כָּל־אֲשֶׁר־ל֔וֹ בַּעֲב֖וּר אֲר֣וֹן הָאֱלֹהִ֑ים וַיֵּ֣לֶךְ דָּוִ֗ד וַיַּעַל֩ אֶת־אֲר֨וֹן הָאֱלֹהִ֜ים מִבֵּ֨ית עֹבֵ֥ד אֱדֹ֛ם עִ֥יר דָּוִ֖ד בְּשִׂמְחָֽה׃}
	\selectlanguage{english}
	
	
	
	\section{Hebrew Reading: Genesis 16:7--10}
	Translate Gen 16:1--6 with the help of notes below the text.
	
	\vspace{0.5cm}
	
	\selectlanguage{hebrew}
	\noindent
	\textsuperscript{7}~\foreignlanguage{hebrew}{וַֽיִּמְצָאָ֞הּ מַלְאַ֧ךְ יְהוָ֛ה עַל־עֵ֥ין הַמַּ֖יִם בַּמִּדְבָּ֑ר עַל־הָעַ֖יִן בְּדֶ֥רֶךְ שֽׁוּר׃} \hspace{0.25cm}
	\textsuperscript{8}~\foreignlanguage{hebrew}{וַיֹּאמַ֗ר הָגָ֞ר שִׁפְחַ֥ת שָׂרַ֛י אֵֽי־מִזֶּ֥ה בָ֖את וְאָ֣נָה תֵלֵ֑כִי וַתֹּ֕אמֶר מִפְּנֵי֙ שָׂרַ֣י גְּבִרְתִּ֔י אָנֹכִ֖י בֹּרַֽחַת׃} \hspace{0.25cm}
	\textsuperscript{9}~\foreignlanguage{hebrew}{וַיֹּ֤אמֶר לָהּ֙ מַלְאַ֣ךְ יְהוָ֔ה שׁ֖וּבִי אֶל־גְּבִרְתֵּ֑ךְ וְהִתְעַנִּ֖י תַּ֥חַת יָדֶֽיהָ׃} \hspace{0.25cm}
	\textsuperscript{10}~\foreignlanguage{hebrew}{וַיֹּ֤אמֶר לָהּ֙ מַלְאַ֣ךְ יְהוָ֔ה הַרְבָּ֥ה אַרְבֶּ֖ה אֶת־זַרְעֵ֑ךְ וְלֹ֥א יִסָּפֵ֖ר מֵרֹֽב׃} \hspace{0.25cm}
	
	\selectlanguage{english}
	
	\vspace{0.25cm}
	
	\hspace*{-0.5cm}\begin{longtable}{p{0.075\linewidth} p{0.1\linewidth}p{0.725\linewidth}}
		16:7 & \foreignlanguage{hebrew}{שׁוּר} & \textit{name of a place or region near the eastern border of Egypt} \\
		16:8 & \foreignlanguage{hebrew}{אֵי־מִזֶּה} & \textit{where from?} \\
		& \foreignlanguage{hebrew}{גְּבִירָה} & \textit{mistress; lady} (title of the queen mother) (cs.\ st.\ \foreignlanguage{hebrew}{גְּבֶרֶת}) \\
		16:9 & \foreignlanguage{hebrew}{ענה} & Hitpa. \textit{to submit} (\textit{HALOT}), \textit{to humble oneself} (BDB) \\
	\end{longtable}
	
	\chapter{Chapter 23}
	
	\renewcommand\arraystretch{1.4}
	
	\section{Vocabulary}
	
	
	\subsection{Verbs}
	
	
	% For the centering of the separation between the two columns see the documentation of the array package, page 2 
	
	\begin{longtable}{>{\raggedleft}p{0.175\linewidth} p{0.75\linewidth}}
		\foreignlanguage{hebrew}{חלה} & Q.\ \textit{to be} or \textit{become weak; become sick, ill} \\ % BDB
		\foreignlanguage{hebrew}{חלק} & Q./Pi.\ \textit{to divide, apportion} \\
		\foreignlanguage{hebrew}{חרשׁ} & Hi.\ \textit{to be silent} \\
		\foreignlanguage{hebrew}{יבשׁ} & Q.\ \textit{to be dry, dried up} (stative verb; SC \foreignlanguage{hebrew}{יָבֵשׁ}; PC \foreignlanguage{hebrew}{יִיבַשׁ}); Hi.\ \textit{to dry up, make dry} \\
		\foreignlanguage{hebrew}{ידה}\textsubscript{2} & Hi.\ \textit{to praise}; Hitpa.\ \textit{to confess} (SC \foreignlanguage{hebrew}{הִתְוַדָּה}) \\
		\foreignlanguage{hebrew}{יטב} & Q.\ \textit{to be good, well, glad, pleasant}; Hi.\ \textit{to do good to, deal well with; to do well; to make something good} \\ % BDB
		\foreignlanguage{hebrew}{יכח} & Hi.\ \textit{to rebuke, reproach; to decide} \\ % HALOT
		\foreignlanguage{hebrew}{יסף} & Q.\ \textit{to add}; Hi.\ \textit{to add} \\
		\foreignlanguage{hebrew}{יצב} & Hitpa.\ \textit{to set oneself, to station oneself; to take one's stand} (cf. \foreignlanguage{hebrew}{נצב} Ni./Hi.)\\ % Only 48 occurrences according to BW10
		\foreignlanguage{hebrew}{ישׁע} & Hi.\ \textit{to help, save}; Ni.\ \textit{to receive help; to be victorious} \\ % HALOT
		\foreignlanguage{hebrew}{יתר} & Ni.\ \textit{to be left over, remain}; Hi.\ \textit{to leave over, leave} \\ % BDB
		\foreignlanguage{hebrew}{רפא} & Q.\ \textit{to heal}; Ni.\ \textit{to be healed} \\
	\end{longtable}
	
	
	\subsection{Nouns}
	
	\begin{longtable}{>{\raggedleft}p{0.175\linewidth} p{0.75\linewidth}}
		\foreignlanguage{hebrew}{גֶּבֶר} & \textit{man} \\ % BDB
		\foreignlanguage{hebrew}{זָר} & \textit{strange, foreign} \\ % BDB
		\foreignlanguage{hebrew}{כָּתֵף} & \textit{shoulder; side; mountain slope} \\ % HALOT
		\foreignlanguage{hebrew}{עֵמֶק} & \textit{valley, vale, lowland} \\ % BDB
		\foreignlanguage{hebrew}{עֲרָבָה} & \textit{desert-plain, steppe} (with the article \foreignlanguage{hebrew}{הָעֲרָבָה} usually a part of the arid region between the Sea of Galilee and the Gulf of Aqaba) \\ % BDB Action point: more precise description necessary? See Ges18
		\foreignlanguage{hebrew}{שְׁאוֺל} & \textit{underworld} (fem.) \\ % BDB
	\end{longtable}
	
	
	
	
	\section{Verbs I \textit{y}: Derived Binyanim}
	
	\subsection{Verbs I \textit{y}: Niphal, Hiphil and Hophal}
	
	Verbs of the type I\,\textit{y} are verbs with originally either /y/ or /w/ as first root consonant. In the Niphal, Hiphil and Hophal most I\,\textit{y} verbs behave like having /w/ as first root consonant. Only few exceptions to this are found in Biblical Hebrew. As the weakness of I\,\textit{y} verbs is in first position, the vowels of the prefix syllables in the Niphal, Hiphil and Hophal interact with the first root consonant in the following ways:
	
	\begin{enumerate}[noitemsep]
		\item In the suffix conjugation and participle Niphal: \textit{*nawšab > nōšaḇ} \foreignlanguage{hebrew}{נוֺשַׁב} in the suffix conjugation (from a primitive form \textit{*naqṭala}) and \textit{*nawšab > nōšāḇ} \foreignlanguage{hebrew}{נוֺשָׁב} in the participle
		\item In the Hiphil of most verbs: \textit{*hawšaba > hōšīḇ} \foreignlanguage{hebrew}{הוֺשִׁיב} (SC) (from a primitive form \textit{*haqṭala}), \textit{*yuhawšib > yōšīḇ} \foreignlanguage{hebrew}{יוֺשִׁיב} (PC) (the same applies to the other Hi.\ forms)
		\item In the Hiphil of a few verbs: \textit{*hayṭaba > hēṭīḇ} \foreignlanguage{hebrew}{הֵיטִיב} (SC) (from a primitive form \textit{*haqṭala}), \textit{*yuhayṭib > yēṭīḇ} \foreignlanguage{hebrew}{יֵיטִיב} (PC) (the same applies to the other Hi.\ forms)
		\item In the Hophal: \textit{*huwrad > hūrad} \foreignlanguage{hebrew}{הוּרַד} (SC), \textit{*yuwrad > yūrad} \foreignlanguage{hebrew}{יוּרַד} (PC) (the same applies to the other Ho.\ forms)
	\end{enumerate}
	
	
	\begin{center}
		\begin{longtable}{|lll|r|r|r|r|}
			\hline
			& & & \multicolumn{1}{c|}{Niphal} & \multicolumn{2}{c|}{Hiphil} & \multicolumn{1}{c|}{Hophal} \\
			\hline
			\endhead
			\hline
			\endfoot
			SC & sg. & 3 m. & \foreignlanguage{hebrew}{נוֺשַׁב} & \foreignlanguage{hebrew}{הוֹשִׁיב} & \foreignlanguage{hebrew}{הֵיטִיב} & \foreignlanguage{hebrew}{הוּשַׁב} \\
			& & 3 f. & \foreignlanguage{hebrew}{נוֺשְׁבָה} & \foreignlanguage{hebrew}{הוֺשִׁ֫יבָה} & \foreignlanguage{hebrew}{הֵיטִ֫יבָה} & \foreignlanguage{hebrew}{} \\
			& & 2 m. & \foreignlanguage{hebrew}{נוֺשַׁ֫בְתָּ} & \foreignlanguage{hebrew}{הוֺשַׁ֫בְתָּ} & \foreignlanguage{hebrew}{הֵיטַַ֫בְתָּ} & \foreignlanguage{hebrew}{} \\
			& & 2 f. & \foreignlanguage{hebrew}{נוֺשַׁבְתְּ} & \foreignlanguage{hebrew}{הוֺשַׁבְתְּ} & \foreignlanguage{hebrew}{הֵיטַ֫בְתְּ} & \foreignlanguage{hebrew}{} \\
			& & 1 c. & \foreignlanguage{hebrew}{נוֺשַׁ֫בְתִּי} & \foreignlanguage{hebrew}{הוֺשַׁ֫בְתִּי} & \foreignlanguage{hebrew}{הֵיטַ֫בְתִּי} & \foreignlanguage{hebrew}{} \\
			\hline
			& pl. & 3 c. & \foreignlanguage{hebrew}{נוֺשְׁבוּ} & \foreignlanguage{hebrew}{הוֺשִׁ֫יבוּ} & \foreignlanguage{hebrew}{הֵיטִ֫יבוּ} & \foreignlanguage{hebrew}{} \\
			& & 2 m. & \foreignlanguage{hebrew}{נוֹשַׁבְתֶּם} & \foreignlanguage{hebrew}{הוֹשַׁבְתֶּם} & \foreignlanguage{hebrew}{הֵיטַבְתֶּם} & \foreignlanguage{hebrew}{} \\
			& & 2 f. & \foreignlanguage{hebrew}{נוֺשַׁבְתֶּן} & \foreignlanguage{hebrew}{הוֺשַׁבְתֶּן} & \foreignlanguage{hebrew}{הֵיטַבְתֶּן} & \foreignlanguage{hebrew}{} \\
			& & 1 c. & \foreignlanguage{hebrew}{נוֺשַׁ֫בְנוּ} & \foreignlanguage{hebrew}{הוֹשַׁ֫בְנוּ} & \foreignlanguage{hebrew}{הֵיטַ֫בְנוּ} & \foreignlanguage{hebrew}{} \\
			\hline
			PC & sg. & 3 m. & \foreignlanguage{hebrew}{יִוָּשֵׁב} & \foreignlanguage{hebrew}{יוֺשִׁיב} & \foreignlanguage{hebrew}{יֵיטִיב} & \foreignlanguage{hebrew}{יוּשַׁב} \\
			& & 3 f. & \foreignlanguage{hebrew}{תִּוָּשֵׁב} & \foreignlanguage{hebrew}{תּוֺשִׁיב} & \foreignlanguage{hebrew}{תֵּיטִיב} & \foreignlanguage{hebrew}{} \\
			& & 2 m. & \foreignlanguage{hebrew}{תִּוָּשֵׁב} & \foreignlanguage{hebrew}{תּוֺשִׁיב} & \foreignlanguage{hebrew}{תֵּיטִיב} & \foreignlanguage{hebrew}{} \\
			& & 2 f. & \foreignlanguage{hebrew}{תִּוָּֽשְׁבִי} & \foreignlanguage{hebrew}{תּוֺשִׁ֫יבִי} & \foreignlanguage{hebrew}{תֵּיטִ֫יבִי} & \foreignlanguage{hebrew}{} \\
			& & 1 c. & \foreignlanguage{hebrew}{אִוָּשֵׁב} & \foreignlanguage{hebrew}{אוֺשִׁיב} & \foreignlanguage{hebrew}{אֵיטִיב} & \foreignlanguage{hebrew}{} \\
			\pagebreak
			\hline
			& pl. & 3 m. & \foreignlanguage{hebrew}{יִוָּֽשְׁבוּ} & \foreignlanguage{hebrew}{יוֺשִׁ֫יבוּ} & \foreignlanguage{hebrew}{יֵיטִי֫בוּ} & \foreignlanguage{hebrew}{} \\
			& & 3 f. & \foreignlanguage{hebrew}{תִּוָּשַׁ֫בְנָה} & \foreignlanguage{hebrew}{תּוֺשֵׁ֫בְנָה} & \foreignlanguage{hebrew}{תֵּיטֵ֫בְנָה} & \foreignlanguage{hebrew}{} \\
			& & 2 m. & \foreignlanguage{hebrew}{תִּוָּֽשְׁבוּ}  & \foreignlanguage{hebrew}{תּוֺשִׁ֫יבוּ} & \foreignlanguage{hebrew}{תֵּיטִ֫יבוּ} & \foreignlanguage{hebrew}{} \\
			& & 2 f. & \foreignlanguage{hebrew}{תִּוָּשַׁ֫בְנָה} & \foreignlanguage{hebrew}{תּוֺשֵׁ֫בְנָה} & \foreignlanguage{hebrew}{תֵּיטֵ֫בְנָה} & \foreignlanguage{hebrew}{} \\
			& & 1 c. & \foreignlanguage{hebrew}{נִוָּשֵׁב} & \foreignlanguage{hebrew}{נוֺשִׁיב} & \foreignlanguage{hebrew}{נֵיטִיב} & \foreignlanguage{hebrew}{} \\
			\hline
			Juss. & sg. & 3 m. & \foreignlanguage{hebrew}{יִוָּשֵׁב} & \foreignlanguage{hebrew}{יוֺשֵׁב} & & \\
			\textit{waw}-PC & sg. & 3 m. & \foreignlanguage{hebrew}{וַיִּוָּלֵד} & \foreignlanguage{hebrew}{וַיּוֺ֫שֶׁב} & \foreignlanguage{hebrew}{‎וַיֵּ֫יטֶב} & \\
			\hline
			Impv. & sg. & m. & \foreignlanguage{hebrew}{הִוָּשֵׁב} & \foreignlanguage{hebrew}{הוֺשֵׁב} & \foreignlanguage{hebrew}{} &  \\
			& & & & \foreignlanguage{hebrew}{הוֺשִׁ֫יבָה} & \foreignlanguage{hebrew}{הֵיטִ֫יבָה} &  \\
			& & f. & \foreignlanguage{hebrew}{הִוָּֽשְׁבִי} & \foreignlanguage{hebrew}{הוֺשִׁ֫יבִי} & \foreignlanguage{hebrew}{} &  \\
			& pl. & m. & \foreignlanguage{hebrew}{הִוָּֽשְׁבוּ} & \foreignlanguage{hebrew}{הוֺשִׁ֫יבוּ} & \foreignlanguage{hebrew}{} &  \\
			\hline
			Inf.\ cs.\ & & & \foreignlanguage{hebrew}{הִוָּשֵׁב} & \foreignlanguage{hebrew}{הוֺשִׁיב} & \foreignlanguage{hebrew}{הֵיטִיב} & \foreignlanguage{hebrew}{הוּשַׁב} \\
			Inf.\ abs.\ & & & & \foreignlanguage{hebrew}{הוֺשֵׁב} & \foreignlanguage{hebrew}{הֵיטֵב} & \\
			\hline
			Part. & sg. & m. & \foreignlanguage{hebrew}{נוֺשָׁב} & \foreignlanguage{hebrew}{מוֺשִׁיב} & \foreignlanguage{hebrew}{מֵיטִיב} & \foreignlanguage{hebrew}{מוּשָׁב} \\
			& & f. & \foreignlanguage{hebrew}{נוֺשֶׁ֫בֶת} & \foreignlanguage{hebrew}{מוֺשֶׁ֫בֶת} & \foreignlanguage{hebrew}{מֵיטִיבָה} & \foreignlanguage{hebrew}{} \\
			& pl. & m. & \foreignlanguage{hebrew}{נוֺשָׁבִים} & \foreignlanguage{hebrew}{מוֺשִׁיבִים} & \foreignlanguage{hebrew}{מֵיטִיבִים} & \foreignlanguage{hebrew}{} \\
			& & f. & \foreignlanguage{hebrew}{נוֹשָׁבוֺת} & \foreignlanguage{hebrew}{מוֺשִׁיבוֺת} & \foreignlanguage{hebrew}{מֵיטִיבוֺת} & \foreignlanguage{hebrew}{} \\
		\end{longtable}
	\end{center}
	
	\vspace{0.5cm}
	
	\noindent \textbf{Notes}
	\nopagebreak
	
	\noindent The primitive forms for the suffix conjugation \textit{*naqṭal} (Ni.) and \textit{*haqṭal} (Hi.) are different from the primitive forms for the corresponding forms of the strong verbs \textit{*niqṭal} and \textit{*hiqṭil}, respectively (cf. Chapters 16 and 17).
	
	In the Niphal prefix conjugation, imperative and infinitives the first root consonant is /w/ in most cases. The /n/ of the Niphal is assimilated to the /w/, e.g., \textit{yinwašiʿ > yiwwāšēăʿ} \foreignlanguage{hebrew}{יִוָּשֵׁ֑עַ} \textit{he will be saved} (pausal form; Prov 28:18). At times, the first root consonant /y/ is preserved, e.g., \foreignlanguage{hebrew}{יִיָּרֶה} \textit{he shall be shot} (Exod 19:13).
	
	Five I\,\textit{y} verbs have Hiphil forms that are based on an original /y/ as R\textsubscript{1}: \foreignlanguage{hebrew}{יטב} Hi.\ \textit{to do good} (74×), \foreignlanguage{hebrew}{ילל} Hi.\ \textit{to howl} (30×), \foreignlanguage{hebrew}{ינק} Hi.\ \textit{to nurse, suckle} (9×), \foreignlanguage{hebrew}{ימן} Hi.\ \textit{to go to the right} (5×), \foreignlanguage{hebrew}{ישׁר} Hi.\ \textit{to make even} (2×).
	
	In \textit{wayyiqṭol} forms the long vowel /ī/ as thematic vowel is replaced by /ē/ (except in pausal forms), e.g., \foreignlanguage{hebrew}{וַיּוֺשֵׁב} \textit{he settled} (transitive; Gen 47:11).
	
	In \textit{wayyiqṭol} verbs without suffix, stress is frequently retracted to the penultimate syllable with shortening of the thematic vowel /ē/ > /æ/, e.g., \foreignlanguage{hebrew}{וַיֹּ֫שֶׁב}, \foreignlanguage{hebrew}{וַתֵּ֫יטֶב}. But forms without retraction of stress are attested, too, e.g., \foreignlanguage{hebrew}{וַיּוֺשֵׁב} (Gen 47:11). With a guttural as third root consonant the vowel /ē/ is shortened to /a/, e.g., \foreignlanguage{hebrew}{וַיֹּ֫שַׁע} \textit{and he saved} (Exod 14:30).
	
	The verb \foreignlanguage{hebrew}{הלך} follows the pattern of the verbs \foreignlanguage{hebrew}{ישׁב}, \foreignlanguage{hebrew}{ירד}, etc., in the Hiphil forms. Illustrative Forms are the following:
	
	\vspace{0.25cm}
	
	\begin{center}
		\begin{tabular}{lllr}
			SC & sg. & 3 m. & \foreignlanguage{hebrew}{הוֺלִיךְ} \\
			& pl. & 3 c. & \foreignlanguage{hebrew}{הוֺלִ֫יכוּ} \\
			PC & sg. & 3 m. & \foreignlanguage{hebrew}{יוֺלִיךְ} \\
			& pl. & 3 m. & \foreignlanguage{hebrew}{יוֺלִ֫יכוּ}  \\
			Inf.\ cs.\ & & & \foreignlanguage{hebrew}{הוֺלִיךְ} \\
		\end{tabular}
	\end{center}
	
	\vspace{0.25cm} 
	
	Frequent doubly weak verbs are \foreignlanguage{hebrew}{יצא} Hi.\ \textit{to bring out, lead out}, \foreignlanguage{hebrew}{ידה} Hi.\ \textit{to praise} and \foreignlanguage{hebrew}{ירא} Ni.\ \textit{to be feared, terrible, awesome}. The verb \foreignlanguage{hebrew}{ירא} Ni.\ is almost exclusively attested in the participle \foreignlanguage{hebrew}{נוֺרָא}, \foreignlanguage{hebrew}{נוֺרָאָה} etc. The verbs \foreignlanguage{hebrew}{יצא} Hi.\ and \foreignlanguage{hebrew}{ידה} Hi.\ have the following characteristic forms:
	
	\vspace{0.25cm}
	
	\begin{center}
		\begin{tabular}{lllrr}
			SC & sg. & 3 m. & \foreignlanguage{hebrew}{} & \foreignlanguage{hebrew}{הוֺצִיא} \\
			& pl. & 3 c. & \foreignlanguage{hebrew}{הוֺדוּ} & \foreignlanguage{hebrew}{הוֺצִיאוּ} \\
			PC & sg. & 3 m. & \foreignlanguage{hebrew}{יוֺדֶה} & \foreignlanguage{hebrew}{יוֺצִיא} \\
			& pl. & 3 m. & \foreignlanguage{hebrew}{יוֺדוּ} & \foreignlanguage{hebrew}{יוֺצִ֫יאוּ} \\
			Inf.\ cs.\ & & & \foreignlanguage{hebrew}{הֹדוֺת} & \foreignlanguage{hebrew}{הוֺצִיא} \\
		\end{tabular}
	\end{center}
	
	\vspace{0.25cm}
	
	The verb \foreignlanguage{hebrew}{יסף} \textit{to add} is used in the Qal only in the suffix conjugation (28×), the inf.\ cs.\ (only 1×) and the part.\ (only 1×). In the Hiphil it is used in all forms (except the inf.\ abs.; 171 times in total), e.g., \foreignlanguage{hebrew}{‎הוֹסַפְתָּ} (SC), \foreignlanguage{hebrew}{‎יוֹסִיף} (PC), \foreignlanguage{hebrew}{‎וַיּ֫וֹסֶף} (\textit{wayyiqṭol}).
	
	Verbs of the type I\,\textit{y} with \foreignlanguage{hebrew}{צ} as second root consonant have forms with a geminated \foreignlanguage{hebrew}{צ} in the Hi.\ and Ho., e.g., \foreignlanguage{hebrew}{וַיַּצִּיגוּ} \textit{and they placed} (1\,Sam 5:2), \foreignlanguage{hebrew}{וַיַּצִּקֻם} \textit{and they poured them out} (with ePP; Josh 7:23), \foreignlanguage{hebrew}{תַּצִּיתוּ} \textit{you shall set on fire} (Josh 8:8). Although forms like these look like forms of I\,\textit{n} verbs, they are derived from the verbs \foreignlanguage{hebrew}{יצג}, \foreignlanguage{hebrew}{יצק} and \foreignlanguage{hebrew}{יצת}, respectively.
	
	The verb \foreignlanguage{hebrew}{ילד} has inf.\ cs.\ forms in the Hophal with a geminated second root consonant instead of a long vowel /ū/ in the prefix syllable: \foreignlanguage{hebrew}{הֻלֶּדֶת} or \foreignlanguage{hebrew}{הוּלֶּדֶת} \textit{hullædæt} in the constructions \foreignlanguage{hebrew}{יוֹם הֻלֶּדֶת} \textit{birthday} (Gen 40:20) or \foreignlanguage{hebrew}{יוֹם הוּלֶּדֶת} \textit{day of birth} (Ezek 16:4).
	
	\subsection{Verbs I \textit{y}: Piel, Pual and Hitpael}
	
	In the Piel, Pual and Hitpael I\,\textit{y} have strong forms in the entire paradigms. In the Piel and Pual the first root consonant /y/ is preserved. In the Hitpael of some verbs the first root consonant is changed to /w/ while in the majority of verbs the /y/ is preserved. Important forms are included in the following table (only attested forms are included).
	
	\vspace{0.5cm}
	
	\begin{center}
		\begin{tabular}{lllrrrr}
			\multicolumn{3}{c}{} & \multicolumn{1}{c}{Piel} & \multicolumn{1}{c}{Pual} & \multicolumn{2}{c}{Hitpael} \\
			SC & sg. & 3 m. & \foreignlanguage{hebrew}{יִסַּר} & \foreignlanguage{hebrew}{יֻסַּד} & \foreignlanguage{hebrew}{הִתְיַצֵּב} & \foreignlanguage{hebrew}{הִתְוַדָּה} \\
			& pl. & 3 m. & \foreignlanguage{hebrew}{יִסְּרוּ} & \foreignlanguage{hebrew}{} & \foreignlanguage{hebrew}{\foreignlanguage{hebrew}{הִתְיַצְּבוּ}} & \foreignlanguage{hebrew}{הִתְוַדּוּ} \\
			PC & sg. & 3 m. & \foreignlanguage{hebrew}{יְיַסֵּר} & \foreignlanguage{hebrew}{} & \foreignlanguage{hebrew}{יִתְיַצֵּב} &  \\
			& pl. & 3 m. & \foreignlanguage{hebrew}{} & \foreignlanguage{hebrew}{} & \foreignlanguage{hebrew}{יִתְיַצְּבוּ} & \foreignlanguage{hebrew}{וַיִּתְוַדּוּ} \\
			Inf.\ cs. & & & \foreignlanguage{hebrew}{יַסֵּד} & \foreignlanguage{hebrew}{} & \foreignlanguage{hebrew}{הִתְיַצֵּב} & \foreignlanguage{hebrew}{הִתְוַדֹּת} \\
			Part. & sg. & m. & \foreignlanguage{hebrew}{מְיַחֵל} & \foreignlanguage{hebrew}{מְיֻסָּד} & \foreignlanguage{hebrew}{} & \foreignlanguage{hebrew}{מִתְוַדֶּה} \\
		\end{tabular}
	\end{center} 
	
	
	\section{Modifying verbs}
	
	Some verbs may be used together with another verb as modifying verbs. When used in this way, they function as adverbial modifiers, i.e., the modifying verb expresses only an adverbial idea whereas the other verb expresses the main verbal idea. Verbs that are frequently used as modifying verbs are: 
	
	\vspace{0.5cm}
	
	
	\begin{tabular}{>{\raggedleft}p{0.15\linewidth} p{0.25\linewidth} p{0.5\linewidth}}
		& Normal meaning & Meaning as modifying verb \\
		\foreignlanguage{hebrew}{יסף} Q./Hi. & \textit{to add} & \textit{to do something again, continue doing something}; with a negative: \textit{to stop doing something; to not do something anymore} \\
		\foreignlanguage{hebrew}{שׁוב} Q. & \textit{to return} & \textit{to do something again} \\
		\foreignlanguage{hebrew}{מהר} Pi. & \textit{to hurry} & \textit{to do something quickly} \\
		\foreignlanguage{hebrew}{רבה} Hi. & \textit{to multiply, increase} & \textit{to do something many times} \\
		\foreignlanguage{hebrew}{שׁכם} Hi. & \textit{to rise early} & \textit{to do something early (in the morning)} \\
	\end{tabular}
	
	\vspace{0.5cm}
	
	In many cases, the modifying verb precedes the modified verb which is in the infinitive construct, usually with the preposition \foreignlanguage{hebrew}{לְ} (1\,Sam 7:13) but also without (Gen 8:12).
	
	\vspace{0.5cm}
	
	\begin{longtable}{>{\raggedleft}p{0.35\linewidth} p{0.55\linewidth}}
		\foreignlanguage{hebrew}{וַיִּכָּנְעוּ הַפְּלִשְׁתִּים וְלֹא־יָסְפוּ עוֹד לָבוֹא בִּגְבוּל יִשְׂרָאֵל} & \textit{And the Philistines were subdued and did not come into the territory of Israel again} (1\,Sam 7:13) \\
		\foreignlanguage{hebrew}{וְלֹא־יָסְפָה שׁוּב־אֵלָיו עוֹד} & \textit{[The dove] did not return to him anymore} (Gen 8:12) \\
		\foreignlanguage{hebrew}{לֹא־יוּכַל בַּעְלָהּ הָרִאשׁוֹן אֲשֶֽר־שִׁלְּחָהּ לָשׁוּב לְקַחְתָּהּ לִהְיוֹת לוֹ לְאִשָּׁה} & \textit{Her first husband, who sent her away, cannot take her back again to be his wife} (Deut 24:4) \\
		\foreignlanguage{hebrew}{וַיֹּאמֶר יִצְחָק אֶל־בְּנוֹ מַה־זֶּה מִהַרְתָּ לִמְצֹא בְּנִי} & \textit{And Isaac said to his son, \enquote{How did you find it so quickly, my son?}} (Gen 27:20) \\
		\foreignlanguage{hebrew}{וַיַּשְׁכֵּם מְשָׁרֵת אִישׁ הָאֱלֹהִים לָקוּם} & \textit{And the servant of the man of God got up early in the morning} (2\,Kgs 6:15) \\
	\end{longtable}
	
	\vspace{0.5cm}
	
	Instead of a verbal form and an infinitive construct two finite verbal forms may be joined together, either in a syndetic construction with the conjunction \foreignlanguage{hebrew}{וְ} (2\,Sam 18:22; 1\,Sam 25:23) or in an asyndetic construction without \foreignlanguage{hebrew}{וְ} (Isa 52:1). The first verb is used as the modifying verb.
	
	\vspace{0.5cm}
	
	\begin{tabular}{>{\raggedleft}p{0.35\linewidth} p{0.55\linewidth}}
		\foreignlanguage{hebrew}{וַיֹּסֶף עוֹד אֲחִימַעַץ בֶּן־צָדוֹק וַיֹּאמֶר אֶל־יוֹאָב ...} & \textit{And Ahimaaz son of Zadok said again to Joab} (2\,Sam 18:22) \\
		\foreignlanguage{hebrew}{וַתֵּרֶא אֲבִיגַיִל אֶת־דָּוִד וַתְּמַהֵר וַתֵּרֶד מֵעַל הַחֲמוֹר} & \textit{And Abigail saw David and quickly got off the donkey} (1\,Sam 25:23) \\
		\foreignlanguage{hebrew}{... כִּי לֹא יוֹסִיף יָבֹא־בָךְ עוֹד עָרֵל וְטָמֵא} & \textit{\dots \space for no uncircumcised or unclean will come into you anymore} (Isa 52:1) \\
	\end{tabular}
	
	
	\section{Exercises}
	
	\subsection{Translation of Verbal Forms}
	
	Translate the following verbal forms. Identify the gender (masc., fem., comm.) and number (sg., pl.) of forms of which the English translation is ambiguous (i.e., \textit{you}, \textit{they}). Mark the stressed syllable if stress is not on the last syllable.
	
	\hspace{0.5cm}
	
	\selectlanguage{hebrew}
	
	\noindent
	1~~\foreignlanguage{hebrew}{אוֹדֶה}  \hspace{0.3cm}
	2~~\foreignlanguage{hebrew}{נוֹדֶה}  \hspace{0.3cm}
	3~~\foreignlanguage{hebrew}{הִתְוַדּוּ}  \hspace{0.3cm}
	4~~\foreignlanguage{hebrew}{הֵיטִיבוּ}  \hspace{0.3cm}
	5~~\foreignlanguage{hebrew}{הֵיטַבְתָּ}  \hspace{0.3cm}
	6~~\foreignlanguage{hebrew}{יֵיטִיב}  \hspace{0.3cm}
	7~~\foreignlanguage{hebrew}{הוֹסַפְתִּי}  \hspace{0.3cm}
	8~~\foreignlanguage{hebrew}{תֹסִיפוּ}  \hspace{0.3cm}
	9~~\foreignlanguage{hebrew}{יוֹסִפוּ}  \hspace{0.3cm}
	10~~\foreignlanguage{hebrew}{הִתְיַצְּבוּ}  \hspace{0.3cm}
	11~~\foreignlanguage{hebrew}{יִתְיַצֵּב}  \hspace{0.3cm}
	12~~\foreignlanguage{hebrew}{הוֹשַׁעְתֶּם}  \hspace{0.3cm}
	13~~\foreignlanguage{hebrew}{הוֹשַׁעְתִּי}  \hspace{0.3cm}
	14~~\foreignlanguage{hebrew}{תּוֹשִׁיעַ}  \hspace{0.3cm}
	15~~\foreignlanguage{hebrew}{‎הִכּוּ}  \hspace{0.3cm}
	16~~\foreignlanguage{hebrew}{וָאוֹלֵךְ}  \hspace{0.3cm}
	17~~\foreignlanguage{hebrew}{‎וַיֹּלִכוּ}  \hspace{0.3cm}
	18~~\foreignlanguage{hebrew}{‎תַּחֲרִישִׁי}  \hspace{0.3cm}
	\selectlanguage{english}
	
	
	\subsection{Translation of Sentences}
	
	Translate the following sentences from the Hebrew Bible. Names of persons and geographical names in these sentences: \foreignlanguage{hebrew}{אַבְשָׁלוֺם}, \foreignlanguage{hebrew}{אַהֲרֹן}, \foreignlanguage{hebrew}{אֵלִיָּהוּ}, \foreignlanguage{hebrew}{בָּבֶל}, \foreignlanguage{hebrew}{דָּוִד}, \foreignlanguage{hebrew}{יְהוּדָה},, \foreignlanguage{hebrew}{יְהוֺשֻׁעַ}, \foreignlanguage{hebrew}{מִצְרַיִם}, \foreignlanguage{hebrew}{צִדְקִיָּהוּ}, \foreignlanguage{hebrew}{מִצְרַיִם}, \foreignlanguage{hebrew}{שָׁאוּל}, \foreignlanguage{hebrew}{שְׁמוּאֵל}
	
	\vspace{0.5cm}
	
	\selectlanguage{hebrew}
	\noindent 
	1~~\foreignlanguage{hebrew}{רְפָאֵ֤נִי יְהוָה֙ וְאֵ֣רָפֵ֔א הוֹשִׁיעֵ֖נִי וְאִוָּשֵׁ֑עָה כִּ֥י תְהִלָּתִ֖י אָֽתָּה׃}  \hspace{0.3cm}
	2~~\foreignlanguage{hebrew}{הִנֵּ֛ה שֶׁ֥בַע שָׁנִ֖ים בָּא֑וֹת שָׂבָ֥ע}\LTRfootnote{\space \foreignlanguage{hebrew}{שָׂבָע} \textit{satiation, sufficiency; plenty}} \foreignlanguage{hebrew}{גָּד֖וֹל בְּכָל־אֶ֥רֶץ מִצְרָֽיִם׃ וְ֠קָמוּ שֶׁ֜בַע שְׁנֵ֤י רָעָב֙ אַחֲרֵיהֶ֔ן וְנִשְׁכַּ֥ח כָּל־הַשָּׂבָ֖ע בְּאֶ֣רֶץ מִצְרָ֑יִם וְכִלָּ֥ה הָרָעָ֖ב אֶת־הָאָֽרֶץ׃ וְלֹֽא־יִוָּדַ֤ע הַשָּׂבָע֙ בָּאָ֔רֶץ מִפְּנֵ֛י הָרָעָ֥ב הַה֖וּא אַחֲרֵי־כֵ֑ן כִּֽי־כָבֵ֥ד ה֖וּא מְאֹֽד׃} \hspace{0.3cm}
	3~~\foreignlanguage{hebrew}{וַיֵּצֵא֙ בַּיּ֣וֹם הַשֵּׁנִ֔י וְהִנֵּ֛ה שְׁנֵֽי־אֲנָשִׁ֥ים עִבְרִ֖ים}\LTRfootnote{\space \foreignlanguage{hebrew}{עִבְרִי} \textit{Hebrew}}  \foreignlanguage{hebrew}{נִצִּ֑ים}\LTRfootnote{\space \foreignlanguage{hebrew}{נצה} Ni.\ \textit{to fight}} \foreignlanguage{hebrew}{וַיֹּ֙אמֶר֙ לָֽרָשָׁ֔ע לָ֥מָּה תַכֶּ֖ה רֵעֶֽךָ׃ וַ֠יֹּאמֶר מִ֣י שָֽׂמְךָ֞ לְאִ֨ישׁ שַׂ֤ר וְשֹׁפֵט֙ עָלֵ֔ינוּ הַלְהָרְגֵ֙נִי֙ אַתָּ֣ה אֹמֵ֔ר כַּאֲשֶׁ֥ר הָרַ֖גְתָּ אֶת־הַמִּצְרִ֑י}\LTRfootnote{\space \foreignlanguage{hebrew}{מִצְרִי} \textit{Egyptian}} \foreignlanguage{hebrew}{וַיִּירָ֤א מֹשֶׁה֙ וַיֹּאמַ֔ר אָכֵ֖ן}\LTRfootnote{\space \foreignlanguage{hebrew}{אָכֵן} \textit{surely}} \foreignlanguage{hebrew}{נוֹדַ֥ע הַדָּבָֽר׃} \hspace{0.3cm}
	4~~\foreignlanguage{hebrew}{ וְעַתָּ֣ה לְכָ֔ה וְאֶֽשְׁלָחֲךָ֖ אֶל־פַּרְעֹ֑ה וְהוֹצֵ֛א אֶת־עַמִּ֥י בְנֵֽי־יִשְׂרָאֵ֖ל מִמִּצְרָֽיִם׃ וַיֹּ֤אמֶר מֹשֶׁה֙ אֶל־הָ֣אֱלֹהִ֔ים מִ֣י אָנֹ֔כִי כִּ֥י אֵלֵ֖ךְ אֶל־פַּרְעֹ֑ה וְכִ֥י אוֹצִ֛יא אֶת־בְּנֵ֥י יִשְׂרָאֵ֖ל מִמִּצְרָֽיִם׃ וַיֹּ֙אמֶר֙ כִּֽי־אֶֽהְיֶ֣ה עִמָּ֔ךְ וְזֶה־לְּךָ֣ הָא֔וֹת כִּ֥י אָנֹכִ֖י שְׁלַחְתִּ֑יךָ בְּהוֹצִֽיאֲךָ֤ אֶת־הָעָם֙ מִמִּצְרַ֔יִם תַּֽעַבְדוּן֙ אֶת־הָ֣אֱלֹהִ֔ים עַ֖ל הָהָ֥ר הַזֶּֽה׃}  \hspace{0.3cm}
	5~~\foreignlanguage{hebrew}{וַיְדַבֵּ֣ר יְהוָה֮ אֶל־מֹשֶׁ֣ה וְאֶֽל־אַהֲרֹן֒ וַיְצַוֵּם֙ אֶל־בְּנֵ֣י יִשְׂרָאֵ֔ל וְאֶל־פַּרְעֹ֖ה מֶ֣לֶךְ מִצְרָ֑יִם לְהוֹצִ֥יא אֶת־בְּנֵֽי־יִשְׂרָאֵ֖ל מֵאֶ֥רֶץ מִצְרָֽיִם׃}  \hspace{0.3cm}
	6~~\foreignlanguage{hebrew}{וַיְדַבֵּ֣ר אֱלֹהִ֔ים אֵ֛ת כָּל־הַדְּבָרִ֥ים הָאֵ֖לֶּה לֵאמֹֽר׃ אָֽנֹכִ֖י֙ יְהוָ֣ה אֱלֹהֶ֑֔יךָ אֲשֶׁ֧ר הוֹצֵאתִ֛יךָ מֵאֶ֥רֶץ מִצְרַ֖יִם מִבֵּ֣֥ית עֲבָדִ֑͏ֽים׃}  \hspace{0.3cm}
	7~~\foreignlanguage{hebrew}{וַיִּשְׁלַ֧ח יְהוָ֛ה אִ֥ישׁ נָבִ֖יא אֶל־בְּנֵ֣י יִשְׂרָאֵ֑ל וַיֹּ֨אמֶר לָהֶ֜ם כֹּה־אָמַ֥ר יְהוָ֣ה ׀ אֱלֹהֵ֣י יִשְׂרָאֵ֗ל אָנֹכִ֞י הֶעֱלֵ֤יתִי אֶתְכֶם֙ מִמִּצְרַ֔יִם וָאֹצִ֥יא אֶתְכֶ֖ם מִבֵּ֥ית עֲבָדִֽים׃}  \hspace{0.3cm}
	8~~\foreignlanguage{hebrew}{וְעַתָּ֣ה יִשְׂרָאֵ֗ל שְׁמַ֤ע אֶל־הַֽחֻקִּים֙ וְאֶל־הַמִּשְׁפָּטִ֔ים אֲשֶׁ֧ר אָֽנֹכִ֛י מְלַמֵּ֥ד אֶתְכֶ֖ם לַעֲשׂ֑וֹת לְמַ֣עַן תִּֽחְי֗וּ וּבָאתֶם֙ וִֽירִשְׁתֶּ֣ם אֶת־הָאָ֔רֶץ אֲשֶׁ֧ר יְהוָ֛ה אֱלֹהֵ֥י אֲבֹתֵיכֶ֖ם נֹתֵ֥ן לָכֶֽם׃ לֹ֣א תֹסִ֗פוּ עַל־הַדָּבָר֙ אֲשֶׁ֤ר אָנֹכִי֙ מְצַוֶּ֣ה אֶתְכֶ֔ם וְלֹ֥א תִגְרְע֖וּ}\LTRfootnote{\space \foreignlanguage{hebrew}{גרע} Q. \textit{to take; to diminish}} \foreignlanguage{hebrew}{מִמֶּ֑נּוּ לִשְׁמֹ֗ר אֶת־מִצְוֺת֙ יְהוָ֣ה אֱלֹֽהֵיכֶ֔ם אֲשֶׁ֥ר אָנֹכִ֖י מְצַוֶּ֥ה אֶתְכֶֽם׃} \hspace{0.3cm}
	9~~\foreignlanguage{hebrew}{וַיֹּסִ֙פוּ֙ בְּנֵ֣י יִשְׂרָאֵ֔ל לַעֲשׂ֥וֹת הָרַ֖ע בְּעֵינֵ֣י יְהוָ֑ה וַיִּתְּנֵ֧ם יְהוָ֛ה בְּיַד־פְּלִשְׁתִּ֖ים אַרְבָּעִ֥ים שָׁנָֽה׃}  \hspace{0.3cm}
	10~~\foreignlanguage{hebrew}{וַיִּשְׁלַ֨ח שָׁא֣וּל מַלְאָכִים֮ לָקַ֣חַת אֶת־דָּוִד֒ וַיַּ֗רְא אֶֽת־לַהֲקַ֤ת}\LTRfootnote{\space \foreignlanguage{hebrew}{לַהֲקָה} \textit{band, company} (BDB)} \foreignlanguage{hebrew}{הַנְּבִיאִים֙ נִבְּאִ֔ים וּשְׁמוּאֵ֕ל עֹמֵ֥ד נִצָּ֖ב עֲלֵיהֶ֑ם וַתְּהִ֞י עַֽל־מַלְאֲכֵ֤י שָׁאוּל֙ ר֣וּחַ אֱלֹהִ֔ים וַיִּֽתְנַבְּא֖וּ גַּם־הֵֽמָּה׃ וַיַּגִּ֣דוּ לְשָׁא֗וּל וַיִּשְׁלַח֙ מַלְאָכִ֣ים אֲחֵרִ֔ים וַיִּֽתְנַבְּא֖וּ גַּם־הֵ֑מָּה וַיֹּ֣סֶף שָׁא֗וּל וַיִּשְׁלַח֙ מַלְאָכִ֣ים שְׁלִשִׁ֔ים וַיִּֽתְנַבְּא֖וּ גַּם־הֵֽמָּה׃} \hspace{0.3cm}
	11~~\foreignlanguage{hebrew}{וַיֹּ֨אמֶר יְהוָ֜ה אֶל־מֹשֶׁ֗ה הֵ֣ן קָרְב֣וּ יָמֶיךָ֮ לָמוּת֒ קְרָ֣א אֶת־יְהוֹשֻׁ֗עַ וְהִֽתְיַצְּב֛וּ בְּאֹ֥הֶל מוֹעֵ֖ד וַאֲצַוֶּ֑נּוּ וַיֵּ֤לֶךְ מֹשֶׁה֙ וִֽיהוֹשֻׁ֔עַ וַיִּֽתְיַצְּב֖וּ בְּאֹ֥הֶל מוֹעֵֽד׃ וַיֵּרָ֧א יְהוָ֛ה בָּאֹ֖הֶל בְּעַמּ֣וּד עָנָ֑ן וַיַּעֲמֹ֛ד עַמּ֥וּד הֶעָנָ֖ן עַל־פֶּ֥תַח הָאֹֽהֶל׃}  \hspace{0.3cm}
	12~~\foreignlanguage{hebrew}{וַיְהִ֣י ׀ כְּכַלּ֣וֹת דָּוִ֗ד לְדַבֵּ֞ר אֶת־הַדְּבָרִ֤ים הָאֵ֙לֶּה֙ אֶל־שָׁא֔וּל וַיֹּ֣אמֶר שָׁא֔וּל הֲקֹלְךָ֥ זֶ֖ה בְּנִ֣י דָוִ֑ד וַיִּשָּׂ֥א שָׁא֛וּל קֹל֖וֹ וַיֵּֽבְךְּ׃}  \hspace{0.3cm}
	13~~\foreignlanguage{hebrew}{כָּל־הַמִּצְוָ֗ה אֲשֶׁ֨ר אָנֹכִ֧י מְצַוְּךָ֛ הַיּ֖וֹם תִּשְׁמְר֣וּן לַעֲשׂ֑וֹת לְמַ֨עַן תִּֽחְי֜וּן וּרְבִיתֶ֗ם וּבָאתֶם֙ וִֽירִשְׁתֶּ֣ם אֶת־הָאָ֔רֶץ אֲשֶׁר־נִשְׁבַּ֥ע יְהוָ֖ה לַאֲבֹתֵיכֶֽם׃ וְזָכַרְתָּ֣ אֶת־כָּל־הַדֶּ֗רֶךְ אֲשֶׁ֨ר הֹלִֽיכֲךָ֜ יְהוָ֧ה אֱלֹהֶ֛יךָ זֶ֛ה}\LTRfootnote{\space \foreignlanguage{hebrew}{זֶה} here as adv. \textit{now, here}} \foreignlanguage{hebrew}{}\LTRfootnote{\space \foreignlanguage{hebrew}{נסה} Pi.\ \textit{to put to the test, try}} \foreignlanguage{hebrew}{לָדַ֜עַת אֶת־אֲשֶׁ֧ר בִּֽלְבָבְךָ֛ הֲתִשְׁמֹ֥ר מִצְוֺתָ֖יו אִם־לֹֽא׃} \hspace{0.3cm}
	14~~\foreignlanguage{hebrew}{כֹּ֚ה אָמַ֣ר יְהוָ֔ה הִנְנִ֨י נֹתֵ֜ן אֶת־הָעִ֥יר הַזֹּ֛את בְּיַ֥ד מֶֽלֶךְ־בָּבֶ֖ל וּלְכָדָֽהּ׃ וְצִדְקִיָּ֙הוּ֙ מֶ֣לֶךְ יְהוּדָ֔ה לֹ֥א יִמָּלֵ֖ט מִיַּ֣ד הַכַּשְׂדִּ֑ים}\LTRfootnote{\space \foreignlanguage{hebrew}{כַּשְׂדִּים} \textit{Chaldeans} (the people ruling over Babylon since 626/5 BCE) (\textit{HALOT})} \foreignlanguage{hebrew}{כִּ֣י הִנָּתֹ֤ן יִנָּתֵן֙ בְּיַ֣ד מֶֽלֶךְ־בָּבֶ֔ל וְדִבֶּר־פִּ֣יו עִם־פִּ֔יו וְעֵינָ֖יו אֶת־עֵינָ֥יו תִּרְאֶֽינָה׃ וּבָבֶ֞ל יוֹלִ֤ךְ אֶת־צִדְקִיָּ֙הוּ֙ וְשָׁ֣ם יִֽהְיֶ֔ה עַד־פָּקְדִ֥י אֹת֖וֹ נְאֻם־יְהוָ֑ה כִּ֧י תִֽלָּחֲמ֛וּ אֶת־הַכַּשְׂדִּ֖ים לֹ֥א תַצְלִֽיחוּ׃}  \hspace{0.3cm}
	15~~\foreignlanguage{hebrew}{וַיֹּ֤אמֶר אֵלִיָּ֙הוּ֙ אֶל־הָעָ֔ם אֲנִ֞י נוֹתַ֧רְתִּי נָבִ֛יא לַיהוָ֖ה לְבַדִּ֑י וּנְבִיאֵ֣י הַבַּ֔עַל אַרְבַּע־מֵא֥וֹת וַחֲמִשִּׁ֖ים אִֽישׁ׃}  \hspace{0.3cm}
	16~~\foreignlanguage{hebrew}{וַיָּבֹא־שָׁ֥ם אֶל־הַמְּעָרָ֖ה}\LTRfootnote{\space \foreignlanguage{hebrew}{מְעָרָה} \textit{cave}} \foreignlanguage{hebrew}{וַיָּ֣לֶן שָׁ֑ם וְהִנֵּ֤ה דְבַר־יְהוָה֙ אֵלָ֔יו וַיֹּ֣אמֶר ל֔וֹ מַה־לְּךָ֥ פֹ֖ה אֵלִיָּֽהוּ׃ וַיֹּאמֶר֩ קַנֹּ֨א קִנֵּ֜אתִי}\LTRfootnote{\space \foreignlanguage{hebrew}{קנא} Pi.\ \textit{to be jealous; to be zealous}} \foreignlanguage{hebrew}{לַיהוָ֣ה ׀ אֱלֹהֵ֣י צְבָא֗וֹת כִּֽי־עָזְב֤וּ בְרִֽיתְךָ֙ בְּנֵ֣י יִשְׂרָאֵ֔ל אֶת־מִזְבְּחֹתֶ֣יךָ הָרָ֔סוּ}\LTRfootnote{\space \foreignlanguage{hebrew}{הרס} Q.\ \textit{to tear down}} \foreignlanguage{hebrew}{וְאֶת־נְבִיאֶ֖יךָ הָרְג֣וּ בֶחָ֑רֶב וֽ͏ָאִוָּתֵ֤ר אֲנִי֙ לְבַדִּ֔י וַיְבַקְשׁ֥וּ אֶת־נַפְשִׁ֖י לְקַחְתָּֽהּ׃} \hspace{0.3cm}
	17~~\foreignlanguage{hebrew}{וַֽיְהִי֙ הֵ֣מָּה בַדֶּ֔רֶךְ וְהַשְּׁמֻעָ֣ה}\LTRfootnote{\space \foreignlanguage{hebrew}{שְׁמוּעָה} \textit{report, news} (part.\ pass.\ f.\ sg.\ \foreignlanguage{hebrew}{שׁמע} \textit{to hear})} \foreignlanguage{hebrew}{בָ֔אָה אֶל־דָּוִ֖ד לֵאמֹ֑ר הִכָּ֤ה אַבְשָׁלוֹם֙ אֶת־כָּל־בְּנֵ֣י הַמֶּ֔לֶךְ וְלֹֽא־נוֹתַ֥ר מֵהֶ֖ם אֶחָֽד׃} \hspace{0.3cm}
	18~~\foreignlanguage{hebrew}{וַיִּשְׁמַ֤ע יְהוָה֙ אֶת־ק֣וֹל דִּבְרֵיכֶ֔ם בְּדַבֶּרְכֶ֖ם אֵלָ֑י וַיֹּ֨אמֶר יְהוָ֜ה אֵלַ֗י שָׁ֠מַעְתִּי אֶת־ק֨וֹל דִּבְרֵ֜י הָעָ֤ם הַזֶּה֙ אֲשֶׁ֣ר דִּבְּר֣וּ אֵלֶ֔יךָ הֵיטִ֖יבוּ כָּל־אֲשֶׁ֥ר דִּבֵּֽרוּ׃}  \hspace{0.3cm}
	19~~\foreignlanguage{hebrew}{הוֹד֣וּ לַיהוָ֣ה כִּי־ט֑וֹב כִּ֖י לְעוֹלָ֣ם חַסְדּֽוֹ׃ }  \hspace{0.3cm}
	20~~\foreignlanguage{hebrew}{אוֹדֶ֣ה יְ֭הוָה בְּכָל־לִבִּ֑י אֲ֝סַפְּרָ֗ה כָּל־נִפְלְאוֹתֶֽיךָ׃}  \hspace{0.3cm}
	\selectlanguage{english}
	
	
	
	\section{Hebrew Reading: Genesis 16:11--16}
	Translate Gen 16:11--16 with the help of notes below the text.
	
	\vspace{0.5cm}
	
	\selectlanguage{hebrew}
	\noindent
	\textsuperscript{11}~\foreignlanguage{hebrew}{וַיֹּ֤אמֶר לָהּ֙ מַלְאַ֣ךְ יְהוָ֔ה הִנָּ֥ךְ הָרָ֖ה וְיֹלַ֣דְתְּ בֵּ֑ן וְקָרָ֤את שְׁמוֹ֙ יִשְׁמָעֵ֔אל כִּֽי־שָׁמַ֥ע יְהוָ֖ה אֶל־עָנְיֵֽךְ׃} \hspace{0.3cm}
	\textsuperscript{12}~\foreignlanguage{hebrew}{וְה֤וּא יִהְיֶה֙ פֶּ֣רֶא אָדָ֔ם יָד֣וֹ בַכֹּ֔ל וְיַ֥ד כֹּ֖ל בּ֑וֹ וְעַל־פְּנֵ֥י כָל־אֶחָ֖יו יִשְׁכֹּֽן׃} \hspace{0.3cm}
	\textsuperscript{13}~\foreignlanguage{hebrew}{וַתִּקְרָ֤א שֵׁם־יְהוָה֙ הַדֹּבֵ֣ר אֵלֶ֔יהָ אַתָּ֖ה אֵ֣ל רֳאִ֑י כִּ֣י אֽ͏ָמְרָ֗ה הֲגַ֥ם הֲלֹ֛ם רָאִ֖יתִי אַחֲרֵ֥י רֹאִֽי׃} \hspace{0.3cm}
	\textsuperscript{14}~\foreignlanguage{hebrew}{עַל־כֵּן֙ קָרָ֣א לַבְּאֵ֔ר בְּאֵ֥ר לַחַ֖י רֹאִ֑י הִנֵּ֥ה בֵין־קָדֵ֖שׁ וּבֵ֥ין בָּֽרֶד׃} \hspace{0.3cm}
	\textsuperscript{15}~\foreignlanguage{hebrew}{וַתֵּ֧לֶד הָגָ֛ר לְאַבְרָ֖ם בֵּ֑ן וַיִּקְרָ֨א אַבְרָ֧ם שֶׁם־בְּנ֛וֹ אֲשֶׁר־יָלְדָ֥ה הָגָ֖ר יִשְׁמָעֵֽאל׃} \hspace{0.3cm}
	\textsuperscript{16}~\foreignlanguage{hebrew}{וְאַבְרָ֕ם בֶּן־שְׁמֹנִ֥ים שָׁנָ֖ה וְשֵׁ֣שׁ שָׁנִ֑ים בְּלֶֽדֶת־הָגָ֥ר אֶת־יִשְׁמָעֵ֖אל לְאַבְרָֽם׃} \hspace{0.3cm}
	\selectlanguage{english}
	
	\vspace{0.25cm}
	
	\hspace*{-0.5cm}\begin{longtable}{p{0.075\linewidth} p{0.1\linewidth}p{0.725\linewidth}}
		16:11 & \foreignlanguage{hebrew}{הָרֶה}* & \textit{pregnant} (adj.) \\
		& \foreignlanguage{hebrew}{וְיֹלַדְתְּ} & This appears to be a \textit{lectio mixta}. One can either read \foreignlanguage{hebrew}{וְיֹלֶדֶת} or \foreignlanguage{hebrew}{וְיָלַדְתְּ} (JM §§\,89\,\textit{j}, 16\,\textit{g}). The vowel signs of the Masoretic Text combine both forms. Other grammars accept this forms as participle fem.\ sg.\ without a helping vowel to break up the final consonant cluster. \\
		& \foreignlanguage{hebrew}{עֳנִי} & \textit{affliction, misery, oppressed situation} \\
		16:12 & \foreignlanguage{hebrew}{פֶּרֶא} & \textit{wild ass}; \foreignlanguage{hebrew}{פֶּרֶא אָדָם} \textit{a man like a wild ass} (\textit{HALOT}) \\
		& \foreignlanguage{hebrew}{עַלְ־פְּנֵי} & \textit{in front of, east of} (BDB); \textit{at the expense of; to the disadvantage of} (\textit{HALOT}) \\
		16:13 & \foreignlanguage{hebrew}{דֹּבֵר} & The verb \foreignlanguage{hebrew}{דבר} is used in the Qal here (part.\ in 39 of the 41 cases of the verb in Qal) \\
		& \foreignlanguage{hebrew}{רֲאִי} & \textit{seeing} (\textit{rɔ̆ʾī}) \\
		& \foreignlanguage{hebrew}{הֲלֹם} & \textit{here} \\
		16:14 & \foreignlanguage{hebrew}{קָדֵשׁ} & (and \foreignlanguage{hebrew}{בֶּרֶד}) names of places
	\end{longtable}
	
	
	\chapter{Chapter 24}
	
	\renewcommand\arraystretch{1.4}
	
	\section{Vocabulary}
	
	\subsection{Verbs}
	
	
	% For the centering of the separation between the two columns see the documentation of the array package, page 2 
	
	\begin{longtable}{>{\raggedleft}p{0.175\linewidth} p{0.75\linewidth}}
		\foreignlanguage{hebrew}{בוא} & Q.\ \textit{to come, to enter}; Hi.\ \textit{to bring, bring in} \\
		\foreignlanguage{hebrew}{בין} & Q.\ \textit{to discern, understand}; Hi.\ \textit{to understand; to give understanding, teach} \\
		\foreignlanguage{hebrew}{כון} & Ni.\ \textit{to be established; to be steadfast; to be permanent; to be ready}; Hi.\ \textit{to establish, set up; to make ready, prepare; to direct}; Polel \textit{to set up, establish} \\
		\foreignlanguage{hebrew}{מות} & Q.\ \textit{to die}; Hi.\ \textit{to kill, put to death} \\
		\foreignlanguage{hebrew}{נוח} & Q.\ \textit{to rest; to settle down; to have rest}; Hi.\ A \textit{to cause to rest, give rest}; Hi.\ B \textit{to lay, set down; to leave, let remain} \\
		\foreignlanguage{hebrew}{סור} & Q.\ \textit{to turn aside}; Hi.\ \textit{to cause to turn aside, depart; to remove, take away} \\
		\foreignlanguage{hebrew}{עור} & Q.\ \textit{to awake; to stir}; Hi.\ \textit{to wake up; to excite; to put into motion} \\ % HALOT
		\foreignlanguage{hebrew}{פוץ} & Q.\ \textit{to be dispersed; disperse} (intr.); Ni.\ \textit{be scattered}; Hi.\ \textit{to scatter} \\ % BDB
		\foreignlanguage{hebrew}{קום} & Q.\ \textit{to arise, stand up, stand}; Hi.\ \textit{to raise, raise up} \\ % BDB
		\foreignlanguage{hebrew}{רום} & Q.\ \textit{to be high; to be raised, uplifted; to rise} (intrans.); Hi.\ \textit{to raise, lift up; to exalt; to set up}; Polel \textit{to raise, lift up; to exalt} \\ % BDB
		\foreignlanguage{hebrew}{ריב} & Q.\ \textit{to strive, contend} \\ % BDB
		\foreignlanguage{hebrew}{שׁוב} & Q.\ \textit{to turn back, return}; Hi.\ \textit{to bring back, return} (trans.) \\
	\end{longtable}
	
	
	\subsection{Nouns}
	
	\begin{longtable}{>{\raggedleft}p{0.175\linewidth} p{0.75\linewidth}}
		\foreignlanguage{hebrew}{אֶבְיוֺן} & \textit{needy, poor} (adj.) \\ % HALOT
		\foreignlanguage{hebrew}{אַחֲרִית} & \textit{end; future; descendants} (fem.) \\ % HALOT
		\foreignlanguage{hebrew}{אֹרַח} & \textit{way, path} \\
		\foreignlanguage{hebrew}{חָזָק} & \textit{strong; heavy, severe; firm, hard} (adj.) \\
		\foreignlanguage{hebrew}{יְאֹר} & \textit{the Nile} (usually with the article \foreignlanguage{hebrew}{הַיְאֹר}, pl.\ \foreignlanguage{hebrew}{יְאֹרִים}; \textit{Nile-arms, Nile-canals}) \textit{river, stream} \\
		\foreignlanguage{hebrew}{יַעַר} & \textit{wood; forest; thicket} \\ % BDB
		\foreignlanguage{hebrew}{כְּסִיל} & \textit{fool} \\
		\foreignlanguage{hebrew}{מִזְמוֺר} & \textit{psalm} \\
		\foreignlanguage{hebrew}{מָלֵא} & \textit{full} (adj.) \\ % HALOT
		\foreignlanguage{hebrew}{מַצָּה} & \textit{unleavened bread, matzah (matzo)} \\
		\foreignlanguage{hebrew}{מָרוֺם} & \textit{height, elevated place} \\ % BDB
		\foreignlanguage{hebrew}{מִשְׁמֶרֶת} & \textit{guard, watch; charge, injunction; function, office} \\ % BDB Action point: move to verb
		\foreignlanguage{hebrew}{צָרָה} & \textit{need; distress; anxiety} \\ % HALOT
		\foreignlanguage{hebrew}{שֶׁלֶם} & a kind of offering: \textit{conclusion offering; salvation offering} \\ % BDB
		\foreignlanguage{hebrew}{שֵׁן} & \textit{tooth, ivory} (gem.\ noun; dual \foreignlanguage{hebrew}{שִׁנַּיִם}) \\
	\end{longtable}
	
	\section{Verbs II \textit{w/y}: Derived Binyanim}
	
	\subsection{Verbs II \textit{w/y}: Niphal, Hiphil and Hophal}
	
	\begin{center}
		\begin{longtable}{|lll|r|r|r|r|}
			\hline
			& & & \multicolumn{1}{c|}{Niphal} & \multicolumn{2}{c|}{Hiphil} & \multicolumn{1}{c|}{Hophal} \\
			& & & \multicolumn{1}{c|}{\foreignlanguage{hebrew}{כון}} & \multicolumn{1}{c}{\foreignlanguage{hebrew}{קום}} & \multicolumn{1}{c|}{\foreignlanguage{hebrew}{בוא}} & \multicolumn{1}{c|}{\foreignlanguage{hebrew}{קום}} \\
			\hline
			\endhead
			\hline
			\endfoot
			SC & sg. & 3 m. & \foreignlanguage{hebrew}{נָכוֺן} & \foreignlanguage{hebrew}{הֵקִים} & \foreignlanguage{hebrew}{הֵבִיא} & \foreignlanguage{hebrew}{הוּשַׁב} \\
			& & 3 f. & \foreignlanguage{hebrew}{נָכ֫וֺנָה}  & \foreignlanguage{hebrew}{הֵקִ֫ימָה} & \foreignlanguage{hebrew}{הֵבִ֫יאָה} & \foreignlanguage{hebrew}{} \\
			& & 2 m. &  & \foreignlanguage{hebrew}{הֲקִימ֫וֺתָ} & \foreignlanguage{hebrew}{הֵבֵ֫אתָ} & \foreignlanguage{hebrew}{} \\
			& & 2 f. &  & \foreignlanguage{hebrew}{הֲקִימוֺת} & \foreignlanguage{hebrew}{הֵבֵאת} & \foreignlanguage{hebrew}{} \\
			& & 1 c. &  & \foreignlanguage{hebrew}{הֲקִימ֫וֺתִי} & \foreignlanguage{hebrew}{הֵבֵ֫אתִי} & \foreignlanguage{hebrew}{} \\
			\hline
			& pl. & 3 c. & \foreignlanguage{hebrew}{נָכ֫וֺנוּ} & \foreignlanguage{hebrew}{הֵקִ֫ימוּ} & \foreignlanguage{hebrew}{הֵבִ֫יאוּ} & \foreignlanguage{hebrew}{} \\
			& & 2 m. &  & \foreignlanguage{hebrew}{הֲקִימוֺתֶם} & \foreignlanguage{hebrew}{הֲבֵאתֶם} & \foreignlanguage{hebrew}{} \\
			& & 2 f. &  & \foreignlanguage{hebrew}{הֲקִימוֺתֶן} & \foreignlanguage{hebrew}{הֲבֵאתֶן} & \foreignlanguage{hebrew}{} \\
			& & 1 c. &  & \foreignlanguage{hebrew}{הֲקִימ֫וֺנוּ} & \foreignlanguage{hebrew}{הֵבֵ֫אנוּ} & \foreignlanguage{hebrew}{} \\
			\hline
			\pagebreak
			\hline
			PC & sg. & 3 m. & \foreignlanguage{hebrew}{יִכּוֹן} & \foreignlanguage{hebrew}{יָקִים} & \foreignlanguage{hebrew}{יָבִיא} & \foreignlanguage{hebrew}{יוּשַׁב} \\
			& & 3 f. & \foreignlanguage{hebrew}{תִּכּוֹן} & \foreignlanguage{hebrew}{תָּקִים} & \foreignlanguage{hebrew}{תָּבִיא} & \foreignlanguage{hebrew}{תּוּמַת} \\
			& & 2 m. & \foreignlanguage{hebrew}{תִּכּוֹן} & \foreignlanguage{hebrew}{תָּקִים} & \foreignlanguage{hebrew}{תָּבִיא} & \foreignlanguage{hebrew}{} \\
			& & 2 f. & & \foreignlanguage{hebrew}{תָּקִ֫ימִי} & \foreignlanguage{hebrew}{תָּבִ֫יאִי} & \foreignlanguage{hebrew}{} \\
			& & 1 c. &  & \foreignlanguage{hebrew}{אָקִים} & \foreignlanguage{hebrew}{אָבִיא} & \foreignlanguage{hebrew}{} \\
			\hline
			& pl. & 3 m. & \foreignlanguage{hebrew}{יִכֹּ֫נוּ} & \foreignlanguage{hebrew}{יָקִ֫ימוּ} & \foreignlanguage{hebrew}{יָבִ֫יאוּ} & \foreignlanguage{hebrew}{יוּמְתוּ} \\
			& & 3 f. & & \foreignlanguage{hebrew}{תְּקִימֶ֫ינָה} & \foreignlanguage{hebrew}{תְּבִיאֶ֫ינָה} & \foreignlanguage{hebrew}{} \\
			& & 2 m. & & \foreignlanguage{hebrew}{תָּקִ֫ימוּ} & \foreignlanguage{hebrew}{תָּבִ֫יאוּ} & \foreignlanguage{hebrew}{} \\
			& & 2 f. & & \foreignlanguage{hebrew}{תְּקִימֶ֫ינָה} & \foreignlanguage{hebrew}{תְּבִיאֶ֫ינָה} & \foreignlanguage{hebrew}{} \\
			& & 1 c. & & \foreignlanguage{hebrew}{נָקִים} & \foreignlanguage{hebrew}{נָבִיא} & \foreignlanguage{hebrew}{} \\
			\hline
			Juss. & sg. & 3 m. & & \foreignlanguage{hebrew}{יָקֵם} & \foreignlanguage{hebrew}{יָבֵא} & \foreignlanguage{hebrew}{} \\
			\textit{waw}-PC & sg. & 3 m. & & \foreignlanguage{hebrew}{וַיָּ֫קֶם} & \foreignlanguage{hebrew}{וַיָּבֵא} & \foreignlanguage{hebrew}{וַיּוּשַׁב} \\
			\hline
			Impv. & sg. & m. & \foreignlanguage{hebrew}{הִכּוֹן} & \foreignlanguage{hebrew}{הָקֵם} & \foreignlanguage{hebrew}{הָבֵא} &  \\
			& & & & \foreignlanguage{hebrew}{הָקִ֫ימָה} & \foreignlanguage{hebrew}{הָבִ֫יאָה} &  \\
			& & f. & & \foreignlanguage{hebrew}{הָקִ֫ימִי} & \foreignlanguage{hebrew}{הָבִ֫יאִי} &  \\
			& pl. & m. & & \foreignlanguage{hebrew}{הָקִ֫ימוּ} & \foreignlanguage{hebrew}{הָבִ֫יאוּ} &  \\
			& & f. & & & \foreignlanguage{hebrew}{} &  \\
			\hline
			Inf.\ cs.\ & & & \foreignlanguage{hebrew}{הִכּוֺן} & \foreignlanguage{hebrew}{הָקִים} & \foreignlanguage{hebrew}{הָבִיא} & \\
			Inf.\ abs.\ & & & \foreignlanguage{hebrew}{הִכּוֺן} & \foreignlanguage{hebrew}{הָקֵם} & \foreignlanguage{hebrew}{הָבֵא} & \\
			\hline
			Part. & sg. & m. & \foreignlanguage{hebrew}{נָכוֹן} & \foreignlanguage{hebrew}{מֵקִים} & \foreignlanguage{hebrew}{מֵבִיא} & \foreignlanguage{hebrew}{מוּשָׁב} \\
			& & f. & \foreignlanguage{hebrew}{נְכוֹנָה} & \foreignlanguage{hebrew}{מְקִימָה} & & \foreignlanguage{hebrew}{} \\
			& pl. & m. & \foreignlanguage{hebrew}{נְכֹנִים} & \foreignlanguage{hebrew}{מְקִימִים} & \foreignlanguage{hebrew}{מְבִיאִים} & \foreignlanguage{hebrew}{מוּשָׁבִים} \\
			& & f. & \foreignlanguage{hebrew}{} & \foreignlanguage{hebrew}{מְקִימוֺת} & & \foreignlanguage{hebrew}{} \\
		\end{longtable}
	\end{center}
	
	\noindent \textbf{Notes}
	\nopagebreak
	
	\noindent In the Niphal, the open prefix syllable in the suffix conjugation and the participle needs to be noted. In the prefix conjugation the /n/ of the Ni.\ is assimilated to the first root consonant, e.g., \textit{*yinkōn > yikkōn} \foreignlanguage{hebrew}{יִכּוֺן}.
	
	Hiphil forms are characterized by an open prefix syllable. If the prefix syllable immediately precedes the stressed syllable, it has a full vowel. Otherwise the vowel is reduced.
	
	In the suffix conjugation Hiphil and the participle, the vowel of the prefix syllable is /ē/ (\textit{ṣere}) (cf.\ the Hiphil form of the strong verb \foreignlanguage{hebrew}{הִכְתִּיב}). In the other forms, i.e., the prefix conjugation, the imperative, and the infinitives, it is /ā/ (\textit{qameṣ}) (cf. the Hiphil prefix conjugation form of the strong verb \foreignlanguage{hebrew}{יַכְתִּיב} etc.)
	
	Before consonantal suffixes a long, unchanging connecting vowel /ō/ is usually inserted between the second root consonant and the suffix which carries the stress, e.g., \foreignlanguage{hebrew}{הֲקִימֹ֫תִי} \textit{I have carried out} (1\,Sam 15:13). However, forms without connecting vowel are attested, too, e.g., \foreignlanguage{hebrew}{הֵנַ֫פְתָּ} (root \foreignlanguage{hebrew}{נוף}) \textit{you have swung [your tool above it]} (Exod 20:25).
	
	In the prefix conjugation, 3/2 f.\ pl.\ forms without connecting vowel are attested, too, e.g., \foreignlanguage{hebrew}{תָּשֵׁ֫בְנָה} (verb \foreignlanguage{hebrew}{שׁוב}), \foreignlanguage{hebrew}{תָּקִ֫ימְנָה} (verb \foreignlanguage{hebrew}{קום}).
	
	The doubly weak verb \foreignlanguage{hebrew}{בוא} is frequently used in the Hiphil. As can be expected, the /ʾ/ is silent at the end of a syllable, but it functions as a consonant when a vowel follows. In the suffix conjugation Hiphil, the verb \foreignlanguage{hebrew}{בוא} has usually forms without connecting vowel, e.g., \foreignlanguage{hebrew}{הֵבֵאתָ} \textit{you have brought}. But with enclitic personal pronouns forms with connecting vowel are used, e.g., \foreignlanguage{hebrew}{הֲבִיאֹתַ֫נִי} \textit{you have brought me} (2 Sam 7:18).
	
	In forms without suffix, II\,\textit{w/y} verbs have distinct jussive and \textit{wayyiqṭol} forms in the Hiphil, e.g., \foreignlanguage{hebrew}{יָקֵם} \textit{may he set up} as opposed to \foreignlanguage{hebrew}{יָקִים} \textit{he will set up}. \textit{waw}-PC forms (\textit{wayyiqṭol}) show retraction of stress and shortening of the thematic vowel, e.g., \foreignlanguage{hebrew}{וַיָּ֫קֶם} \textit{wayyā́qæm}; \foreignlanguage{hebrew}{וַיָּ֫שֶׁב} \textit{wayyā́šæḇ}.
	
	Long imperative forms of II\,\textit{w/y} verbs have the long vowel /ī/ between the first and second root consonant in the Hiphil, e.g., \foreignlanguage{hebrew}{הָקִ֫ימָה} and \foreignlanguage{hebrew}{הָבִ֫יאָה}.
	
	In the Hiphil, all II\,\textit{w/y} verbs have the same thematic vowel irrespective of their thematic vowel in the Qal.
	
	\begin{center}
		\begin{tabular}{lrrr}
			& \multicolumn{3}{c}{Suffix Conjugation} \\
			& \multicolumn{1}{c}{\textit{ā} (< *\textit{a})} & \multicolumn{1}{c}{\textit{ē} (< *\textit{i})} & \multicolumn{1}{c}{\textit{ō} (< *\textit{u})} \\
			Q. & \foreignlanguage{hebrew}{קָם} & \foreignlanguage{hebrew}{מֵת} & \foreignlanguage{hebrew}{בּוֺשׁ} \\
			Hi. & \foreignlanguage{hebrew}{הֵקִים} & \foreignlanguage{hebrew}{הֵמִית} & \foreignlanguage{hebrew}{הֱבִישׁוֺתָ} \\
		\end{tabular}
		
		\vspace{0.5cm}
		
		\begin{tabular}{lrrr}
			& \multicolumn{3}{c}{Prefix Conjugation} \\
			& \multicolumn{1}{c}{\textit{ū}} & \multicolumn{1}{c}{\textit{ī}} & \multicolumn{1}{c}{\textit{ō} (< *\textit{ā})} \\
			Q. & \foreignlanguage{hebrew}{יָקוּם} & \foreignlanguage{hebrew}{יָבִין} & \foreignlanguage{hebrew}{תֵּבֹ֫שׁוּ} \\
			Hi. & \foreignlanguage{hebrew}{יָקִים} & \foreignlanguage{hebrew}{יָבִין} & \foreignlanguage{hebrew}{תָּבִ֫ישׁוּ} \\
		\end{tabular}
	\end{center}
	
	The verb \foreignlanguage{hebrew}{נוח} Q.\ \textit{to rest} has two distinct sets of Hiphil forms with different meanings. Hiphil~A follows the pattern of the verb \foreignlanguage{hebrew}{קום} in the Hiphil. In  Hiphil B the first root consonant is geminated. The meaning of \foreignlanguage{hebrew}{נוח} in Hiphil A is \textit{to cause to rest, to give rest}. In Hiphil B the meaning is \textit{to lay, set down; to leave, let remain}. The verb \foreignlanguage{hebrew}{נוח} occurs in Hiphil A about 34 times and in Hiphil B about 69 times.
	
	\vspace{0.25cm}
	\begin{center}
		\begin{tabular}{lllrr}
			& & & \multicolumn{1}{c}{Hiphil A} & \multicolumn{1}{c}{Hiphil B} \\
			SC & sg. & 3 m. & \foreignlanguage{hebrew}{הֵנִיחַ} & \foreignlanguage{hebrew}{הִנִּיחַ} \\
			PC & sg. & 3 m. & \foreignlanguage{hebrew}{יָנִיחַ} & \foreignlanguage{hebrew}{יַנִּיחַ} \\
			\textit{waw}-PC & sg. & 3 m. & \foreignlanguage{hebrew}{וַיָּ֫נַח} & \foreignlanguage{hebrew}{וַיַּנַּח} \\
			Impv. & sg. & m. & & \foreignlanguage{hebrew}{הַנַּח} \\
			& pl. & m. & \foreignlanguage{hebrew}{הָנִ֫יחוּ} & \foreignlanguage{hebrew}{הַנִּ֫יחוּ} \\
			Inf.\ cs. & & & \foreignlanguage{hebrew}{הָנִיחַ} & \foreignlanguage{hebrew}{הַנִּיחַ} \\
			Part. & sg. & m. & \foreignlanguage{hebrew}{מֵנִיחַ} & \foreignlanguage{hebrew}{מַנִּיחַ} \\
		\end{tabular}
	\end{center}
	\vspace{0.25cm}
	
	In forms of the \textit{waw}-PC and the imperative without suffixes the vowel between the two root consonants is /a/ (\textit{pataḥ}) because of the guttural /ḥ/ \foreignlanguage{hebrew}{ח}. The Hiphil A \textit{waw}-PC form \foreignlanguage{hebrew}{וַיָּ֫נַח} is identical to the corresponding Qal form; the context helps to distinguish the two forms.
	
	The verb \foreignlanguage{hebrew}{בושׁ} Q.\ \textit{to be ashamed} has two distinct Hiphil forms with different meanings. There are forms that follow the other II\,\textit{w/y} verbs such as \foreignlanguage{hebrew}{קום}, e.g., \foreignlanguage{hebrew}{הֱבִישׁוֺתָ}, \foreignlanguage{hebrew}{תָּבִ֫ישׁוּ}. These forms have the meaning \textit{to put to shame}. The variant forms follow the I\,\textit{y} verbs, e.g., \foreignlanguage{hebrew}{}, \foreignlanguage{hebrew}{הוֺבִישׁ}, \foreignlanguage{hebrew}{הֹבִישׁוּ}. These forms have the meaning \textit{to be put to shame, be ashamed}.
	
	Hophal forms of II\,\textit{w/y} verbs follow the pattern of I\,\textit{y} verbs. Originally the Hophal forms had a short vowel in the prefix syllable, e.g., \textit{*huqam}. As the short vowel /u/ in the open syllable had to be reduced, the characteristic vowel of the passive would have been lost. The short vowel was therefore lengthened \textit{*huqam > hūqam} by analogy with the I\,\textit{y} verbs.
	
	Hophal forms with a vocalic suffix like \foreignlanguage{hebrew}{הֻמְתוּ} \textit{they were killed} (2\,Sam 21:9) and \foreignlanguage{hebrew}{הוּבְאוּ} \textit{they were brought in} (Gen 43:18) are the only forms of II\,\textit{w/y} verbs without a vowel between the first and the second root consonant in Masoretic Hebrew.
	
	
	\subsection{Verbs II \textit{w/y}: Polel, Polal and Hitpolel}
	
	With very few (and late) exceptions, II\,\textit{w/y} verbs are not used in the Piel, Pual or Hitpael, e.g., \foreignlanguage{hebrew}{לְקַיֵּם} \textit{to confirm} (Esth 9:29). Instead, they are used in the Polel, Polal and Hitpolel. These binyanim are characterized by a long unchangeable vowel /ō/ after the first root consonant and reduplication of the second root consonant. As a result, forms of these binyanim are easy to identify. The vowel /ō/ after the first root consonant is often spelled plene but may be spelled defectively, too. The meaning of the Polel, Polal and Hitpolel corresponds with the Piel, Pual and Hitpael, respectively. In Biblical Hebrew, there are 172 instances of Polel forms, 13 Polal forms and 70 Hitpolel forms.
	
	The following table contains characteristic forms. For Polal the verb \foreignlanguage{hebrew}{כון} Po.\ \textit{to set up, establish} is used and for the Hitpolel the verb \foreignlanguage{hebrew}{בין} Hitpo.\ \textit{to behave intelligently}, because these verbs are relatively frequent. Forms of other verbs supplement the two verbs, when important forms are missing.
	
	\vspace{0.5cm}
	
	\begin{center}
		\begin{tabular}{|lll|r|r|r|}
			\hline
			\multicolumn{3}{|c|}{} & \multicolumn{1}{c|}{Polel} & \multicolumn{1}{c|}{Polal} & \multicolumn{1}{c|}{Hitpolel} \\
			\hline
			SC & sg. & 3 m. & \foreignlanguage{hebrew}{כּוֺנֵן} & & \foreignlanguage{hebrew}{הִתְבּוֹנָ֑ן} \\
			& & 2 m. & \foreignlanguage{hebrew}{כּוֺנַ֫נְתָּ} & \foreignlanguage{hebrew}{חוֹלָ֑לְתָּ} & \foreignlanguage{hebrew}{הִתְבֹּנַ֫נְתָּ} \\
			%			 & & 1 c. & & &  \\
			\hline
			& pl. & 3 c. & \foreignlanguage{hebrew}{כֹּוֺנְנוּ} & \foreignlanguage{hebrew}{כּוֹנָ֑נוּ} & \foreignlanguage{hebrew}{הִתְבּוֹנָ֑נוּ} \\
			%			 & & 2 m. & & & \\
			%			 & & 2 f. & & &  \\
			%			 & & 1 c. & & & \\
			\hline
			PC & sg. & 3 m. & \foreignlanguage{hebrew}{יְכוֺנֵן} & & \foreignlanguage{hebrew}{יִתְבּוֹנָ֑ן} \\
			%			 & & 3 f. & \foreignlanguage{hebrew}{תְּכוֺנֵן} & & \\
			%			 & & 2 m. & \foreignlanguage{hebrew}{תְּכוֺנֵן} & & \\
			%			 & & 2 f. & \foreignlanguage{hebrew}{תְּכוֹנְנִי} & & \\
			%			 & & 1 c. & \foreignlanguage{hebrew}{אֲכוֺנֵן} & & \\
			& pl. & 3 c. & \foreignlanguage{hebrew}{יְכוֺנְנוּ}* & \foreignlanguage{hebrew}{יְחוֹלָ֑לוּ} & \foreignlanguage{hebrew}{יִתְבּוֺנְנוּ} \\
			%			 & & 2 f. & \foreignlanguage{hebrew}{תְּכוֺנֵנָּה} & & \\
			%			 & & 2 m. & \foreignlanguage{hebrew}{תְּכוֺנְנוּ} & & \\
			%			 & & 2 f. & \foreignlanguage{hebrew}{תְּכוֺנֵנָּה} & & \\
			%			 & & 1 c. & \foreignlanguage{hebrew}{נְכוֺנֵן} & & \\
			\hline
			Impv. & sg. & 2 m. & \foreignlanguage{hebrew}{כּוֺנֵן} & & \foreignlanguage{hebrew}{הִתְבּוֹנֵן} \\
			%			 & & 2 f. & \foreignlanguage{hebrew}{כּוֺנְנִי}* \\
			& pl. & 2 m. & \foreignlanguage{hebrew}{כּוֺנְנוּ} & & \foreignlanguage{hebrew}{הִתְבּוֹנְנוּ} \\
			%			 & & 2 f. & \foreignlanguage{hebrew}{כּוֺנֵנָּה} \\
			\hline
			Inf.\ cs. & & & \foreignlanguage{hebrew}{כּוֺנֵן} & & \\
			\hline
			Part. & sg. & m. & \foreignlanguage{hebrew}{מְכוֺנֵן}* & \foreignlanguage{hebrew}{מְרוֹמַם} & \foreignlanguage{hebrew}{מִתְגּוֹרֵר} \\
			& pl. & m. & \foreignlanguage{hebrew}{מְכוֺנְנִים}* & & \foreignlanguage{hebrew}{מִתְקוֹמְמִים} \\
			\hline
		\end{tabular}
	\end{center}
	
	\vspace{0.5cm}
	
	% Old table with all forms of the Polel
	%\begin{center}
	%	\begin{longtable}{|lll|r|}
		%		\hline
		%		SC & sg. & 3 m. & \foreignlanguage{hebrew}{כּוֺנֵן} \\
		%		 & & 3 f. & \foreignlanguage{hebrew}{כּוֺנְנָה} \\
		%		 & & 2 m. & \foreignlanguage{hebrew}{כּוֺנַנְתָּ} \\
		%		 & & 2 f. & \foreignlanguage{hebrew}{כּוֺנַנְתְּ} \\
		%		 & & 1 c. & \foreignlanguage{hebrew}{כּוֺנַנְתִּי} \\
		%		\hline
		%		 & pl. & 3 c. & \foreignlanguage{hebrew}{כֹּוֺנְנוּ} \\
		%		 & & 2 m. & \foreignlanguage{hebrew}{כּוֺנַנְתֶּם} \\
		%		 & & 2 f. & \foreignlanguage{hebrew}{כּוֺנַנְתֶּן} \\
		%		 & & 1 c. & \foreignlanguage{hebrew}{כּוֺנַנּוּ} \\
		%		 \hline
		%		PC & sg. & 3 m. & \foreignlanguage{hebrew}{יְכוֺנֵן} \\
		%		 & & 3 f. & \foreignlanguage{hebrew}{תְּכוֺנֵן} \\
		%		 & & 2 m. & \foreignlanguage{hebrew}{תְּכוֺנֵן} \\
		%		 & & 2 f. & \foreignlanguage{hebrew}{תְּכוֹנְנִי} \\
		%		 & & 1 c. & \foreignlanguage{hebrew}{אֲכוֺנֵן} \\
		%		 \hline
		%		 & pl. & 3 c. & \foreignlanguage{hebrew}{יְכוֺנְנוּ} \\
		%		 & & 2 f. & \foreignlanguage{hebrew}{תְּכוֺנֵנָּה} \\
		%		 & & 2 m. & \foreignlanguage{hebrew}{תְּכוֺנְנוּ} \\
		%		 & & 2 f. & \foreignlanguage{hebrew}{תְּכוֺנֵנָּה} \\
		%		 & & 1 c. & \foreignlanguage{hebrew}{נְכוֺנֵן} \\
		%		 \hline
		%		Impv. & sg. & 2 m. & \foreignlanguage{hebrew}{כּוֺנֵן} \\
		%		 & & & \foreignlanguage{hebrew}{כּוֺנְנָה} \\
		%		 & & 2 f. & \foreignlanguage{hebrew}{כּוֺנְנִי} \\
		%		\hline
		%		 & pl. & 2 m. & \foreignlanguage{hebrew}{כּוֺנְנוּ} \\
		%		 & & 2 f. & \foreignlanguage{hebrew}{כּוֺנֵנָּה} \\
		%		\hline
		%		Inf.\ cs. & & & \foreignlanguage{hebrew}{כּוֺנֵן} \\
		%		\hline
		%		Inf.\ abs. & & & \foreignlanguage{hebrew}{} \\
		%		\hline
		%		Part. & sg. & m. & \foreignlanguage{hebrew}{מְכוֺנֵן} \\
		%		 & pl. & m. & \foreignlanguage{hebrew}{מְכוֺנְנִים} \\
		%	\end{longtable}
	%\end{center}
	
	\noindent \textbf{Notes}
	\nopagebreak
	
	\noindent The immediate context helps to distinguish between ambiguous forms of the Polel (active) and Polal (passive).
	
	Participles have a /m/ \foreignlanguage{hebrew}{מְ} as prefix consonant.
	
	In the Polel and Hitpolel, the vowel /e/ after the second root consonant is changed to /a/ in suffix conjugation forms with consonantal sufffix due to Philippi's Law.
	
	In the Hitpolel, pausal forms have the vowel /ā/ between the reduplicated root consonant.
	
	
	
	\section{Exercises}
	
	\subsection{Translation of Verbal Forms}
	
	Translate the following verbal forms. Identify the gender (masc., fem., comm.) and number (sg., pl.) of forms of which the English translation is ambiguous (i.e., \textit{you}, \textit{they}). Mark the stressed syllable if stress is not on the last syllable.
	
	\hspace{0.5cm}
	
	\selectlanguage{hebrew}
	
	\noindent
	1~~\foreignlanguage{hebrew}{הֵבִיא}  \hspace{0.3cm}
	2~~\foreignlanguage{hebrew}{תָּבִיא}  \hspace{0.3cm}
	3~~\foreignlanguage{hebrew}{הֵבֵאתָ}  \hspace{0.3cm}
	4~~\foreignlanguage{hebrew}{יָבִיאוּ}  \hspace{0.3cm}
	5~~\foreignlanguage{hebrew}{תִּכּוֹן}  \hspace{0.3cm}
	6~~\foreignlanguage{hebrew}{יִכֹּנוּ}  \hspace{0.3cm}
	7~~\foreignlanguage{hebrew}{נָכוֹנוּ}  \hspace{0.3cm}
	8~~\foreignlanguage{hebrew}{הֵמִיתוּ}  \hspace{0.3cm}
	9~~\foreignlanguage{hebrew}{הֲמִתֶּם}  \hspace{0.3cm}
	10~~\foreignlanguage{hebrew}{וַיָּמִיתוּ}  \hspace{0.3cm}
	11~~\foreignlanguage{hebrew}{הֲסִרֹתִי}  \hspace{0.3cm}
	12~~\foreignlanguage{hebrew}{הָסִירוּ}  \hspace{0.3cm}
	13~~\foreignlanguage{hebrew}{יָסִיר}  \hspace{0.3cm}
	14~~\foreignlanguage{hebrew}{יָסֵר}  \hspace{0.3cm}
	15~~\foreignlanguage{hebrew}{‎הֱשִׁיבוֹתָ}  \hspace{0.3cm}
	16~~\foreignlanguage{hebrew}{נָשִׁיב}  \hspace{0.3cm}
	17~~\foreignlanguage{hebrew}{‎הִנִּיחוּ}  \hspace{0.3cm}
	18~~\foreignlanguage{hebrew}{‎יַנִּיחוּ }  \hspace{0.3cm}
	\selectlanguage{english}
	
	
	\subsection{Translation of Sentences}
	
	Translate the following sentences from the Hebrew Bible. Names of persons and geographical names in these sentences: \foreignlanguage{hebrew}{אֲבִימֶלֶךְ}, \foreignlanguage{hebrew}{אַבְרָהָם}, \foreignlanguage{hebrew}{אַבְשָׁלוֺם}, \foreignlanguage{hebrew}{אַמְנוֺן}, \foreignlanguage{hebrew}{דָּוִד}, \foreignlanguage{hebrew}{חֶבְרוֺן}, \foreignlanguage{hebrew}{חָפְנִי}, \foreignlanguage{hebrew}{יֹאשִׁיָּהוּ}, \foreignlanguage{hebrew}{יְהוֺאָחָז}, \foreignlanguage{hebrew}{יְהוּדָה}, \foreignlanguage{hebrew}{יָרָבְעָם}, \foreignlanguage{hebrew}{יְרוּשָׁלִַם}, \foreignlanguage{hebrew}{מְגִדּוֺ}, \foreignlanguage{hebrew}{מֹשֶׁה}, \foreignlanguage{hebrew}{נְכֹה}, \foreignlanguage{hebrew}{פִּינְחָס}, \foreignlanguage{hebrew}{קַיִן}, \foreignlanguage{hebrew}{רְאוּבֶן}, \foreignlanguage{hebrew}{שָׁאוּל}, \foreignlanguage{hebrew}{שִׁישַׁק}, \foreignlanguage{hebrew}{שְׁלֹמֹה}, \foreignlanguage{hebrew}{שְׁמוּאֵל}
	
	\vspace{0.5cm}
	
	\selectlanguage{hebrew}
	\noindent 
	1~~\foreignlanguage{hebrew}{וַיָּבִ֜אוּ אֶת־אֲר֣וֹן יְהוָ֗ה וַיַּצִּ֤גוּ}\LTRfootnote{\space \foreignlanguage{hebrew}{יצג} Hi.\ \textit{to set, place}} \foreignlanguage{hebrew}{אֹתוֹ֙ בִּמְקוֹמ֔וֹ בְּת֣וֹךְ הָאֹ֔הֶל אֲשֶׁ֥ר נָטָה־ל֖וֹ דָּוִ֑ד וַיַּ֨עַל דָּוִ֥ד עֹל֛וֹת לִפְנֵ֥י יְהוָ֖ה וּשְׁלָמִֽים׃}\LTRfootnote{\space \foreignlanguage{hebrew}{שֶׁלֶם} a kind of offering: \textit{conclusion offering, salvation offering} (see \textit{HALOT} for details)} \hspace{0.3cm}
	2~~\foreignlanguage{hebrew}{וֽ͏ַיְהִ֖י מִקֵּ֣ץ יָמִ֑ים}\LTRfootnote{\space \foreignlanguage{hebrew}{מִקֵּץ יָמִים} \textit{after some time}, literally \textit{after the end of days}} \foreignlanguage{hebrew}{וַיָּבֵ֨א קַ֜יִן מִפְּרִ֧י הֽ͏ָאֲדָמָ֛ה מִנְחָ֖ה לַֽיהוָֽה׃} \hspace{0.3cm}
	3~~\foreignlanguage{hebrew}{וַיִּקְרָ֨א אֲבִימֶ֜לֶךְ לְאַבְרָהָ֗ם וַיֹּ֨אמֶר ל֜וֹ מֶֽה־עָשִׂ֤יתָ לָּ֙נוּ֙ וּמֶֽה־חָטָ֣אתִי לָ֔ךְ כִּֽי־הֵבֵ֧אתָ עָלַ֛י וְעַל־מַמְלַכְתִּ֖י חֲטָאָ֣ה}\LTRfootnote{\space \foreignlanguage{hebrew}{חֲטָאָה} \textit{sin, guilt}} \foreignlanguage{hebrew}{גְדֹלָ֑ה מַעֲשִׂים֙ אֲשֶׁ֣ר לֹא־יֵֽעָשׂ֔וּ עָשִׂ֖יתָ עִמָּדִֽי׃} \hspace{0.3cm}
	4~~\foreignlanguage{hebrew}{וַיֹּ֤אמֶר רְאוּבֵן֙ אֶל־אָבִ֣יו לֵאמֹ֔ר אֶת־שְׁנֵ֤י בָנַי֙ תָּמִ֔ית אִם־לֹ֥א אֲבִיאֶ֖נּוּ אֵלֶ֑יךָ תְּנָ֤ה אֹתוֹ֙ עַל־יָדִ֔י וַאֲנִ֖י אֲשִׁיבֶ֥נּוּ אֵלֶֽיךָ׃}  \hspace{0.3cm}
	5~~\foreignlanguage{hebrew}{כֹּ֚ה אָמַ֣ר יְהוָ֔ה הִנְנִ֨י מֵבִ֥יא רָעָ֛ה אֶל־הַמָּק֥וֹם הַזֶּ֖ה וְעַל־יֹֽשְׁבָ֑יו אֵ֚ת כָּל־דִּבְרֵ֣י הַסֵּ֔פֶר אֲשֶׁ֥ר קָרָ֖א מֶ֥לֶךְ יְהוּדָֽה׃}  \hspace{0.3cm}
	6~~\foreignlanguage{hebrew}{וַיְבַקֵּ֥שׁ שְׁלֹמֹ֖ה לְהָמִ֣ית אֶת־יָרָבְעָ֑ם וַיָּ֣קָם יָרָבְעָ֗ם וַיִּבְרַ֤ח מִצְרַ֙יִם֙ אֶל־שִׁישַׁ֣ק מֶֽלֶךְ־מִצְרַ֔יִם וַיְהִ֥י בְמִצְרַ֖יִם עַד־מ֥וֹת שְׁלֹמֹֽה׃}  \hspace{0.3cm}
	7~~\foreignlanguage{hebrew}{בְּיָמָ֡יו עָלָה֩ פַרְעֹ֨ה נְכֹ֧ה מֶֽלֶךְ־מִצְרַ֛יִם עַל־מֶ֥לֶךְ אַשּׁ֖וּר עַל־נְהַר־פְּרָ֑ת וַיֵּ֨לֶךְ הַמֶּ֤לֶךְ יֹאשִׁיָּ֙הוּ֙ לִקְרָאת֔וֹ וַיְמִיתֵ֙הוּ֙ בִּמְגִדּ֔וֹ כִּרְאֹת֖וֹ אֹתֽוֹ׃ וַיַּרְכִּבֻ֨הוּ עֲבָדָ֥יו מֵת֙ מִמְּגִדּ֔וֹ וַיְבִאֻ֙הוּ֙ יְר֣וּשָׁלִַ֔ם וַֽיִּקְבְּרֻ֖הוּ בִּקְבֻֽרָת֑וֹ}\LTRfootnote{\space \foreignlanguage{hebrew}{קְבֻרָה} \textit{grave, burial}} \foreignlanguage{hebrew}{וַיִּקַּ֣ח עַם־הָאָ֗רֶץ אֶת־יְהֽוֹאָחָז֙ בֶּן־יֹ֣אשִׁיָּ֔הוּ וַיִּמְשְׁח֥וּ אֹת֛וֹ וַיַּמְלִ֥יכוּ אֹת֖וֹ תַּ֥חַת אָבִֽיו׃} \hspace{0.3cm}
	8~~\foreignlanguage{hebrew}{וַיְצַו֩ אַבְשָׁל֨וֹם אֶת־נְעָרָ֜יו לֵאמֹ֗ר רְא֣וּ נָ֠א כְּט֨וֹב לֵב־אַמְנ֤וֹן בַּיַּ֙יִן֙ וְאָמַרְתִּ֣י אֲלֵיכֶ֔ם הַכּ֧וּ אֶת־אַמְנ֛וֹן וַהֲמִתֶּ֥ם אֹת֖וֹ אַל־תִּירָ֑אוּ הֲל֗וֹא כִּ֤י אָֽנֹכִי֙ צִוִּ֣יתִי אֶתְכֶ֔ם חִזְק֖וּ וִהְי֥וּ לִבְנֵי־חָֽיִל׃}  \hspace{0.3cm}
	9~~\foreignlanguage{hebrew}{מַכֵּ֥ה אִ֛ישׁ וָמֵ֖ת מ֥וֹת יוּמָֽת׃}  \hspace{0.3cm}
	10~~\foreignlanguage{hebrew}{לֹֽא־יוּמְת֤וּ אָבוֹת֙ עַל־בָּנִ֔ים וּבָנִ֖ים לֹא־יוּמְת֣וּ עַל־אָב֑וֹת אִ֥יש בְּחֶטְא֖וֹ}\LTRfootnote{\space \foreignlanguage{hebrew}{חֵטְא} \textit{ḥēṭ(ʾ)} \textit{sin, guilt}} \foreignlanguage{hebrew}{יוּמָֽתוּ׃} \hspace{0.3cm}
	11~~\foreignlanguage{hebrew}{וְזֶה־לְּךָ֣ הָא֗וֹת אֲשֶׁ֤ר יָבֹא֙ אֶל־שְׁנֵ֣י בָנֶ֔יךָ אֶל־חָפְנִ֖י וּפִֽינְחָ֑ס בְּי֥וֹם אֶחָ֖ד יָמ֥וּתוּ שְׁנֵיהֶֽם׃ וַהֲקִימֹתִ֥י לִי֙ כֹּהֵ֣ן נֶאֱמָ֔ן כַּאֲשֶׁ֛ר בִּלְבָבִ֥י וּבְנַפְשִׁ֖י יַעֲשֶׂ֑ה וּבָנִ֤יתִי לוֹ֙ בַּ֣יִת נֶאֱמָ֔ן וְהִתְהַלֵּ֥ךְ לִפְנֵֽי־מְשִׁיחִ֖י}\LTRfootnote{\space \foreignlanguage{hebrew}{מָשִׁיחַ} \textit{anointed}} \foreignlanguage{hebrew}{כָּל־הַיָּמִֽים׃} \hspace{0.3cm}
	12~~\foreignlanguage{hebrew}{וְהִנֵּ֨ה אָנֹכִ֜י עִמָּ֗ךְ וּשְׁמַרְתִּ֙יךָ֙ בְּכֹ֣ל אֲשֶׁר־תֵּלֵ֔ךְ וַהֲשִׁ֣בֹתִ֔יךָ אֶל־הָאֲדָמָ֖ה הַזֹּ֑את כִּ֚י לֹ֣א אֽ͏ֶעֱזָבְךָ֔ עַ֚ד אֲשֶׁ֣ר אִם־עָשִׂ֔יתִי אֵ֥ת אֲשֶׁר־דִּבַּ֖רְתִּי לָֽךְ׃}  \hspace{0.3cm}
	13~~\foreignlanguage{hebrew}{וַיֹּ֨אמֶר אֲלֵהֶ֣ם ׀ רְאוּבֵן֮ אַל־תִּשְׁפְּכוּ־דָם֒ הַשְׁלִ֣יכוּ אֹת֗וֹ אֶל־הַבּ֤וֹר הַזֶּה֙ אֲשֶׁ֣ר בַּמִּדְבָּ֔ר וְיָ֖ד אַל־תִּשְׁלְחוּ־ב֑וֹ לְמַ֗עַן הַצִּ֤יל אֹתוֹ֙ מִיָּדָ֔ם לַהֲשִׁיב֖וֹ אֶל־אָבִֽיו׃}  \hspace{0.3cm}
	14~~\foreignlanguage{hebrew}{וַיַּשְׁכִּ֤מוּ אַשְׁדּוֹדִים֙}\LTRfootnote{\space \foreignlanguage{hebrew}{אַשְׁדּוֺדִי} \textit{person from Ashdod} (one of the five cities of the Philistines)} \foreignlanguage{hebrew}{מִֽמָּחֳרָ֔ת}\LTRfootnote{\space \foreignlanguage{hebrew}{מִמָּחֳרָת} \textit{mimmɔḥɔ̆rāt} \textit{on the following day}} \foreignlanguage{hebrew}{וְהִנֵּ֣ה דָג֗וֹן}\LTRfootnote{\space \foreignlanguage{hebrew}{דָּגוֺן} Philistine deity} \foreignlanguage{hebrew}{נֹפֵ֤ל לְפָנָיו֙ אַ֔רְצָה לִפְנֵ֖י אֲר֣וֹן יְהוָ֑ה וַיִּקְחוּ֙ אֶת־דָּג֔וֹן וַיָּשִׁ֥בוּ אֹת֖וֹ לִמְקוֹמֽוֹ׃} \hspace{0.3cm}
	15~~\foreignlanguage{hebrew}{וַיִּקַּ֛ח יְהוָ֥ה אֱלֹהִ֖ים אֶת־הָֽאָדָ֑ם וַיַּנִּחֵ֣הוּ בְגַן־עֵ֔דֶן}\LTRfootnote{\space \foreignlanguage{hebrew}{גַּן־עֵדֶן} \textit{Garden of Eden}} \foreignlanguage{hebrew}{לְעָבְדָ֖הּ וּלְשָׁמְרָֽהּ׃} \hspace{0.3cm}
	16~~\foreignlanguage{hebrew}{וַיְדַבֵּ֨ר שְׁמוּאֵ֜ל אֶל־הָעָ֗ם אֵ֚ת מִשְׁפַּ֣ט הַמְּלֻכָ֔ה}\LTRfootnote{\space \foreignlanguage{hebrew}{מְלוּכָה} \textit{kingship, kingdom}} \foreignlanguage{hebrew}{וַיִּכְתֹּ֣ב בַּסֵּ֔פֶר וַיַּנַּ֖ח לִפְנֵ֣י יְהוָ֑ה וַיְשַׁלַּ֧ח שְׁמוּאֵ֛ל אֶת־כָּל־הָעָ֖ם אִ֥ישׁ לְבֵיתֽוֹ׃} \hspace{0.3cm}
	17~~\foreignlanguage{hebrew}{זָכוֹר֙}\LTRfootnote{\space \foreignlanguage{hebrew}{זָכוֺר} here used as an imperative} \foreignlanguage{hebrew}{אֶת־הַדָּבָ֔ר אֲשֶׁ֨ר צִוָּ֥ה אֶתְכֶ֛ם מֹשֶׁ֥ה עֶֽבֶד־יְהוָ֖ה לֵאמֹ֑ר יְהוָ֤ה אֱלֹהֵיכֶם֙ מֵנִ֣יחַ לָכֶ֔ם וְנָתַ֥ן לָכֶ֖ם אֶת־הָאָ֥רֶץ הַזֹּֽאת׃} \hspace{0.3cm}
	18~~\foreignlanguage{hebrew}{וַיִּשְׁכַּ֥ב דָּוִ֖ד עִם־אֲבֹתָ֑יו וַיִּקָּבֵ֖ר בְּעִ֥יר דָּוִֽד׃ וְהַיָּמִ֗ים אֲשֶׁ֨ר מָלַ֤ךְ דָּוִד֙ עַל־יִשְׂרָאֵ֔ל אַרְבָּעִ֖ים שָׁנָ֑ה בְּחֶבְר֤וֹן מָלַךְ֙ שֶׁ֣בַע שָׁנִ֔ים וּבִירוּשָׁלַ֣͏ִם מָלַ֔ךְ שְׁלֹשִׁ֥ים וְשָׁלֹ֖שׁ שָׁנִֽים׃ וּשְׁלֹמֹ֕ה יָשַׁ֕ב עַל־כִּסֵּ֖א דָּוִ֣ד אָבִ֑יו וַתִּכֹּ֥ן מַלְכֻת֖וֹ מְאֹֽד׃} \hspace{0.3cm}
	19~~\foreignlanguage{hebrew}{וַיֹּ֣אמֶר שְׁמוּאֵ֗ל אֶל־כָּל־בֵּ֣ית יִשְׂרָאֵל֮ לֵאמֹר֒ אִם־בְּכָל־לְבַבְכֶ֗ם אַתֶּ֤ם שָׁבִים֙ אֶל־יְהוָ֔ה הָסִ֜ירוּ אֶת־אֱלֹהֵ֧י הַנֵּכָ֛ר}\LTRfootnote{\space \foreignlanguage{hebrew}{נֵכָר} \textit{foreignness}} \foreignlanguage{hebrew}{מִתּוֹכְכֶ֖ם וְהָעַשְׁתָּר֑וֹת}\LTRfootnote{\space \foreignlanguage{hebrew}{עַשְׁתֹּרֶת} \textit{Astarte} (name of a female deity); pl. \foreignlanguage{hebrew}{עַשְׁתָּרוֹת}} \foreignlanguage{hebrew}{וְהָכִ֨ינוּ לְבַבְכֶ֤ם אֶל־יְהוָה֙ וְעִבְדֻ֣הוּ לְבַדּ֔וֹ וְיַצֵּ֥ל אֶתְכֶ֖ם מִיַּ֥ד פְּלִשְׁתִּֽים׃ וַיָּסִ֙ירוּ֙ בְּנֵ֣י יִשְׂרָאֵ֔ל אֶת־הַבְּעָלִ֖ים וְאֶת־הָעַשְׁתָּרֹ֑ת וַיַּעַבְד֥וּ אֶת־יְהוָ֖ה לְבַדּֽוֹ׃} \hspace{0.3cm}
	20~~\foreignlanguage{hebrew}{כִּ֣י ׀ יִמְלְא֣וּ יָמֶ֗יךָ וְשָֽׁכַבְתָּ֙ אֶת־אֲבֹתֶ֔יךָ וַהֲקִימֹתִ֤י אֶֽת־זַרְעֲךָ֙ אַחֲרֶ֔יךָ אֲשֶׁ֥ר יֵצֵ֖א מִמֵּעֶ֑יךָ}\LTRfootnote{\space \foreignlanguage{hebrew}{מֵעִים}* \textit{internal organs, inward parts }(intestines, bowels), \textit{belly}, here as source of procreation (BDB)} \foreignlanguage{hebrew}{וַהֲכִינֹתִ֖י אֶת־מַמְלַכְתּֽוֹ׃ ה֥וּא יִבְנֶה־בַּ֖יִת לִשְׁמִ֑י וְכֹנַנְתִּ֛י אֶת־כִּסֵּ֥א מַמְלַכְתּ֖וֹ עַד־עוֹלָֽם׃ אֲנִי֙ אֶהְיֶה־לּ֣וֹ לְאָ֔ב וְה֖וּא יִהְיֶה־לִּ֣י לְבֵ֑ן אֲשֶׁר֙ בְּהַ֣עֲוֺת֔וֹ}\LTRfootnote{\space \foreignlanguage{hebrew}{עוה} Hi. \textit{to go astray}} \foreignlanguage{hebrew}{וְהֹֽכַחְתִּיו֙ בְּשֵׁ֣בֶט אֲנָשִׁ֔ים וּבְנִגְעֵ֖י בְּנֵ֥י אָדָֽם׃ וְחַסְדִּ֖י לֹא־יָס֣וּר מִמֶּ֑נּוּ כַּאֲשֶׁ֤ר הֲסִרֹ֙תִי֙ מֵעִ֣ם שָׁא֔וּל אֲשֶׁ֥ר הֲסִרֹ֖תִי מִלְּפָנֶֽיךָ׃ וְנֶאְמַ֨ן בֵּיתְךָ֧ וּמַֽמְלַכְתְּךָ֛ עַד־עוֹלָ֖ם לְפָנֶ֑יךָ כִּֽסְאֲךָ֔ יִהְיֶ֥ה נָכ֖וֹן עַד־עוֹלָֽם׃} \hspace{0.3cm}
	\selectlanguage{english}
	
	
	\section{Hebrew Reading: 1 Kings 19:1--8}
	Translate 1\,Kings 19:1--8 with the help of notes below the text.
	
	\vspace{0.5cm}
	
	\selectlanguage{hebrew}
	\noindent
	\textsuperscript{1}~\foreignlanguage{hebrew}{וַיַּגֵּ֤ד אַחְאָב֙ לְאִיזֶ֔בֶל אֵ֛ת כָּל־אֲשֶׁ֥ר עָשָׂ֖ה אֵלִיָּ֑הוּ וְאֵ֨ת כָּל־אֲשֶׁ֥ר הָרַ֛ג אֶת־כָּל־הַנְּבִיאִ֖ים בֶּחָֽרֶב׃} \hspace{0.3cm}
	\textsuperscript{2}~\foreignlanguage{hebrew}{וַתִּשְׁלַ֤ח אִיזֶ֙בֶל֙ מַלְאָ֔ךְ אֶל־אֵלִיָּ֖הוּ לֵאמֹ֑ר כֹּֽה־יַעֲשׂ֤וּן אֱלֹהִים֙ וְכֹ֣ה יוֹסִפ֔וּן כִּֽי־כָעֵ֤ת מָחָר֙ אָשִׂ֣ים אֶֽת־נַפְשְׁךָ֔ כְּנֶ֖פֶשׁ אַחַ֥ד מֵהֶֽם׃} \hspace{0.3cm}
	\textsuperscript{3}~\foreignlanguage{hebrew}{וַיַּ֗רְא וַיָּ֙קָם֙ וַיֵּ֣לֶךְ אֶל־נַפְשׁ֔וֹ וַיָּבֹ֕א בְּאֵ֥ר שֶׁ֖בַע אֲשֶׁ֣ר לִֽיהוּדָ֑ה וַיַּנַּ֥ח אֶֽת־נַעֲר֖וֹ שָֽׁם׃} \hspace{0.3cm}
	\textsuperscript{4}~\foreignlanguage{hebrew}{וְהֽוּא־הָלַ֤ךְ בַּמִּדְבָּר֙ דֶּ֣רֶךְ י֔וֹם וַיָּבֹ֕א וַיֵּ֕שֶׁב תַּ֖חַת רֹ֣תֶם אֶחָ֑ד וַיִּשְׁאַ֤ל אֶת־נַפְשׁוֹ֙ לָמ֔וּת וַיֹּ֣אמֶר ׀ רַ֗ב עַתָּ֤ה יְהוָה֙ קַ֣ח נַפְשִׁ֔י כִּֽי־לֹא־ט֥וֹב אָנֹכִ֖י מֵאֲבֹתָֽי׃} \hspace{0.3cm}
	\textsuperscript{5}~\foreignlanguage{hebrew}{וַיִּשְׁכַּב֙ וַיִּישַׁ֔ן תַּ֖חַת רֹ֣תֶם אֶחָ֑ד וְהִנֵּֽה־זֶ֤ה מַלְאָךְ֙ נֹגֵ֣עַ בּ֔וֹ וַיֹּ֥אמֶר ל֖וֹ ק֥וּם אֱכֽוֹל׃} \hspace{0.3cm}
	\textsuperscript{6}~\foreignlanguage{hebrew}{וַיַּבֵּ֕ט וְהִנֵּ֧ה מְרַאֲשֹׁתָ֛יו עֻגַ֥ת רְצָפִ֖ים וְצַפַּ֣חַת מָ֑יִם וַיֹּ֣אכַל וַיֵּ֔שְׁתְּ וַיָּ֖שָׁב וַיִּשְׁכָּֽב׃} \hspace{0.3cm}
	\textsuperscript{7}~\foreignlanguage{hebrew}{וַיָּשָׁב֩ מַלְאַ֨ךְ יְהוָ֤ה ׀ שֵׁנִית֙ וַיִּגַּע־בּ֔וֹ וַיֹּ֖אמֶר ק֣וּם אֱכֹ֑ל כִּ֛י רַ֥ב מִמְּךָ֖ הַדָּֽרֶךְ׃} \hspace{0.3cm}
	\textsuperscript{8}~\foreignlanguage{hebrew}{וַיָּ֖קָם וַיֹּ֣אכַל וַיִּשְׁתֶּ֑ה וַיֵּ֜לֶךְ בְּכֹ֣חַ ׀ הָאֲכִילָ֣ה הַהִ֗יא אַרְבָּעִ֥ים יוֹם֙ וְאַרְבָּעִ֣ים לַ֔יְלָה עַ֛ד הַ֥ר הָאֱלֹהִ֖ים חֹרֵֽב׃} \hspace{0.3cm}
	\selectlanguage{english}
	
	\vspace{0.25cm}
	
	\hspace*{-0.5cm}\begin{longtable}{p{0.075\linewidth} p{0.15\linewidth}p{0.675\linewidth}}
		19:3 & \foreignlanguage{hebrew}{בְּאֵר שֶׁבַע} & name of a place in the Negev (cf. Gen 21:31; 26:33) \\
		19:4 & \foreignlanguage{hebrew}{רֹתֶם} & \textit{broom-plant} (\textit{Retama raetam}) (m.\ or f.) \\
		& \foreignlanguage{hebrew}{רַב} & here \textit{enough!} \\
		19:5 & \foreignlanguage{hebrew}{ישׁן} & Q.\ \textit{to sleep} \\
		& \foreignlanguage{hebrew}{וְהִנֵּה־זֶה} & \textit{and look, there!} \\ % Gesenius, 18th ed., p. 295a
		19:6 & \foreignlanguage{hebrew}{מְרַאֲשֹׁתָיו} & as a local adverbial phrase \textit{at his head-place} (of someone lying down) (BDB) from the noun \foreignlanguage{hebrew}{מֽרַאֲשׁוֺת}* \\
		& \foreignlanguage{hebrew}{עֻגָה} & \textit{round flat loaf} (of bread, which is quickly baked in ashes or on glowing baking stones; \textit{HALOT}) \\
		& \foreignlanguage{hebrew}{רֶצֶף} & \textit{glowing coal} \\ % HALOT
		& \foreignlanguage{hebrew}{צַפַּחַת} & \textit{pitcher} \\ % HALOT
		19:8 & \foreignlanguage{hebrew}{אֲכִילָה} & \textit{meal, food} \\ % BDB and HALOT
	\end{longtable}
	
	
	\chapter{Chapter 25}
	
	\renewcommand\arraystretch{1.4}
	
	\section{Vocabulary}
	
	\subsection{Verbs}
	
	
	% For the centering of the separation between the two columns see the documentation of the array package, page 2 
	
	\begin{longtable}{>{\raggedleft}p{0.175\linewidth} p{0.75\linewidth}}
		\foreignlanguage{hebrew}{ברא} & Q.\ \textit{to create} \\
		\foreignlanguage{hebrew}{חלל} & Hi.\ \textit{to begin}; Pi.\ \textit{to profane; to put into use} \\ % HALOT
		\foreignlanguage{hebrew}{חתת} & Ni.\ \textit{to be dismayed; to be terrified}; Q.\ \textit{to be shattered; to be filled with terror}  \\
		\foreignlanguage{hebrew}{יעץ} & Q.\ \textit{to advice, counsel}; Ni.\ \textit{to consult together; to exchange council} \\ % BDB
		\foreignlanguage{hebrew}{פעל} & Q.\ \textit{to do; to make} \\
		\foreignlanguage{hebrew}{פרר} & Hi.\ \textit{to break, violate; to frustrate, make ineffectual} \\ % 55 times; 44 times Hi.
		\foreignlanguage{hebrew}{רחק} & Q.\ \textit{to be far, distant}; Hi.\ \textit{to remove; to depart, withdraw, distance oneself} \\ % Qal BDB, Hi. HALOT
		\foreignlanguage{hebrew}{רעע} & Q.\ \textit{to be bad, evil, displeasing}; Hi. \textit{to do evil; to treat badly} \\ % HALOT
		\foreignlanguage{hebrew}{שׁמם} & Q.\ \textit{to be uninhabited, be deserted; to shudder, be appalled}; Ni.\ \textit{to be made uninhabited, deserted; to be made to tremble} \\ % HALOT
	\end{longtable}
	
	
	\subsection{Nouns}
	
	\begin{longtable}{>{\raggedleft}p{0.175\linewidth} p{0.75\linewidth}}
		\foreignlanguage{hebrew}{אַלְמָנָה} & \textit{widow} \\
		\foreignlanguage{hebrew}{דְּבַשׁ} & \textit{honey} \\
		\foreignlanguage{hebrew}{חֵץ} & \textit{arrow} (gem. noun; pl.\ \foreignlanguage{hebrew}{חִצִּים}) \\
		\foreignlanguage{hebrew}{מַחֲשָׁבָה} & \textit{thought; device, plan, purpose} \\ % BDB
		\foreignlanguage{hebrew}{עָמָל} & \textit{trouble; labor; toil} \\ % BDB
		\foreignlanguage{hebrew}{עֶרְוָה} & \textit{nakedness; genital area} \\
		\foreignlanguage{hebrew}{צֵל} & \textit{shadow, shade} (gem. noun; with ePP \foreignlanguage{hebrew}{צִלְּךָ}) \\
		\foreignlanguage{hebrew}{קָטֹן} & \textit{small; young; unimportant} (adj.; cs.\ st.\ \foreignlanguage{hebrew}{קְטֹן}; only in m.\ sg., other forms from \foreignlanguage{hebrew}{קָטָן}) \\ % HALOT
		\foreignlanguage{hebrew}{קָטָן} & \textit{small; young; unimportant} (adj.; pl.\ \foreignlanguage{hebrew}{קְטַנִּים}) \\ % HALOT
		\foreignlanguage{hebrew}{קָצִיר} & \textit{harvest; harvest crops} \\
		\foreignlanguage{hebrew}{שָׁוְא} & \textit{emptiness; nothingness} \\ % BDB (vanity not included)
		\foreignlanguage{hebrew}{שְׁמָמָה} & \textit{desolation; devastation; waste} \\ % DCH
	\end{longtable}
	
	
	%\subsection{Other Parts of Speech}
	%
	%\begin{longtable}{>{\raggedleft}p{0.175\linewidth} p{0.75\linewidth}}
	%		\foreignlanguage{hebrew}{בַּל} & \textit{not} (negative used in poetry) \\
	%\end{longtable}
	
	\section{Verbs II Geminate: Derived Binyanim}
	
	\subsection{Verbs II Geminate: Niphal, Hiphil and Hophal}
	
	The Niphal forms are characterized by an open prefix syllable in the suffix conjugation and the participle. In the prefix conjugation, the imperative and infinitives the /n/ is assimilated to the first root consonant, e.g., \textit{yinsab > yissaḇ} \foreignlanguage{hebrew}{יִסַּב}.
	
	In the Hiphil, many forms are characterized by open prefix syllables. In prefix conjugation forms, the prefix syllable may be open. In this case, the second root consonant is geminated when when suffixes are attached. But the first root consonant may be doubled in prefix conjugation forms, as well (so-called aramaizing forms).
	
	Hophal forms follow the pattern of I\,\textit{y} verbs in order to preserve the original short vowel /u/ in the prefix syllable: \textit{*husab > hūsaḇ} \foreignlanguage{hebrew}{הוּסַב}.
	
	As in the Qal, vocalic suffixes are not stressed. A connecting vowel is inserted between the geminated second root consonant and a consonantal suffix. The connecting vowel bears the stress.
	
	When suffixes are added, the gemination of the second root consonant is preserved.
	
	
	\vspace{0.25cm}
	
	\noindent The table only contains selected forms because of the relative infrequency of II~gem.\ verbs in the derived binyanim.
	
	% II gem. verbs (not exhaustive):
	% Niphal
	% Verb בזז Ni. 3 times
	% Verb חתת Ni. 28 times
	% Verb מדד Ni. 3 times
	% Verb מסס Ni. 19 times (for the infinitives cf. 2 Sam 17:	10; Ps 68.3)
	% Verb סבב Ni. 20 times
	% Verb קלל Ni. 11 times
	% The inf. abs. \foreignlanguage{hebrew}{הִבּוֺז} (Isa 24:3) has been ignored (cf. GKC §67t)
	%
	% Hiphil
	% Verb  Hi.
	% Verb חלל Hi. 52 times
	% Verb סבב Hi. 32 times
	% Verb עזז Hi. 2 times
	% Verb פרר Hi. 40 times
	% Verb צרר Hi. 12 times
	% Verb קלל Hi. 13 times
	% Verb קרר Hi. 2 times
	% Verb רעע Hi. 68 times
	% Verb שׁמם Hi. 17 times
	% Verb תלל Hi. 7 times
	% Verb תמם Hi. 8 times
	% Aramaizing forms with חלל סבב תמם
	%
	% Hophal
	% Verb חלל Ho. 1 time
	% Verb פרר Ho. 3 times
	% Verb שמם Ho. 4 times
	% Verb תלל Ho. 1 time
	
	\begin{center}
		\begin{longtable}{|lll|r|r|r|}
			\hline
			& & & \multicolumn{1}{c|}{Niphal} & \multicolumn{1}{c|}{Hiphil} & \multicolumn{1}{c|}{Hophal} \\
			\hline
			SC & sg. & 3 m. & \foreignlanguage{hebrew}{נָסַב} & \foreignlanguage{hebrew}{הֵסֵב} & \foreignlanguage{hebrew}{הוּסַב} \\
			& & 3 f. & \foreignlanguage{hebrew}{נָסַ֫בָּה}  & \foreignlanguage{hebrew}{} & \\
			& & 2 m. &  & \foreignlanguage{hebrew}{הֲסִבֹּ֫תָ} & \\
			& & 1 c. & \foreignlanguage{hebrew}{נְסַבֹּתִי} & \foreignlanguage{hebrew}{הֲסִבּ֫וֺתִי} &  \\
			& pl. & 3 c. & \foreignlanguage{hebrew}{נָסַ֫בּוּ} & \foreignlanguage{hebrew}{הֵסַ֫בּוּ} &  \\
			\hline
			PC & sg. & 3 m. & \foreignlanguage{hebrew}{יִסַּב} & \foreignlanguage{hebrew}{יַסֵּב}/\foreignlanguage{hebrew}{יָסֵב} & \foreignlanguage{hebrew}{יוּסַב} \\
			& pl. & 3 m. & \foreignlanguage{hebrew}{יִסַּ֫בּוּ} & \foreignlanguage{hebrew}{יַסֵּ֫בּוּ}/\foreignlanguage{hebrew}{יָסֵ֫בּוּ} & \\
			& & 3 f. & & \foreignlanguage{hebrew}{תְּסִבֶּינָה} & \\
			\hline
			Juss. & sg. & 3 m. & & \foreignlanguage{hebrew}{יָסֵב} & \foreignlanguage{hebrew}{} \\
			\textit{waw}-PC & sg. & 3 m. & & \foreignlanguage{hebrew}{וַיַּסֵּב}/\foreignlanguage{hebrew}{וַיָּ֫סֶב} & \foreignlanguage{hebrew}{וַיּוּסַב}  \\
			\hline
			Impv. & sg. & m. & \foreignlanguage{hebrew}{} & \foreignlanguage{hebrew}{הָסֵב} &  \\
			& pl. & m. & & \foreignlanguage{hebrew}{הָסַ֫בּוּ} &  \\
			\hline
			Inf.\ cs.\ & & & \foreignlanguage{hebrew}{הִסֵּב} & \foreignlanguage{hebrew}{הָסֵב} & \\
			Inf.\ abs.\ & & & \foreignlanguage{hebrew}{הִסֵּב} & \foreignlanguage{hebrew}{} & \\
			\hline
			\pagebreak
			\hline
			Part. & sg. & m. & \foreignlanguage{hebrew}{נָסַב} & \foreignlanguage{hebrew}{מֵסֵב} &  \\
			& & f. & \foreignlanguage{hebrew}{נְסַבָּה} & \foreignlanguage{hebrew}{} &  \\
			& pl. & m. & \foreignlanguage{hebrew}{נְסַבִּים} & \foreignlanguage{hebrew}{} &  \\
			& & f. & \foreignlanguage{hebrew}{נְסַבּוֺת} & \foreignlanguage{hebrew}{} &  \\
			\hline
		\end{longtable}
	\end{center}
	
	
	\noindent \textbf{Notes}
	\nopagebreak
	
	\noindent In the suffix conjugation Niphal, the thematic vowel between the first and second root consonant may be /ō/ or /ē/ instead of /a/, e.g., \foreignlanguage{hebrew}{וְנָבֹ֫זּוּ} \textit{and they will be plundered}, \foreignlanguage{hebrew}{נָמֵס} \textit{it is melted} (but as a pausal form \foreignlanguage{hebrew}{נָמָ֑ס}).
	
	The I gutt.\ and II gem.\ verb \foreignlanguage{hebrew}{חתת} has Niphal prefix conjugation forms with a lengthened prefix vowel as result of the loss of the gemination, e.g., \foreignlanguage{hebrew}{יֵחַת} (\textit{*yinḥat > *yiḥḥat > yēḥat}). In the suffix conjugation, the 3 m.\ sg.\ form \foreignlanguage{hebrew}{נִחַת} is attested once.
	
	In the Hiphil the thematic vowel of the 3 f.\ sg.\ and 3 c.\ pl.\ forms may be either /a/ or /ē/, e.g., \foreignlanguage{hebrew}{הֵסַ֫בּוּ} \textit{they brought round} (1\,Sam 5:9--10), \foreignlanguage{hebrew}{הֵחֵ֫לָּה} \textit{she/it began} (Judg 20:40), \foreignlanguage{hebrew}{הֵחֵּ֫לּוּ} \textit{they began} (1\,Sam 3:2).
	
	With a guttural or /r/ as second root consonant the thematic vowel may be lengthened when suffixes are added, e.g., \foreignlanguage{hebrew}{וַהֲצֵרוֹתִי} \textit{and I will cause distress} (Jer 10:18), \foreignlanguage{hebrew}{הֲרֵעֹתֶם} \textit{you have done something bad} (Gen 43:6).
	
	% The form \foreignlanguage{hebrew}{יִשַּׁחוּ} in Qoh 12:4 is not included because of the uncertainties around it (Qal aramaizing form?)
	
	Forms without a connecting vowel between the second root consonant and the consonantal suffix are also found, e.g., \foreignlanguage{hebrew}{וְהֵפַרְתָּ֫ה} \textit{and you shall make ineffectual} (2\,Sam 15:34). In this form the gemination of the second root consonant is lost.
	
	Aramaizing forms with a geminated first root occur a number of times, e.g., \foreignlanguage{hebrew}{וַיַּסֵּב} \textit{and he caused to go around} (Josh 6:11), \foreignlanguage{hebrew}{וַיַּסֵּבּוּ} \textit{and they brought} (1\,Sam 5:8), \foreignlanguage{hebrew}{יַתֵּם} \textit{let him count out the entire amount} (2\,Kgs 22:4), \foreignlanguage{hebrew}{יַחֵל} (with loss of the gemination of the /ḥ/ and preservation of the short vowel) \textit{let him [not] make invalid} (Num 30:3).
	
	In rare cases, the gemination of the second root consonant is missing, e.g., \foreignlanguage{hebrew}{וְנָבְקָה} \textit{and it shall be disturbed} (3 fem.\ sg.\ \textit{waw}-SC Ni.\ from the verb \foreignlanguage{hebrew}{בקק}; Jes 19:3), \foreignlanguage{hebrew}{הֵעֵ֫זָה} \textit{she made [her face] bold} (3 fem.\ sg.\ SC Hi.\ from the verb \foreignlanguage{hebrew}{עזז}; Prov 7:13).
	
	Contamination with forms of II\,\textit{w/y} verbs may happen occasionally, e.g., \foreignlanguage{hebrew}{הִבּוֹז תִּבּוֹז} (verb \foreignlanguage{hebrew}{עזז} Ni.) \textit{[the earth] will be plundered} (Isa 24:3), \foreignlanguage{hebrew}{יַשִּׁים} (\foreignlanguage{hebrew}{שׁמם} Hi.) \textit{he will devastate} (Jer 49:20). Other deviating forms are attested as well.
	
	
	\subsection{Verbs II Geminate: Polel, Polal and Hitpolel}
	
	Verbs II gem.\ are used in the Piel, Pual and Hitpael relatively frequently. Like II\,\textit{w/y} verbs, they are used in the Polel, Polal and Hitpolel, binyanim, which are characterized by a long unchangeable vowel /ō/ after the first root consonant and reduplication of the second root consonant. These binyanim correspond with the Piel, Pual and Hitpael in terms of their meaning.
	
	The following table contains characteristic forms. The verb \foreignlanguage{hebrew}{עלל} Polel \textit{to act arbitrarily (toward)} is used, because it occurs in all three binyanim Polel, Polal and Hitpolel. Forms of other verbs supplement this verb, when important forms are missing.
	
	\vspace{0.5cm}
	
	\begin{center}
		\begin{longtable}{|lll|r|r|r|}
			\hline
			\multicolumn{3}{|c|}{} & \multicolumn{1}{c|}{Polel} & \multicolumn{1}{c|}{Polal} & \multicolumn{1}{c|}{Hitpolel} \\
			\hline
			SC & sg. & 3 m. & \foreignlanguage{hebrew}{עוֹלֵל} & \foreignlanguage{hebrew}{עוֹלַל} & \\
			& & 3 f. & & & \foreignlanguage{hebrew}{הִתְהֹלָ֑לוּ} \\
			& & 2 m. & \foreignlanguage{hebrew}{עוֹלַ֫לְתָּ} &  & \\
			& pl. & 3 c. & \foreignlanguage{hebrew}{עֹרְרוּ} & & \foreignlanguage{hebrew}{הִתְהֹלְלוּ} \\
			%			 & & 2 m. & & & \\
			%			 & & 2 f. & & &  \\
			%			 & & 1 c. & & & \\
			\hline
			PC & sg. & 3 m. & \foreignlanguage{hebrew}{יְשֹׁדֵד} & & \foreignlanguage{hebrew}{יִתְגֹּדַד} \\
			%			 & & 3 f. & \foreignlanguage{hebrew}{} & & \\
			%			 & & 2 m. & \foreignlanguage{hebrew}{} & & \\
			%			 & & 2 f. & \foreignlanguage{hebrew}{} & & \\
			%			 & & 1 c. & \foreignlanguage{hebrew}{} & & \\
			& pl. & 3 c. & \foreignlanguage{hebrew}{יְעוֹלְלוּ} & \foreignlanguage{hebrew}{יְרוֹפָ֑פוּ} & \foreignlanguage{hebrew}{יִתְהֹלְלוּ} \\
			%			 & & 2 f. & \foreignlanguage{hebrew}{} & & \\
			%			 & & 2 m. & \foreignlanguage{hebrew}{} & & \\
			%			 & & 2 f. & \foreignlanguage{hebrew}{} & & \\
			%			 & & 1 c. & \foreignlanguage{hebrew}{} & & \\
			\hline
			Impv. & sg. & 2 m. & \foreignlanguage{hebrew}{עוֹלֵל} & & \\
			%			 & & 2 f. & \foreignlanguage{hebrew}{}* \\
			& pl. & 2 m. & & & \foreignlanguage{hebrew}{הִתְקוֹשְׁשׁוּ} \\
			%			 & & 2 f. & \foreignlanguage{hebrew}{} \\
			\hline
			Inf.\ cs. & & & \foreignlanguage{hebrew}{קֹשֵׁשׁ} & & \foreignlanguage{hebrew}{הִתְעוֹלֵל} \\
			Inf.\ abs. & & & \foreignlanguage{hebrew}{עוֹלֵל} & & \\
			\hline
			Part. & sg. & m. & \foreignlanguage{hebrew}{מְקֹשֵׁשׁ	} & \foreignlanguage{hebrew}{מְהוֹלָל} & \foreignlanguage{hebrew}{מִתְגֹּלֵל} \\
			& pl. & m. & \foreignlanguage{hebrew}{מְחֹקְקִים} & & \foreignlanguage{hebrew}{מִתְגֹּדְדִים} \\
			\hline
		\end{longtable}
	\end{center}
	
	\noindent \textbf{Notes}
	\nopagebreak
	
	\noindent The immediate context helps to distinguish between ambiguous forms of the Polel (active) and Polal (passive).
	
	Participles have a /m/ \foreignlanguage{hebrew}{מְ} as prefix consonant.
	
	In pausal forms of the Hitpolel /ā/ is found instead of /ē/, e.g., \foreignlanguage{hebrew}{תִּתְגּוֹדָֽדִי} \textit{you cut yourself} (Jer 47:5). % RSV/ESV gash for cut
	
	\section{The Marker of Subordination  \foreignlanguage{hebrew}{שֶׁ ּ}}
	
	The marker of subordination \foreignlanguage{hebrew}{שֶׁ ּ} is placed in front of the following word with a \textit{dageš forte} in the first consonant of this word. At times, one finds \textit{pataḥ} instead of \textit{segol}, i.e., \foreignlanguage{hebrew}{שַׁ ּ}. Before /r/ and /ʾ/ the vowel is changed to /ā/ and the \textit{dageš forte} is lost leading to \foreignlanguage{hebrew}{שָׁ}
	
	The particle \foreignlanguage{hebrew}{שֶׁ ּ} functions similarly to the much more frequent \foreignlanguage{hebrew}{אֲשֶׁר}. Important functions are:
	
	\begin{itemize}[noitemsep]
		\item[--] marking relative clauses, e.g., \foreignlanguage{hebrew}{כַּחוֹל שֶׁעַל־שְׂפַת הַיָּם} \textit{as the sand which is upon the shore of the sea} (Judg 7:12; cf. \foreignlanguage{hebrew}{וְכַחוֹל אֲשֶׁר עַל־שְׂפַת הַיָּם} in Gen 22:17)
		\item[--] as an alternative construction for a construct chain or an enclitic personal pronoun, e.g., \foreignlanguage{hebrew}{כַּרְמִי שֶׁלִּי} \textit{my vineyard} (Song 1:6)
		\item[--] introducing an object clause, e.g., \foreignlanguage{hebrew}{וְעָשִׂיתָ לִּי אוֹת שָׁאַתָּה מְדַבֵּר עִמִּי} \textit{and give me a sign that it is you who speak with me} (Judg 6:17); \foreignlanguage{hebrew}{וְיָדַעְתִּי גַם־אָנִי שֶׁמִּקְרֶה אֶחָד יִקְרֶה אֶת־כֻּלָּם} \textit{Yet I know that one fate will befall them all} (Qoh 2:14)
		\item[--] introducing a causal subordinate clause, e.g., \foreignlanguage{hebrew}{פִּתְחִי־לִי אֲחֹתִי רַעְיָתִי יוֹנָתִי תַמָּתִי שֶׁרֹּאשִׁי נִמְלָא־טָל} \textit{Open to me, my sister, my beloved, my dove, my perfect one, for my head is filled with dew} (Song 5:2)
		\item[--] introducing a final subordinate clause, e.g., \foreignlanguage{hebrew}{וְהָאֱלֹהִים עָשָׂה שֶׁיִּרְאוּ מִלְּפָנָיו} \textit{and God has done this so that one would fear him} (Qoh 3:14)
		\item[--] with prepositions in various functions
	\end{itemize} 
	
	The subordination marker \foreignlanguage{hebrew}{שֶׁ ּ} occurs 138 times in the Hebrew Bible and is mainly used in Late Biblical Hebrew or in a few earlier texts of northern origin. It is especially frequent in Song of Songs and Qohelet.
	
	\section{Asyndetic Relative Clauses}
	
	Relative clauses are usually introduced with the subordination marker \foreignlanguage{hebrew}{אֲשֶׁר} or much less frequently \foreignlanguage{hebrew}{שֶׁ ּ}. They are then called \emph{syndetic} relative clauses. It is possible, however, to attach a relative clause directly to the antecedent without a subordination marker. Both verbal clauses (Jer 13:20; Deut 32:17; Exod 18:20; Jer 5:15) and verbless clauses (Isa 51:7) may function as \emph{asyndetic} relative clauses.
	
	\vspace{0.5cm}
	
	\begin{longtable}{>{\raggedleft}p{0.35\linewidth} p{0.55\linewidth}}
		\foreignlanguage{hebrew}{אַיֵּה הָעֵדֶר נִתַּן־לָךְ} & \textit{Where is the flock that was given to you?} (Jer 13:20) \\
		\foreignlanguage{hebrew}{וְהוֹדַעְתָּ לָהֶם אֶת־הַדֶּרֶךְ יֵלְכוּ בָהּ וְאֶת־הַמַּעֲשֶׂה אֲשֶׁר יַעֲשׂוּן} & \textit{\dots \space and you shall make known to them the way in which they must walk and what they must do} (Exod 18:20; followed by a noun with a syndetic relative clause) \\
		\foreignlanguage{hebrew}{יִזְבְּחוּ לַשֵּׁדִים לֹא אֱלֹהַ אֱלֹהִים לֹא יְדָעוּם} & \textit{They sacrificed to demons that were no god, [to] gods they did not know} (Deut 32:17) \\
		\foreignlanguage{hebrew}{גּוֹי לֹא־תֵדַע לְשֹׁנוֹ} & \textit{a people whose language you do not know} (Jer 5:15) \\
		\foreignlanguage{hebrew}{עַם תּוֹרָתִי בְלִבָּם} & \textit{the people in whose heart is my instruction} (Isa 51:7) \\
	\end{longtable}
	
	
	
	\section{Exercises}
	
	\subsection{Translation of Verbal Forms}
	
	Translate the following verbal forms. Identify the gender (masc., fem., comm.) and number (sg., pl.) of forms of which the English translation is ambiguous (i.e., \textit{you}, \textit{they}). Mark the stressed syllable if stress is not on the last syllable.
	
	\hspace{0.5cm}
	
	\selectlanguage{hebrew}
	
	\noindent
	1~~\foreignlanguage{hebrew}{הֵחֵלּוּ}  \hspace{0.3cm}
	2~~\foreignlanguage{hebrew}{הַחִלֹּתִי}  \hspace{0.3cm}
	3~~\foreignlanguage{hebrew}{וַיָּחֵלּוּ}  \hspace{0.3cm}
	4~~\foreignlanguage{hebrew}{וַיָּחֶל}  \hspace{0.3cm}
	5~~\foreignlanguage{hebrew}{יֵחַתּוּ}  \hspace{0.3cm}
	6~~\foreignlanguage{hebrew}{תֵּחַתּוּ}  \hspace{0.3cm}
	7~~\foreignlanguage{hebrew}{הֵפַר}  \hspace{0.3cm}
	8~~\foreignlanguage{hebrew}{הֵפַרְתָּה}  \hspace{0.3cm}
	9~~\foreignlanguage{hebrew}{הֵפֵרוּ}  \hspace{0.3cm}
	10~~\foreignlanguage{hebrew}{תָּפֵרוּ}  \hspace{0.3cm}
	11~~\foreignlanguage{hebrew}{הֲרֵעֹתֶם}  \hspace{0.3cm}
	12~~\foreignlanguage{hebrew}{הֲרֵעֹתָה}  \hspace{0.3cm}
	13~~\foreignlanguage{hebrew}{וַיָּרֵעוּ}  \hspace{0.3cm}
	14~~\foreignlanguage{hebrew}{מְרֵעִים}  \hspace{0.3cm}
	15~~\foreignlanguage{hebrew}{‎יִשֹּׁם}  \hspace{0.3cm}
	16~~\foreignlanguage{hebrew}{שָׁמְמוּ}  \hspace{0.3cm}
	17~~\foreignlanguage{hebrew}{‎בָּרָאתִי}  \hspace{0.3cm}
	18~~\foreignlanguage{hebrew}{‎תְּטַמְּאוּ}  \hspace{0.3cm}
	19~~\foreignlanguage{hebrew}{‎יִוָּעֲצוּ}  \hspace{0.3cm}
	20~~\foreignlanguage{hebrew}{‎נוֹעֲצוּ}  \hspace{0.3cm}
	\selectlanguage{english}
	
	
	\subsection{Translation of Sentences}
	
	Translate the following sentences from the Hebrew Bible. Names of persons and geographical names in these sentences: \foreignlanguage{hebrew}{אַבְרָהָם}, \foreignlanguage{hebrew}{אַבְשָׁלוֺם}, \foreignlanguage{hebrew}{אֲחִיתֹפֶל}, \foreignlanguage{hebrew}{גִּלְעָד}, \foreignlanguage{hebrew}{חוּשַׁי}, \foreignlanguage{hebrew}{יְהוֹשֻׁעַ}, \foreignlanguage{hebrew}{יוֺסֵף}, \foreignlanguage{hebrew}{מֹשֶׁה}, \foreignlanguage{hebrew}{עַי} (almost always with the article), \foreignlanguage{hebrew}{עַמּוֺן}, \foreignlanguage{hebrew}{עָמְרִי}, \foreignlanguage{hebrew}{צָדוֹק}
	
	
	\vspace{0.5cm}
	
	\selectlanguage{hebrew}
	\noindent 
	1~~\foreignlanguage{hebrew}{וַיֹּ֤אמֶר יְהוָה֙ אֶל־יְהוֹשֻׁ֔עַ הַיּ֣וֹם הַזֶּ֗ה אָחֵל֙ גַּדֶּלְךָ֔ בְּעֵינֵ֖י כָּל־יִשְׂרָאֵ֑ל אֲשֶׁר֙ יֵֽדְע֔וּן כִּ֗י כַּאֲשֶׁ֥ר הָיִ֛יתִי עִ‏ם־מֹשֶׁ֖ה אֶהְיֶ֥ה עִמָּֽךְ׃}  \hspace{0.3cm}
	2~~\foreignlanguage{hebrew}{‏וַיֹּאמְר֨וּ הָעָ֜ם שָׂרֵ֤י גִלְעָד֙ אִ֣ישׁ אֶל־רֵעֵ֔הוּ מִ֣י הָאִ֔ישׁ אֲשֶׁ֣ר יָחֵ֔ל לְהִלָּחֵ֖ם בִּבְנֵ֣י עַמּ֑וֹן יִֽהְיֶ֣ה לְרֹ֔אשׁ לְכֹ֖ל יֹשְׁבֵ֥י גִלְעָֽד׃}  \hspace{0.3cm}
	3~~\foreignlanguage{hebrew}{אֲדֹנָ֣י יְהוִ֗ה אַתָּ֤ה הַֽחִלּ֨וֹתָ֙ לְהַרְא֣וֹת אֶֽת־עַבְדְּךָ֔ אֶ֨ת־גָּדְלְךָ֔}\LTRfootnote{\space \foreignlanguage{hebrew}{גֹּדֶל} \textit{greatness}} \foreignlanguage{hebrew}{וְאֶת־יָדְךָ֖ הַחֲזָקָ֑ה אֲשֶׁ֤ר מִי־אֵל֙ בַּשָּׁמַ֣יִם וּבָאָ֔רֶץ אֲשֶׁר־יַעֲשֶׂ֥ה כְמַעֲשֶׂ֖יךָ וְכִגְבוּרֹתֶֽךָ׃} \hspace{0.3cm}
	4~~\foreignlanguage{hebrew}{‏וַתְּחִלֶּ֜ינָה שֶׁ֣בַע שְׁנֵ֤י הָרָעָב֙ לָב֔וֹא כַּאֲשֶׁ֖ר אָמַ֣ר יוֹסֵ֑ף וַיְהִ֤י רָעָב֙ בְּכָל־הָ֣אֲרָצ֔וֹת וּבְכָל־אֶ֥רֶץ מִצְרַ֖יִם הָ֥יָה לָֽחֶם׃}  \hspace{0.3cm}
	5~~\foreignlanguage{hebrew}{רְאֵה נָתַ֨ן יְהוָ֧ה אֱלֹהֶ֛יךָ לְפָנֶ֖יךָ אֶת־הָאָ֑רֶץ עֲלֵ֣ה רֵ֗שׁ כַּאֲשֶׁר֩ דִּבֶּ֨ר יְהוָ֜ה אֱלֹהֵ֤י אֲבֹתֶ֙יךָ֙ לָ֔ךְ אַל־תִּירָ֖א וְאַל־תֵּחָֽת׃}  \hspace{0.3cm}
	6~~\foreignlanguage{hebrew}{וַיֹּ֨אמֶר יְהוָ֤ה אֶל־יְהוֹשֻׁ֙עַ֙ אַל־תִּירָ֣א וְאַל־תֵּחָ֔ת קַ֣ח עִמְּךָ֗ אֵ֚ת כָּל־עַ֣ם הַמִּלְחָמָ֔ה וְק֖וּם עֲלֵ֣ה הָעָ֑י רְאֵ֣ה ׀ נָתַ֣תִּי בְיָדְךָ֗ אֶת־מֶ֤לֶךְ הָעַי֙ וְאֶת־עַמּ֔וֹ וְאֶת־עִיר֖וֹ וְאֶת־אַרְצֽוֹ׃}  \hspace{0.3cm}
	7~~\foreignlanguage{hebrew}{וַיֹּ֥אמֶר חוּשַׁ֖י אֶל־אַבְשָׁל֑וֹם לֹֽא־טוֹבָ֧ה הָעֵצָ֛ה אֲשֶׁר־יָעַ֥ץ אֲחִיתֹ֖פֶל בַּפַּ֥עַם הַזֹּֽאת׃}  \hspace{0.3cm}
	8~~\foreignlanguage{hebrew}{ וַיֹּ֤אמֶר אַבְשָׁלוֹם֙ וְכָל־אִ֣ישׁ יִשְׂרָאֵ֔ל טוֹבָ֗ה עֲצַת֙ חוּשַׁ֣י הָאַרְכִּ֔י}\LTRfootnote{\space \foreignlanguage{hebrew}{אַרְכִּי} \textit{Archite} (gentilic noun)} \foreignlanguage{hebrew}{מֵעֲצַ֖ת אֲחִיתֹ֑פֶל וַיהוָ֣ה צִוָּ֗ה לְהָפֵ֞ר אֶת־עֲצַ֤ת אֲחִיתֹ֨פֶל֙ הַטּוֹבָ֔ה לְבַעֲב֗וּר הָבִ֧יא יְהוָ֛ה אֶל־אַבְשָׁל֖וֹם אֶת־הָרָעָֽה׃} \hspace{0.3cm}
	9~~\foreignlanguage{hebrew}{וַיֹּ֨אמֶר יְהוָ֜ה אֶל־מֹשֶׁ֗ה הֵ֣ן קָרְב֣וּ יָמֶיךָ֮ לָמוּת֒ קְרָ֣א אֶת־יְהוֹשֻׁ֗עַ וְהִֽתְיַצְּב֛וּ בְּאֹ֥הֶל מוֹעֵ֖ד וַאֲצַוֶּ֑נּוּ וַיֵּ֤לֶךְ מֹשֶׁה֙ וִֽיהוֹשֻׁ֔עַ וַיִּֽתְיַצְּב֖וּ בְּאֹ֥הֶל מוֹעֵֽד׃ וַיֵּרָ֧א יְהוָ֛ה בָּאֹ֖הֶל בְּעַמּ֣וּד עָנָ֑ן וַיַּעֲמֹ֛ד עַמּ֥וּד הֶעָנָ֖ן עַל־פֶּ֥תַח הָאֹֽהֶל׃ וַיֹּ֤אמֶר יְהוָה֙ אֶל־מֹשֶׁ֔ה הִנְּךָ֥ שֹׁכֵ֖ב עִם־אֲבֹתֶ֑יךָ וְקָם֩ הָעָ֨ם הַזֶּ֜ה וְזָנָ֣ה ׀ אַחֲרֵ֣י ׀ אֱלֹהֵ֣י נֵֽכַר}\LTRfootnote{\space \foreignlanguage{hebrew}{נֵכָר} \textit{foreignness; foreign}} \foreignlanguage{hebrew}{־הָאָ֗רֶץ אֲשֶׁ֨ר ה֤וּא בָא־שָׁ֙מָּה֙ בְּקִרְבּ֔וֹ וַעֲזָבַ֕נִי וְהֵפֵר֙ אֶת־בְּרִיתִ֔י אֲשֶׁ֥ר כָּרַ֖תִּי אִתּֽוֹ׃ וְחָרָ֣ה אַפִּ֣י ב֣וֹ בַיּוֹם־הַ֠הוּא וַעֲזַבְתִּ֞ים וְהִסְתַּרְתִּ֨י פָנַ֤י מֵהֶם֙ וְהָיָ֣ה לֶֽאֱכֹ֔ל וּמְצָאֻ֛הוּ רָע֥וֹת רַבּ֖וֹת וְצָר֑וֹת וְאָמַר֙ בַּיּ֣וֹם הַה֔וּא הֲלֹ֗א עַ֣ל כִּֽי־אֵ֤ין אֱלֹהַי֙ בְּקִרְבִּ֔י מְצָא֖וּנִי הָרָע֥וֹת הָאֵֽלֶּה׃ וְאָנֹכִ֗י הַסְתֵּ֨ר אַסְתִּ֤יר פָּנַי֙ בַּיּ֣וֹם הַה֔וּא עַ֥ל כָּל־הָרָעָ֖ה אֲשֶׁ֣ר עָשָׂ֑ה כִּ֣י פָנָ֔ה אֶל־אֱלֹהִ֖ים אֲחֵרִֽים׃} \hspace{0.3cm}
	10~~\foreignlanguage{hebrew}{וַיַּעֲשֶׂ֥ה עָמְרִ֛י הָרַ֖ע בְּעֵינֵ֣י יְהוָ֑ה וַיָּ֕רַע מִכֹּ֖ל אֲשֶׁ֥ר לְפָנָֽיו׃}  \hspace{0.3cm}
	11~~\foreignlanguage{hebrew}{וַיֹּ֤אמֶר הַמֶּ֙לֶךְ֙ לְצָד֔וֹק הָשֵׁ֛ב אֶת־אֲר֥וֹן הָאֱלֹהִ֖ים הָעִ֑יר אִם־אֶמְצָ֥א חֵן֙ בְּעֵינֵ֣י יְהוָ֔ה וֶהֱשִׁבַ֕נִי וְהִרְאַ֥נִי אֹת֖וֹ וְאֶת־נָוֵֽהוּ׃}\LTRfootnote{\space \foreignlanguage{hebrew}{נָוֶה} \textit{grazing place; stopping place, settlement} (\textit{HALOT})} \hspace{0.3cm}
	12~~\foreignlanguage{hebrew}{וַיֹּ֤אמֶר אֵלָיו֙ הָעֶ֔בֶד אוּלַי֙ לֹא־תֹאבֶ֣ה הָֽאִשָּׁ֔ה לָלֶ֥כֶת אַחֲרַ֖י אֶל־הָאָ֣רֶץ הַזֹּ֑את הֶֽהָשֵׁ֤ב אָשִׁיב֙ אֶת־בִּנְךָ֔ אֶל־הָאָ֖רֶץ אֲשֶׁר־יָצָ֥אתָ מִשָּֽׁם׃ וַיֹּ֥אמֶר אֵלָ֖יו אַבְרָהָ֑ם הִשָּׁ֣מֶר}\LTRfootnote{\space \foreignlanguage{hebrew}{שׁמר} Ni.\ \textit{to be on one's guard}} \foreignlanguage{hebrew}{לְךָ֔ פֶּן־תָּשִׁ֥יב אֶת־בְּנִ֖י שָֽׁמָּה׃} \hspace{0.3cm}
	\selectlanguage{english}
	
	
	
	\section{Hebrew Reading: 1 Kings 19:9--14}
	Translate 1\,Kings 19:9--14 with the help of notes below the text.
	
	\vspace{0.5cm}
	
	\selectlanguage{hebrew}
	\noindent
	\textsuperscript{9}~\foreignlanguage{hebrew}{וַיָּבֹא־שָׁ֥ם אֶל־הַמְּעָרָ֖ה וַיָּ֣לֶן שָׁ֑ם וְהִנֵּ֤ה דְבַר־יְהוָה֙ אֵלָ֔יו וַיֹּ֣אמֶר ל֔וֹ מַה־לְּךָ֥ פֹ֖ה אֵלִיָּֽהוּ׃}
	\textsuperscript{10}~\foreignlanguage{hebrew}{וַיֹּאמֶר֩ קַנֹּ֨א קִנֵּ֜אתִי לַיהוָ֣ה ׀ אֱלֹהֵ֣י צְבָא֗וֹת כִּֽי־עָזְב֤וּ בְרִֽיתְךָ֙ בְּנֵ֣י יִשְׂרָאֵ֔ל אֶת־מִזְבְּחֹתֶ֣יךָ הָרָ֔סוּ וְאֶת־נְבִיאֶ֖יךָ הָרְג֣וּ בֶחָ֑רֶב וֽ͏ָאִוָּתֵ֤ר אֲנִי֙ לְבַדִּ֔י וַיְבַקְשׁ֥וּ אֶת־נַפְשִׁ֖י לְקַחְתָּֽהּ׃} \hspace{0.3cm}
	\textsuperscript{11}~\foreignlanguage{hebrew}{וַיֹּ֗אמֶר צֵ֣א וְעָמַדְתָּ֣ בָהָר֮ לִפְנֵ֣י יְהוָה֒ וְהִנֵּ֧ה יְהוָ֣ה עֹבֵ֗ר וְר֣וּחַ גְּדוֹלָ֡ה וְחָזָ֞ק מְפָרֵק֩ הָרִ֨ים וּמְשַׁבֵּ֤ר סְלָעִים֙ לִפְנֵ֣י יְהוָ֔ה לֹ֥א בָר֖וּחַ יְהוָ֑ה וְאַחַ֤ר הָר֨וּחַ רַ֔עַשׁ לֹ֥א בָרַ֖עַשׁ יְהוָֽה׃} \hspace{0.3cm}
	\textsuperscript{12}~\foreignlanguage{hebrew}{וְאַחַ֤ר הָרַ֙עַשׁ֙ אֵ֔שׁ לֹ֥א בָאֵ֖שׁ יְהוָ֑ה וְאַחַ֣ר הָאֵ֔שׁ ק֖וֹל דְּמָמָ֥ה דַקָּֽה׃} \hspace{0.3cm}
	\textsuperscript{13}~\foreignlanguage{hebrew}{וַיְהִ֣י ׀ כִּשְׁמֹ֣עַ אֵלִיָּ֗הוּ וַיָּ֤לֶט פָּנָיו֙ בְּאַדַּרְתּ֔וֹ וַיֵּצֵ֕א וַֽיַּעֲמֹ֖ד פֶּ֣תַח הַמְּעָרָ֑ה וְהִנֵּ֤ה אֵלָיו֙ ק֔וֹל וַיֹּ֕אמֶר מַה־לְּךָ֥ פֹ֖ה אֵלִיָּֽהוּ׃} \hspace{0.3cm}
	\textsuperscript{14}~\foreignlanguage{hebrew}{וַיֹּאמֶר֩ קַנֹּ֨א קִנֵּ֜אתִי לַיהוָ֣ה ׀ אֱלֹהֵ֣י צְבָא֗וֹת כִּֽי־עָזְב֤וּ בְרִֽיתְךָ֙ בְּנֵ֣י יִשְׂרָאֵ֔ל אֶת־מִזְבְּחֹתֶ֣יךָ הָרָ֔סוּ וְאֶת־נְבִיאֶ֖יךָ הָרְג֣וּ בֶחָ֑רֶב וָאִוָּתֵ֤ר אֲנִי֙ לְבַדִּ֔י וַיְבַקְשׁ֥וּ אֶת־נַפְשִׁ֖י לְקַחְתָּֽהּ׃} \hspace{0.3cm}
	\selectlanguage{english}
	
	\vspace{0.25cm}
	
	\hspace*{-0.5cm}\begin{longtable}{p{0.075\linewidth} p{0.15\linewidth}p{0.675\linewidth}}
		19:9 & \foreignlanguage{hebrew}{מְעָרָה} & \textit{cave} \\
		19:10 & \foreignlanguage{hebrew}{קנא} & Pi.\ \textit{to be jealous; to be zealous} \\
		& \foreignlanguage{hebrew}{הרס} & Q.\ \textit{to tear down; to throw down} \\
		19:11 & \foreignlanguage{hebrew}{פרק} & Pi.\ \textit{to tear off; to tear out} \\
		& \foreignlanguage{hebrew}{שׁבר} & Pi.\ \textit{to shatter; to break} \\
		& \foreignlanguage{hebrew}{רַעַשׁ} & \textit{earthquake} \\
		19:13 & \foreignlanguage{hebrew}{דְּמָמָה} & \textit{calm} (cessation of strong movement of air) (\textit{HALOT}) \\
		& \foreignlanguage{hebrew}{דַּק} & \textit{thin; scarce; fine}; here: \textit{small, soft} \\
		& \foreignlanguage{hebrew}{לוט} & Hi.\ \textit{to wrap} \\
		& \foreignlanguage{hebrew}{אַדֶּרֶת} & \textit{cloak, robe} (fem.) \\
		& \foreignlanguage{hebrew}{מְעָרָה} & \textit{cave} \\
		19:14 & \foreignlanguage{hebrew}{קנא} & Pi.\ \textit{to be jealous; to be zealous} \\
		& \foreignlanguage{hebrew}{הרס} & Q.\ \textit{to tear down; to throw down} \\
	\end{longtable}
	
	
	\chapter{Chapter 26}
	
	\renewcommand\arraystretch{1.4}
	
	\section{Vocabulary}
	
	\subsection{Verbs}
	
	% For the centering of the separation between the two columns see the documentation of the array package, page 2 
	
	\begin{longtable}{>{\raggedleft}p{0.175\linewidth} p{0.75\linewidth}}
		\foreignlanguage{hebrew}{חוה} & Hištaphel \textit{to bow down, prostrate; to worship} \\
		\foreignlanguage{hebrew}{ירה}\textsubscript{3} & Hi.\ \textit{to instruct, teach} (cf. \foreignlanguage{hebrew}{תּוֺרָה}) \\
		\foreignlanguage{hebrew}{נדח} & Hi.\ \textit{to thrust; thrust out; thrust away} \\ % BDB
	\end{longtable}
	
	
	\subsection{Nouns}
	
	\begin{longtable}{>{\raggedleft}p{0.175\linewidth} p{0.75\linewidth}}
		\foreignlanguage{hebrew}{אַחֲרוֹן} & \textit{last, latter} (temp.); \textit{at the back; western} (local) (adj.) \\
		\foreignlanguage{hebrew}{אֱנוֺשׁ} & \textit{mankind, (all) human beings, man; (some) men; single human being} (mostly poet.) \\
		\foreignlanguage{hebrew}{אַרְיֵה}/\foreignlanguage{hebrew}{אֲרִי} & \textit{lion} \\
		\foreignlanguage{hebrew}{בְּתוּלָה} & \textit{virgin} \\ % HALOT
		\foreignlanguage{hebrew}{גַּיְא} & \textit{valley} (\textit{gay(ʾ)}; pl.\ \foreignlanguage{hebrew}{גֵּאָיוֺת}) \\
		\foreignlanguage{hebrew}{מְדִינָה} & \textit{province} \\
		\foreignlanguage{hebrew}{מוּסָר} & \textit{discipline; training; exhortation, warning} \\ % HALOT
		\foreignlanguage{hebrew}{עֶלְיוֺן} & \textit{highest, most high} \\
		\foreignlanguage{hebrew}{פֶּסַח} & \textit{Passover, Pesach} \\
		\foreignlanguage{hebrew}{רֵאשִׁית} & \textit{beginning; first, chief} \\ % BDB
		\foreignlanguage{hebrew}{רְחֹב} & \textit{broad open place, plaza} \\
	\end{longtable}
	
	\subsection{Other Parts of Speech}
	
	\begin{longtable}{>{\raggedleft}p{0.175\linewidth} p{0.75\linewidth}}
		\foreignlanguage{hebrew}{הוֹי} & \textit{ah! alas! ha!} (interjection expressing usually dissatisfaction and pain) \\
	\end{longtable}
	
	
	\section{The Verb \foreignlanguage{hebrew}{חוה} Hištaphel}
	
	The III\,\textit{y} verb \foreignlanguage{hebrew}{חוה} \textit{to bow down, prostrate; to worship} occurs about 170 times in the Hebrew Bible. It is only used in the Hištaphel (or more precisely Hištap̄ʿel), a binyan that is characterized by a causative prefix consonant /š/, the consonant /t/ which marks reflexivity and, in the suffix conjugation, the imperative and the infinitive construct, a prosthetic /h/. At the same time, \foreignlanguage{hebrew}{חוה} is the only verb used in the Hištaphel.
	
	The following forms can be mentioned as examples:
	
	\begin{longtable}{lllr}
		SC & sg. & 3 m. & \foreignlanguage{hebrew}{הִשְׁתַּחֲוָה} \\
		& & 2 m. & \foreignlanguage{hebrew}{הִשְׁתַּחֲוִיתָ} \\
		& & 1 c. & \foreignlanguage{hebrew}{הִשְׁתַּחֲוֵיתִי} \\
		& pl. & 3 c. & \foreignlanguage{hebrew}{הִשְׁתַּחֲווּ} \\
		& & 2 m. & \foreignlanguage{hebrew}{וְהִשְׁתַּחֲוִיתֶם} \\
		PC & sg. & 3 m. & \foreignlanguage{hebrew}{יִשְׁתַּחֲוֶה} \\
		& & 2 m. & \foreignlanguage{hebrew}{תִּשְׁתַּחֲוֶה} \\
		& & 1 c. & \foreignlanguage{hebrew}{אֶשְׁתַּחֲוֶה} \\
		& pl. & 2 m. & \foreignlanguage{hebrew}{תִּשְׁתַּחֲווּ} \\
		& & 1 c. & \foreignlanguage{hebrew}{נִשְׁתַּחֲוֶה} \\
		\textit{waw}-PC & sg. & 3 m. & \foreignlanguage{hebrew}{וַיִּשְׁתַּ֫חוּ} \\
		& pl. & 3 m. & \foreignlanguage{hebrew}{וַיִּשְׁתַּחֲווּ} \\
		Impv. & pl. & m. & \foreignlanguage{hebrew}{הִשְׁתַּחֲווּ} \\
		Inf.\ cs. & & & \foreignlanguage{hebrew}{הִשְׁתַּחֲוֹת} \\
		Part. & sg. & m. & \foreignlanguage{hebrew}{מִשְׁתַּחֲוֶה} \\
		& pl. & m. & \foreignlanguage{hebrew}{מִשְׁתַּחֲוִים} \\
		
	\end{longtable}
	
	\noindent \textbf{Notes}
	\nopagebreak
	
	\noindent The verb shows the morphological features of III\,\textit{y} verbs.
	
	In the \textit{wayyiqṭol}-form without suffixes (3 m./f.\ sg.) the apocopate forms have a long vowel /ū/ at the end: \textit{*wayyištaḥw > wayyištaḥuw > wayyištáḥū} \foreignlanguage{hebrew}{וַיִּשְׁתַּ֫חוּ}. In pausal forms the \textit{pataḥ} is lengthened to \textit{qameṣ}, e.g., \foreignlanguage{hebrew}{וַתִּשְׁתָּ֫חוּ}.
	
	Traditionally, these forms were considered Hitpaʿlel forms of a verb \foreignlanguage{hebrew}{שׁחה} with metathesis of the /t/ and the /š/ and reduplication of the third element of the root, first as a semivowel /w/ and then as a vocalic element as usual with III\,\textit{y} verbs.\footnote{\space Cf.\ GKC §\,75\,\textit{kk}; BDB 1005a.}
	
	Since the Ugaritic language, a Central Semitic language from the period of 1400--1200 BCE that is similar to Hebrew, was deciphered in 1930, most scholars analyze these forms as Hištap̄ʿel of a root \foreignlanguage{hebrew}{חוה}, because in Ugaritic a Hištap̄ʿel binyan including forms of a verb \textit{ḥwy} with similar forms and identical meaning as the Hebrew verb \foreignlanguage{hebrew}{חוה} is clearly attested, e.g., \textit{tštḥwy} \textit{she prostrates herself}.
	
	
	\section{The Qal Passive}
	
	Some verbal forms appear to be Pual or Hophal forms. It can be shown, however, that these forms are remnants of a Qal passive binyan which is the passive counterpart to the Qal in the same way as the Hophal and the Pual are the passive counterparts to the Hiphil and the Piel. The Qal passive was being phased out during the period of Biblical Hebrew but the existing forms in the biblical texts were preserved. The following
	arguments for the existence of the Qal passive can be mentioned:
	
	\begin{enumerate}[noitemsep]
		\item These alleged Pual or Hophal forms do not have active counterparts in the Piel or	Hiphil, respectively.
		\item These forms have a meaning that serves as the passive to the meaning of the verbs in question in the Qal.
		\item The verb \foreignlanguage{hebrew}{לקח} has suffix conjugation forms that are vocalized as Pual and prefix conjugation forms that look like Hophal forms, e.g., \foreignlanguage{hebrew}{לֻקַּח} (3 m.\ sg.\ SC), \foreignlanguage{hebrew}{יֻקַּח} (3 m.\ sg.\ PC; with assimilation of the /l/). Yet, the meaning of the verb is identical, namely passive to Qal forms: \textit{to be taken}. This situation would be unusual for how the binyanim function in Biblical Hebrew without the assumption of a Qal passive.
		\item Classical Arabic has passive forms in Stem I, the equivalent of the Qal binyan in Hebrew, e.g., \textit{fuʿila} \textit{it was done} as opposed to the active form \textit{faʿala} \textit{he did}. Internal passive forms are also attested in Old Aramaic and Biblical Aramaic.
	\end{enumerate}
	
	Qal passive forms are characterized by the vowel sequence \textit{u–a} which is characteristic of passive forms. The following examples
	are illustrative:
	
	\begin{itemize}[noitemsep]
		\item[--] suffix conjugation: \foreignlanguage{hebrew}{לֻקַּח} (3 m.\ sg.), \foreignlanguage{hebrew}{כֹּרָ֑תָה}, \foreignlanguage{hebrew}{לֻקֳחָה} (3 f.\ sg.), \foreignlanguage{hebrew}{יֻלְּדוּ} (3 c.\ pl.)
		\item[--] prefix conjugation: \foreignlanguage{hebrew}{יֻקּ͏ַח} (verb \foreignlanguage{hebrew}{לקח}), \foreignlanguage{hebrew}{יֻתּ͏ַן} (verb \foreignlanguage{hebrew}{נתן}), \foreignlanguage{hebrew}{יוּ͏שׁ͏ַת} (verb \foreignlanguage{hebrew}{שׁית}), \foreignlanguage{hebrew}{וַתּ͏ֻקּ͏ַח} (verb \foreignlanguage{hebrew}{לקח}; 3 f.\ sg.\ \textit{waw}-PC)
		\item[--] participle: \foreignlanguage{hebrew}{אֻכּ͏ָל} (m.\ sg.; Exod 3:2); \foreignlanguage{hebrew}{לֻקָּח} (2\,Kgs 2:10)
	\end{itemize}
	
	These forms show the characteristics of weak verbs like the assimilation of /n/ or /l/ as in the case of the verb \foreignlanguage{hebrew}{לקח}. The following illustrate the use of the Qal passive:
	
	\begin{longtable}{>{\raggedleft}p{0.35\linewidth} p{0.55\linewidth}}
		\foreignlanguage{hebrew}{וַיְשַׁלְּחֵהוּ יְהוָה אֱלֹהִים מִגַּן־עֵדֶן לַעֲבֹד אֶת־הָאֲדָמָה אֲשֶׁר לֻקַּח מִשָּׁם} & \textit{And the Lord God sent him away from the garden of Eden to work the ground from which he had been taken} (Gen 3:23) \\
		\foreignlanguage{hebrew}{לְזֹאת יִקָּרֵא אִשָּׁה כִּי מֵאִישׁ לֻקֳחָה־זֹּאת} & \textit{This one shall be called woman because from man she has been taken} (Gen 2:23b) \\
		\foreignlanguage{hebrew}{וַתֻּקַּח הָאִשָּׁה בֵּית פַּרְעֹה} & \textit{And the woman was taken into Pharaoh’s house} (Gen 12:15) \\
	\end{longtable}
	
	Qal passive forms were already recognized as such by Jewish grammarians during the Middle Ages. In the middle of the 19th century they were rediscovered as Qal passive forms by western scholars and not considered Pual of Hophal forms anymore. Modern lexica recognize Qal passive forms to different degrees. For the identification of Qal passive forms it is recommended to consult \textit{HALOT} or Wilhelm Gesenius, \textit{Hebräisches und aramäisches Handwörterbuch über das Alte Testament}, ed. H. Donner, 18th ed. (Berlin: Springer, 2013). BDB does \emph{not} recognize the Qal passive and therefore should not be consulted for this matter. Electronic versions of the Hebrew Bible with morphological tags recognize only a part of all (possible) Qal passive forms. According to the Westminster Hebrew Morphology tagging, the Qal passive is attested 87 times in the Hebrew Bible. The most frequent verbs in the Qal passive are \foreignlanguage{hebrew}{ילד} (27 times), \foreignlanguage{hebrew}{לקח} (15 times) and \foreignlanguage{hebrew}{נתן} (8 times).\footnote{\space Hebrew Masoretic Text with Westminster Hebrew Morphology (HMT-W4), Groves-Wheeler Westminster Hebrew Morphology, v.\ 4.22, version 3.1 in Accordance 14.0.11.}
	
	% The following section is based on Lettinga/von Siebenthal, § 42 S (384)
	
	\section{Rare Binyanim}
	
	The following rare binyanim in Biblical Hebrew need to be mentioned:
	
	\begin{itemize}[noitemsep]
		\item[--] Poʿel (with the passive Poʿal and the reflexive Hitpoʿel), e.g., \foreignlanguage{hebrew}{שֹׁרֵשׁ} \textit{it took root} (Isa 40:24)
		\item[--] Paʿlel (with the passive Puʿlal), e.g., \foreignlanguage{hebrew}{שַׁאֲנָן} \textit{he was at ease} (root \foreignlanguage{hebrew}{שׁאן}; Jer 48:11), passive \foreignlanguage{hebrew}{אֻמְלַל} \textit{it became feeble} (root \foreignlanguage{hebrew}{אמל}; Joel 1:10)
		\item[--] Pilpel (with the passive Polpal and the reflexive Hitpalpel), e.g., \foreignlanguage{hebrew}{כִּלְכַּל} \textit{he sustained} (with /a/ in the second syllable; 2\,Sam 19:33); \foreignlanguage{hebrew}{וְכָלְכְּלוּ} \textit{wəḵɔlkəlū} \textit{and they were supplied with food} (1\,Kgs 20:27)
		
	\end{itemize}
	
	The forms of these binyanim are not easily recognizable as verbal forms, unless they have the familiar verbal suffixes or prefixes. The lexicon will help to identify these forms as verbal forms.
	
	Pilpel forms occur 50 times in the Hebrew Bible. No less than 24 of these occurrences are from the verb \foreignlanguage{hebrew}{כול} Pilpel \textit{to sustain} (with food). Pilpel forms are easy to identify because of the reduplication of the two root consonants. The vowel structure of Pilpel forms is identical to Piel forms. The difference with Piel forms is that Pilpel forms have two different consonants where Piel forms have the geminated second root consonant. In the same way, the vowel structure of Polpal and Hitpalpel forms is identical to Hitpael and Pual forms, respectively. Pilpel forms are mainly derived from II gem.\ verbs and less frequently from II\,\textit{w/y} verbs, if roots are counted and not occurrences of Pilpel forms.\footnote{\space Alternatively, Pilpel forms can be derived from biconsonantal roots with reduplication of both root consonants. BDB and \textit{HALOT} do this only for the verb \foreignlanguage{hebrew}{טאטא}.}
	
	% The 18\textsuperscript{th} edition of Gesenius' Handwörterbuch (in German) does this relatively often.}

Characteristic Pilpel forms of the verb \foreignlanguage{hebrew}{כול} are \foreignlanguage{hebrew}{כִּלְכַּל} (SC), \foreignlanguage{hebrew}{יְכַלְכֵּל} (PC), \foreignlanguage{hebrew}{וַיְכַלְכֵּל} (\textit{waw}-PC), \foreignlanguage{hebrew}{וְכִלְכַּלְתִּי} (\textit{waw}-SC), \foreignlanguage{hebrew}{כַּלְכֵּל} (inf.\ cs.), \foreignlanguage{hebrew}{מְכַלְכֵּל}.

In terms of vowel structure, Pilpel forms are identical to the forms of quadriliteral verbs with four root consonants. In the Hebrew Bible only very few forms of quadriliteral verbs attested. One verb that is worth mentioning is the verb \foreignlanguage{hebrew}{תרגם}. In the Hebrew Bible, it is attested only once as a participle Polpal \foreignlanguage{hebrew}{מְתֻרְגָּם} \textit{translated} (Ezra 4:7). It is the normal word for \textit{to translate} from post-biblical Hebrew to Modern Israeli Hebrew (cf.\ the noun \foreignlanguage{hebrew}{תַּרְגוּם} \textit{translation; Targum}).

\section{Final and Consecutive Constructions}

Final and consecutive constructions indicate the purpose or the consequence or result of a state of affairs, respectively. They can be expressed by subordinate clauses, the use of indirect volitive forms and prepositions.

A frequently used conjunction introducing final subordinate clauses is (\foreignlanguage{hebrew}{אֲשֶׁר}) \foreignlanguage{hebrew}{לְמַעַן} (Deut 8:1; Josh 3:4). Less frequent are the conjunctions \foreignlanguage{hebrew}{בַּעֲבוּר} (Exod 19:9) and \foreignlanguage{hebrew}{אֲשֶׁר} (Deut 4:40). For negative final subordinate clauses the conjunction \foreignlanguage{hebrew}{פֶּן} is used (Exod 20:19).

\begin{longtable}{>{\raggedleft}p{0.35\linewidth} p{0.55\linewidth}}
	\foreignlanguage{hebrew}{כָּל־הַמִּצְוָה אֲשֶׁר אָנֹכִי מְצַוְּךָ הַיּוֹם תִּשְׁמְרוּן לַעֲשׂוֹת לְמַעַן תִּחְיוּן וּרְבִיתֶם וּבָאתֶם וִירִשְׁתֶּם אֶת־הָאָרֶץ אֲשֶׁר־נִשְׁבַּע יְהוָה לַאֲבֹתֵיכֶם׃} & \textit{You shall carefully keep all the instruction that I command you today, in order that you will live and become numerous and go in and take possession of the land that the Lord swore to your fathers} (Deut 8:1) \\
	\foreignlanguage{hebrew}{אַל־תִּקְרְבוּ אֵלָיו לְמַעַן אֲשֶׁר־תֵּדְעוּ אֶת־הַדֶּרֶךְ אֲשֶׁר תֵּלְכוּ־בָהּ} & \textit{Do not come close to [the ark], in order that you know the way which you shall go} (Josh 3:4) \\
	\foreignlanguage{hebrew}{וַיֹּאמֶר יְהוָה אֶל־מֹשֶׁה הִנֵּה אָנֹכִי בָּא אֵלֶיךָ בְּעַב הֶעָנָן בַּעֲבוּר יִשְׁמַע הָעָם בְּדַבְּרִי עִמָּךְ} & \textit{The Lord said to Moses, \enquote{Behold, I am coming to you in the thick cloud, in order that the people may hear when I speak with you}} (Exod 19:9) \\
	\foreignlanguage{hebrew}{וְשָׁמַרְתָּ אֶת־חֻקָּיו וְאֶת־מִצְוֹתָיו אֲשֶׁר אָנֹכִי מְצַוְּךָ הַיּוֹם אֲשֶׁר יִיטַב לְךָ וּלְבָנֶיךָ אַחֲרֶיךָ} & \textit{\dots \space and you shall keep his statutes and his commandments that I command you today, in order that it may be well with you and you children after you} (Deut 4:40) \\
	\foreignlanguage{hebrew}{וְאַל־יְדַבֵּר עִמָּנוּ אֱלֹהִים פֶּן־נָמוּת} & \textit{\dots \space but do not let God speak with with us, lest we die} (Exod 20:19) \\
\end{longtable}

Consecutive subordinate clauses are mostly introduced with \foreignlanguage{hebrew}{כִּי} (Exod 3:11) and only rarely with the conjunctions \foreignlanguage{hebrew}{לְמַעַן} (Jer 27:15) and \foreignlanguage{hebrew}{אֲשֶׁר} (Gen 13:16).

\begin{longtable}{>{\raggedleft}p{0.35\linewidth} p{0.55\linewidth}}
	\foreignlanguage{hebrew}{‎מִי אָנֹכִי כִּי אֵלֵךְ אֶל־פַּרְעֹה וְכִי אוֹצִיא אֶת־בְּנֵי יִשְׂרָאֵל מִמִּצְרָיִם} & \textit{Who am I that I should go to Pharaoh and that I should lead the Israelites out of Egypt} (Exod 3:11) \\
	\foreignlanguage{hebrew}{כִּי לֹא שְׁלַחְתִּים נְאֻם־יְהוָה וְהֵם נִבְּאִים בִּשְׁמִי לַשָּׁקֶר לְמַעַן הַדִּיחִי אֶתְכֶם} & \textit{For I have not sent them, and yet they prophesy in my name falsely, with the result that I will drive you out} (Jer 27:15) \\
	\foreignlanguage{hebrew}{וְשַׂמְתִּי אֶת־זַרְעֲךָ כַּעֲפַר הָאָרֶץ אֲשֶׁר אִם־יוּכַל אִישׁ לִמְנוֹת אֶת־עֲפַר הָאָרֶץ גַּם־זַרְעֲךָ יִמָּנֶה} & \textit{And I will make your descendants as the dust of the earth, so that if someone can count the dust of the earth, your descendants too can be counted} (Gen 13:16) \\
\end{longtable}

Two clauses that are coordinated with \foreignlanguage{hebrew}{וְ} with an indirect volitive mood in the second the clause are also frequently used to express a final or consecutive relation. In the second clause, a cohortative (Gen 27:4), an imperative (2\,Sam 21:3) or a third person jussive form (Judg 6:30) may be used.

% The first two examples are from Lettinga/von Siebenthal, §78 (715)

\begin{longtable}{>{\raggedleft}p{0.35\linewidth} p{0.55\linewidth}}
	\foreignlanguage{hebrew}{וַעֲשֵׂה־לִי מַטְעַמִּים כַּאֲשֶׁר אָהַבְתִּי וְהָבִיאָה לִּי וְאֹכֵלָה} & \textit{\dots \space and make for me tasty food as I like [it] and bring it to me so that I may eat [it]} (Gen 27:4) \\
	\foreignlanguage{hebrew}{מָה אֶעֱשֶׂה לָכֶם וּבַמָּה אֲכַפֵּר וּבָרְכוּ אֶת־נַחֲלַת יְהוָה} & \textit{What shall I do for you and with what shall I make atonement so that you may bless the heritage of the Lord} (2\,Sam 21:3) \\
	\foreignlanguage{hebrew}{הוֹצֵא אֶת־בִּנְךָ וְיָמֹת} & \textit{Bring out your son that he may die} (Judg 6:30) \\
\end{longtable}

As prepositions with final meaning can be mentioned \foreignlanguage{hebrew}{לְ} (Exod 16:15) and -- followed by the inf.\ cs.\ -- \foreignlanguage{hebrew}{לְמַעַן} (1\,Sam 17:28), \foreignlanguage{hebrew}{בַּעֲבוּר} (Exod 20:20) and with negative meaning \foreignlanguage{hebrew}{לְבִלְתִּי} (1\,Kgs 15:17).

\begin{longtable}{>{\raggedleft}p{0.35\linewidth} p{0.55\linewidth}}
	\foreignlanguage{hebrew}{הוּא הַלֶּחֶם אֲשֶׁר נָתַן יְהוָה לָכֶם לְאָכְלָה} & \textit{That is the bread that the Lord has given you to eat} (i.e., with the intention that you may eat it; Exod 16:15) \\
	\foreignlanguage{hebrew}{כִּי לְמַעַן רְאוֹת הַמִּלְחָמָה יָרָֽדְתָּ} & \textit{\dots \space for you came down to see the battle} (1\,Sam 17:28) \\
	\foreignlanguage{hebrew}{וַיֹּאמֶר מֹשֶׁה אֶל־הָעָם אַל־תִּירָאוּ כִּי לְבַעֲבוּר נַסּוֹת אֶתְכֶם בָּא הָאֱלֹהִים} & \textit{And Moses said to the people, \enquote{Do not be afraid, for God has come to test you}} (Exod 20:20) \\
	\foreignlanguage{hebrew}{וַיִּבֶן אֶת־הָרָמָה לְבִלְתִּי תֵּת יֹצֵא וָבָא לְאָסָא מֶלֶךְ יְהוּדָה} & \textit{And he fortified Ramah in order to not allow anyone belonging to Asa king of Judah to go out [for battle] or go in} (1\,Kgs 15:17) \\
\end{longtable}

The preposition \foreignlanguage{hebrew}{לְ} is also used with consecutive meaning (Ps 78:25).

\vspace{0.5cm}

\begin{tabular}{>{\raggedleft}p{0.35\linewidth} p{0.55\linewidth}}
	\foreignlanguage{hebrew}{צֵידָה שָׁלַח לָהֶם לָשֹׂבַע} & \textit{\dots \space he sent them provisions with satiety as result} (Ps 78:25) \\
\end{tabular}


\section{Concessive Subordinate Clauses}

% The definition of concessive is a paraphrase of Hadumod Bussmann, Routledge Dictionary of Language and Linguistics (London: Routledge, 1996), 93.

A concessive subordinate clause expresses a condition that even if it is fulfilled, does not lead to the result that is expressed in the main clause. In English, the conjunction \textit{although} among others introduces concessive subordinate clauses. In Biblical Hebrew, the most frequent concessive conjunctions are \foreignlanguage{hebrew}{כִּי} (Exod 13:17), \foreignlanguage{hebrew}{גַּם כִּי} (Ps 23:4a) and \foreignlanguage{hebrew}{גַּם} (Jer 36:25).

% The conjunction אם (Lettinga/von Siebenthal, Ges18, HALOT) (e.g., Num 22:18; Job 9:15; the best example is possibly Job 9:20) brings this too close to hypothetical conditional clauses. For this reason it was not included here.
% The conjunctions על and על אשר with concessive meaning (Lettinga/von Siebenthal, Ges18, BDB) is too infrequent to be included; Ges18 and BDB mention only Isa 53:9; Job 16:17.

\vspace{0.5cm}

\begin{longtable}{>{\raggedleft}p{0.35\linewidth} p{0.55\linewidth}}
	\foreignlanguage{hebrew}{וַיְהִי בְּשַׁלַּח פַּרְעֹה אֶת־הָעָם וְלֹא־נָחָם אֱלֹהִים דֶּרֶךְ אֶרֶץ פְּלִשְׁתִּים כִּי קָרוֹב הוּא} & \textit{When Pharaoh let the people go, God did not lead them by way of the land of the Philistines, although it was near} (Exod 13:17) \\
	\foreignlanguage{hebrew}{גַּם כִּי־אֵלֵךְ בְּגֵיא צַלְמָוֶת לֹא־אִירָא רָע} & \textit{Even though I walk in the valley of deep darkness, I do not fear evil} (Ps 23:4) \\
	\foreignlanguage{hebrew}{וְגַם אֶלְנָתָן וּדְלָיָהוּ וּגְמַרְיָהוּ הִפְגִּעוּ בַמֶּלֶךְ לְבִלְתִּי שְׂרֹף אֶת־הַמְּגִלָּה וְלֹא שָׁמַע אֲלֵיהֶם} & \textit{Even though Elnathan and Delaiah and Gemariah urged the king not to burn the scroll, he did not listen to them} (Jer 36:25) \\
\end{longtable}


\section{Exercises}

%\subsection{Translation of Verbal Forms}
%
%Translate the following verbal forms. Identify the gender (masc., fem., comm.) and number (sg., pl.) of forms of which the English translation is ambiguous (i.e., \textit{you}, \textit{they}). Mark the stressed syllable if stress is not on the last syllable.
%
%\hspace{0.5cm}
%
%\selectlanguage{hebrew}
%
%\noindent
%1~~\foreignlanguage{hebrew}{אוֹדֶה}  \hspace{0.3cm}
%2~~\foreignlanguage{hebrew}{נוֹדֶה}  \hspace{0.3cm}
%3~~\foreignlanguage{hebrew}{הִתְוַדּוּ}  \hspace{0.3cm}
%4~~\foreignlanguage{hebrew}{הֵיטִיבוּ}  \hspace{0.3cm}
%5~~\foreignlanguage{hebrew}{הֵיטַבְתָּ}  \hspace{0.3cm}
%6~~\foreignlanguage{hebrew}{יֵיטִיב}  \hspace{0.3cm}
%7~~\foreignlanguage{hebrew}{הוֹסַפְתִּי}  \hspace{0.3cm}
%8~~\foreignlanguage{hebrew}{תֹסִיפוּ}  \hspace{0.3cm}
%9~~\foreignlanguage{hebrew}{יוֹסִפוּ}  \hspace{0.3cm}
%10~~\foreignlanguage{hebrew}{הִתְיַצְּבוּ}  \hspace{0.3cm}
%11~~\foreignlanguage{hebrew}{יִתְיַצֵּב}  \hspace{0.3cm}
%12~~\foreignlanguage{hebrew}{הוֹשַׁעְתֶּם}  \hspace{0.3cm}
%13~~\foreignlanguage{hebrew}{הוֹשַׁעְתִּי}  \hspace{0.3cm}
%14~~\foreignlanguage{hebrew}{תּוֹשִׁיעַ}  \hspace{0.3cm}
%15~~\foreignlanguage{hebrew}{‎הִכּוּ}  \hspace{0.3cm}
%16~~\foreignlanguage{hebrew}{וָאוֹלֵךְ}  \hspace{0.3cm}
%17~~\foreignlanguage{hebrew}{‎וַיֹּלִכוּ}  \hspace{0.3cm}
%18~~\foreignlanguage{hebrew}{‎תַּחֲרִישִׁי }  \hspace{0.3cm}
%\selectlanguage{english}


\subsection{Translation of Sentences}

Translate the following sentences from the Hebrew Bible. Names of persons and geographical names in these sentences: \foreignlanguage{hebrew}{אֲבִגַיִל}, \foreignlanguage{hebrew}{אָדָם},  \foreignlanguage{hebrew}{אֲדֹנִיָּהוּ}, \foreignlanguage{hebrew}{אֱנוֺשׁ}, \foreignlanguage{hebrew}{בַּת־שֶׁבַע}, \foreignlanguage{hebrew}{דָּוִד}, \foreignlanguage{hebrew}{הֶבֶל}, \foreignlanguage{hebrew}{יְהוֺנָתָן}, \foreignlanguage{hebrew}{יְהוֺשֻׁעַ}, \foreignlanguage{hebrew}{יָרָבְעָם}, \foreignlanguage{hebrew}{יְרִיחוֺ}, \foreignlanguage{hebrew}{יִרְמְיָהוּ}, \foreignlanguage{hebrew}{מְפִיבֹשֶׁת}, \foreignlanguage{hebrew}{קַיִן}, \foreignlanguage{hebrew}{שָׁאוּל}, \foreignlanguage{hebrew}{שִׁישַׁק}, \foreignlanguage{hebrew}{שְׁלֹמֹה}, \foreignlanguage{hebrew}{שֵׁת}

\vspace{0.5cm}

\selectlanguage{hebrew}
\noindent
1~~\foreignlanguage{hebrew}{בְּזֵעַ֤ת}\LTRfootnote{\space \foreignlanguage{hebrew}{זֵעָה} \textit{sweat}} \foreignlanguage{hebrew}{אַפֶּ֙יךָ֙ תֹּ֣אכַל לֶ֔חֶם עַ֤ד שֽׁוּבְךָ֙ אֶל־הָ֣אֲדָמָ֔ה כִּ֥י מִמֶּ֖נָּה לֻקָּ֑חְתָּ כִּֽי־עָפָ֣ר אַ֔תָּה וְאֶל־עָפָ֖ר תָּשֽׁוּב׃} \hspace{0.3cm}
2~~\foreignlanguage{hebrew}{וַֽיְשַׁלְּחֵ֛הוּ יְהוָ֥ה אֱלֹהִ֖ים מִגַּן־עֵ֑דֶן}\LTRfootnote{\space \foreignlanguage{hebrew}{גַּן־עֵדֶן} \textit{Garden of Eden}} \foreignlanguage{hebrew}{לַֽעֲבֹד֙ אֶת־הָ֣אֲדָמָ֔ה אֲשֶׁ֥ר לֻקַּ֖ח מִשָּֽׁם׃} \hspace{0.3cm}
3~~\foreignlanguage{hebrew}{וַיֵּ֨דַע אָדָ֥ם עוֹד֙ אֶת־אִשְׁתּ֔וֹ וַתֵּ֣לֶד בֵּ֔ן וַתִּקְרָ֥א אֶת־שְׁמ֖וֹ שֵׁ֑ת כִּ֣י שָֽׁת־לִ֤י אֱלֹהִים֙ זֶ֣רַע אַחֵ֔ר תַּ֣חַת הֶ֔בֶל כִּ֥י הֲרָג֖וֹ קָֽיִן׃ וּלְשֵׁ֤ת גַּם־הוּא֙ יֻלַּד־בֵּ֔ן וַיִּקְרָ֥א אֶת־שְׁמ֖וֹ אֱנ֑וֹשׁ אָ֣ז הוּחַ֔ל לִקְרֹ֖א בְּשֵׁ֥ם יְהוָֽה׃}  \hspace{0.3cm}
4~~\foreignlanguage{hebrew}{‏וַיָּבֹא מְפִיבֹ֨שֶׁת בֶּן־יְהוֹנָתָ֤ן בֶּן־שָׁאוּל֙ אֶל־דָּוִ֔ד וַיִּפֹּ֥ל עַל־פָּנָ֖יו וַיִּשְׁתָּ֑חוּ וַיֹּ֤אמֶר דָּוִד֙ מְפִיבֹ֔שֶׁת וַיֹּ֖אמֶר הִנֵּ֥ה עַבְדֶּֽךָ׃} \hspace{0.3cm}
5~~\foreignlanguage{hebrew}{וַתֵּ֤רֶא אֲבִיגַ֙יִל֙ אֶת־דָּוִ֔ד וַתְּמַהֵ֕ר וַתֵּ֖רֶד מֵעַ֣ל הַחֲמ֑וֹר וַתִּפֹּ֞ל לְאַפֵּ֤י דָוִד֙ עַל־פָּנֶ֔יהָ וַתִּשְׁתַּ֖חוּ אָֽרֶץ׃}  \hspace{0.3cm}
6~~\foreignlanguage{hebrew}{וַתָּבֹ֤א בַת־שֶׁ֙בַע֙ אֶל־הַמֶּ֣לֶךְ שְׁלֹמֹ֔ה לְדַבֶּר־ל֖וֹ עַל־אֲדֹנִיָּ֑הוּ וַיָּקָם֩ הַמֶּ֨לֶךְ לִקְרָאתָ֜הּ וַיִּשְׁתַּ֣חוּ לָ֗הּ וַיֵּ֙שֶׁב֙ עַל־כִּסְא֔וֹ וַיָּ֤שֶׂם כִּסֵּא֙ לְאֵ֣ם הַמֶּ֔לֶךְ וַתֵּ֖שֶׁב לִֽימִינֽוֹ׃}  \hspace{0.3cm}
7~~\foreignlanguage{hebrew}{‏וַיַּחֲלֹ֥ם}\LTRfootnote{\space \foreignlanguage{hebrew}{חלם} Q.\ \textit{to dream}} \foreignlanguage{hebrew}{עוֹד֙ חֲל֣וֹם אַחֵ֔ר וַיְסַפֵּ֥ר אֹת֖וֹ לְאֶחָ֑יו וַיֹּ֗אמֶר הִנֵּ֨ה חָלַ֤מְתִּֽי חֲלוֹם֙ ע֔וֹד וְהִנֵּ֧ה הַשֶּׁ֣מֶשׁ וְהַיָּרֵ֗חַ}\LTRfootnote{\space \foreignlanguage{hebrew}{יָרֵחַ} \textit{moon}} \foreignlanguage{hebrew}{וְאַחַ֤ד עָשָׂר֙ כּֽוֹכָבִ֔ים}\LTRfootnote{\space \foreignlanguage{hebrew}{כּוֹכָב} \textit{star}} \foreignlanguage{hebrew}{מִֽשְׁתַּחֲוִ֖ים לִֽי׃‏ וַיְסַפֵּ֣ר אֶל־אָבִיו֮ וְאֶל־אֶחָיו֒ וַיִּגְעַר}\LTRfootnote{\space \foreignlanguage{hebrew}{גער} Q.\ \textit{to rebuke}} \foreignlanguage{hebrew}{־בּ֣וֹ אָבִ֔יו וַיֹּ֣אמֶר ל֔וֹ מָ֛ה הַחֲל֥וֹם הַזֶּ֖ה אֲשֶׁ֣ר חָלָ֑מְתָּ הֲב֣וֹא נָב֗וֹא אֲנִי֙ וְאִמְּךָ֣ וְאַחֶ֔יךָ לְהִשְׁתַּחֲוֹ֥ת לְךָ֖ אָֽרְצָה׃} \hspace{0.3cm}
9~~\foreignlanguage{hebrew}{וַיְהִ֗י בִּֽהְי֣וֹת יְהוֹשֻׁעַ֮ בִּירִיחוֹ֒ וַיִּשָּׂ֤א עֵינָיו֙ וַיַּ֔רְא וְהִנֵּה־אִישׁ֙ עֹמֵ֣ד לְנֶגְדּ֔וֹ וְחַרְבּ֥וֹ שְׁלוּפָ֖ה}\LTRfootnote{\space \foreignlanguage{hebrew}{שׁלף} Q.\ \textit{to draw out, off}} \foreignlanguage{hebrew}{בְּיָד֑וֹ וַיֵּ֨לֶךְ יְהוֹשֻׁ֤עַ אֵלָיו֙ וַיֹּ֣אמֶר ל֔וֹ הֲלָ֥נוּ אַתָּ֖ה אִם־לְצָרֵֽינוּ׃ וַיֹּ֣אמֶר ׀ לֹ֗א כִּ֛י אֲנִ֥י שַׂר־צְבָֽא־יְהוָ֖ה עַתָּ֣ה בָ֑אתִי וַיִּפֹּל֩ יְהוֹשֻׁ֨עַ אֶל־פָּנָ֥יו אַ֙רְצָה֙ וַיִּשְׁתָּ֔חוּ וַיֹּ֣אמֶר ל֔וֹ מָ֥ה אֲדֹנִ֖י מְדַבֵּ֥ר אֶל־עַבְדּֽוֹ׃ וַיֹּאמֶר֩ שַׂר־צְבָ֨א יְהוָ֜ה אֶל־יְהוֹשֻׁ֗עַ שַׁל}\LTRfootnote{\space \foreignlanguage{hebrew}{נשׁל} Q.\ \textit{loosen} (sandal)} \foreignlanguage{hebrew}{־נַֽעֲלְךָ֙}\LTRfootnote{\space \foreignlanguage{hebrew}{נַעַל} \textit{sandal}} \foreignlanguage{hebrew}{מֵעַ֣ל רַגְלֶ֔ךָ כִּ֣י הַמָּק֗וֹם אֲשֶׁ֥ר אַתָּ֛ה עֹמֵ֥ד עָלָ֖יו קֹ֣דֶשׁ ה֑וּא וַיַּ֥עַשׂ יְהוֹשֻׁ֖עַ כֵּֽן׃} \hspace{0.3cm}
10~~\foreignlanguage{hebrew}{הַדָּבָר֙ אֲשֶׁ֣ר הָיָ֣ה אֶֽל־יִרְמְיָ֔הוּ מֵאֵ֥ת יְהוָ֖ה לֵאמֹֽר׃ עֲמֹ֗ד בְּשַׁ֙עַר֙ בֵּ֣ית יְהוָ֔ה וְקָרָ֣אתָ שָּׁ֔ם אֶת־הַדָּבָ֖ר הַזֶּ֑ה וְאָמַרְתָּ֞ שִׁמְע֣וּ דְבַר־יְהוָ֗ה כָּל־יְהוּדָה֙ הַבָּאִים֙ בַּשְּׁעָרִ֣ים הָאֵ֔לֶּה לְהִֽשְׁתַּחֲוֺ֖ת לַיהוָֽה׃ כֹּֽה־אָמַ֞ר יְהוָ֤ה צְבָאוֹת֙ אֱלֹהֵ֣י יִשְׂרָאֵ֔ל הֵיטִ֥יבוּ דַרְכֵיכֶ֖ם וּמַֽעַלְלֵיכֶ֑ם}\LTRfootnote{\space \foreignlanguage{hebrew}{מַעֲלָלִים} \textit{deeds}} \foreignlanguage{hebrew}{וַאֲשַׁכְּנָ֣ה}\LTRfootnote{\space \foreignlanguage{hebrew}{שׁכן} Pi.\ \textit{to cause to dwell}} \foreignlanguage{hebrew}{אֶתְכֶ֔ם בַּמָּק֥וֹם הַזֶּֽה׃} \hspace{0.3cm}
11~~\foreignlanguage{hebrew}{‏וַיִּכְרֹ֨ת יְהוָ֤ה אִתָּם֙ בְּרִ֔ית וַיְצַוֵּ֣ם לֵאמֹ֔ר לֹ֥א תִֽירְא֖וּ אֱלֹהִ֣ים אֲחֵרִ֑ים וְלֹא־תִשְׁתַּחֲו֣וּ לָהֶ֔ם וְלֹ֣א תַעַבְד֔וּם וְלֹ֥א תִזְבְּח֖וּ לָהֶֽם׃ כִּ֣י אִֽם־אֶת־יְהוָ֗ה אֲשֶׁר֩ הֶעֱלָ֨ה אֶתְכֶ֜ם מֵאֶ֧רֶץ מִצְרַ֛יִם בְּכֹ֧חַ גָּד֛וֹל וּבִזְר֥וֹעַ נְטוּיָ֖ה אֹת֣וֹ תִירָ֑אוּ וְל֥וֹ תִֽשְׁתַּחֲו֖וּ וְל֥וֹ תִזְבָּֽחוּ׃} \hspace{0.3cm}
12~~\foreignlanguage{hebrew}{וַיְבַקֵּ֥שׁ שְׁלֹמֹ֖ה לְהָמִ֣ית אֶת־יָרָבְעָ֑ם וַיָּ֣קָם יָרָבְעָ֗ם וַיִּבְרַ֤ח מִצְרַ֙יִם֙ אֶל־שִׁישַׁ֣ק מֶֽלֶךְ־מִצְרַ֔יִם וַיְהִ֥י בְמִצְרַ֖יִם עַד־מ֥וֹת שְׁלֹמֹֽה׃}  \hspace{0.3cm}
\selectlanguage{english}


\section{Hebrew Reading: 1 Kings 19:15--21}
Translate 1\,Kings 19:15--21 with the help of notes below the text.

\vspace{0.5cm}

\selectlanguage{hebrew}
\noindent
\textsuperscript{15}~\foreignlanguage{hebrew}{וַיֹּ֤אמֶר יְהוָה֙ אֵלָ֔יו לֵ֛ךְ שׁ֥וּב לְדַרְכְּךָ֖ מִדְבַּ֣רָה דַמָּ֑שֶׂק וּבָ֗אתָ וּמָשַׁחְתָּ֧ אֶת־חֲזָאֵ֛ל לְמֶ֖לֶךְ עַל־אֲרָֽם׃} \hspace{0.3cm}
\textsuperscript{16}~\foreignlanguage{hebrew}{וְאֵת֙ יֵה֣וּא בֶן־נִמְשִׁ֔י תִּמְשַׁ֥ח לְמֶ֖לֶךְ עַל־יִשְׂרָאֵ֑ל וְאֶת־אֱלִישָׁ֤ע בֶּן־שָׁפָט֙ מֵאָבֵ֣ל מְחוֹלָ֔ה תִּמְשַׁ֥ח לְנָבִ֖יא תַּחְתֶּֽיךָ׃} \hspace{0.3cm}
\textsuperscript{17}~\foreignlanguage{hebrew}{וְהָיָ֗ה הַנִּמְלָ֛ט מֵחֶ֥רֶב חֲזָאֵ֖ל יָמִ֣ית יֵה֑וּא וְהַנִּמְלָ֛ט מֵחֶ֥רֶב יֵה֖וּא יָמִ֥ית אֱלִישָֽׁע׃} \hspace{0.3cm}
\textsuperscript{18}~\foreignlanguage{hebrew}{וְהִשְׁאַרְתִּ֥י בְיִשְׂרָאֵ֖ל שִׁבְעַ֣ת אֲלָפִ֑ים כָּל־הַבִּרְכַּ֗יִם אֲשֶׁ֤ר לֹֽא־כָֽרְעוּ֙ לַבַּ֔עַל וְכָ֨ל־הַפֶּ֔ה אֲשֶׁ֥ר לֹֽא־נָשַׁ֖ק לֽוֹ׃} \hspace{0.3cm}
\textsuperscript{19}~\foreignlanguage{hebrew}{וַיֵּ֣לֶךְ מִ֠שָּׁם וַיִּמְצָ֞א אֶת־אֱלִישָׁ֤ע בֶּן־שָׁפָט֙ וְה֣וּא חֹרֵ֔שׁ שְׁנֵים־עָשָׂ֤ר צְמָדִים֙ לְפָנָ֔יו וְה֖וּא בִּשְׁנֵ֣ים הֶעָשָׂ֑ר וַיַּעֲבֹ֤ר אֵלִיָּ֙הוּ֙ אֵלָ֔יו וַיַּשְׁלֵ֥ךְ אַדַּרְתּ֖וֹ אֵלָֽיו׃} \hspace{0.3cm}
\textsuperscript{20}~\foreignlanguage{hebrew}{וַיַּעֲזֹ֣ב אֶת־הַבָּקָ֗ר וַיָּ֙רָץ֙ אַחֲרֵ֣י אֽ͏ֵלִיָּ֔הוּ וַיֹּ֗אמֶר אֶשְּׁקָה־נָּא֙ לְאָבִ֣י וּלְאִמִּ֔י וְאֵלְכָ֖ה אַחֲרֶ֑יךָ וַיֹּ֤אמֶר לוֹ֙ לֵ֣ךְ שׁ֔וּב כִּ֥י מֶה־עָשִׂ֖יתִי לָֽךְ׃} \hspace{0.3cm}
\textsuperscript{21}~\foreignlanguage{hebrew}{וַיָּ֨שָׁב מֵאַחֲרָ֜יו וַיִּקַּ֣ח אֶת־צֶ֧מֶד הַבָּקָ֣ר וַיִּזְבָּחֵ֗הוּ וּבִכְלִ֤י הַבָּקָר֙ בִּשְּׁלָ֣ם הַבָּשָׂ֔ר וַיִּתֵּ֥ן לָעָ֖ם וַיֹּאכֵ֑לוּ וַיָּ֗קָם וַיֵּ֛לֶךְ אַחֲרֵ֥י אֵלִיָּ֖הוּ וַיְשָׁרְתֵֽהוּ׃} \hspace{0.3cm}
\selectlanguage{english}

\vspace{0.25cm}

\hspace*{-0.5cm}\begin{longtable}{p{0.075\linewidth} p{0.15\linewidth}p{0.675\linewidth}}
	19:15 & \foreignlanguage{hebrew}{דַּמֶּשֶׁק} & \textit{Damascus} \\
	19:16 & \foreignlanguage{hebrew}{אָבֵל מְחוֺלָה} & name of a place \\
	19:18 & \foreignlanguage{hebrew}{בֶּרֶךְ} & \textit{knee} \\
	& \foreignlanguage{hebrew}{כרע} Q.\ & \textit{to bow down, kneel} \\
	& \foreignlanguage{hebrew}{נשׁק} Q.\ & \textit{to kiss} \\
	19:19 & \foreignlanguage{hebrew}{חרשׁ} Q.\ & \textit{to plow} \\
	& \foreignlanguage{hebrew}{צֶמֶד} & \textit{couple, pair} (of oxen) \\
	& \foreignlanguage{hebrew}{אַדֶּרֶת} & \textit{cloak, robe} \\
	19:20 & \foreignlanguage{hebrew}{נשׁק} Q.\ & see 19:18 \\
	19:21 & \foreignlanguage{hebrew}{צֶמֶד} & see 19:19 \\
	& \foreignlanguage{hebrew}{בשׁל} Pi.\ & \textit{to cook} \\
\end{longtable}

\backmatter


\chapter{Ketiv and Qere in the Hebrew Bible}

% As this is a textual phenomenon and not a grammatical one, Ketiv and Qere is explained in an appendix.

The text of the Hebrew Bible was finalized at the end of the 1st century and the early 2nd century CE. Since the 2nd century CE, the then still unpointed text remained stable; changes were not allowed. As a consequence, a system was developed by the Masoretes that allowed for alternative readings.

For this purpose, the alternative reading was put in the margin of the biblical text in the manuscript. This is called \textit{Qere} \foreignlanguage{hebrew}{קְרֵי}, Aramaic for \textit{(what is to be) read}. A small \foreignlanguage{hebrew}{ק̇} with a dot on top (an abbreviation for \foreignlanguage{hebrew}{קְרֵי}) under the alternative reading indicates it.

The reading for which there is a variant is indicated by a small circle (in Latin \textit{circellus}) that was placed on top of the Hebrew word or words. This word or these words are called \textit{Ketiv} \foreignlanguage{hebrew}{כְּתִיב} \textit{(what is) written}. 

The vowel signs of the Qere and the accents are placed with the Ketiv in the main text and not with the Qere in the margin. When the vowels are incompatible with the word they are placed with (as in 2\,Kgs 18:27 for example), it is an indication that there is a Qere in the margin. Otherwise, one always needs to look for a Qere in the margin in order not to miss it.

It needs to be noted that the Qere practically replaces the Ketiv so that only the Qere is read and not the Ketiv, although the Ketiv may be interesting in itself.

Qere may be found in the following circumstances in the Hebrew Bible:

\begin{itemize}[noitemsep]
	\item[--] Correction of obvious mistakes, e.g., Ketiv \foreignlanguage{hebrew}{עברנו} \textit{our crossing} with the Qere \foreignlanguage{hebrew}{עָבְרָם} \textit{their crossing} (Josh 5:1)
	\item[--] Adjustment to spelling conventions, e.g., Ketiv \foreignlanguage{hebrew}{ותלוש} \textit{and she kneaded} with an exceptional vowel letter \foreignlanguage{hebrew}{ו} for a short vowel with the Qere \foreignlanguage{hebrew}{וַתָּ֫לָשׁ} \textit{wattā́lɔš} \textit{and she kneaded} (2\,Sam 13:8)
	\item[--] Adjustments to grammatical norms, e.g., Ketiv \foreignlanguage{hebrew}{בהשדה} with unusual preservation of the article after a proclitic preposition with the Qere \foreignlanguage{hebrew}{בַשָּׂדֶה} \textit{in the field} (2\,Kgs 7:12)
	\item[--] Preferred readings, e.g., Ketiv \foreignlanguage{hebrew}{הבאיש} (vocalized \foreignlanguage{hebrew}{הִבְאִישׁ}) \textit{he became hated} with the Qere \foreignlanguage{hebrew}{הֹבִישׁ} \textit{he became ashamed} (Isa 30:5)
\end{itemize}

A Qere may be even found where is there no word in the main text that it is meant to replace. This is a case of \foreignlanguage{hebrew}{קרי ולא כתיב},\footnote{\space In vocalized Aramaic \foreignlanguage{hebrew}{קְרֵי וְלָא כְתִיב}.} e.g., Ketiv \foreignlanguage{hebrew}{הנה ימים} \textit{and see, days} with the Qere \foreignlanguage{hebrew}{הִנֵּה יָמִים בָּאִים} \textit{and see, days are coming}. 
The opposite, an element in the main text that is not to be read, thus dropped, is called \foreignlanguage{hebrew}{כתיב ולא קרי},\footnote{\space In vocalized Aramaic \foreignlanguage{hebrew}{כְּתִיב וְלָא קְרֵי}.} e.g., Ketiv \foreignlanguage{hebrew}{יסלח נא} \textit{may he forgive} with the particle \foreignlanguage{hebrew}{נָא} with the Qere \foreignlanguage{hebrew}{יִסְלַח} (2\,Kgs 5:18).


\chapter{Key to the Exercises in Chapter 3--26}

\section*{Chapter 3}
\subsection*{Frequent Verbal Forms}

\textbf{1.}~~Judg 6:29 \hspace{0.2cm}
\textbf{2.}~~2\,Sam~23:22 \hspace{0.2cm} 
\textbf{3.}~~Gen 29:22 \hspace{0.2cm} 
\textbf{4.}~~Gen 29:28 \hspace{0.2cm} 
\textbf{5.}~~Judg 3:16 \hspace{0.2cm} 
\textbf{6.}~~Judg 14:10 \hspace{0.2cm} 
\textbf{7.}~~2\,Kgs 23:34 \hspace{0.2cm} 
\textbf{8.}~~Gen 20:2 \hspace{0.2cm} 
\textbf{9.}~~Gen 24:61 \hspace{0.2cm} 
\textbf{10.}~~Exod 24:8 \hspace{0.2cm} 
\textbf{11.}~~Num 22:41 \hspace{0.2cm} 
\textbf{12.}~~Judg 9:43 \hspace{0.2cm} 
\textbf{13.}~~1\,Sam 31:4 \hspace{0.2cm} 
\textbf{14.}~~1\,Kgs 17:23 \hspace{0.2cm} 
\textbf{15.}~~2\,Kgs 23:35 \hspace{0.2cm} 
\textbf{16.}~~Gen 21:27 \hspace{0.2cm} 
\textbf{17.}~~2\,Kgs 13:5 \hspace{0.2cm} 
\textbf{18.}~~1\,Sam 9:17 \hspace{0.2cm} 
\textbf{19.}~~Gen 6:12 \hspace{0.2cm} 
\textbf{20.}~~Gen 29:2 \hspace{0.2cm} 
\textbf{21.}~~Judg 9:36 \hspace{0.2cm} 
\textbf{22.}~~2\,Kgs 16:12 \hspace{0.2cm} 
\textbf{23.}~~Gen 22:4 \hspace{0.2cm} 
\textbf{24.}~~Exod 2:15 \hspace{0.2cm}

\subsection*{Exercises with the Article}

\selectlanguage{hebrew}
הָאָב הָאִישׁ הָאֱלֹהִים הָאָרֶץ הַבֵּן הַבָּקָר הַדָּבָר הַדָּם הַזָּהָב חַחֶרֶב הַיּוֺם הַיֶּלֶד הַכֶּסֶף הַמִּזְבֵּחַ הַמֶּלֶךְ הַמָּקוֺם הַמִּשְׁתֶּה הָעֶבֶד הָעַיִן הָעִיר הָעָם הַצֹּאן הַשָּׂדֶה
\selectlanguage{english}

\bigskip

\noindent Note: The word \foreignlanguage{hebrew}{פַּרְעֹה} is never used with the article.

\section*{Chapter 4}

\subsection*{Frequent Verbal Forms}
\textbf{1.}~~1\,Kgs 22:27 \hspace{0.2cm} 
\textbf{1.}~~1\,Sam 17:27 \hspace{0.2cm} 
\textbf{3.}~~Num 23:2 \hspace{0.2cm} 
\textbf{4.}~~1\,Sam 18:24 \hspace{0.2cm} 
\textbf{5.}~~Exod 6:13 \hspace{0.2cm} 
\textbf{6.}~~1\,Kgs 19:4 \hspace{0.2cm} 
\textbf{7.}~~Gen 28:9 \hspace{0.2cm} 
\textbf{8.}~~Gen 26:17 \hspace{0.2cm} 
\textbf{9.}~~Gen 28:10 \hspace{0.2cm} 
\textbf{10.}~~Judg 16:1 \hspace{0.2cm} 
\textbf{11.}~~2\,Sam 3:21 \hspace{0.2cm} 
\textbf{12.}~~2\,Kgs 2:25 \hspace{0.2cm} 
\textbf{13.}~~Gen 19:23 \hspace{0.2cm} 
\textbf{14.}~~Num 22:36 \hspace{0.2cm} 
\textbf{15.}~~2\,Sam 3:23 \hspace{0.2cm} 
\textbf{16.}~~1\,Sam 4:7 \hspace{0.2cm} 
\textbf{17.}~~Gen 30:16 \hspace{0.2cm} 
\textbf{18.}~~Gen 20:3 \hspace{0.2cm} 
\textbf{19.}~~Exod 3:1 \hspace{0.2cm} 
\textbf{20.}~~Exod 7:10 \hspace{0.2cm} 
\textbf{21.}~~Exod 24:18 \hspace{0.2cm} 
\textbf{22.}~~Num 22:20 \hspace{0.2cm} 
\textbf{23.}~~Judg 13:9 \hspace{0.2cm} 
\textbf{24.}~~1\,Sam 2:27 \hspace{0.2cm}
\textbf{25.}~~Judg 9:51 \hspace{0.2cm}
\textbf{26.}~~Judg 11:1 \hspace{0.2cm}
\textbf{27.}~~Gen 12:10 \hspace{0.2cm}

\section*{Chapter 5}

\subsection*{Translation of Verbal Forms}
\textbf{1.} \textit{they said}, comm.\  \hspace{0.3cm}
\textbf{2.} \textit{I went}, comm.\ \foreignlanguage{hebrew}{הָלַ֫כְתִּי}
\textbf{3.} \textit{he took}, masc.\ \hspace{0.3cm}
\textbf{4.} \textit{she took}, fem.\ \hspace{0.3cm}
\textbf{5.} \textit{I~sent}, comm. \foreignlanguage{hebrew}{שָׁלַ֫חְתִּי} \hspace{0.3cm}
\textbf{6.} \textit{you know/knew}, masc. \foreignlanguage{hebrew}{יָדַ֫עְתָּ} \hspace{0.3cm}
\textbf{7.} \textit{they sat}, comm.\ \hspace{0.3cm}
\textbf{8.} \textit{we went}, comm.\ \foreignlanguage{hebrew}{הָלַ֫כְנוּ} \hspace{0.3cm}
\textbf{9.} \textit{you heard}, fem.\ \hspace{0.3cm}
\textbf{10.} \textit{he fell}, masc.\ \hspace{0.3cm}
\textbf{11.} \textit{you kept}, masc.\ \hspace{0.3cm}
\textbf{12.} \textit{you asked}, fem.\ \hspace{0.3cm}
\textbf{13.} \textit{you bore}, fem.\ \hspace{0.3cm}
\textbf{14.} \textit{you ate}, masc.\ \hspace{0.3cm}
\textbf{15.} \textit{you became king}, masc.\ \foreignlanguage{hebrew}{מָלַ֫כְתָּ} \hspace{0.3cm}
\textbf{16.} \textit{you know/knew}, fem.\ \hspace{0.3cm}
\textbf{17.} \textit{she said}, fem.\ \hspace{0.3cm}
\textbf{18.} \textit{they gave}, comm.\ \hspace{0.3cm}
\textbf{19.} \textit{you gave}, masc.\ \foreignlanguage{hebrew}{נָתַ֫תָּה} \hspace{0.3cm}
\textbf{20.} \textit{you gave}, masc.\ \foreignlanguage{hebrew}{נָתַ֫תָּ}



\subsection*{Translation of Sentences}
\textbf{1.}~~Deut 29:5 \hspace{0.2cm}
\textbf{2.}~~2\,Kgs 1:14 \hspace{0.2cm}
\textbf{3.}~~1\,Sam 5:1 \hspace{0.2cm}
\textbf{4.}~~1\,Kgs 8:39 \hspace{0.2cm}
\textbf{5.}~~Deut 5:28 \hspace{0.2cm}
\textbf{6.}~~2\,Kgs 5:6 \hspace{0.2cm}
\textbf{7.}~~Josh 2:9 \hspace{0.2cm}
\textbf{8.}~~2\,Kgs 20:14 \hspace{0.2cm}
\textbf{9.}~~Exod 2:25 \hspace{0.2cm}
\textbf{10.}~~Gen 21:17 \hspace{0.2cm}
\textbf{11.}~~1\,Sam 17:11 \hspace{0.2cm}
\textbf{12.}~~Judg 11:15 \hspace{0.2cm}
\textbf{13.}~~1\,Kgs 15:20 \hspace{0.2cm}
\textbf{14.}~~Exod 5:1


\section*{Chapter 6}

\subsection*{Translation of Verbal Forms}
\textbf{1.} \textit{you will remember}, masc.\ pl.  \hspace{0.3cm}
\textbf{2.} \textit{I will cut}, comm.\ 
\textbf{3.} \textit{you/she will serve}, masc.\ sg./fem.\ sg. \hspace{0.3cm}
\textbf{4.} \textit{you will keep}, masc.\ pl.\ \hspace{0.3cm}
\textbf{5.} \textit{you/she will draw near}, masc.\ sg./fem.\ sg. \hspace{0.3cm}
\textbf{6.} \textit{you will remember}, fem.\ sg. \hspace{0.3cm}
\textbf{7.} \textit{we will listen} \hspace{0.3cm}
\textbf{8.} \textit{I will send} \hspace{0.3cm}
\textbf{9.} \textit{they shall reign}, masc.\ pl. \foreignlanguage{hebrew}{יִמְלֹ֫כוּ} \hspace{0.3cm}
\textbf{10.} \textit{you will send}, fem.\ pl.\ \foreignlanguage{hebrew}{תִּשְׁלַ֫חְנָה} \hspace{0.3cm}



\subsection*{Translation of Sentences}
\textbf{1.}~~1\,Sam 13:13 \hspace{0.2cm}
\textbf{2.}~~Gen 21:1 \hspace{0.2cm}
\textbf{3.}~~Judg 7:24 \hspace{0.2cm}
\textbf{4.}~~Judg 11:28 \hspace{0.2cm}
\textbf{5.}~~Exod 14:29 \hspace{0.2cm}
\textbf{6.}~~1\,Kgs 4:1 \hspace{0.2cm}
\textbf{7.}~~1\,Sam 4:1 \hspace{0.2cm}
\textbf{8.}~~Gen 41:56 \hspace{0.2cm}
\textbf{9.}~~1\,Kgs 2:30 \hspace{0.2cm}
\textbf{10.}~~Judg 21:8 \hspace{0.2cm}
\textbf{11.}~~2\,Kgs 21:21 \hspace{0.2cm}
\textbf{12.}~~Exod 4:30 \hspace{0.2cm}
\textbf{13.}~~1\,Kgs 15:20 \hspace{0.2cm}
\textbf{14.}~~Judg 6:8 \hspace{0.2cm}
\textbf{14.}~~Exod 5:1 \hspace{0.2cm}
\textbf{16.}~~1 Sam 11:3 \hspace{0.2cm}
\textbf{14.}~~Exod 5:1 \hspace{0.2cm}
\textbf{15.}~~1\,Kgs 20:2 \hspace{0.2cm} 
\textbf{16.}~~1\,Sam 11:3 \hspace{0.2cm}
\textbf{17.}~~Exod 4:31 \hspace{0.2cm}
\textbf{18.}~~Gen 16:2 \hspace{0.2cm}
\textbf{19.}~~2\,Kgs 18:31 \hspace{0.2cm}
\textbf{20.}~~1\,Sam 8:21 \hspace{0.2cm}
\textbf{21.}~~Exod 7:13 \hspace{0.2cm}
\textbf{22.}~~Judg 10:16 \hspace{0.2cm}
\textbf{23.}~~1\,Sam 14:36 \hspace{0.2cm}
\textbf{24.}~~Exod 3:12 \hspace{0.2cm}

\section*{Chapter 7}

\subsection*{Translation of Sentences}
\textbf{1.}~~Gen 21:32 \hspace{0.2cm}
\textbf{2.}~~1\,Sam 16:10 \hspace{0.2cm}
\textbf{3.}~~Exod 18:25 \hspace{0.2cm}
\textbf{4.}~~Judg 8:18 \hspace{0.2cm}
\textbf{5.}~~1\,Sam 22:21 \hspace{0.2cm}
\textbf{6.}~~1\,Kgs 19:1 \hspace{0.2cm}
\textbf{7.}~~Judg 8:17 \hspace{0.2cm}
\textbf{8.}~~2\,Kgs 23:20 \hspace{0.2cm}
\textbf{9.}~~Josh 21:45 \hspace{0.2cm}
\textbf{10.}~~Gen 19:29 \hspace{0.2cm}
\textbf{11.}~~Gen 42:9 \hspace{0.2cm}
\textbf{12.}~~Judg 2:13 \hspace{0.2cm}
\textbf{13.}~~Josh 10:22 \hspace{0.2cm}
\textbf{14.}~~1\,Kgs 22:37 \hspace{0.2cm}
\textbf{15.}~~1\,Sam 14:36 \hspace{0.2cm}
\textbf{16.}~~Lev 9:8 \hspace{0.2cm}
\textbf{17.}~~1\,Sam 11:15 \hspace{0.2cm}
\textbf{18.}~~1\,Sam 16:4 \hspace{0.2cm}
\textbf{19.}~~Gen 20:14 \hspace{0.2cm}

\section*{Chapter 8}

\subsection*{Translation of Sentences}
\textbf{1.}~~Ps 106:48 \hspace{0.2cm}
\textbf{2.}~~Jos. 6:20 \hspace{0.2cm}
\textbf{3.}~~2\,Sam. 11:2 \hspace{0.2cm}
\textbf{4.}~~Gen 19:17 \hspace{0.2cm}
\textbf{5.}~~Gen 24:30 \hspace{0.2cm}
\textbf{6.}~~Deut 4:11 \hspace{0.2cm}
\textbf{7.}~~Josh 10:19 \hspace{0.2cm}
\textbf{8.}~~1\,Kgs 8:22 \hspace{0.2cm}
\textbf{9.}~~1\,Sam 26:5 \hspace{0.2cm}
\textbf{10.}~~1\,Sam 3:15 \hspace{0.2cm}
\textbf{11.}~~1\,Sam 16:22 \hspace{0.2cm}
\textbf{12.}~~1\,Kgs 3:15 \hspace{0.2cm}
\textbf{13.}~~1\,Kgs 10:8 \hspace{0.2cm}
\textbf{14.}~~2\,Kgs 5:9 \hspace{0.2cm}
\textbf{15.}~~Judg 12:4 \hspace{0.2cm}
\textbf{16.}~~1\,Sam 7:5 \hspace{0.2cm}
\textbf{17.}~~1\,Sam 30:3 \hspace{0.2cm}
\textbf{18.}~~Exod 24:16 \hspace{0.2cm}
\textbf{19.}~~Judg 9:36 \hspace{0.2cm}
\textbf{20.}~~2\,Sam 7:2 \hspace{0.2cm}
\textbf{21.}~~Deut 28:3 \hspace{0.2cm}
\textbf{22.}~~Ps 106:48 \hspace{0.2cm}

\section*{Chapter 9}

\subsection*{Translation of Sentences}
\textbf{1.}~~1\,Kgs 22:19 \hspace{0.2cm}
\textbf{2.}~~1\,Kgs 18:19--20 \hspace{0.2cm}
\textbf{3.}~~2\,Kgs 18:5 \hspace{0.2cm}
\textbf{4.}~~Gen 22:10 \hspace{0.2cm}
\textbf{5.}~~Gen 24:30 \hspace{0.2cm}
\textbf{6.}~~Gen 27:34 \hspace{0.2cm}
\textbf{7.}~~Gen 35:22 \hspace{0.2cm}
\textbf{8.}~~Gen 36:31  \hspace{0.2cm}
\textbf{9.}~~Num 21:23  \hspace{0.2cm}
\textbf{10.}~~Deut 5:23  \hspace{0.2cm}
\textbf{11.}~~Josh 24:22  \hspace{0.2cm}
\textbf{12.}~~Ruth 1:1  \hspace{0.2cm}
\textbf{13.}~~1\,Sam 15:1  \hspace{0.2cm}
\textbf{13.}~~1\,Sam 15:1  \hspace{0.2cm}
\textbf{15.}~~2\,Kgs 23.20 \hspace{0.2cm}
\textbf{15.}~~1\,Kgs 5:21 \hspace{0.2cm}
\textbf{16.}~~1\,Kgs 21:27  \hspace{0.2cm}
\textbf{17.}~~1\,Sam 24:21  \hspace{0.2cm}
\textbf{18.}~~Deut 6:17 \hspace{0.2cm}
\textbf{19.}~~1\,Kgs 11:11 \hspace{0.2cm}
\textbf{20.}~~2\,Sam 5:10 \hspace{0.2cm}

\section*{Chapter 10}
\textbf{1.}~~2\,Kgs 10:18 \hspace{0.2cm}
\textbf{2.}~~1\,Kgs 13:31 \hspace{0.2cm}
\textbf{3.}~~2\,Kgs 13:9 \hspace{0.2cm}
\textbf{4.}~~1\,Sam 3:4 \hspace{0.2cm}
\textbf{5.}~~2\,Sam 15:10 \hspace{0.2cm}
\textbf{6.}~~Gen 42:2 \hspace{0.2cm}
\textbf{7.}~~Gen 37:29 \hspace{0.2cm}
\textbf{8.}~~Gen 20:11  \hspace{0.2cm}
\textbf{9.}~~Josh 1:3  \hspace{0.2cm}
\textbf{10.}~~Judg 8:14  \hspace{0.2cm}
\textbf{11.}~~2\,Kgs 10:9  \hspace{0.2cm}
\textbf{12.}~~Josh 9:26  \hspace{0.2cm}
\textbf{13.}~~2\,Sam 7:8  \hspace{0.2cm}
\textbf{14.}~~Josh 10:8  \hspace{0.2cm}
\textbf{15.}~~2\,Kgs 10:14 \hspace{0.2cm}
\textbf{16.}~~2\,Sam 20:1  \hspace{0.2cm}
\textbf{17.}~~Deut 14:27 \hspace{0.2cm}
\textbf{18.}~~Gen 28:16  \hspace{0.2cm}
\textbf{19.}~~Deut 29:13--14 \hspace{0.2cm}
\textbf{20.}~~Jer 31:31--33 \hspace{0.2cm}

\section*{Chapter 11}

\subsection*{Translation of Verbal Forms}
\textbf{1.} \textit{and they ate}, fem.\ pl.  \hspace{0.3cm}
\textbf{2.} \textit{eat!} or \textit{eating}, impv.\ m.\ sg.\ or inf.\ cs.\ \hspace{0.3cm}
\textbf{3.} \textit{eat!}, impv.\ masc.\ pl. \hspace{0.3cm}
\textbf{4.} \textit{I said} \hspace{0.15cm} \foreignlanguage{hebrew}{אָמַ֫רְתִּי} \hspace{0.3cm}
\textbf{5.} \textit{I will say} \hspace{0.3cm}
\textbf{6.} \textit{saying}, part.\ act.\ m.\ sg.\ \hspace{0.3cm}
\textbf{7.} \textit{you will draw near} m.\ pl.\ \hspace{0.3cm}
\textbf{8.} \textit{draw near} impv.\ m.\ sg.\ \hspace{0.3cm}
\textbf{9.} \textit{and I fell} \hspace{0.3cm}
\textbf{10.} \textit{I fell} \hspace{0.15cm} \foreignlanguage{hebrew}{נָפַ֫לְתִּי} \hspace{0.3cm}
\textbf{11.} \textit{and he grasped} \hspace{0.3cm}
\textbf{12.} \textit{and he touched} \hspace{0.3cm}

\subsection*{Translation of Sentences}
\textbf{1.}~~Gen 37:8 \hspace{0.2cm}
\textbf{2.}~~Gen 43:27 \hspace{0.2cm}
\textbf{3.}~~2\,Sam 13:4 \hspace{0.2cm}
\textbf{4.}~~Gen 29:22--25 \hspace{0.2cm}
\textbf{5.}~~Exod 4.29--30 \hspace{0.2cm}
\textbf{6.}~~Gen 45:4 \hspace{0.2cm}
\textbf{7.}~~Josh 3:9 \hspace{0.2cm}
\textbf{8.}~~Josh 14:6  \hspace{0.2cm}
\textbf{9.}~~2\,Kgs 2:5  \hspace{0.2cm}
\textbf{10.}~~Gen 12:9  \hspace{0.2cm}
\textbf{11.}~~Gen 46:1  \hspace{0.2cm}
\textbf{12.}~~Prov 3:1  \hspace{0.2cm}
\textbf{13.}~~Gen 1:31  \hspace{0.2cm}
\textbf{14.}~~Gen 24:7  \hspace{0.2cm}
\textbf{15.}~~Gen 28:13 \hspace{0.2cm}

\section*{Chapter 12}

\subsection*{Translation of Verbal Forms}

\textbf{1.} \textit{they built} \hspace{0.3cm}
\textbf{2.} \textit{I will uncover} \hspace{0.3cm}
\textbf{3.} \textit{you found} \foreignlanguage{hebrew}{מָצָ֫אתָ} \hspace{0.3cm}
\textbf{4.} \textit{I was not willing} \foreignlanguage{hebrew}{אָבִ֫יתִי} \hspace{0.3cm}
\textbf{5.} \textit{you will sin} m.\ pl.\ \hspace{0.3cm}
\textbf{6.} \textit{you will live} m.\ pl.\ \hspace{0.3cm}
\textbf{7.} \textit{and they camped} \hspace{0.3cm}
\textbf{8.} \textit{fill!} impv.\ m.\ pl.\ \hspace{0.3cm}
\textbf{9.} \textit{and he stretched} \hspace{0.3cm}
\textbf{10.} \textit{and they carried} \hspace{0.3cm}
\textbf{11.} \textit{and she went up}, \textit{and you went} m.\ sg.\ \foreignlanguage{hebrew}{וַתַּ֫עַל} \hspace{0.3cm}
\textbf{12.} \textit{you answered} \hspace{0.3cm}
\textbf{13.} \textit{someone calling} \hspace{0.3cm}
\textbf{14.} \textit{drink!} \hspace{0.3cm}
\textbf{15.} \textit{we will drink} \hspace{0.3cm}


\subsection*{Translation of Sentences}

\textbf{1.}~~Gen 26:30 \hspace{0.2cm}
\textbf{2.}~~Gen 4:9 \hspace{0.2cm}
\textbf{3.}~~1\,Sam 10:24 \hspace{0.2cm}
\textbf{4.}~~2\,Kgs 2:3 \hspace{0.2cm}
\textbf{5.}~~Josh 7:20 \hspace{0.2cm}
\textbf{6.}~~1\,Sam 29:9 \hspace{0.2cm}
\textbf{7.}~~1\,Sam 11:1 \hspace{0.2cm}
\textbf{8.}~~Gen 50:7  \hspace{0.2cm}
\textbf{9.}~~2\,Kgs 22:8  \hspace{0.2cm}
\textbf{10.}~~Josh 5:13  \hspace{0.2cm}

\section*{Chapter 13}

\subsection*{Translation of Verbal Forms}

\textbf{1.} \textit{I will go} \hspace{0.3cm}
\textbf{2.} \textit{we will go} \hspace{0.3cm}
\textbf{3.} \textit{you will go} m.\ pl. \hspace{0.3cm}
\textbf{4.} \textit{go!} m.\ sg. \hspace{0.3cm}
\textbf{5.} \textit{she will be unclean}/\textit{you will be unclean} fem.\ sg./m.\ sg. \hspace{0.3cm}
\textbf{6.} \textit{I went out} \foreignlanguage{hebrew}{יָצָ֫אתִי} \hspace{0.3cm}
\textbf{7.} \textit{they went out} \hspace{0.3cm}
\textbf{8.} \textit{they will go out} \hspace{0.3cm}
\textbf{9.} \textit{people going out} m.\ pl. \hspace{0.3cm}
\textbf{10.} \textit{we took possession of} \hspace{0.3cm}
\textbf{11.} \textit{you will take possession of} m.\ pl. \hspace{0.3cm}
\textbf{12.} \textit{and they took possession of answered} m.\ pl. \hspace{0.3cm}
\textbf{13.} \textit{I am/was afraid} \foreignlanguage{hebrew}{יָרֵ֫אתִי} \hspace{0.3cm}
\textbf{14.} \textit{you will be afraid}  m.\ pl.  \hspace{0.3cm}
\textbf{15.} \textit{you will see}  m.\ pl.  \hspace{0.3cm}
\textbf{16.} \textit{they sat down}  \hspace{0.3cm}
\textbf{17.} \textit{and they sat down} \hspace{0.3cm}
\textbf{18.} \textit{sit down!} m.\ sg. \hspace{0.3cm}
\textbf{19.} \textit{sit down!} m.\ pl. \hspace{0.3cm}
\textbf{20.} \textit{you went down} m.\ sg. \foreignlanguage{hebrew}{יָרַ֫דְתָּ} \hspace{0.3cm}



\subsection*{Translation of Sentences}
\textbf{1.}~~Josh 3:6 \hspace{0.2cm}
\textbf{2.}~~Josh 6:6 \hspace{0.2cm}
\textbf{3.}~~Num 12:4 \hspace{0.2cm}
\textbf{4.}~~Exod 19:20 \hspace{0.2cm}
\textbf{5.}~~Gen 19:30 \hspace{0.2cm}
\textbf{6.}~~Judg 21:2 \hspace{0.2cm}
\textbf{7.}~~1\,Sam 23:17 \hspace{0.2cm}
\textbf{8.}~~1\,Kgs 2:36  \hspace{0.2cm}
\textbf{9.}~~Gen 11:4--5  \hspace{0.2cm}
\textbf{10.}~~Isa 6:8  \hspace{0.2cm}
\textbf{11.}~~1\,Kgs 22:6  \hspace{0.2cm}
\textbf{12.}~~Gen 4:17  \hspace{0.2cm}
\textbf{13.}~~Num 19:22  \hspace{0.2cm}
\textbf{14.}~~2\,Sam 7:27 \hspace{0.2cm}
\textbf{15.}~~Num 20:14-18  \hspace{0.2cm}
\textbf{16.}~~Judg 2:14  \hspace{0.2cm}
\textbf{17.}~~2\,Sam 13:21 \hspace{0.2cm}
\textbf{18.}~~1\,Sam 6:20 \hspace{0.2cm}

\section*{Chapter 14}

\subsection*{Translation of Verbal Forms}
\textbf{1.} \textit{they shall be ashamed} (m.\ pl.) \hspace{0.15cm} \foreignlanguage{hebrew}{יֵבֹ֫שׁוּ} \hspace{0.3cm}
\textbf{2.} \textit{we were ashamed} \hspace{0.15cm} \foreignlanguage{hebrew}{בֹּ֫שְׁנוּ} \hspace{0.3cm}
\textbf{3.} \textit{and they understood} (m.\ pl.) \hspace{0.15cm} \foreignlanguage{hebrew}{וַיָּבִ֫ינוּ} \hspace{0.3cm}
\textbf{4.} \textit{come!} (f.\ sg.) \hspace{0.15cm} \foreignlanguage{hebrew}{בֹּ֫אִי} \hspace{0.3cm}
\textbf{5.} \textit{I have sojourned} \hspace{0.15cm} \foreignlanguage{hebrew}{גַּ֫רְתִּי} \hspace{0.3cm}
\textbf{6.} \textit{sojourner} (part.) or \textit{he sojourned} (SC 3 m.\ sg.) \hspace{0.3cm}
\textbf{7.} \textit{you will die} (m.\ sg.) or \textit{she will die} \hspace{0.3cm}
\textbf{8.} \textit{and they fled} (m.\ pl.) \hspace{0.3cm}
\textbf{9.} \textit{turn aside!} (m.\ sg.) \hspace{0.15cm} \foreignlanguage{hebrew}{ס֫וּרָה} \hspace{0.3cm}
\textbf{10.} \textit{turn aside!} (m.\ sg.) or \textit{turning aside} (inf.\ cs.) \hspace{0.15cm} \textbf{11.} \textit{they stood up} \hspace{0.15cm} \foreignlanguage{hebrew}{קָ֫מוּ} \hspace{0.3cm}
\textbf{12.} \textit{and he stood up} \hspace{0.15cm} \foreignlanguage{hebrew}{וַיָּ֫קָם} \hspace{0.3cm}
\textbf{13.} \textit{runners} (m.\ pl.) \hspace{0.3cm}
\textbf{14.} \textit{you ran} (m.\ sg.) \hspace{0.15cm} \foreignlanguage{hebrew}{רַ֫צְתָּה} \hspace{0.3cm}
\textbf{15.} \textit{put!} (m.\ pl.) \hspace{0.15cm} \foreignlanguage{hebrew}{שִׂ֫ימוּ} \hspace{0.3cm}
\textbf{16.} \textit{you put} (m.\ pl.) \hspace{0.3cm}
\textbf{17.} \textit{and he returned} \hspace{0.15cm} \foreignlanguage{hebrew}{וַיָּ֫שָׁב} \hspace{0.3cm}
\textbf{18.} \textit{and I returned} \hspace{0.3cm}

\subsection*{Numerals}

Ezra 2:3~~\foreignlanguage{hebrew}{פַּרְעֹשׁ} \hspace{0.1cm} 2,172 \hspace{0.3cm}
Ezra 2:4~~\foreignlanguage{hebrew}{שְׁפַטְיָה} \hspace{0.1cm} 372 \hspace{0.3cm}
Ezra 2:5~~\foreignlanguage{hebrew}{פַּחַת מוֹאָב} \hspace{0.1cm} 2,812 \hspace{0.3cm}
Ezra 2:7~~\hspace{0.1cm} \foreignlanguage{hebrew}{עֵילָם} \hspace{0.1cm} 1,254 \hspace{0.3cm}
Ezra 2:8~~\foreignlanguage{hebrew}{זַתּוּא} \hspace{0.1cm} 945 \hspace{0.3cm}
Ezra 2:9~~\foreignlanguage{hebrew}{זַכָּי} \hspace{0.1cm} 760 \hspace{0.3cm}
Ezra 2:10~~\foreignlanguage{hebrew}{בָּנִי} \hspace{0.1cm} 642 \hspace{0.3cm}
Ezra 2:11~~\foreignlanguage{hebrew}{בֵבָי} \hspace{0.1cm} 623 \hspace{0.3cm}
Ezra 2:12~~\foreignlanguage{hebrew}{עַזְגָּד} \hspace{0.1cm} 1,222 \hspace{0.3cm}
Ezra 2:13~~\foreignlanguage{hebrew}{אֲדֹנִיקָם} \hspace{0.1cm} 666 \hspace{0.3cm}
Ezra 2:14~~\foreignlanguage{hebrew}{בִּגְוָי} \hspace{0.1cm}2,056 \hspace{0.3cm}
Ezra 2:15~~\foreignlanguage{hebrew}{עָדִין} \hspace{0.1cm} 454 \hspace{0.3cm}
Ezra 2:16~~\foreignlanguage{hebrew}{אָטֵר} \hspace{0.1cm} 98 \hspace{0.3cm}
Ezra 2:17~~\foreignlanguage{hebrew}{בֵּצָי} \hspace{0.1cm} 323 \hspace{0.3cm}


\subsection*{Translation of Sentences}
\textbf{1.}~~1\,Sam 22:11 \hspace{0.2cm}
\textbf{2.}~~2\,Sam 12:19 \hspace{0.2cm}
\textbf{3.}~~2\,Kgs 16:11 \hspace{0.2cm}
\textbf{4.}~~1\,Sam 26:5 \hspace{0.2cm}
\textbf{5.}~~1\,Kgs 17:22 \hspace{0.2cm}
\textbf{6.}~~2\,Kgs 1:4 \hspace{0.2cm}
\textbf{7.}~~1\,Sam 15:26 \hspace{0.2cm}
\textbf{8.}~~Josh 24:29  \hspace{0.2cm}
\textbf{9.}~~Exod 14:26  \hspace{0.2cm}
\textbf{10.}~~Josh 3:12 \hspace{0.2cm}
\textbf{11.}~~Gen 2:16--17  \hspace{0.2cm}
\textbf{12.}~~Ruth 1:1--7  \hspace{0.2cm}


\section*{Chapter 15}

\subsection*{Translation of Verbal Forms}
\textbf{1.} \textit{and they measured} (m.\ pl.) \hspace{0.15cm} \foreignlanguage{hebrew}{וַיָּמֹ֫דּוּ} \hspace{0.3cm}
\textbf{2.} \textit{you will measure} (m.\ pl.) \hspace{0.15cm} \foreignlanguage{hebrew}{תָּמֹ֫דּוּ} \hspace{0.3cm}
\textbf{3.} \textit{and they surrounded} (m.\ pl.) \hspace{0.15cm} \foreignlanguage{hebrew}{וַיָּסֹ֫בּוּ} \hspace{0.3cm}
\textbf{4.} \textit{and he surrounded} \hspace{0.3cm}
\textbf{5.} \textit{go around!} (m.\ pl.) \hspace{0.15cm} \foreignlanguage{hebrew}{סֹ֫בּוּ} \hspace{0.3cm}
\textbf{6.} \textit{and you shall go around} (m.\ pl.) \hspace{0.3cm}
\textbf{7.} \textit{and he went around} \hspace{0.3cm}
\textbf{8.} \textit{they will surround} (fem.\ pl.) \hspace{0.15cm} \foreignlanguage{hebrew}{תְסֻבֶּ֫ינָה} \hspace{0.3cm}
\textbf{9.} \textit{and we went around} (m.\ pl.) \hspace{0.15cm} \foreignlanguage{hebrew}{וַנָּ֫סָב} \hspace{0.3cm}
\textbf{10.} \textit{someone devastating} (m.\ sg.) \hspace{0.3cm}
\textbf{11.} \textit{he will be desolate} \foreignlanguage{hebrew}{} \hspace{0.3cm}
\textbf{12.} \textit{they will be desolate} (m.\ pl.) \hspace{0.15cm} \foreignlanguage{hebrew}{יָשֹׁ֫מּוּ} \hspace{0.3cm}
\textbf{13.} \textit{they will be complete} (m.\ pl.) \hspace{0.15cm} \foreignlanguage{hebrew}{יִתַּ֫מּוּ} \hspace{0.3cm}
\textbf{14.} \textit{they were complete} (m.\ pl.) \hspace{0.15cm} \foreignlanguage{hebrew}{תַּ֫מּוּ} \hspace{0.3cm}
\textbf{15.} \textit{he was complete} \hspace{0.3cm}
\textbf{16.} \textit{he will show favor} \hspace{0.3cm}
\textbf{17.} \textit{and he showed favor} \hspace{0.15cm} \foreignlanguage{hebrew}{וַיָּ֫חָן} \hspace{0.3cm}
\textbf{18.} \textit{I will show favor} \hspace{0.3cm}
\textbf{19.} \textit{and I will show favor} (m.\ pl.) \hspace{0.15cm} \foreignlanguage{hebrew}{} \hspace{0.3cm}
\textbf{20.} \textit{have mercy on me!} (m.\ sg.) \hspace{0.15cm} \foreignlanguage{hebrew}{חָנֵּ֫נִי} \hspace{0.3cm}


\subsection*{Translation of Sentences}
\textbf{1.}~~Ps 30:11 \hspace{0.2cm}
\textbf{2.}~~Josh 4:1--2 \hspace{0.2cm}
\textbf{3.}~~Josh 6:2--4 \hspace{0.2cm}
\textbf{4.}~~Judg 4:8 \hspace{0.2cm}
\textbf{5.}~~Exod 19:5--6 \hspace{0.2cm}
\textbf{6.}~~1\,Kgs 13:8 \hspace{0.2cm}
\textbf{7.}~~Gen 28:20--21 \hspace{0.2cm}
\textbf{8.}~~1\,Sam 7:12  \hspace{0.2cm}
\textbf{9.}~~Deut 5:3  \hspace{0.2cm}
\textbf{10.}~~1\,Kgs 19:9 \hspace{0.2cm}


\section*{Chapter 16}

\subsection*{Translation of Verbal Forms}
\textbf{1.} \textit{he fought} \hspace{0.3cm}
\textbf{2.} \textit{they fought} (m.\ pl.) \hspace{0.3cm}
\textbf{3.} \textit{they fought} (m.\ pl.) \hspace{0.15cm} \foreignlanguage{hebrew}{נִלְחָ֫מוּ} \hspace{0.3cm}
\textbf{4.} \textit{I fought} \hspace{0.3cm}
\textbf{5.} \textit{he will fight} \hspace{0.3cm}
\textbf{6.} \textit{and he fought} \foreignlanguage{hebrew}{וַיִּלָּ֫חֶם} \hspace{0.3cm}
\textbf{7.} \textit{fight!} (m.\ pl.)  \hspace{0.3cm}
\textbf{8.} \textit{(we will fight} \hspace{0.3cm}
\textbf{9.} \textit{you will fight} (m.\ pl.)\hspace{0.3cm}
\textbf{10.} \textit{fight!} (m.\ pl..) \hspace{0.3cm}
\textbf{11.} \textit{I escaped} \foreignlanguage{hebrew}{נִמְלַ֫טְתִּי} \hspace{0.3cm}
\textbf{12.} \textit{they escaped} (m.\ pl.) \hspace{0.3cm}
\textbf{13.} \textit{I will escape} \hspace{0.3cm}
\textbf{14.} \textit{she will escape}/\textit{you will escape} (m.\ sg.) \hspace{0.3cm}
\textbf{15.} \textit{you swore} (m.\ sg.) \foreignlanguage{hebrew}{נִשְׁבַּ֫עְתָּ} \hspace{0.3cm}
\textbf{16.} \textit{you swore} (m.\ pl.) \hspace{0.3cm}
\textbf{17.} \textit{we swore} \hspace{0.15cm} \foreignlanguage{hebrew}{וַיָּ֫חָן} \hspace{0.3cm}
\textbf{18.} \textit{I swore} \hspace{0.3cm}
\textbf{19.} \textit{you will swear} (m.\ pl.) \hspace{0.15cm} \foreignlanguage{hebrew}{} \hspace{0.3cm}
\textbf{20.} \textit{and he swore} (m.\ sg.) \hspace{0.15cm} \foreignlanguage{hebrew}{חָנֵּ֫נִי}

\subsection*{Translation of Sentences}
\textbf{1.}~~Deut 1:34--36a \hspace{0.2cm}
\textbf{2.}~~1\,Sam 24:7 \hspace{0.2cm}
\textbf{3.}~~2\,Sam 11:11 \hspace{0.2cm}
\textbf{4.}~~1\,Kgs 18:40 \hspace{0.2cm}
\textbf{5.}~~1\,Sam 17:55 \hspace{0.2cm}
\textbf{6.}~~Josh 21:43 \hspace{0.2cm}
\textbf{7.}~~Josh 24:14--16 \hspace{0.2cm}
\textbf{8.}~~1\,Kgs 12:22--24  \hspace{0.2cm}
\textbf{9.}~~2\,Kgs 12:18--19  \hspace{0.2cm}
\textbf{10.}~~Gen 44:17 \hspace{0.2cm}

\section*{Chapter 17}

\subsection*{Translation of Verbal Forms}
\textbf{1.} \textit{they rose early} \hspace{0.15cm} \foreignlanguage{hebrew}{הִשְׁכִּ֫ימוּ} \hspace{0.3cm}
\textbf{2.} \textit{they will believe} \hspace{0.15cm} \foreignlanguage{hebrew}{יַאֲמִ֫ינוּ} \hspace{0.3cm}
\textbf{3.} \textit{[people] destroying} \hspace{0.3cm}
\textbf{4.} \textit{he believed} \hspace{0.3cm}
\textbf{5.} \textit{I will throw} \hspace{0.3cm}
\textbf{6.} \textit{and he rose early} \hspace{0.3cm}
\textbf{7.} \textit{they believed} \hspace{0.15cm} \foreignlanguage{hebrew}{הַאֲמִ֫ינוּ} \hspace{0.3cm}
\textbf{8.} \textit{believe!} (impv.\ masc.\ pl.) \hspace{0.15cm} \foreignlanguage{hebrew}{הַאֲמִ֫ינוּ} \hspace{0.3cm}
\textbf{9.} \textit{she will throw/you will throw} (masc.\ sg.) \hspace{0.3cm}
\textbf{10.} \textit{she threw} \hspace{0.15cm} \foreignlanguage{hebrew}{הִשְׁלִ֫יכָה} \hspace{0.3cm}
\textbf{11.} \textit{throw!/throwing} (inf.\ cs.) \hspace{0.3cm}
\textbf{12.} \textit{and they threw} \hspace{0.15cm} \foreignlanguage{hebrew}{וַיַּשְׁלִ֫יכוּ} \hspace{0.3cm}
\textbf{13.} \textit{I threw} \hspace{0.15cm} \foreignlanguage{hebrew}{הִשְׁלַ֫כְתִּי} \hspace{0.3cm}
\textbf{14.} \textit{you believed} \hspace{0.3cm}
\textbf{15.} \textit{I cut off} \hspace{0.15cm} \foreignlanguage{hebrew}{הִכְרַ֫תִּי} \hspace{0.3cm}


\subsection*{Translation of Sentences}
\textbf{1.}~~Josh 10:11 \hspace{0.2cm}
\textbf{2.}~~Deut 4:3 \hspace{0.2cm}
\textbf{3.}~~Jer 28:15 \hspace{0.2cm}
\textbf{4.}~~Exod 22:30 \hspace{0.2cm}
\textbf{5.}~~1\,Sam 24:21--23 \hspace{0.2cm}
\textbf{6.}~~Gen 19:14 \hspace{0.2cm}
\textbf{7.}~~Num 27:22--23 \hspace{0.2cm}
\textbf{8.}~~2\,Kgs 10:28  \hspace{0.2cm}
\textbf{9.}~~1\,Kgs 16:12  \hspace{0.2cm}
\textbf{10.}~~1\,Sam 26:15--16a \hspace{0.2cm}
\textbf{11.}~~Exod 4:1--3  \hspace{0.2cm}
\textbf{12.}~~2\,Chr 20:20  \hspace{0.2cm}


\section*{Chapter 18}

\subsection*{Translation of Verbal Forms}
\textbf{1.} 3 masc.\ sg.\ PC Ho.\ \textit{he will be put under a ban} \hspace{0.3cm}
\textbf{2.} 3 masc.\ sg.\ PC Ho.\ \textit{he was visited} \hspace{0.3cm}
\textbf{3.} 3 masc.\ sg.\ PC Ho.\ \textit{he shall be placed} \hspace{0.3cm}
\textbf{4.} ptc.\ masc.\ sg.\ Ho.\ \hspace{0.3cm}
\textbf{5.} ptc.\ fem.\ sg.\ Ho.\ \hspace{0.3cm}
\textbf{6.} ptc.\ masc.\ sg.\ Ho.\ \hspace{0.3cm}
\textbf{7.} ptc.\ masc.\ sg.\ Ho.\ \hspace{0.3cm}
\textbf{8.} 1 comm.\ sg.\ SC Ho.\ \textit{I was thrown} \hspace{0.15cm} \foreignlanguage{hebrew}{הָשְׁל֫כְתִּי}  \hspace{0.3cm}
\textbf{9.} ptc.\ masc.\ sg.\ Ho.\ \hspace{0.3cm}
\textbf{10.} 3 masc.\ sg.\ PC Ho.\ \textit{they will be thrown} \hspace{0.15cm} \foreignlanguage{hebrew}{יֻשְׁלָ֑֫כוּ} \hspace{0.3cm}


\subsection*{Translation of Sentences}
\textbf{1.}~~Lev 16:10 \hspace{0.2cm}
\textbf{2.}~~2\,Kgs 4:32 \hspace{0.2cm}
\textbf{3.}~~2\,Sam 19:1 \hspace{0.2cm}
\textbf{4.}~~Ps 14:7 \hspace{0.2cm}
\textbf{5.}~~2\,Kgs 10:14  \hspace{0.2cm}
\textbf{6.}~~2\,Sam 19:16 \hspace{0.2cm}
\textbf{7.}~~Deut 23:16--17 \hspace{0.2cm}
\textbf{8.}~~Josh 6:21  \hspace{0.2cm}
\textbf{9.}~~Hos 14:2  \hspace{0.2cm}
\textbf{10.}~~1\,Sam 18:5 \hspace{0.2cm}
\textbf{11.}~~Gen 39:2--3  \hspace{0.2cm}
\textbf{12.}~~2\,Kgs 8:16--19  \hspace{0.2cm}


\section*{Chapter 19}

\subsection*{Translation of Verbal Forms}
\textbf{1.} \textit{you spoke} \hspace{0.15cm} \foreignlanguage{hebrew}{דִּבַּ֫רְתָּ} \hspace{0.3cm}
\textbf{2.} \textit{you will speak} \hspace{0.3cm}
\textbf{3.} \textit{they spoke} \hspace{0.3cm}
\textbf{4.} \textit{they will speak} \hspace{0.15cm} \foreignlanguage{hebrew}{יְדַבֵּ֑֫רוּ} \hspace{0.3cm}
\textbf{5.} \textit{I spoke} \hspace{0.15cm} \foreignlanguage{hebrew}{דִּבַּ֫רְתִּי} \hspace{0.3cm}
\textbf{6.} \textit{you will speak} (masc.\ pl.) \hspace{0.3cm}
\textbf{7.} \textit{you hasten} \hspace{0.3cm}
\textbf{8.} \textit{I will speak} \hspace{0.3cm}
\textbf{9.} \textit{to speak} (inf.\ cs.) \hspace{0.3cm}
\textbf{10.} \textit{speak!} (impv.\ masc.\ pl.) \hspace{0.3cm}
\textbf{11.} \textit{someone speaking} (ptc. masc.\ sg.) \hspace{0.3cm}
\textbf{12.} \textit{we spoke} \hspace{0.15cm} \foreignlanguage{hebrew}{דִּבַּ֫רְנוּ} \hspace{0.3cm}
\textbf{13.} \textit{he will speak} \hspace{0.3cm}
\textbf{14.} \textit{you taught} \hspace{0.3cm}
\textbf{15.} \textit{they will serve} \hspace{0.15cm} \foreignlanguage{hebrew}{} \hspace{0.3cm}
\textbf{16.} \textit{they/you will speak} (fem.\ pl.) \hspace{0.15cm} \foreignlanguage{hebrew}{תְּדַבֵּ֫רְנָה} \hspace{0.3cm}
\textbf{17.} \textit{you will bless} \hspace{0.3cm}
\textbf{18.} \textit{and they spoke} \hspace{0.15cm} \foreignlanguage{hebrew}{‎וַיְדַבֵּ֑֫רוּ} \hspace{0.3cm}
\textbf{19.} \textit{and he blessed} \hspace{0.15cm} \foreignlanguage{hebrew}{וַיְבָ֫רֶךְ} \hspace{0.3cm}
\textbf{20.} \textit{and they sought} \hspace{0.3cm}


\subsection*{Translation of Sentences}
\textbf{1.}~~Exod 2:15 \hspace{0.2cm}
\textbf{2.}~~Exod 4:19 \hspace{0.2cm}
\textbf{3.}~~Judg 6:29 \hspace{0.2cm}
\textbf{4.}~~1\,Sam 26:2 \hspace{0.2cm}
\textbf{5.}~~Gen 17:15--16  \hspace{0.2cm}
\textbf{6.}~~Judg 13:24 \hspace{0.2cm}
\textbf{7.}~~Exod 19:10 \hspace{0.2cm}
\textbf{8.}~~Exod 19:14  \hspace{0.2cm}
\textbf{9.}~~Exod 5:1--2  \hspace{0.2cm}
\textbf{10.}~~1\,Sam 13:2 \hspace{0.2cm}

\section*{Chapter 20}

\subsection*{Translation of Verbal Forms}
\textbf{1.} \textit{they were sent} \hspace{0.3cm}
\textbf{2.} \textit{she will be blessed} (fem.\ sg.)/\textit{you will be blessed} (masc.\ sg.) \hspace{0.3cm}
\textbf{3.} \textit{he will be sought} \hspace{0.3cm}
\textbf{4.} \textit{you will be sought} (fem.\ sg.) \hspace{0.3cm}
\textbf{5.} \textit{he will be sent} \hspace{0.15cm} \foreignlanguage{hebrew}{} \hspace{0.3cm}
\textbf{6.} \textit{she will be cursed} (fem.\ sg.)/\textit{you will be cursed} (masc.\ sg.) \hspace{0.3cm}
\textbf{7.} \textit{he walked around/walk about!} (impv.\ masc.\ sg.) \hspace{0.3cm}
\textbf{8.} \textit{I walked about} \hspace{0.15cm} \foreignlanguage{hebrew}{\foreignlanguage{hebrew}{הִתְהַלַּ֫כְתִּי}} \hspace{0.3cm}
\textbf{9.} \textit{we walked about} \hspace{0.15cm} \foreignlanguage{hebrew}{הִתְהַלַּ֫כְנוּ} \hspace{0.3cm}
\textbf{10.} \textit{they walked about} \hspace{0.3cm}
\textbf{11.} \textit{and they walked about} \hspace{0.3cm}
\textbf{12.} \textit{he will pray} \hspace{0.3cm}
\textbf{13.} \textit{and she prayed} (fem.\ sg.)/\textit{and you prayed} (masc.\ sg.) \hspace{0.3cm}
\textbf{14.} \textit{I prayed} \hspace{0.15cm} \foreignlanguage{hebrew}{הִתְפַּלָּ֫לְתִּי} \hspace{0.3cm}
\textbf{15.} \textit{they sanctified themselves}/\textit{sanctify yourselves} (impv. masc.\ pl.) \hspace{0.3cm}


\subsection*{Translation of Sentences}
\textbf{1.}~~Ps 113:2 \hspace{0.2cm}
\textbf{2.}~~2\,Chr 26:16b--18 \hspace{0.2cm}
\textbf{3.}~~1\,Kgs 21:14--16 \hspace{0.2cm}
\textbf{4.}~~Deut 9:20 Qere \hspace{0.2cm}
\textbf{5.}~~1\,Sam 4:9  \hspace{0.2cm}
\textbf{6.}~~1\,Sam 7:5 \hspace{0.2cm}
\textbf{7.}~~1\,Sam 12:19 \hspace{0.2cm}
\textbf{8.}~~2\,Kgs 6:17  \hspace{0.2cm}
\textbf{9.}~~1\,Kgs 20:22  \hspace{0.2cm}
\textbf{10.}~~2\,Kgs 19:15 \hspace{0.2cm}
\textbf{11.}~~Gen 5:24  \hspace{0.2cm}
\textbf{12.}~~2\,Sam 13:20  \hspace{0.2cm}


\section*{Chapter 21}

\subsection*{Translation of Verbal Forms}
\textbf{1.} \textit{they looked} \hspace{0.15cm} \foreignlanguage{hebrew}{הִבִּ֫יטוּ} \hspace{0.3cm}
\textbf{2.} \textit{you looked} (m.\ pl.) \hspace{0.3cm}
\textbf{3.} \textit{I will look} \hspace{0.3cm}
\textbf{4.} \textit{they will look} (m.\ pl.) \hspace{0.15cm} \foreignlanguage{hebrew}{יַבִּ֫יטוּ} \hspace{0.3cm}
\textbf{5.} \textit{I told} \hspace{0.3cm}
\textbf{6.} \textit{they told}  \hspace{0.15cm} \foreignlanguage{hebrew}{הִגִּ֫ידוּ} \hspace{0.3cm}
\textbf{7.} \textit{you told} (m.\ sg.) \hspace{0.15cm} \foreignlanguage{hebrew}{הִגַּ֫דְתָּ} \hspace{0.3cm}
\textbf{8.} \textit{(she told} \hspace{0.15cm} \foreignlanguage{hebrew}{הִגִּ֫ידָה} \hspace{0.3cm}
\textbf{9.} \textit{and he told} \hspace{0.3cm}
\textbf{10.} \textit{you will tell} (f.\ sg.) \hspace{0.15cm} \foreignlanguage{hebrew}{תַּגִּ֫ידִי} \hspace{0.3cm}
\textbf{11.} \textit{you will tell} (m.\ pl.) \hspace{0.15cm} \foreignlanguage{hebrew}{תַּגִּ֫ידוּ} \hspace{0.3cm}
\textbf{12.} \textit{I will tell} \hspace{0.3cm}
\textbf{13.} \textit{we will tell} \hspace{0.3cm}
\textbf{14.} \textit{and they told}/ (m.\ pl.) \hspace{0.15cm} \foreignlanguage{hebrew}{וַיַּגִּ֫דוּ} \hspace{0.3cm}
\textbf{15.} \textit{tell!} (m.\ sg.) \hspace{0.15cm} \foreignlanguage{hebrew}{‎הַגִּ֫ידָה} \hspace{0.3cm}
\textbf{16.} \textit{tell!} (m.\ sg.) \hspace{0.3cm}
\textbf{17.} \textit{I regretted} (Ni.)/\textit{I comforted} (Pi.) \hspace{0.15cm} \foreignlanguage{hebrew}{‎נִחַ֫מְתִּי} \hspace{0.3cm}
\textbf{18.} \textit{they will comfort} \hspace{0.3cm}
\textbf{19.} \textit{and he comforted} \hspace{0.3cm}
\textbf{20.} \textit{they rescued} \hspace{0.15cm} \foreignlanguage{hebrew}{הִצִּ֫ילוּ} \hspace{0.3cm}
\textbf{21.} \textit{we rescued} \hspace{0.15cm} \foreignlanguage{hebrew}{הִצַּ֫לְנוּ} \hspace{0.3cm}
\textbf{22.} \textit{and he rescued} \hspace{0.3cm}
\textbf{23.} \textit{rescue!} (m.\ pl.) \hspace{0.15cm} \foreignlanguage{hebrew}{הַצִּ֫ילוּ} \hspace{0.3cm}
\textbf{24.} \textit{they will reach} \hspace{0.15cm} \foreignlanguage{hebrew}{יַשִּׂ֫יגוּ}



\subsection*{Translation of Sentences}
\textbf{1.}~~Exod 3:6 \hspace{0.2cm}
\textbf{2.}~~Gen 12:18 \hspace{0.2cm}
\textbf{3.}~~Gen 46:31 \hspace{0.2cm}
\textbf{4.}~~Gen 47:1 \hspace{0.2cm}
\textbf{5.}~~Judg 14:1--3  \hspace{0.2cm}
\textbf{6.}~~2\,Sam 12:18--19 \hspace{0.2cm}
\textbf{7.}~~Gen 22:20  \hspace{0.2cm}
\textbf{8.}~~Gen 31:22 \hspace{0.2cm}
\textbf{9.}~~1\,Sam 15:10--11  \hspace{0.2cm}
\textbf{10.}~~Gen 6:5--6 \hspace{0.2cm}
\textbf{11.}~~Exod 17:9  \hspace{0.2cm}
\textbf{12.}~~1\,Sam 17:37  \hspace{0.2cm}
\textbf{13.}~~1\,Sam 30:8  \hspace{0.2cm}
\textbf{14.}~~2\,Sam 12:24 Qere \hspace{0.2cm}

\section*{Chapter 22}

\subsection*{Translation of Verbal Forms}
\textbf{1.} \textit{he finished} \hspace{0.3cm}
\textbf{2.} \textit{they finished}  \hspace{0.3cm}
\textbf{3.} \textit{you finished} (masc.\ pl.) \hspace{0.3cm}
\textbf{4.} \textit{I finished} \hspace{0.15cm} \foreignlanguage{hebrew}{כִּלִּ֫יתִי} \hspace{0.3cm}
\textbf{5.} \textit{I will finish} \hspace{0.3cm}
\textbf{6.} \textit{they will finish} \hspace{0.3cm}
\textbf{7.} \textit{she covered} \hspace{0.3cm}
\textbf{8.} \textit{you will cover} (fem.\ sg.) \hspace{0.3cm}
\textbf{9.} \textit{he prophecied, someone prophecying} \hspace{0.3cm}
\textbf{10.} \textit{they prophecied} \hspace{0.3cm}
\textbf{11.} \textit{you prophecied} \hspace{0.15cm} \foreignlanguage{hebrew}{נִבֵּ֫אתָ} \hspace{0.3cm}
\textbf{12.} \textit{she will prophecy} (fem.\ sg.)/\textit{you will prophecy} (masc.\ sg.) \hspace{0.3cm}
\textbf{13.} \textit{you struck} \hspace{0.15cm} \foreignlanguage{hebrew}{הִכִּ֫יתָ} \hspace{0.3cm}
\textbf{14.} \textit{he struck} \hspace{0.3cm}
\textbf{15.} \textit{they struck} \hspace{0.3cm}
\textbf{16.} \textit{I struck} \hspace{0.15cm} \foreignlanguage{hebrew}{הִכֵּ֫יתִי} \hspace{0.3cm}
\textbf{17.} \textit{you struck} (masc.\ pl.) \hspace{0.3cm}
\textbf{18.} \textit{he will strike} \hspace{0.3cm}
\textbf{19.} \textit{they will strike} \hspace{0.3cm}
\textbf{20.} \textit{I will strike} \hspace{0.3cm}
\textbf{21.} \textit{and they struck} (masc.\ pl.) \hspace{0.3cm}
\textbf{22.} \textit{and we struck} \hspace{0.3cm}
\textbf{23.} \textit{they appeared} \hspace{0.3cm}
\textbf{24.} \textit{they will appear} (masc.\ pl.) \hspace{0.3cm}
\textbf{25.} \textit{he showed} \hspace{0.3cm}
\textbf{26.} \textit{I commanded} \hspace{0.15cm} \foreignlanguage{hebrew}{צִוִּ֫יתִי} \hspace{0.3cm}
\textbf{27.} \textit{you commanded} \hspace{0.15cm} \foreignlanguage{hebrew}{צִוִּ֫יתָ} \hspace{0.3cm}
\textbf{28.} \textit{and he commanded} \hspace{0.3cm}
\textbf{29.} \textit{they brought up} \hspace{0.3cm}
\textbf{30.} \textit{I brought up} \hspace{0.15cm} \foreignlanguage{hebrew}{הֶעֱלֵ֫יתִי} \hspace{0.3cm}


\subsection*{Translation of Sentences}
\textbf{1.}~~Exod 7:6 \hspace{0.2cm}
\textbf{2.}~~Exod 4:28 \hspace{0.2cm}
\textbf{3.}~~1 Kgs 6:14 \hspace{0.2cm}
\textbf{4.}~~Gen 2:2 \hspace{0.2cm}
\textbf{5.}~~1\,Sam 19:20 \hspace{0.2cm}
\textbf{6.}~~Judg 1:8 \hspace{0.2cm}
\textbf{7.}~~Deut 8:1  \hspace{0.2cm}
\textbf{8.}~~1\,Sam 13:14 \hspace{0.2cm}
\textbf{9.}~~Gen 26:24--25  \hspace{0.2cm}
\textbf{10.}~~Exod 24:15--18 \hspace{0.2cm}
\textbf{11.}~~2\,Kings 19:1--2  \hspace{0.2cm}
\textbf{12.}~~Jer 25:15, 17  \hspace{0.2cm}
\textbf{13.}~~1\,Sam 10:18  \hspace{0.2cm}
\textbf{14.}~~2\,Sam 6:12 \hspace{0.2cm}


\section*{Chapter 23}

\subsection*{Translation of Verbal Forms}
\textbf{1.} \textit{I will praise} \hspace{0.3cm}
\textbf{2.} \textit{we will praise} \hspace{0.3cm}
\textbf{3.} \textit{they confessed/confess!} (impv.\ masc.\ pl.) \hspace{0.3cm}
\textbf{4.} \textit{they did good/do good!} (impv.\ masc.\ pl.) \hspace{0.15cm} \foreignlanguage{hebrew}{הֵיטִ֫יבוּ} \hspace{0.3cm}
\textbf{5.} \textit{you did good} (masc.\ sg.) \hspace{0.15cm} \foreignlanguage{hebrew}{הֵיטַ֫בְתָּ} \hspace{0.3cm}
\textbf{6.} \textit{he will do good} \hspace{0.3cm}
\textbf{7.} \textit{I added} \hspace{0.15cm} \foreignlanguage{hebrew}{הוֹסַ֫פְתִּי} \hspace{0.3cm}
\textbf{8.} \textit{you will add} (masc.\ pl.) \hspace{0.15cm} \foreignlanguage{hebrew}{תֹסִ֫יפוּ} \hspace{0.3cm}
\textbf{9.} \textit{they will add}  (masc.\ pl.) \hspace{0.15cm} \foreignlanguage{hebrew}{יוֹסִ֫פוּ} \hspace{0.3cm}
\textbf{10.} \textit{they took their stand/take your stand} \hspace{0.3cm}
\textbf{11.} \textit{he will take his stand} \hspace{0.3cm}
\textbf{12.} \textit{you rescued} \hspace{0.3cm}
\textbf{13.} \textit{I rescued} \hspace{0.15cm} \foreignlanguage{hebrew}{הוֹשַׁ֫עְתִּי} \hspace{0.3cm}
\textbf{14.} \textit{she will rescue} (fem.\ sg.)/\textit{you will rescue} (masc.\ sg.) \hspace{0.15cm} \foreignlanguage{hebrew}{} \hspace{0.3cm}
\textbf{15.} \textit{they struck} \hspace{0.3cm}
\textbf{16.} \textit{and I led} \hspace{0.3cm}
\textbf{17.} \textit{and they led} (masc.\ pl.) \hspace{0.15cm} \foreignlanguage{hebrew}{‎וַיֹּלִ֫כוּ} \hspace{0.3cm}
\textbf{18.} \textit{you will be silent} (fem.\ sg.) \hspace{0.15cm} \foreignlanguage{hebrew}{‎תַּחֲרִישִׁי} \hspace{0.3cm}


\subsection*{Translation of Sentences}
\textbf{1.}~~Jer 17:14 \hspace{0.2cm}
\textbf{2.}~~Exod 2:13--14 \hspace{0.2cm}
\textbf{3.}~~Exod 3:10--12 \hspace{0.2cm}
\textbf{4.}~~Exod 6:13 \hspace{0.2cm}
\textbf{5.}~~Exod 20:1--2 \hspace{0.2cm}
\textbf{6.}~~Judg 6:8 \hspace{0.2cm}
\textbf{7.}~~Deut 4:1--2  \hspace{0.2cm}
\textbf{8.}~~Judg 13:1 \hspace{0.2cm}
\textbf{9.}~~1\,Sam 19:20--21  \hspace{0.2cm}
\textbf{10.}~~Deut 31:14--15 \hspace{0.2cm}
\textbf{11.}~~1\,Sam 24:17  \hspace{0.2cm}
\textbf{12.}~~Deut 8:1--2 Qere  \hspace{0.2cm}
\textbf{13.}~~Jer 32:3b--5 Qere \hspace{0.2cm}
\textbf{14.}~~1\,Kgs 18:22 \hspace{0.2cm}
\textbf{15.}~~1\,Kgs 19:9--10 \hspace{0.2cm}
\textbf{16.}~~2\,Sam 13:30 \hspace{0.2cm}
\textbf{17.}~~Ps 136:1 \hspace{0.2cm}
\textbf{18.}~~Ps 9:2 \hspace{0.2cm}

\section*{Chapter 24}

\subsection*{Translation of Verbal Forms}
\textbf{1.} \textit{he brought} \hspace{0.3cm}
\textbf{2.} \textit{she will bring/you will bring} (2 f.\ sg./ 2 m. sg.) \hspace{0.3cm}
\textbf{3.} \textit{you brought} (m.\ sg.) \foreignlanguage{hebrew}{הֵבֵ֫אתָ} \hspace{0.3cm}
\textbf{4.} \textit{they will look} (m.) \foreignlanguage{hebrew}{יָבִ֫יאוּ} \hspace{0.3cm}
\textbf{5.} \textit{she will be established/ you will be established} \hspace{0.3cm}
\textbf{6.} \textit{they will be established}  \hspace{0.15cm} \foreignlanguage{hebrew}{יִכֹּ֫נוּ} \hspace{0.3cm}
\textbf{7.} \textit{they were established} (c.\ pl.) \hspace{0.15cm} \foreignlanguage{hebrew}{נָכ֫וֹנוּ} \hspace{0.3cm}
\textbf{8.} \textit{they killed} \hspace{0.15cm} \foreignlanguage{hebrew}{הֵמִ֫יתוּ} \hspace{0.3cm}
\textbf{9.} \textit{you killed} (m.\ pl.) \hspace{0.3cm}
\textbf{10.} \textit{and they killed} (m.\ pl.) \hspace{0.15cm} \foreignlanguage{hebrew}{וַיָּמִ֫יתוּ} \hspace{0.3cm}
\textbf{11.} \textit{you removed} (m.\ sg.) \hspace{0.15cm} \foreignlanguage{hebrew}{הֲסִרֹ֫תִי} \hspace{0.3cm}
\textbf{12.} \textit{remove!} \foreignlanguage{hebrew}{הָסִ֫ירוּ} \hspace{0.3cm}
\textbf{13.} \textit{he will remove} \hspace{0.3cm}
\textbf{14.} \textit{may he remove} \hspace{0.3cm}
\textbf{15.} \textit{you brought back} (2 m.\ sg.) \hspace{0.15cm} \foreignlanguage{hebrew}{הֱשִׁיב֫וֹתָ} \hspace{0.3cm}
\textbf{16.} \textit{we will bring back} \hspace{0.3cm}
\textbf{17.} \textit{they set} \foreignlanguage{hebrew}{‎הִנִּ֫יחוּ} \hspace{0.3cm}
\textbf{18.} \textit{they will set} (2 m.\ pl.) \hspace{0.15cm} \foreignlanguage{hebrew}{יַנִּ֫יחוּ} \hspace{0.3cm}



\subsection*{Translation of Sentences}
\textbf{1.}~~2\,Sam 6:17 \hspace{0.2cm}
\textbf{2.}~~Gen 4:3 \hspace{0.2cm}
\textbf{3.}~~Gen 20:9 \hspace{0.2cm}
\textbf{4.}~~Gen 42:37 \hspace{0.2cm}
\textbf{5.}~~2\,Kgs 22:16 \hspace{0.2cm}
\textbf{6.}~~1\,Kgs 11:40 \hspace{0.2cm}
\textbf{7.}~~2\,Kgs 23:29--30  \hspace{0.2cm}
\textbf{8.}~~2\,Sam 13:28 \hspace{0.2cm}
\textbf{9.}~~Exod 21:12  \hspace{0.2cm}
\textbf{10.}~~Deut 24:16 \hspace{0.2cm}
\textbf{11.}~~1\,Sam 2:34--35  \hspace{0.2cm}
\textbf{12.}~~Gen 28:15  \hspace{0.2cm}
\textbf{13.}~~Gen 37:22 \hspace{0.2cm}
\textbf{14.}~~1\,Sam 5:3 \hspace{0.2cm}
\textbf{15.}~~Gen 2:15 \hspace{0.2cm}
\textbf{16.}~~1\,Sam 10:25 \hspace{0.2cm}
\textbf{17.}~~Josh 1:13 \hspace{0.2cm}
\textbf{18.}~~1\,Kgs 2:10--12 \hspace{0.2cm}
\textbf{19.}~~1\,Sam 7:3--4\hspace{0.2cm}
\textbf{20.}~~2\,Sam 7:12--16 \hspace{0.2cm}

\section*{Chapter 25}

\subsection*{Translation of Verbal Forms}
\textbf{1.} \textit{they began} \hspace{0.15cm} \foreignlanguage{hebrew}{הֵחֵ֫לּוּ} \hspace{0.3cm}
\textbf{2.} \textit{I began} \hspace{0.15cm} \foreignlanguage{hebrew}{הַחִלֹּ֫תִי} \hspace{0.3cm}
\textbf{3.} \textit{and they began} \hspace{0.15cm} \foreignlanguage{hebrew}{וַיָּחֵ֫לּוּ} \hspace{0.3cm}
\textbf{4.} \textit{and he began} \hspace{0.15cm} \foreignlanguage{hebrew}{וַיָּ֫חֶל} \hspace{0.3cm}
\textbf{5.} \textit{they will be dismayed} (m.\ pl.) \hspace{0.15cm} \foreignlanguage{hebrew}{יֵחַ֫תּוּ} \hspace{0.3cm}
\textbf{6.} \textit{you/they will be dismayed} (2 m.\ pl./3 f.\ pl.)  \hspace{0.15cm} \foreignlanguage{hebrew}{תֵּחַ֫תּוּ} \hspace{0.3cm}
\textbf{7.} \textit{they violated} \hspace{0.3cm}
\textbf{8.} \textit{you violated} (m.\ sg.) \hspace{0.15cm} \hspace{0.15cm} \foreignlanguage{hebrew}{הֵפַ֫רְתָּה} \hspace{0.3cm}
\textbf{9.} \textit{they violated} (m.\ pl.) \hspace{0.15cm} \foreignlanguage{hebrew}{הֵפֵ֫רוּ} \hspace{0.3cm}
\textbf{10.} \textit{you/they will violate} (2 m.\ pl./3 f.\ pl.) \hspace{0.15cm} \foreignlanguage{hebrew}{תָּפֵ֫רוּ} \hspace{0.3cm}
\textbf{11.} \textit{you did evil} (m.\ pl.) \hspace{0.3cm}
\textbf{12.} \textit{you did evil}  (m.\ sg.) \hspace{0.15cm} \foreignlanguage{hebrew}{הֲרֵעֹ֫תָה} \hspace{0.3cm}
\textbf{13.} \textit{and they did evil}  (m.\ pl.) \hspace{0.15cm} \foreignlanguage{hebrew}{וַיָּרֵ֫עוּ} \hspace{0.3cm}
\textbf{14.} \textit{evildoers} (m.) \hspace{0.3cm}
\textbf{15.} \textit{he will shudder} \hspace{0.3cm}
\textbf{16.} \textit{they shuddered} \hspace{0.3cm}
\textbf{17.} \textit{I created} \hspace{0.15cm} \foreignlanguage{hebrew}{‎בָּרָ֫אתִי} \hspace{0.3cm}
\textbf{18.} \textit{you/they will desecrate} (2 m.\ pl./3 f.\ pl.) \hspace{0.3cm}
\textbf{19.} \textit{they will consult together} (m.) \hspace{0.3cm}
\textbf{20.} \textit{they consulted together} \hspace{0.3cm}


\subsection*{Translation of Sentences}
\textbf{1.}~~Josh 3:7 \hspace{0.2cm}
\textbf{2.}~~Judg 10:18 \hspace{0.2cm}
\textbf{3.}~~Deut 3:24 \hspace{0.2cm}
\textbf{4.}~~Gen 41:54 \hspace{0.2cm}
\textbf{5.}~~Deut 1:21 \hspace{0.2cm}
\textbf{6.}~~Josh 8:1 \hspace{0.2cm}
\textbf{7.}~~2\,Sam 17.7  \hspace{0.2cm}
\textbf{8.}~~2\,Sam 17.14  \hspace{0.2cm}
\textbf{9.}~~Deut 31:14-1-8 \hspace{0.2cm}
\textbf{10.}~~1\,Kgs 16:25  \hspace{0.2cm}
\textbf{11.}~~2\,Sam 15:25  \hspace{0.2cm}
\textbf{12.}~~Gen 24:4--5  \hspace{0.2cm}


\section*{Chapter 26}

\subsection*{Translation of Sentences}
\textbf{1.}~~Gen 3:19 \hspace{0.2cm}
\textbf{2.}~~Gen 3:23 \hspace{0.2cm}
\textbf{3.}~~Gen 4:25--26 \hspace{0.2cm}
\textbf{4.}~~2\,Sam 9:6 \hspace{0.2cm}
\textbf{5.}~~1\,Sam 25:23 \hspace{0.2cm}
\textbf{6.}~~1\,Kgs 2:19 \hspace{0.2cm}
\textbf{7.}~~Gen 37:9--10  \hspace{0.2cm}
\textbf{8.}~~Josh 5:13--15  \hspace{0.2cm}
\textbf{9.}~~Josh 5:13--15 \hspace{0.2cm}
\textbf{10.}~~Jer 7:1--3  \hspace{0.2cm}
\textbf{11.}~~2\,Kgs 17:35--36  \hspace{0.2cm}
\textbf{12.}~~1\,Kgs 11:40 \hspace{0.2cm}


\chapter{Vocabulary in Alphabetical Order}

\renewcommand\arraystretch{1.3}

% The following code for using a longtable in a multicol environent is based on the answer in this post: https://tex.stackexchange.com/questions/45980/balancing-long-table-inside-multicol-in-latex
% It is a workaround for the problem that longtable and multicol are not compatible because longtable can only be used in one-column environment.
% This workaround requires manual pagebreaks.

%\begin{multicols}{2}
%
%\setbox\ltmcbox\vbox{
	%	\makeatletter\col@number\@ne
	%	\begin{longtable}{>{\raggedleft}p{0.25\linewidth} p{0.65\linewidth}}
		%		Text & Text \\
		%		Text & Text \\
		%		Text & Text \\
		%		Text & Text \\
		%		Text & Text \\
		%		Text & Text \\
		%	\end{longtable}
	%	\unskip
	%	\unpenalty
	%	\unpenalty}
%\unvbox\ltmcbox
%	
%\end{multicols}


\begin{center}
	\LARGE\foreignlanguage{hebrew}{א}
\end{center}

\begin{multicols}{2}
	
	\setbox\ltmcbox\vbox{
		\makeatletter\col@number\@ne
		\begin{longtable}{>{\raggedleft}p{0.25\linewidth} p{0.65\linewidth}}
			\foreignlanguage{hebrew}{אָב} & \textit{father} \\
			\foreignlanguage{hebrew}{אבד} & Q.\ \textit{to perish} \\
			\foreignlanguage{hebrew}{אבה} & Q.\ \textit{to be willing, consent} (almost always with a negative) \\
			\foreignlanguage{hebrew}{אֶבְיוֺן} & \textit{needy, poor} (adj.) \\ % HALOT
			\foreignlanguage{hebrew}{אֶבֶן} & \textit{stone} (f., pl.\ \foreignlanguage{hebrew}{אֲבָנִים})\\
			\foreignlanguage{hebrew}{אָדוֺן} & \textit{lord} (with ref. to God \foreignlanguage{hebrew}{אֲדֹנָי} \textit{my Lord}) \\
			\foreignlanguage{hebrew}{אָדָם} & \textit{man, mankind} \\
			\foreignlanguage{hebrew}{אֲדָמָה} & \textit{ground, land} \\
			\foreignlanguage{hebrew}{אהב} & Q.\ \textit{to love} (stative verb, pausal form SC \foreignlanguage{hebrew}{אָהֵ֑ב}) \\
			\foreignlanguage{hebrew}{אֹהֶל} & \textit{tent} \\
			\foreignlanguage{hebrew}{אוֺ} & \textit{or} \\
			\foreignlanguage{hebrew}{אָוֶן} & \textit{disaster; sin, injustice; deception, nothingness; idolatry} \\ % HALOT
			\foreignlanguage{hebrew}{אוֺצָר} & \textit{treasure, supplies} \\ % HALOT
			\foreignlanguage{hebrew}{אוֺר} & \textit{light} \\
			\foreignlanguage{hebrew}{אוֺת} & \textit{sign} (fem., pl.\ \foreignlanguage{hebrew}{אֹתוֺת}) \\
			\foreignlanguage{hebrew}{אָז} & \textit{then, at that time} (often with a PC form with past tense meaning) \\
			\foreignlanguage{hebrew}{אֹזֶן} & \textit{ear} (du. \foreignlanguage{hebrew}{אָזְנַיִם} \textit{ʾɔznáyim}) \\
			\foreignlanguage{hebrew}{אָח} & \textit{brother} \\
			\foreignlanguage{hebrew}{אָחוֺת} & \textit{sister} \\
			\foreignlanguage{hebrew}{אחז} & Q.\ \textit{to grasp, take hold, take possession} \\
			\foreignlanguage{hebrew}{אֲחֻזָּה} & \textit{possession, property} \\
			\foreignlanguage{hebrew}{אַחַר} & \textit{behind, afterwards} (adv.), \textit{behind, after} (prep.) \\
			\foreignlanguage{hebrew}{אַחֵר} & \textit{another, different}
			(m.\ pl.\ \foreignlanguage{hebrew}{אֲחֵרִים}, f.\ sg.\ \foreignlanguage{hebrew}{אַחֶרֶת}, f.\ pl.\ \foreignlanguage{hebrew}{אַחֵרוֺת}) \\
			\foreignlanguage{hebrew}{אַחֲרוֹן} & \textit{last, latter} (temp.); \textit{at the back; western} (local) \\
			\foreignlanguage{hebrew}{אַחֲרֵי} & \textit{after, behind} \\
			\foreignlanguage{hebrew}{אַחֲרֵי־כֵן} & \textit{afterwards} (adv.) \\
			\foreignlanguage{hebrew}{אַחֲרִית} & \textit{end; future; descendants} (fem.) \\ % HALOT
			\foreignlanguage{hebrew}{אֵי} & \textit{where?} \\
			\foreignlanguage{hebrew}{אֹיֵב} & \textit{enemy} \\
			\foreignlanguage{hebrew}{אַיֵּה} & \textit{where?} \\
			\foreignlanguage{hebrew}{אַיִל} & \textit{ram} \\
			\foreignlanguage{hebrew}{אַיִן} & \textit{there is not} (particle of non-existence) \\
			\foreignlanguage{hebrew}{אִישׁ} & \textit{man} \\
			\foreignlanguage{hebrew}{אַךְ} & \textit{yea, surely, only} \\ % HALOT
			\foreignlanguage{hebrew}{אכל} & Q.\ \textit{to eat} \\
			\foreignlanguage{hebrew}{אַל} & \textit{not} (with the jussive or cohortative) \\
			\foreignlanguage{hebrew}{אֶל} & \textit{to, towards} (spatial), \textit{at, by} (local), \textit{against, according to} \\
			\foreignlanguage{hebrew}{אֵל} & \textit{God, god} \\
			\foreignlanguage{hebrew}{אֵ֫לֶּה} & \textit{these} (comm.) \\
			\foreignlanguage{hebrew}{אֱלֹהִים} & \textit{God, god, gods} \\
			\foreignlanguage{hebrew}{אֱלוֺהַּ} & \textit{god, God} \\
			\foreignlanguage{hebrew}{אַלְמָנָה} & \textit{widow} \\
			\foreignlanguage{hebrew}{אִם} & \textit{if} (conj.) \\
			\foreignlanguage{hebrew}{אָמָה} & \textit{slave, handmaid} \\
			\foreignlanguage{hebrew}{אַמָּה} & \textit{cubit} \\
		\end{longtable}
		\unskip
		\unpenalty
		\unpenalty}
	\unvbox\ltmcbox
\end{multicols}

\newpage

\begin{multicols}{2}
	
	\setbox\ltmcbox\vbox{
		\makeatletter\col@number\@ne
		\begin{longtable}{>{\raggedleft}p{0.25\linewidth} p{0.65\linewidth}}
			\foreignlanguage{hebrew}{אֱמוּנָה} & \textit{steadfastness; trustworthiness, faithfulness} \\ % HALOT
			\foreignlanguage{hebrew}{אמן} & Hi. \textit{to believe, think; to believe in; to have trust in}, Ni. \textit{to be reliable, faithful; to be permanent, endure} \\ % HALOT
			\foreignlanguage{hebrew}{אמר} & Q.\ \textit{to say} \\
			\foreignlanguage{hebrew}{אֱמֹרִי} & \textit{Amorite} \\
			\foreignlanguage{hebrew}{אָ֫נָה} & \textit{where to?} (interrogative adverb \foreignlanguage{hebrew}{אָן} \textit{where to?} with directional He) \\
			\foreignlanguage{hebrew}{אֱנוֺשׁ} & \textit{mankind, (all) human beings, man; (some) men; single human being} (mostly poet.) \\
			\foreignlanguage{hebrew}{אסף} & Q.\ \textit{to gather, remove} \\
			\foreignlanguage{hebrew}{אסר} & Q.\ \textit{to tie, bind, imprison} \\
			\foreignlanguage{hebrew}{אַף} & \textit{also, even} (adv.) \\
			\foreignlanguage{hebrew}{אַף} & \textit{nose, anger, nostrils} (gem. noun; dual \foreignlanguage{hebrew}{אַפַּיִם}) \\
			\foreignlanguage{hebrew}{אַפַּיִם} & \textit{face} (dual of \foreignlanguage{hebrew}{אַף})  \\
			\foreignlanguage{hebrew}{אֵצֶל} & \textit{in proximity to, beside} (prep.) \\
			\foreignlanguage{hebrew}{אֲרוֺן} & \textit{chest, ark} (with the article \foreignlanguage{hebrew}{הָאָרוֺן}) \\
			\foreignlanguage{hebrew}{אֶרֶז} & \textit{cedar} \\
			\foreignlanguage{hebrew}{אֹרַח} & \textit{way, path} \\
			\foreignlanguage{hebrew}{אַרְיֵה}/\foreignlanguage{hebrew}{אֲרִי} & \textit{lion} \\
			\foreignlanguage{hebrew}{אֹרֶךְ} & \textit{length} \\
			\foreignlanguage{hebrew}{אֶרֶץ} & \textit{earth, land} (fem.) (pl. \foreignlanguage{hebrew}{אֲרָצוֺת}; with the article \foreignlanguage{hebrew}{הָאָ֫רֶץ}, pausal form \foreignlanguage{hebrew}{אָ֑רֶץ}) \\
			\foreignlanguage{hebrew}{ארר} & Q.\ \textit{to curse} \\
			\foreignlanguage{hebrew}{אֵשׁ} & \textit{fire} (fem.) \\
			\foreignlanguage{hebrew}{אִשָּׁה} & \textit{woman} \\
			\foreignlanguage{hebrew}{אֵת} & \textit{with, close to, beside} \\
			\foreignlanguage{hebrew}{אֵת} & object marker indicating the direct object \\
		\end{longtable}
		\unskip
		\unpenalty
		\unpenalty}
	
	\unvbox\ltmcbox
	
\end{multicols}

\medskip

\begin{center}
	\LARGE\foreignlanguage{hebrew}{ב}
\end{center}

\begin{multicols}{2}
	
	\setbox\ltmcbox\vbox{
		\makeatletter\col@number\@ne
		\begin{longtable}{>{\raggedleft}p{0.25\linewidth} p{0.65\linewidth}}
			\foreignlanguage{hebrew}{בְּ} & \textit{in, at, with} (local and temporal)\\
			\foreignlanguage{hebrew}{בְּאֵר} & \textit{well} (of underground water), \textit{watering place} (fem.) \\	% HALOT
			\foreignlanguage{hebrew}{בֶּגֶד} & \textit{garment, clothing} \\
			\foreignlanguage{hebrew}{בְּהֵמָה} & \textit{beast, animal, cattle} \\
			\foreignlanguage{hebrew}{בוא} & Q.\ \textit{to come, to enter}; Hi.\ \textit{to bring, bring in} \\
			\foreignlanguage{hebrew}{בוֺר} & \textit{cistern, pit} \\
			\foreignlanguage{hebrew}{בושׁ} & Q.\ \textit{to be ashamed} \\
			\foreignlanguage{hebrew}{בחר} & Q.\ \textit{to choose} (with prepositional object with \foreignlanguage{hebrew}{בְּ} or direct object) \\
			\foreignlanguage{hebrew}{בטח} & Q.\ \textit{to trust, be confident} \\
			\foreignlanguage{hebrew}{בֶּטֶן} & \textit{belly; internal organs} (fem.) \\ % HALOT
			\foreignlanguage{hebrew}{בְּטֶרֶם} & \textit{before} (conj. or prep.) \\
			\foreignlanguage{hebrew}{בין} & Q.\ \textit{to discern, understand}; Hi.\ \textit{to understand; to give understanding, teach} \\
			\foreignlanguage{hebrew}{בֵּין} & \textit{between} (prep., usually \foreignlanguage{hebrew}{בֵּין ... וּבֵין} or \foreignlanguage{hebrew}{ בֵּין ... לְ}) \\
			\foreignlanguage{hebrew}{בַּיִת} & \textit{house} \\
			\foreignlanguage{hebrew}{בכה} & Q.\ \textit{to weep, weep for} \\ % HALOT
			\foreignlanguage{hebrew}{בְּכֹר} & \textit{firstborn} \\
			\foreignlanguage{hebrew}{בַּל} & \textit{not} (negative used in poetry) \\
			\foreignlanguage{hebrew}{בְּלִי} & \textit{not; without} \\
			\foreignlanguage{hebrew}{בָּמָה} & \textit{hill, high place (often as place of worship)} \\
			\foreignlanguage{hebrew}{בֵּן} & \textit{son} \\
			\foreignlanguage{hebrew}{בנה} & Q.\ \textit{to build} \\
			\foreignlanguage{hebrew}{בַּעֲבוּר} & \textit{because of, for the sake of} (prep.): \textit{in order that} (conj.) (prep. \foreignlanguage{hebrew}{בְּ} + noun \foreignlanguage{hebrew}{עֲבוּר}) \\
		\end{longtable}
		\unskip
		\unpenalty
		\unpenalty}
	\unvbox\ltmcbox
	
\end{multicols}

\newpage

\begin{multicols}{2}
	
	\setbox\ltmcbox\vbox{
		\makeatletter\col@number\@ne
		\begin{longtable}{>{\raggedleft}p{0.25\linewidth} p{0.65\linewidth}}
			\foreignlanguage{hebrew}{בַּעַד} & \textit{behind; through, out of; round about; for the benefit of, for} (prep.; cs.\ st.\ \foreignlanguage{hebrew}{בְּעַד}, with ePP \foreignlanguage{hebrew}{בַּעֲדוֺ}) \\ % HALOT
			\foreignlanguage{hebrew}{בַּעַל} & \textit{owner, lord, Baal} \\
			\foreignlanguage{hebrew}{בער} & Q.\ \textit{to burn} (intransitive) \\
			\foreignlanguage{hebrew}{בָּקָר} & \textit{cattle, herd} (coll.) \\
			\foreignlanguage{hebrew}{בֹּקֶר} & \textit{morning} \\
			\foreignlanguage{hebrew}{בקשׁ} & Pi.\ \textit{to seek}  \\ % BDB
			\foreignlanguage{hebrew}{ברא} & Q.\ \textit{to create} \\
			\foreignlanguage{hebrew}{בַּרְזֶל} & \textit{iron} \\
			\foreignlanguage{hebrew}{ברח} & Q.\ \textit{to run away, flee} \\
			\foreignlanguage{hebrew}{בְּרִית} & \textit{covenant, agreement, contract} (fem.) \\
			\foreignlanguage{hebrew}{ברך} & Pi.\ \textit{to bless} (in the Q.\ only part.\ pass.) \\ % HALOT
			\foreignlanguage{hebrew}{בְּרָכָה} & \textit{blessing} (cs.\ st.\ \foreignlanguage{hebrew}{בִּרְכַּת}, but with enclitic pronoun \foreignlanguage{hebrew}{בִּרְכָתִי}) \\
			\foreignlanguage{hebrew}{בָּשָׂר} & \textit{flesh, meat} \\
			\foreignlanguage{hebrew}{בַּת} & \textit{daughter} \\
			\foreignlanguage{hebrew}{בְּתוּלָה} & \textit{virgin} \\
		\end{longtable}
		\unskip
		\unpenalty
		\unpenalty}
	\unvbox\ltmcbox
	
\end{multicols}

\medskip

\begin{center}
	\LARGE\foreignlanguage{hebrew}{ג}
\end{center}

\begin{multicols}{2}
	
	\setbox\ltmcbox\vbox{
		\makeatletter\col@number\@ne
		\begin{longtable}{>{\raggedleft}p{0.25\linewidth} p{0.65\linewidth}}
			\foreignlanguage{hebrew}{גאל} & Q.\ \textit{to redeem} \\
			\foreignlanguage{hebrew}{גְּבוּל} & \textit{border, boundary, territory} \\
			\foreignlanguage{hebrew}{גִּבּוֺר} & \textit{hero} (as substantive), \textit{strong, mighty} (adj.) \\
			\foreignlanguage{hebrew}{גְּבוּרָה} & \textit{strength, might} \\
			\foreignlanguage{hebrew}{גִּבְעָה} & \textit{hill} \\
			\foreignlanguage{hebrew}{גֶּבֶר} & \textit{man} \\ % BDB
			\foreignlanguage{hebrew}{גָדוֺל} & \textit{great} \\
			\foreignlanguage{hebrew}{גדל} & Q.\ \textit{to be great, become great, grow up} (stative verb); Hi.\ \textit{to make great; to make great things}; Pi.\ \textit{to cause to grow; make great, powerful} \\
			\foreignlanguage{hebrew}{גּוֺי} & \textit{nation, people} (pl.\ \foreignlanguage{hebrew}{גּוֺיִם})\\
			\foreignlanguage{hebrew}{גור} & Q.\ \textit{to sojourn, dwell as alien and dependent} \\ % BDB + HALOT
			\foreignlanguage{hebrew}{גּוֺרָל} & \textit{lot; allocation by lot} \\ % HALOT
			\foreignlanguage{hebrew}{גלה} & Q.\ \textit{to uncover, remove, go into exile} \\
			\foreignlanguage{hebrew}{גַּם} & \textit{also, even} \\
			\foreignlanguage{hebrew}{גָּמָל} & \textit{camel} (pl. \foreignlanguage{hebrew}{גְּמַלִּים}) \\
			\foreignlanguage{hebrew}{גֶּפֶן} & \textit{vine} \\
			\foreignlanguage{hebrew}{גֵּר} & \textit{sojourner} \\
			\foreignlanguage{hebrew}{גרשׁ} & Pi.\ \textit{to drive out} \\
		\end{longtable}
		\unskip
		\unpenalty
		\unpenalty}
	\unvbox\ltmcbox
	
\end{multicols}

\medskip

\begin{center}
	\LARGE\foreignlanguage{hebrew}{ד}
\end{center}

\begin{multicols}{2}
	
	\setbox\ltmcbox\vbox{
		\makeatletter\col@number\@ne
		\begin{longtable}{>{\raggedleft}p{0.25\linewidth} p{0.65\linewidth}}
			\foreignlanguage{hebrew}{דבק} & Q.\ \textit{to cling, cleave, keep close} \\
			\foreignlanguage{hebrew}{דבר} & Pi.\ \textit{to speak} \\
			\foreignlanguage{hebrew}{דָּבָר} & \textit{word, matter, thing} \\
			\foreignlanguage{hebrew}{דְּבַשׁ} & \textit{honey} \\
			\foreignlanguage{hebrew}{דוֺר} & \textit{generation} (pl. \foreignlanguage{hebrew}{דֹּרוֺת}) \\
			\foreignlanguage{hebrew}{דֶּלֶת} & \textit{door} \\
			\foreignlanguage{hebrew}{דָּם} & \textit{blood} \\
			\foreignlanguage{hebrew}{דַּעַת} & \textit{knowledge} (inf.\ cs.\ Q. \foreignlanguage{hebrew}{ידע}) \\
			\foreignlanguage{hebrew}{דרך} & Q.\ \textit{to tread} \\
			\foreignlanguage{hebrew}{דֶּרֶךְ} & \textit{way, road} \\
			\foreignlanguage{hebrew}{דרשׁ} & Q.\ \textit{to seek, resort to} \\
		\end{longtable}
		\unskip
		\unpenalty
		\unpenalty}
	\unvbox\ltmcbox
	
\end{multicols}

\newpage

\begin{center}
	\LARGE\foreignlanguage{hebrew}{ה}
\end{center}

\begin{multicols}{2}
	
	\setbox\ltmcbox\vbox{
		\makeatletter\col@number\@ne
		\begin{longtable}{>{\raggedleft}p{0.25\linewidth} p{0.65\linewidth}}
			\foreignlanguage{hebrew}{הוֹי} & \textit{ah! alas! ha!} (interjection expressing usually dissatisfaction and pain) \\
			\foreignlanguage{hebrew}{היה} & Q.\ \textit{t be, to become} \\
			\foreignlanguage{hebrew}{הֵיכָל} & \textit{palace; temple} \\
			\foreignlanguage{hebrew}{הלך} & Q.\ \textit{to go}; Hitpa.\ \textit{to walk; to walk about; to move to and fro} \\ % BDB
			\foreignlanguage{hebrew}{הלל} & Pi.\ \textit{to praise}; Hitpa.\ \textit{to boast} \\
			\foreignlanguage{hebrew}{הָמוֺן} & \textit{sound, murmur, roar; crowd; abundance} \\
			\foreignlanguage{hebrew}{הֵן} & \textit{behold} (points to the word it precedes; cf. \foreignlanguage{hebrew}{הִנֵּה}); \textit{if} (conditional conj.); \textit{if} (for indirect questions) \\
			\foreignlanguage{hebrew}{הִנֵּה} & \textit{look, behold} \\
			\foreignlanguage{hebrew}{הפך} & Q.\ \textit{to turn; to overthrow; to change} \\ % HALOT
			\foreignlanguage{hebrew}{הָר} & \textit{mountain, hill, hill-country} \\
			\foreignlanguage{hebrew}{הַרְבֵּה} & \textit{greatly, exceedingly, much} (adv.; orig. inf.\ abs.\ Hi. of \foreignlanguage{hebrew}{רבה}) \\
			\foreignlanguage{hebrew}{הרג} & Q.\ \textit{to kill, slay} \\
		\end{longtable}
		\unskip
		\unpenalty
		\unpenalty}
	\unvbox\ltmcbox
	
\end{multicols}

\medskip

\begin{center}
	\LARGE\foreignlanguage{hebrew}{ז}
\end{center}

\begin{multicols}{2}
	
	\setbox\ltmcbox\vbox{
		\makeatletter\col@number\@ne
		\begin{longtable}{>{\raggedleft}p{0.25\linewidth} p{0.65\linewidth}}
			\foreignlanguage{hebrew}{זֹאת} ,\foreignlanguage{hebrew}{זֶה} & \textit{this} (masc., fem.) \\
			\foreignlanguage{hebrew}{זבח} & Q.\ \textit{to slaughter for sacrifice, slaughter} \\
			\foreignlanguage{hebrew}{זֶבַח} & \textit{sacrifice} \\
			\foreignlanguage{hebrew}{זָהָב} & \textit{gold} \\
			\foreignlanguage{hebrew}{זכר} & Q.\ \textit{to remember} \\
			\foreignlanguage{hebrew}{זָכָר} & \textit{male} (adj.) \\
			\foreignlanguage{hebrew}{זנה} & Q. \textit{to commit fornication; to be unfaithful} (in a relationship with God) \\ % HALOT
			\foreignlanguage{hebrew}{זעק} & Q. \textit{to cry, cry out, call} (cf. \foreignlanguage{hebrew}{צעק}) \\
			\foreignlanguage{hebrew}{זָקֵן} & \textit{old, old man} (\foreignlanguage{hebrew}{הַזְּקֵנִים} \textit{the elders}) \\
			\foreignlanguage{hebrew}{זָר} & \textit{strange, foreign} \\ % BDB
			\foreignlanguage{hebrew}{זְרוֺעַ} & \textit{arm, forearm} (fem.) \\
			\foreignlanguage{hebrew}{זרע} & Q.\ \textit{to sow, scatter seed} \\
			\foreignlanguage{hebrew}{זֶרַע} & \textit{seed, sowing, offspring, descendants} \\ % BDB
		\end{longtable}
		\unskip
		\unpenalty
		\unpenalty}
	\unvbox\ltmcbox
	
\end{multicols}

\medskip

\begin{center}
	\LARGE\foreignlanguage{hebrew}{ח}
\end{center}

\begin{multicols}{2}
	
	\setbox\ltmcbox\vbox{
		\makeatletter\col@number\@ne
		\begin{longtable}{>{\raggedleft}p{0.25\linewidth} p{0.65\linewidth}}
			\foreignlanguage{hebrew}{חָג} & \textit{festival, festival-gathering, pilgrim-feast} (pl. \foreignlanguage{hebrew}{חַגִּים}; abs.\ st.\ also \foreignlanguage{hebrew}{חַג}) \\
			\foreignlanguage{hebrew}{חדל} & Q.\ \textit{to cease, come to an end; to cease, leave off} \\ % BDB
			\foreignlanguage{hebrew}{חָדָשׁ} & \textit{new} \\
			\foreignlanguage{hebrew}{חֹדֶשׁ} & \textit{new moon, month} \\
			\foreignlanguage{hebrew}{חוה} & Hištaphel \textit{to bow down, prostrate; to worship} \\
			\foreignlanguage{hebrew}{חוֺמָה} & \textit{city wall, wall} \\ % HALOT
			\foreignlanguage{hebrew}{חוּץ} & \textit{outside, street} (m.; pl.\ \foreignlanguage{hebrew}{חוּצוֺת}) \\
			\foreignlanguage{hebrew}{חזה} & Q.\ \textit{to see, behold} (almost always poetic) \\ % BDB
		\end{longtable}
		\unskip
		\unpenalty
		\unpenalty}
	\unvbox\ltmcbox
	
\end{multicols}

\newpage

\begin{multicols}{2}
	
	\setbox\ltmcbox\vbox{
		\makeatletter\col@number\@ne
		\begin{longtable}{>{\raggedleft}p{0.25\linewidth} p{0.65\linewidth}}
			\foreignlanguage{hebrew}{חזק} & Q.\ \textit{to be firm, strong} (stative verb, PC \foreignlanguage{hebrew}{יֶחֱזַק} but SC \foreignlanguage{hebrew}{חָזַק}); Pi.\ \textit{to make firm, strong; repair}; Hi.\ \textit{to seize, grasp; to keep hold of}; Hitpa.\ \textit{to show oneself courageous; to prove oneself strong} \\
			\foreignlanguage{hebrew}{חָזָק} & \textit{strong; heavy, severe; firm, hard} (adj.) \\
			\foreignlanguage{hebrew}{חטא} & Q.\ \textit{to sin} \\
			\foreignlanguage{hebrew}{חַטָּאת} & \textit{sin, sin offering} \\
			\foreignlanguage{hebrew}{חַי} & \textit{alive, living} (pl. \foreignlanguage{hebrew}{חַיִּים}) \\
			\foreignlanguage{hebrew}{חיה} & Q.\ \textit{to be alive, stay alive, revive, recover, return to life} \\ % HALOT
			\foreignlanguage{hebrew}{חַיָּה} & \textit{animals} (usually collective) \\
			\foreignlanguage{hebrew}{חַיִל} & \textit{strength, wealth, army} \\
			\foreignlanguage{hebrew}{חָכָם} & \textit{wise} \\
			\foreignlanguage{hebrew}{חָכְמָה} & \textit{wisdom} \\
			\foreignlanguage{hebrew}{חֵלֶב} & \textit{fat; best, choice part} \\ % HALOT
			\foreignlanguage{hebrew}{חלה} & Q.\ \textit{to be} or \textit{become weak; become sick, ill} \\ % BDB
			\foreignlanguage{hebrew}{חֲלוֹם} & \textit{dream} (masc.; pl.\ \foreignlanguage{hebrew}{חֲלֹמוֺת})  \\
			\foreignlanguage{hebrew}{חלל} & Hi.\ \textit{to begin}; Pi.\ \textit{to profane; to put into use} \\ % HALOT
			\foreignlanguage{hebrew}{חָלָל} & \textit{slain, pierced} \\ % HALOT
			\foreignlanguage{hebrew}{חלק} & Q./Pi.\ \textit{to divide, apportion} \\
			\foreignlanguage{hebrew}{חֵלֶק} & \textit{portion, share, possession} \\
			\foreignlanguage{hebrew}{חֵמָה} & \textit{heat; rage, wrath; poison} \\ % BDB
			\foreignlanguage{hebrew}{חֲמוֺר} & \textit{donkey, he-ass} \\
			\foreignlanguage{hebrew}{חָמָס} & \textit{violence, wrong} \\
			\foreignlanguage{hebrew}{חֵן} & \textit{favor, grace} \\
			\foreignlanguage{hebrew}{חנה} & Q.\ \textit{to encamp} \\
			\foreignlanguage{hebrew}{חנן} & Q.\ \textit{to show favor, be gracious} \\
			\foreignlanguage{hebrew}{חֶסֶד} & \textit{joint obligation, loyalty, faithfulness, goodness, graciousness} \\ % HALOT
			\foreignlanguage{hebrew}{חפץ} & Q.\ \textit{to take pleasure in, desire; to delight in} (freq.\ with prep.\ obj.\ with \foreignlanguage{hebrew}{בְּ}) \\ % HALOT
			\foreignlanguage{hebrew}{חֵץ} & \textit{arrow} (gem.\ noun; pl.\ \foreignlanguage{hebrew}{חִצִּים}) \\
			\foreignlanguage{hebrew}{חֲצִי} & \textit{half, middle} \\
			\foreignlanguage{hebrew}{חָצֵר} & \textit{court, enclosure} \\ % BDB
			\foreignlanguage{hebrew}{חֹק} & \textit{statute} (pl.\ \foreignlanguage{hebrew}{חֻקִּים}) \\
			\foreignlanguage{hebrew}{חֻקָּה} & \textit{statute} \\
			\foreignlanguage{hebrew}{חֶרֶב} & \textit{sword} (fem.) \\
			\foreignlanguage{hebrew}{חרה} & Q.\ \textit{to burn, kindle} (of anger \foreignlanguage{hebrew}{אַף}) \\
			\foreignlanguage{hebrew}{חרם} & Hi. \textit{to put under a ban} \\ % HALOT
			\foreignlanguage{hebrew}{חֶרְפָּה} & \textit{disgrace, shame; reviling, taunt} \\ % HALOT
			\foreignlanguage{hebrew}{חרשׁ} & Hi.\ \textit{to be silent} \\
			\foreignlanguage{hebrew}{חשׁב} & Q.\ \textit{to think, consider, regard, plan, devise} \\
			\foreignlanguage{hebrew}{חֹשֶׁךְ} & \textit{darkness} \\
			\foreignlanguage{hebrew}{חתת} & Ni.\ \textit{to be dismayed; to be terrified}; Q.\ \textit{to be shattered; to be filled with terror}  \\
		\end{longtable}	
		\unskip
		\unpenalty
		\unpenalty}
	\unvbox\ltmcbox
	
\end{multicols}

\medskip

\begin{center}
	\LARGE\foreignlanguage{hebrew}{ט}
\end{center}

\begin{multicols}{2}
	
	\setbox\ltmcbox\vbox{
		\makeatletter\col@number\@ne
		\begin{longtable}{>{\raggedleft}p{0.25\linewidth} p{0.65\linewidth}}
			\foreignlanguage{hebrew}{טָהוֺר} & \textit{pure; ceremionally clean} (adj.) \\
			\foreignlanguage{hebrew}{טהר} & Q.\ \textit{to be clean} (stative verb, SC \foreignlanguage{hebrew}{טָהֵר}, PC \foreignlanguage{hebrew}{יִטְהַר}); Pi.\ \textit{to cleanse; to pronounce clean} \\
			\foreignlanguage{hebrew}{טוֺב} & \textit{good} \\
			\foreignlanguage{hebrew}{טמא} & Q.\ \textit{to be unclean, become unclean} (stative verb, SC \foreignlanguage{hebrew}{טָמֵא}); Pi.\ \textit{to defile; to desecrate} \\ % HALOT for Piel
			\foreignlanguage{hebrew}{טָמֵא} & \textit{unclean} (adj., m.\ pl.\ \foreignlanguage{hebrew}{טְמֵאִים}) \\
			\foreignlanguage{hebrew}{טֶרֶם} & \textit{not yet} (adv.); \textit{before} (conj.); \foreignlanguage{hebrew}{בְּטֶרֶם} \textit{before} (conj.\ or prep.) \\
		\end{longtable}
		\unskip
		\unpenalty
		\unpenalty}
	\unvbox\ltmcbox
	
\end{multicols}

\newpage

\begin{center}
	\LARGE\foreignlanguage{hebrew}{י}
\end{center}

\begin{multicols}{2}
	
	\setbox\ltmcbox\vbox{
		\makeatletter\col@number\@ne
		\begin{longtable}{>{\raggedleft}p{0.25\linewidth} p{0.65\linewidth}}
			\foreignlanguage{hebrew}{יְאֹר} & \textit{the Nile} (usually with the article \foreignlanguage{hebrew}{הַיאֹר}, pl.\ \foreignlanguage{hebrew}{יְאֹרִים}; \textit{Nile-arms, Nile-canals}) \textit{river, stream} \\
			\foreignlanguage{hebrew}{יבשׁ} & Q.\ \textit{to be dry, dried up} (stative verb; SC \foreignlanguage{hebrew}{יָבֵשׁ}; PC \foreignlanguage{hebrew}{יִיבַשׁ}); Hi.\ \textit{to dry up, make dry} \\
			\foreignlanguage{hebrew}{ידה}\textsubscript{2} & Hi.\ \textit{to praise}; Hitpa.\ \textit{to confess} (SC \foreignlanguage{hebrew}{הִתְוַדָּה}) \\
			\foreignlanguage{hebrew}{ידע} & \textit{to know} \\
			\foreignlanguage{hebrew}{יוֺם} & \textit{day} (\foreignlanguage{hebrew}{הַיּוֺם} as adv.\ \textit{today}) \\
			\foreignlanguage{hebrew}{יוֹמָם} & \textit{by day} (adv.) \\
			\foreignlanguage{hebrew}{יטב} & Q.\ \textit{to be good, well, glad, pleasant}; Hi. \textit{to do good to, deal well with; to do well; to make something good} \\ % BDB
			\foreignlanguage{hebrew}{יַיִן} & \textit{wine} \\
			\foreignlanguage{hebrew}{יכח} & Hi.\ \textit{to rebuke, reproach; to decide} \\ % HALOT
			\foreignlanguage{hebrew}{ילד} & Q.\ \textit{to bear, beget} \\
			\foreignlanguage{hebrew}{יֶלֶד} & \textit{boy} \\
			\foreignlanguage{hebrew}{יָם} & \textit{sea}, west (pl.\ \foreignlanguage{hebrew}{יַמִּים}) \\
			\foreignlanguage{hebrew}{יָמִין} & \textit{right side, right hand, south} \\
			\foreignlanguage{hebrew}{יסף} & Q.\ \textit{to add}; Hi.\ \textit{to add} \\
			\foreignlanguage{hebrew}{יַעַן} ,\foreignlanguage{hebrew}{יַעַן אֲשֶׁר}  & \textit{because} \\
			\foreignlanguage{hebrew}{יעץ} & Q.\ \textit{to advice, counsel}; Ni.\ \textit{to consult together; to exchange council} \\ % BDB
			\foreignlanguage{hebrew}{יַעַר} & \textit{wood; forest; thicket} \\ % BDB
			\foreignlanguage{hebrew}{יצא} & Q.\ \textit{to go out, come out} \\
			\foreignlanguage{hebrew}{יצב} & Hitpa.\ \textit{to set oneself, to station oneself; to take one's stand} (cf. \foreignlanguage{hebrew}{נצב} Ni./Hi.)\\ % Only 48 occurrences according to BW10
			\foreignlanguage{hebrew}{יצק} & Q.\ \textit{to pour out} \\
			\foreignlanguage{hebrew}{ירא} & Q.\ \textit{to fear, be afraid} (stative verb, SC \foreignlanguage{hebrew}{יָרֵא}; with direct object or prep. \foreignlanguage{hebrew}{מִן} or \foreignlanguage{hebrew}{מִפְּנֵי} indicate the obect) \\
			\foreignlanguage{hebrew}{ירד} & Q.\ \textit{to go down, come down, descend} \\
			\foreignlanguage{hebrew}{ירה}\textsubscript{3} & Hi.\ \textit{to instruct, teach} (cf. \foreignlanguage{hebrew}{תּוֺרָה}) \\
			\foreignlanguage{hebrew}{ירשׁ} & \textit{to take possession of, inherit, dispossess} \\
			\foreignlanguage{hebrew}{יֵשׁ} & \textit{there is} (particle of existence) \\
			\foreignlanguage{hebrew}{ישׁב} & Q.\ \textit{to sit, remain, dwell} \\
			\foreignlanguage{hebrew}{יְשׁוּעָה} & \textit{help, salvation} \\ % HALOT
			\foreignlanguage{hebrew}{ישׁע} & Hi.\ \textit{to help, save}; Ni.\ \textit{to receive help; to be victorious} \\ % HALOT
			\foreignlanguage{hebrew}{יָשָׁר} & \textit{straight, right} \\
			\foreignlanguage{hebrew}{יתר} & Ni.\ \textit{to be left over, remain}; Hi.\ \textit{to leave over, leave} \\ % BDB
			\foreignlanguage{hebrew}{יֶתֶר} & \textit{rest, remainder} \\
		\end{longtable}
		\unskip
		\unpenalty
		\unpenalty}
	\unvbox\ltmcbox
	
\end{multicols}

\begin{center}
	\LARGE\foreignlanguage{hebrew}{כ}
\end{center}

\begin{multicols}{2}
	
	\setbox\ltmcbox\vbox{
		\makeatletter\col@number\@ne
		\begin{longtable}{>{\raggedleft}p{0.25\linewidth} p{0.65\linewidth}}
			\foreignlanguage{hebrew}{כְּ} & \textit{like, as, according to} \\
			\foreignlanguage{hebrew}{כַּאֲשֶׁר} & \textit{as, according to} (modal), \textit{when} (temporal), \textit{because} (causal) (conj.) \\
			\foreignlanguage{hebrew}{כבד} & Q.\ \textit{to be heavy; to be insensitive, dull; to he honored} (stative verb, SC \foreignlanguage{hebrew}{כָּבֵד}, PC \foreignlanguage{hebrew}{יִכְבַּד}); Pi.\ \textit{to make honorable, honor}; Ni.\ \textit{to be honored, enjoy honor} \\
			\foreignlanguage{hebrew}{כָּבֵד} & \textit{heavy; oppressing; weighty} (adj.) \\
			\foreignlanguage{hebrew}{כָּבוֺד} & \textit{glory, honor} \\
			\foreignlanguage{hebrew}{כבס} & Pi.\ \textit{to wash} (normally clothes) (SC \foreignlanguage{hebrew}{כִּבֶּס})  \\
			\foreignlanguage{hebrew}{כֶּבֶשׂ} & \textit{young ram} \\ % HALOT
			\foreignlanguage{hebrew}{כֹּה} & \textit{thus, here} (adv.) \\
		\end{longtable}
		\unskip
		\unpenalty
		\unpenalty}
	\unvbox\ltmcbox
	
\end{multicols}

\newpage

\begin{multicols}{2}
	
	\setbox\ltmcbox\vbox{
		\makeatletter\col@number\@ne
		\begin{longtable}{>{\raggedleft}p{0.25\linewidth} p{0.65\linewidth}}
			
			\foreignlanguage{hebrew}{כֹּהֵן} & \textit{priest} \\
			\foreignlanguage{hebrew}{כון} & Ni.\ \textit{to be established; to be steadfast; to be permanent; to be ready}; Hi.\ \textit{to establish, set up; to make ready, prepare; to direct}; Polel \textit{to set up, establish} \\
			\foreignlanguage{hebrew}{כֹּחַ} & \textit{strength, power} \\
			\foreignlanguage{hebrew}{כִּי} & \textit{for, because} (causal), \textit{if} (conditional), \textit{when} (temporal), \textit{that}, etc. (conj.) \\
			\foreignlanguage{hebrew}{כִּכָּר} & \textit{round loaf, talent, vicinity} (esp. the region of the Jordan) (fem.) \\
			\foreignlanguage{hebrew}{כֹּל} & \textit{the whole, every, all} \\
			\foreignlanguage{hebrew}{כלה} & Q.\ \textit{to be complete, at an end, finished, accomplished, spent}; Pi.\ \textit{to complete, bring to an end, finish; to accomplish; to cause to cease} \\ % BDB
			\foreignlanguage{hebrew}{כְּלִי} & \textit{article, utensil, vessel, weapon} \\
			\foreignlanguage{hebrew}{כֵּן} & \textit{so, thus} (Modern Hebrew: \textit{yes}) \\
			\foreignlanguage{hebrew}{כָּנָף} & \textit{wing; edge, extremity; skirt} (of a garment) (fem.) \\
			\foreignlanguage{hebrew}{כִּסֵּא} & \textit{seat, chair, throne} \\
			\foreignlanguage{hebrew}{כסה} & Pi.\ \textit{to cover} \\
			\foreignlanguage{hebrew}{כְּסִיל} & \textit{fool} \\
			\foreignlanguage{hebrew}{כֶּסֶף} & \textit{silver} \\
			\foreignlanguage{hebrew}{כעס} & Hi.\ \textit{to provoke to anger} \\
			\foreignlanguage{hebrew}{כַּף} & \textit{the hollow, flat of the hand, hand, sole (of the foot)} (dual \foreignlanguage{hebrew}{כַּפַּיִם}) \\
			\foreignlanguage{hebrew}{כְּרוּב} & \textit{cherub} \\
			\foreignlanguage{hebrew}{כֶּרֶם} & \textit{vineyard} \\
			\foreignlanguage{hebrew}{כרת} & Q.\ \textit{to cut}, Hi.\ \textit{cut off, exterminate} \\ % BDB and HALOT
			\foreignlanguage{hebrew}{כשׁל} & Q.\ \textit{to stumble, stagger}; Hi.\ \textit{to cause to stumble, stagger} \\
			\foreignlanguage{hebrew}{כתב} & Q.\ \textit{to write} \\
			\foreignlanguage{hebrew}{כָּתֵף} & \textit{shoulder; side; mountain slope} \\ % HALOT
		\end{longtable}
		\unskip
		\unpenalty
		\unpenalty}
	\unvbox\ltmcbox
	
\end{multicols}

\medskip

\begin{center}
	\LARGE\foreignlanguage{hebrew}{ל}
\end{center}

\begin{multicols}{2}
	
	\setbox\ltmcbox\vbox{
		\makeatletter\col@number\@ne
		\begin{longtable}{>{\raggedleft}p{0.25\linewidth} p{0.65\linewidth}}
			\foreignlanguage{hebrew}{לְ} & \textit{to, for, regarding}, also indicating indirect objects \\
			\foreignlanguage{hebrew}{לֹא} & \textit{not, no} \\
			\foreignlanguage{hebrew}{לֵאמֹר} & introduction of direct speech \\
			\foreignlanguage{hebrew}{לֵב} & \textit{heart} (pl.\ \foreignlanguage{hebrew}{לִבּוֺת}) \\
			\foreignlanguage{hebrew}{לֵבָב} & \textit{heart} (cs.\ st. \foreignlanguage{hebrew}{לְבַב}) \\
			\foreignlanguage{hebrew}{לְבַד} & \textit{alone, by itsef} (with ePP \foreignlanguage{hebrew}{לְבַדּוֺ}; prep. \foreignlanguage{hebrew}{לְ} with noun \foreignlanguage{hebrew}{בַּד} \textit{separation}) \\
			\foreignlanguage{hebrew}{לְבִלְתִּי} & \textit{not} (negative used with the inf.\,cs.; \foreignlanguage{hebrew}{לְ} + \foreignlanguage{hebrew}{בִּלְתִּי}) \\
			\foreignlanguage{hebrew}{לבשׁ} & Q.\ \textit{to put on} (a garment), \textit{to clothe oneself, clothe} (stative verb, \\
			& SC \foreignlanguage{hebrew}{לָבֵשׁ}, PC \foreignlanguage{hebrew}{יִלְבַּשׁ}) \\
			\foreignlanguage{hebrew}{לחם} & Ni.\ \textit{to fight, do battle} (with prep. objects) \\
			\foreignlanguage{hebrew}{לֶחֶם} & \textit{bread, food} \\
			\foreignlanguage{hebrew}{לַ֫יְלָה} & \textit{night} \\
			\foreignlanguage{hebrew}{לין} & Q.\ \textit{to spend the night, stay overnight; to stay, dwell} \\ % HALOT
			\foreignlanguage{hebrew}{לכד} & Q.\ \textit{to capture, seize, take} \\
			\foreignlanguage{hebrew}{לָכֵן} & \textit{therefore} (\foreignlanguage{hebrew}{לְ} + \foreignlanguage{hebrew}{כֵּן}) \\
			\foreignlanguage{hebrew}{למד} & Qal \textit{to learn}; Pi. \textit{to teach}  \\
			\foreignlanguage{hebrew}{לָ֫מָּה} & \textit{why?} (preposition \foreignlanguage{hebrew}{לְ} + \foreignlanguage{hebrew}{מָה}) \\
			\foreignlanguage{hebrew}{לְמַ֫עַן} & \textit{in order to, so that} (conj.); \textit{on account of, for the sake of} (prep.) (\foreignlanguage{hebrew}{לְ} + \foreignlanguage{hebrew}{מַעַן}; prep. and conj.) \\
		\end{longtable}
		\unskip
		\unpenalty
		\unpenalty}
	\unvbox\ltmcbox
	
\end{multicols}

\begin{multicols}{2}
	
	\setbox\ltmcbox\vbox{
		\makeatletter\col@number\@ne
		\begin{longtable}{>{\raggedleft}p{0.25\linewidth} p{0.65\linewidth}}
			\foreignlanguage{hebrew}{לִפְנֵי} & \textit{in front of, before} (local and temporal) (prep. \foreignlanguage{hebrew}{לְ} + \foreignlanguage{hebrew}{פָּנִים}) \\
			\foreignlanguage{hebrew}{לקח} & Q.\ \textit{to take} \\
			\foreignlanguage{hebrew}{לִקְרַאת} & \textit{toward, against} (prep.; \foreignlanguage{hebrew}{לְ} + inf.\ cs.\ of the verb \foreignlanguage{hebrew}{קרא}\textsubscript{2} \textit{encounter}) \\
			\foreignlanguage{hebrew}{לָשׁוֺן} & \textit{tongue; language} (m.; pl.\ \foreignlanguage{hebrew}{לְשֹׁנוֺת}) \\
		\end{longtable}
		\unskip
		\unpenalty
		\unpenalty}
	\unvbox\ltmcbox
	
\end{multicols}

\medskip

\begin{center}
	\LARGE\foreignlanguage{hebrew}{מ}
\end{center}

\begin{multicols}{2}
	
	\setbox\ltmcbox\vbox{
		\makeatletter\col@number\@ne
		\begin{longtable}{>{\raggedleft}p{0.25\linewidth} p{0.65\linewidth}}
			\foreignlanguage{hebrew}{מְאֹד} & \textit{very} (degree particle); as a noun \textit{strength, power} \\
			\foreignlanguage{hebrew}{מֵאַ֫יִן} & \textit{whence? where from?} (prep.\ \foreignlanguage{hebrew}{מִן} + \foreignlanguage{hebrew}{אַיִן}) \\
			\foreignlanguage{hebrew}{מאן} & Pi.\ \textit{to refuse}  \\ % Only 46 occurrences, but mainly in Gen-Jer
			\foreignlanguage{hebrew}{מאס} & Q.\ \textit{to reject, despise} \\
			\foreignlanguage{hebrew}{מִגְדָּל} & \textit{tower} \\
			\foreignlanguage{hebrew}{מָגֵן} & \textit{shield} (pl.\ \foreignlanguage{hebrew}{מָגִנִּים}) \\
			\foreignlanguage{hebrew}{מִגְרָשׁ} & \textit{pastureland (belonging to a city); outskirts} \\ % HALOT
			\foreignlanguage{hebrew}{מִדְבָּר} & \textit{wilderness, desert} \\
			\foreignlanguage{hebrew}{מדד} & Q.\ \textit{to measure} \\
			\foreignlanguage{hebrew}{מִדָּה} & \textit{measured length; measurement} \\ % HALOT
			\foreignlanguage{hebrew}{מָדוּעַ} & \textit{why?} \\
			\foreignlanguage{hebrew}{מְדִינָה} & \textit{province} \\
			\foreignlanguage{hebrew}{מָה} & \textit{what?} \\
			\foreignlanguage{hebrew}{מהר} & Pi.\ \textit{to hasten; to do something hastily}  \\
			\foreignlanguage{hebrew}{מַהֵר} & \textit{quickly, speedily} (adv.; inf.\ abs.\ of \foreignlanguage{hebrew}{מהר} Pi.\ \textit{to hasten}) \\
			\foreignlanguage{hebrew}{מוּסָר} & \textit{discipline; training; exhortation, warning} \\
			\foreignlanguage{hebrew}{מוֺעֵד} & \textit{appointed time, appointed place, meeting} (\foreignlanguage{hebrew}{אֹהֶל מוֺעֵד} \textit{tent of meeting}) \\
			\foreignlanguage{hebrew}{מות} & Q.\ \textit{to die} (SC 3 m.\ sg.\ \foreignlanguage{hebrew}{מֵת}; PC 3 m.\ sg.\ \foreignlanguage{hebrew}{יָמוּת}; part. m.\ sg.\ \foreignlanguage{hebrew}{מֵת}); Hi.\ \textit{to kill, put to death} \\
			\foreignlanguage{hebrew}{מָוֶת} & \textit{death} \\
			\foreignlanguage{hebrew}{מִזְבֵּחַ} & \textit{altar} (pl. \foreignlanguage{hebrew}{מִזְבְּחוֺת}) \\
			\foreignlanguage{hebrew}{מִזְמוֺר} & \textit{psalm} \\
			\foreignlanguage{hebrew}{מִזְרָח} & \textit{sunrise; east} \\
			\foreignlanguage{hebrew}{מַחֲנֶה} & \textit{camp} \\
			\foreignlanguage{hebrew}{מָחָר} & \textit{tomorrow} (adv.) \\
			\foreignlanguage{hebrew}{מַחֲשָׁבָה} & \textit{thought; device, plan, purpose} \\ % BDB
			\foreignlanguage{hebrew}{מַטֶּה} & \textit{staff, tribe} \\
			\foreignlanguage{hebrew}{מִי} & \textit{who?} \\
			\foreignlanguage{hebrew}{מַיִם} & \textit{water} \\
			\foreignlanguage{hebrew}{מַכָּה} & \textit{blow; wound; plague} \\
			\foreignlanguage{hebrew}{מכר} & Q.\ \textit{to sell} \\
			\foreignlanguage{hebrew}{מלא} & Q.\ \textit{to be full} (stative verb; SC \foreignlanguage{hebrew}{מָלֵא}; with noun to indicate the substance) \\
			\foreignlanguage{hebrew}{מָלֵא} & \textit{full} (adj.) \\ % HALOT
			\foreignlanguage{hebrew}{מַלְאָךְ} & \textit{messenger, angel} \\
			\foreignlanguage{hebrew}{מְלָאכָה} & \textit{occupation, work} (\textit{məlā(ʾ)ḵā}; cs.\ st.\ \foreignlanguage{hebrew}{מְלֶ֫אכֶת}) \\
			\foreignlanguage{hebrew}{מִלְחָמָה} & \textit{battle, war} \\
			\foreignlanguage{hebrew}{מלט} & Ni.\ \textit{to escape, be delivered} \\
			\foreignlanguage{hebrew}{מלך} & Q.\ \textit{to reign} (as king) \textit{to become king} \\
			\foreignlanguage{hebrew}{מֶלֶךְ} & \textit{king} \\
			\foreignlanguage{hebrew}{מַלְכוּת} & \textit{royalty; royal power; reign; kingdom} \\
			\foreignlanguage{hebrew}{מַמְלָכָה} & \textit{kingdom, sovereignty, dominion, reign} (cs.\ st.\ \foreignlanguage{hebrew}{מַמְלֶכֶת}) \\
			\foreignlanguage{hebrew}{מִמָּעָל} & \textit{above} (adv.) (\foreignlanguage{hebrew}{מִמָּעָל לְ} prep. on top of, above) \\
			\foreignlanguage{hebrew}{מִן} & \textit{from, away from, out of}; in comparisons: \textit{(more) than} \\
		\end{longtable}
		\unskip
		\unpenalty
		\unpenalty}
	\unvbox\ltmcbox
	
\end{multicols}

\begin{multicols}{2}
	
	\setbox\ltmcbox\vbox{
		\makeatletter\col@number\@ne
		\begin{longtable}{>{\raggedleft}p{0.25\linewidth} p{0.65\linewidth}}
			\foreignlanguage{hebrew}{מִנְחָה} & \textit{gift, tribute, offering} \\
			\foreignlanguage{hebrew}{מִסְפָּר} & \textit{number} \\ % once also "tale" (Judg 7.15)
			\foreignlanguage{hebrew}{מְעַט} & \textit{a little, fewness, a few} \\
			\foreignlanguage{hebrew}{מַ֫עְלָה} & \textit{upwards, above} (adv.) (noun \foreignlanguage{hebrew}{מַעַל} \textit{higher part}) \\
			\foreignlanguage{hebrew}{מַעֲשֶׂה} & \textit{deed, work} \\
			\foreignlanguage{hebrew}{מצא} & Q.\ \textit{to find} \\
			\foreignlanguage{hebrew}{מַצָּה} & \textit{unleavened bread, matzah (matzo)} \\
			\foreignlanguage{hebrew}{מִצְוָה} & \textit{commandment} \\
			\foreignlanguage{hebrew}{מִקְדָּשׁ} & \textit{sacred place, sanctuary} \\ % BDB
			\foreignlanguage{hebrew}{מָקוֺם} & \textit{place} (pl. \foreignlanguage{hebrew}{מְקֹמוֺת}) \\
			\foreignlanguage{hebrew}{מִקְנֶה} & \textit{livestock as property, cattle} \\
			\foreignlanguage{hebrew}{מַרְאֶה} & \textit{sight, appearance, vision} \\ % BDB
			\foreignlanguage{hebrew}{מָרוֺם} & \textit{height, elevated place} \\
			\foreignlanguage{hebrew}{משׁח} & Q.\ \textit{to anoint, to smear} \\
			\foreignlanguage{hebrew}{מִשְׁכָּן} & \textit{dwelling-place, abode} \\
			\foreignlanguage{hebrew}{משׁל} & Q.\ \textit{to rule, reign} \\
			\foreignlanguage{hebrew}{מִשְׁמֶרֶת} & \textit{guard, watch; charge, injunction; function, office} \\ % BDB
			\foreignlanguage{hebrew}{מִשְׁפָּחָה} & \textit{extended family, clan} \\ % HALOT
			\foreignlanguage{hebrew}{מִשְׁפָט} & \textit{judgment} \\
			\foreignlanguage{hebrew}{מִשְׁתֶּה} & \textit{feast, banquet} \\ % only 46 ocurrences
			\foreignlanguage{hebrew}{מָתַי} & \textit{when?} \\
		\end{longtable}
		\unskip
		\unpenalty
		\unpenalty}
	\unvbox\ltmcbox
	
\end{multicols}

\medskip

\begin{center}
	\LARGE\foreignlanguage{hebrew}{נ}
\end{center}

\begin{multicols}{2}
	
	\setbox\ltmcbox\vbox{
		\makeatletter\col@number\@ne
		\begin{longtable}{>{\raggedleft}p{0.25\linewidth} p{0.65\linewidth}}
			\foreignlanguage{hebrew}{נְאֻם} & \textit{utterance, declaration, announcement} \\ % BDB and HALOT
			\foreignlanguage{hebrew}{נבא} & Ni.\ \textit{to be in a prophetic trance, behave like a} \foreignlanguage{hebrew}{נָבִיא}; Hitpa.\ \textit{to exhibit the behavior of a} \foreignlanguage{hebrew}{נָבִיא} \\ % HALOT
			\foreignlanguage{hebrew}{נבט} & Hi.\ \textit{to look} \\
			\foreignlanguage{hebrew}{נָבִיא} & \textit{prophet} \\
			\foreignlanguage{hebrew}{נֶגֶב} & \textit{south-country, Negev, south} \\
			\foreignlanguage{hebrew}{נגד} & Hi.\ \textit{to tell, inform, report} \\
			\foreignlanguage{hebrew}{נֶגֶד} & \textit{in front of, in sight of, opposite to} \\
			\foreignlanguage{hebrew}{נגע} & Q.\ \textit{to touch, reach, strike} \\
			\foreignlanguage{hebrew}{נֶגַע} & Q.\ \textit{stroke, plague, blow, mark} \\
			\foreignlanguage{hebrew}{נגשׁ} & Qal/Ni.\ \textit{to draw near, approach} \\
			\foreignlanguage{hebrew}{נֵדֶר} & \textit{vow} \\
			\foreignlanguage{hebrew}{נָהָר} & \textit{river, stream} (pl.\ \foreignlanguage{hebrew}{נְהָרוֺת}; \foreignlanguage{hebrew}{נְהַר פְּרָת} \textit{Euphrates}) \\
			\foreignlanguage{hebrew}{נדח} & Hi.\ \textit{to thrust; thrust out; thrust away} \\ % BDB
			\foreignlanguage{hebrew}{נוח} & Hi.\ A \textit{to cause to rest, give rest}; Hi.\ B \textit{to lay, set down; to leave, let remain} \\
			\foreignlanguage{hebrew}{נוס} & Q.\ \textit{to flee, escape} \\
			\foreignlanguage{hebrew}{נחל} & Q.\ \textit{to get a possession; to take a possession; to inherit} \\
			\foreignlanguage{hebrew}{נַחַל} & \textit{river valley, wadi; stream} \\
			\foreignlanguage{hebrew}{נַחֲלָה} & \textit{possession, property, inheritance} \\
			\foreignlanguage{hebrew}{נחם} & Ni.\ \textit{to regret; to be sorry; to console oneself}; Pi.\ \textit{to comfort} \\
			\foreignlanguage{hebrew}{נְחֹשֶׁת} & \textit{copper, bronze} \\
			\foreignlanguage{hebrew}{נטה} & Q. \textit{to stretch out, spread out, extend, bend, turn, incline}; Hi. \textit{to turn, incline} \\
			\foreignlanguage{hebrew}{נטע} & Q.\ \textit{to plant} \\
			\foreignlanguage{hebrew}{נכה} & Hi. \textit{to smite, to strike; to strike dead} \\
			\foreignlanguage{hebrew}{נסע} & \textit{to pull up or out, set out, journey} \\
			\foreignlanguage{hebrew}{נַעַר} & \textit{boy, lad, youth} \\
		\end{longtable}
		\unskip
		\unpenalty
		\unpenalty}
	\unvbox\ltmcbox
	
\end{multicols}

\newpage

\begin{multicols}{2}
	
	\setbox\ltmcbox\vbox{
		\makeatletter\col@number\@ne
		\begin{longtable}{>{\raggedleft}p{0.25\linewidth} p{0.65\linewidth}}
			\foreignlanguage{hebrew}{נַעֲרָה} & \textit{young woman, girl; maid, attendant, servant} \\
			\foreignlanguage{hebrew}{נפל} & \textit{to fall} \\
			\foreignlanguage{hebrew}{נֶפֶשׁ} & \textit{soul, self, life, living being, person, desire, throat} (f.) \\
			\foreignlanguage{hebrew}{נצב} & Ni.\ \textit{to take one's stand, to stand} \\
			\foreignlanguage{hebrew}{נצל} & Hi.\ \textit{to take away, snatch away; to rescue; to deliver from} \\
			\foreignlanguage{hebrew}{נצר} & Q.\ \textit{to watch, guard, keep} \\
			\foreignlanguage{hebrew}{נשׂא} & Q.\ \textit{to lift, carry, take} \\
			\foreignlanguage{hebrew}{נשׂג} & Hi.\ \textit{to reach; to overtake} \\
			\foreignlanguage{hebrew}{נָשִׂיא} & \textit{chief, prince} \\ % BDB
			\foreignlanguage{hebrew}{נתן} & Q.\ \textit{to give, to put, to make into, to allow} \\
		\end{longtable}
		\unskip
		\unpenalty
		\unpenalty}
	\unvbox\ltmcbox
	
\end{multicols}

\medskip

\begin{center}
	\LARGE\foreignlanguage{hebrew}{ס}
\end{center}

\begin{multicols}{2}
	
	\setbox\ltmcbox\vbox{
		\makeatletter\col@number\@ne
		\begin{longtable}{>{\raggedleft}p{0.25\linewidth} p{0.65\linewidth}}
			\foreignlanguage{hebrew}{סבב} & Q.\ \textit{to turn about, go around, surround} \\
			\foreignlanguage{hebrew}{סָבִיב} & \textit{around, round about} (adv. and prep., as a noun \textit{surroundings}) \\
			\foreignlanguage{hebrew}{סְבִיבוֺת} & \textit{around, round about} (prep.; pl.\ of \foreignlanguage{hebrew}{סָבִיב}, with ePP \foreignlanguage{hebrew}{סְבִיבֹתֵיהֶם}) \\
			\foreignlanguage{hebrew}{סגר} & Q.\ \textit{to shut, close}; Hi.\ \textit{to deliver, hand over; to shut} \\
			\foreignlanguage{hebrew}{סוּס} & \textit{horse} \\
			\foreignlanguage{hebrew}{סור} & Q.\ \textit{to turn aside}; Hi.\ \textit{to cause to turn aside, depart; to remove, take away} \\
			\foreignlanguage{hebrew}{סֶלַע} & \textit{rock, cliff} \\
			\foreignlanguage{hebrew}{ספר} & Q.\ \textit{to count}; Pi.\ \textit{to report; tell; to make known, announce}  \\
			\foreignlanguage{hebrew}{סֵפֶר} & \textit{something written, letter, scroll} \\ % HALOT
			\foreignlanguage{hebrew}{סֹפֵר} & \textit{secretary, scribe} \\
			\foreignlanguage{hebrew}{סתר} & Hi.\ \textit{to hide, conceal} (transitive); Ni.\ \textit{to hide oneself; to be hid, concealed} \\
		\end{longtable}
		\unskip
		\unpenalty
		\unpenalty}
	\unvbox\ltmcbox
	
\end{multicols}

\medskip

\begin{center}
	\LARGE\foreignlanguage{hebrew}{ע}
\end{center}

\begin{multicols}{2}
	
	\setbox\ltmcbox\vbox{
		\makeatletter\col@number\@ne
		\begin{longtable}{>{\raggedleft}p{0.25\linewidth} p{0.65\linewidth}}
			\foreignlanguage{hebrew}{עבד} & Q.\ \textit{to serve, work} \\
			\foreignlanguage{hebrew}{עֶבֶד} & \textit{slave, servant} \\
			\foreignlanguage{hebrew}{עֲבֹדָה} & \textit{work, service} \\
			\foreignlanguage{hebrew}{עבר} & Q.\ \textit{to pass over, through, by, pass on} \\
			\foreignlanguage{hebrew}{עֵבֶר} & \textit{region across or beyond; side} \\ % BDB
			\foreignlanguage{hebrew}{עַד} & \textit{up to, as far as, until} \\
			\foreignlanguage{hebrew}{עֵד} & \textit{witness} \\
			\foreignlanguage{hebrew}{עֵדָה} & \textit{congregation} \\
			\foreignlanguage{hebrew}{עֵדוּת} & \textit{witness, testimony}, pl. \textit{laws, legal provisions} (pl. \foreignlanguage{hebrew}{עֵדֹת}; \\
			& with ePP also \foreignlanguage{hebrew}{עֵדְוֹתָיו} \textit{ʿēdwōtāw} \textit{his legal provisions}, etc.) \\
			\foreignlanguage{hebrew}{עוֺד} & \textit{still, yet, again, besides} (adv.) \\
			\foreignlanguage{hebrew}{עוֹלָם} & \textit{long time, duration, future time, long time back} \\
			\foreignlanguage{hebrew}{עָוֹן} & \textit{iniquity, guilt} (pl.\ \foreignlanguage{hebrew}{עֲוֹנוֺת})\\ % BDB
			\foreignlanguage{hebrew}{עוֺף} & \textit{birds, fowl; winged insects} (everything that flies) (coll.) \\ % BDB, HALOT
			\foreignlanguage{hebrew}{עור} & Q.\ \textit{to awake; to stir}; Hi.\ \textit{to wake up; to excite; to put into motion} \\ % HALOT
		\end{longtable}
		\unskip
		\unpenalty
		\unpenalty}
	\unvbox\ltmcbox
	
\end{multicols}

\newpage

\begin{multicols}{2}
	
	\setbox\ltmcbox\vbox{
		\makeatletter\col@number\@ne
		\begin{longtable}{>{\raggedleft}p{0.25\linewidth} p{0.65\linewidth}}
			\foreignlanguage{hebrew}{עוֺר} & \textit{skin; animal skin; leather} \\
			\foreignlanguage{hebrew}{עֹז} & \textit{might, strength} \\
			\foreignlanguage{hebrew}{עזב} & Q.\ \textit{to leave, forsake} \\
			\foreignlanguage{hebrew}{עזר} & Q.\ \textit{to help, assist}  \\
			\foreignlanguage{hebrew}{עַיִן} & \textit{eye, spring} (fem.) \\
			\foreignlanguage{hebrew}{עִיר} & \textit{city, town} (fem.) \\
			\foreignlanguage{hebrew}{עַל} & \textit{on, upon, above, at, by, against, on account of, because of, about,} \\
			& \textit{concerning} \\
			\foreignlanguage{hebrew}{עַל אֲשֶׁר} & \textit{because} \\
			\foreignlanguage{hebrew}{עלה} & Q.\ \textit{to go up, ascend} \\
			\foreignlanguage{hebrew}{עֹלָה} & \textit{(whole) burnt offering} \\
			\foreignlanguage{hebrew}{עֶלְיוֺן} & \textit{highest, most high} \\
			\foreignlanguage{hebrew}{עַל־כֵּן} & \textit{therefore} \\
			\foreignlanguage{hebrew}{עַם} & \textit{people, nation} (also \foreignlanguage{hebrew}{עָם}) \\
			\foreignlanguage{hebrew}{עִם} & \textit{with, close to, beside} \\
			\foreignlanguage{hebrew}{עמד} & Q.\ \textit{to take one's stand, stand} \\
			\foreignlanguage{hebrew}{עַמּוּד} & \textit{pillar; column} (cf. \foreignlanguage{hebrew}{עמד}) \\
			\foreignlanguage{hebrew}{עָמָל} & \textit{trouble; labor; toil} \\
			\foreignlanguage{hebrew}{עֵמֶק} & \textit{valley, vale, lowland} \\ % BDB
			\foreignlanguage{hebrew}{ענה} & Q.\ \textit{to answer, respond} \\
			\foreignlanguage{hebrew}{ענה}\textsubscript{2} & Q.\ \textit{to be bowed down; to be downcast, afflicted}; Pi.\ \textit{to oppress; to mishandle, humble, afflict} \\ % mostly BDB
			\foreignlanguage{hebrew}{עָנִי} & \textit{poor; afflicted; humble} \\
			\foreignlanguage{hebrew}{עָנָן} & \textit{cloud, cloud-mass} \\
			\foreignlanguage{hebrew}{עָפָר} & \textit{dry earth; dust} \\ % BDB
			\foreignlanguage{hebrew}{עֵץ} & \textit{tree, wood}, sometimes \textit{trees} (coll.) (pl.\ \foreignlanguage{hebrew}{עֵצִים}) \\
			\foreignlanguage{hebrew}{עֵצָה} & \textit{counsel, advice; plan} \\ % BDB, HALOT
			\foreignlanguage{hebrew}{עֶצֶם} & \textit{bone} (m.\ or f.; pl.\ \foreignlanguage{hebrew}{עֲצָמוֺת}) \\
			\foreignlanguage{hebrew}{עֶרֶב} & \textit{evening} \\
			\foreignlanguage{hebrew}{עֲרָבָה} & \textit{desert-plain, steppe} (with the article \foreignlanguage{hebrew}{הָעֲרָבָה} usually a part of the arid region between the Sea of Galilea and the Gulf of Aqaba) \\
			\foreignlanguage{hebrew}{עֶרְוָה} & \textit{nakedness; genital area} \\
			\foreignlanguage{hebrew}{ערך} & Q.\ \textit{to arrange, set in order; to draw up a battle formation}  \\
			\foreignlanguage{hebrew}{עשׂה} & Q.\ \textit{to do, to make} \\
			\foreignlanguage{hebrew}{עֵת} & \textit{time} (f.; gem. noun; with enclitic pronoun \foreignlanguage{hebrew}{עִתּוֺ}) \\
			\foreignlanguage{hebrew}{עַתָּה} & \textit{now} (adv. of time) \\
		\end{longtable}
		\unskip
		\unpenalty
		\unpenalty}
	\unvbox\ltmcbox
	
\end{multicols}

\medskip

\begin{center}
	\LARGE\foreignlanguage{hebrew}{פ}
\end{center}

\begin{multicols}{2}
	
	\setbox\ltmcbox\vbox{
		\makeatletter\col@number\@ne
		\begin{longtable}{>{\raggedleft}p{0.25\linewidth} p{0.65\linewidth}}
			\foreignlanguage{hebrew}{פֵּאָה} & \textit{corner; side} \\
			\foreignlanguage{hebrew}{פדה} & Q.\ \textit{to buy out, redeem, ransom} \\ % HALOT, BDB
			\foreignlanguage{hebrew}{פֶּה} & \textit{mouth} \\
			\foreignlanguage{hebrew}{פֹּה} & \textit{here} (adv.) \\
			\foreignlanguage{hebrew}{פוץ} & Q.\ \textit{to be dispersed; disperse}; Ni.\ \textit{be scattered}; Hi.\ \textit{to scatter} \\ % BDB
			\foreignlanguage{hebrew}{פלא} & Ni.\ \textit{to be surpassing, extraordinary} \\ % BDB
			\foreignlanguage{hebrew}{פלל} & Hitpa.\ \textit{to pray} \\
			\foreignlanguage{hebrew}{פְּלִשְׁתִּים} & \textit{Philistines} (sg.\ \foreignlanguage{hebrew}{פְּלִשְׁתִּי}) \\
			\foreignlanguage{hebrew}{פֶן} & \textit{so that not, lest} (conj.) \\
			\foreignlanguage{hebrew}{פנה} & Q.\ \textit{to turn (turn to one side, head in a particular direction; turn round; turn away and go on further)} \\ % HALOT
			\foreignlanguage{hebrew}{פָּנִים} & \textit{face, surface} (only pl.) \\
			\foreignlanguage{hebrew}{פֶּסַח} & \textit{Passover, Pesach} \\
			\foreignlanguage{hebrew}{פעל} & Q.\ \textit{to do; to make} \\
			\foreignlanguage{hebrew}{פַּעַם} & \textit{step, pace; foot; time} (Tagalog \textit{beses}) \\
			\foreignlanguage{hebrew}{פקד} & Q.\ \textit{to visit, attend to, muster, appoint}; Hi.\ \textit{to set (over), make overseer; commit, entrust; deposit} \\
		\end{longtable}
		\unskip
		\unpenalty
		\unpenalty}
	\unvbox\ltmcbox
	
\end{multicols}

\begin{multicols}{2}
	
	\setbox\ltmcbox\vbox{
		\makeatletter\col@number\@ne
		\begin{longtable}{>{\raggedleft}p{0.25\linewidth} p{0.65\linewidth}}
			\foreignlanguage{hebrew}{פַּר} & \textit{bull, steer} (gem.\ noun; pl. \foreignlanguage{hebrew}{פָרִים}) \\ % HALOT
			\foreignlanguage{hebrew}{פְּרִי} & \textit{fruit} \\
			\foreignlanguage{hebrew}{פַּרְעֹה} & \textit{Pharao} \\
			\foreignlanguage{hebrew}{פרר} & Hi.\ \textit{to break, violate; to frustrate, make ineffectual} \\
			\foreignlanguage{hebrew}{פרשׂ} & Q.\ \textit{to spread out, spread} \\
			\foreignlanguage{hebrew}{פָּרָשׁ} & \textit{horseman, charioteer; team of horses, horses for a chariot} (pl.\ \foreignlanguage{hebrew}{פָּרָשִׁים}) \\ % HALOT
			\foreignlanguage{hebrew}{פֶּשַׁע} & \textit{transgression, crime} \\ % BDB, HALOT
			\foreignlanguage{hebrew}{פתח} & Q.\ \textit{to open}; Pi.\ \textit{to let loose; to untie; to liberate}  \\
			\foreignlanguage{hebrew}{פֶּתַח} & \textit{opening, doorway, entrance} \\
		\end{longtable}
		\unskip
		\unpenalty
		\unpenalty}
	\unvbox\ltmcbox
	
\end{multicols}

\medskip

\begin{center}
	\LARGE\foreignlanguage{hebrew}{צ}
\end{center}

\begin{multicols}{2}
	
	\setbox\ltmcbox\vbox{
		\makeatletter\col@number\@ne
		\begin{longtable}{>{\raggedleft}p{0.25\linewidth} p{0.65\linewidth}}
			\foreignlanguage{hebrew}{צֹאן} & \textit{small cattle, sheep and goats, flock, flocks} (coll., fem.) \\
			\foreignlanguage{hebrew}{צָבָא} & \textit{army, host, war, warfare} (pl.\ \foreignlanguage{hebrew}{צְבָאוֺת}) \\
			\foreignlanguage{hebrew}{צַדִּיק} & \textit{just, righteous} \\
			\foreignlanguage{hebrew}{צֶדֶק} & \textit{rightness; righteousness} \\ % BDB
			\foreignlanguage{hebrew}{צְדָקָה} & \textit{righteousness} \\
			\foreignlanguage{hebrew}{צוה} & Pi.\ \textit{to give an order, to command; to order, to instruct; to charge, to commission} (cf. the noun \foreignlanguage{hebrew}{מִצְוָה}) \\
			\foreignlanguage{hebrew}{צוּר} & \textit{rock; rocky hill, mountain} \\ % HALOT
			\foreignlanguage{hebrew}{צֵל} & \textit{shadow, shade} (gem. noun; with ePP \foreignlanguage{hebrew}{צִלְּךָ}) \\
			\foreignlanguage{hebrew}{צלח} & Q.\ \textit{to force entry to; to succeed, be successful}; Hi.\ \textit{to be successful; to make something a success} \\ % HALOT (BDB splits the verb into two roots)
			\foreignlanguage{hebrew}{צעק} & Q.\ \textit{to cry, cry out, call} \\
			\foreignlanguage{hebrew}{צָפוֺן} & \textit{north} \\
			\foreignlanguage{hebrew}{צַר} & \textit{enemy} (pl.\ \foreignlanguage{hebrew}{צָרִים}) \\
			\foreignlanguage{hebrew}{צָרָה} & \textit{need; distress; anxiety} \\ % HALOT
		\end{longtable}
		\unskip
		\unpenalty
		\unpenalty}
	\unvbox\ltmcbox
	
\end{multicols}

\medskip

\begin{center}
	\LARGE\foreignlanguage{hebrew}{ק}
\end{center}

\begin{multicols}{2}
	
	\setbox\ltmcbox\vbox{
		\makeatletter\col@number\@ne
		\begin{longtable}{>{\raggedleft}p{0.25\linewidth} p{0.65\linewidth}}
			\foreignlanguage{hebrew}{קבץ} & Q.\ \textit{to gather, collect} \\
			\foreignlanguage{hebrew}{קבר} & Q.\ \textit{to bury} \\
			\foreignlanguage{hebrew}{קֶבֶר} & \textit{grave} \\
			\foreignlanguage{hebrew}{קָדוֺשׁ} & \textit{holy} \\
			\foreignlanguage{hebrew}{קָדִים} & \textit{East; east wind} \\ % BDB
			\foreignlanguage{hebrew}{קֶדֶם} & \textit{front; east; before, earlier; prehistoric times, primeval times} \\
			\foreignlanguage{hebrew}{קדשׁ} & Q.\ \textit{to be holy} (stative verb; SC pausal form \foreignlanguage{hebrew}{קָדֵ֑שׁוּ}; PC \foreignlanguage{hebrew}{יִקְדַּשׁ}); Pi.\ \textit{to sanctify, make holy; to declare holy; to dedicate}; Hi.\ \textit{to set apart, devote, consecrate; regard} or \textit{treat as sacred} \\
			\foreignlanguage{hebrew}{קֹדֶשׁ} & \textit{holiness} \\
			\foreignlanguage{hebrew}{קָהָל} & \textit{assembly} \\
			\foreignlanguage{hebrew}{קוֺל} & \textit{voice, sound} (masc., pl.\ \foreignlanguage{hebrew}{קֹלוֺת}) \\
			\foreignlanguage{hebrew}{קום} & Q.\ \textit{to arise, stand up, stand}; Hi.\ \textit{to raise, raise up} \\ % BDB
			\foreignlanguage{hebrew}{קָטֹן} & \textit{small; young; unimportant} (adj.; cs.\ st.\ \foreignlanguage{hebrew}{קְטֹן}; only in m.\ sg., other forms from \foreignlanguage{hebrew}{קָטָן}) \\ % HALOT
			\foreignlanguage{hebrew}{קָטָן} & \textit{small; young; unimportant} (adj.; pl.\ \foreignlanguage{hebrew}{קְטַנִּים}) \\ % HALOT
			\foreignlanguage{hebrew}{קטר} & Hi.\ \textit{to cause to go up in smoke}; Pi.\ \textit{to make a sacrifice go up in smoke} \\ % HALOT
		\end{longtable}
		\unskip
		\unpenalty
		\unpenalty}
	\unvbox\ltmcbox
	
\end{multicols}

\begin{multicols}{2}
	
	\setbox\ltmcbox\vbox{
		\makeatletter\col@number\@ne
		\begin{longtable}{>{\raggedleft}p{0.25\linewidth} p{0.65\linewidth}}
			\foreignlanguage{hebrew}{קְטֹרֶת} & \textit{smoke, odor of (burning) sacrifice; incense} (fem.) \\ % HALOT
			\foreignlanguage{hebrew}{קִיר} & \textit{wall} (pl. \foreignlanguage{hebrew}{קִירוֺת}) \\
			\foreignlanguage{hebrew}{קלל} & Q.\  \textit{to be small, insignificant; to be swift}; Pi. \textit{to curse}  \\
			\foreignlanguage{hebrew}{קנה} & Q.\ \textit{to buy; to acquire; ; to create} \\
			\foreignlanguage{hebrew}{קָנֶה} & \textit{stalk, reed} \\
			\foreignlanguage{hebrew}{קֵץ} & \textit{end} (gem. noun; with ePP \foreignlanguage{hebrew}{קִצּוֺ}) \\
			\foreignlanguage{hebrew}{קָצֶה} & \textit{edge, end, extremity} \\	% HALOT
			\foreignlanguage{hebrew}{קָצִיר} & \textit{harvest; harvest crops} \\
			\foreignlanguage{hebrew}{קרא} & Q.\ \textit{to call, proclaim, read} \\
			\foreignlanguage{hebrew}{קרב} & Q.\ \textit{to draw near, approach} (stative verb; SC \foreignlanguage{hebrew}{קָרַב} and \foreignlanguage{hebrew}{קָרֵב}; PC \foreignlanguage{hebrew}{יִקְרַב}) \\
			\foreignlanguage{hebrew}{קֶרֶב} & \textit{inward part, midst} (with ePP \foreignlanguage{hebrew}{קִרְבּוֺ}) \\
			\foreignlanguage{hebrew}{קָרְבָּן} & \textit{offering, gift} (only in Lev, Num, Ezek) \\
			\foreignlanguage{hebrew}{קָרוֺב} & \textit{near} \\
			\foreignlanguage{hebrew}{קֶרֶן} & \textit{horn} \\
			\foreignlanguage{hebrew}{קרע} & Q.\ \textit{to tear} \\
			\foreignlanguage{hebrew}{קֶשֶׁת} & \textit{bow} (fem.)\\
		\end{longtable}
		\unskip
		\unpenalty
		\unpenalty}
	\unvbox\ltmcbox
	
\end{multicols}

\medskip

\begin{center}
	\LARGE\foreignlanguage{hebrew}{ר}
\end{center}

\begin{multicols}{2}
	
	\setbox\ltmcbox\vbox{
		\makeatletter\col@number\@ne
		\begin{longtable}{>{\raggedleft}p{0.25\linewidth} p{0.65\linewidth}}
			\foreignlanguage{hebrew}{ראה} & Q.\ \textit{to see}; Ni.\ \textit{to appear, become visible; to present oneself}; Hi.\ \textit{to show, let someone see something} \\ % HALOT
			\foreignlanguage{hebrew}{רֹאשׁ} & \textit{head} \\
			\foreignlanguage{hebrew}{רֵאשִׁית} & \textit{beginning; first, chief} \\
			\foreignlanguage{hebrew}{רַב} & \textit{much, many, great} (pl. \foreignlanguage{hebrew}{רַבִּים}) \\
			\foreignlanguage{hebrew}{רֹב} & \textit{multitude, abundance, greatness} \\
			\foreignlanguage{hebrew}{רבה} & Q.\ \textit{to become numerous, increase; to become great}; Hi.\ \textit{to make numerous; to make great} \\ % HALOT
			\foreignlanguage{hebrew}{רדף} & \textit{to pursue, chase, persecute} \\
			\foreignlanguage{hebrew}{רוּחַ} & \textit{spirit, wind} (fem.\ or masc.) \\
			\foreignlanguage{hebrew}{רום} & Q.\ \textit{to be high; to be raised, uplifted; to rise} (intrans.); Hi.\ \textit{to raise, lift up; to exalt; to set up}; Polel \textit{to raise, lift up; to exalt} \\ % BDB
			\foreignlanguage{hebrew}{רוץ} & Q.\ \textit{to run} \\
			\foreignlanguage{hebrew}{רֹחַב} & \textit{breadth, width} \\
			\foreignlanguage{hebrew}{רְחֹב} & \textit{broad open place, plaza} \\
			\foreignlanguage{hebrew}{רָחוֺק} & \textit{distant, far} (adj.) \\
			\foreignlanguage{hebrew}{רחץ} & Q.\ \textit{to wash, bathe} \\
			\foreignlanguage{hebrew}{רחק} & Q.\ \textit{to be far, distant}; Hi.\ \textit{to remove; to depart, withdraw, distance oneself} \\ % Qal BDB, Hi. HALOT
			\foreignlanguage{hebrew}{ריב} & Q.\ \textit{to strive, contend} \\ % BDB
			\foreignlanguage{hebrew}{רִיב} & \textit{dispute, lawsuit, quarrel} \\ % HALOT
			\foreignlanguage{hebrew}{רכב} & Q.\ \textit{to mount and ride, ride}; Hi.\ \textit{to cause ride, to make to ride} \\ % BDB
			\foreignlanguage{hebrew}{רֶכֶב} & \textit{chariotry, chariots} (collective), \textit{chariot} (rare meaning) \\
			\foreignlanguage{hebrew}{רַע} & \textit{bad, evil} (pl. \foreignlanguage{hebrew}{רָעִים}; gem. adj.) \\
			\foreignlanguage{hebrew}{רֵעַ} & \textit{friend, companion, fellow} \\
			\foreignlanguage{hebrew}{רָעָב} & \textit{famine, hunger} \\
			\foreignlanguage{hebrew}{רעה} & Q.\ \textit{to pasture, tend, graze} (part.\ \foreignlanguage{hebrew}{רֹעֶה} \textit{shepherd}) \\
			\foreignlanguage{hebrew}{רָעָה} & \textit{evil; distress; misery; disaster, calamity} (root \foreignlanguage{hebrew}{רעע}) \\
			\foreignlanguage{hebrew}{רעע} & Q.\ \textit{to be bad, evil, displeasing}; Hi. \textit{to do evil; to treat badly} \\ % HALOT
		\end{longtable}
		\unskip
		\unpenalty
		\unpenalty}
	\unvbox\ltmcbox
	
\end{multicols}

\newpage

\begin{multicols}{2}
	
	\setbox\ltmcbox\vbox{
		\makeatletter\col@number\@ne
		\begin{longtable}{>{\raggedleft}p{0.25\linewidth} p{0.65\linewidth}}
			\foreignlanguage{hebrew}{רפא} & Q.\ \textit{to heal}; Ni.\ \textit{to be healed} \\
			\foreignlanguage{hebrew}{רצה} & Q.\ \textit{to be pleased with; to accept favorably} \\
			\foreignlanguage{hebrew}{רָצוֺן} & \textit{goodwill; favor; acceptance} \\
			\foreignlanguage{hebrew}{רַק} & \textit{only, still, but, however, nevertheless, surely} \\
			\foreignlanguage{hebrew}{רָשָׁע} & \textit{wicked, guilty} \\
		\end{longtable}
		\unskip
		\unpenalty
		\unpenalty}
	\unvbox\ltmcbox
	
\end{multicols}

\medskip

\begin{center}
	\LARGE\foreignlanguage{hebrew}{שׂ}
\end{center}

\begin{multicols}{2}
	
	\setbox\ltmcbox\vbox{
		\makeatletter\col@number\@ne
		\begin{longtable}{>{\raggedleft}p{0.25\linewidth} p{0.65\linewidth}}
			\foreignlanguage{hebrew}{שׂבע} & Q. \textit{be sated, to eat one's fill, satisfy oneself with} (stative verb, SC pausal form \foreignlanguage{hebrew}{שָׂבֵ֑עוּ}, PC \foreignlanguage{hebrew}{יִשְׂבַּע}) (with noun phrase) \\ % BDB, HALOT
			\foreignlanguage{hebrew}{שֶׂה} & \textit{a small livestock beast} (\textit{sheep} or \textit{goat}) \\
			\foreignlanguage{hebrew}{שָׂדֶה} & \textit{field} (pl. \foreignlanguage{hebrew}{שָׂדוֺת}) \\
			\foreignlanguage{hebrew}{שׂים} & Q.\ \textit{to put, to make into, to give} \\
			\foreignlanguage{hebrew}{שׂכל} & Hi.\ \textit{to understand; to have insight; to make wise, insightful; to achieve success} \\
			\foreignlanguage{hebrew}{שְׂמֹאל} & \textit{left side, left, north} (\textit{śəmō(ʾ)l} with silent \foreignlanguage{hebrew}{א}) \\
			\foreignlanguage{hebrew}{שׂמח} & Q.\ \textit{to rejoice, be glad} (stative verb) \\
			\foreignlanguage{hebrew}{שִׂמְחָה} & \textit{joy, gladness} \\ % BDB
			\foreignlanguage{hebrew}{שׂנא} & Q.\ \textit{to hate} (stative verb, PC \foreignlanguage{hebrew}{שָׁנֵא}) \\
			\foreignlanguage{hebrew}{שָׂעִיר} & \textit{billy-goat, buck} \\
			\foreignlanguage{hebrew}{שָׂפָה} & \textit{lip, speech, edge} \\
			\foreignlanguage{hebrew}{שַׂר} & \textit{ruler, official, prince, captain, chieftain, chief} (pl.\ \foreignlanguage{hebrew}{שָׂרִים}) \\ % BDB
			\foreignlanguage{hebrew}{שׂרף} & Q.\ \textit{to burn completely} (transitive) \\
		\end{longtable}
		\unskip
		\unpenalty
		\unpenalty}
	\unvbox\ltmcbox
	
\end{multicols}

\medskip

\begin{center}
	\LARGE\foreignlanguage{hebrew}{שׁ}
\end{center}

\begin{multicols}{2}
	
	\setbox\ltmcbox\vbox{
		\makeatletter\col@number\@ne
		\begin{longtable}{>{\raggedleft}p{0.25\linewidth} p{0.65\linewidth}}
			\foreignlanguage{hebrew}{שְׁאוֺל} & \textit{underworld} (fem.) \\ % BDB
			\foreignlanguage{hebrew}{שׁאל} & Q.\ \textit{to ask, inquire} \\
			\foreignlanguage{hebrew}{שׁאר} & Ni.\ \textit{be left over, remain}; Hi.\ \textit{to leave over, spare} \\ % BDB
			\foreignlanguage{hebrew}{שְׁאֵרִית} & \textit{remainder; remnant} (fem.) \\ % HALOT
			\foreignlanguage{hebrew}{שֵׁבֶט} & \textit{rod, staff, scepter, tribe} \\
			\foreignlanguage{hebrew}{שׁבע} & Ni.\ \textit{to swear, take an oath} \\
			\foreignlanguage{hebrew}{שׁבר} & Q.\ \textit{to break, break in pieces} \\
			\foreignlanguage{hebrew}{שׁבת} & Q.\ \textit{to cease; to desist; to rest}; Hi.\ \textit{to cause to cease, put an end to exterminate, destroy; to remove} \\
			\foreignlanguage{hebrew}{שַׁבָּת} & \textit{Sabbath} (f.\ or m.; pl.\ \foreignlanguage{hebrew}{שַׁבָּתוֺת}) \\
			\foreignlanguage{hebrew}{שׁדד} & Q.\ \textit{to devastate, despoil, deal violently with} \\
			\foreignlanguage{hebrew}{שׁדד} & Q.\ \textit{to devastate, despoil, deal violently with} \\ % HALOT
			\foreignlanguage{hebrew}{שׁוב} & Q.\ \textit{to turn back, return}; Hi.\ \textit{to bring back, return} (trans.) \\
			\foreignlanguage{hebrew}{שָׁוְא} & \textit{emptiness; nothingness} \\ % BDB (vanity not included)
			\foreignlanguage{hebrew}{שׁוֺפָר} & \textit{horn} (as a wind instrument) \\
			\foreignlanguage{hebrew}{שׁוֺר} & \textit{a single beast, bovid} (\textit{bull}, etc.) \\
			\foreignlanguage{hebrew}{שׁחט} & Q.\ \textit{to slaughter} \\
			\foreignlanguage{hebrew}{שׁחת} & Hi./Pi.\ \textit{to ruin, destroy; annihilate, exterminate} \\ % HALOT
			\foreignlanguage{hebrew}{שׁיר} & Q.\ \textit{to sing} (PC \foreignlanguage{hebrew}{יָשִׁיר}) \\
			\foreignlanguage{hebrew}{שִׁיר} & \textit{song} \\
			\foreignlanguage{hebrew}{שׁית} & Q.\ \textit{to set, put} (PC \foreignlanguage{hebrew}{יָשִׁית}) \\
			
		\end{longtable}
		\unskip
		\unpenalty
		\unpenalty}
	\unvbox\ltmcbox
	
\end{multicols}

\begin{multicols}{2}
	
	\setbox\ltmcbox\vbox{
		\makeatletter\col@number\@ne
		\begin{longtable}{>{\raggedleft}p{0.25\linewidth} p{0.65\linewidth}}
			\foreignlanguage{hebrew}{שׁכב} & Q.\ \textit{to lie down} (3 m.\ sg.\ PC \foreignlanguage{hebrew}{יִשְׁכַּב}) \\
			\foreignlanguage{hebrew}{שׁכח} & Q.\ \textit{to forget} \\
			\foreignlanguage{hebrew}{שׁכם} & Hi. \textit{to do something early; to rise early} \\
			\foreignlanguage{hebrew}{שְׁכֶם} & \textit{shoulder, back} \\
			\foreignlanguage{hebrew}{שׁכן} & Q.\ \textit{to settle down, dwell} \\
			\foreignlanguage{hebrew}{שָׁלוֺם} & \textit{welfare, peace, completeness} \\
			\foreignlanguage{hebrew}{שׁלח} & Q.\ \textit{to send}; Pi.\ \textit{to send off; to send out, forth; to let go, set free}  \\ % BDB
			\foreignlanguage{hebrew}{שֻׁלְחָן} & \textit{table} \\
			\foreignlanguage{hebrew}{שׁלך} & Hi.\ \textit{to throw} \\
			\foreignlanguage{hebrew}{שָׁלָל} & \textit{spoil, plunder, booty} \\
			\foreignlanguage{hebrew}{שׁלם} & Pi.\ \textit{to make whole} or \textit{good, restore; make good}, i.e., \textit{pay} (vows); \textit{to requite, recompense, reward} \\ % BDB
			\foreignlanguage{hebrew}{שֶׁלֶם} & a kind of offering: \textit{conclusion offering; salvation offering} \\ % BDB
			\foreignlanguage{hebrew}{שָׁם} & \textit{there} (adv.) \\
			\foreignlanguage{hebrew}{שֵׁם} & \textit{name} (pl. \foreignlanguage{hebrew}{שֵׁמוֺת}) \\
			\foreignlanguage{hebrew}{שׁמד} & Hi.\ \textit{to annihilate, exterminate; destroy} \\
			\foreignlanguage{hebrew}{שָׁמַיִם} & \textit{heaven, sky} \\
			\foreignlanguage{hebrew}{שׁמם} & Q.\ \textit{to be uninhabited, be deserted; to shudder, be appalled}; Ni.\ \textit{to be made uninhabited, deserted; to be made to tremble} \\ % HALOT
			\foreignlanguage{hebrew}{שְׁמָמָה} & \textit{desolation; devastation; waste} \\ % DCH
			\foreignlanguage{hebrew}{שֶׁמֶן} & \textit{oil, fat, fatness} \\
			\foreignlanguage{hebrew}{שׁמע} & Q.\ \textit{to hear, to listen to} \\
			\foreignlanguage{hebrew}{שׁמר} & Q.\ \textit{to keep, watch, preserve} \\
			\foreignlanguage{hebrew}{שֶׁמֶשׁ} & \textit{sun} (fem.) \\
			\foreignlanguage{hebrew}{שֵׁן} & \textit{tooth, ivory} (gem. noun; dual \foreignlanguage{hebrew}{שִׁנַּיִם}) \\
			\foreignlanguage{hebrew}{שָׁנָה} & \textit{year} (pl. \foreignlanguage{hebrew}{שָׁנִים}) \\
			\foreignlanguage{hebrew}{שַׁעַר} & \textit{gate} \\
			\foreignlanguage{hebrew}{שִׁפְחָה} & \textit{maid, maid-servant} (pl.\ \foreignlanguage{hebrew}{שְׁפָחוֺת}) \\
			\foreignlanguage{hebrew}{שׁפט} & Q.\ \textit{to judge, govern} \\
			\foreignlanguage{hebrew}{שֹׁפֵט} & \textit{judge, ruler, governor} (part.\ act.\ Qal) \\
			\foreignlanguage{hebrew}{שׁפך} & Q.\ \textit{to pour, pour out}  \\
			\foreignlanguage{hebrew}{שׁקה} & Hi.\ \textit{to give to drink; to cause to drink water} (causative counterpart to the verb \foreignlanguage{hebrew}{שׁתה} Q.\ \textit{to drink}) \\ % BDB
			\foreignlanguage{hebrew}{שֶׁקֶל} & \textit{weight; shekel} (measurement of weight; about 11\,g) \\
			\foreignlanguage{hebrew}{שֶׁקֶר} & \textit{breach of faith, lie} \\ % HALOT
			\foreignlanguage{hebrew}{שׁרת} & Pi.\ \textit{to serve} \\ % BDB
			\foreignlanguage{hebrew}{שׁתה} & Q.\ \textit{to drink} \\
		\end{longtable}
		\unskip
		\unpenalty
		\unpenalty}
	\unvbox\ltmcbox
	
\end{multicols}

\medskip

\begin{center}
	\LARGE\foreignlanguage{hebrew}{ת}
\end{center}

\begin{multicols}{2}
	
	\setbox\ltmcbox\vbox{
		\makeatletter\col@number\@ne
		\begin{longtable}{>{\raggedleft}p{0.25\linewidth} p{0.65\linewidth}}
			\foreignlanguage{hebrew}{תְּהִלָּה} & \textit{glory, praise; song of praise} \\
			\foreignlanguage{hebrew}{תָּוֶךְ} & \textit{midst} (cs.\ st.\ \foreignlanguage{hebrew}{תּוֺךְ}) \\
			\foreignlanguage{hebrew}{תוֺעֵבָה} & \textit{abomination, abhorrence} \\
			\foreignlanguage{hebrew}{תּוֺרָה} & \textit{instruction} \\
			\foreignlanguage{hebrew}{תַּחַת} & \textit{under, beneath, instead of, in place of} \\
			\foreignlanguage{hebrew}{תָּמִיד} & \textit{continuously, always} (adv.), \textit{continuity} (noun) \\
			\foreignlanguage{hebrew}{תָּמִים} & \textit{complete; without fault; perfect; impeccable; honest, devout} (adj.) \\ % HALOT
			\foreignlanguage{hebrew}{תמם} & Q.\ \textit{to be complete, finished} \\
			\foreignlanguage{hebrew}{תְּפִלָּה} & \textit{prayer} \\
			\foreignlanguage{hebrew}{תפשׂ} & Q.\ \textit{to lay hold of, seize} \\
		\end{longtable}
		\unskip
		\unpenalty
		\unpenalty}
	\unvbox\ltmcbox
	
\end{multicols}

\begin{multicols}{2}
	
	\setbox\ltmcbox\vbox{
		\makeatletter\col@number\@ne
		\begin{longtable}{>{\raggedleft}p{0.25\linewidth} p{0.65\linewidth}}
			\foreignlanguage{hebrew}{תקע} & Q.\ \textit{to give a blow, blast; clap; drive in} \\ % BDB
			\foreignlanguage{hebrew}{תְּרוּמָה} & \textit{contribution, offering} \\
		\end{longtable}
		\unskip
		\unpenalty
		\unpenalty}
	\unvbox\ltmcbox
	
\end{multicols}


\end{document}
